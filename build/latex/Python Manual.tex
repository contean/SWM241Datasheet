%% Generated by Sphinx.
\def\sphinxdocclass{report}
\documentclass[letterpaper,10pt,english]{sphinxmanual}
\ifdefined\pdfpxdimen
   \let\sphinxpxdimen\pdfpxdimen\else\newdimen\sphinxpxdimen
\fi \sphinxpxdimen=.75bp\relax
\ifdefined\pdfimageresolution
    \pdfimageresolution= \numexpr \dimexpr1in\relax/\sphinxpxdimen\relax
\fi
%% let collapsible pdf bookmarks panel have high depth per default
\PassOptionsToPackage{bookmarksdepth=5}{hyperref}

\PassOptionsToPackage{booktabs}{sphinx}
\PassOptionsToPackage{colorrows}{sphinx}

\PassOptionsToPackage{warn}{textcomp}
\usepackage[utf8]{inputenc}
\ifdefined\DeclareUnicodeCharacter
% support both utf8 and utf8x syntaxes
  \ifdefined\DeclareUnicodeCharacterAsOptional
    \def\sphinxDUC#1{\DeclareUnicodeCharacter{"#1}}
  \else
    \let\sphinxDUC\DeclareUnicodeCharacter
  \fi
  \sphinxDUC{00A0}{\nobreakspace}
  \sphinxDUC{2500}{\sphinxunichar{2500}}
  \sphinxDUC{2502}{\sphinxunichar{2502}}
  \sphinxDUC{2514}{\sphinxunichar{2514}}
  \sphinxDUC{251C}{\sphinxunichar{251C}}
  \sphinxDUC{2572}{\textbackslash}
\fi
\usepackage{cmap}
\usepackage[T1]{fontenc}
\usepackage{amsmath,amssymb,amstext}
\usepackage{babel}



\usepackage{tgtermes}
\usepackage{tgheros}
\renewcommand{\ttdefault}{txtt}



\usepackage[Sonny]{fncychap}
\ChNameVar{\Large\normalfont\sffamily}
\ChTitleVar{\Large\normalfont\sffamily}
\usepackage{sphinx}

\fvset{fontsize=auto}
\usepackage{geometry}


% Include hyperref last.
\usepackage{hyperref}
% Fix anchor placement for figures with captions.
\usepackage{hypcap}% it must be loaded after hyperref.
% Set up styles of URL: it should be placed after hyperref.
\urlstyle{same}

\addto\captionsenglish{\renewcommand{\contentsname}{目录:}}

\usepackage{sphinxmessages}
\setcounter{tocdepth}{1}


\hypersetup{unicode=true}
\usepackage{CJKutf8}
\AtBeginDocument{\begin{CJK}{UTF8}{gbsn}}
\AtEndDocument{\end{CJK}}


\title{PYTHON Documentation}
\date{Nov 30, 2023}
\release{v1.0}
\author{xxxx}
\newcommand{\sphinxlogo}{\vbox{}}
\renewcommand{\releasename}{Release}
\makeindex
\begin{document}

\ifdefined\shorthandoff
  \ifnum\catcode`\=\string=\active\shorthandoff{=}\fi
  \ifnum\catcode`\"=\active\shorthandoff{"}\fi
\fi

\pagestyle{empty}
\sphinxmaketitle
\pagestyle{plain}
\sphinxtableofcontents
\pagestyle{normal}
\phantomsection\label{\detokenize{index::doc}}


\sphinxstepscope


\chapter{相关文档}
\label{\detokenize{SWM241/_u76f8_u5173_u6587_u6863/index:id1}}\label{\detokenize{SWM241/_u76f8_u5173_u6587_u6863/index::doc}}
\sphinxstepscope


\section{缩写表}
\label{\detokenize{SWM241/_u76f8_u5173_u6587_u6863/_u7f29_u5199_u8868:id1}}\label{\detokenize{SWM241/_u76f8_u5173_u6587_u6863/_u7f29_u5199_u8868::doc}}

\begin{savenotes}\sphinxattablestart
\sphinxthistablewithglobalstyle
\centering
\begin{tabular}[t]{\X{50}{100}\X{50}{100}}
\sphinxtoprule
\sphinxtableatstartofbodyhook
\sphinxAtStartPar
名称             |
&
\sphinxAtStartPar
描述                                        |
\\
\sphinxhline
\sphinxAtStartPar
ACMP
&
\sphinxAtStartPar
Analog Comparator Controller
\\
\sphinxhline
\sphinxAtStartPar
ADC
&
\sphinxAtStartPar
Analog\sphinxhyphen{}to\sphinxhyphen{}Digital Converter
\\
\sphinxhline
\sphinxAtStartPar
AES
&
\sphinxAtStartPar
Advanced Encryption Standard
\\
\sphinxhline
\sphinxAtStartPar
APB
&
\sphinxAtStartPar
Advanced Peripheral Bus
\\
\sphinxhline
\sphinxAtStartPar
AHB
&
\sphinxAtStartPar
Advanced High\sphinxhyphen{}Performance Bus
\\
\sphinxhline
\sphinxAtStartPar
BOD
&
\sphinxAtStartPar
Brown\sphinxhyphen{}out Detection
\\
\sphinxhline
\sphinxAtStartPar
CAN
&
\sphinxAtStartPar
Controller Area Network
\\
\sphinxhline
\sphinxAtStartPar
PWM
&
\sphinxAtStartPar
Pulse Width Modulation
\\
\sphinxhline
\sphinxAtStartPar
FIFO
&
\sphinxAtStartPar
First In, First Out
\\
\sphinxhline
\sphinxAtStartPar
GPIO
&
\sphinxAtStartPar
General\sphinxhyphen{}Purpose Input/Output
\\
\sphinxhline
\sphinxAtStartPar
IAP
&
\sphinxAtStartPar
In Application Programming
\\
\sphinxhline
\sphinxAtStartPar
ICP
&
\sphinxAtStartPar
In Circuit Programming
\\
\sphinxhline
\sphinxAtStartPar
ISP
&
\sphinxAtStartPar
In System Programming
\\
\sphinxhline
\sphinxAtStartPar
LDO
&
\sphinxAtStartPar
Low Dropout Regulator
\\
\sphinxhline
\sphinxAtStartPar
MPU
&
\sphinxAtStartPar
Memory Protection Unit
\\
\sphinxhline
\sphinxAtStartPar
NVIC
&
\sphinxAtStartPar
Nested Vectored Interrupt Controller
\\
\sphinxhline
\sphinxAtStartPar
DMA
&
\sphinxAtStartPar
Direct Memory Access
\\
\sphinxhline
\sphinxAtStartPar
PLL
&
\sphinxAtStartPar
Phase\sphinxhyphen{}Locked Loop
\\
\sphinxbottomrule
\end{tabular}
\sphinxtableafterendhook\par
\sphinxattableend\end{savenotes}

\sphinxstepscope


\section{寄存器描述列表缩写约定}
\label{\detokenize{SWM241/_u76f8_u5173_u6587_u6863/_u5bc4_u5b58_u5668_u63cf_u8ff0_u5217_u8868_u7f29_u5199_u7ea6_u5b9a:id1}}\label{\detokenize{SWM241/_u76f8_u5173_u6587_u6863/_u5bc4_u5b58_u5668_u63cf_u8ff0_u5217_u8868_u7f29_u5199_u7ea6_u5b9a::doc}}

\begin{savenotes}\sphinxattablestart
\sphinxthistablewithglobalstyle
\centering
\begin{tabular}[t]{\X{50}{100}\X{50}{100}}
\sphinxtoprule
\sphinxtableatstartofbodyhook
\sphinxAtStartPar
名称             |
&
\sphinxAtStartPar
描述                                        |
\\
\sphinxhline
\sphinxAtStartPar
RO
&
\sphinxAtStartPar
只读(read only)                           |
\\
\sphinxhline
\sphinxAtStartPar
WO
&
\sphinxAtStartPar
只写(write only)                          |
\\
\sphinxhline
\sphinxAtStartPar
R/W
&
\sphinxAtStartPar
读/写(read / write)                       |
\\
\sphinxhline
\sphinxAtStartPar
R/W0C
&
\sphinxAtStartPar
写0清零(read/write 0 clear)               |
\\
\sphinxhline
\sphinxAtStartPar
R/W1C
&
\sphinxAtStartPar
写1清零(read/write 1 clear)               |
\\
\sphinxhline
\sphinxAtStartPar
AC
&
\sphinxAtStartPar
自动清零(auto clear)                      |
\\
\sphinxhline
\sphinxAtStartPar
RC
&
\sphinxAtStartPar
读清零(read clear)                        |
\\
\sphinxhline\begin{itemize}
\item {} 
\end{itemize}
&
\sphinxAtStartPar
保留(reserve)                             |
\\
\sphinxbottomrule
\end{tabular}
\sphinxtableafterendhook\par
\sphinxattableend\end{savenotes}

\sphinxstepscope


\section{命名规则说明}
\label{\detokenize{SWM241/_u76f8_u5173_u6587_u6863/_u547d_u540d_u89c4_u5219_u8bf4_u660e:id1}}\label{\detokenize{SWM241/_u76f8_u5173_u6587_u6863/_u547d_u540d_u89c4_u5219_u8bf4_u660e::doc}}
\sphinxAtStartPar
\sphinxincludegraphics{{SWM241/相关文档/media命名规则说明002}.emf}

\sphinxstepscope

\sphinxAtStartPar
概述
==

\sphinxAtStartPar
SWM241系列32位MCU(以下简称SWM241)内嵌ARM® Cortex®\sphinxhyphen{}M0内核,凭借其出色的性能以及高可靠性、代码密度大等突出特点,可应用于工业控制、电机控制、白色家电等多种领域。

\sphinxAtStartPar
SWM241支持片上包含精度为1\%以内的48MHz时钟,最高支持48MHz时钟输出。同时提供最大为128K字节的FLASH和最大8K字节的SRAM。此外,芯片支持ISP(在系统编程)操作及IAP(在应用编程),用户可自定义BOOT程序。

\sphinxAtStartPar
SWM241外设串行总线包括最多4个UART接口、2个SPI通信接口(支持主/从选择)及2个I2C接口(支持主/从选择)、1个CAN模块,此外还具有1个16位看门狗定时器,8组32位加强型定时器(可触发DMA),4路8通道16bit
PWM,1个(最多12通道)通道12bit、1MSPS的逐次逼近型ADC模块,以及段码屏SLCD驱动器和大电流SLED驱动器,具备宽电压输入特性,并提供欠压中断及复位功能。

\sphinxstepscope

\sphinxAtStartPar
特性
==
\begin{itemize}
\item {} 
\sphinxAtStartPar
内核
\begin{itemize}
\item {} 
\sphinxAtStartPar
32位ARM® Cortex®\sphinxhyphen{}M0 内核

\item {} 
\sphinxAtStartPar
24位系统定时器

\item {} 
\sphinxAtStartPar
工作频率最高48MHz

\item {} 
\sphinxAtStartPar
硬件单周期乘法

\item {} 
\sphinxAtStartPar
集成嵌套向量中断控制器(NVIC),提供最多32个、4级可配置优先级的中断

\item {} 
\sphinxAtStartPar
通过SWD接口烧录

\end{itemize}

\item {} 
\sphinxAtStartPar
内置LDO
\begin{itemize}
\item {} 
\sphinxAtStartPar
供电电压范围2.5V至5.5V

\end{itemize}

\item {} 
\sphinxAtStartPar
SRAM存储器
\begin{itemize}
\item {} 
\sphinxAtStartPar
8KB

\end{itemize}

\item {} 
\sphinxAtStartPar
FLASH存储器
\begin{itemize}
\item {} 
\sphinxAtStartPar
128KB

\item {} 
\sphinxAtStartPar
支持用户定制ISP(在系统编程)更新用户程序

\item {} 
\sphinxAtStartPar
支持自定义BOOT程序

\item {} 
\sphinxAtStartPar
REMAP功能

\end{itemize}

\item {} 
\sphinxAtStartPar
串行接口
\begin{itemize}
\item {} 
\sphinxAtStartPar
UART*4,具有独立8字节FIFO,最高支持主时钟16分频

\item {} 
\sphinxAtStartPar
SPI*2,具有8字节独立FIFO,支持SPI、SSI协议,支持master/slave模式

\item {} 
\sphinxAtStartPar
I2C*2,支持7位、10位地址方式,支持master/slave模式

\item {} 
\sphinxAtStartPar
CAN*1,支持协议2.0A (11Bit标识符) 和2.0B(29Bit标识符)

\end{itemize}

\item {} 
\sphinxAtStartPar
PWM控制模块
\begin{itemize}
\item {} 
\sphinxAtStartPar
独立4通道16位PWM产生器,互补模式下可扩展为8通道

\item {} 
\sphinxAtStartPar
提供新周期开始中断,高电平结束中断以及中心对称模式下的半周期中断

\item {} 
\sphinxAtStartPar
具有普通、中心对称输出模式

\item {} 
\sphinxAtStartPar
死区控制

\item {} 
\sphinxAtStartPar
由硬件完成与ADC的交互

\end{itemize}

\item {} 
\sphinxAtStartPar
定时器模块
\begin{itemize}
\item {} 
\sphinxAtStartPar
8路32位加强定时器,支持计数器、捕获、脉冲发送等功能,1路支持HALL接口

\item {} 
\sphinxAtStartPar
16位看门狗定时器,溢出后可配置触发中断或复位芯片

\item {} 
\sphinxAtStartPar
内置低功耗定时器模块,使用内部32KB时钟,休眠计数并自唤醒

\end{itemize}

\item {} 
\sphinxAtStartPar
DMA模块
\begin{itemize}
\item {} 
\sphinxAtStartPar
共计2通道,支持UART/SPI/ADC模块及存储模块间数据交互

\end{itemize}

\item {} 
\sphinxAtStartPar
GPIO
\begin{itemize}
\item {} 
\sphinxAtStartPar
最多可达46个GPIO

\item {} 
\sphinxAtStartPar
可配置4种IO模式
\begin{itemize}
\item {} 
\sphinxAtStartPar
上拉输入

\item {} 
\sphinxAtStartPar
下拉输入

\item {} 
\sphinxAtStartPar
推挽输出

\item {} 
\sphinxAtStartPar
开漏输出

\end{itemize}

\item {} 
\sphinxAtStartPar
灵活的中断配置
\begin{itemize}
\item {} 
\sphinxAtStartPar
触发类型设置(边沿检测、电平检测)

\item {} 
\sphinxAtStartPar
触发电平设置(高电平、低电平)

\item {} 
\sphinxAtStartPar
触发边沿设置(上升沿、下降沿、双边沿)

\end{itemize}

\end{itemize}

\item {} 
\sphinxAtStartPar
模拟外设
\begin{itemize}
\item {} 
\sphinxAtStartPar
12位12通道高精度SAR ADC
\begin{itemize}
\item {} 
\sphinxAtStartPar
采样率高达1MSPS

\item {} 
\sphinxAtStartPar
支持single/scan两种模式

\item {} 
\sphinxAtStartPar
独立结果寄存器

\item {} 
\sphinxAtStartPar
提供独立FIFO

\item {} 
\sphinxAtStartPar
可由软件/PWM/TIMER触发

\item {} 
\sphinxAtStartPar
支持DMA

\end{itemize}

\end{itemize}

\item {} 
\sphinxAtStartPar
欠压检测
\begin{itemize}
\item {} 
\sphinxAtStartPar
支持欠压检测

\item {} 
\sphinxAtStartPar
支持欠压中断和复位选择

\end{itemize}

\item {} 
\sphinxAtStartPar
时钟源
\begin{itemize}
\item {} 
\sphinxAtStartPar
48MHz精度可达1\%的片内时钟源

\item {} 
\sphinxAtStartPar
最高支持48MHz时钟

\item {} 
\sphinxAtStartPar
32KHz片内时钟源

\item {} 
\sphinxAtStartPar
片外2~32MHz片外晶振

\item {} 
\sphinxAtStartPar
片外32KHz时钟,供RTC使用

\end{itemize}

\item {} 
\sphinxAtStartPar
SAFETY
\begin{itemize}
\item {} 
\sphinxAtStartPar
模拟电路配置写保护

\item {} 
\sphinxAtStartPar
时钟配置写保护

\item {} 
\sphinxAtStartPar
IO配置写保护

\item {} 
\sphinxAtStartPar
各模块中断使能写保护

\item {} 
\sphinxAtStartPar
非法地址访问检测

\end{itemize}

\item {} 
\sphinxAtStartPar
SLED

\item {} 
\sphinxAtStartPar
各模块中断使能写保护支持1/4占空比或1/8占空比

\item {} 
\sphinxAtStartPar
具有4或8个COM输出引脚和20个SEG输出引脚

\item {} 
\sphinxAtStartPar
8路大电流驱动I/O口

\item {} 
\sphinxAtStartPar
SLED刷新频率可灵活配置

\item {} 
\sphinxAtStartPar
间隔时间可灵活配置

\item {} 
\sphinxAtStartPar
灰度时间可灵活配置

\item {} 
\sphinxAtStartPar
SLCD

\item {} 
\sphinxAtStartPar
LCD帧频率设置

\item {} 
\sphinxAtStartPar
支持按键扫描

\item {} 
\sphinxAtStartPar
支持TYPE B驱动波形

\item {} 
\sphinxAtStartPar
根据图形显示数据,产生驱动电平控制时序

\item {} 
\sphinxAtStartPar
CRC
\begin{itemize}
\item {} 
\sphinxAtStartPar
支持CRC\sphinxhyphen{}32码多项式

\item {} 
\sphinxAtStartPar
支持CRC\sphinxhyphen{}16码多项式

\item {} 
\sphinxAtStartPar
支持输出结果设置,包括翻转、取反

\item {} 
\sphinxAtStartPar
支持初始值自定义

\item {} 
\sphinxAtStartPar
支持输入可选择取反

\end{itemize}

\item {} 
\sphinxAtStartPar
DIV
\begin{itemize}
\item {} 
\sphinxAtStartPar
支持32位整数除法运算及求余运算

\item {} 
\sphinxAtStartPar
支持32位开方运算,支持小数位

\item {} 
\sphinxAtStartPar
除法单次运算耗时32个时钟,不包括读写寄存器时间

\item {} 
\sphinxAtStartPar
开方单次运算耗时16/32个时钟,不包括读写寄存器时间

\item {} 
\sphinxAtStartPar
运算启动自动清除运算使能查询

\item {} 
\sphinxAtStartPar
提供运算进行标志和完成标志

\item {} 
\sphinxAtStartPar
支持有符号数和无符号数运算

\end{itemize}

\item {} 
\sphinxAtStartPar
其他
\begin{itemize}
\item {} 
\sphinxAtStartPar
自定义BOOT程序

\item {} 
\sphinxAtStartPar
128BIT 独立ID

\end{itemize}

\item {} 
\sphinxAtStartPar
低功耗
\begin{itemize}
\item {} 
\sphinxAtStartPar
正常模式:\sphinxhref{mailto:13mA@48MHz}{13mA@48MHz}

\item {} 
\sphinxAtStartPar
浅睡眠:30uA

\end{itemize}

\item {} 
\sphinxAtStartPar
环境
\begin{itemize}
\item {} 
\sphinxAtStartPar
工作温度:\sphinxhyphen{}40℃~105℃

\item {} 
\sphinxAtStartPar
保存温度:\sphinxhyphen{}50℃~150℃

\item {} 
\sphinxAtStartPar
湿度等级:MSL3

\end{itemize}

\item {} 
\sphinxAtStartPar
封装
\begin{itemize}
\item {} 
\sphinxAtStartPar
LQFP44

\item {} 
\sphinxAtStartPar
LQFP32

\end{itemize}

\item {} 
\sphinxAtStartPar
应用范围
\begin{itemize}
\item {} 
\sphinxAtStartPar
仪器仪表

\item {} 
\sphinxAtStartPar
工业控制

\item {} 
\sphinxAtStartPar
电机驱动

\item {} 
\sphinxAtStartPar
白色家电

\item {} 
\sphinxAtStartPar
可穿戴设备

\end{itemize}

\end{itemize}

\sphinxstepscope


\chapter{选型指南}
\label{\detokenize{SWM241/_u9009_u578b_u6307_u5357:id1}}\label{\detokenize{SWM241/_u9009_u578b_u6307_u5357::doc}}
\sphinxAtStartPar
表格 3‑1 SWM241系列MCU选型表


\begin{savenotes}\sphinxattablestart
\sphinxthistablewithglobalstyle
\centering
\begin{tabular}[t]{\X{33}{99}\X{33}{99}\X{33}{99}}
\sphinxtoprule
\sphinxtableatstartofbodyhook
\sphinxAtStartPar
\sphinxstylestrong{Peripherals}
&
\sphinxAtStartPar
\sphinxstylestrong{SWM241PBT7\sphinxhyphen{}80}
&
\sphinxAtStartPar
\sphinxstylestrong{SWM241KBT7\sphinxhyphen{}80}
\\
\sphinxhline
\sphinxAtStartPar
Voltage (V)
&
\sphinxAtStartPar
2.5\textasciitilde{}5.5
&
\sphinxAtStartPar
2.5\textasciitilde{}5.5
\\
\sphinxhline
\sphinxAtStartPar
Flash (KB)
&
\sphinxAtStartPar
128
&
\sphinxAtStartPar
128
\\
\sphinxhline
\sphinxAtStartPar
SRAM (KB)
&
\sphinxAtStartPar
8
&
\sphinxAtStartPar
8
\\
\sphinxhline
\sphinxAtStartPar
I/O
&
\sphinxAtStartPar
40
&
\sphinxAtStartPar
28
\\
\sphinxhline
\sphinxAtStartPar
Timer
&
\sphinxAtStartPar
8+1
&
\sphinxAtStartPar
8+1
\\
\sphinxhline
\sphinxAtStartPar
RTC
&
\sphinxAtStartPar
1
&
\sphinxAtStartPar
1
\\
\sphinxhline
\sphinxAtStartPar
PWM
&
\sphinxAtStartPar
15
&
\sphinxAtStartPar
12
\\
\sphinxhline
\sphinxAtStartPar
DMA
&
\sphinxAtStartPar
2
&
\sphinxAtStartPar
2
\\
\sphinxhline
\sphinxAtStartPar
SAR ADC
&
\sphinxAtStartPar
1(12)
&
\sphinxAtStartPar
1(8)
\\
\sphinxhline
\sphinxAtStartPar
UART
&
\sphinxAtStartPar
4
&
\sphinxAtStartPar
4
\\
\sphinxhline
\sphinxAtStartPar
I2C
&
\sphinxAtStartPar
2
&
\sphinxAtStartPar
2
\\
\sphinxhline
\sphinxAtStartPar
SPI
&
\sphinxAtStartPar
2
&
\sphinxAtStartPar
2
\\
\sphinxhline
\sphinxAtStartPar
CAN
&
\sphinxAtStartPar
1
&
\sphinxAtStartPar
1
\\
\sphinxhline
\sphinxAtStartPar
SLED
&
\sphinxAtStartPar
4x19
&
\sphinxAtStartPar
8x13
\\
\sphinxhline
\sphinxAtStartPar
SLCD
&
\sphinxAtStartPar
4x30
&
\sphinxAtStartPar
3x16
\\
\sphinxhline
\sphinxAtStartPar
CRC
&
\sphinxAtStartPar
1
&
\sphinxAtStartPar
1
\\
\sphinxhline
\sphinxAtStartPar
DIV
&
\sphinxAtStartPar
1
&
\sphinxAtStartPar
1
\\
\sphinxhline
\sphinxAtStartPar
SAFETY
&
\sphinxAtStartPar
1
&
\sphinxAtStartPar
1
\\
\sphinxhline
\sphinxAtStartPar
Package
&
\sphinxAtStartPar
LQFP44
&
\sphinxAtStartPar
LQFP32
\\
\sphinxbottomrule
\end{tabular}
\sphinxtableafterendhook\par
\sphinxattableend\end{savenotes}

\sphinxstepscope


\chapter{功能方框图}
\label{\detokenize{SWM241/_u529f_u80fd_u65b9_u6846_u56fe:id1}}\label{\detokenize{SWM241/_u529f_u80fd_u65b9_u6846_u56fe::doc}}
\sphinxAtStartPar
\sphinxincludegraphics{{SWM241/media功能方框图002}.emf}

\sphinxAtStartPar
图 4‑1功能方框图

\sphinxstepscope


\chapter{\sphinxstylestrong{管脚配置}}
\label{\detokenize{SWM241/_u7ba1_u811a_u914d_u7f6e:id1}}\label{\detokenize{SWM241/_u7ba1_u811a_u914d_u7f6e::doc}}

\section{SWM241PBT7}
\label{\detokenize{SWM241/_u7ba1_u811a_u914d_u7f6e:swm241pbt7}}
\sphinxAtStartPar
\sphinxincludegraphics{{SWM241/media管脚配置002}.emf}

\sphinxAtStartPar
图 5‑1 PBT7封装管脚配置图


\section{SWM241KBT7}
\label{\detokenize{SWM241/_u7ba1_u811a_u914d_u7f6e:swm241kbt7}}
\sphinxAtStartPar
\sphinxincludegraphics{{SWM241/media管脚配置003}.emf}

\sphinxAtStartPar
图 5‑2 KBT7封装管脚配置图


\section{管脚定义}
\label{\detokenize{SWM241/_u7ba1_u811a_u914d_u7f6e:id2}}

\begin{savenotes}
\sphinxatlongtablestart
\sphinxthistablewithglobalstyle
\makeatletter
  \LTleft \@totalleftmargin plus1fill
  \LTright\dimexpr\columnwidth-\@totalleftmargin-\linewidth\relax plus1fill
\makeatother
\begin{longtable}{\X{20}{100}\X{20}{100}\X{20}{100}\X{20}{100}\X{20}{100}}
\sphinxtoprule
\endfirsthead

\multicolumn{5}{c}{\sphinxnorowcolor
    \makebox[0pt]{\sphinxtablecontinued{\tablename\ \thetable{} \textendash{} continued from previous page}}%
}\\
\sphinxtoprule
\endhead

\sphinxbottomrule
\multicolumn{5}{r}{\sphinxnorowcolor
    \makebox[0pt][r]{\sphinxtablecontinued{continues on next page}}%
}\\
\endfoot

\endlastfoot
\sphinxtableatstartofbodyhook

\sphinxAtStartPar
管  {\color{red}\bfseries{}|}脚  {\color{red}\bfseries{}|}号  |
&
\begin{DUlineblock}{0em}
\item[] | |
\end{DUlineblock}
&
\sphinxAtStartPar
\sphinxstylestrong{管脚 | * 名称} | 型 |
&
\sphinxAtStartPar
类  | **
*  | |
&
\sphinxAtStartPar
复用功能**                         | | |
\\
\sphinxhline
\sphinxAtStartPar
\sphinxstylestrong{PBT7}
&
\sphinxAtStartPar
** KBT 7**
&&&\\
\sphinxhline
\sphinxAtStartPar
1
&
\sphinxAtStartPar
/
&
\sphinxAtStartPar
A4
&
\sphinxAtStartPar
I/O
&
\sphinxAtStartPar
UART0TX

\sphinxAtStartPar
PWM1AN

\sphinxAtStartPar
T7I

\sphinxAtStartPar
T7O

\sphinxAtStartPar
LCD\_SEG19
\\
\sphinxhline
\sphinxAtStartPar
2
&
\sphinxAtStartPar
1
&
\sphinxAtStartPar
A5
&
\sphinxAtStartPar
I/O
&
\sphinxAtStartPar
UART0RX

\sphinxAtStartPar
UART0TX

\sphinxAtStartPar
PWM0BN

\sphinxAtStartPar
ISP
\\
\sphinxhline
\sphinxAtStartPar
3
&
\sphinxAtStartPar
2
&
\sphinxAtStartPar
RESETn
&
\sphinxAtStartPar
I/O
&
\sphinxAtStartPar
—
\\
\sphinxhline
\sphinxAtStartPar
4
&
\sphinxAtStartPar
/
&
\sphinxAtStartPar
A6
&
\sphinxAtStartPar
I/O
&
\sphinxAtStartPar
PWM\_CLK0

\sphinxAtStartPar
LCD\_SEG20

\sphinxAtStartPar
XLI
\\
\sphinxhline
\sphinxAtStartPar
5
&
\sphinxAtStartPar
/
&
\sphinxAtStartPar
D0
&
\sphinxAtStartPar
I/O
&
\sphinxAtStartPar
UART0RTS

\sphinxAtStartPar
PWM0AN

\sphinxAtStartPar
LCD\_SEG21

\sphinxAtStartPar
XLO
\\
\sphinxhline
\sphinxAtStartPar
6
&
\sphinxAtStartPar
3
&
\sphinxAtStartPar
D1
&
\sphinxAtStartPar
I/O
&
\sphinxAtStartPar
UART0CTS

\sphinxAtStartPar
PWM0A

\sphinxAtStartPar
LCD\_SEG22
\\
\sphinxhline
\sphinxAtStartPar
7
&
\sphinxAtStartPar
4
&
\sphinxAtStartPar
D2
&
\sphinxAtStartPar
I/O
&
\sphinxAtStartPar
XI
\\
\sphinxhline
\sphinxAtStartPar
8
&
\sphinxAtStartPar
5
&
\sphinxAtStartPar
D3
&
\sphinxAtStartPar
I/O
&
\sphinxAtStartPar
PWM\_CLK1

\sphinxAtStartPar
XO
\\
\sphinxhline
\sphinxAtStartPar
9
&
\sphinxAtStartPar
6
&
\sphinxAtStartPar
CAP
&
\sphinxAtStartPar
S
&
\sphinxAtStartPar
注:需要对地连接一个1uF电容            |
\\
\sphinxhline
\sphinxAtStartPar
10
&
\sphinxAtStartPar
7
&
\sphinxAtStartPar
VSS
&
\sphinxAtStartPar
S
&
\sphinxAtStartPar
—
\\
\sphinxhline
\sphinxAtStartPar
11
&
\sphinxAtStartPar
8
&
\sphinxAtStartPar
VDD
&
\sphinxAtStartPar
S
&
\sphinxAtStartPar
—
\\
\sphinxhline
\sphinxAtStartPar
12
&
\sphinxAtStartPar
/
&
\sphinxAtStartPar
D4
&
\sphinxAtStartPar
I/O
&
\sphinxAtStartPar
I2C0SCL

\sphinxAtStartPar
HALL1

\sphinxAtStartPar
LCD\_SEG23
\\
\sphinxhline
\sphinxAtStartPar
13
&
\sphinxAtStartPar
/
&
\sphinxAtStartPar
D5
&
\sphinxAtStartPar
I/O
&
\sphinxAtStartPar
I2C0SDA

\sphinxAtStartPar
HALL2

\sphinxAtStartPar
LCD\_COM0
\\
\sphinxhline
\sphinxAtStartPar
14
&
\sphinxAtStartPar
9
&
\sphinxAtStartPar
D6
&
\sphinxAtStartPar
I/O
&
\sphinxAtStartPar
I2C0SCL

\sphinxAtStartPar
UART1RTS

\sphinxAtStartPar
CAN0TX

\sphinxAtStartPar
PWM0BRK

\sphinxAtStartPar
HALL0

\sphinxAtStartPar
LED\_COM0

\sphinxAtStartPar
LCD\_COM1
\\
\sphinxhline
\sphinxAtStartPar
15
&
\sphinxAtStartPar
10
&
\sphinxAtStartPar
D7
&
\sphinxAtStartPar
I/O
&
\sphinxAtStartPar
I2C0SDA

\sphinxAtStartPar
UART1CTS

\sphinxAtStartPar
CAN0RX

\sphinxAtStartPar
PWM2AN

\sphinxAtStartPar
LED\_COM1

\sphinxAtStartPar
LCD\_COM2
\\
\sphinxhline
\sphinxAtStartPar
16
&
\sphinxAtStartPar
11
&
\sphinxAtStartPar
D8
&
\sphinxAtStartPar
I/O
&
\sphinxAtStartPar
PWM2A

\sphinxAtStartPar
HALL0

\sphinxAtStartPar
T3I

\sphinxAtStartPar
T3O

\sphinxAtStartPar
LED\_COM2

\sphinxAtStartPar
LCD\_COM3
\\
\sphinxhline
\sphinxAtStartPar
17
&
\sphinxAtStartPar
12
&
\sphinxAtStartPar
D9
&
\sphinxAtStartPar
I/O
&
\sphinxAtStartPar
UART1TX

\sphinxAtStartPar
HALL1

\sphinxAtStartPar
T3I

\sphinxAtStartPar
T3O

\sphinxAtStartPar
LED\_COM3
\\
\sphinxhline
\sphinxAtStartPar
/
&
\sphinxAtStartPar
13
&
\sphinxAtStartPar
A0
&
\sphinxAtStartPar
I/O
&
\sphinxAtStartPar
SPI0SSEL

\sphinxAtStartPar
UART1RX

\sphinxAtStartPar
HALL2

\sphinxAtStartPar
LED\_COM4
\\
\sphinxhline
\sphinxAtStartPar
/
&
\sphinxAtStartPar
14
&
\sphinxAtStartPar
A1
&
\sphinxAtStartPar
I/O
&
\sphinxAtStartPar
I2C1SCL

\sphinxAtStartPar
SPI0SCLK

\sphinxAtStartPar
PWMBRK1

\sphinxAtStartPar
LED\_COM5
\\
\sphinxhline
\sphinxAtStartPar
/
&
\sphinxAtStartPar
15
&
\sphinxAtStartPar
A2
&
\sphinxAtStartPar
I/O
&
\sphinxAtStartPar
I2C1SDA

\sphinxAtStartPar
SPI0MISO

\sphinxAtStartPar
UART2TX

\sphinxAtStartPar
LED\_COM6
\\
\sphinxhline
\sphinxAtStartPar
/
&
\sphinxAtStartPar
16
&
\sphinxAtStartPar
C0
&
\sphinxAtStartPar
I/O
&
\sphinxAtStartPar
SPI0MOSI

\sphinxAtStartPar
UART2RX

\sphinxAtStartPar
PWM2BN

\sphinxAtStartPar
LED\_COM7

\sphinxAtStartPar
LCD\_SEG24
\\
\sphinxhline
\sphinxAtStartPar
18
&
\sphinxAtStartPar
/
&
\sphinxAtStartPar
C1
&
\sphinxAtStartPar
I/O
&
\sphinxAtStartPar
SPI1MOSI

\sphinxAtStartPar
PWM2B

\sphinxAtStartPar
T5O

\sphinxAtStartPar
LED\_SEG17

\sphinxAtStartPar
LCD\_SEG25
\\
\sphinxhline
\sphinxAtStartPar
19
&
\sphinxAtStartPar
/
&
\sphinxAtStartPar
C2
&
\sphinxAtStartPar
I/O
&
\sphinxAtStartPar
I2C0SDA

\sphinxAtStartPar
SPI1MISO

\sphinxAtStartPar
T5I

\sphinxAtStartPar
LED\_SEG18

\sphinxAtStartPar
LCD\_SEG26
\\
\sphinxhline
\sphinxAtStartPar
20
&
\sphinxAtStartPar
/
&
\sphinxAtStartPar
C3
&
\sphinxAtStartPar
I/O
&
\sphinxAtStartPar
I2C0SCL

\sphinxAtStartPar
SPI1SCLK

\sphinxAtStartPar
PWM2AN

\sphinxAtStartPar
T1I

\sphinxAtStartPar
T1O

\sphinxAtStartPar
LED\_SEG19

\sphinxAtStartPar
LCD\_SEG27
\\
\sphinxhline
\sphinxAtStartPar
21
&
\sphinxAtStartPar
/
&
\sphinxAtStartPar
D10
&
\sphinxAtStartPar
I/O
&
\sphinxAtStartPar
I2C1SCL

\sphinxAtStartPar
SPI0SCLK

\sphinxAtStartPar
PWM2B

\sphinxAtStartPar
RTC\_1HZ

\sphinxAtStartPar
LCD\_SEG28
\\
\sphinxhline
\sphinxAtStartPar
22
&
\sphinxAtStartPar
/
&
\sphinxAtStartPar
D11
&
\sphinxAtStartPar
I/O
&
\sphinxAtStartPar
I2C1SDA

\sphinxAtStartPar
SPI0MISO

\sphinxAtStartPar
UART3RX

\sphinxAtStartPar
PWM2A

\sphinxAtStartPar
LCD\_SEG29
\\
\sphinxhline
\sphinxAtStartPar
23
&
\sphinxAtStartPar
/
&
\sphinxAtStartPar
D12
&
\sphinxAtStartPar
I/O
&
\sphinxAtStartPar
SPI0MOSI

\sphinxAtStartPar
UART3TX

\sphinxAtStartPar
PWM2B

\sphinxAtStartPar
T4I

\sphinxAtStartPar
T4O

\sphinxAtStartPar
LCD\_SEG30
\\
\sphinxhline
\sphinxAtStartPar
24
&
\sphinxAtStartPar
17
&
\sphinxAtStartPar
B0
&
\sphinxAtStartPar
I/O
&
\sphinxAtStartPar
SPI1MOSI

\sphinxAtStartPar
UART3TX

\sphinxAtStartPar
PWM2A

\sphinxAtStartPar
T2I

\sphinxAtStartPar
T2O

\sphinxAtStartPar
LCD\_SEG31
\\
\sphinxhline
\sphinxAtStartPar
25
&
\sphinxAtStartPar
18
&
\sphinxAtStartPar
B1
&
\sphinxAtStartPar
I/O
&
\sphinxAtStartPar
SPI1MISO

\sphinxAtStartPar
UART3RX

\sphinxAtStartPar
PWM2B

\sphinxAtStartPar
T1I

\sphinxAtStartPar
T1O

\sphinxAtStartPar
LED\_SEG0

\sphinxAtStartPar
LCD\_SEG0
\\
\sphinxhline
\sphinxAtStartPar
26
&
\sphinxAtStartPar
19
&
\sphinxAtStartPar
B2
&
\sphinxAtStartPar
I/O
&
\sphinxAtStartPar
I2C0SCL

\sphinxAtStartPar
SPI0SCLK

\sphinxAtStartPar
PWM3B

\sphinxAtStartPar
T2I

\sphinxAtStartPar
T2O

\sphinxAtStartPar
LED\_SEG1

\sphinxAtStartPar
LCD\_SEG1
\\
\sphinxhline
\sphinxAtStartPar
27
&
\sphinxAtStartPar
20
&
\sphinxAtStartPar
B3
&
\sphinxAtStartPar
I/O
&
\sphinxAtStartPar
I2C0SDA

\sphinxAtStartPar
I2C1SCL

\sphinxAtStartPar
SPI0MISO

\sphinxAtStartPar
UART2RX

\sphinxAtStartPar
PWM3BN

\sphinxAtStartPar
T3I

\sphinxAtStartPar
T3O

\sphinxAtStartPar
LED\_SEG2

\sphinxAtStartPar
LCD\_SEG2
\\
\sphinxhline
\sphinxAtStartPar
28
&
\sphinxAtStartPar
21
&
\sphinxAtStartPar
B4
&
\sphinxAtStartPar
I/O
&
\sphinxAtStartPar
I2C1SDA

\sphinxAtStartPar
SPI0MOSI

\sphinxAtStartPar
UART2TX

\sphinxAtStartPar
CAN0TX

\sphinxAtStartPar
PWM2BN

\sphinxAtStartPar
T4I

\sphinxAtStartPar
T4O

\sphinxAtStartPar
LED\_SEG3

\sphinxAtStartPar
LCD\_SEG3
\\
\sphinxhline
\sphinxAtStartPar
29
&
\sphinxAtStartPar
22
&
\sphinxAtStartPar
B5
&
\sphinxAtStartPar
I/O
&
\sphinxAtStartPar
SPI1MOSI

\sphinxAtStartPar
UART3TX

\sphinxAtStartPar
CAN0RX

\sphinxAtStartPar
PWM2AN

\sphinxAtStartPar
T5I

\sphinxAtStartPar
T5O

\sphinxAtStartPar
LED\_SEG4

\sphinxAtStartPar
LCD\_SEG4
\\
\sphinxhline
\sphinxAtStartPar
30
&
\sphinxAtStartPar
23
&
\sphinxAtStartPar
B6
&
\sphinxAtStartPar
I/O
&
\sphinxAtStartPar
I2C1SDA

\sphinxAtStartPar
SPI1MISO

\sphinxAtStartPar
UART3RX

\sphinxAtStartPar
T6I

\sphinxAtStartPar
T6O

\sphinxAtStartPar
LED\_SEG5

\sphinxAtStartPar
LCD\_SEG5
\\
\sphinxhline
\sphinxAtStartPar
31
&
\sphinxAtStartPar
24
&
\sphinxAtStartPar
B7
&
\sphinxAtStartPar
I/O
&
\sphinxAtStartPar
I2C1SCL

\sphinxAtStartPar
SPI1SCLK

\sphinxAtStartPar
T7I

\sphinxAtStartPar
T7O

\sphinxAtStartPar
LED\_SEG6

\sphinxAtStartPar
LCD\_SEG6
\\
\sphinxhline
\sphinxAtStartPar
32
&
\sphinxAtStartPar
/
&
\sphinxAtStartPar
B8
&
\sphinxAtStartPar
I/O
&
\sphinxAtStartPar
UART1TX

\sphinxAtStartPar
LCD\_SEG7
\\
\sphinxhline
\sphinxAtStartPar
33
&
\sphinxAtStartPar
25
&
\sphinxAtStartPar
B9
&
\sphinxAtStartPar
I/O
&
\sphinxAtStartPar
UART1RX

\sphinxAtStartPar
LED\_SEG7

\sphinxAtStartPar
LCD\_SEG8

\sphinxAtStartPar
ADC0\_CH8
\\
\sphinxhline
\sphinxAtStartPar
34
&
\sphinxAtStartPar
/
&
\sphinxAtStartPar
A7
&
\sphinxAtStartPar
I/O
&
\sphinxAtStartPar
SPI1SSEL

\sphinxAtStartPar
UART2RTS

\sphinxAtStartPar
LED\_SEG12

\sphinxAtStartPar
LCD\_SEG9

\sphinxAtStartPar
ADC0\_CH7
\\
\sphinxhline
\sphinxAtStartPar
35
&
\sphinxAtStartPar
/
&
\sphinxAtStartPar
A8
&
\sphinxAtStartPar
I/O
&
\sphinxAtStartPar
UART2CTS

\sphinxAtStartPar
LED\_SEG13

\sphinxAtStartPar
LCD\_SEG10

\sphinxAtStartPar
ADC0\_CH6
\\
\sphinxhline
\sphinxAtStartPar
36
&
\sphinxAtStartPar
/
&
\sphinxAtStartPar
A9
&
\sphinxAtStartPar
I/O
&
\sphinxAtStartPar
PWM0B

\sphinxAtStartPar
LED\_SEG14

\sphinxAtStartPar
LCD\_SEG11

\sphinxAtStartPar
ADC0\_CH5
\\
\sphinxhline
\sphinxAtStartPar
37
&
\sphinxAtStartPar
26
&
\sphinxAtStartPar
A10
&
\sphinxAtStartPar
I/O
&
\sphinxAtStartPar
SWCLK

\sphinxAtStartPar
T7I

\sphinxAtStartPar
T7O

\sphinxAtStartPar
LED\_SEG15

\sphinxAtStartPar
LCD\_SEG12

\sphinxAtStartPar
ADC0\_CH4
\\
\sphinxhline
\sphinxAtStartPar
38
&
\sphinxAtStartPar
27
&
\sphinxAtStartPar
A11
&
\sphinxAtStartPar
I/O
&
\sphinxAtStartPar
SWDIO

\sphinxAtStartPar
PWM3A

\sphinxAtStartPar
LED\_SEG16

\sphinxAtStartPar
LCD\_SEG13

\sphinxAtStartPar
ADC0\_CH3
\\
\sphinxhline
\sphinxAtStartPar
39
&
\sphinxAtStartPar
/
&
\sphinxAtStartPar
A12
&
\sphinxAtStartPar
I/O
&
\sphinxAtStartPar
T6I

\sphinxAtStartPar
T6O

\sphinxAtStartPar
LCD\_SEG14

\sphinxAtStartPar
ADC0\_CH2
\\
\sphinxhline
\sphinxAtStartPar
40
&
\sphinxAtStartPar
28
&
\sphinxAtStartPar
A13
&
\sphinxAtStartPar
I/O
&
\sphinxAtStartPar
PWM3B

\sphinxAtStartPar
ADC0\_CH1
\\
\sphinxhline
\sphinxAtStartPar
41
&
\sphinxAtStartPar
29
&
\sphinxAtStartPar
A14
&
\sphinxAtStartPar
I/O
&
\sphinxAtStartPar
SPI0CLK

\sphinxAtStartPar
PWM3AN

\sphinxAtStartPar
ADC0\_CH0

\sphinxAtStartPar
ADC0\_REFP
\\
\sphinxhline
\sphinxAtStartPar
/
&
\sphinxAtStartPar
/
&
\sphinxAtStartPar
A15
&
\sphinxAtStartPar
I/O
&
\sphinxAtStartPar
PWM3BN
\\
\sphinxhline
\sphinxAtStartPar
42
&
\sphinxAtStartPar
30
&
\sphinxAtStartPar
D13
&&
\sphinxAtStartPar
I2C0SDA

\sphinxAtStartPar
SPI0MISO

\sphinxAtStartPar
UART0RX

\sphinxAtStartPar
T0O

\sphinxAtStartPar
LED\_SEG8

\sphinxAtStartPar
LCD\_SEG15

\sphinxAtStartPar
ADC0\_CH9
\\
\sphinxhline
\sphinxAtStartPar
43
&
\sphinxAtStartPar
31
&
\sphinxAtStartPar
D14
&&
\sphinxAtStartPar
SPI0MOSI

\sphinxAtStartPar
UART0TX

\sphinxAtStartPar
PWM1BN

\sphinxAtStartPar
T0I

\sphinxAtStartPar
LED\_SEG9

\sphinxAtStartPar
LCD\_SEG16

\sphinxAtStartPar
ADC0\_CH10
\\
\sphinxhline
\sphinxAtStartPar
44
&
\sphinxAtStartPar
32
&
\sphinxAtStartPar
A3
&
\sphinxAtStartPar
I/O
&
\sphinxAtStartPar
I2C0SCL

\sphinxAtStartPar
SPI1SSEL

\sphinxAtStartPar
PWM1A

\sphinxAtStartPar
T2I

\sphinxAtStartPar
T2O

\sphinxAtStartPar
LED\_SEG11

\sphinxAtStartPar
LCD\_SEG18

\sphinxAtStartPar
ADC0\_CH11
\\
\sphinxbottomrule
\end{longtable}
\sphinxtableafterendhook
\sphinxatlongtableend
\end{savenotes}

\sphinxAtStartPar
\sphinxstyleemphasis{注1:I=输入,O=输出,S=电源}

\sphinxAtStartPar
\sphinxstyleemphasis{注2:LED\_COMx(x=0, …, 7)集成大电流SINK能力}


\section{功能描述}
\label{\detokenize{SWM241/_u7ba1_u811a_u914d_u7f6e:id7}}

\begin{savenotes}
\sphinxatlongtablestart
\sphinxthistablewithglobalstyle
\makeatletter
  \LTleft \@totalleftmargin plus1fill
  \LTright\dimexpr\columnwidth-\@totalleftmargin-\linewidth\relax plus1fill
\makeatother
\begin{longtable}{\X{50}{100}\X{50}{100}}
\sphinxtoprule
\endfirsthead

\multicolumn{2}{c}{\sphinxnorowcolor
    \makebox[0pt]{\sphinxtablecontinued{\tablename\ \thetable{} \textendash{} continued from previous page}}%
}\\
\sphinxtoprule
\endhead

\sphinxbottomrule
\multicolumn{2}{r}{\sphinxnorowcolor
    \makebox[0pt][r]{\sphinxtablecontinued{continues on next page}}%
}\\
\endfoot

\endlastfoot
\sphinxtableatstartofbodyhook

\sphinxAtStartPar
管脚名称    | 描述
&
\begin{DUlineblock}{0em}
\item[] 
\end{DUlineblock}
\\
\sphinxhline
\sphinxAtStartPar
I2CxSCL
&
\sphinxAtStartPar
I2Cx模块时钟引脚                                       |
\\
\sphinxhline
\sphinxAtStartPar
I2CxSDA
&
\sphinxAtStartPar
I2Cx模块数据引脚                                       |
\\
\sphinxhline
\sphinxAtStartPar
SPIxSSEL
&
\sphinxAtStartPar
SPIx模块片选引脚                                       |
\\
\sphinxhline
\sphinxAtStartPar
SPIxCLK
&
\sphinxAtStartPar
SPIx模块时钟引脚                                       |
\\
\sphinxhline
\sphinxAtStartPar
SPIxMISO
&
\sphinxAtStartPar
SPIx模块主机接收引脚                                   |
\\
\sphinxhline
\sphinxAtStartPar
SPIxMOSI
&
\sphinxAtStartPar
SPIx模块主机发送引脚                                   |
\\
\sphinxhline
\sphinxAtStartPar
UARTxRX
&
\sphinxAtStartPar
UARTx模块数据接收引脚                                  |
\\
\sphinxhline
\sphinxAtStartPar
UARTxTX
&
\sphinxAtStartPar
UARTx模块数据发送引脚                                  |
\\
\sphinxhline
\sphinxAtStartPar
UARTxCTS
&
\sphinxAtStartPar
UARTx模块发送允许引脚                                  |
\\
\sphinxhline
\sphinxAtStartPar
UARTxRTS
&
\sphinxAtStartPar
UARTx模块接收请求引脚                                  |
\\
\sphinxhline
\sphinxAtStartPar
CANxRX
&
\sphinxAtStartPar
CANx模块接收引脚                                       |
\\
\sphinxhline
\sphinxAtStartPar
CANxTX
&
\sphinxAtStartPar
CANx模块发送引脚                                       |
\\
\sphinxhline
\sphinxAtStartPar
PWMxA
&
\sphinxAtStartPar
PWM模块第x组A路输出引脚                                |
\\
\sphinxhline
\sphinxAtStartPar
PWMxB
&
\sphinxAtStartPar
PWM模块第x组B路输出引脚                                |
\\
\sphinxhline
\sphinxAtStartPar
PWMxAN
&
\sphinxAtStartPar
PWM模块第x组A路反向输出引脚                            |
\\
\sphinxhline
\sphinxAtStartPar
PWMxBN
&
\sphinxAtStartPar
PWM模块第x组B路反向输出引脚                            |
\\
\sphinxhline
\sphinxAtStartPar
PWM\_CLKx
&
\sphinxAtStartPar
PWM PULSE引脚                                          |
\\
\sphinxhline
\sphinxAtStartPar
PWMxBRK
&
\sphinxAtStartPar
PWMx模块的BRAKE输出引脚                                |
\\
\sphinxhline
\sphinxAtStartPar
TxI
&
\sphinxAtStartPar
TIMERx模块输入捕获引脚                                 |
\\
\sphinxhline
\sphinxAtStartPar
TxO
&
\sphinxAtStartPar
TIMERx模块输出捕获引脚                                 |
\\
\sphinxhline
\sphinxAtStartPar
HALLx
&
\sphinxAtStartPar
霍尔模块输入引脚x                                      |
\\
\sphinxhline
\sphinxAtStartPar
LED\_COMx
&
\sphinxAtStartPar
LED模块COM引脚x                                        |
\\
\sphinxhline
\sphinxAtStartPar
LED\_SEGx
&
\sphinxAtStartPar
LED模块SEG引脚x                                        |
\\
\sphinxhline
\sphinxAtStartPar
LCD\_COMx
&
\sphinxAtStartPar
LCD模块COM引脚x                                        |
\\
\sphinxhline
\sphinxAtStartPar
LCD\_SEGx
&
\sphinxAtStartPar
LCD模块SEG引脚x                                        |
\\
\sphinxhline
\sphinxAtStartPar
ADC0\_CHx
&
\sphinxAtStartPar
ADC0模块通道x输入引脚                                  |
\\
\sphinxhline
\sphinxAtStartPar
ADC0\_REFP
&
\sphinxAtStartPar
ADC0 REFP基准正向输入引脚                              |
\\
\sphinxhline
\sphinxAtStartPar
ISP
&
\sphinxAtStartPar
ISP功能引脚                                            |
\\
\sphinxhline
\sphinxAtStartPar
RESETn
&
\sphinxAtStartPar
芯片复位功能引脚,低电平复位                           |
\\
\sphinxhline
\sphinxAtStartPar
RTC\_1HZ
&
\sphinxAtStartPar
RTC 1Hz 输出引脚                                       |
\\
\sphinxhline
\sphinxAtStartPar
XLI
&
\sphinxAtStartPar
外部低频晶振输入引脚                                   |
\\
\sphinxhline
\sphinxAtStartPar
XLO
&
\sphinxAtStartPar
外部低频晶振输出引脚                                   |
\\
\sphinxhline
\sphinxAtStartPar
XI
&
\sphinxAtStartPar
外部高频晶振输入引脚                                   |
\\
\sphinxhline
\sphinxAtStartPar
XO
&
\sphinxAtStartPar
外部高频晶振输出引脚                                   |
\\
\sphinxhline
\sphinxAtStartPar
CAP
&
\sphinxAtStartPar
LDO电容引脚                                            |
\\
\sphinxhline
\sphinxAtStartPar
VSS
&
\sphinxAtStartPar
芯片电源地引脚                                         |
\\
\sphinxhline
\sphinxAtStartPar
VDD
&
\sphinxAtStartPar
芯片电源引脚                                           |
\\
\sphinxhline
\sphinxAtStartPar
SWCLK
&
\sphinxAtStartPar
下载器SWCLK引脚                                        |
\\
\sphinxhline
\sphinxAtStartPar
SWDIO
&
\sphinxAtStartPar
下载器SWDIO引脚                                        |
\\
\sphinxbottomrule
\end{longtable}
\sphinxtableafterendhook
\sphinxatlongtableend
\end{savenotes}


\section{管脚复用功能}
\label{\detokenize{SWM241/_u7ba1_u811a_u914d_u7f6e:id8}}
\sphinxAtStartPar
PORTA管脚功能通过PORTCON模块的PORTA\_FUNC0和PORTA\_FUNC1寄存器对应位域配置。请使用驱动库提供的PORT\_Init函数和功能宏定义配置管脚功能,提高代码可读性。

\sphinxAtStartPar
表格 5‑1 PORTA复用功能表


\begin{savenotes}\sphinxattablestart
\sphinxthistablewithglobalstyle
\centering
\begin{tabular}[t]{\X{8}{96}\X{8}{96}\X{8}{96}\X{8}{96}\X{8}{96}\X{8}{96}\X{8}{96}\X{8}{96}\X{8}{96}\X{8}{96}\X{8}{96}\X{8}{96}}
\sphinxtoprule
\sphinxtableatstartofbodyhook
\sphinxAtStartPar
管脚
&
\begin{DUlineblock}{0em}
\item[] 功 | 能0  |
\end{DUlineblock}
&
\sphinxAtStartPar
功 | 能 | 1  |
&
\begin{DUlineblock}{0em}
\item[] 
\item[] {\color{red}\bfseries{}|}
\end{DUlineblock}
&
\begin{DUlineblock}{0em}
\item[] 
\item[] {\color{red}\bfseries{}|}
\end{DUlineblock}

\sphinxAtStartPar
能 |
3
&
\begin{DUlineblock}{0em}
\item[] 
\item[] 
\item[] 功
|
\end{DUlineblock}

\sphinxAtStartPar
4
&
\begin{DUlineblock}{0em}
\item[] 
\item[] 功
\item[] 能
\end{DUlineblock}

\sphinxAtStartPar
5
&
\begin{DUlineblock}{0em}
\item[] 
\item[] 
\end{DUlineblock}
\begin{description}
\sphinxlineitem{功 |}
\sphinxAtStartPar
能

\end{description}

\sphinxAtStartPar
6
&
\begin{DUlineblock}{0em}
\item[] 
\item[] {\color{red}\bfseries{}|}
\end{DUlineblock}

\sphinxAtStartPar
能 |
7
&
\begin{DUlineblock}{0em}
\item[] 
\item[] 
\item[] {\color{red}\bfseries{}|}
\end{DUlineblock}

\sphinxAtStartPar
8
&
\begin{DUlineblock}{0em}
\item[] 其
\item[] 他
|
 |
\end{DUlineblock}

\sphinxAtStartPar
15
&
\begin{DUlineblock}{0em}
\item[] 
\end{DUlineblock}
\\
\sphinxhline
\sphinxAtStartPar
PA0
&
\sphinxAtStartPar
A0
&&
\sphinxAtStartPar
U AR T1 RX
&
\sphinxAtStartPar
H AL L2
&
\sphinxAtStartPar
LE {\color{red}\bfseries{}D\_} CO M4
&&&&&&\\
\sphinxhline
\sphinxAtStartPar
PA1
&
\sphinxAtStartPar
A1
&
\sphinxAtStartPar
I 2C 1S CL
&
\sphinxAtStartPar
S PI 0C LK
&
\sphinxAtStartPar
P WM BR K1
&
\sphinxAtStartPar
LE {\color{red}\bfseries{}D\_} CO M5
&&&&&&\\
\sphinxhline
\sphinxAtStartPar
PA2
&
\sphinxAtStartPar
A2
&
\sphinxAtStartPar
I 2C 1S DA
&
\sphinxAtStartPar
SP I0 MI SO
&
\sphinxAtStartPar
U AR T2 TX
&
\sphinxAtStartPar
LE {\color{red}\bfseries{}D\_} CO M6
&&&&&&\\
\sphinxhline
\sphinxAtStartPar
PA3
&
\sphinxAtStartPar
A3/  A {\color{red}\bfseries{}DC0\_} CH11
&
\sphinxAtStartPar
I 2C 0S CL
&
\sphinxAtStartPar
SP I1 SS EL
&
\sphinxAtStartPar
P WM 1A
&
\sphinxAtStartPar
T 2I
&
\sphinxAtStartPar
T 2O
&
\sphinxAtStartPar
L ED \_S EG 11
&&&
\sphinxAtStartPar
L CD \_S EG 18
&\\
\sphinxhline
\sphinxAtStartPar
PA4
&
\sphinxAtStartPar
A4
&
\sphinxAtStartPar
U AR T0 TX
&
\sphinxAtStartPar
PW M1 AN
&
\sphinxAtStartPar
T 7I
&
\sphinxAtStartPar
T 7O
&&&&&
\sphinxAtStartPar
L CD \_S EG 19
&\\
\sphinxhline
\sphinxAtStartPar
PA5
&
\sphinxAtStartPar
A5
&
\sphinxAtStartPar
U AR T0 RX
&
\sphinxAtStartPar
U AR T0 TX
&
\sphinxAtStartPar
PW M0 BN
&&&&&&\begin{itemize}
\item {} 
\end{itemize}
&
\sphinxAtStartPar
ISP
\\
\sphinxhline
\sphinxAtStartPar
PA6
&
\sphinxAtStartPar
A6
&
\sphinxAtStartPar
PW {\color{red}\bfseries{}M\_} CL K0
&&&&&&&&
\sphinxAtStartPar
L CD \_S EG 20
&
\sphinxAtStartPar
XLI
\\
\sphinxhline
\sphinxAtStartPar
PA7
&
\sphinxAtStartPar
A7/   Adc0 \_CH7
&
\sphinxAtStartPar
SP I1 SS EL
&
\sphinxAtStartPar
UA RT 2R TS
&
\sphinxAtStartPar
L ED \_S EG 12
&&&&&&
\sphinxAtStartPar
LC {\color{red}\bfseries{}D\_} SE G9
&\\
\sphinxhline
\sphinxAtStartPar
PA8
&
\sphinxAtStartPar
A8/   ADC0 \_CH6
&
\sphinxAtStartPar
UA RT 2C TS
&
\sphinxAtStartPar
L ED \_S EG 13
&&&&&&&
\sphinxAtStartPar
L CD \_S EG 10
&\\
\sphinxhline
\sphinxAtStartPar
PA9
&
\sphinxAtStartPar
A9/   ADC0 \_CH5
&
\sphinxAtStartPar
P WM 0B
&
\sphinxAtStartPar
L ED \_S EG 14
&&&&&&&&\\
\sphinxhline
\sphinxAtStartPar
PA10
&
\sphinxAtStartPar
A10/   ADC0 \_CH4
&
\sphinxAtStartPar
S WC LK
&
\sphinxAtStartPar
T 7I
&
\sphinxAtStartPar
T 7O
&
\sphinxAtStartPar
L ED \_S EG 15
&&&&&
\sphinxAtStartPar
L CD \_S EG 12
&\\
\sphinxhline
\sphinxAtStartPar
PA11
&
\sphinxAtStartPar
A11/   ADC0 \_CH3
&
\sphinxAtStartPar
S WD IO
&
\sphinxAtStartPar
P WM 3A
&
\sphinxAtStartPar
L ED \_S EG 16
&&&&&&
\sphinxAtStartPar
L CD \_S EG 13
&\\
\sphinxhline
\sphinxAtStartPar
PA12
&
\sphinxAtStartPar
A12/   ADC0 \_CH2
&
\sphinxAtStartPar
T 6I
&
\sphinxAtStartPar
T 6O
&&&&&&&
\sphinxAtStartPar
L CD \_S EG 14
&\\
\sphinxhline
\sphinxAtStartPar
PA13
&
\sphinxAtStartPar
A13/   ADC0 \_CH1
&
\sphinxAtStartPar
P WM 3B
&&&&&&&&&\\
\sphinxhline
\sphinxAtStartPar
PA14
&
\sphinxAtStartPar
A14/  A {\color{red}\bfseries{}DC0\_} CH0/  A {\color{red}\bfseries{}DC0\_} REFP
&
\sphinxAtStartPar
S PI 0C LK
&
\sphinxAtStartPar
PW M3 AN
&&&&&&&&\\
\sphinxhline
\sphinxAtStartPar
PA15
&
\sphinxAtStartPar
A15
&
\sphinxAtStartPar
PW M3 BN
&&&&&&&&&\\
\sphinxbottomrule
\end{tabular}
\sphinxtableafterendhook\par
\sphinxattableend\end{savenotes}

\sphinxAtStartPar
\sphinxstyleemphasis{注:PA15专用于SLCD模块电源域控制,使用SLCD时需要设置为输出高电平为SLCD供电,不使用SLCD时设置为输出低电平。}

\begin{DUlineblock}{0em}
\item[] 
\end{DUlineblock}

\sphinxAtStartPar
PORTB管脚功能通过PORTCON模块的PORTB\_FUNC0和PORTB\_FUNC1寄存器对应位域配置。请使用驱动库提供的PORT\_Init函数和功能宏定义配置管脚功能,提高代码可读性。

\sphinxAtStartPar
表格 5‑2 PORTB复用功能


\begin{savenotes}\sphinxattablestart
\sphinxthistablewithglobalstyle
\centering
\begin{tabular}[t]{\X{8}{96}\X{8}{96}\X{8}{96}\X{8}{96}\X{8}{96}\X{8}{96}\X{8}{96}\X{8}{96}\X{8}{96}\X{8}{96}\X{8}{96}\X{8}{96}}
\sphinxtoprule
\sphinxtableatstartofbodyhook
\sphinxAtStartPar
管脚
&
\begin{DUlineblock}{0em}
\item[] 功 | 能0  |
\end{DUlineblock}
&
\sphinxAtStartPar
功  | 能1 | |
&
\begin{DUlineblock}{0em}
\item[] 功
\item[] 能
|
\end{DUlineblock}
&
\begin{DUlineblock}{0em}
\item[] 功
\item[] 能4
|
\end{DUlineblock}
&
\begin{DUlineblock}{0em}
\item[] 功 |
\item[] 能 |
|
\end{DUlineblock}
&
\sphinxAtStartPar
功  | 能6 | 5  |
&
\sphinxAtStartPar
功  | 能7 | |
&
\begin{DUlineblock}{0em}
\item[] 功
\item[] 能
|
\end{DUlineblock}
&
\begin{DUlineblock}{0em}
\item[] 其
\end{DUlineblock}

\sphinxAtStartPar
8  |
&
\begin{DUlineblock}{0em}
\item[] 
\end{DUlineblock}

\sphinxAtStartPar
5  |
&
\sphinxAtStartPar
他  |
\\
\sphinxhline
\sphinxAtStartPar
PB0
&
\sphinxAtStartPar
B0
&
\sphinxAtStartPar
SP I1M OSI
&
\sphinxAtStartPar
U AR T3 TX
&
\sphinxAtStartPar
PW M2A
&
\sphinxAtStartPar
T2I
&
\sphinxAtStartPar
T 2O
&&&&
\sphinxAtStartPar
LCD \_SE G31
&\\
\sphinxhline
\sphinxAtStartPar
PB1
&
\sphinxAtStartPar
B1
&
\sphinxAtStartPar
SP I1M ISO
&
\sphinxAtStartPar
U AR T3 RX
&
\sphinxAtStartPar
PW M2B
&
\sphinxAtStartPar
T1I
&
\sphinxAtStartPar
T 1O
&
\sphinxAtStartPar
LE D\_S EG0
&&&
\sphinxAtStartPar
LC D\_S EG0
&\\
\sphinxhline
\sphinxAtStartPar
PB2
&
\sphinxAtStartPar
B2
&
\sphinxAtStartPar
I 2C0 SCL
&
\sphinxAtStartPar
S PI 0C LK
&
\sphinxAtStartPar
PW M3B
&
\sphinxAtStartPar
T2I
&
\sphinxAtStartPar
T 2O
&
\sphinxAtStartPar
LE D\_S EG1
&&&
\sphinxAtStartPar
LC D\_S EG1
&\\
\sphinxhline
\sphinxAtStartPar
PB3
&
\sphinxAtStartPar
B3
&
\sphinxAtStartPar
I 2C0 SDA
&
\sphinxAtStartPar
I 2C 0S CL
&
\sphinxAtStartPar
SP I0M ISO
&
\sphinxAtStartPar
U ART 2RX
&
\sphinxAtStartPar
PW M3 BN
&
\sphinxAtStartPar
T3I
&
\sphinxAtStartPar
T3O
&
\sphinxAtStartPar
LE {\color{red}\bfseries{}D\_} SE G2
&
\sphinxAtStartPar
LC D\_S EG2
&\\
\sphinxhline
\sphinxAtStartPar
PB4
&
\sphinxAtStartPar
B4
&
\sphinxAtStartPar
I 2C1 SDA
&
\sphinxAtStartPar
SP I0 MO SI
&
\sphinxAtStartPar
U ART 2TX
&
\sphinxAtStartPar
CAN 0TX
&
\sphinxAtStartPar
PW M2 BN
&
\sphinxAtStartPar
T4I
&
\sphinxAtStartPar
T4O
&
\sphinxAtStartPar
LE {\color{red}\bfseries{}D\_} SE G3
&
\sphinxAtStartPar
LC D\_S EG3
&\\
\sphinxhline
\sphinxAtStartPar
PB5
&
\sphinxAtStartPar
B5
&&
\sphinxAtStartPar
SP I1 MO SI
&
\sphinxAtStartPar
U ART 3TX
&
\sphinxAtStartPar
CAN 0RX
&
\sphinxAtStartPar
PW M2 AN
&
\sphinxAtStartPar
T5I
&
\sphinxAtStartPar
T5O
&
\sphinxAtStartPar
LE {\color{red}\bfseries{}D\_} SE G4
&
\sphinxAtStartPar
LC D\_S EG4
&\\
\sphinxhline
\sphinxAtStartPar
PB6
&
\sphinxAtStartPar
B6
&&
\sphinxAtStartPar
I 2C 1S DA
&
\sphinxAtStartPar
SP I1M ISO
&
\sphinxAtStartPar
U ART 3RX
&
\sphinxAtStartPar
T 6I
&
\sphinxAtStartPar
T6O
&
\sphinxAtStartPar
LE D\_S EG5
&&
\sphinxAtStartPar
LC D\_S EG5
&\\
\sphinxhline
\sphinxAtStartPar
PB7
&
\sphinxAtStartPar
B7
&
\sphinxAtStartPar
I 2C1 SCL
&
\sphinxAtStartPar
S PI 1C LK
&
\sphinxAtStartPar
T7I
&
\sphinxAtStartPar
T7O
&
\sphinxAtStartPar
LE {\color{red}\bfseries{}D\_} SE G6
&&&&
\sphinxAtStartPar
LC D\_S EG6
&\\
\sphinxhline
\sphinxAtStartPar
PB8
&
\sphinxAtStartPar
B8
&
\sphinxAtStartPar
U ART 1TX
&&&&&&&&
\sphinxAtStartPar
LC D\_S EG7
&\\
\sphinxhline
\sphinxAtStartPar
PB9
&
\sphinxAtStartPar
B9/   ADC0 \_CH8
&
\sphinxAtStartPar
U ART 1RX
&
\sphinxAtStartPar
LE {\color{red}\bfseries{}D\_} SE G7
&&&&&&&
\sphinxAtStartPar
LC D\_S EG8
&\\
\sphinxhline
\sphinxAtStartPar
PB10
&
\sphinxAtStartPar
B10
&&&&&&&&&&\\
\sphinxhline
\sphinxAtStartPar
PB11
&
\sphinxAtStartPar
B11
&&&&&&&&&&\\
\sphinxhline
\sphinxAtStartPar
PB12
&
\sphinxAtStartPar
B12
&&&&&&&&&&\\
\sphinxhline
\sphinxAtStartPar
PB13
&
\sphinxAtStartPar
B13
&&&&&&&&&&\\
\sphinxhline
\sphinxAtStartPar
PB14
&
\sphinxAtStartPar
B14
&&&&&&&&&&\\
\sphinxhline
\sphinxAtStartPar
PB15
&
\sphinxAtStartPar
B15
&&&&&&&&&&\\
\sphinxbottomrule
\end{tabular}
\sphinxtableafterendhook\par
\sphinxattableend\end{savenotes}

\sphinxAtStartPar
PORTC管脚功能通过PORTCON模块的PORTC\_FUNC0和PORTC\_FUNC1寄存器对应位域配置。请使用驱动库提供的PORT\_Init函数和功能宏定义配置管脚功能,提高代码可读性。

\sphinxAtStartPar
表格 5‑3 PORTC复用功能


\begin{savenotes}\sphinxattablestart
\sphinxthistablewithglobalstyle
\centering
\begin{tabular}[t]{\X{8}{96}\X{8}{96}\X{8}{96}\X{8}{96}\X{8}{96}\X{8}{96}\X{8}{96}\X{8}{96}\X{8}{96}\X{8}{96}\X{8}{96}\X{8}{96}}
\sphinxtoprule
\sphinxtableatstartofbodyhook
\sphinxAtStartPar
管脚
&
\begin{DUlineblock}{0em}
\item[] 功 | 能0  |
\end{DUlineblock}
&
\sphinxAtStartPar
功  | 能1 | |
&
\begin{DUlineblock}{0em}
\item[] 功
\item[] 能
|
\end{DUlineblock}
&
\begin{DUlineblock}{0em}
\item[] 功
\item[] 能4
|
\end{DUlineblock}
&
\begin{DUlineblock}{0em}
\item[] 功 |
\item[] 能 |
|
\end{DUlineblock}
&
\sphinxAtStartPar
功  | 能6 | 5  |
&
\sphinxAtStartPar
功  | 能7 | |
&
\begin{DUlineblock}{0em}
\item[] 功
\item[] 能
|
\end{DUlineblock}
&
\begin{DUlineblock}{0em}
\item[] 其
他
\end{DUlineblock}

\sphinxAtStartPar
8  |
&
\begin{DUlineblock}{0em}
\item[] 
\end{DUlineblock}

\sphinxAtStartPar
15  |
&
\begin{DUlineblock}{0em}
\item[] 
\end{DUlineblock}
\\
\sphinxhline
\sphinxAtStartPar
PC0
&
\sphinxAtStartPar
C0
&
\sphinxAtStartPar
SP I0M OSI
&
\sphinxAtStartPar
U AR T2 RX
&
\sphinxAtStartPar
PWM 2BN
&
\sphinxAtStartPar
LE D\_C OM7
&&&&&
\sphinxAtStartPar
LCD \_SE G24
&\\
\sphinxhline
\sphinxAtStartPar
PC1
&
\sphinxAtStartPar
C1
&
\sphinxAtStartPar
SP I1M OSI
&
\sphinxAtStartPar
P WM 2B
&
\sphinxAtStartPar
T5O
&
\sphinxAtStartPar
LED \_SE G17
&&&&&
\sphinxAtStartPar
LCD \_SE G25
&\\
\sphinxhline
\sphinxAtStartPar
PC2
&
\sphinxAtStartPar
C2
&
\sphinxAtStartPar
I 2C0 SDA
&
\sphinxAtStartPar
SP I1 MI SO
&
\sphinxAtStartPar
T5I
&
\sphinxAtStartPar
LED \_SE G18
&&&&&
\sphinxAtStartPar
LCD \_SE G26
&\\
\sphinxhline
\sphinxAtStartPar
PC3
&
\sphinxAtStartPar
C3
&
\sphinxAtStartPar
I 2C0 SCL
&
\sphinxAtStartPar
S PI 1C LK
&
\sphinxAtStartPar
PWM 2AN
&
\sphinxAtStartPar
T1I
&
\sphinxAtStartPar
T 1O
&
\sphinxAtStartPar
LED \_SE G19
&&&
\sphinxAtStartPar
LCD \_SE G27
&\\
\sphinxhline
\sphinxAtStartPar
PC4
&
\sphinxAtStartPar
C4
&&&&&&&&&&\\
\sphinxhline
\sphinxAtStartPar
PC5
&
\sphinxAtStartPar
C5
&&&&&&&&&&\\
\sphinxhline
\sphinxAtStartPar
PC6
&
\sphinxAtStartPar
C6
&&&&&&&&&&\\
\sphinxhline
\sphinxAtStartPar
PC7
&
\sphinxAtStartPar
C7
&&&&&&&&&&\\
\sphinxhline
\sphinxAtStartPar
PC8
&
\sphinxAtStartPar
C8
&&&&&&&&&&\\
\sphinxhline
\sphinxAtStartPar
PC9
&
\sphinxAtStartPar
C9
&&&&&&&&&&\\
\sphinxhline
\sphinxAtStartPar
PC10
&
\sphinxAtStartPar
C10
&&&&&&&&&&\\
\sphinxhline
\sphinxAtStartPar
PC11
&
\sphinxAtStartPar
C11
&&&&&&&&&&\\
\sphinxhline
\sphinxAtStartPar
PC12
&
\sphinxAtStartPar
C12
&&&&&&&&&&\\
\sphinxhline
\sphinxAtStartPar
PC13
&
\sphinxAtStartPar
C13
&&&&&&&&&&\\
\sphinxhline
\sphinxAtStartPar
PC14
&
\sphinxAtStartPar
C14
&&&&&&&&&&\\
\sphinxhline
\sphinxAtStartPar
PC15
&
\sphinxAtStartPar
C15
&&&&&&&&&&\\
\sphinxbottomrule
\end{tabular}
\sphinxtableafterendhook\par
\sphinxattableend\end{savenotes}

\sphinxAtStartPar
PORTD管脚功能通过PORTCON模块的PORTD\_FUNC0和PORTD\_FUNC1寄存器对应位域配置。请使用驱动库提供的PORT\_Init函数和功能宏定义配置管脚功能,提高代码可读性。

\sphinxAtStartPar
表格 5‑4 PORTD复用功能


\begin{savenotes}\sphinxattablestart
\sphinxthistablewithglobalstyle
\centering
\begin{tabular}[t]{\X{8}{96}\X{8}{96}\X{8}{96}\X{8}{96}\X{8}{96}\X{8}{96}\X{8}{96}\X{8}{96}\X{8}{96}\X{8}{96}\X{8}{96}\X{8}{96}}
\sphinxtoprule
\sphinxtableatstartofbodyhook
\sphinxAtStartPar
管脚
&
\begin{DUlineblock}{0em}
\item[] 功 | 能0  |
\end{DUlineblock}
&
\sphinxAtStartPar
功  | 能1 | |
&
\begin{DUlineblock}{0em}
\item[] 功
\item[] 能
|
\end{DUlineblock}
&
\begin{DUlineblock}{0em}
\item[] 功
\item[] 能
\end{DUlineblock}

\sphinxAtStartPar
3  |
&
\sphinxAtStartPar
功 | 能 | |
&
\sphinxAtStartPar
功 | 能 | 5  |
&
\begin{DUlineblock}{0em}
\item[] 功
\item[] 能
|
\end{DUlineblock}
&
\begin{DUlineblock}{0em}
\item[] 功
\item[] 能
\end{DUlineblock}

\sphinxAtStartPar
7  |
&
\sphinxAtStartPar
其 他 | |
&
\sphinxAtStartPar
15 |
&
\begin{DUlineblock}{0em}
\item[] 
\end{DUlineblock}
\\
\sphinxhline
\sphinxAtStartPar
PD0
&
\sphinxAtStartPar
D0
&
\sphinxAtStartPar
UA RT0 RTS
&
\sphinxAtStartPar
PW M0 AN
&&&&&&&
\sphinxAtStartPar
L CD \_S EG 21
&
\sphinxAtStartPar
XLO
\\
\sphinxhline
\sphinxAtStartPar
PD1
&
\sphinxAtStartPar
D1
&
\sphinxAtStartPar
UA RT0 CTS
&
\sphinxAtStartPar
P WM 0A
&&&&&&&
\sphinxAtStartPar
L CD \_S EG 22
&\\
\sphinxhline
\sphinxAtStartPar
PD2
&
\sphinxAtStartPar
D2
&&&&&&&&&&
\sphinxAtStartPar
XI
\\
\sphinxhline
\sphinxAtStartPar
PD3
&
\sphinxAtStartPar
D3
&
\sphinxAtStartPar
PW M\_C LK1
&&&&&&&&&
\sphinxAtStartPar
XO
\\
\sphinxhline
\sphinxAtStartPar
PD4
&
\sphinxAtStartPar
D4
&
\sphinxAtStartPar
I 2C0 SCL
&
\sphinxAtStartPar
H AL L1
&&&&&&&
\sphinxAtStartPar
L CD \_S EG 23
&\\
\sphinxhline
\sphinxAtStartPar
PD5
&
\sphinxAtStartPar
D5
&
\sphinxAtStartPar
I 2C0 SDA
&
\sphinxAtStartPar
H AL L2
&&&&&&&
\sphinxAtStartPar
LC {\color{red}\bfseries{}D\_} CO M0
&\\
\sphinxhline
\sphinxAtStartPar
PD6
&
\sphinxAtStartPar
D6
&
\sphinxAtStartPar
I 2C0 SCL
&
\sphinxAtStartPar
UA RT 1R TS
&
\sphinxAtStartPar
CA N0 TX
&
\sphinxAtStartPar
P WM BR K0
&
\sphinxAtStartPar
H AL L0
&
\sphinxAtStartPar
LE {\color{red}\bfseries{}D\_} CO M0
&&&
\sphinxAtStartPar
LC {\color{red}\bfseries{}D\_} CO M1
&\\
\sphinxhline
\sphinxAtStartPar
PD7
&
\sphinxAtStartPar
D7
&
\sphinxAtStartPar
I 2C0 SDA
&
\sphinxAtStartPar
UA RT 1C TS
&
\sphinxAtStartPar
CA N0 RX
&
\sphinxAtStartPar
PW M2 AN
&
\sphinxAtStartPar
LE {\color{red}\bfseries{}D\_} CO M1
&&&&
\sphinxAtStartPar
LC {\color{red}\bfseries{}D\_} CO M2
&\\
\sphinxhline
\sphinxAtStartPar
PD8
&
\sphinxAtStartPar
D8
&
\sphinxAtStartPar
PW M2A
&
\sphinxAtStartPar
H AL L0
&
\sphinxAtStartPar
T 3I
&
\sphinxAtStartPar
T 3O
&
\sphinxAtStartPar
LE {\color{red}\bfseries{}D\_} CO M2
&&&&
\sphinxAtStartPar
LC {\color{red}\bfseries{}D\_} CO M3
&\\
\sphinxhline
\sphinxAtStartPar
PD9
&
\sphinxAtStartPar
D9
&
\sphinxAtStartPar
U ART 1TX
&
\sphinxAtStartPar
H AL L1
&
\sphinxAtStartPar
T 3I
&
\sphinxAtStartPar
T 3O
&
\sphinxAtStartPar
LE {\color{red}\bfseries{}D\_} CO M3
&&&&\begin{itemize}
\item {} 
\end{itemize}
&\\
\sphinxhline
\sphinxAtStartPar
PD10
&
\sphinxAtStartPar
D10
&
\sphinxAtStartPar
I 2C1 SCL
&
\sphinxAtStartPar
S PI 0C LK
&
\sphinxAtStartPar
P WM 2B
&
\sphinxAtStartPar
R TC \_1 HZ
&&&&&
\sphinxAtStartPar
L CD \_S EG 28
&\\
\sphinxhline
\sphinxAtStartPar
PD11
&
\sphinxAtStartPar
D11
&
\sphinxAtStartPar
I 2C1 SDA
&
\sphinxAtStartPar
SP I0 MI SO
&
\sphinxAtStartPar
U AR T3 RX
&
\sphinxAtStartPar
P WM 2A
&&&&&
\sphinxAtStartPar
L CD \_S EG 29
&\\
\sphinxhline
\sphinxAtStartPar
PD12
&
\sphinxAtStartPar
D12
&
\sphinxAtStartPar
SP I0M OSI
&
\sphinxAtStartPar
U AR T3 TX
&
\sphinxAtStartPar
P WM 2B
&
\sphinxAtStartPar
T 4I
&
\sphinxAtStartPar
T 4O
&&&&
\sphinxAtStartPar
L CD \_S EG 30
&\\
\sphinxhline
\sphinxAtStartPar
PD13
&
\sphinxAtStartPar
D13/   ADC0 \_CH9
&
\sphinxAtStartPar
I 2C0 SDA
&
\sphinxAtStartPar
SP I0 MI SO
&
\sphinxAtStartPar
U AR T0 RX
&
\sphinxAtStartPar
T 0O
&
\sphinxAtStartPar
LE {\color{red}\bfseries{}D\_} SE G8
&&&&
\sphinxAtStartPar
L CD \_S EG 15
&\\
\sphinxhline
\sphinxAtStartPar
PD14
&
\sphinxAtStartPar
D14/  A {\color{red}\bfseries{}DC0\_} CH10
&
\sphinxAtStartPar
SP I0M OSI
&
\sphinxAtStartPar
U AR T0 TX
&
\sphinxAtStartPar
PW M1 BN
&
\sphinxAtStartPar
T 0I
&&
\sphinxAtStartPar
LE {\color{red}\bfseries{}D\_} SE G9
&&&
\sphinxAtStartPar
L CD \_S EG 16
&\\
\sphinxhline
\sphinxAtStartPar
PD15
&
\sphinxAtStartPar
D15
&
\sphinxAtStartPar
SP I0S SEL
&
\sphinxAtStartPar
P WM 1B
&
\sphinxAtStartPar
T 0I
&
\sphinxAtStartPar
T 0O
&
\sphinxAtStartPar
L ED \_S EG 10
&&&&
\sphinxAtStartPar
L CD \_S EG 17
&\\
\sphinxbottomrule
\end{tabular}
\sphinxtableafterendhook\par
\sphinxattableend\end{savenotes}

\sphinxstepscope


\chapter{功能描述}
\label{\detokenize{SWM241/_u529f_u80fd_u63cf_u8ff0/index:id1}}\label{\detokenize{SWM241/_u529f_u80fd_u63cf_u8ff0/index::doc}}
\sphinxstepscope


\section{存储器映射}
\label{\detokenize{SWM241/_u529f_u80fd_u63cf_u8ff0/_u5b58_u50a8_u5668_u6620_u5c04:id1}}\label{\detokenize{SWM241/_u529f_u80fd_u63cf_u8ff0/_u5b58_u50a8_u5668_u6620_u5c04::doc}}
\sphinxAtStartPar
SWM241控制器为32位通用控制器,提供了4G字节寻址空间,如下表所示。数据格式仅支持小端格式(Little\sphinxhyphen{}Endian),各模块具体寄存器排布及操作说明在后章节有详细描述。

\sphinxAtStartPar
表格 6‑1存储器映射


\begin{savenotes}
\sphinxatlongtablestart
\sphinxthistablewithglobalstyle
\makeatletter
  \LTleft \@totalleftmargin plus1fill
  \LTright\dimexpr\columnwidth-\@totalleftmargin-\linewidth\relax plus1fill
\makeatother
\begin{longtable}{\X{33}{99}\X{33}{99}\X{33}{99}}
\sphinxtoprule
\endfirsthead

\multicolumn{3}{c}{\sphinxnorowcolor
    \makebox[0pt]{\sphinxtablecontinued{\tablename\ \thetable{} \textendash{} continued from previous page}}%
}\\
\sphinxtoprule
\endhead

\sphinxbottomrule
\multicolumn{3}{r}{\sphinxnorowcolor
    \makebox[0pt][r]{\sphinxtablecontinued{continues on next page}}%
}\\
\endfoot

\endlastfoot
\sphinxtableatstartofbodyhook

\sphinxAtStartPar
起始         |
&
\sphinxAtStartPar
结束          |
&
\sphinxAtStartPar
描述                         |
\\
\sphinxhline
\sphinxAtStartPar
存储器       |
&
\begin{DUlineblock}{0em}
\item[] 
\end{DUlineblock}
&
\begin{DUlineblock}{0em}
\item[] 
\end{DUlineblock}
\\
\sphinxhline
\sphinxAtStartPar
0x00000000
&\begin{itemize}
\item {} 
\end{itemize}
&
\sphinxAtStartPar
FLASH
\\
\sphinxhline
\sphinxAtStartPar
0x20000000
&\begin{itemize}
\item {} 
\end{itemize}
&
\sphinxAtStartPar
SRAM
\\
\sphinxhline
\sphinxAtStartPar
AHB总线外设  |
&
\begin{DUlineblock}{0em}
\item[] 
\end{DUlineblock}
&
\begin{DUlineblock}{0em}
\item[] 
\end{DUlineblock}
\\
\sphinxhline
\sphinxAtStartPar
0x40000000
&
\sphinxAtStartPar
0x400007FF
&
\sphinxAtStartPar
SYSCON
\\
\sphinxhline
\sphinxAtStartPar
0x40000800
&
\sphinxAtStartPar
0x40000FFF
&
\sphinxAtStartPar
DMA
\\
\sphinxhline
\sphinxAtStartPar
0x40001000
&
\sphinxAtStartPar
0x400017FF
&
\sphinxAtStartPar
INTCTRL
\\
\sphinxhline
\sphinxAtStartPar
0x40002800
&
\sphinxAtStartPar
0x40002FFF
&
\sphinxAtStartPar
CRC
\\
\sphinxhline
\sphinxAtStartPar
0x40003800
&
\sphinxAtStartPar
0x40003FFF
&
\sphinxAtStartPar
DIVIDER
\\
\sphinxhline
\sphinxAtStartPar
APB1总线外设 |
&
\begin{DUlineblock}{0em}
\item[] 
\end{DUlineblock}
&
\begin{DUlineblock}{0em}
\item[] 
\end{DUlineblock}
\\
\sphinxhline
\sphinxAtStartPar
0x40040000
&
\sphinxAtStartPar
0x400407FF
&
\sphinxAtStartPar
GPIOA
\\
\sphinxhline
\sphinxAtStartPar
0x40040800
&
\sphinxAtStartPar
0x40040FFF
&
\sphinxAtStartPar
GPIOB
\\
\sphinxhline
\sphinxAtStartPar
0x40041000
&
\sphinxAtStartPar
0x400417FF
&
\sphinxAtStartPar
GPIOC
\\
\sphinxhline
\sphinxAtStartPar
0x40041800
&
\sphinxAtStartPar
0x40041FFF
&
\sphinxAtStartPar
GPIOD
\\
\sphinxhline
\sphinxAtStartPar
0x40042000
&
\sphinxAtStartPar
0x400427FF
&
\sphinxAtStartPar
UART0
\\
\sphinxhline
\sphinxAtStartPar
0x40042800
&
\sphinxAtStartPar
0x40042FFF
&
\sphinxAtStartPar
UART1
\\
\sphinxhline
\sphinxAtStartPar
0x40043000
&
\sphinxAtStartPar
0x400437FF
&
\sphinxAtStartPar
UART2
\\
\sphinxhline
\sphinxAtStartPar
0x40043800
&
\sphinxAtStartPar
0x40043FFF
&
\sphinxAtStartPar
UART3
\\
\sphinxhline
\sphinxAtStartPar
0x40044000
&
\sphinxAtStartPar
0x400447FF
&
\sphinxAtStartPar
SPI0
\\
\sphinxhline
\sphinxAtStartPar
0x40044800
&
\sphinxAtStartPar
0x40044FFF
&
\sphinxAtStartPar
SPI1
\\
\sphinxhline
\sphinxAtStartPar
0x40046000
&
\sphinxAtStartPar
0x400467FF
&
\sphinxAtStartPar
PWM
\\
\sphinxhline
\sphinxAtStartPar
0x40046800
&
\sphinxAtStartPar
0x40046FFF
&
\sphinxAtStartPar
TIMER
\\
\sphinxhline
\sphinxAtStartPar
0x40049000
&
\sphinxAtStartPar
0x400497FF
&
\sphinxAtStartPar
SARADC0
\\
\sphinxhline
\sphinxAtStartPar
0x4004A000
&
\sphinxAtStartPar
0x4004A7FF
&
\sphinxAtStartPar
FLASHC
\\
\sphinxhline
\sphinxAtStartPar
0x4004B800
&
\sphinxAtStartPar
0x4004BFFF
&
\sphinxAtStartPar
RTC
\\
\sphinxhline
\sphinxAtStartPar
APB2总线外设 |
&
\begin{DUlineblock}{0em}
\item[] 
\end{DUlineblock}
&
\begin{DUlineblock}{0em}
\item[] 
\end{DUlineblock}
\\
\sphinxhline
\sphinxAtStartPar
0x400A0000
&
\sphinxAtStartPar
0x400A07FF
&
\sphinxAtStartPar
PORTCON
\\
\sphinxhline
\sphinxAtStartPar
0x400A0800
&
\sphinxAtStartPar
0x400A0FFF
&
\sphinxAtStartPar
WDT
\\
\sphinxhline
\sphinxAtStartPar
0x400A6000
&
\sphinxAtStartPar
0x400A67FF
&
\sphinxAtStartPar
I2C0
\\
\sphinxhline
\sphinxAtStartPar
0x400A6800
&
\sphinxAtStartPar
0x400A6FFF
&
\sphinxAtStartPar
I2C1
\\
\sphinxhline
\sphinxAtStartPar
0x400A8000
&
\sphinxAtStartPar
0x400A8FFF
&
\sphinxAtStartPar
CAN0
\\
\sphinxhline
\sphinxAtStartPar
0x400A9800
&
\sphinxAtStartPar
0x400A9FFF
&
\sphinxAtStartPar
SLCD
\\
\sphinxhline
\sphinxAtStartPar
0x400AA000
&
\sphinxAtStartPar
0x400AA7FF
&
\sphinxAtStartPar
ANACON
\\
\sphinxhline
\sphinxAtStartPar
0x400AA800
&
\sphinxAtStartPar
0x400AAFFF
&
\sphinxAtStartPar
SLED
\\
\sphinxhline
\sphinxAtStartPar
0x400AB000
&
\sphinxAtStartPar
0x400AB7FF
&
\sphinxAtStartPar
SAFETY
\\
\sphinxhline
\sphinxAtStartPar
核内部控制器 |
&
\begin{DUlineblock}{0em}
\item[] 
\end{DUlineblock}
&
\begin{DUlineblock}{0em}
\item[] 
\end{DUlineblock}
\\
\sphinxhline
\sphinxAtStartPar
0xE000E010
&
\sphinxAtStartPar
0xE000E01F
&
\sphinxAtStartPar
系统定时控制寄存器           |
\\
\sphinxhline
\sphinxAtStartPar
0xE000E100
&
\sphinxAtStartPar
0xE000E4EF
&
\sphinxAtStartPar
NVIC中断控制寄存器器         |
\\
\sphinxhline
\sphinxAtStartPar
0xE000ED00
&
\sphinxAtStartPar
0xE000ED3F
&
\sphinxAtStartPar
系统控制寄存器               |
\\
\sphinxbottomrule
\end{longtable}
\sphinxtableafterendhook
\sphinxatlongtableend
\end{savenotes}

\sphinxstepscope


\section{中断控制器(NVIC)}
\label{\detokenize{SWM241/_u529f_u80fd_u63cf_u8ff0/_u4e2d_u65ad_u63a7_u5236_u5668:nvic}}\label{\detokenize{SWM241/_u529f_u80fd_u63cf_u8ff0/_u4e2d_u65ad_u63a7_u5236_u5668::doc}}
\sphinxAtStartPar
概述
\textasciitilde{}\textasciitilde{}

\sphinxAtStartPar
Cortex\sphinxhyphen{}M0提供了“嵌套向量中断控制器(NVIC)”用以管理中断事件。

\sphinxAtStartPar
中断优先级分为4级,可通过中断优先级配置寄存器(IRQn)进行配置。中断发生时,内核比较中断优先级,并自动获取入口地址,并保护环境,将指定寄存器中数据入栈,无需软件参与。中断服务程序结束后,由硬件完成出栈工作。同时支持“尾链”模式及“迟至”模式,有效的优化了中断发生及背对背中断的执行效率,提高了中断
的实时性。

\sphinxAtStartPar
更多细节请参阅“Cortex®\sphinxhyphen{}M0 技术参考手册”及“ARM® CoreSight技术参考手册”。

\sphinxAtStartPar
特性
\textasciitilde{}\textasciitilde{}
\begin{itemize}
\item {} 
\sphinxAtStartPar
支持嵌套及向量中断

\item {} 
\sphinxAtStartPar
硬件完成现场的保存和恢复

\item {} 
\sphinxAtStartPar
动态改变优先级

\item {} 
\sphinxAtStartPar
确定的中断时间

\end{itemize}


\subsection{功能描述}
\label{\detokenize{SWM241/_u529f_u80fd_u63cf_u8ff0/_u4e2d_u65ad_u63a7_u5236_u5668:id1}}

\subsubsection{中断向量表}
\label{\detokenize{SWM241/_u529f_u80fd_u63cf_u8ff0/_u4e2d_u65ad_u63a7_u5236_u5668:id2}}
\sphinxAtStartPar
SWM241提供了32个中断供外设与核交互,其排列如表格 6‑2所示。可以通过中断配置模块,将任意模块或具体IO的中断连接至指定中断编号。具体使用参考中断配置模块。

\sphinxAtStartPar
表格 6‑2中断编号及对应外设


\begin{savenotes}
\sphinxatlongtablestart
\sphinxthistablewithglobalstyle
\makeatletter
  \LTleft \@totalleftmargin plus1fill
  \LTright\dimexpr\columnwidth-\@totalleftmargin-\linewidth\relax plus1fill
\makeatother
\begin{longtable}{\X{50}{100}\X{50}{100}}
\sphinxtoprule
\endfirsthead

\multicolumn{2}{c}{\sphinxnorowcolor
    \makebox[0pt]{\sphinxtablecontinued{\tablename\ \thetable{} \textendash{} continued from previous page}}%
}\\
\sphinxtoprule
\endhead

\sphinxbottomrule
\multicolumn{2}{r}{\sphinxnorowcolor
    \makebox[0pt][r]{\sphinxtablecontinued{continues on next page}}%
}\\
\endfoot

\endlastfoot
\sphinxtableatstartofbodyhook

\sphinxAtStartPar
中断(IRQ编号)
&
\sphinxAtStartPar
外设
\\
\sphinxhline
\sphinxAtStartPar
0
&
\sphinxAtStartPar
UART0
\\
\sphinxhline
\sphinxAtStartPar
1
&
\sphinxAtStartPar
TIMER0
\\
\sphinxhline
\sphinxAtStartPar
2
&
\sphinxAtStartPar
SPI0
\\
\sphinxhline
\sphinxAtStartPar
3
&
\sphinxAtStartPar
UART1
\\
\sphinxhline
\sphinxAtStartPar
4
&
\sphinxAtStartPar
UART2
\\
\sphinxhline
\sphinxAtStartPar
5
&
\sphinxAtStartPar
TIMER1
\\
\sphinxhline
\sphinxAtStartPar
6
&
\sphinxAtStartPar
DMA
\\
\sphinxhline
\sphinxAtStartPar
7
&
\sphinxAtStartPar
PWM\_CH0
\\
\sphinxhline
\sphinxAtStartPar
8
&
\sphinxAtStartPar
I2C1
\\
\sphinxhline
\sphinxAtStartPar
9
&
\sphinxAtStartPar
TIMER2
\\
\sphinxhline
\sphinxAtStartPar
10
&
\sphinxAtStartPar
TIMER3
\\
\sphinxhline
\sphinxAtStartPar
11
&
\sphinxAtStartPar
WDT
\\
\sphinxhline
\sphinxAtStartPar
12
&
\sphinxAtStartPar
I2C0
\\
\sphinxhline
\sphinxAtStartPar
13
&
\sphinxAtStartPar
UART3
\\
\sphinxhline
\sphinxAtStartPar
14
&
\sphinxAtStartPar
SARADC0
\\
\sphinxhline
\sphinxAtStartPar
15
&
\sphinxAtStartPar
TIMER4
\\
\sphinxhline
\sphinxAtStartPar
16
&
\sphinxAtStartPar
CAN/GPIOD1
\\
\sphinxhline
\sphinxAtStartPar
17
&
\sphinxAtStartPar
GPIOC2/GPIOB1
\\
\sphinxhline
\sphinxAtStartPar
18
&
\sphinxAtStartPar
GPIOC3/TIMER5
\\
\sphinxhline
\sphinxAtStartPar
19
&
\sphinxAtStartPar
GPIOD6/ GPIOA0
\\
\sphinxhline
\sphinxAtStartPar
20
&
\sphinxAtStartPar
TIMER6/GPIOC1
\\
\sphinxhline
\sphinxAtStartPar
21
&
\sphinxAtStartPar
GPIOD8/GPIOA1
\\
\sphinxhline
\sphinxAtStartPar
22
&
\sphinxAtStartPar
GPIOD9/GPIOB7
\\
\sphinxhline
\sphinxAtStartPar
23
&
\sphinxAtStartPar
GPIOB5/ GPIOD10
\\
\sphinxhline
\sphinxAtStartPar
24
&
\sphinxAtStartPar
GPIOD13/GPIOB2/GPIOA2
\\
\sphinxhline
\sphinxAtStartPar
25
&
\sphinxAtStartPar
TIMER7/ GPIOD12/XTAL\_STOP\_DET
\\
\sphinxhline
\sphinxAtStartPar
26
&
\sphinxAtStartPar
GPIOA /PWM\_CH1
\\
\sphinxhline
\sphinxAtStartPar
27
&
\sphinxAtStartPar
GPIOB /PWM\_CH2
\\
\sphinxhline
\sphinxAtStartPar
28
&
\sphinxAtStartPar
PWM\_HALT/GPIOD11/ BOD
\\
\sphinxhline
\sphinxAtStartPar
29
&
\sphinxAtStartPar
SAFETY/GPIOC/PWM\_CH3
\\
\sphinxhline
\sphinxAtStartPar
30
&
\sphinxAtStartPar
HALL/CAN/ SPI1
\\
\sphinxhline
\sphinxAtStartPar
31
&
\sphinxAtStartPar
RTC /GPIOD
\\
\sphinxhline
\sphinxAtStartPar
NMI
&
\sphinxAtStartPar
SYSTEM
\\
\sphinxbottomrule
\end{longtable}
\sphinxtableafterendhook
\sphinxatlongtableend
\end{savenotes}

\sphinxstepscope


\section{系统管理(SYSCON)}
\label{\detokenize{SWM241/_u529f_u80fd_u63cf_u8ff0/_u7cfb_u7edf_u7ba1_u7406:syscon}}\label{\detokenize{SWM241/_u529f_u80fd_u63cf_u8ff0/_u7cfb_u7edf_u7ba1_u7406::doc}}
\sphinxAtStartPar
概述
\textasciitilde{}\textasciitilde{}

\sphinxAtStartPar
系统管理为整个芯片提供时钟源,包括系统时钟切换、外设时钟门控、工作模式选择、数据备份以及版本控制等功能。还可通过单独时钟的开或关,时钟源选择来进行功耗控制。

\sphinxAtStartPar
特性
\textasciitilde{}\textasciitilde{}
\begin{itemize}
\item {} 
\sphinxAtStartPar
时钟控制

\item {} 
\sphinxAtStartPar
工作模式选择

\item {} 
\sphinxAtStartPar
休眠使能

\item {} 
\sphinxAtStartPar
RTC唤醒设置

\item {} 
\sphinxAtStartPar
端口唤醒设置

\item {} 
\sphinxAtStartPar
BOD掉电检测控制

\item {} 
\sphinxAtStartPar
复位控制及状态

\item {} 
\sphinxAtStartPar
UID

\end{itemize}


\subsection{模块结构框图}
\label{\detokenize{SWM241/_u529f_u80fd_u63cf_u8ff0/_u7cfb_u7edf_u7ba1_u7406:id1}}
\sphinxAtStartPar
\sphinxincludegraphics{{SWM241/功能描述/media系统管理002}.emf}

\sphinxAtStartPar
图 6‑1时钟结构框图

\begin{DUlineblock}{0em}
\item[] 注:系统时钟: \sphinxstyleemphasis{SYSCLK}; \sphinxstyleemphasis{AHB} 时钟: \sphinxstyleemphasis{HCLK}; \sphinxstyleemphasis{APB1} 时钟:
\sphinxstyleemphasis{PCLK1}; \sphinxstyleemphasis{APB2} 时钟: \sphinxstyleemphasis{PCLK2}。
\item[] \sphinxstyleemphasis{SYSCLK = HCLK = PCLK1 = 2* PCLK2}。
\end{DUlineblock}

\sphinxAtStartPar
例如:系统时钟 \sphinxstyleemphasis{SYSCLK} 选择的是 \sphinxstyleemphasis{48MHz},那么 \sphinxstyleemphasis{PCLK1=48MHz}, \sphinxstyleemphasis{PCLK2=24MHz}。


\subsection{功能描述}
\label{\detokenize{SWM241/_u529f_u80fd_u63cf_u8ff0/_u7cfb_u7edf_u7ba1_u7406:id2}}

\subsubsection{时钟控制}
\label{\detokenize{SWM241/_u529f_u80fd_u63cf_u8ff0/_u7cfb_u7edf_u7ba1_u7406:id3}}
\sphinxAtStartPar
SWM241有下列时钟源可供使用:
\begin{itemize}
\item {} 
\sphinxAtStartPar
内部高频振荡器(HRC):内部高频振荡器为片内时钟源,无需连接任何外部器件。频率为48MHz,通过HRCCR寄存器进行切换,可提供较精确的固定频率时钟.

\item {} 
\sphinxAtStartPar
内部低频振荡器(LRC):内部低频振荡器为片内时钟源,无需连接任何外部器件。频率为32KHz。

\item {} 
\sphinxAtStartPar
外部振荡器(XTAL):外部振荡器可接2\textasciitilde{}32MHz频率.

\item {} 
\sphinxAtStartPar
外部低频振荡器(XTAL\_32K):外部低频振荡器,支持32.768KHz时钟接入

\end{itemize}

\sphinxAtStartPar
对于主时钟选择,通过CLKSEL寄存器SYS位,选择内部高频时钟或其他时钟。

\sphinxAtStartPar
CLKSEL寄存器SYS位,选择内部高频时钟(HRC),时钟源为48MHz,此时CLKSEL寄存器CLK位无效。

\sphinxAtStartPar
CLKSEL 寄存器SYS位,选择CLK,通过CLK位可选择时钟源为片内高频RC振荡器(48MHz)、片外高频晶体振荡器(3\textasciitilde{}32MHz)、片外低频晶体振荡器(32.768KHz)、片内低频RC振荡器(32KHz),此时CLKSEL寄存器CLK\_DIVX位有效:
\begin{itemize}
\item {} 
\sphinxAtStartPar
CLK\_DIVX = 0时,选择内部HRC时钟不分频

\item {} 
\sphinxAtStartPar
CLK\_DIVX = 1时,选择内部HRC时钟8分频

\end{itemize}

\sphinxAtStartPar
当CLKSEL 寄存器CLK选择片外时钟。选择片外时钟前,需将相应引脚输入使能通过INEN\_x寄存器打开,并通过寄存器PORTx\_SEL将相应引脚换至外接晶振功能,且将XTALCR寄存器中外接晶振使能位使能。完成上述操作后,需根据外部晶振起振时间,使用软件产生一定时间,确保晶振稳定震荡,最后将CLK
SEL寄存器中相应位设置为片外震荡器。

\sphinxAtStartPar
注意:执行时钟切换时,需要保证目标时钟使能及通路打开,在内部HRC时钟相互切换时,需要先切换至32KHz时钟

\sphinxAtStartPar
对于ADC时钟,通过CLKSEL寄存器ADC、ADCDIV位配置:

\sphinxAtStartPar
通过CLKSEL.ADC选择SARADC时钟为片内高频RC振荡器(HRC:48MHz)、片外高频晶体振荡器(2\textasciitilde{}32MHz)。可通过CLKSEL.ADCDIV选择SARADC时钟源分频,可选不分频、4分频、8分频。

\sphinxAtStartPar
内部HRC及LRC可通过HRCCR寄存器ON位与LRCCR寄存器ON位进行关闭操作,关闭前需确认时钟已切换,并未使用即将执行关闭操作的时钟。

\sphinxAtStartPar
外设时钟控制功能可控制外设时钟打开及关闭,如:
\begin{itemize}
\item {} 
\sphinxAtStartPar
GPIO

\item {} 
\sphinxAtStartPar
SARADC

\item {} 
\sphinxAtStartPar
SLED

\item {} 
\sphinxAtStartPar
DIV

\item {} 
\sphinxAtStartPar
CRC

\item {} 
\sphinxAtStartPar
I2C

\item {} 
\sphinxAtStartPar
SPI

\item {} 
\sphinxAtStartPar
PWM

\item {} 
\sphinxAtStartPar
TIMER

\item {} 
\sphinxAtStartPar
WDT

\item {} 
\sphinxAtStartPar
UART

\item {} 
\sphinxAtStartPar
RTC

\end{itemize}

\sphinxAtStartPar
上电后,以上模块均处于时钟关闭状态,需要通过设置CLKEN0与CLKEN1寄存器进行时钟使能,否则访问对应模块寄存器操作无效。

\sphinxAtStartPar
复位
\textasciicircum{}\textasciicircum{}

\sphinxAtStartPar
复位源包括上电/BOD/WDT/外部复位引脚/低功耗管理复位/system reset及芯片各个模块独立软复位(ADC/RTC/DIV/I2C等模块)。

\sphinxAtStartPar
当以下事件中的一个发生时,产生一个系统复位,可复位芯片全局:
\begin{itemize}
\item {} 
\sphinxAtStartPar
上电复位

\item {} 
\sphinxAtStartPar
外部复位引脚复位

\item {} 
\sphinxAtStartPar
WDT看门狗计数复位

\item {} 
\sphinxAtStartPar
BOD掉电复位

\end{itemize}

\sphinxAtStartPar
可通过查看RSTSR复位状态寄存器中的复位状态标志位识别复位事件来源

\sphinxAtStartPar
各模块独立软复位只复位该模块。


\subsubsection{休眠与唤醒设置}
\label{\detokenize{SWM241/_u529f_u80fd_u63cf_u8ff0/_u7cfb_u7edf_u7ba1_u7406:id4}}
\sphinxAtStartPar
SWM241系列提供浅睡眠(SLEEP),通过SLEEP寄存器进行使能操作。

\sphinxAtStartPar
浅睡眠模式

\sphinxAtStartPar
浅睡眠模式下,芯片进入保持状态,所有时钟关闭,在功耗较低的前提下保持数据。可以通过配置任意I/O引脚进行唤醒操作,也可以通过RTC定时器进行唤醒操作,或者两种唤醒操作同时存在。IO唤醒操作同样为下降沿唤醒。唤醒后,程序从睡眠使能语句继续执行。

\sphinxAtStartPar
在sleep之前,需要将时钟切换为内部高频。

\sphinxAtStartPar
注意:浅睡眠模式使能前需保证LRC(32KHz)时钟为使能状态,且将所有不需要唤醒操作的IO输入使能关闭(PORTCON模块中INEN\_x寄存器)。

\sphinxAtStartPar
RTC唤醒

\sphinxAtStartPar
浅睡眠模式下,通过SYSCON模块中RTCWKSR寄存器及RTCWKCR寄存器进行定时器唤醒操作。流程如下:
\begin{itemize}
\item {} 
\sphinxAtStartPar
关闭所有不需要唤醒功能的IO输入使能(PORTCON模块中INEN\_x寄存器)

\item {} 
\sphinxAtStartPar
配置RTC时钟源及唤醒时间

\item {} 
\sphinxAtStartPar
使能唤醒源,设置RTCWKCR寄存器EN位为1 (使能前需通过写1清除RTCWKSR寄存器FLAG位)

\item {} 
\sphinxAtStartPar
使能RTC,RTC开始计数

\item {} 
\sphinxAtStartPar
SLEEP寄存器SLEEP位置1后,芯片进入浅睡眠模式,RTC计到设置值后唤醒芯片

\item {} 
\sphinxAtStartPar
唤醒后,RTCWKSR寄存器FLAG位为1(可通过对该位写1进行清除)

\end{itemize}

\sphinxAtStartPar
端口唤醒

\sphinxAtStartPar
浅睡眠模式下,可指定任意IO进行唤醒操作。示意图如图 6‑2所示。

\sphinxAtStartPar
具体流程如下:
\begin{itemize}
\item {} 
\sphinxAtStartPar
确认LRC(32KHz时钟)为使能状态

\item {} 
\sphinxAtStartPar
将需要执行唤醒操作的引脚对应PxWKEN寄存器及INEN\_x寄存器指定位配置为1,使能相应端口对应位输入使能及唤醒功能

\item {} 
\sphinxAtStartPar
SLEEP寄存器BIT{[}0{]} = 1后,芯片进入浅睡眠模式

\item {} 
\sphinxAtStartPar
唤醒端口可配置为GPIO端口,以及UART模块RX端口或I2C模块DAT端口,当配置端口对应位产生下降沿时,芯片被唤醒,继续执行程序。使用通讯接口进行唤醒时,需保证通讯采样速率低于32KHz,避免出现数据丢失现象

\item {} 
\sphinxAtStartPar
唤醒后,端口对应PxWKSR寄存器对应位被置1,可通过对该位写1进行清除(该位对进入休眠无影响)

\end{itemize}

\sphinxAtStartPar
注意:所有配置为唤醒功能的引脚,执行唤醒过程时只能有一个产生下降沿,对应引脚必须保证为高电平。为保证功耗最低,需确认所有输入使能引脚无悬空输入状态
\begin{quote}

\sphinxAtStartPar
\sphinxincludegraphics{{SWM241/功能描述/media系统管理003}.emf}
\end{quote}

\sphinxAtStartPar
图 6‑2端口唤醒示意图


\subsubsection{BOD掉电检测}
\label{\detokenize{SWM241/_u529f_u80fd_u63cf_u8ff0/_u7cfb_u7edf_u7ba1_u7406:bod}}
\sphinxAtStartPar
芯片提供了低电压中断及复位功能,置PERWP寄存器ANACFGR位,关闭写保护,并通过BODCR寄存器进行配置。

\sphinxAtStartPar
通过配置寄存器BODCR,可选择电压低压1.7V/1.9V/2.1V/2.7V/3.5V产生复位。当电压低于配置电压时,将产生低压复位。该功能为常开功能,系统复位后默认低压阈值为1.7V。

\sphinxAtStartPar
通过配置寄存器BODCR,可选择电压低压1.9V/2.1V/2.3V/2.5V/2.7V/3.5V/4.1V产生中断,当电压低于配置电压时,将产生BOD中断信号,通过查询BODSR寄存器IF位可以获取状态。

\sphinxAtStartPar
IF位为中断状态位,当芯片供电电压从配置电压以上变为低于配置电压时,该位将被置1(沿触发)。此时若IE位为非屏蔽状态(IE = 1),则NVIC控制器将接收到BOD中断。该中断可通过向IF位写1清除。清除后,若电压低于配置电压,IF 位会再次置位,再次产生中断。


\subsubsection{用户ID}
\label{\detokenize{SWM241/_u529f_u80fd_u63cf_u8ff0/_u7cfb_u7edf_u7ba1_u7406:id}}
\sphinxAtStartPar
芯片可以提供唯一96BIT ID号用于加密使用。


\subsection{寄存器映射}
\label{\detokenize{SWM241/_u529f_u80fd_u63cf_u8ff0/_u7cfb_u7edf_u7ba1_u7406:id5}}

\begin{savenotes}
\sphinxatlongtablestart
\sphinxthistablewithglobalstyle
\makeatletter
  \LTleft \@totalleftmargin plus1fill
  \LTright\dimexpr\columnwidth-\@totalleftmargin-\linewidth\relax plus1fill
\makeatother
\begin{longtable}{\X{20}{100}\X{20}{100}\X{20}{100}\X{20}{100}\X{20}{100}}
\sphinxtoprule
\endfirsthead

\multicolumn{5}{c}{\sphinxnorowcolor
    \makebox[0pt]{\sphinxtablecontinued{\tablename\ \thetable{} \textendash{} continued from previous page}}%
}\\
\sphinxtoprule
\endhead

\sphinxbottomrule
\multicolumn{5}{r}{\sphinxnorowcolor
    \makebox[0pt][r]{\sphinxtablecontinued{continues on next page}}%
}\\
\endfoot

\endlastfoot
\sphinxtableatstartofbodyhook

\sphinxAtStartPar
名称   |
&
\begin{DUlineblock}{0em}
\item[] 偏移 |
\end{DUlineblock}
&
\begin{DUlineblock}{0em}
\item[] 
\item[] |
|
\end{DUlineblock}
&
\begin{DUlineblock}{0em}
\item[] 
\end{DUlineblock}
\begin{quote}

\begin{DUlineblock}{0em}
\item[] 
\item[] 
\end{DUlineblock}
\end{quote}
&
\sphinxAtStartPar
描述                       | | | |
\\
\sphinxhline
\sphinxAtStartPar
SYSCONBASE:0 {\color{red}\bfseries{}|}x40000000
&
\begin{DUlineblock}{0em}
\item[] 
\end{DUlineblock}
&&&\\
\sphinxhline
\sphinxAtStartPar
CLKSEL
&
\sphinxAtStartPar
0x00
&&
\sphinxAtStartPar
0x C2801
&
\sphinxAtStartPar
时钟选择控制寄存器         |
\\
\sphinxhline
\sphinxAtStartPar
CLKDIVX\_ON
&
\sphinxAtStartPar
0x04
&&
\sphinxAtStartPar
0x 00000
&
\sphinxAtStartPar
源时钟控制寄存器           |
\\
\sphinxhline
\sphinxAtStartPar
CLKEN0
&
\sphinxAtStartPar
0x08
&&
\sphinxAtStartPar
0x 00000
&
\sphinxAtStartPar
时钟门控控制寄存器0        |
\\
\sphinxhline
\sphinxAtStartPar
CLKEN1
&
\sphinxAtStartPar
0x0C
&&
\sphinxAtStartPar
0x 00000
&
\sphinxAtStartPar
时钟门控控制寄存器1        |
\\
\sphinxhline
\sphinxAtStartPar
SLEEP
&
\sphinxAtStartPar
0x10
&&
\sphinxAtStartPar
0x 00000
&
\sphinxAtStartPar
系统模式控制寄存器         |
\\
\sphinxhline
\sphinxAtStartPar
RSTSR
&
\sphinxAtStartPar
0x024
&&
\sphinxAtStartPar
0x 00001
&
\sphinxAtStartPar
芯片复位状态寄存器         |
\\
\sphinxhline
\sphinxAtStartPar
RTCWKCR
&
\sphinxAtStartPar
0x30
&&
\sphinxAtStartPar
0x 00000
&
\sphinxAtStartPar
RTC唤醒使能控制寄存器      |
\\
\sphinxhline
\sphinxAtStartPar
RTCWKSR
&
\sphinxAtStartPar
0x34
&&
\sphinxAtStartPar
0x 00000
&
\sphinxAtStartPar
RTC唤醒标志寄存器          |
\\
\sphinxhline
\sphinxAtStartPar
CHIP\_ID0
&
\sphinxAtStartPar
0x80
&&
\sphinxAtStartPar
—
&
\sphinxAtStartPar
芯片128位ID寄存器0         |
\\
\sphinxhline
\sphinxAtStartPar
CHIP\_ID1
&
\sphinxAtStartPar
0x84
&&
\sphinxAtStartPar
—
&
\sphinxAtStartPar
芯片128位ID寄存器1         |
\\
\sphinxhline
\sphinxAtStartPar
CHIP\_ID2
&
\sphinxAtStartPar
0x88
&&
\sphinxAtStartPar
—
&
\sphinxAtStartPar
芯片128位ID寄存器2         |
\\
\sphinxhline
\sphinxAtStartPar
CHIP\_ID3
&
\sphinxAtStartPar
0x8C
&&
\sphinxAtStartPar
—
&
\sphinxAtStartPar
芯片128位ID寄存器3         |
\\
\sphinxhline
\sphinxAtStartPar
PRNGCR
&
\sphinxAtStartPar
0x0d0
&&
\sphinxAtStartPar
0x 00001
&
\sphinxAtStartPar
伪随机数控制寄存器         |
\\
\sphinxhline
\sphinxAtStartPar
PRNGDL
&
\sphinxAtStartPar
0x0d4
&&
\sphinxAtStartPar
0x 00000
&\\
\sphinxhline
\sphinxAtStartPar
PRNGDH
&
\sphinxAtStartPar
0x0d8
&&
\sphinxAtStartPar
0x 00000
&\\
\sphinxhline
\sphinxAtStartPar
PAWKEN
&
\sphinxAtStartPar
0x100
&&
\sphinxAtStartPar
0x 00000
&
\sphinxAtStartPar
PORTA唤醒使能控制寄存器    |
\\
\sphinxhline
\sphinxAtStartPar
PBWKEN
&
\sphinxAtStartPar
0x104
&&
\sphinxAtStartPar
0x 00000
&
\sphinxAtStartPar
PORTB唤醒使能控制寄存器    |
\\
\sphinxhline
\sphinxAtStartPar
PCWKEN
&
\sphinxAtStartPar
0x108
&&
\sphinxAtStartPar
0x 00000
&
\sphinxAtStartPar
PORTC唤醒使能控制寄存器    |
\\
\sphinxhline
\sphinxAtStartPar
PDWKEN
&
\sphinxAtStartPar
0x10C
&&
\sphinxAtStartPar
0x 00000
&
\sphinxAtStartPar
PORTD唤醒使能控制寄存器    |
\\
\sphinxhline
\sphinxAtStartPar
PAWKSR
&
\sphinxAtStartPar
0x130
&&
\sphinxAtStartPar
0x 00000
&
\sphinxAtStartPar
PORTA唤醒状态寄存器        |
\\
\sphinxhline
\sphinxAtStartPar
PBWKSR
&
\sphinxAtStartPar
0x134
&&
\sphinxAtStartPar
0x 00000
&
\sphinxAtStartPar
PORTB唤醒状态寄存器        |
\\
\sphinxhline
\sphinxAtStartPar
PCWKSR
&
\sphinxAtStartPar
0x138
&&
\sphinxAtStartPar
0x 00000
&
\sphinxAtStartPar
PORTC唤醒状态寄存器        |
\\
\sphinxhline
\sphinxAtStartPar
PDWKSR
&
\sphinxAtStartPar
0x13C
&&
\sphinxAtStartPar
0x 00000
&
\sphinxAtStartPar
PORTD唤醒状态寄存器        |
\\
\sphinxhline
\sphinxAtStartPar
PRSTEN
&
\sphinxAtStartPar
0x720
&&
\sphinxAtStartPar
0x 00000
&
\sphinxAtStartPar
芯片复位屏蔽寄存器         |
\\
\sphinxhline
\sphinxAtStartPar
PRSTR1
&
\sphinxAtStartPar
0x724
&&
\sphinxAtStartPar
0x 00000
&
\sphinxAtStartPar
芯片复位配置寄存器1        |
\\
\sphinxhline
\sphinxAtStartPar
PRSTR2
&
\sphinxAtStartPar
0x728
&&
\sphinxAtStartPar
0x 00000
&
\sphinxAtStartPar
芯片复位配置寄存器2        |
\\
\sphinxhline
\sphinxAtStartPar
ANACONBASE:0 {\color{red}\bfseries{}|}x400AA000
&
\begin{DUlineblock}{0em}
\item[] 
\end{DUlineblock}
&&&\\
\sphinxhline
\sphinxAtStartPar
HRCCR
&
\sphinxAtStartPar
0x00
&&
\sphinxAtStartPar
0x 00001
&
\sphinxAtStartPar
内部高频RC振荡器配置寄存器 |
\\
\sphinxhline
\sphinxAtStartPar
BODCR
&
\sphinxAtStartPar
0x10
&&
\sphinxAtStartPar
0x 00000
&
\sphinxAtStartPar
BOD控制寄存器              |
\\
\sphinxhline
\sphinxAtStartPar
BODSR
&
\sphinxAtStartPar
0x14
&&
\sphinxAtStartPar
0x 00000
&
\sphinxAtStartPar
BOD中断状态寄存器          |
\\
\sphinxhline
\sphinxAtStartPar
XTALCR
&
\sphinxAtStartPar
0x20
&&
\sphinxAtStartPar
0x 00000
&
\sphinxAtStartPar
晶体振荡器控制寄存器       |
\\
\sphinxhline
\sphinxAtStartPar
XTALSR
&
\sphinxAtStartPar
0x24
&&
\sphinxAtStartPar
0x 00000
&
\sphinxAtStartPar
晶体振荡器状态寄存器       |
\\
\sphinxhline
\sphinxAtStartPar
LRCCR
&
\sphinxAtStartPar
0x50
&&
\sphinxAtStartPar
0x 00001
&
\sphinxAtStartPar
芯片内部低频RC配置寄存器   |
\\
\sphinxbottomrule
\end{longtable}
\sphinxtableafterendhook
\sphinxatlongtableend
\end{savenotes}


\subsection{寄存器描述}
\label{\detokenize{SWM241/_u529f_u80fd_u63cf_u8ff0/_u7cfb_u7edf_u7ba1_u7406:id10}}

\subsubsection{时钟选择控制寄存器CLKSEL}
\label{\detokenize{SWM241/_u529f_u80fd_u63cf_u8ff0/_u7cfb_u7edf_u7ba1_u7406:clksel}}

\begin{savenotes}\sphinxattablestart
\sphinxthistablewithglobalstyle
\centering
\begin{tabular}[t]{\X{20}{100}\X{20}{100}\X{20}{100}\X{20}{100}\X{20}{100}}
\sphinxtoprule
\sphinxtableatstartofbodyhook
\sphinxAtStartPar
寄存器 |
&
\begin{DUlineblock}{0em}
\item[] 偏移 |
\end{DUlineblock}
&
\begin{DUlineblock}{0em}
\item[] 
\item[] {\color{red}\bfseries{}|}
\end{DUlineblock}
&
\sphinxAtStartPar
复位值 |    描 | |
&
\begin{DUlineblock}{0em}
\item[] |
  |
\end{DUlineblock}
\\
\sphinxhline
\sphinxAtStartPar
CLKSEL
&
\sphinxAtStartPar
0x00
&&
\sphinxAtStartPar
0 CC2801
&
\sphinxAtStartPar
时钟选择控制寄存器         |
\\
\sphinxbottomrule
\end{tabular}
\sphinxtableafterendhook\par
\sphinxattableend\end{savenotes}


\begin{savenotes}\sphinxattablestart
\sphinxthistablewithglobalstyle
\centering
\begin{tabular}[t]{\X{12}{96}\X{12}{96}\X{12}{96}\X{12}{96}\X{12}{96}\X{12}{96}\X{12}{96}\X{12}{96}}
\sphinxtoprule
\sphinxtableatstartofbodyhook
\sphinxAtStartPar
31
&
\sphinxAtStartPar
30
&
\sphinxAtStartPar
29
&
\sphinxAtStartPar
28
&
\sphinxAtStartPar
27
&
\sphinxAtStartPar
26
&
\sphinxAtStartPar
25
&
\sphinxAtStartPar
24
\\
\sphinxhline\begin{itemize}
\item {} 
\end{itemize}
&&&&&&&\\
\sphinxhline
\sphinxAtStartPar
23
&
\sphinxAtStartPar
22
&
\sphinxAtStartPar
21
&
\sphinxAtStartPar
20
&
\sphinxAtStartPar
19
&
\sphinxAtStartPar
18
&
\sphinxAtStartPar
17
&
\sphinxAtStartPar
16
\\
\sphinxhline\begin{itemize}
\item {} 
\end{itemize}
&&&&&&\begin{itemize}
\item {} 
\end{itemize}
&
\sphinxAtStartPar
ADC
\\
\sphinxhline
\sphinxAtStartPar
15
&
\sphinxAtStartPar
14
&
\sphinxAtStartPar
13
&
\sphinxAtStartPar
12
&
\sphinxAtStartPar
11
&
\sphinxAtStartPar
10
&
\sphinxAtStartPar
9
&
\sphinxAtStartPar
8
\\
\sphinxhline
\sphinxAtStartPar
RTCTRM
&&
\sphinxAtStartPar
WDT
&&\begin{itemize}
\item {} 
\end{itemize}
&&&\\
\sphinxhline
\sphinxAtStartPar
7
&
\sphinxAtStartPar
6
&
\sphinxAtStartPar
5
&
\sphinxAtStartPar
4
&
\sphinxAtStartPar
3
&
\sphinxAtStartPar
2
&
\sphinxAtStartPar
1
&
\sphinxAtStartPar
0
\\
\sphinxhline\begin{itemize}
\item {} 
\end{itemize}
&&
\sphinxAtStartPar
RTC
&
\sphinxAtStartPar
CLK
&&&&
\sphinxAtStartPar
SYS
\\
\sphinxbottomrule
\end{tabular}
\sphinxtableafterendhook\par
\sphinxattableend\end{savenotes}


\begin{savenotes}\sphinxattablestart
\sphinxthistablewithglobalstyle
\centering
\begin{tabular}[t]{\X{33}{99}\X{33}{99}\X{33}{99}}
\sphinxtoprule
\sphinxtableatstartofbodyhook
\sphinxAtStartPar
位域 |
&
\sphinxAtStartPar
名称     | |
&
\sphinxAtStartPar
描述                                        | |
\\
\sphinxhline
\sphinxAtStartPar
31:25
&\begin{itemize}
\item {} 
\end{itemize}
&\begin{itemize}
\item {} 
\end{itemize}
\\
\sphinxhline
\sphinxAtStartPar
24
&
\sphinxAtStartPar
WKUP
&
\sphinxAtStartPar
SLEEP唤醒时钟选择                           |

\sphinxAtStartPar
1:片外低频晶体振荡器(32.768KHz)          |

\sphinxAtStartPar
0:内部低频RC振荡器(32KHz)                |
\\
\sphinxhline
\sphinxAtStartPar
23:20
&\begin{itemize}
\item {} 
\end{itemize}
&\begin{itemize}
\item {} 
\end{itemize}
\\
\sphinxhline
\sphinxAtStartPar
19:18
&
\sphinxAtStartPar
ADCDIV
&
\sphinxAtStartPar
SARADC采样时钟选择,对所有SARADC均有效      |

\sphinxAtStartPar
0x:时钟源1分频                             |

\sphinxAtStartPar
10:时钟源4分频                             |

\sphinxAtStartPar
11:时钟源8分频                             |

\sphinxAtStartPar
注:SARADC采样时钟在进行不同源选择时,      | 将SARADC时钟使能关闭,再进行时钟源切换。 |
\\
\sphinxhline
\sphinxAtStartPar
17:16
&
\sphinxAtStartPar
ADC
&
\sphinxAtStartPar
SARADC时钟源选择,对所有SARADC均有效        |

\sphinxAtStartPar
01:片外高频晶体振荡器(2\textasciitilde{}32MHz)           |

\sphinxAtStartPar
00:片内高频RC振荡器(48MHz)               |

\sphinxAtStartPar
其他:保留                                  |
\\
\sphinxhline
\sphinxAtStartPar
15:14
&
\sphinxAtStartPar
RTCTRM
&
\sphinxAtStartPar
RTC TRIM参考时钟选择                        |

\sphinxAtStartPar
11:XTAL/8                                  |

\sphinxAtStartPar
10:XTAL/4                                  |

\sphinxAtStartPar
01:XTAL/2                                  |

\sphinxAtStartPar
00:片外高频晶体振荡器(XTAL)              |
\\
\sphinxhline
\sphinxAtStartPar
13:12
&
\sphinxAtStartPar
WDT
&
\sphinxAtStartPar
WDT计数时钟选择                             |

\sphinxAtStartPar
10:片内低频RC振荡器(32KHz)               |

\sphinxAtStartPar
其他:保留                                  |
\\
\sphinxhline
\sphinxAtStartPar
11:6
&\begin{itemize}
\item {} 
\end{itemize}
&\begin{itemize}
\item {} 
\end{itemize}
\\
\sphinxhline
\sphinxAtStartPar
5
&
\sphinxAtStartPar
RTC
&
\sphinxAtStartPar
32K时钟选择                                 |

\sphinxAtStartPar
1:片外低频晶体振荡器(32.768KHz)          |

\sphinxAtStartPar
0:内部低频RC振荡器(32KHz)                |
\\
\sphinxhline
\sphinxAtStartPar
4:2
&
\sphinxAtStartPar
CLK
&
\sphinxAtStartPar
CLK时钟选择                                 |

\sphinxAtStartPar
1xx:片内高频RC振荡器(HRC:48MHz)         |

\sphinxAtStartPar
011:片外高频晶体振荡器(XTAL:2\textasciitilde{}32MHz)    |

\sphinxAtStartPar
:片外低频晶体振荡器(XTAL\_32K:32.768KHz) |

\sphinxAtStartPar
001:保留                                   |

\sphinxAtStartPar
000:片内低频RC振荡器(LRC:32KHz)         |
\\
\sphinxhline
\sphinxAtStartPar
1
&
\sphinxAtStartPar
CLK\_DIVX
&
\sphinxAtStartPar
CLK分频选择                                 |

\sphinxAtStartPar
1:CLK/8分频                                |

\sphinxAtStartPar
0:CLK                                      |
\\
\sphinxhline
\sphinxAtStartPar
0
&
\sphinxAtStartPar
SYS
&
\sphinxAtStartPar
系统时钟选择                                |

\sphinxAtStartPar
1:HRC(48MHz)                             |

\sphinxAtStartPar
0:CLK                                      |

\sphinxAtStartPar
注:更改CLK或CLK\_DIVX设置时                 | 将此位先切换为1,再进行时钟源或分频切换  |
\\
\sphinxbottomrule
\end{tabular}
\sphinxtableafterendhook\par
\sphinxattableend\end{savenotes}


\subsubsection{源时钟选择控制寄存器CLKDIVX\_ON}
\label{\detokenize{SWM241/_u529f_u80fd_u63cf_u8ff0/_u7cfb_u7edf_u7ba1_u7406:clkdivx-on}}

\begin{savenotes}\sphinxattablestart
\sphinxthistablewithglobalstyle
\centering
\begin{tabular}[t]{\X{20}{100}\X{20}{100}\X{20}{100}\X{20}{100}\X{20}{100}}
\sphinxtoprule
\sphinxtableatstartofbodyhook
\sphinxAtStartPar
寄存器 |
&
\begin{DUlineblock}{0em}
\item[] 偏移 |
\end{DUlineblock}
&
\begin{DUlineblock}{0em}
\item[] 
\item[] {\color{red}\bfseries{}|}
\end{DUlineblock}
&
\sphinxAtStartPar
复位值 |    描 | |
&
\begin{DUlineblock}{0em}
\item[] |
  |
\end{DUlineblock}
\\
\sphinxhline
\sphinxAtStartPar
CLKDIVX\_ON
&
\sphinxAtStartPar
0x04
&&
\sphinxAtStartPar
0 000000
&
\sphinxAtStartPar
源时钟控制寄存器           |
\\
\sphinxbottomrule
\end{tabular}
\sphinxtableafterendhook\par
\sphinxattableend\end{savenotes}


\begin{savenotes}\sphinxattablestart
\sphinxthistablewithglobalstyle
\centering
\begin{tabular}[t]{\X{12}{96}\X{12}{96}\X{12}{96}\X{12}{96}\X{12}{96}\X{12}{96}\X{12}{96}\X{12}{96}}
\sphinxtoprule
\sphinxtableatstartofbodyhook
\sphinxAtStartPar
31
&
\sphinxAtStartPar
30
&
\sphinxAtStartPar
29
&
\sphinxAtStartPar
28
&
\sphinxAtStartPar
27
&
\sphinxAtStartPar
26
&
\sphinxAtStartPar
25
&
\sphinxAtStartPar
24
\\
\sphinxhline\begin{itemize}
\item {} 
\end{itemize}
&&&&&&&\\
\sphinxhline
\sphinxAtStartPar
23
&
\sphinxAtStartPar
22
&
\sphinxAtStartPar
21
&
\sphinxAtStartPar
20
&
\sphinxAtStartPar
19
&
\sphinxAtStartPar
18
&
\sphinxAtStartPar
17
&
\sphinxAtStartPar
16
\\
\sphinxhline\begin{itemize}
\item {} 
\end{itemize}
&&&&&&&\\
\sphinxhline
\sphinxAtStartPar
15
&
\sphinxAtStartPar
14
&
\sphinxAtStartPar
13
&
\sphinxAtStartPar
12
&
\sphinxAtStartPar
11
&
\sphinxAtStartPar
10
&
\sphinxAtStartPar
9
&
\sphinxAtStartPar
8
\\
\sphinxhline\begin{itemize}
\item {} 
\end{itemize}
&&&&&&&\\
\sphinxhline
\sphinxAtStartPar
7
&
\sphinxAtStartPar
6
&
\sphinxAtStartPar
5
&
\sphinxAtStartPar
4
&
\sphinxAtStartPar
3
&
\sphinxAtStartPar
2
&
\sphinxAtStartPar
1
&
\sphinxAtStartPar
0
\\
\sphinxhline\begin{itemize}
\item {} 
\end{itemize}
&&&&&&&\\
\sphinxbottomrule
\end{tabular}
\sphinxtableafterendhook\par
\sphinxattableend\end{savenotes}


\begin{savenotes}\sphinxattablestart
\sphinxthistablewithglobalstyle
\centering
\begin{tabular}[t]{\X{33}{99}\X{33}{99}\X{33}{99}}
\sphinxtoprule
\sphinxtableatstartofbodyhook
\sphinxAtStartPar
位域 |
&
\sphinxAtStartPar
名称     | |
&
\sphinxAtStartPar
描述                                        | |
\\
\sphinxhline
\sphinxAtStartPar
31:1
&\begin{itemize}
\item {} 
\end{itemize}
&\begin{itemize}
\item {} 
\end{itemize}
\\
\sphinxhline
\sphinxAtStartPar
0
&
\sphinxAtStartPar
CLKDIV\_ON
&
\sphinxAtStartPar
DIVCLK时钟门控                              |

\sphinxAtStartPar
1:关闭                                     |

\sphinxAtStartPar
0:打开                                     |

\sphinxAtStartPar
注:                                        | IV时,需保证此位为1,在关闭状态下进行更改 |

\sphinxAtStartPar
系统时钟选择不同时钟切换时,若需要在SRCDI | K或SRCCLK内部时钟源之间进行切换,则系统时钟 | 切换回HRC,然后将该位置为1后再进行切换。 |

\sphinxAtStartPar
3:若系统时钟已选择了HRC作为时钟源,并需要 | RC频率时,系统时钟需要先切至其他时钟源,  | 改变HRC频率,最后再将系统时钟切换回HRC。 |
\\
\sphinxbottomrule
\end{tabular}
\sphinxtableafterendhook\par
\sphinxattableend\end{savenotes}


\subsubsection{时钟门控控制寄存器0 CLKEN0}
\label{\detokenize{SWM241/_u529f_u80fd_u63cf_u8ff0/_u7cfb_u7edf_u7ba1_u7406:clken0}}

\begin{savenotes}\sphinxattablestart
\sphinxthistablewithglobalstyle
\centering
\begin{tabular}[t]{\X{20}{100}\X{20}{100}\X{20}{100}\X{20}{100}\X{20}{100}}
\sphinxtoprule
\sphinxtableatstartofbodyhook
\sphinxAtStartPar
寄存器 |
&
\begin{DUlineblock}{0em}
\item[] 偏移 |
\end{DUlineblock}
&
\begin{DUlineblock}{0em}
\item[] 
\item[] {\color{red}\bfseries{}|}
\end{DUlineblock}
&
\sphinxAtStartPar
复位值 |    描 | |
&
\begin{DUlineblock}{0em}
\item[] |
  |
\end{DUlineblock}
\\
\sphinxhline
\sphinxAtStartPar
CLKEN0
&
\sphinxAtStartPar
0x08
&&
\sphinxAtStartPar
0 000000
&
\sphinxAtStartPar
时钟门控控制寄存器0        |
\\
\sphinxbottomrule
\end{tabular}
\sphinxtableafterendhook\par
\sphinxattableend\end{savenotes}


\begin{savenotes}\sphinxattablestart
\sphinxthistablewithglobalstyle
\centering
\begin{tabular}[t]{\X{12}{96}\X{12}{96}\X{12}{96}\X{12}{96}\X{12}{96}\X{12}{96}\X{12}{96}\X{12}{96}}
\sphinxtoprule
\sphinxtableatstartofbodyhook
\sphinxAtStartPar
31
&
\sphinxAtStartPar
30
&
\sphinxAtStartPar
29
&
\sphinxAtStartPar
28
&
\sphinxAtStartPar
27
&
\sphinxAtStartPar
26
&
\sphinxAtStartPar
25
&
\sphinxAtStartPar
24
\\
\sphinxhline
\sphinxAtStartPar
SLED
&\begin{itemize}
\item {} 
\end{itemize}
&&
\sphinxAtStartPar
CAN
&&&&\begin{itemize}
\item {} 
\end{itemize}
\\
\sphinxhline
\sphinxAtStartPar
23
&
\sphinxAtStartPar
22
&
\sphinxAtStartPar
21
&
\sphinxAtStartPar
20
&
\sphinxAtStartPar
19
&
\sphinxAtStartPar
18
&
\sphinxAtStartPar
17
&
\sphinxAtStartPar
16
\\
\sphinxhline\begin{itemize}
\item {} 
\end{itemize}
&&
\sphinxAtStartPar
DIV
&\begin{itemize}
\item {} 
\end{itemize}
&
\sphinxAtStartPar
CRC
&\begin{itemize}
\item {} 
\end{itemize}
&&\\
\sphinxhline
\sphinxAtStartPar
15
&
\sphinxAtStartPar
14
&
\sphinxAtStartPar
13
&
\sphinxAtStartPar
12
&
\sphinxAtStartPar
11
&
\sphinxAtStartPar
10
&
\sphinxAtStartPar
9
&
\sphinxAtStartPar
8
\\
\sphinxhline
\sphinxAtStartPar
I2C0
&
\sphinxAtStartPar
SPI1
&&
\sphinxAtStartPar
PWM
&&
\sphinxAtStartPar
WDT
&&\\
\sphinxhline
\sphinxAtStartPar
7
&
\sphinxAtStartPar
6
&
\sphinxAtStartPar
5
&
\sphinxAtStartPar
4
&
\sphinxAtStartPar
3
&
\sphinxAtStartPar
2
&
\sphinxAtStartPar
1
&
\sphinxAtStartPar
0
\\
\sphinxhline
\sphinxAtStartPar
UART1
&
\sphinxAtStartPar
UART0
&\begin{itemize}
\item {} 
\end{itemize}
&&&&&\\
\sphinxbottomrule
\end{tabular}
\sphinxtableafterendhook\par
\sphinxattableend\end{savenotes}


\begin{savenotes}\sphinxattablestart
\sphinxthistablewithglobalstyle
\centering
\begin{tabular}[t]{\X{33}{99}\X{33}{99}\X{33}{99}}
\sphinxtoprule
\sphinxtableatstartofbodyhook
\sphinxAtStartPar
位域 |
&
\sphinxAtStartPar
名称     | |
&
\sphinxAtStartPar
描述                                        | |
\\
\sphinxhline
\sphinxAtStartPar
31
&
\sphinxAtStartPar
SLED
&
\sphinxAtStartPar
SLED时钟使能                                |
\\
\sphinxhline
\sphinxAtStartPar
30
&\begin{itemize}
\item {} 
\end{itemize}
&\begin{itemize}
\item {} 
\end{itemize}
\\
\sphinxhline
\sphinxAtStartPar
29
&
\sphinxAtStartPar
SLCD
&
\sphinxAtStartPar
SLCD模块时钟使能                            |
\\
\sphinxhline
\sphinxAtStartPar
28
&
\sphinxAtStartPar
CAN
&
\sphinxAtStartPar
CAN模块时钟使能                             |
\\
\sphinxhline
\sphinxAtStartPar
27
&\begin{itemize}
\item {} 
\end{itemize}
&\begin{itemize}
\item {} 
\end{itemize}
\\
\sphinxhline
\sphinxAtStartPar
26
&
\sphinxAtStartPar
ADC
&
\sphinxAtStartPar
SARADC数字控制时钟使能                      |
\\
\sphinxhline
\sphinxAtStartPar
25
&
\sphinxAtStartPar
ANAC
&
\sphinxAtStartPar
ANACON时钟使能                              |

\sphinxAtStartPar
注:包括HRC/BOD/XTAL/LRC/OPA/CMP时钟使能    |
\\
\sphinxhline
\sphinxAtStartPar
24:22
&\begin{itemize}
\item {} 
\end{itemize}
&\begin{itemize}
\item {} 
\end{itemize}
\\
\sphinxhline
\sphinxAtStartPar
21
&
\sphinxAtStartPar
DIV
&
\sphinxAtStartPar
DIVIDER时钟使能                             |
\\
\sphinxhline
\sphinxAtStartPar
20
&\begin{itemize}
\item {} 
\end{itemize}
&\begin{itemize}
\item {} 
\end{itemize}
\\
\sphinxhline
\sphinxAtStartPar
19
&
\sphinxAtStartPar
CRC
&
\sphinxAtStartPar
CRC时钟使能                                 |
\\
\sphinxhline
\sphinxAtStartPar
18:17
&\begin{itemize}
\item {} 
\end{itemize}
&\begin{itemize}
\item {} 
\end{itemize}
\\
\sphinxhline
\sphinxAtStartPar
16
&
\sphinxAtStartPar
I2C1
&
\sphinxAtStartPar
I2C1时钟使能                                |
\\
\sphinxhline
\sphinxAtStartPar
15
&
\sphinxAtStartPar
I2C0
&
\sphinxAtStartPar
I2C0时钟使能                                |
\\
\sphinxhline
\sphinxAtStartPar
14
&
\sphinxAtStartPar
SPI1
&
\sphinxAtStartPar
SPI1时钟使能                                |
\\
\sphinxhline
\sphinxAtStartPar
13
&
\sphinxAtStartPar
SPI0
&
\sphinxAtStartPar
SPI0时钟使能                                |
\\
\sphinxhline
\sphinxAtStartPar
12
&
\sphinxAtStartPar
PWM
&
\sphinxAtStartPar
PWM时钟使能                                 |
\\
\sphinxhline
\sphinxAtStartPar
11
&
\sphinxAtStartPar
TIMER
&
\sphinxAtStartPar
TIMER时钟使能                               |
\\
\sphinxhline
\sphinxAtStartPar
10
&
\sphinxAtStartPar
WDT
&
\sphinxAtStartPar
WDT时钟使能                                 |
\\
\sphinxhline
\sphinxAtStartPar
9
&
\sphinxAtStartPar
UART3
&
\sphinxAtStartPar
UART3时钟使能                               |
\\
\sphinxhline
\sphinxAtStartPar
8
&
\sphinxAtStartPar
UART2
&
\sphinxAtStartPar
UART2时钟使能                               |
\\
\sphinxhline
\sphinxAtStartPar
7
&
\sphinxAtStartPar
UART1
&
\sphinxAtStartPar
UART1时钟使能                               |
\\
\sphinxhline
\sphinxAtStartPar
6
&
\sphinxAtStartPar
UART0
&
\sphinxAtStartPar
UART0时钟使能                               |
\\
\sphinxhline
\sphinxAtStartPar
5:4
&\begin{itemize}
\item {} 
\end{itemize}
&\begin{itemize}
\item {} 
\end{itemize}
\\
\sphinxhline
\sphinxAtStartPar
3
&
\sphinxAtStartPar
GPIOD
&
\sphinxAtStartPar
GPIOD时钟使能                               |
\\
\sphinxhline
\sphinxAtStartPar
2
&
\sphinxAtStartPar
GPIOC
&
\sphinxAtStartPar
GPIOC时钟使能                               |
\\
\sphinxhline
\sphinxAtStartPar
1
&
\sphinxAtStartPar
GPIOB
&
\sphinxAtStartPar
GPIOB时钟使能                               |
\\
\sphinxhline
\sphinxAtStartPar
0
&
\sphinxAtStartPar
GPIOA
&
\sphinxAtStartPar
GPIOA时钟使能                               |
\\
\sphinxbottomrule
\end{tabular}
\sphinxtableafterendhook\par
\sphinxattableend\end{savenotes}


\subsubsection{时钟门控控制寄存器1 CLKEN1}
\label{\detokenize{SWM241/_u529f_u80fd_u63cf_u8ff0/_u7cfb_u7edf_u7ba1_u7406:clken1}}

\begin{savenotes}\sphinxattablestart
\sphinxthistablewithglobalstyle
\centering
\begin{tabular}[t]{\X{20}{100}\X{20}{100}\X{20}{100}\X{20}{100}\X{20}{100}}
\sphinxtoprule
\sphinxtableatstartofbodyhook
\sphinxAtStartPar
寄存器 |
&
\begin{DUlineblock}{0em}
\item[] 偏移 |
\end{DUlineblock}
&
\begin{DUlineblock}{0em}
\item[] 
\item[] {\color{red}\bfseries{}|}
\end{DUlineblock}
&
\sphinxAtStartPar
复位值 |    描 | |
&
\begin{DUlineblock}{0em}
\item[] |
  |
\end{DUlineblock}
\\
\sphinxhline
\sphinxAtStartPar
CLKEN1
&
\sphinxAtStartPar
0x0C
&&
\sphinxAtStartPar
0 000000
&
\sphinxAtStartPar
时钟门控控制寄存器1        |
\\
\sphinxbottomrule
\end{tabular}
\sphinxtableafterendhook\par
\sphinxattableend\end{savenotes}


\begin{savenotes}\sphinxattablestart
\sphinxthistablewithglobalstyle
\centering
\begin{tabular}[t]{\X{12}{96}\X{12}{96}\X{12}{96}\X{12}{96}\X{12}{96}\X{12}{96}\X{12}{96}\X{12}{96}}
\sphinxtoprule
\sphinxtableatstartofbodyhook
\sphinxAtStartPar
31
&
\sphinxAtStartPar
30
&
\sphinxAtStartPar
29
&
\sphinxAtStartPar
28
&
\sphinxAtStartPar
27
&
\sphinxAtStartPar
26
&
\sphinxAtStartPar
25
&
\sphinxAtStartPar
24
\\
\sphinxhline\begin{itemize}
\item {} 
\end{itemize}
&&&&&&&\\
\sphinxhline
\sphinxAtStartPar
23
&
\sphinxAtStartPar
22
&
\sphinxAtStartPar
21
&
\sphinxAtStartPar
20
&
\sphinxAtStartPar
19
&
\sphinxAtStartPar
18
&
\sphinxAtStartPar
17
&
\sphinxAtStartPar
16
\\
\sphinxhline\begin{itemize}
\item {} 
\end{itemize}
&&&&&\begin{itemize}
\item {} 
\end{itemize}
&&\\
\sphinxhline
\sphinxAtStartPar
15
&
\sphinxAtStartPar
14
&
\sphinxAtStartPar
13
&
\sphinxAtStartPar
12
&
\sphinxAtStartPar
11
&
\sphinxAtStartPar
10
&
\sphinxAtStartPar
9
&
\sphinxAtStartPar
8
\\
\sphinxhline\begin{itemize}
\item {} 
\end{itemize}
&&&&&&&\\
\sphinxhline
\sphinxAtStartPar
7
&
\sphinxAtStartPar
6
&
\sphinxAtStartPar
5
&
\sphinxAtStartPar
4
&
\sphinxAtStartPar
3
&
\sphinxAtStartPar
2
&
\sphinxAtStartPar
1
&
\sphinxAtStartPar
0
\\
\sphinxhline\begin{itemize}
\item {} 
\end{itemize}
&&&&&&&\\
\sphinxbottomrule
\end{tabular}
\sphinxtableafterendhook\par
\sphinxattableend\end{savenotes}


\begin{savenotes}\sphinxattablestart
\sphinxthistablewithglobalstyle
\centering
\begin{tabular}[t]{\X{33}{99}\X{33}{99}\X{33}{99}}
\sphinxtoprule
\sphinxtableatstartofbodyhook
\sphinxAtStartPar
位域 |
&
\sphinxAtStartPar
名称     | |
&
\sphinxAtStartPar
描述                                        | |
\\
\sphinxhline
\sphinxAtStartPar
31:20
&\begin{itemize}
\item {} 
\end{itemize}
&\begin{itemize}
\item {} 
\end{itemize}
\\
\sphinxhline
\sphinxAtStartPar
19
&
\sphinxAtStartPar
RTC
&
\sphinxAtStartPar
RTC时钟使能                                 |
\\
\sphinxhline
\sphinxAtStartPar
18:0
&\begin{itemize}
\item {} 
\end{itemize}
&\begin{itemize}
\item {} 
\end{itemize}
\\
\sphinxbottomrule
\end{tabular}
\sphinxtableafterendhook\par
\sphinxattableend\end{savenotes}

\sphinxAtStartPar
注:RTC时钟使能后, RTC模块使能寄存器置1,RTC在工作时钟使能下开始工作。


\subsubsection{系统模式控制寄存器SLEEP}
\label{\detokenize{SWM241/_u529f_u80fd_u63cf_u8ff0/_u7cfb_u7edf_u7ba1_u7406:sleep}}

\begin{savenotes}\sphinxattablestart
\sphinxthistablewithglobalstyle
\centering
\begin{tabular}[t]{\X{20}{100}\X{20}{100}\X{20}{100}\X{20}{100}\X{20}{100}}
\sphinxtoprule
\sphinxtableatstartofbodyhook
\sphinxAtStartPar
寄存器 |
&
\begin{DUlineblock}{0em}
\item[] 偏移 |
\end{DUlineblock}
&
\begin{DUlineblock}{0em}
\item[] 
\item[] {\color{red}\bfseries{}|}
\end{DUlineblock}
&
\sphinxAtStartPar
复位值 |    描 | |
&
\begin{DUlineblock}{0em}
\item[] |
  |
\end{DUlineblock}
\\
\sphinxhline
\sphinxAtStartPar
SLEEP
&
\sphinxAtStartPar
0x10
&&
\sphinxAtStartPar
0 000000
&
\sphinxAtStartPar
系统模式控制寄存器         |
\\
\sphinxbottomrule
\end{tabular}
\sphinxtableafterendhook\par
\sphinxattableend\end{savenotes}


\begin{savenotes}\sphinxattablestart
\sphinxthistablewithglobalstyle
\centering
\begin{tabular}[t]{\X{12}{96}\X{12}{96}\X{12}{96}\X{12}{96}\X{12}{96}\X{12}{96}\X{12}{96}\X{12}{96}}
\sphinxtoprule
\sphinxtableatstartofbodyhook
\sphinxAtStartPar
31
&
\sphinxAtStartPar
30
&
\sphinxAtStartPar
29
&
\sphinxAtStartPar
28
&
\sphinxAtStartPar
27
&
\sphinxAtStartPar
26
&
\sphinxAtStartPar
25
&
\sphinxAtStartPar
24
\\
\sphinxhline\begin{itemize}
\item {} 
\end{itemize}
&&&&&&&\\
\sphinxhline
\sphinxAtStartPar
23
&
\sphinxAtStartPar
22
&
\sphinxAtStartPar
21
&
\sphinxAtStartPar
20
&
\sphinxAtStartPar
19
&
\sphinxAtStartPar
18
&
\sphinxAtStartPar
17
&
\sphinxAtStartPar
16
\\
\sphinxhline\begin{itemize}
\item {} 
\end{itemize}
&&&&&&&\\
\sphinxhline
\sphinxAtStartPar
15
&
\sphinxAtStartPar
14
&
\sphinxAtStartPar
13
&
\sphinxAtStartPar
12
&
\sphinxAtStartPar
11
&
\sphinxAtStartPar
10
&
\sphinxAtStartPar
9
&
\sphinxAtStartPar
8
\\
\sphinxhline\begin{itemize}
\item {} 
\end{itemize}
&&&&&&&\\
\sphinxhline
\sphinxAtStartPar
7
&
\sphinxAtStartPar
6
&
\sphinxAtStartPar
5
&
\sphinxAtStartPar
4
&
\sphinxAtStartPar
3
&
\sphinxAtStartPar
2
&
\sphinxAtStartPar
1
&
\sphinxAtStartPar
0
\\
\sphinxhline\begin{itemize}
\item {} 
\end{itemize}
&&&&&&&\\
\sphinxbottomrule
\end{tabular}
\sphinxtableafterendhook\par
\sphinxattableend\end{savenotes}


\begin{savenotes}\sphinxattablestart
\sphinxthistablewithglobalstyle
\centering
\begin{tabular}[t]{\X{33}{99}\X{33}{99}\X{33}{99}}
\sphinxtoprule
\sphinxtableatstartofbodyhook
\sphinxAtStartPar
位域 |
&
\sphinxAtStartPar
名称     | |
&
\sphinxAtStartPar
描述                                        | |
\\
\sphinxhline
\sphinxAtStartPar
31:1
&\begin{itemize}
\item {} 
\end{itemize}
&\begin{itemize}
\item {} 
\end{itemize}
\\
\sphinxhline
\sphinxAtStartPar
0
&
\sphinxAtStartPar
SLEEP
&
\sphinxAtStartPar
将该位置1后,系统将进入SLEEP模式            |
\\
\sphinxbottomrule
\end{tabular}
\sphinxtableafterendhook\par
\sphinxattableend\end{savenotes}


\subsubsection{芯片复位状态寄存器RSTSR}
\label{\detokenize{SWM241/_u529f_u80fd_u63cf_u8ff0/_u7cfb_u7edf_u7ba1_u7406:rstsr}}

\begin{savenotes}\sphinxattablestart
\sphinxthistablewithglobalstyle
\centering
\begin{tabular}[t]{\X{20}{100}\X{20}{100}\X{20}{100}\X{20}{100}\X{20}{100}}
\sphinxtoprule
\sphinxtableatstartofbodyhook
\sphinxAtStartPar
寄存器 |
&
\begin{DUlineblock}{0em}
\item[] 偏移 |
\end{DUlineblock}
&
\begin{DUlineblock}{0em}
\item[] 
\item[] {\color{red}\bfseries{}|}
\end{DUlineblock}
&
\sphinxAtStartPar
复位值 |    描 | |
&
\begin{DUlineblock}{0em}
\item[] |
  |
\end{DUlineblock}
\\
\sphinxhline
\sphinxAtStartPar
RSTSR
&
\sphinxAtStartPar
0x024
&&
\sphinxAtStartPar
0 000001
&
\sphinxAtStartPar
芯片复位状态寄存器         |
\\
\sphinxbottomrule
\end{tabular}
\sphinxtableafterendhook\par
\sphinxattableend\end{savenotes}


\begin{savenotes}\sphinxattablestart
\sphinxthistablewithglobalstyle
\centering
\begin{tabular}[t]{\X{12}{96}\X{12}{96}\X{12}{96}\X{12}{96}\X{12}{96}\X{12}{96}\X{12}{96}\X{12}{96}}
\sphinxtoprule
\sphinxtableatstartofbodyhook
\sphinxAtStartPar
31
&
\sphinxAtStartPar
30
&
\sphinxAtStartPar
29
&
\sphinxAtStartPar
28
&
\sphinxAtStartPar
27
&
\sphinxAtStartPar
26
&
\sphinxAtStartPar
25
&
\sphinxAtStartPar
24
\\
\sphinxhline\begin{itemize}
\item {} 
\end{itemize}
&&&&&&&\\
\sphinxhline
\sphinxAtStartPar
23
&
\sphinxAtStartPar
22
&
\sphinxAtStartPar
21
&
\sphinxAtStartPar
20
&
\sphinxAtStartPar
19
&
\sphinxAtStartPar
18
&
\sphinxAtStartPar
17
&
\sphinxAtStartPar
16
\\
\sphinxhline\begin{itemize}
\item {} 
\end{itemize}
&&&&&&&\\
\sphinxhline
\sphinxAtStartPar
15
&
\sphinxAtStartPar
14
&
\sphinxAtStartPar
13
&
\sphinxAtStartPar
12
&
\sphinxAtStartPar
11
&
\sphinxAtStartPar
10
&
\sphinxAtStartPar
9
&
\sphinxAtStartPar
8
\\
\sphinxhline\begin{itemize}
\item {} 
\end{itemize}
&&&&&&&\\
\sphinxhline
\sphinxAtStartPar
7
&
\sphinxAtStartPar
6
&
\sphinxAtStartPar
5
&
\sphinxAtStartPar
4
&
\sphinxAtStartPar
3
&
\sphinxAtStartPar
2
&
\sphinxAtStartPar
1
&
\sphinxAtStartPar
0
\\
\sphinxhline\begin{itemize}
\item {} 
\end{itemize}
&&&
\sphinxAtStartPar
IAA
&\begin{itemize}
\item {} 
\end{itemize}
&&
\sphinxAtStartPar
WDT
&
\sphinxAtStartPar
POR
\\
\sphinxbottomrule
\end{tabular}
\sphinxtableafterendhook\par
\sphinxattableend\end{savenotes}


\begin{savenotes}\sphinxattablestart
\sphinxthistablewithglobalstyle
\centering
\begin{tabular}[t]{\X{33}{99}\X{33}{99}\X{33}{99}}
\sphinxtoprule
\sphinxtableatstartofbodyhook
\sphinxAtStartPar
位域 |
&
\sphinxAtStartPar
名称     | |
&
\sphinxAtStartPar
描述                                        | |
\\
\sphinxhline
\sphinxAtStartPar
31:5
&\begin{itemize}
\item {} 
\end{itemize}
&\begin{itemize}
\item {} 
\end{itemize}
\\
\sphinxhline
\sphinxAtStartPar
4
&
\sphinxAtStartPar
IAA
&
\sphinxAtStartPar
非法地址访问复位状态标志寄存器              |

\sphinxAtStartPar
1:表示出现非法地址访问复位                 |

\sphinxAtStartPar
写1清零                                     |
\\
\sphinxhline
\sphinxAtStartPar
3:2
&\begin{itemize}
\item {} 
\end{itemize}
&\begin{itemize}
\item {} 
\end{itemize}
\\
\sphinxhline
\sphinxAtStartPar
1
&
\sphinxAtStartPar
WDT
&
\sphinxAtStartPar
WDT复位状态标志寄存器,写1清零              |

\sphinxAtStartPar
1:出现WDT复位                              |

\sphinxAtStartPar
0:未出现WDT复位                            |
\\
\sphinxhline
\sphinxAtStartPar
0
&
\sphinxAtStartPar
POR
&
\sphinxAtStartPar
POR复位状态标志寄存器,写1清零              |

\sphinxAtStartPar
1:出现POR复位                              |

\sphinxAtStartPar
0:未出现POR复位                            |
\\
\sphinxbottomrule
\end{tabular}
\sphinxtableafterendhook\par
\sphinxattableend\end{savenotes}


\subsubsection{RTC唤醒使能控制寄存器RTCWKCR}
\label{\detokenize{SWM241/_u529f_u80fd_u63cf_u8ff0/_u7cfb_u7edf_u7ba1_u7406:rtcrtcwkcr}}

\begin{savenotes}\sphinxattablestart
\sphinxthistablewithglobalstyle
\centering
\begin{tabular}[t]{\X{20}{100}\X{20}{100}\X{20}{100}\X{20}{100}\X{20}{100}}
\sphinxtoprule
\sphinxtableatstartofbodyhook
\sphinxAtStartPar
寄存器 |
&
\begin{DUlineblock}{0em}
\item[] 偏移 |
\end{DUlineblock}
&
\begin{DUlineblock}{0em}
\item[] 
\item[] {\color{red}\bfseries{}|}
\end{DUlineblock}
&
\sphinxAtStartPar
复位值 |    描 | |
&
\begin{DUlineblock}{0em}
\item[] |
  |
\end{DUlineblock}
\\
\sphinxhline
\sphinxAtStartPar
RTCWKCR
&
\sphinxAtStartPar
0x30
&&
\sphinxAtStartPar
0 000000
&
\sphinxAtStartPar
RTC唤醒使能控制寄存器      |
\\
\sphinxbottomrule
\end{tabular}
\sphinxtableafterendhook\par
\sphinxattableend\end{savenotes}


\begin{savenotes}\sphinxattablestart
\sphinxthistablewithglobalstyle
\centering
\begin{tabular}[t]{\X{12}{96}\X{12}{96}\X{12}{96}\X{12}{96}\X{12}{96}\X{12}{96}\X{12}{96}\X{12}{96}}
\sphinxtoprule
\sphinxtableatstartofbodyhook
\sphinxAtStartPar
31
&
\sphinxAtStartPar
30
&
\sphinxAtStartPar
29
&
\sphinxAtStartPar
28
&
\sphinxAtStartPar
27
&
\sphinxAtStartPar
26
&
\sphinxAtStartPar
25
&
\sphinxAtStartPar
24
\\
\sphinxhline\begin{itemize}
\item {} 
\end{itemize}
&&&&&&&\\
\sphinxhline
\sphinxAtStartPar
23
&
\sphinxAtStartPar
22
&
\sphinxAtStartPar
21
&
\sphinxAtStartPar
20
&
\sphinxAtStartPar
19
&
\sphinxAtStartPar
18
&
\sphinxAtStartPar
17
&
\sphinxAtStartPar
16
\\
\sphinxhline\begin{itemize}
\item {} 
\end{itemize}
&&&&&&&\\
\sphinxhline
\sphinxAtStartPar
15
&
\sphinxAtStartPar
14
&
\sphinxAtStartPar
13
&
\sphinxAtStartPar
12
&
\sphinxAtStartPar
11
&
\sphinxAtStartPar
10
&
\sphinxAtStartPar
9
&
\sphinxAtStartPar
8
\\
\sphinxhline\begin{itemize}
\item {} 
\end{itemize}
&&&&&&&\\
\sphinxhline
\sphinxAtStartPar
7
&
\sphinxAtStartPar
6
&
\sphinxAtStartPar
5
&
\sphinxAtStartPar
4
&
\sphinxAtStartPar
3
&
\sphinxAtStartPar
2
&
\sphinxAtStartPar
1
&
\sphinxAtStartPar
0
\\
\sphinxhline\begin{itemize}
\item {} 
\end{itemize}
&&&&&&&
\sphinxAtStartPar
EN
\\
\sphinxbottomrule
\end{tabular}
\sphinxtableafterendhook\par
\sphinxattableend\end{savenotes}


\begin{savenotes}\sphinxattablestart
\sphinxthistablewithglobalstyle
\centering
\begin{tabular}[t]{\X{33}{99}\X{33}{99}\X{33}{99}}
\sphinxtoprule
\sphinxtableatstartofbodyhook
\sphinxAtStartPar
位域 |
&
\sphinxAtStartPar
名称     | |
&
\sphinxAtStartPar
描述                                        | |
\\
\sphinxhline
\sphinxAtStartPar
31:1
&\begin{itemize}
\item {} 
\end{itemize}
&\begin{itemize}
\item {} 
\end{itemize}
\\
\sphinxhline
\sphinxAtStartPar
0
&
\sphinxAtStartPar
EN
&
\sphinxAtStartPar
基本RTC唤醒使能寄存器                       |

\sphinxAtStartPar
1:使能基本RTC唤醒功能                      |

\sphinxAtStartPar
0:禁止基本RTC唤醒功能                      |
\\
\sphinxbottomrule
\end{tabular}
\sphinxtableafterendhook\par
\sphinxattableend\end{savenotes}


\subsubsection{RTC唤醒标志寄存器RTCWKSR}
\label{\detokenize{SWM241/_u529f_u80fd_u63cf_u8ff0/_u7cfb_u7edf_u7ba1_u7406:rtcrtcwksr}}

\begin{savenotes}\sphinxattablestart
\sphinxthistablewithglobalstyle
\centering
\begin{tabular}[t]{\X{20}{100}\X{20}{100}\X{20}{100}\X{20}{100}\X{20}{100}}
\sphinxtoprule
\sphinxtableatstartofbodyhook
\sphinxAtStartPar
寄存器 |
&
\begin{DUlineblock}{0em}
\item[] 偏移 |
\end{DUlineblock}
&
\begin{DUlineblock}{0em}
\item[] 
\item[] {\color{red}\bfseries{}|}
\end{DUlineblock}
&
\sphinxAtStartPar
复位值 |    描 | |
&
\begin{DUlineblock}{0em}
\item[] |
  |
\end{DUlineblock}
\\
\sphinxhline
\sphinxAtStartPar
RTCWKSR
&
\sphinxAtStartPar
0x34
&&
\sphinxAtStartPar
0 000000
&
\sphinxAtStartPar
RTC唤醒标志寄存器          |
\\
\sphinxbottomrule
\end{tabular}
\sphinxtableafterendhook\par
\sphinxattableend\end{savenotes}


\begin{savenotes}\sphinxattablestart
\sphinxthistablewithglobalstyle
\centering
\begin{tabular}[t]{\X{12}{96}\X{12}{96}\X{12}{96}\X{12}{96}\X{12}{96}\X{12}{96}\X{12}{96}\X{12}{96}}
\sphinxtoprule
\sphinxtableatstartofbodyhook
\sphinxAtStartPar
31
&
\sphinxAtStartPar
30
&
\sphinxAtStartPar
29
&
\sphinxAtStartPar
28
&
\sphinxAtStartPar
27
&
\sphinxAtStartPar
26
&
\sphinxAtStartPar
25
&
\sphinxAtStartPar
24
\\
\sphinxhline\begin{itemize}
\item {} 
\end{itemize}
&&&&&&&\\
\sphinxhline
\sphinxAtStartPar
23
&
\sphinxAtStartPar
22
&
\sphinxAtStartPar
21
&
\sphinxAtStartPar
20
&
\sphinxAtStartPar
19
&
\sphinxAtStartPar
18
&
\sphinxAtStartPar
17
&
\sphinxAtStartPar
16
\\
\sphinxhline\begin{itemize}
\item {} 
\end{itemize}
&&&&&&&\\
\sphinxhline
\sphinxAtStartPar
15
&
\sphinxAtStartPar
14
&
\sphinxAtStartPar
13
&
\sphinxAtStartPar
12
&
\sphinxAtStartPar
11
&
\sphinxAtStartPar
10
&
\sphinxAtStartPar
9
&
\sphinxAtStartPar
8
\\
\sphinxhline\begin{itemize}
\item {} 
\end{itemize}
&&&&&&&\\
\sphinxhline
\sphinxAtStartPar
7
&
\sphinxAtStartPar
6
&
\sphinxAtStartPar
5
&
\sphinxAtStartPar
4
&
\sphinxAtStartPar
3
&
\sphinxAtStartPar
2
&
\sphinxAtStartPar
1
&
\sphinxAtStartPar
0
\\
\sphinxhline\begin{itemize}
\item {} 
\end{itemize}
&&&&&&&\\
\sphinxbottomrule
\end{tabular}
\sphinxtableafterendhook\par
\sphinxattableend\end{savenotes}


\begin{savenotes}\sphinxattablestart
\sphinxthistablewithglobalstyle
\centering
\begin{tabular}[t]{\X{33}{99}\X{33}{99}\X{33}{99}}
\sphinxtoprule
\sphinxtableatstartofbodyhook
\sphinxAtStartPar
位域 |
&
\sphinxAtStartPar
名称     | |
&
\sphinxAtStartPar
描述                                        | |
\\
\sphinxhline
\sphinxAtStartPar
31:1
&\begin{itemize}
\item {} 
\end{itemize}
&\begin{itemize}
\item {} 
\end{itemize}
\\
\sphinxhline
\sphinxAtStartPar
0
&
\sphinxAtStartPar
FLAG
&
\sphinxAtStartPar
基本RTC唤醒标志,写1清除                    |

\sphinxAtStartPar
0:未产生唤醒标志                           |

\sphinxAtStartPar
1:已产生唤醒标志                           |
\\
\sphinxbottomrule
\end{tabular}
\sphinxtableafterendhook\par
\sphinxattableend\end{savenotes}


\subsubsection{芯片128位ID寄存器0CHIP\_ID0}
\label{\detokenize{SWM241/_u529f_u80fd_u63cf_u8ff0/_u7cfb_u7edf_u7ba1_u7406:id0chip-id0}}

\begin{savenotes}\sphinxattablestart
\sphinxthistablewithglobalstyle
\centering
\begin{tabular}[t]{\X{20}{100}\X{20}{100}\X{20}{100}\X{20}{100}\X{20}{100}}
\sphinxtoprule
\sphinxtableatstartofbodyhook
\sphinxAtStartPar
寄存器 |
&
\begin{DUlineblock}{0em}
\item[] 偏移 |
\end{DUlineblock}
&
\begin{DUlineblock}{0em}
\item[] 
\item[] {\color{red}\bfseries{}|}
\end{DUlineblock}
&
\sphinxAtStartPar
复位值 |    描 | |
&
\begin{DUlineblock}{0em}
\item[] |
  |
\end{DUlineblock}
\\
\sphinxhline
\sphinxAtStartPar
CHIP\_ID0
&
\sphinxAtStartPar
0x80
&&
\sphinxAtStartPar
—
&
\sphinxAtStartPar
芯片128位ID寄存器0         |
\\
\sphinxbottomrule
\end{tabular}
\sphinxtableafterendhook\par
\sphinxattableend\end{savenotes}


\begin{savenotes}\sphinxattablestart
\sphinxthistablewithglobalstyle
\centering
\begin{tabular}[t]{\X{12}{96}\X{12}{96}\X{12}{96}\X{12}{96}\X{12}{96}\X{12}{96}\X{12}{96}\X{12}{96}}
\sphinxtoprule
\sphinxtableatstartofbodyhook
\sphinxAtStartPar
31
&
\sphinxAtStartPar
30
&
\sphinxAtStartPar
29
&
\sphinxAtStartPar
28
&
\sphinxAtStartPar
27
&
\sphinxAtStartPar
26
&
\sphinxAtStartPar
25
&
\sphinxAtStartPar
24
\\
\sphinxhline
\sphinxAtStartPar
ID0
&&&&&&&\\
\sphinxhline
\sphinxAtStartPar
23
&
\sphinxAtStartPar
22
&
\sphinxAtStartPar
21
&
\sphinxAtStartPar
20
&
\sphinxAtStartPar
19
&
\sphinxAtStartPar
18
&
\sphinxAtStartPar
17
&
\sphinxAtStartPar
16
\\
\sphinxhline
\sphinxAtStartPar
ID0
&&&&&&&\\
\sphinxhline
\sphinxAtStartPar
15
&
\sphinxAtStartPar
14
&
\sphinxAtStartPar
13
&
\sphinxAtStartPar
12
&
\sphinxAtStartPar
11
&
\sphinxAtStartPar
10
&
\sphinxAtStartPar
9
&
\sphinxAtStartPar
8
\\
\sphinxhline
\sphinxAtStartPar
ID0
&&&&&&&\\
\sphinxhline
\sphinxAtStartPar
7
&
\sphinxAtStartPar
6
&
\sphinxAtStartPar
5
&
\sphinxAtStartPar
4
&
\sphinxAtStartPar
3
&
\sphinxAtStartPar
2
&
\sphinxAtStartPar
1
&
\sphinxAtStartPar
0
\\
\sphinxhline
\sphinxAtStartPar
ID0
&&&&&&&\\
\sphinxbottomrule
\end{tabular}
\sphinxtableafterendhook\par
\sphinxattableend\end{savenotes}


\begin{savenotes}\sphinxattablestart
\sphinxthistablewithglobalstyle
\centering
\begin{tabular}[t]{\X{33}{99}\X{33}{99}\X{33}{99}}
\sphinxtoprule
\sphinxtableatstartofbodyhook
\sphinxAtStartPar
位域 |
&
\sphinxAtStartPar
名称     | |
&
\sphinxAtStartPar
描述                                        | |
\\
\sphinxhline
\sphinxAtStartPar
31:0
&
\sphinxAtStartPar
ID0
&
\sphinxAtStartPar
芯片128位ID寄存器0                          |
\\
\sphinxbottomrule
\end{tabular}
\sphinxtableafterendhook\par
\sphinxattableend\end{savenotes}


\subsubsection{芯片128位ID寄存器1CHIP\_ID1}
\label{\detokenize{SWM241/_u529f_u80fd_u63cf_u8ff0/_u7cfb_u7edf_u7ba1_u7406:id1chip-id1}}

\begin{savenotes}\sphinxattablestart
\sphinxthistablewithglobalstyle
\centering
\begin{tabular}[t]{\X{20}{100}\X{20}{100}\X{20}{100}\X{20}{100}\X{20}{100}}
\sphinxtoprule
\sphinxtableatstartofbodyhook
\sphinxAtStartPar
寄存器 |
&
\begin{DUlineblock}{0em}
\item[] 偏移 |
\end{DUlineblock}
&
\begin{DUlineblock}{0em}
\item[] 
\item[] {\color{red}\bfseries{}|}
\end{DUlineblock}
&
\sphinxAtStartPar
复位值 |    描 | |
&
\begin{DUlineblock}{0em}
\item[] |
  |
\end{DUlineblock}
\\
\sphinxhline
\sphinxAtStartPar
CHIP\_ID1
&
\sphinxAtStartPar
0x84
&&
\sphinxAtStartPar
—
&
\sphinxAtStartPar
芯片128位ID寄存器1         |
\\
\sphinxbottomrule
\end{tabular}
\sphinxtableafterendhook\par
\sphinxattableend\end{savenotes}


\begin{savenotes}\sphinxattablestart
\sphinxthistablewithglobalstyle
\centering
\begin{tabular}[t]{\X{12}{96}\X{12}{96}\X{12}{96}\X{12}{96}\X{12}{96}\X{12}{96}\X{12}{96}\X{12}{96}}
\sphinxtoprule
\sphinxtableatstartofbodyhook
\sphinxAtStartPar
31
&
\sphinxAtStartPar
30
&
\sphinxAtStartPar
29
&
\sphinxAtStartPar
28
&
\sphinxAtStartPar
27
&
\sphinxAtStartPar
26
&
\sphinxAtStartPar
25
&
\sphinxAtStartPar
24
\\
\sphinxhline
\sphinxAtStartPar
ID1
&&&&&&&\\
\sphinxhline
\sphinxAtStartPar
23
&
\sphinxAtStartPar
22
&
\sphinxAtStartPar
21
&
\sphinxAtStartPar
20
&
\sphinxAtStartPar
19
&
\sphinxAtStartPar
18
&
\sphinxAtStartPar
17
&
\sphinxAtStartPar
16
\\
\sphinxhline
\sphinxAtStartPar
ID1
&&&&&&&\\
\sphinxhline
\sphinxAtStartPar
15
&
\sphinxAtStartPar
14
&
\sphinxAtStartPar
13
&
\sphinxAtStartPar
12
&
\sphinxAtStartPar
11
&
\sphinxAtStartPar
10
&
\sphinxAtStartPar
9
&
\sphinxAtStartPar
8
\\
\sphinxhline
\sphinxAtStartPar
ID1
&&&&&&&\\
\sphinxhline
\sphinxAtStartPar
7
&
\sphinxAtStartPar
6
&
\sphinxAtStartPar
5
&
\sphinxAtStartPar
4
&
\sphinxAtStartPar
3
&
\sphinxAtStartPar
2
&
\sphinxAtStartPar
1
&
\sphinxAtStartPar
0
\\
\sphinxhline
\sphinxAtStartPar
ID1
&&&&&&&\\
\sphinxbottomrule
\end{tabular}
\sphinxtableafterendhook\par
\sphinxattableend\end{savenotes}


\begin{savenotes}\sphinxattablestart
\sphinxthistablewithglobalstyle
\centering
\begin{tabular}[t]{\X{33}{99}\X{33}{99}\X{33}{99}}
\sphinxtoprule
\sphinxtableatstartofbodyhook
\sphinxAtStartPar
位域 |
&
\sphinxAtStartPar
名称     | |
&
\sphinxAtStartPar
描述                                        | |
\\
\sphinxhline
\sphinxAtStartPar
31:0
&
\sphinxAtStartPar
ID1
&
\sphinxAtStartPar
芯片128位ID寄存器1                          |
\\
\sphinxbottomrule
\end{tabular}
\sphinxtableafterendhook\par
\sphinxattableend\end{savenotes}


\subsubsection{芯片128位ID寄存器2CHIP\_ID2}
\label{\detokenize{SWM241/_u529f_u80fd_u63cf_u8ff0/_u7cfb_u7edf_u7ba1_u7406:id2chip-id2}}

\begin{savenotes}\sphinxattablestart
\sphinxthistablewithglobalstyle
\centering
\begin{tabular}[t]{\X{20}{100}\X{20}{100}\X{20}{100}\X{20}{100}\X{20}{100}}
\sphinxtoprule
\sphinxtableatstartofbodyhook
\sphinxAtStartPar
寄存器 |
&
\begin{DUlineblock}{0em}
\item[] 偏移 |
\end{DUlineblock}
&
\begin{DUlineblock}{0em}
\item[] 
\item[] {\color{red}\bfseries{}|}
\end{DUlineblock}
&
\sphinxAtStartPar
复位值 |    描 | |
&
\begin{DUlineblock}{0em}
\item[] |
  |
\end{DUlineblock}
\\
\sphinxhline
\sphinxAtStartPar
CHIP\_ID2
&
\sphinxAtStartPar
0x88
&&
\sphinxAtStartPar
—
&
\sphinxAtStartPar
芯片128位ID寄存器2         |
\\
\sphinxbottomrule
\end{tabular}
\sphinxtableafterendhook\par
\sphinxattableend\end{savenotes}


\begin{savenotes}\sphinxattablestart
\sphinxthistablewithglobalstyle
\centering
\begin{tabular}[t]{\X{12}{96}\X{12}{96}\X{12}{96}\X{12}{96}\X{12}{96}\X{12}{96}\X{12}{96}\X{12}{96}}
\sphinxtoprule
\sphinxtableatstartofbodyhook
\sphinxAtStartPar
31
&
\sphinxAtStartPar
30
&
\sphinxAtStartPar
29
&
\sphinxAtStartPar
28
&
\sphinxAtStartPar
27
&
\sphinxAtStartPar
26
&
\sphinxAtStartPar
25
&
\sphinxAtStartPar
24
\\
\sphinxhline
\sphinxAtStartPar
ID2
&&&&&&&\\
\sphinxhline
\sphinxAtStartPar
23
&
\sphinxAtStartPar
22
&
\sphinxAtStartPar
21
&
\sphinxAtStartPar
20
&
\sphinxAtStartPar
19
&
\sphinxAtStartPar
18
&
\sphinxAtStartPar
17
&
\sphinxAtStartPar
16
\\
\sphinxhline
\sphinxAtStartPar
ID2
&&&&&&&\\
\sphinxhline
\sphinxAtStartPar
15
&
\sphinxAtStartPar
14
&
\sphinxAtStartPar
13
&
\sphinxAtStartPar
12
&
\sphinxAtStartPar
11
&
\sphinxAtStartPar
10
&
\sphinxAtStartPar
9
&
\sphinxAtStartPar
8
\\
\sphinxhline
\sphinxAtStartPar
ID2
&&&&&&&\\
\sphinxhline
\sphinxAtStartPar
7
&
\sphinxAtStartPar
6
&
\sphinxAtStartPar
5
&
\sphinxAtStartPar
4
&
\sphinxAtStartPar
3
&
\sphinxAtStartPar
2
&
\sphinxAtStartPar
1
&
\sphinxAtStartPar
0
\\
\sphinxhline
\sphinxAtStartPar
ID2
&&&&&&&\\
\sphinxbottomrule
\end{tabular}
\sphinxtableafterendhook\par
\sphinxattableend\end{savenotes}


\begin{savenotes}\sphinxattablestart
\sphinxthistablewithglobalstyle
\centering
\begin{tabular}[t]{\X{33}{99}\X{33}{99}\X{33}{99}}
\sphinxtoprule
\sphinxtableatstartofbodyhook
\sphinxAtStartPar
位域 |
&
\sphinxAtStartPar
名称     | |
&
\sphinxAtStartPar
描述                                        | |
\\
\sphinxhline
\sphinxAtStartPar
31:0
&
\sphinxAtStartPar
ID2
&
\sphinxAtStartPar
芯片128位ID寄存器2                          |
\\
\sphinxbottomrule
\end{tabular}
\sphinxtableafterendhook\par
\sphinxattableend\end{savenotes}


\subsubsection{芯片128位ID寄存器3CHIP\_ID3}
\label{\detokenize{SWM241/_u529f_u80fd_u63cf_u8ff0/_u7cfb_u7edf_u7ba1_u7406:id3chip-id3}}

\begin{savenotes}\sphinxattablestart
\sphinxthistablewithglobalstyle
\centering
\begin{tabular}[t]{\X{20}{100}\X{20}{100}\X{20}{100}\X{20}{100}\X{20}{100}}
\sphinxtoprule
\sphinxtableatstartofbodyhook
\sphinxAtStartPar
寄存器 |
&
\begin{DUlineblock}{0em}
\item[] 偏移 |
\end{DUlineblock}
&
\begin{DUlineblock}{0em}
\item[] 
\item[] {\color{red}\bfseries{}|}
\end{DUlineblock}
&
\sphinxAtStartPar
复位值 |    描 | |
&
\begin{DUlineblock}{0em}
\item[] |
  |
\end{DUlineblock}
\\
\sphinxhline
\sphinxAtStartPar
CHIP\_ID3
&
\sphinxAtStartPar
0x8C
&&
\sphinxAtStartPar
—
&
\sphinxAtStartPar
芯片128位ID寄存器3         |
\\
\sphinxbottomrule
\end{tabular}
\sphinxtableafterendhook\par
\sphinxattableend\end{savenotes}


\begin{savenotes}\sphinxattablestart
\sphinxthistablewithglobalstyle
\centering
\begin{tabular}[t]{\X{12}{96}\X{12}{96}\X{12}{96}\X{12}{96}\X{12}{96}\X{12}{96}\X{12}{96}\X{12}{96}}
\sphinxtoprule
\sphinxtableatstartofbodyhook
\sphinxAtStartPar
31
&
\sphinxAtStartPar
30
&
\sphinxAtStartPar
29
&
\sphinxAtStartPar
28
&
\sphinxAtStartPar
27
&
\sphinxAtStartPar
26
&
\sphinxAtStartPar
25
&
\sphinxAtStartPar
24
\\
\sphinxhline
\sphinxAtStartPar
ID3
&&&&&&&\\
\sphinxhline
\sphinxAtStartPar
23
&
\sphinxAtStartPar
22
&
\sphinxAtStartPar
21
&
\sphinxAtStartPar
20
&
\sphinxAtStartPar
19
&
\sphinxAtStartPar
18
&
\sphinxAtStartPar
17
&
\sphinxAtStartPar
16
\\
\sphinxhline
\sphinxAtStartPar
ID3
&&&&&&&\\
\sphinxhline
\sphinxAtStartPar
15
&
\sphinxAtStartPar
14
&
\sphinxAtStartPar
13
&
\sphinxAtStartPar
12
&
\sphinxAtStartPar
11
&
\sphinxAtStartPar
10
&
\sphinxAtStartPar
9
&
\sphinxAtStartPar
8
\\
\sphinxhline
\sphinxAtStartPar
ID3
&&&&&&&\\
\sphinxhline
\sphinxAtStartPar
7
&
\sphinxAtStartPar
6
&
\sphinxAtStartPar
5
&
\sphinxAtStartPar
4
&
\sphinxAtStartPar
3
&
\sphinxAtStartPar
2
&
\sphinxAtStartPar
1
&
\sphinxAtStartPar
0
\\
\sphinxhline
\sphinxAtStartPar
ID3
&&&&&&&\\
\sphinxbottomrule
\end{tabular}
\sphinxtableafterendhook\par
\sphinxattableend\end{savenotes}


\begin{savenotes}\sphinxattablestart
\sphinxthistablewithglobalstyle
\centering
\begin{tabular}[t]{\X{33}{99}\X{33}{99}\X{33}{99}}
\sphinxtoprule
\sphinxtableatstartofbodyhook
\sphinxAtStartPar
位域 |
&
\sphinxAtStartPar
名称     | |
&
\sphinxAtStartPar
描述                                        | |
\\
\sphinxhline
\sphinxAtStartPar
31:0
&
\sphinxAtStartPar
ID3
&
\sphinxAtStartPar
芯片128位ID寄存器3                          |
\\
\sphinxbottomrule
\end{tabular}
\sphinxtableafterendhook\par
\sphinxattableend\end{savenotes}

\sphinxAtStartPar
\sphinxstylestrong{伪随机数控制寄存器PRNGCR}


\begin{savenotes}\sphinxattablestart
\sphinxthistablewithglobalstyle
\centering
\begin{tabular}[t]{\X{20}{100}\X{20}{100}\X{20}{100}\X{20}{100}\X{20}{100}}
\sphinxtoprule
\sphinxtableatstartofbodyhook
\sphinxAtStartPar
寄存器 |
&
\begin{DUlineblock}{0em}
\item[] 偏移 |
\end{DUlineblock}
&
\begin{DUlineblock}{0em}
\item[] 
\item[] {\color{red}\bfseries{}|}
\end{DUlineblock}
&
\sphinxAtStartPar
复位值 |    描 | |
&
\begin{DUlineblock}{0em}
\item[] |
  |
\end{DUlineblock}
\\
\sphinxhline
\sphinxAtStartPar
PRNGCR
&
\sphinxAtStartPar
0x0d0
&&
\sphinxAtStartPar
0 000001
&
\sphinxAtStartPar
伪随机数控制寄存器         |
\\
\sphinxbottomrule
\end{tabular}
\sphinxtableafterendhook\par
\sphinxattableend\end{savenotes}


\begin{savenotes}\sphinxattablestart
\sphinxthistablewithglobalstyle
\centering
\begin{tabular}[t]{\X{12}{96}\X{12}{96}\X{12}{96}\X{12}{96}\X{12}{96}\X{12}{96}\X{12}{96}\X{12}{96}}
\sphinxtoprule
\sphinxtableatstartofbodyhook
\sphinxAtStartPar
31
&
\sphinxAtStartPar
30
&
\sphinxAtStartPar
29
&
\sphinxAtStartPar
28
&
\sphinxAtStartPar
27
&
\sphinxAtStartPar
26
&
\sphinxAtStartPar
25
&
\sphinxAtStartPar
24
\\
\sphinxhline\begin{itemize}
\item {} 
\end{itemize}
&&&&&&&\\
\sphinxhline
\sphinxAtStartPar
23
&
\sphinxAtStartPar
22
&
\sphinxAtStartPar
21
&
\sphinxAtStartPar
20
&
\sphinxAtStartPar
19
&
\sphinxAtStartPar
18
&
\sphinxAtStartPar
17
&
\sphinxAtStartPar
16
\\
\sphinxhline\begin{itemize}
\item {} 
\end{itemize}
&&&&&&&\\
\sphinxhline
\sphinxAtStartPar
15
&
\sphinxAtStartPar
14
&
\sphinxAtStartPar
13
&
\sphinxAtStartPar
12
&
\sphinxAtStartPar
11
&
\sphinxAtStartPar
10
&
\sphinxAtStartPar
9
&
\sphinxAtStartPar
8
\\
\sphinxhline\begin{itemize}
\item {} 
\end{itemize}
&&&&&&&
\sphinxAtStartPar
RDY
\\
\sphinxhline
\sphinxAtStartPar
7
&
\sphinxAtStartPar
6
&
\sphinxAtStartPar
5
&
\sphinxAtStartPar
4
&
\sphinxAtStartPar
3
&
\sphinxAtStartPar
2
&
\sphinxAtStartPar
1
&
\sphinxAtStartPar
0
\\
\sphinxhline\begin{itemize}
\item {} 
\end{itemize}
&&&&&
\sphinxAtStartPar
CLK
&&
\sphinxAtStartPar
S CLR
\\
\sphinxbottomrule
\end{tabular}
\sphinxtableafterendhook\par
\sphinxattableend\end{savenotes}


\begin{savenotes}\sphinxattablestart
\sphinxthistablewithglobalstyle
\centering
\begin{tabular}[t]{\X{33}{99}\X{33}{99}\X{33}{99}}
\sphinxtoprule
\sphinxtableatstartofbodyhook
\sphinxAtStartPar
位域 |
&
\sphinxAtStartPar
名称     | |
&
\sphinxAtStartPar
描述                                        | |
\\
\sphinxhline
\sphinxAtStartPar
31:9
&\begin{itemize}
\item {} 
\end{itemize}
&\begin{itemize}
\item {} 
\end{itemize}
\\
\sphinxhline
\sphinxAtStartPar
8
&
\sphinxAtStartPar
RDY
&
\sphinxAtStartPar
PRNG随机数准备标志,RO                      |

\sphinxAtStartPar
当检测到该信号为1时                         | 以读取PRNG\_DATAL和PRNG\_DATAH,并且每次需 | RNG\_DATAL和PRNG\_DATAH两个寄存器一并读出。 |
\\
\sphinxhline
\sphinxAtStartPar
7:3
&\begin{itemize}
\item {} 
\end{itemize}
&\begin{itemize}
\item {} 
\end{itemize}
\\
\sphinxhline
\sphinxAtStartPar
2:1
&
\sphinxAtStartPar
CLK
&
\sphinxAtStartPar
PRNG随机数发生器时钟配置                    |

\sphinxAtStartPar
Bit1:为0时,所有时钟无效;为1时,正常工作  |

\sphinxAtStartPar
Bit0:为0时,三个时钟工作模式(HRC、LR      | TAL);为1时,两个时钟工作模式(HRC、LRC) |
\\
\sphinxhline
\sphinxAtStartPar
0
&
\sphinxAtStartPar
SEEDCLR
&
\sphinxAtStartPar
PRNG种子清零寄存器                          |

\sphinxAtStartPar
1:种子清零。此时随机数发生器不工作         |

\sphinxAtStartPar
0:随机数发生器工作                         |

\sphinxAtStartPar
注:该信号                                  | 效,则其为高的时间不能短于LRC一个周期。  |
\\
\sphinxbottomrule
\end{tabular}
\sphinxtableafterendhook\par
\sphinxattableend\end{savenotes}

\sphinxAtStartPar
\sphinxstylestrong{伪随机数输出寄存器低32位数据PRNGDL}


\begin{savenotes}\sphinxattablestart
\sphinxthistablewithglobalstyle
\centering
\begin{tabular}[t]{\X{20}{100}\X{20}{100}\X{20}{100}\X{20}{100}\X{20}{100}}
\sphinxtoprule
\sphinxtableatstartofbodyhook
\sphinxAtStartPar
寄存器 |
&
\begin{DUlineblock}{0em}
\item[] 偏移 |
\end{DUlineblock}
&
\begin{DUlineblock}{0em}
\item[] 
\item[] {\color{red}\bfseries{}|}
\end{DUlineblock}
&
\sphinxAtStartPar
复位值 |    描 | |
&
\begin{DUlineblock}{0em}
\item[] |
  |
\end{DUlineblock}
\\
\sphinxhline
\sphinxAtStartPar
PRNGDL
&
\sphinxAtStartPar
0x0d4
&&
\sphinxAtStartPar
0 000000
&\\
\sphinxbottomrule
\end{tabular}
\sphinxtableafterendhook\par
\sphinxattableend\end{savenotes}


\begin{savenotes}\sphinxattablestart
\sphinxthistablewithglobalstyle
\centering
\begin{tabular}[t]{\X{12}{96}\X{12}{96}\X{12}{96}\X{12}{96}\X{12}{96}\X{12}{96}\X{12}{96}\X{12}{96}}
\sphinxtoprule
\sphinxtableatstartofbodyhook
\sphinxAtStartPar
31
&
\sphinxAtStartPar
30
&
\sphinxAtStartPar
29
&
\sphinxAtStartPar
28
&
\sphinxAtStartPar
27
&
\sphinxAtStartPar
26
&
\sphinxAtStartPar
25
&
\sphinxAtStartPar
24
\\
\sphinxhline
\sphinxAtStartPar
DATAL
&&&&&&&\\
\sphinxhline
\sphinxAtStartPar
23
&
\sphinxAtStartPar
22
&
\sphinxAtStartPar
21
&
\sphinxAtStartPar
20
&
\sphinxAtStartPar
19
&
\sphinxAtStartPar
18
&
\sphinxAtStartPar
17
&
\sphinxAtStartPar
16
\\
\sphinxhline
\sphinxAtStartPar
DATAL
&&&&&&&\\
\sphinxhline
\sphinxAtStartPar
15
&
\sphinxAtStartPar
14
&
\sphinxAtStartPar
13
&
\sphinxAtStartPar
12
&
\sphinxAtStartPar
11
&
\sphinxAtStartPar
10
&
\sphinxAtStartPar
9
&
\sphinxAtStartPar
8
\\
\sphinxhline
\sphinxAtStartPar
DATAL
&&&&&&&\\
\sphinxhline
\sphinxAtStartPar
7
&
\sphinxAtStartPar
6
&
\sphinxAtStartPar
5
&
\sphinxAtStartPar
4
&
\sphinxAtStartPar
3
&
\sphinxAtStartPar
2
&
\sphinxAtStartPar
1
&
\sphinxAtStartPar
0
\\
\sphinxhline
\sphinxAtStartPar
DATAL
&&&&&&&\\
\sphinxbottomrule
\end{tabular}
\sphinxtableafterendhook\par
\sphinxattableend\end{savenotes}


\begin{savenotes}\sphinxattablestart
\sphinxthistablewithglobalstyle
\centering
\begin{tabular}[t]{\X{33}{99}\X{33}{99}\X{33}{99}}
\sphinxtoprule
\sphinxtableatstartofbodyhook
\sphinxAtStartPar
位域 |
&
\sphinxAtStartPar
名称     | |
&
\sphinxAtStartPar
描述                                        | |
\\
\sphinxhline
\sphinxAtStartPar
31:0
&
\sphinxAtStartPar
DATAL
&
\sphinxAtStartPar
随机数输出寄存器低32位数据                  |
\\
\sphinxbottomrule
\end{tabular}
\sphinxtableafterendhook\par
\sphinxattableend\end{savenotes}

\sphinxAtStartPar
\sphinxstylestrong{伪随机数输出寄存器高32位数据PRNGDH}


\begin{savenotes}\sphinxattablestart
\sphinxthistablewithglobalstyle
\centering
\begin{tabular}[t]{\X{20}{100}\X{20}{100}\X{20}{100}\X{20}{100}\X{20}{100}}
\sphinxtoprule
\sphinxtableatstartofbodyhook
\sphinxAtStartPar
寄存器 |
&
\begin{DUlineblock}{0em}
\item[] 偏移 |
\end{DUlineblock}
&
\begin{DUlineblock}{0em}
\item[] 
\item[] {\color{red}\bfseries{}|}
\end{DUlineblock}
&
\sphinxAtStartPar
复位值 |    描 | |
&
\begin{DUlineblock}{0em}
\item[] |
  |
\end{DUlineblock}
\\
\sphinxhline
\sphinxAtStartPar
PRNGDH
&
\sphinxAtStartPar
0x0d8
&&
\sphinxAtStartPar
0 000000
&\\
\sphinxbottomrule
\end{tabular}
\sphinxtableafterendhook\par
\sphinxattableend\end{savenotes}


\begin{savenotes}\sphinxattablestart
\sphinxthistablewithglobalstyle
\centering
\begin{tabular}[t]{\X{12}{96}\X{12}{96}\X{12}{96}\X{12}{96}\X{12}{96}\X{12}{96}\X{12}{96}\X{12}{96}}
\sphinxtoprule
\sphinxtableatstartofbodyhook
\sphinxAtStartPar
31
&
\sphinxAtStartPar
30
&
\sphinxAtStartPar
29
&
\sphinxAtStartPar
28
&
\sphinxAtStartPar
27
&
\sphinxAtStartPar
26
&
\sphinxAtStartPar
25
&
\sphinxAtStartPar
24
\\
\sphinxhline\begin{itemize}
\item {} 
\end{itemize}
&
\sphinxAtStartPar
DADAH
&&&&&&\\
\sphinxhline
\sphinxAtStartPar
23
&
\sphinxAtStartPar
22
&
\sphinxAtStartPar
21
&
\sphinxAtStartPar
20
&
\sphinxAtStartPar
19
&
\sphinxAtStartPar
18
&
\sphinxAtStartPar
17
&
\sphinxAtStartPar
16
\\
\sphinxhline
\sphinxAtStartPar
DADAH
&&&&&&&\\
\sphinxhline
\sphinxAtStartPar
15
&
\sphinxAtStartPar
14
&
\sphinxAtStartPar
13
&
\sphinxAtStartPar
12
&
\sphinxAtStartPar
11
&
\sphinxAtStartPar
10
&
\sphinxAtStartPar
9
&
\sphinxAtStartPar
8
\\
\sphinxhline
\sphinxAtStartPar
DADAH
&&&&&&&\\
\sphinxhline
\sphinxAtStartPar
7
&
\sphinxAtStartPar
6
&
\sphinxAtStartPar
5
&
\sphinxAtStartPar
4
&
\sphinxAtStartPar
3
&
\sphinxAtStartPar
2
&
\sphinxAtStartPar
1
&
\sphinxAtStartPar
0
\\
\sphinxhline
\sphinxAtStartPar
DADAH
&&&&&&&\\
\sphinxbottomrule
\end{tabular}
\sphinxtableafterendhook\par
\sphinxattableend\end{savenotes}


\begin{savenotes}\sphinxattablestart
\sphinxthistablewithglobalstyle
\centering
\begin{tabular}[t]{\X{33}{99}\X{33}{99}\X{33}{99}}
\sphinxtoprule
\sphinxtableatstartofbodyhook
\sphinxAtStartPar
位域 |
&
\sphinxAtStartPar
名称     | |
&
\sphinxAtStartPar
描述                                        | |
\\
\sphinxhline
\sphinxAtStartPar
31
&\begin{itemize}
\item {} 
\end{itemize}
&\begin{itemize}
\item {} 
\end{itemize}
\\
\sphinxhline
\sphinxAtStartPar
30:0
&
\sphinxAtStartPar
DATAH
&
\sphinxAtStartPar
随机数输出寄存器高31位数据                  |
\\
\sphinxbottomrule
\end{tabular}
\sphinxtableafterendhook\par
\sphinxattableend\end{savenotes}


\subsubsection{PORTA唤醒使能控制寄存器PAWKEN}
\label{\detokenize{SWM241/_u529f_u80fd_u63cf_u8ff0/_u7cfb_u7edf_u7ba1_u7406:portapawken}}

\begin{savenotes}\sphinxattablestart
\sphinxthistablewithglobalstyle
\centering
\begin{tabular}[t]{\X{20}{100}\X{20}{100}\X{20}{100}\X{20}{100}\X{20}{100}}
\sphinxtoprule
\sphinxtableatstartofbodyhook
\sphinxAtStartPar
寄存器 |
&
\begin{DUlineblock}{0em}
\item[] 偏移 |
\end{DUlineblock}
&
\begin{DUlineblock}{0em}
\item[] 
\item[] {\color{red}\bfseries{}|}
\end{DUlineblock}
&
\sphinxAtStartPar
复位值 |    描 | |
&
\begin{DUlineblock}{0em}
\item[] |
  |
\end{DUlineblock}
\\
\sphinxhline
\sphinxAtStartPar
PAWKEN
&
\sphinxAtStartPar
0x100
&&
\sphinxAtStartPar
0 000000
&
\sphinxAtStartPar
PORTA唤醒使能控制寄存器    |
\\
\sphinxbottomrule
\end{tabular}
\sphinxtableafterendhook\par
\sphinxattableend\end{savenotes}


\begin{savenotes}\sphinxattablestart
\sphinxthistablewithglobalstyle
\centering
\begin{tabular}[t]{\X{12}{96}\X{12}{96}\X{12}{96}\X{12}{96}\X{12}{96}\X{12}{96}\X{12}{96}\X{12}{96}}
\sphinxtoprule
\sphinxtableatstartofbodyhook
\sphinxAtStartPar
31
&
\sphinxAtStartPar
30
&
\sphinxAtStartPar
29
&
\sphinxAtStartPar
28
&
\sphinxAtStartPar
27
&
\sphinxAtStartPar
26
&
\sphinxAtStartPar
25
&
\sphinxAtStartPar
24
\\
\sphinxhline\begin{itemize}
\item {} 
\end{itemize}
&&&&&&&\\
\sphinxhline
\sphinxAtStartPar
23
&
\sphinxAtStartPar
22
&
\sphinxAtStartPar
21
&
\sphinxAtStartPar
20
&
\sphinxAtStartPar
19
&
\sphinxAtStartPar
18
&
\sphinxAtStartPar
17
&
\sphinxAtStartPar
16
\\
\sphinxhline\begin{itemize}
\item {} 
\end{itemize}
&&&&&&&\\
\sphinxhline
\sphinxAtStartPar
15
&
\sphinxAtStartPar
14
&
\sphinxAtStartPar
13
&
\sphinxAtStartPar
12
&
\sphinxAtStartPar
11
&
\sphinxAtStartPar
10
&
\sphinxAtStartPar
9
&
\sphinxAtStartPar
8
\\
\sphinxhline
\sphinxAtStartPar
PAWKEN15
&
\sphinxAtStartPar
PA WKEN14
&
\sphinxAtStartPar
PA N13
&
\sphinxAtStartPar
PA N12
&&
\sphinxAtStartPar
PA N10
&
\sphinxAtStartPar
PA N9
&
\sphinxAtStartPar
P EN8
\\
\sphinxhline
\sphinxAtStartPar
7
&
\sphinxAtStartPar
6
&
\sphinxAtStartPar
5
&
\sphinxAtStartPar
4
&
\sphinxAtStartPar
3
&
\sphinxAtStartPar
2
&
\sphinxAtStartPar
1
&
\sphinxAtStartPar
0
\\
\sphinxhline
\sphinxAtStartPar
PAWKEN7
&
\sphinxAtStartPar
P AWKEN6
&
\sphinxAtStartPar
P EN5
&
\sphinxAtStartPar
P EN4
&
\sphinxAtStartPar
PA N3
&
\sphinxAtStartPar
P EN2
&
\sphinxAtStartPar
PA N1
&
\sphinxAtStartPar
P EN0
\\
\sphinxbottomrule
\end{tabular}
\sphinxtableafterendhook\par
\sphinxattableend\end{savenotes}


\begin{savenotes}\sphinxattablestart
\sphinxthistablewithglobalstyle
\centering
\begin{tabular}[t]{\X{33}{99}\X{33}{99}\X{33}{99}}
\sphinxtoprule
\sphinxtableatstartofbodyhook
\sphinxAtStartPar
位域 |
&
\sphinxAtStartPar
名称     | |
&
\sphinxAtStartPar
描述                                        | |
\\
\sphinxhline
\sphinxAtStartPar
31:16
&\begin{itemize}
\item {} 
\end{itemize}
&\begin{itemize}
\item {} 
\end{itemize}
\\
\sphinxhline
\sphinxAtStartPar
15
&
\sphinxAtStartPar
PAWKEN15
&
\sphinxAtStartPar
PA15输入唤醒使能                            |

\sphinxAtStartPar
1:使能 0:禁能                             |
\\
\sphinxhline
\sphinxAtStartPar
14
&
\sphinxAtStartPar
PAWKEN14
&
\sphinxAtStartPar
PA14输入唤醒使能                            |

\sphinxAtStartPar
1:使能 0:禁能                             |
\\
\sphinxhline
\sphinxAtStartPar
13
&
\sphinxAtStartPar
PAWKEN13
&
\sphinxAtStartPar
PA13输入唤醒使能                            |

\sphinxAtStartPar
1:使能 0:禁能                             |
\\
\sphinxhline
\sphinxAtStartPar
12
&
\sphinxAtStartPar
PAWKEN12
&
\sphinxAtStartPar
PA12输入唤醒使能                            |

\sphinxAtStartPar
1:使能 0:禁能                             |
\\
\sphinxhline
\sphinxAtStartPar
11
&
\sphinxAtStartPar
PAWKEN11
&
\sphinxAtStartPar
PA11输入唤醒使能                            |

\sphinxAtStartPar
1:使能 0:禁能                             |
\\
\sphinxhline
\sphinxAtStartPar
10
&
\sphinxAtStartPar
PAWKEN10
&
\sphinxAtStartPar
PA10输入唤醒使能                            |

\sphinxAtStartPar
1:使能 0:禁能                             |
\\
\sphinxhline
\sphinxAtStartPar
9
&
\sphinxAtStartPar
PAWKEN9
&
\sphinxAtStartPar
PA9输入唤醒使能                             |

\sphinxAtStartPar
1:使能 0:禁能                             |
\\
\sphinxhline
\sphinxAtStartPar
8
&
\sphinxAtStartPar
PAWKEN8
&
\sphinxAtStartPar
PA8输入唤醒使能                             |

\sphinxAtStartPar
1:使能 0:禁能                             |
\\
\sphinxhline
\sphinxAtStartPar
7
&
\sphinxAtStartPar
PAWKEN7
&
\sphinxAtStartPar
PA7输入唤醒使能                             |

\sphinxAtStartPar
1:使能 0:禁能                             |
\\
\sphinxhline
\sphinxAtStartPar
6
&
\sphinxAtStartPar
PAWKEN6
&
\sphinxAtStartPar
PA6输入唤醒使能                             |

\sphinxAtStartPar
1:使能 0:禁能                             |
\\
\sphinxhline
\sphinxAtStartPar
5
&
\sphinxAtStartPar
PAWKEN5
&
\sphinxAtStartPar
PA5输入唤醒使能                             |

\sphinxAtStartPar
1:使能 0:禁能                             |
\\
\sphinxhline
\sphinxAtStartPar
4
&
\sphinxAtStartPar
PAWKEN4
&
\sphinxAtStartPar
PA4输入唤醒使能                             |

\sphinxAtStartPar
1:使能 0:禁能                             |
\\
\sphinxhline
\sphinxAtStartPar
3
&
\sphinxAtStartPar
PAWKEN3
&
\sphinxAtStartPar
PA3输入唤醒使能                             |

\sphinxAtStartPar
1:使能 0:禁能                             |
\\
\sphinxhline
\sphinxAtStartPar
2
&
\sphinxAtStartPar
PAWKEN2
&
\sphinxAtStartPar
PA2输入唤醒使能                             |

\sphinxAtStartPar
1:使能 0:禁能                             |
\\
\sphinxhline
\sphinxAtStartPar
1
&
\sphinxAtStartPar
PAWKEN1
&
\sphinxAtStartPar
PA1输入唤醒使能                             |

\sphinxAtStartPar
1:使能 0:禁能                             |
\\
\sphinxhline
\sphinxAtStartPar
0
&
\sphinxAtStartPar
PAWKEN0
&
\sphinxAtStartPar
PA0输入唤醒使能                             |

\sphinxAtStartPar
1:使能 0:禁能                             |
\\
\sphinxbottomrule
\end{tabular}
\sphinxtableafterendhook\par
\sphinxattableend\end{savenotes}


\subsubsection{PORTB唤醒使能控制寄存器PBWKEN}
\label{\detokenize{SWM241/_u529f_u80fd_u63cf_u8ff0/_u7cfb_u7edf_u7ba1_u7406:portbpbwken}}

\begin{savenotes}\sphinxattablestart
\sphinxthistablewithglobalstyle
\centering
\begin{tabular}[t]{\X{20}{100}\X{20}{100}\X{20}{100}\X{20}{100}\X{20}{100}}
\sphinxtoprule
\sphinxtableatstartofbodyhook
\sphinxAtStartPar
寄存器 |
&
\begin{DUlineblock}{0em}
\item[] 偏移 |
\end{DUlineblock}
&
\begin{DUlineblock}{0em}
\item[] 
\item[] {\color{red}\bfseries{}|}
\end{DUlineblock}
&
\sphinxAtStartPar
复位值 |    描 | |
&
\begin{DUlineblock}{0em}
\item[] |
  |
\end{DUlineblock}
\\
\sphinxhline
\sphinxAtStartPar
PBWKEN
&
\sphinxAtStartPar
0x104
&&
\sphinxAtStartPar
0 000000
&
\sphinxAtStartPar
PORTB唤醒使能控制寄存器    |
\\
\sphinxbottomrule
\end{tabular}
\sphinxtableafterendhook\par
\sphinxattableend\end{savenotes}


\begin{savenotes}\sphinxattablestart
\sphinxthistablewithglobalstyle
\centering
\begin{tabular}[t]{\X{12}{96}\X{12}{96}\X{12}{96}\X{12}{96}\X{12}{96}\X{12}{96}\X{12}{96}\X{12}{96}}
\sphinxtoprule
\sphinxtableatstartofbodyhook
\sphinxAtStartPar
31
&
\sphinxAtStartPar
30
&
\sphinxAtStartPar
29
&
\sphinxAtStartPar
28
&
\sphinxAtStartPar
27
&
\sphinxAtStartPar
26
&
\sphinxAtStartPar
25
&
\sphinxAtStartPar
24
\\
\sphinxhline\begin{itemize}
\item {} 
\end{itemize}
&&&&&&&\\
\sphinxhline
\sphinxAtStartPar
23
&
\sphinxAtStartPar
22
&
\sphinxAtStartPar
21
&
\sphinxAtStartPar
20
&
\sphinxAtStartPar
19
&
\sphinxAtStartPar
18
&
\sphinxAtStartPar
17
&
\sphinxAtStartPar
16
\\
\sphinxhline\begin{itemize}
\item {} 
\end{itemize}
&&&&&&&\\
\sphinxhline
\sphinxAtStartPar
15
&
\sphinxAtStartPar
14
&
\sphinxAtStartPar
13
&
\sphinxAtStartPar
12
&
\sphinxAtStartPar
11
&
\sphinxAtStartPar
10
&
\sphinxAtStartPar
9
&
\sphinxAtStartPar
8
\\
\sphinxhline\begin{itemize}
\item {} 
\end{itemize}
&&&&&&
\sphinxAtStartPar
PB N9
&
\sphinxAtStartPar
P EN8
\\
\sphinxhline
\sphinxAtStartPar
7
&
\sphinxAtStartPar
6
&
\sphinxAtStartPar
5
&
\sphinxAtStartPar
4
&
\sphinxAtStartPar
3
&
\sphinxAtStartPar
2
&
\sphinxAtStartPar
1
&
\sphinxAtStartPar
0
\\
\sphinxhline
\sphinxAtStartPar
PBWKEN7
&
\sphinxAtStartPar
P BWKEN6
&
\sphinxAtStartPar
P EN5
&
\sphinxAtStartPar
P EN4
&
\sphinxAtStartPar
PB N3
&
\sphinxAtStartPar
P EN2
&
\sphinxAtStartPar
PB N1
&
\sphinxAtStartPar
P EN0
\\
\sphinxbottomrule
\end{tabular}
\sphinxtableafterendhook\par
\sphinxattableend\end{savenotes}


\begin{savenotes}\sphinxattablestart
\sphinxthistablewithglobalstyle
\centering
\begin{tabular}[t]{\X{33}{99}\X{33}{99}\X{33}{99}}
\sphinxtoprule
\sphinxtableatstartofbodyhook
\sphinxAtStartPar
位域 |
&
\sphinxAtStartPar
名称     | |
&
\sphinxAtStartPar
描述                                        | |
\\
\sphinxhline
\sphinxAtStartPar
31:10
&\begin{itemize}
\item {} 
\end{itemize}
&\begin{itemize}
\item {} 
\end{itemize}
\\
\sphinxhline
\sphinxAtStartPar
9
&
\sphinxAtStartPar
PBWKEN9
&
\sphinxAtStartPar
PB9输入唤醒使能                             |

\sphinxAtStartPar
1:使能 0:禁能                             |
\\
\sphinxhline
\sphinxAtStartPar
8
&
\sphinxAtStartPar
PBWKEN8
&
\sphinxAtStartPar
PB8输入唤醒使能                             |

\sphinxAtStartPar
1:使能 0:禁能                             |
\\
\sphinxhline
\sphinxAtStartPar
7
&
\sphinxAtStartPar
PBWKEN7
&
\sphinxAtStartPar
PB7输入唤醒使能                             |

\sphinxAtStartPar
1:使能 0:禁能                             |
\\
\sphinxhline
\sphinxAtStartPar
6
&
\sphinxAtStartPar
PBWKEN6
&
\sphinxAtStartPar
PB6输入唤醒使能                             |

\sphinxAtStartPar
1:使能 0:禁能                             |
\\
\sphinxhline
\sphinxAtStartPar
5
&
\sphinxAtStartPar
PBWKEN5
&
\sphinxAtStartPar
PB5输入唤醒使能                             |

\sphinxAtStartPar
1:使能 0:禁能                             |
\\
\sphinxhline
\sphinxAtStartPar
4
&
\sphinxAtStartPar
PBWKEN4
&
\sphinxAtStartPar
PB4输入唤醒使能                             |

\sphinxAtStartPar
1:使能 0:禁能                             |
\\
\sphinxhline
\sphinxAtStartPar
3
&
\sphinxAtStartPar
PBWKEN3
&
\sphinxAtStartPar
PB3输入唤醒使能                             |

\sphinxAtStartPar
1:使能 0:禁能                             |
\\
\sphinxhline
\sphinxAtStartPar
2
&
\sphinxAtStartPar
PBWKEN2
&
\sphinxAtStartPar
PB2输入唤醒使能                             |

\sphinxAtStartPar
1:使能 0:禁能                             |
\\
\sphinxhline
\sphinxAtStartPar
1
&
\sphinxAtStartPar
PBWKEN1
&
\sphinxAtStartPar
PB1输入唤醒使能                             |

\sphinxAtStartPar
1:使能 0:禁能                             |
\\
\sphinxhline
\sphinxAtStartPar
0
&
\sphinxAtStartPar
PBWKEN0
&
\sphinxAtStartPar
PB0输入唤醒使能                             |

\sphinxAtStartPar
1:使能 0:禁能                             |
\\
\sphinxbottomrule
\end{tabular}
\sphinxtableafterendhook\par
\sphinxattableend\end{savenotes}


\subsubsection{PORTC唤醒使能控制寄存器PCWKEN}
\label{\detokenize{SWM241/_u529f_u80fd_u63cf_u8ff0/_u7cfb_u7edf_u7ba1_u7406:portcpcwken}}

\begin{savenotes}\sphinxattablestart
\sphinxthistablewithglobalstyle
\centering
\begin{tabular}[t]{\X{20}{100}\X{20}{100}\X{20}{100}\X{20}{100}\X{20}{100}}
\sphinxtoprule
\sphinxtableatstartofbodyhook
\sphinxAtStartPar
寄存器 |
&
\begin{DUlineblock}{0em}
\item[] 偏移 |
\end{DUlineblock}
&
\begin{DUlineblock}{0em}
\item[] 
\item[] {\color{red}\bfseries{}|}
\end{DUlineblock}
&
\sphinxAtStartPar
复位值 |    描 | |
&
\begin{DUlineblock}{0em}
\item[] |
  |
\end{DUlineblock}
\\
\sphinxhline
\sphinxAtStartPar
PCWKEN
&
\sphinxAtStartPar
0x108
&&
\sphinxAtStartPar
0 000000
&
\sphinxAtStartPar
PORTC唤醒使能控制寄存器    |
\\
\sphinxbottomrule
\end{tabular}
\sphinxtableafterendhook\par
\sphinxattableend\end{savenotes}


\begin{savenotes}\sphinxattablestart
\sphinxthistablewithglobalstyle
\centering
\begin{tabular}[t]{\X{12}{96}\X{12}{96}\X{12}{96}\X{12}{96}\X{12}{96}\X{12}{96}\X{12}{96}\X{12}{96}}
\sphinxtoprule
\sphinxtableatstartofbodyhook
\sphinxAtStartPar
31
&
\sphinxAtStartPar
30
&
\sphinxAtStartPar
29
&
\sphinxAtStartPar
28
&
\sphinxAtStartPar
27
&
\sphinxAtStartPar
26
&
\sphinxAtStartPar
25
&
\sphinxAtStartPar
24
\\
\sphinxhline\begin{itemize}
\item {} 
\end{itemize}
&&&&&&&\\
\sphinxhline
\sphinxAtStartPar
23
&
\sphinxAtStartPar
22
&
\sphinxAtStartPar
21
&
\sphinxAtStartPar
20
&
\sphinxAtStartPar
19
&
\sphinxAtStartPar
18
&
\sphinxAtStartPar
17
&
\sphinxAtStartPar
16
\\
\sphinxhline\begin{itemize}
\item {} 
\end{itemize}
&&&&&&&\\
\sphinxhline
\sphinxAtStartPar
15
&
\sphinxAtStartPar
14
&
\sphinxAtStartPar
13
&
\sphinxAtStartPar
12
&
\sphinxAtStartPar
11
&
\sphinxAtStartPar
10
&
\sphinxAtStartPar
9
&
\sphinxAtStartPar
8
\\
\sphinxhline\begin{itemize}
\item {} 
\end{itemize}
&&&&&&&\\
\sphinxhline
\sphinxAtStartPar
7
&
\sphinxAtStartPar
6
&
\sphinxAtStartPar
5
&
\sphinxAtStartPar
4
&
\sphinxAtStartPar
3
&
\sphinxAtStartPar
2
&
\sphinxAtStartPar
1
&
\sphinxAtStartPar
0
\\
\sphinxhline\begin{itemize}
\item {} 
\end{itemize}
&&&&
\sphinxAtStartPar
PC N3
&
\sphinxAtStartPar
P EN2
&
\sphinxAtStartPar
PC N1
&
\sphinxAtStartPar
P EN0
\\
\sphinxbottomrule
\end{tabular}
\sphinxtableafterendhook\par
\sphinxattableend\end{savenotes}


\begin{savenotes}\sphinxattablestart
\sphinxthistablewithglobalstyle
\centering
\begin{tabular}[t]{\X{33}{99}\X{33}{99}\X{33}{99}}
\sphinxtoprule
\sphinxtableatstartofbodyhook
\sphinxAtStartPar
位域 |
&
\sphinxAtStartPar
名称     | |
&
\sphinxAtStartPar
描述                                        | |
\\
\sphinxhline
\sphinxAtStartPar
31:4
&\begin{itemize}
\item {} 
\end{itemize}
&\begin{itemize}
\item {} 
\end{itemize}
\\
\sphinxhline
\sphinxAtStartPar
3
&
\sphinxAtStartPar
PCWKEN3
&
\sphinxAtStartPar
PC3输入唤醒使能                             |

\sphinxAtStartPar
1:使能 0:禁能                             |
\\
\sphinxhline
\sphinxAtStartPar
2
&
\sphinxAtStartPar
PCWKEN2
&
\sphinxAtStartPar
PC2输入唤醒使能                             |

\sphinxAtStartPar
1:使能 0:禁能                             |
\\
\sphinxhline
\sphinxAtStartPar
1
&
\sphinxAtStartPar
PCWKEN1
&
\sphinxAtStartPar
PC1输入唤醒使能                             |

\sphinxAtStartPar
1:使能 0:禁能                             |
\\
\sphinxhline
\sphinxAtStartPar
0
&
\sphinxAtStartPar
PCWKEN0
&
\sphinxAtStartPar
PC0输入唤醒使能                             |

\sphinxAtStartPar
1:使能 0:禁能                             |
\\
\sphinxbottomrule
\end{tabular}
\sphinxtableafterendhook\par
\sphinxattableend\end{savenotes}


\subsubsection{PORTD唤醒使能控制寄存器PDWKEN}
\label{\detokenize{SWM241/_u529f_u80fd_u63cf_u8ff0/_u7cfb_u7edf_u7ba1_u7406:portdpdwken}}

\begin{savenotes}\sphinxattablestart
\sphinxthistablewithglobalstyle
\centering
\begin{tabular}[t]{\X{20}{100}\X{20}{100}\X{20}{100}\X{20}{100}\X{20}{100}}
\sphinxtoprule
\sphinxtableatstartofbodyhook
\sphinxAtStartPar
寄存器 |
&
\begin{DUlineblock}{0em}
\item[] 偏移 |
\end{DUlineblock}
&
\begin{DUlineblock}{0em}
\item[] 
\item[] {\color{red}\bfseries{}|}
\end{DUlineblock}
&
\sphinxAtStartPar
复位值 |    描 | |
&
\begin{DUlineblock}{0em}
\item[] |
  |
\end{DUlineblock}
\\
\sphinxhline
\sphinxAtStartPar
PDWKEN
&
\sphinxAtStartPar
0x10C
&&
\sphinxAtStartPar
0 000000
&
\sphinxAtStartPar
PORTD唤醒使能控制寄存器    |
\\
\sphinxbottomrule
\end{tabular}
\sphinxtableafterendhook\par
\sphinxattableend\end{savenotes}


\begin{savenotes}\sphinxattablestart
\sphinxthistablewithglobalstyle
\centering
\begin{tabular}[t]{\X{12}{96}\X{12}{96}\X{12}{96}\X{12}{96}\X{12}{96}\X{12}{96}\X{12}{96}\X{12}{96}}
\sphinxtoprule
\sphinxtableatstartofbodyhook
\sphinxAtStartPar
31
&
\sphinxAtStartPar
30
&
\sphinxAtStartPar
29
&
\sphinxAtStartPar
28
&
\sphinxAtStartPar
27
&
\sphinxAtStartPar
26
&
\sphinxAtStartPar
25
&
\sphinxAtStartPar
24
\\
\sphinxhline\begin{itemize}
\item {} 
\end{itemize}
&&&&&&&\\
\sphinxhline
\sphinxAtStartPar
23
&
\sphinxAtStartPar
22
&
\sphinxAtStartPar
21
&
\sphinxAtStartPar
20
&
\sphinxAtStartPar
19
&
\sphinxAtStartPar
18
&
\sphinxAtStartPar
17
&
\sphinxAtStartPar
16
\\
\sphinxhline\begin{itemize}
\item {} 
\end{itemize}
&&&&&&&\\
\sphinxhline
\sphinxAtStartPar
15
&
\sphinxAtStartPar
14
&
\sphinxAtStartPar
13
&
\sphinxAtStartPar
12
&
\sphinxAtStartPar
11
&
\sphinxAtStartPar
10
&
\sphinxAtStartPar
9
&
\sphinxAtStartPar
8
\\
\sphinxhline\begin{itemize}
\item {} 
\end{itemize}
&&&&&&
\sphinxAtStartPar
PD N9
&
\sphinxAtStartPar
P EN8
\\
\sphinxhline
\sphinxAtStartPar
7
&
\sphinxAtStartPar
6
&
\sphinxAtStartPar
5
&
\sphinxAtStartPar
4
&
\sphinxAtStartPar
3
&
\sphinxAtStartPar
2
&
\sphinxAtStartPar
1
&
\sphinxAtStartPar
0
\\
\sphinxhline
\sphinxAtStartPar
PDWKEN7
&
\sphinxAtStartPar
P DWKEN6
&
\sphinxAtStartPar
PD N5
&
\sphinxAtStartPar
P EN4
&
\sphinxAtStartPar
PD N3
&
\sphinxAtStartPar
P EN2
&
\sphinxAtStartPar
PD N1
&
\sphinxAtStartPar
P EN0
\\
\sphinxbottomrule
\end{tabular}
\sphinxtableafterendhook\par
\sphinxattableend\end{savenotes}


\begin{savenotes}\sphinxattablestart
\sphinxthistablewithglobalstyle
\centering
\begin{tabular}[t]{\X{33}{99}\X{33}{99}\X{33}{99}}
\sphinxtoprule
\sphinxtableatstartofbodyhook
\sphinxAtStartPar
位域 |
&
\sphinxAtStartPar
名称     | |
&
\sphinxAtStartPar
描述                                        | |
\\
\sphinxhline
\sphinxAtStartPar
31:10
&\begin{itemize}
\item {} 
\end{itemize}
&\begin{itemize}
\item {} 
\end{itemize}
\\
\sphinxhline
\sphinxAtStartPar
9
&
\sphinxAtStartPar
PDWKEN9
&
\sphinxAtStartPar
PD9输入唤醒使能                             |

\sphinxAtStartPar
1:使能 0:禁能                             |
\\
\sphinxhline
\sphinxAtStartPar
8
&
\sphinxAtStartPar
PDWKEN8
&
\sphinxAtStartPar
PD8输入唤醒使能                             |

\sphinxAtStartPar
1:使能 0:禁能                             |
\\
\sphinxhline
\sphinxAtStartPar
7
&
\sphinxAtStartPar
PDWKEN7
&
\sphinxAtStartPar
PD7输入唤醒使能                             |

\sphinxAtStartPar
1:使能 0:禁能                             |
\\
\sphinxhline
\sphinxAtStartPar
6
&
\sphinxAtStartPar
PDWKEN6
&
\sphinxAtStartPar
PD6输入唤醒使能                             |

\sphinxAtStartPar
1:使能 0:禁能                             |
\\
\sphinxhline
\sphinxAtStartPar
5
&
\sphinxAtStartPar
PDWKEN5
&
\sphinxAtStartPar
PD5输入唤醒使能                             |

\sphinxAtStartPar
1:使能 0:禁能                             |
\\
\sphinxhline
\sphinxAtStartPar
4
&
\sphinxAtStartPar
PDWKEN4
&
\sphinxAtStartPar
PD4输入唤醒使能                             |

\sphinxAtStartPar
1:使能 0:禁能                             |
\\
\sphinxhline
\sphinxAtStartPar
3
&
\sphinxAtStartPar
PDWKEN3
&
\sphinxAtStartPar
PD3输入唤醒使能                             |

\sphinxAtStartPar
1:使能 0:禁能                             |
\\
\sphinxhline
\sphinxAtStartPar
2
&
\sphinxAtStartPar
PDWKEN2
&
\sphinxAtStartPar
PD2输入唤醒使能                             |

\sphinxAtStartPar
1:使能 0:禁能                             |
\\
\sphinxhline
\sphinxAtStartPar
1
&
\sphinxAtStartPar
PDWKEN1
&
\sphinxAtStartPar
PD1输入唤醒使能                             |

\sphinxAtStartPar
1:使能 0:禁能                             |
\\
\sphinxhline
\sphinxAtStartPar
0
&
\sphinxAtStartPar
PDWKEN0
&
\sphinxAtStartPar
PD0输入唤醒使能                             |

\sphinxAtStartPar
1:使能 0:禁能                             |
\\
\sphinxbottomrule
\end{tabular}
\sphinxtableafterendhook\par
\sphinxattableend\end{savenotes}


\subsubsection{PORTA唤醒状态寄存器PAWKSR}
\label{\detokenize{SWM241/_u529f_u80fd_u63cf_u8ff0/_u7cfb_u7edf_u7ba1_u7406:portapawksr}}

\begin{savenotes}\sphinxattablestart
\sphinxthistablewithglobalstyle
\centering
\begin{tabular}[t]{\X{20}{100}\X{20}{100}\X{20}{100}\X{20}{100}\X{20}{100}}
\sphinxtoprule
\sphinxtableatstartofbodyhook
\sphinxAtStartPar
寄存器 |
&
\begin{DUlineblock}{0em}
\item[] 偏移 |
\end{DUlineblock}
&
\begin{DUlineblock}{0em}
\item[] 
\item[] {\color{red}\bfseries{}|}
\end{DUlineblock}
&
\sphinxAtStartPar
复位值 |    描 | |
&
\begin{DUlineblock}{0em}
\item[] |
  |
\end{DUlineblock}
\\
\sphinxhline
\sphinxAtStartPar
PAWKSR
&
\sphinxAtStartPar
0x130
&&
\sphinxAtStartPar
0 000000
&
\sphinxAtStartPar
PORTA唤醒状态寄存器        |
\\
\sphinxbottomrule
\end{tabular}
\sphinxtableafterendhook\par
\sphinxattableend\end{savenotes}


\begin{savenotes}\sphinxattablestart
\sphinxthistablewithglobalstyle
\centering
\begin{tabular}[t]{\X{12}{96}\X{12}{96}\X{12}{96}\X{12}{96}\X{12}{96}\X{12}{96}\X{12}{96}\X{12}{96}}
\sphinxtoprule
\sphinxtableatstartofbodyhook
\sphinxAtStartPar
31
&
\sphinxAtStartPar
30
&
\sphinxAtStartPar
29
&
\sphinxAtStartPar
28
&
\sphinxAtStartPar
27
&
\sphinxAtStartPar
26
&
\sphinxAtStartPar
25
&
\sphinxAtStartPar
24
\\
\sphinxhline\begin{itemize}
\item {} 
\end{itemize}
&&&&&&&\\
\sphinxhline
\sphinxAtStartPar
23
&
\sphinxAtStartPar
22
&
\sphinxAtStartPar
21
&
\sphinxAtStartPar
20
&
\sphinxAtStartPar
19
&
\sphinxAtStartPar
18
&
\sphinxAtStartPar
17
&
\sphinxAtStartPar
16
\\
\sphinxhline\begin{itemize}
\item {} 
\end{itemize}
&&&&&&&\\
\sphinxhline
\sphinxAtStartPar
15
&
\sphinxAtStartPar
14
&
\sphinxAtStartPar
13
&
\sphinxAtStartPar
12
&
\sphinxAtStartPar
11
&
\sphinxAtStartPar
10
&
\sphinxAtStartPar
9
&
\sphinxAtStartPar
8
\\
\sphinxhline
\sphinxAtStartPar
PAWKSR15
&
\sphinxAtStartPar
PA WKSR14
&
\sphinxAtStartPar
PA R13
&
\sphinxAtStartPar
PA R12
&&
\sphinxAtStartPar
PA R10
&
\sphinxAtStartPar
PA R9
&
\sphinxAtStartPar
P SR8
\\
\sphinxhline
\sphinxAtStartPar
7
&
\sphinxAtStartPar
6
&
\sphinxAtStartPar
5
&
\sphinxAtStartPar
4
&
\sphinxAtStartPar
3
&
\sphinxAtStartPar
2
&
\sphinxAtStartPar
1
&
\sphinxAtStartPar
0
\\
\sphinxhline
\sphinxAtStartPar
PAWKSR7
&
\sphinxAtStartPar
P AWKSR6
&
\sphinxAtStartPar
P SR5
&
\sphinxAtStartPar
P SR4
&
\sphinxAtStartPar
PA R3
&
\sphinxAtStartPar
P SR2
&
\sphinxAtStartPar
PA R1
&
\sphinxAtStartPar
P SR0
\\
\sphinxbottomrule
\end{tabular}
\sphinxtableafterendhook\par
\sphinxattableend\end{savenotes}


\begin{savenotes}\sphinxattablestart
\sphinxthistablewithglobalstyle
\centering
\begin{tabular}[t]{\X{33}{99}\X{33}{99}\X{33}{99}}
\sphinxtoprule
\sphinxtableatstartofbodyhook
\sphinxAtStartPar
位域 |
&
\sphinxAtStartPar
名称     | |
&
\sphinxAtStartPar
描述                                        | |
\\
\sphinxhline
\sphinxAtStartPar
31:16
&\begin{itemize}
\item {} 
\end{itemize}
&\begin{itemize}
\item {} 
\end{itemize}
\\
\sphinxhline
\sphinxAtStartPar
15
&
\sphinxAtStartPar
PAWKSR15
&
\sphinxAtStartPar
PA15输入唤醒状态标志位                      |

\sphinxAtStartPar
唤醒后硬件置1,软件写1清除                  |

\sphinxAtStartPar
1:唤醒 0:未唤醒                           |
\\
\sphinxhline
\sphinxAtStartPar
14
&
\sphinxAtStartPar
PAWKSR14
&
\sphinxAtStartPar
PA14输入唤醒状态标志位                      |

\sphinxAtStartPar
唤醒后硬件置1,软件写1清除                  |

\sphinxAtStartPar
1:唤醒 0:未唤醒                           |
\\
\sphinxhline
\sphinxAtStartPar
13
&
\sphinxAtStartPar
PAWKSR13
&
\sphinxAtStartPar
PA13输入唤醒状态标志位                      |

\sphinxAtStartPar
唤醒后硬件置1,软件写1清除                  |

\sphinxAtStartPar
1:唤醒 0:未唤醒                           |
\\
\sphinxhline
\sphinxAtStartPar
12
&
\sphinxAtStartPar
PAWKSR12
&
\sphinxAtStartPar
PA12输入唤醒状态标志位                      |

\sphinxAtStartPar
唤醒后硬件置1,软件写1清除                  |

\sphinxAtStartPar
1:唤醒 0:未唤醒                           |
\\
\sphinxhline
\sphinxAtStartPar
11
&
\sphinxAtStartPar
PAWKSR11
&
\sphinxAtStartPar
PA11输入唤醒状态标志位                      |

\sphinxAtStartPar
唤醒后硬件置1,软件写1清除                  |

\sphinxAtStartPar
1:唤醒 0:未唤醒                           |
\\
\sphinxhline
\sphinxAtStartPar
10
&
\sphinxAtStartPar
PAWKSR10
&
\sphinxAtStartPar
PA10输入唤醒状态标志位                      |

\sphinxAtStartPar
唤醒后硬件置1,软件写1清除                  |

\sphinxAtStartPar
1:唤醒 0:未唤醒                           |
\\
\sphinxhline
\sphinxAtStartPar
9
&
\sphinxAtStartPar
PAWKSR9
&
\sphinxAtStartPar
PA9输入唤醒状态标志位                       |

\sphinxAtStartPar
唤醒后硬件置1,软件写1清除                  |

\sphinxAtStartPar
1:唤醒 0:未唤醒                           |
\\
\sphinxhline
\sphinxAtStartPar
8
&
\sphinxAtStartPar
PAWKSR8
&
\sphinxAtStartPar
PA8输入唤醒状态标志位                       |

\sphinxAtStartPar
唤醒后硬件置1,软件写1清除                  |

\sphinxAtStartPar
1:唤醒 0:未唤醒                           |
\\
\sphinxhline
\sphinxAtStartPar
7
&
\sphinxAtStartPar
PAWKSR7
&
\sphinxAtStartPar
PA7输入唤醒状态标志位                       |

\sphinxAtStartPar
唤醒后硬件置1,软件写1清除                  |

\sphinxAtStartPar
1:唤醒 0:未唤醒                           |
\\
\sphinxhline
\sphinxAtStartPar
6
&
\sphinxAtStartPar
PAWKSR6
&
\sphinxAtStartPar
PA6输入唤醒状态标志位                       |

\sphinxAtStartPar
唤醒后硬件置1,软件写1清除                  |

\sphinxAtStartPar
1:唤醒 0:未唤醒                           |
\\
\sphinxhline
\sphinxAtStartPar
5
&
\sphinxAtStartPar
PAWKSR5
&
\sphinxAtStartPar
PA5输入唤醒状态标志位                       |

\sphinxAtStartPar
唤醒后硬件置1,软件写1清除                  |

\sphinxAtStartPar
1:唤醒 0:未唤醒                           |
\\
\sphinxhline
\sphinxAtStartPar
4
&
\sphinxAtStartPar
PAWKSR4
&
\sphinxAtStartPar
PA4输入唤醒状态标志位                       |

\sphinxAtStartPar
唤醒后硬件置1,软件写1清除                  |

\sphinxAtStartPar
1:唤醒 0:未唤醒                           |
\\
\sphinxhline
\sphinxAtStartPar
3
&
\sphinxAtStartPar
PAWKSR3
&
\sphinxAtStartPar
PA3输入唤醒状态标志位                       |

\sphinxAtStartPar
唤醒后硬件置1,软件写1清除                  |

\sphinxAtStartPar
1:唤醒 0:未唤醒                           |
\\
\sphinxhline
\sphinxAtStartPar
2
&
\sphinxAtStartPar
PAWKSR2
&
\sphinxAtStartPar
PA2输入唤醒状态标志位                       |

\sphinxAtStartPar
唤醒后硬件置1,软件写1清除                  |

\sphinxAtStartPar
1:唤醒 0:未唤醒                           |
\\
\sphinxhline
\sphinxAtStartPar
1
&
\sphinxAtStartPar
PAWKSR1
&
\sphinxAtStartPar
PA1输入唤醒状态标志位                       |

\sphinxAtStartPar
唤醒后硬件置1,软件写1清除                  |

\sphinxAtStartPar
1:唤醒 0:未唤醒                           |
\\
\sphinxhline
\sphinxAtStartPar
0
&
\sphinxAtStartPar
PAWKSR0
&
\sphinxAtStartPar
PA0输入唤醒状态标志位                       |

\sphinxAtStartPar
唤醒后硬件置1,软件写1清除                  |

\sphinxAtStartPar
1:唤醒 0:未唤醒                           |
\\
\sphinxbottomrule
\end{tabular}
\sphinxtableafterendhook\par
\sphinxattableend\end{savenotes}


\subsubsection{PORTB唤醒状态寄存器PBWKSR}
\label{\detokenize{SWM241/_u529f_u80fd_u63cf_u8ff0/_u7cfb_u7edf_u7ba1_u7406:portbpbwksr}}

\begin{savenotes}\sphinxattablestart
\sphinxthistablewithglobalstyle
\centering
\begin{tabular}[t]{\X{20}{100}\X{20}{100}\X{20}{100}\X{20}{100}\X{20}{100}}
\sphinxtoprule
\sphinxtableatstartofbodyhook
\sphinxAtStartPar
寄存器 |
&
\begin{DUlineblock}{0em}
\item[] 偏移 |
\end{DUlineblock}
&
\begin{DUlineblock}{0em}
\item[] 
\item[] {\color{red}\bfseries{}|}
\end{DUlineblock}
&
\sphinxAtStartPar
复位值 |    描 | |
&
\begin{DUlineblock}{0em}
\item[] |
  |
\end{DUlineblock}
\\
\sphinxhline
\sphinxAtStartPar
PBWKSR
&
\sphinxAtStartPar
0x134
&&
\sphinxAtStartPar
0 000000
&
\sphinxAtStartPar
PORTB唤醒状态寄存器        |
\\
\sphinxbottomrule
\end{tabular}
\sphinxtableafterendhook\par
\sphinxattableend\end{savenotes}


\begin{savenotes}\sphinxattablestart
\sphinxthistablewithglobalstyle
\centering
\begin{tabular}[t]{\X{12}{96}\X{12}{96}\X{12}{96}\X{12}{96}\X{12}{96}\X{12}{96}\X{12}{96}\X{12}{96}}
\sphinxtoprule
\sphinxtableatstartofbodyhook
\sphinxAtStartPar
31
&
\sphinxAtStartPar
30
&
\sphinxAtStartPar
29
&
\sphinxAtStartPar
28
&
\sphinxAtStartPar
27
&
\sphinxAtStartPar
26
&
\sphinxAtStartPar
25
&
\sphinxAtStartPar
24
\\
\sphinxhline\begin{itemize}
\item {} 
\end{itemize}
&&&&&&&\\
\sphinxhline
\sphinxAtStartPar
23
&
\sphinxAtStartPar
22
&
\sphinxAtStartPar
21
&
\sphinxAtStartPar
20
&
\sphinxAtStartPar
19
&
\sphinxAtStartPar
18
&
\sphinxAtStartPar
17
&
\sphinxAtStartPar
16
\\
\sphinxhline\begin{itemize}
\item {} 
\end{itemize}
&&&&&&&\\
\sphinxhline
\sphinxAtStartPar
15
&
\sphinxAtStartPar
14
&
\sphinxAtStartPar
13
&
\sphinxAtStartPar
12
&
\sphinxAtStartPar
11
&
\sphinxAtStartPar
10
&
\sphinxAtStartPar
9
&
\sphinxAtStartPar
8
\\
\sphinxhline\begin{itemize}
\item {} 
\end{itemize}
&&&&&&
\sphinxAtStartPar
PB R9
&
\sphinxAtStartPar
P SR8
\\
\sphinxhline
\sphinxAtStartPar
7
&
\sphinxAtStartPar
6
&
\sphinxAtStartPar
5
&
\sphinxAtStartPar
4
&
\sphinxAtStartPar
3
&
\sphinxAtStartPar
2
&
\sphinxAtStartPar
1
&
\sphinxAtStartPar
0
\\
\sphinxhline
\sphinxAtStartPar
PBWKSR7
&
\sphinxAtStartPar
P BWKSR6
&
\sphinxAtStartPar
P SR5
&
\sphinxAtStartPar
P SR4
&
\sphinxAtStartPar
PB R3
&
\sphinxAtStartPar
P SR2
&
\sphinxAtStartPar
PB R1
&
\sphinxAtStartPar
P SR0
\\
\sphinxbottomrule
\end{tabular}
\sphinxtableafterendhook\par
\sphinxattableend\end{savenotes}


\begin{savenotes}\sphinxattablestart
\sphinxthistablewithglobalstyle
\centering
\begin{tabular}[t]{\X{33}{99}\X{33}{99}\X{33}{99}}
\sphinxtoprule
\sphinxtableatstartofbodyhook
\sphinxAtStartPar
位域 |
&
\sphinxAtStartPar
名称     | |
&
\sphinxAtStartPar
描述                                        | |
\\
\sphinxhline
\sphinxAtStartPar
31:10
&\begin{itemize}
\item {} 
\end{itemize}
&\begin{itemize}
\item {} 
\end{itemize}
\\
\sphinxhline
\sphinxAtStartPar
9
&
\sphinxAtStartPar
PBWKSR9
&
\sphinxAtStartPar
PB9输入唤醒状态标志位                       |

\sphinxAtStartPar
唤醒后硬件置1,软件写1清除                  |

\sphinxAtStartPar
1:唤醒 0:未唤醒                           |
\\
\sphinxhline
\sphinxAtStartPar
8
&
\sphinxAtStartPar
PBWKSR8
&
\sphinxAtStartPar
PB8输入唤醒状态标志位                       |

\sphinxAtStartPar
唤醒后硬件置1,软件写1清除                  |

\sphinxAtStartPar
1:唤醒 0:未唤醒                           |
\\
\sphinxhline
\sphinxAtStartPar
7
&
\sphinxAtStartPar
PBWKSR7
&
\sphinxAtStartPar
PB7输入唤醒状态标志位                       |

\sphinxAtStartPar
唤醒后硬件置1,软件写1清除                  |

\sphinxAtStartPar
1:唤醒 0:未唤醒                           |
\\
\sphinxhline
\sphinxAtStartPar
6
&
\sphinxAtStartPar
PBWKSR6
&
\sphinxAtStartPar
PB6输入唤醒状态标志位                       |

\sphinxAtStartPar
唤醒后硬件置1,软件写1清除                  |

\sphinxAtStartPar
1:唤醒 0:未唤醒                           |
\\
\sphinxhline
\sphinxAtStartPar
5
&
\sphinxAtStartPar
PBWKSR5
&
\sphinxAtStartPar
PB5输入唤醒状态标志位                       |

\sphinxAtStartPar
唤醒后硬件置1,软件写1清除                  |

\sphinxAtStartPar
1:唤醒 0:未唤醒                           |
\\
\sphinxhline
\sphinxAtStartPar
4
&
\sphinxAtStartPar
PBWKSR4
&
\sphinxAtStartPar
PB4输入唤醒状态标志位                       |

\sphinxAtStartPar
唤醒后硬件置1,软件写1清除                  |

\sphinxAtStartPar
1:唤醒 0:未唤醒                           |
\\
\sphinxhline
\sphinxAtStartPar
3
&
\sphinxAtStartPar
PBWKSR3
&
\sphinxAtStartPar
PB3输入唤醒状态标志位                       |

\sphinxAtStartPar
唤醒后硬件置1,软件写1清除                  |

\sphinxAtStartPar
1:唤醒 0:未唤醒                           |
\\
\sphinxhline
\sphinxAtStartPar
2
&
\sphinxAtStartPar
PBWKSR2
&
\sphinxAtStartPar
PB2输入唤醒状态标志位                       |

\sphinxAtStartPar
唤醒后硬件置1,软件写1清除                  |

\sphinxAtStartPar
1:唤醒 0:未唤醒                           |
\\
\sphinxhline
\sphinxAtStartPar
1
&
\sphinxAtStartPar
PBWKSR1
&
\sphinxAtStartPar
PB1输入唤醒状态标志位                       |

\sphinxAtStartPar
唤醒后硬件置1,软件写1清除                  |

\sphinxAtStartPar
1:唤醒 0:未唤醒                           |
\\
\sphinxhline
\sphinxAtStartPar
0
&
\sphinxAtStartPar
PBWKSR0
&
\sphinxAtStartPar
PB0输入唤醒状态标志位                       |

\sphinxAtStartPar
唤醒后硬件置1,软件写1清除                  |

\sphinxAtStartPar
1:唤醒 0:未唤醒                           |
\\
\sphinxbottomrule
\end{tabular}
\sphinxtableafterendhook\par
\sphinxattableend\end{savenotes}


\subsubsection{PORTC唤醒状态寄存器PCWKSR}
\label{\detokenize{SWM241/_u529f_u80fd_u63cf_u8ff0/_u7cfb_u7edf_u7ba1_u7406:portcpcwksr}}

\begin{savenotes}\sphinxattablestart
\sphinxthistablewithglobalstyle
\centering
\begin{tabular}[t]{\X{20}{100}\X{20}{100}\X{20}{100}\X{20}{100}\X{20}{100}}
\sphinxtoprule
\sphinxtableatstartofbodyhook
\sphinxAtStartPar
寄存器 |
&
\begin{DUlineblock}{0em}
\item[] 偏移 |
\end{DUlineblock}
&
\begin{DUlineblock}{0em}
\item[] 
\item[] {\color{red}\bfseries{}|}
\end{DUlineblock}
&
\sphinxAtStartPar
复位值 |    描 | |
&
\begin{DUlineblock}{0em}
\item[] |
  |
\end{DUlineblock}
\\
\sphinxhline
\sphinxAtStartPar
PCWKSR
&
\sphinxAtStartPar
0x138
&&
\sphinxAtStartPar
0 000000
&
\sphinxAtStartPar
PORTC唤醒状态寄存器        |
\\
\sphinxbottomrule
\end{tabular}
\sphinxtableafterendhook\par
\sphinxattableend\end{savenotes}


\begin{savenotes}\sphinxattablestart
\sphinxthistablewithglobalstyle
\centering
\begin{tabular}[t]{\X{12}{96}\X{12}{96}\X{12}{96}\X{12}{96}\X{12}{96}\X{12}{96}\X{12}{96}\X{12}{96}}
\sphinxtoprule
\sphinxtableatstartofbodyhook
\sphinxAtStartPar
31
&
\sphinxAtStartPar
30
&
\sphinxAtStartPar
29
&
\sphinxAtStartPar
28
&
\sphinxAtStartPar
27
&
\sphinxAtStartPar
26
&
\sphinxAtStartPar
25
&
\sphinxAtStartPar
24
\\
\sphinxhline\begin{itemize}
\item {} 
\end{itemize}
&&&&&&&\\
\sphinxhline
\sphinxAtStartPar
23
&
\sphinxAtStartPar
22
&
\sphinxAtStartPar
21
&
\sphinxAtStartPar
20
&
\sphinxAtStartPar
19
&
\sphinxAtStartPar
18
&
\sphinxAtStartPar
17
&
\sphinxAtStartPar
16
\\
\sphinxhline\begin{itemize}
\item {} 
\end{itemize}
&&&&&&&\\
\sphinxhline
\sphinxAtStartPar
15
&
\sphinxAtStartPar
14
&
\sphinxAtStartPar
13
&
\sphinxAtStartPar
12
&
\sphinxAtStartPar
11
&
\sphinxAtStartPar
10
&
\sphinxAtStartPar
9
&
\sphinxAtStartPar
8
\\
\sphinxhline\begin{itemize}
\item {} 
\end{itemize}
&&&&&&&\\
\sphinxhline
\sphinxAtStartPar
7
&
\sphinxAtStartPar
6
&
\sphinxAtStartPar
5
&
\sphinxAtStartPar
4
&
\sphinxAtStartPar
3
&
\sphinxAtStartPar
2
&
\sphinxAtStartPar
1
&
\sphinxAtStartPar
0
\\
\sphinxhline\begin{itemize}
\item {} 
\end{itemize}
&&&&
\sphinxAtStartPar
P SR3
&
\sphinxAtStartPar
P SR2
&
\sphinxAtStartPar
P SR1
&
\sphinxAtStartPar
P SR0
\\
\sphinxbottomrule
\end{tabular}
\sphinxtableafterendhook\par
\sphinxattableend\end{savenotes}


\begin{savenotes}\sphinxattablestart
\sphinxthistablewithglobalstyle
\centering
\begin{tabular}[t]{\X{33}{99}\X{33}{99}\X{33}{99}}
\sphinxtoprule
\sphinxtableatstartofbodyhook
\sphinxAtStartPar
位域 |
&
\sphinxAtStartPar
名称     | |
&
\sphinxAtStartPar
描述                                        | |
\\
\sphinxhline
\sphinxAtStartPar
31:4
&\begin{itemize}
\item {} 
\end{itemize}
&\begin{itemize}
\item {} 
\end{itemize}
\\
\sphinxhline
\sphinxAtStartPar
3
&
\sphinxAtStartPar
PCWKSR3
&
\sphinxAtStartPar
PC3输入唤醒状态标志位                       |

\sphinxAtStartPar
唤醒后硬件置1,软件写1清除                  |

\sphinxAtStartPar
1:唤醒 0:未唤醒                           |
\\
\sphinxhline
\sphinxAtStartPar
2
&
\sphinxAtStartPar
PCWKSR2
&
\sphinxAtStartPar
PC2输入唤醒状态标志位                       |

\sphinxAtStartPar
唤醒后硬件置1,软件写1清除                  |

\sphinxAtStartPar
1:唤醒 0:未唤醒                           |
\\
\sphinxhline
\sphinxAtStartPar
1
&
\sphinxAtStartPar
PCWKSR1
&
\sphinxAtStartPar
PC1输入唤醒状态标志位                       |

\sphinxAtStartPar
唤醒后硬件置1,软件写1清除                  |

\sphinxAtStartPar
1:唤醒 0:未唤醒                           |
\\
\sphinxhline
\sphinxAtStartPar
0
&
\sphinxAtStartPar
PCWKSR0
&
\sphinxAtStartPar
PC0输入唤醒状态标志位                       |

\sphinxAtStartPar
唤醒后硬件置1,软件写1清除                  |

\sphinxAtStartPar
1:唤醒 0:未唤醒                           |
\\
\sphinxbottomrule
\end{tabular}
\sphinxtableafterendhook\par
\sphinxattableend\end{savenotes}


\subsubsection{PORTD唤醒状态寄存器PDWKSR}
\label{\detokenize{SWM241/_u529f_u80fd_u63cf_u8ff0/_u7cfb_u7edf_u7ba1_u7406:portdpdwksr}}

\begin{savenotes}\sphinxattablestart
\sphinxthistablewithglobalstyle
\centering
\begin{tabular}[t]{\X{20}{100}\X{20}{100}\X{20}{100}\X{20}{100}\X{20}{100}}
\sphinxtoprule
\sphinxtableatstartofbodyhook
\sphinxAtStartPar
寄存器 |
&
\begin{DUlineblock}{0em}
\item[] 偏移 |
\end{DUlineblock}
&
\begin{DUlineblock}{0em}
\item[] 
\item[] {\color{red}\bfseries{}|}
\end{DUlineblock}
&
\sphinxAtStartPar
复位值 |    描 | |
&
\begin{DUlineblock}{0em}
\item[] |
  |
\end{DUlineblock}
\\
\sphinxhline
\sphinxAtStartPar
PDWKSR
&
\sphinxAtStartPar
0x13C
&&
\sphinxAtStartPar
0 000000
&
\sphinxAtStartPar
PORTD唤醒状态寄存器        |
\\
\sphinxbottomrule
\end{tabular}
\sphinxtableafterendhook\par
\sphinxattableend\end{savenotes}


\begin{savenotes}\sphinxattablestart
\sphinxthistablewithglobalstyle
\centering
\begin{tabular}[t]{\X{12}{96}\X{12}{96}\X{12}{96}\X{12}{96}\X{12}{96}\X{12}{96}\X{12}{96}\X{12}{96}}
\sphinxtoprule
\sphinxtableatstartofbodyhook
\sphinxAtStartPar
31
&
\sphinxAtStartPar
30
&
\sphinxAtStartPar
29
&
\sphinxAtStartPar
28
&
\sphinxAtStartPar
27
&
\sphinxAtStartPar
26
&
\sphinxAtStartPar
25
&
\sphinxAtStartPar
24
\\
\sphinxhline\begin{itemize}
\item {} 
\end{itemize}
&&&&&&&\\
\sphinxhline
\sphinxAtStartPar
23
&
\sphinxAtStartPar
22
&
\sphinxAtStartPar
21
&
\sphinxAtStartPar
20
&
\sphinxAtStartPar
19
&
\sphinxAtStartPar
18
&
\sphinxAtStartPar
17
&
\sphinxAtStartPar
16
\\
\sphinxhline\begin{itemize}
\item {} 
\end{itemize}
&&&&&&&\\
\sphinxhline
\sphinxAtStartPar
15
&
\sphinxAtStartPar
14
&
\sphinxAtStartPar
13
&
\sphinxAtStartPar
12
&
\sphinxAtStartPar
11
&
\sphinxAtStartPar
10
&
\sphinxAtStartPar
9
&
\sphinxAtStartPar
8
\\
\sphinxhline\begin{itemize}
\item {} 
\end{itemize}
&&&&&&
\sphinxAtStartPar
P SR9
&
\sphinxAtStartPar
P SR8
\\
\sphinxhline
\sphinxAtStartPar
7
&
\sphinxAtStartPar
6
&
\sphinxAtStartPar
5
&
\sphinxAtStartPar
4
&
\sphinxAtStartPar
3
&
\sphinxAtStartPar
2
&
\sphinxAtStartPar
1
&
\sphinxAtStartPar
0
\\
\sphinxhline
\sphinxAtStartPar
PDWKSR7
&
\sphinxAtStartPar
PD WKSR6
&
\sphinxAtStartPar
P SR5
&
\sphinxAtStartPar
P SR4
&
\sphinxAtStartPar
P SR3
&
\sphinxAtStartPar
P SR2
&
\sphinxAtStartPar
P SR1
&
\sphinxAtStartPar
P SR0
\\
\sphinxbottomrule
\end{tabular}
\sphinxtableafterendhook\par
\sphinxattableend\end{savenotes}


\begin{savenotes}\sphinxattablestart
\sphinxthistablewithglobalstyle
\centering
\begin{tabular}[t]{\X{33}{99}\X{33}{99}\X{33}{99}}
\sphinxtoprule
\sphinxtableatstartofbodyhook
\sphinxAtStartPar
位域 |
&
\sphinxAtStartPar
名称     | |
&
\sphinxAtStartPar
描述                                        | |
\\
\sphinxhline
\sphinxAtStartPar
31:10
&\begin{itemize}
\item {} 
\end{itemize}
&\begin{itemize}
\item {} 
\end{itemize}
\\
\sphinxhline
\sphinxAtStartPar
9
&
\sphinxAtStartPar
PDWKSR9
&
\sphinxAtStartPar
PD9输入唤醒状态标志位                       |

\sphinxAtStartPar
唤醒后硬件置1,软件写1清除                  |

\sphinxAtStartPar
1:唤醒 0:未唤醒                           |
\\
\sphinxhline
\sphinxAtStartPar
8
&
\sphinxAtStartPar
PDWKSR8
&
\sphinxAtStartPar
PD8输入唤醒状态标志位                       |

\sphinxAtStartPar
唤醒后硬件置1,软件写1清除                  |

\sphinxAtStartPar
1:唤醒 0:未唤醒                           |
\\
\sphinxhline
\sphinxAtStartPar
7
&
\sphinxAtStartPar
PDWKSR7
&
\sphinxAtStartPar
PD7输入唤醒状态标志位                       |

\sphinxAtStartPar
唤醒后硬件置1,软件写1清除                  |

\sphinxAtStartPar
1:唤醒 0:未唤醒                           |
\\
\sphinxhline
\sphinxAtStartPar
6
&
\sphinxAtStartPar
PDWKSR6
&
\sphinxAtStartPar
PD6输入唤醒状态标志位                       |

\sphinxAtStartPar
唤醒后硬件置1,软件写1清除                  |

\sphinxAtStartPar
1:唤醒 0:未唤醒                           |
\\
\sphinxhline
\sphinxAtStartPar
5
&
\sphinxAtStartPar
PDWKSR5
&
\sphinxAtStartPar
PD5输入唤醒状态标志位                       |

\sphinxAtStartPar
唤醒后硬件置1,软件写1清除                  |

\sphinxAtStartPar
1:唤醒 0:未唤醒                           |
\\
\sphinxhline
\sphinxAtStartPar
4
&
\sphinxAtStartPar
PDWKSR4
&
\sphinxAtStartPar
PD4输入唤醒状态标志位                       |

\sphinxAtStartPar
唤醒后硬件置1,软件写1清除                  |

\sphinxAtStartPar
1:唤醒 0:未唤醒                           |
\\
\sphinxhline
\sphinxAtStartPar
3
&
\sphinxAtStartPar
PDWKSR3
&
\sphinxAtStartPar
PD3输入唤醒状态标志位                       |

\sphinxAtStartPar
唤醒后硬件置1,软件写1清除                  |

\sphinxAtStartPar
1:唤醒 0:未唤醒                           |
\\
\sphinxhline
\sphinxAtStartPar
2
&
\sphinxAtStartPar
PDWKSR2
&
\sphinxAtStartPar
PD2输入唤醒状态标志位                       |

\sphinxAtStartPar
唤醒后硬件置1,软件写1清除                  |

\sphinxAtStartPar
1:唤醒 0:未唤醒                           |
\\
\sphinxhline
\sphinxAtStartPar
1
&
\sphinxAtStartPar
PDWKSR1
&
\sphinxAtStartPar
PD1输入唤醒状态标志位                       |

\sphinxAtStartPar
唤醒后硬件置1,软件写1清除                  |

\sphinxAtStartPar
1:唤醒 0:未唤醒                           |
\\
\sphinxhline
\sphinxAtStartPar
0
&
\sphinxAtStartPar
PDWKSR0
&
\sphinxAtStartPar
PD0输入唤醒状态标志位                       |

\sphinxAtStartPar
唤醒后硬件置1,软件写1清除                  |

\sphinxAtStartPar
1:唤醒 0:未唤醒                           |
\\
\sphinxbottomrule
\end{tabular}
\sphinxtableafterendhook\par
\sphinxattableend\end{savenotes}


\subsubsection{芯片复位屏蔽寄存器PRSTEN}
\label{\detokenize{SWM241/_u529f_u80fd_u63cf_u8ff0/_u7cfb_u7edf_u7ba1_u7406:prsten}}

\begin{savenotes}\sphinxattablestart
\sphinxthistablewithglobalstyle
\centering
\begin{tabular}[t]{\X{20}{100}\X{20}{100}\X{20}{100}\X{20}{100}\X{20}{100}}
\sphinxtoprule
\sphinxtableatstartofbodyhook
\sphinxAtStartPar
寄存器 |
&
\begin{DUlineblock}{0em}
\item[] 偏移 |
\end{DUlineblock}
&
\begin{DUlineblock}{0em}
\item[] 
\item[] {\color{red}\bfseries{}|}
\end{DUlineblock}
&
\sphinxAtStartPar
复位值 |    描 | |
&
\begin{DUlineblock}{0em}
\item[] |
  |
\end{DUlineblock}
\\
\sphinxhline
\sphinxAtStartPar
PRSTEN
&
\sphinxAtStartPar
0x720
&&
\sphinxAtStartPar
0 000000
&
\sphinxAtStartPar
芯片复位屏蔽寄存器         |
\\
\sphinxbottomrule
\end{tabular}
\sphinxtableafterendhook\par
\sphinxattableend\end{savenotes}


\begin{savenotes}\sphinxattablestart
\sphinxthistablewithglobalstyle
\centering
\begin{tabular}[t]{\X{12}{96}\X{12}{96}\X{12}{96}\X{12}{96}\X{12}{96}\X{12}{96}\X{12}{96}\X{12}{96}}
\sphinxtoprule
\sphinxtableatstartofbodyhook
\sphinxAtStartPar
31
&
\sphinxAtStartPar
30
&
\sphinxAtStartPar
29
&
\sphinxAtStartPar
28
&
\sphinxAtStartPar
27
&
\sphinxAtStartPar
26
&
\sphinxAtStartPar
25
&
\sphinxAtStartPar
24
\\
\sphinxhline\begin{itemize}
\item {} 
\end{itemize}
&&&&&&&\\
\sphinxhline
\sphinxAtStartPar
23
&
\sphinxAtStartPar
22
&
\sphinxAtStartPar
21
&
\sphinxAtStartPar
20
&
\sphinxAtStartPar
19
&
\sphinxAtStartPar
18
&
\sphinxAtStartPar
17
&
\sphinxAtStartPar
16
\\
\sphinxhline\begin{itemize}
\item {} 
\end{itemize}
&&&&&&&\\
\sphinxhline
\sphinxAtStartPar
15
&
\sphinxAtStartPar
14
&
\sphinxAtStartPar
13
&
\sphinxAtStartPar
12
&
\sphinxAtStartPar
11
&
\sphinxAtStartPar
10
&
\sphinxAtStartPar
9
&
\sphinxAtStartPar
8
\\
\sphinxhline\begin{itemize}
\item {} 
\end{itemize}
&&&&&&&\\
\sphinxhline
\sphinxAtStartPar
7
&
\sphinxAtStartPar
6
&
\sphinxAtStartPar
5
&
\sphinxAtStartPar
4
&
\sphinxAtStartPar
3
&
\sphinxAtStartPar
2
&
\sphinxAtStartPar
1
&
\sphinxAtStartPar
0
\\
\sphinxhline
\sphinxAtStartPar
PRSTEN
&&&&&&&\\
\sphinxbottomrule
\end{tabular}
\sphinxtableafterendhook\par
\sphinxattableend\end{savenotes}


\begin{savenotes}\sphinxattablestart
\sphinxthistablewithglobalstyle
\centering
\begin{tabular}[t]{\X{33}{99}\X{33}{99}\X{33}{99}}
\sphinxtoprule
\sphinxtableatstartofbodyhook
\sphinxAtStartPar
位域 |
&
\sphinxAtStartPar
名称     | |
&
\sphinxAtStartPar
描述                                        | |
\\
\sphinxhline
\sphinxAtStartPar
31:8
&\begin{itemize}
\item {} 
\end{itemize}
&\begin{itemize}
\item {} 
\end{itemize}
\\
\sphinxhline
\sphinxAtStartPar
7:0
&
\sphinxAtStartPar
PRSTEN
&
\sphinxAtStartPar
只有当该寄存器                              | 0x55时,才能对PRSTR1和PRSTR2进行写操作。 |
\\
\sphinxbottomrule
\end{tabular}
\sphinxtableafterendhook\par
\sphinxattableend\end{savenotes}


\subsubsection{芯片复位配置寄存器1 PRSTR1}
\label{\detokenize{SWM241/_u529f_u80fd_u63cf_u8ff0/_u7cfb_u7edf_u7ba1_u7406:prstr1}}

\begin{savenotes}\sphinxattablestart
\sphinxthistablewithglobalstyle
\centering
\begin{tabular}[t]{\X{20}{100}\X{20}{100}\X{20}{100}\X{20}{100}\X{20}{100}}
\sphinxtoprule
\sphinxtableatstartofbodyhook
\sphinxAtStartPar
寄存器 |
&
\begin{DUlineblock}{0em}
\item[] 偏移 |
\end{DUlineblock}
&
\begin{DUlineblock}{0em}
\item[] 
\item[] {\color{red}\bfseries{}|}
\end{DUlineblock}
&
\sphinxAtStartPar
复位值 |    描 | |
&
\begin{DUlineblock}{0em}
\item[] |
  |
\end{DUlineblock}
\\
\sphinxhline
\sphinxAtStartPar
PRSTR1
&
\sphinxAtStartPar
0x724
&&
\sphinxAtStartPar
0 000000
&
\sphinxAtStartPar
芯片复位配置寄存器1        |
\\
\sphinxbottomrule
\end{tabular}
\sphinxtableafterendhook\par
\sphinxattableend\end{savenotes}


\begin{savenotes}\sphinxattablestart
\sphinxthistablewithglobalstyle
\centering
\begin{tabular}[t]{\X{12}{96}\X{12}{96}\X{12}{96}\X{12}{96}\X{12}{96}\X{12}{96}\X{12}{96}\X{12}{96}}
\sphinxtoprule
\sphinxtableatstartofbodyhook
\sphinxAtStartPar
31
&
\sphinxAtStartPar
30
&
\sphinxAtStartPar
29
&
\sphinxAtStartPar
28
&
\sphinxAtStartPar
27
&
\sphinxAtStartPar
26
&
\sphinxAtStartPar
25
&
\sphinxAtStartPar
24
\\
\sphinxhline
\sphinxAtStartPar
SLED
&\begin{itemize}
\item {} 
\end{itemize}
&&
\sphinxAtStartPar
CAN
&\begin{itemize}
\item {} 
\end{itemize}
&
\sphinxAtStartPar
S DC0
&\begin{itemize}
\item {} 
\end{itemize}
&\\
\sphinxhline
\sphinxAtStartPar
23
&
\sphinxAtStartPar
22
&
\sphinxAtStartPar
21
&
\sphinxAtStartPar
20
&
\sphinxAtStartPar
19
&
\sphinxAtStartPar
18
&
\sphinxAtStartPar
17
&
\sphinxAtStartPar
16
\\
\sphinxhline\begin{itemize}
\item {} 
\end{itemize}
&&
\sphinxAtStartPar
DIV
&\begin{itemize}
\item {} 
\end{itemize}
&
\sphinxAtStartPar
CRC
&\begin{itemize}
\item {} 
\end{itemize}
&&\\
\sphinxhline
\sphinxAtStartPar
15
&
\sphinxAtStartPar
14
&
\sphinxAtStartPar
13
&
\sphinxAtStartPar
12
&
\sphinxAtStartPar
11
&
\sphinxAtStartPar
10
&
\sphinxAtStartPar
9
&
\sphinxAtStartPar
8
\\
\sphinxhline
\sphinxAtStartPar
I2C0
&
\sphinxAtStartPar
SPI1
&&
\sphinxAtStartPar
PWM
&&
\sphinxAtStartPar
WDT
&&\\
\sphinxhline
\sphinxAtStartPar
7
&
\sphinxAtStartPar
6
&
\sphinxAtStartPar
5
&
\sphinxAtStartPar
4
&
\sphinxAtStartPar
3
&
\sphinxAtStartPar
2
&
\sphinxAtStartPar
1
&
\sphinxAtStartPar
0
\\
\sphinxhline
\sphinxAtStartPar
UART1
&
\sphinxAtStartPar
UART0
&\begin{itemize}
\item {} 
\end{itemize}
&&&&&\\
\sphinxbottomrule
\end{tabular}
\sphinxtableafterendhook\par
\sphinxattableend\end{savenotes}


\begin{savenotes}\sphinxattablestart
\sphinxthistablewithglobalstyle
\centering
\begin{tabular}[t]{\X{33}{99}\X{33}{99}\X{33}{99}}
\sphinxtoprule
\sphinxtableatstartofbodyhook
\sphinxAtStartPar
位域 |
&
\sphinxAtStartPar
名称     | |
&
\sphinxAtStartPar
描述                                        | |
\\
\sphinxhline
\sphinxAtStartPar
31
&
\sphinxAtStartPar
SLED
&
\sphinxAtStartPar
SLED模块复位配置位                          |

\sphinxAtStartPar
将该位置1,则复位该模块。                   |
\\
\sphinxhline
\sphinxAtStartPar
30
&\begin{itemize}
\item {} 
\end{itemize}
&\begin{itemize}
\item {} 
\end{itemize}
\\
\sphinxhline
\sphinxAtStartPar
29
&
\sphinxAtStartPar
SLCD
&\\
\sphinxhline
\sphinxAtStartPar
28
&
\sphinxAtStartPar
CAN
&
\sphinxAtStartPar
CAN模块复位配置位                           |
\\
\sphinxhline
\sphinxAtStartPar
27
&\begin{itemize}
\item {} 
\end{itemize}
&\begin{itemize}
\item {} 
\end{itemize}
\\
\sphinxhline
\sphinxAtStartPar
26
&
\sphinxAtStartPar
SARADC0
&
\sphinxAtStartPar
SARADC0模块数字部分复位配置位               |

\sphinxAtStartPar
将该位置1,则复位该模块。                   |
\\
\sphinxhline
\sphinxAtStartPar
25
&\begin{itemize}
\item {} 
\end{itemize}
&
\sphinxAtStartPar
*保留*                                    |

\sphinxAtStartPar
该位必须置0                                 |
\\
\sphinxhline
\sphinxAtStartPar
14:24
&\begin{itemize}
\item {} 
\end{itemize}
&\begin{itemize}
\item {} 
\end{itemize}
\\
\sphinxhline
\sphinxAtStartPar
21
&
\sphinxAtStartPar
DIV
&
\sphinxAtStartPar
DIV模块复位配置位                           |

\sphinxAtStartPar
将该位置1,则复位该模块。                   |
\\
\sphinxhline
\sphinxAtStartPar
20
&\begin{itemize}
\item {} 
\end{itemize}
&\begin{itemize}
\item {} 
\end{itemize}
\\
\sphinxhline
\sphinxAtStartPar
19
&
\sphinxAtStartPar
CRC
&
\sphinxAtStartPar
CRC模块复位配置位                           |

\sphinxAtStartPar
将该位置1,则复位该模块。                   |
\\
\sphinxhline
\sphinxAtStartPar
18:17
&\begin{itemize}
\item {} 
\end{itemize}
&\begin{itemize}
\item {} 
\end{itemize}
\\
\sphinxhline
\sphinxAtStartPar
16
&
\sphinxAtStartPar
I2C1
&
\sphinxAtStartPar
I2C1模块复位配置位                          |

\sphinxAtStartPar
将该位置1,则复位该模块。                   |
\\
\sphinxhline
\sphinxAtStartPar
15
&
\sphinxAtStartPar
I2C0
&
\sphinxAtStartPar
I2C0模块复位配置位                          |

\sphinxAtStartPar
将该位置1,则复位该模块。                   |
\\
\sphinxhline
\sphinxAtStartPar
14
&
\sphinxAtStartPar
SPI1
&
\sphinxAtStartPar
SPI1模块复位配置位                          |

\sphinxAtStartPar
将该位置1,则复位该模块。                   |
\\
\sphinxhline
\sphinxAtStartPar
13
&
\sphinxAtStartPar
SPI0
&
\sphinxAtStartPar
SPI0模块复位配置位                          |

\sphinxAtStartPar
将该位置1,则复位该模块。                   |
\\
\sphinxhline
\sphinxAtStartPar
12
&
\sphinxAtStartPar
PWM
&
\sphinxAtStartPar
PWM模块复位配置位                           |

\sphinxAtStartPar
将该位置1,则复位该模块。                   |
\\
\sphinxhline
\sphinxAtStartPar
11
&
\sphinxAtStartPar
TIMER
&
\sphinxAtStartPar
TIMER模块复位配置位                         |

\sphinxAtStartPar
将该位置1,则复位该模块。                   |
\\
\sphinxhline
\sphinxAtStartPar
10
&
\sphinxAtStartPar
WDT
&
\sphinxAtStartPar
WDT模块复位配置位                           |

\sphinxAtStartPar
将该位置1,则复位该模块。                   |
\\
\sphinxhline
\sphinxAtStartPar
9
&
\sphinxAtStartPar
UART3
&
\sphinxAtStartPar
UART3模块复位配置位                         |

\sphinxAtStartPar
将该位置1,则复位该模块。                   |
\\
\sphinxhline
\sphinxAtStartPar
8
&
\sphinxAtStartPar
UART2
&
\sphinxAtStartPar
UART2模块复位配置位                         |

\sphinxAtStartPar
将该位置1,则复位该模块。                   |
\\
\sphinxhline
\sphinxAtStartPar
7
&
\sphinxAtStartPar
UART1
&
\sphinxAtStartPar
UART1模块复位配置位                         |

\sphinxAtStartPar
将该位置1,则复位该模块。                   |
\\
\sphinxhline
\sphinxAtStartPar
6
&
\sphinxAtStartPar
UART0
&
\sphinxAtStartPar
UART0模块复位配置位                         |

\sphinxAtStartPar
将该位置1,则复位该模块。                   |
\\
\sphinxhline
\sphinxAtStartPar
5:4
&\begin{itemize}
\item {} 
\end{itemize}
&\begin{itemize}
\item {} 
\end{itemize}
\\
\sphinxhline
\sphinxAtStartPar
3
&
\sphinxAtStartPar
GPIOD
&
\sphinxAtStartPar
GPIOD模块复位配置位                         |

\sphinxAtStartPar
将该位置1,则复位该模块。                   |
\\
\sphinxhline
\sphinxAtStartPar
2
&
\sphinxAtStartPar
GPIOC
&
\sphinxAtStartPar
GPIOC模块复位配置位                         |

\sphinxAtStartPar
将该位置1,则复位该模块。                   |
\\
\sphinxhline
\sphinxAtStartPar
1
&
\sphinxAtStartPar
GPIOB
&
\sphinxAtStartPar
GPIOB模块复位配置位                         |

\sphinxAtStartPar
将该位置1,则复位该模块。                   |
\\
\sphinxhline
\sphinxAtStartPar
0
&
\sphinxAtStartPar
GPIOA
&
\sphinxAtStartPar
GPIOA模块复位配置位                         |

\sphinxAtStartPar
将该位置1,则复位该模块。                   |
\\
\sphinxbottomrule
\end{tabular}
\sphinxtableafterendhook\par
\sphinxattableend\end{savenotes}


\subsubsection{芯片复位配置寄存器2 PRSTR2}
\label{\detokenize{SWM241/_u529f_u80fd_u63cf_u8ff0/_u7cfb_u7edf_u7ba1_u7406:prstr2}}

\begin{savenotes}\sphinxattablestart
\sphinxthistablewithglobalstyle
\centering
\begin{tabular}[t]{\X{20}{100}\X{20}{100}\X{20}{100}\X{20}{100}\X{20}{100}}
\sphinxtoprule
\sphinxtableatstartofbodyhook
\sphinxAtStartPar
寄存器 |
&
\begin{DUlineblock}{0em}
\item[] 偏移 |
\end{DUlineblock}
&
\begin{DUlineblock}{0em}
\item[] 
\item[] {\color{red}\bfseries{}|}
\end{DUlineblock}
&
\sphinxAtStartPar
复位值 |    描 | |
&
\begin{DUlineblock}{0em}
\item[] |
  |
\end{DUlineblock}
\\
\sphinxhline
\sphinxAtStartPar
PRSTR2
&
\sphinxAtStartPar
0x728
&&
\sphinxAtStartPar
0 000000
&
\sphinxAtStartPar
芯片复位配置寄存器2        |
\\
\sphinxbottomrule
\end{tabular}
\sphinxtableafterendhook\par
\sphinxattableend\end{savenotes}


\begin{savenotes}\sphinxattablestart
\sphinxthistablewithglobalstyle
\centering
\begin{tabular}[t]{\X{12}{96}\X{12}{96}\X{12}{96}\X{12}{96}\X{12}{96}\X{12}{96}\X{12}{96}\X{12}{96}}
\sphinxtoprule
\sphinxtableatstartofbodyhook
\sphinxAtStartPar
31
&
\sphinxAtStartPar
30
&
\sphinxAtStartPar
29
&
\sphinxAtStartPar
28
&
\sphinxAtStartPar
27
&
\sphinxAtStartPar
26
&
\sphinxAtStartPar
25
&
\sphinxAtStartPar
24
\\
\sphinxhline\begin{itemize}
\item {} 
\end{itemize}
&&&&&&&\\
\sphinxhline
\sphinxAtStartPar
23
&
\sphinxAtStartPar
22
&
\sphinxAtStartPar
21
&
\sphinxAtStartPar
20
&
\sphinxAtStartPar
19
&
\sphinxAtStartPar
18
&
\sphinxAtStartPar
17
&
\sphinxAtStartPar
16
\\
\sphinxhline\begin{itemize}
\item {} 
\end{itemize}
&&&&
\sphinxAtStartPar
RTC
&\begin{itemize}
\item {} 
\end{itemize}
&&\\
\sphinxhline
\sphinxAtStartPar
15
&
\sphinxAtStartPar
14
&
\sphinxAtStartPar
13
&
\sphinxAtStartPar
12
&
\sphinxAtStartPar
11
&
\sphinxAtStartPar
10
&
\sphinxAtStartPar
9
&
\sphinxAtStartPar
8
\\
\sphinxhline\begin{itemize}
\item {} 
\end{itemize}
&&&&&&&\\
\sphinxhline
\sphinxAtStartPar
7
&
\sphinxAtStartPar
6
&
\sphinxAtStartPar
5
&
\sphinxAtStartPar
4
&
\sphinxAtStartPar
3
&
\sphinxAtStartPar
2
&
\sphinxAtStartPar
1
&
\sphinxAtStartPar
0
\\
\sphinxhline\begin{itemize}
\item {} 
\end{itemize}
&&&&&&&\\
\sphinxbottomrule
\end{tabular}
\sphinxtableafterendhook\par
\sphinxattableend\end{savenotes}


\begin{savenotes}\sphinxattablestart
\sphinxthistablewithglobalstyle
\centering
\begin{tabular}[t]{\X{33}{99}\X{33}{99}\X{33}{99}}
\sphinxtoprule
\sphinxtableatstartofbodyhook
\sphinxAtStartPar
位域 |
&
\sphinxAtStartPar
名称     | |
&
\sphinxAtStartPar
描述                                        | |
\\
\sphinxhline
\sphinxAtStartPar
31:20
&\begin{itemize}
\item {} 
\end{itemize}
&\begin{itemize}
\item {} 
\end{itemize}
\\
\sphinxhline
\sphinxAtStartPar
19
&
\sphinxAtStartPar
RTC
&
\sphinxAtStartPar
RTC\_BASE模块复位配置位                      |

\sphinxAtStartPar
将该位置1,则复位该模块。                   |
\\
\sphinxhline
\sphinxAtStartPar
18:0
&\begin{itemize}
\item {} 
\end{itemize}
&\begin{itemize}
\item {} 
\end{itemize}
\\
\sphinxbottomrule
\end{tabular}
\sphinxtableafterendhook\par
\sphinxattableend\end{savenotes}


\subsubsection{内部高频RC振荡器配置寄存器HRCCR}
\label{\detokenize{SWM241/_u529f_u80fd_u63cf_u8ff0/_u7cfb_u7edf_u7ba1_u7406:rchrccr}}

\begin{savenotes}\sphinxattablestart
\sphinxthistablewithglobalstyle
\centering
\begin{tabular}[t]{\X{20}{100}\X{20}{100}\X{20}{100}\X{20}{100}\X{20}{100}}
\sphinxtoprule
\sphinxtableatstartofbodyhook
\sphinxAtStartPar
寄存器 |
&
\begin{DUlineblock}{0em}
\item[] 偏移 |
\end{DUlineblock}
&
\begin{DUlineblock}{0em}
\item[] 
\item[] {\color{red}\bfseries{}|}
\end{DUlineblock}
&
\sphinxAtStartPar
复位值 |    描 | |
&
\begin{DUlineblock}{0em}
\item[] |
  |
\end{DUlineblock}
\\
\sphinxhline
\sphinxAtStartPar
HRCCR
&
\sphinxAtStartPar
0x00
&&
\sphinxAtStartPar
0 000001
&
\sphinxAtStartPar
内部高频RC振荡器配置寄存器 |
\\
\sphinxbottomrule
\end{tabular}
\sphinxtableafterendhook\par
\sphinxattableend\end{savenotes}


\begin{savenotes}\sphinxattablestart
\sphinxthistablewithglobalstyle
\centering
\begin{tabular}[t]{\X{12}{96}\X{12}{96}\X{12}{96}\X{12}{96}\X{12}{96}\X{12}{96}\X{12}{96}\X{12}{96}}
\sphinxtoprule
\sphinxtableatstartofbodyhook
\sphinxAtStartPar
31
&
\sphinxAtStartPar
30
&
\sphinxAtStartPar
29
&
\sphinxAtStartPar
28
&
\sphinxAtStartPar
27
&
\sphinxAtStartPar
26
&
\sphinxAtStartPar
25
&
\sphinxAtStartPar
24
\\
\sphinxhline\begin{itemize}
\item {} 
\end{itemize}
&&&&&&&\\
\sphinxhline
\sphinxAtStartPar
23
&
\sphinxAtStartPar
22
&
\sphinxAtStartPar
21
&
\sphinxAtStartPar
20
&
\sphinxAtStartPar
19
&
\sphinxAtStartPar
18
&
\sphinxAtStartPar
17
&
\sphinxAtStartPar
16
\\
\sphinxhline\begin{itemize}
\item {} 
\end{itemize}
&&&&&&&\\
\sphinxhline
\sphinxAtStartPar
15
&
\sphinxAtStartPar
14
&
\sphinxAtStartPar
13
&
\sphinxAtStartPar
12
&
\sphinxAtStartPar
11
&
\sphinxAtStartPar
10
&
\sphinxAtStartPar
9
&
\sphinxAtStartPar
8
\\
\sphinxhline\begin{itemize}
\item {} 
\end{itemize}
&&&&&&&\\
\sphinxhline
\sphinxAtStartPar
7
&
\sphinxAtStartPar
6
&
\sphinxAtStartPar
5
&
\sphinxAtStartPar
4
&
\sphinxAtStartPar
3
&
\sphinxAtStartPar
2
&
\sphinxAtStartPar
1
&
\sphinxAtStartPar
0
\\
\sphinxhline\begin{itemize}
\item {} 
\end{itemize}
&&&&&&&
\sphinxAtStartPar
ON
\\
\sphinxbottomrule
\end{tabular}
\sphinxtableafterendhook\par
\sphinxattableend\end{savenotes}


\begin{savenotes}\sphinxattablestart
\sphinxthistablewithglobalstyle
\centering
\begin{tabular}[t]{\X{33}{99}\X{33}{99}\X{33}{99}}
\sphinxtoprule
\sphinxtableatstartofbodyhook
\sphinxAtStartPar
位域 |
&
\sphinxAtStartPar
名称     | |
&
\sphinxAtStartPar
描述                                        | |
\\
\sphinxhline
\sphinxAtStartPar
31:2
&\begin{itemize}
\item {} 
\end{itemize}
&\begin{itemize}
\item {} 
\end{itemize}
\\
\sphinxhline
\sphinxAtStartPar
0
&
\sphinxAtStartPar
ON
&
\sphinxAtStartPar
内部高频RC振荡器使能                        |

\sphinxAtStartPar
0:关闭                                     |

\sphinxAtStartPar
1:开启                                     |
\\
\sphinxbottomrule
\end{tabular}
\sphinxtableafterendhook\par
\sphinxattableend\end{savenotes}


\subsubsection{BOD控制寄存器BODCR}
\label{\detokenize{SWM241/_u529f_u80fd_u63cf_u8ff0/_u7cfb_u7edf_u7ba1_u7406:bodbodcr}}

\begin{savenotes}\sphinxattablestart
\sphinxthistablewithglobalstyle
\centering
\begin{tabular}[t]{\X{20}{100}\X{20}{100}\X{20}{100}\X{20}{100}\X{20}{100}}
\sphinxtoprule
\sphinxtableatstartofbodyhook
\sphinxAtStartPar
寄存器 |
&
\begin{DUlineblock}{0em}
\item[] 偏移 |
\end{DUlineblock}
&
\begin{DUlineblock}{0em}
\item[] 
\item[] {\color{red}\bfseries{}|}
\end{DUlineblock}
&
\sphinxAtStartPar
复位值 |    描 | |
&
\begin{DUlineblock}{0em}
\item[] |
  |
\end{DUlineblock}
\\
\sphinxhline
\sphinxAtStartPar
BODCR
&
\sphinxAtStartPar
0x10
&&
\sphinxAtStartPar
0 000000
&
\sphinxAtStartPar
BOD控制寄存器              |
\\
\sphinxbottomrule
\end{tabular}
\sphinxtableafterendhook\par
\sphinxattableend\end{savenotes}


\begin{savenotes}\sphinxattablestart
\sphinxthistablewithglobalstyle
\centering
\begin{tabular}[t]{\X{12}{96}\X{12}{96}\X{12}{96}\X{12}{96}\X{12}{96}\X{12}{96}\X{12}{96}\X{12}{96}}
\sphinxtoprule
\sphinxtableatstartofbodyhook
\sphinxAtStartPar
31
&
\sphinxAtStartPar
30
&
\sphinxAtStartPar
29
&
\sphinxAtStartPar
28
&
\sphinxAtStartPar
27
&
\sphinxAtStartPar
26
&
\sphinxAtStartPar
25
&
\sphinxAtStartPar
24
\\
\sphinxhline\begin{itemize}
\item {} 
\end{itemize}
&&&&&&&\\
\sphinxhline
\sphinxAtStartPar
23
&
\sphinxAtStartPar
22
&
\sphinxAtStartPar
21
&
\sphinxAtStartPar
20
&
\sphinxAtStartPar
19
&
\sphinxAtStartPar
18
&
\sphinxAtStartPar
17
&
\sphinxAtStartPar
16
\\
\sphinxhline\begin{itemize}
\item {} 
\end{itemize}
&&&&&&&\\
\sphinxhline
\sphinxAtStartPar
15
&
\sphinxAtStartPar
14
&
\sphinxAtStartPar
13
&
\sphinxAtStartPar
12
&
\sphinxAtStartPar
11
&
\sphinxAtStartPar
10
&
\sphinxAtStartPar
9
&
\sphinxAtStartPar
8
\\
\sphinxhline\begin{itemize}
\item {} 
\end{itemize}
&&&&&&&\\
\sphinxhline
\sphinxAtStartPar
7
&
\sphinxAtStartPar
6
&
\sphinxAtStartPar
5
&
\sphinxAtStartPar
4
&
\sphinxAtStartPar
3
&
\sphinxAtStartPar
2
&
\sphinxAtStartPar
1
&
\sphinxAtStartPar
0
\\
\sphinxhline
\sphinxAtStartPar
RSTLVL
&
\sphinxAtStartPar
INTLVL
&&&\begin{itemize}
\item {} 
\end{itemize}
&&
\sphinxAtStartPar
IE
&\begin{itemize}
\item {} 
\end{itemize}
\\
\sphinxbottomrule
\end{tabular}
\sphinxtableafterendhook\par
\sphinxattableend\end{savenotes}


\begin{savenotes}\sphinxattablestart
\sphinxthistablewithglobalstyle
\centering
\begin{tabular}[t]{\X{33}{99}\X{33}{99}\X{33}{99}}
\sphinxtoprule
\sphinxtableatstartofbodyhook
\sphinxAtStartPar
位域 |
&
\sphinxAtStartPar
名称     | |
&
\sphinxAtStartPar
描述                                        | |
\\
\sphinxhline
\sphinxAtStartPar
31:7
&\begin{itemize}
\item {} 
\end{itemize}
&\begin{itemize}
\item {} 
\end{itemize}
\\
\sphinxhline
\sphinxAtStartPar
9:7
&
\sphinxAtStartPar
RSTLVL
&
\sphinxAtStartPar
BOD复位电位配置寄存器                       |

\sphinxAtStartPar
000: BOD 1.7V产生复位                       |

\sphinxAtStartPar
001: BOD 1.9V产生复位                       |

\sphinxAtStartPar
010: BOD 2.1V产生复位                       |

\sphinxAtStartPar
011: BOD 2.7V产生复位                       |

\sphinxAtStartPar
100: BOD 3.5V产生复位                       |
\\
\sphinxhline
\sphinxAtStartPar
6:4
&
\sphinxAtStartPar
INTLVL
&
\sphinxAtStartPar
BOD中断电位配置寄存器                       |

\sphinxAtStartPar
000: BOD 1.9V产生中断                       |

\sphinxAtStartPar
001: BOD 2.1V产生中断                       |

\sphinxAtStartPar
010: BOD 2.3V产生中断                       |

\sphinxAtStartPar
011: BOD 2.5V产生中断                       |

\sphinxAtStartPar
100: BOD 2.7V产生中断                       |

\sphinxAtStartPar
101: BOD 3.5V产生中断                       |

\sphinxAtStartPar
110: BOD 4.1V产生中断                       |
\\
\sphinxhline
\sphinxAtStartPar
3:2
&\begin{itemize}
\item {} 
\end{itemize}
&\begin{itemize}
\item {} 
\end{itemize}
\\
\sphinxhline
\sphinxAtStartPar
1
&
\sphinxAtStartPar
IE
&
\sphinxAtStartPar
BOD中断功能使能寄存器                       |

\sphinxAtStartPar
1:使能                                     |

\sphinxAtStartPar
0:关闭                                     |
\\
\sphinxhline
\sphinxAtStartPar
0
&\begin{itemize}
\item {} 
\end{itemize}
&\begin{itemize}
\item {} 
\end{itemize}
\\
\sphinxbottomrule
\end{tabular}
\sphinxtableafterendhook\par
\sphinxattableend\end{savenotes}


\subsubsection{BOD中断状态寄存器BODSR}
\label{\detokenize{SWM241/_u529f_u80fd_u63cf_u8ff0/_u7cfb_u7edf_u7ba1_u7406:bodbodsr}}

\begin{savenotes}\sphinxattablestart
\sphinxthistablewithglobalstyle
\centering
\begin{tabular}[t]{\X{20}{100}\X{20}{100}\X{20}{100}\X{20}{100}\X{20}{100}}
\sphinxtoprule
\sphinxtableatstartofbodyhook
\sphinxAtStartPar
寄存器 |
&
\begin{DUlineblock}{0em}
\item[] 偏移 |
\end{DUlineblock}
&
\begin{DUlineblock}{0em}
\item[] 
\item[] {\color{red}\bfseries{}|}
\end{DUlineblock}
&
\sphinxAtStartPar
复位值 |    描 | |
&
\begin{DUlineblock}{0em}
\item[] |
  |
\end{DUlineblock}
\\
\sphinxhline
\sphinxAtStartPar
BODSR
&
\sphinxAtStartPar
0x14
&&&
\sphinxAtStartPar
BOD中断状态寄存器          |
\\
\sphinxbottomrule
\end{tabular}
\sphinxtableafterendhook\par
\sphinxattableend\end{savenotes}


\begin{savenotes}\sphinxattablestart
\sphinxthistablewithglobalstyle
\centering
\begin{tabular}[t]{\X{12}{96}\X{12}{96}\X{12}{96}\X{12}{96}\X{12}{96}\X{12}{96}\X{12}{96}\X{12}{96}}
\sphinxtoprule
\sphinxtableatstartofbodyhook
\sphinxAtStartPar
31
&
\sphinxAtStartPar
30
&
\sphinxAtStartPar
29
&
\sphinxAtStartPar
28
&
\sphinxAtStartPar
27
&
\sphinxAtStartPar
26
&
\sphinxAtStartPar
25
&
\sphinxAtStartPar
24
\\
\sphinxhline\begin{itemize}
\item {} 
\end{itemize}
&&&&&&&\\
\sphinxhline
\sphinxAtStartPar
23
&
\sphinxAtStartPar
22
&
\sphinxAtStartPar
21
&
\sphinxAtStartPar
20
&
\sphinxAtStartPar
19
&
\sphinxAtStartPar
18
&
\sphinxAtStartPar
17
&
\sphinxAtStartPar
16
\\
\sphinxhline\begin{itemize}
\item {} 
\end{itemize}
&&&&&&&\\
\sphinxhline
\sphinxAtStartPar
15
&
\sphinxAtStartPar
14
&
\sphinxAtStartPar
13
&
\sphinxAtStartPar
12
&
\sphinxAtStartPar
11
&
\sphinxAtStartPar
10
&
\sphinxAtStartPar
9
&
\sphinxAtStartPar
8
\\
\sphinxhline\begin{itemize}
\item {} 
\end{itemize}
&&&&&&&\\
\sphinxhline
\sphinxAtStartPar
7
&
\sphinxAtStartPar
6
&
\sphinxAtStartPar
5
&
\sphinxAtStartPar
4
&
\sphinxAtStartPar
3
&
\sphinxAtStartPar
2
&
\sphinxAtStartPar
1
&
\sphinxAtStartPar
0
\\
\sphinxhline\begin{itemize}
\item {} 
\end{itemize}
&&&&&&&
\sphinxAtStartPar
IF
\\
\sphinxbottomrule
\end{tabular}
\sphinxtableafterendhook\par
\sphinxattableend\end{savenotes}


\begin{savenotes}\sphinxattablestart
\sphinxthistablewithglobalstyle
\centering
\begin{tabular}[t]{\X{33}{99}\X{33}{99}\X{33}{99}}
\sphinxtoprule
\sphinxtableatstartofbodyhook
\sphinxAtStartPar
位域 |
&
\sphinxAtStartPar
名称     | |
&
\sphinxAtStartPar
描述                                        | |
\\
\sphinxhline
\sphinxAtStartPar
31:1
&\begin{itemize}
\item {} 
\end{itemize}
&\begin{itemize}
\item {} 
\end{itemize}
\\
\sphinxhline
\sphinxAtStartPar
0
&
\sphinxAtStartPar
IF
&
\sphinxAtStartPar
BOD中断状态标志位,写1清除                  |

\sphinxAtStartPar
1:已触发中断电压                           |

\sphinxAtStartPar
0:未触发中断电压                           |

\sphinxAtStartPar
注:只有当BODCR.IE=1时,BODSR.IF才会置位    |
\\
\sphinxbottomrule
\end{tabular}
\sphinxtableafterendhook\par
\sphinxattableend\end{savenotes}


\subsubsection{晶体振荡器控制寄存器XTALCR}
\label{\detokenize{SWM241/_u529f_u80fd_u63cf_u8ff0/_u7cfb_u7edf_u7ba1_u7406:xtalcr}}

\begin{savenotes}\sphinxattablestart
\sphinxthistablewithglobalstyle
\centering
\begin{tabular}[t]{\X{20}{100}\X{20}{100}\X{20}{100}\X{20}{100}\X{20}{100}}
\sphinxtoprule
\sphinxtableatstartofbodyhook
\sphinxAtStartPar
寄存器 |
&
\begin{DUlineblock}{0em}
\item[] 偏移 |
\end{DUlineblock}
&
\begin{DUlineblock}{0em}
\item[] 
\item[] {\color{red}\bfseries{}|}
\end{DUlineblock}
&
\sphinxAtStartPar
复位值 |    描 | |
&
\begin{DUlineblock}{0em}
\item[] |
  |
\end{DUlineblock}
\\
\sphinxhline
\sphinxAtStartPar
XTALCR
&
\sphinxAtStartPar
0x20
&&
\sphinxAtStartPar
0 000000
&
\sphinxAtStartPar
晶体振荡器控制寄存器       |
\\
\sphinxbottomrule
\end{tabular}
\sphinxtableafterendhook\par
\sphinxattableend\end{savenotes}


\begin{savenotes}\sphinxattablestart
\sphinxthistablewithglobalstyle
\centering
\begin{tabular}[t]{\X{12}{96}\X{12}{96}\X{12}{96}\X{12}{96}\X{12}{96}\X{12}{96}\X{12}{96}\X{12}{96}}
\sphinxtoprule
\sphinxtableatstartofbodyhook
\sphinxAtStartPar
31
&
\sphinxAtStartPar
30
&
\sphinxAtStartPar
29
&
\sphinxAtStartPar
28
&
\sphinxAtStartPar
27
&
\sphinxAtStartPar
26
&
\sphinxAtStartPar
25
&
\sphinxAtStartPar
24
\\
\sphinxhline\begin{itemize}
\item {} 
\end{itemize}
&&&&&&&\\
\sphinxhline
\sphinxAtStartPar
23
&
\sphinxAtStartPar
22
&
\sphinxAtStartPar
21
&
\sphinxAtStartPar
20
&
\sphinxAtStartPar
19
&
\sphinxAtStartPar
18
&
\sphinxAtStartPar
17
&
\sphinxAtStartPar
16
\\
\sphinxhline\begin{itemize}
\item {} 
\end{itemize}
&&&
\sphinxAtStartPar
DRV
&&&&\\
\sphinxhline
\sphinxAtStartPar
15
&
\sphinxAtStartPar
14
&
\sphinxAtStartPar
13
&
\sphinxAtStartPar
12
&
\sphinxAtStartPar
11
&
\sphinxAtStartPar
10
&
\sphinxAtStartPar
9
&
\sphinxAtStartPar
8
\\
\sphinxhline\begin{itemize}
\item {} 
\end{itemize}
&&&&&&&\\
\sphinxhline
\sphinxAtStartPar
7
&
\sphinxAtStartPar
6
&
\sphinxAtStartPar
5
&
\sphinxAtStartPar
4
&
\sphinxAtStartPar
3
&
\sphinxAtStartPar
2
&
\sphinxAtStartPar
1
&
\sphinxAtStartPar
0
\\
\sphinxhline\begin{itemize}
\item {} 
\end{itemize}
&&
\sphinxAtStartPar
DET
&&\begin{itemize}
\item {} 
\end{itemize}
&&
\sphinxAtStartPar
ON
&\\
\sphinxbottomrule
\end{tabular}
\sphinxtableafterendhook\par
\sphinxattableend\end{savenotes}


\begin{savenotes}\sphinxattablestart
\sphinxthistablewithglobalstyle
\centering
\begin{tabular}[t]{\X{33}{99}\X{33}{99}\X{33}{99}}
\sphinxtoprule
\sphinxtableatstartofbodyhook
\sphinxAtStartPar
位域 |
&
\sphinxAtStartPar
名称     | |
&
\sphinxAtStartPar
描述                                        | |
\\
\sphinxhline
\sphinxAtStartPar
31:21
&\begin{itemize}
\item {} 
\end{itemize}
&\begin{itemize}
\item {} 
\end{itemize}
\\
\sphinxhline
\sphinxAtStartPar
20:16
&
\sphinxAtStartPar
DRV
&
\sphinxAtStartPar
高频晶体振荡器驱动能力控制信号              |

\sphinxAtStartPar
每bit位控制的驱动能力一样,将该寄存器配置   | it为1,则表示有几倍的驱动能力,可微调频率 |
\\
\sphinxhline
\sphinxAtStartPar
15:12
&\begin{itemize}
\item {} 
\end{itemize}
&\begin{itemize}
\item {} 
\end{itemize}
\\
\sphinxhline
\sphinxAtStartPar
11:8
&
\sphinxAtStartPar
32KDRV
&
\sphinxAtStartPar
32K低频晶振偏置电流控制信号,32K            | 驱动能力,可微调频率                        |
\\
\sphinxhline
\sphinxAtStartPar
7:6
&\begin{itemize}
\item {} 
\end{itemize}
&\begin{itemize}
\item {} 
\end{itemize}
\\
\sphinxhline
\sphinxAtStartPar
5
&
\sphinxAtStartPar
DET
&
\sphinxAtStartPar
外接高频晶振停振检测                        |

\sphinxAtStartPar
0:关闭                                     |

\sphinxAtStartPar
1:开启                                     |
\\
\sphinxhline
\sphinxAtStartPar
4
&
\sphinxAtStartPar
32KDET
&
\sphinxAtStartPar
外接低频晶振停振检测                        |

\sphinxAtStartPar
0:关闭                                     |

\sphinxAtStartPar
1:开启                                     |
\\
\sphinxhline
\sphinxAtStartPar
3:2
&\begin{itemize}
\item {} 
\end{itemize}
&\begin{itemize}
\item {} 
\end{itemize}
\\
\sphinxhline
\sphinxAtStartPar
1
&
\sphinxAtStartPar
ON
&
\sphinxAtStartPar
外接高频晶振使能                            |

\sphinxAtStartPar
0:关闭                                     |

\sphinxAtStartPar
1:开启                                     |
\\
\sphinxhline
\sphinxAtStartPar
0
&
\sphinxAtStartPar
32KON
&
\sphinxAtStartPar
外接低频晶振使能                            |

\sphinxAtStartPar
0:关闭                                     |

\sphinxAtStartPar
1:开启                                     |
\\
\sphinxbottomrule
\end{tabular}
\sphinxtableafterendhook\par
\sphinxattableend\end{savenotes}


\subsubsection{晶体振荡器状态寄存器XTALSR}
\label{\detokenize{SWM241/_u529f_u80fd_u63cf_u8ff0/_u7cfb_u7edf_u7ba1_u7406:xtalsr}}

\begin{savenotes}\sphinxattablestart
\sphinxthistablewithglobalstyle
\centering
\begin{tabular}[t]{\X{20}{100}\X{20}{100}\X{20}{100}\X{20}{100}\X{20}{100}}
\sphinxtoprule
\sphinxtableatstartofbodyhook
\sphinxAtStartPar
寄存器 |
&
\begin{DUlineblock}{0em}
\item[] 偏移 |
\end{DUlineblock}
&
\begin{DUlineblock}{0em}
\item[] 
\item[] {\color{red}\bfseries{}|}
\end{DUlineblock}
&
\sphinxAtStartPar
复位值 |    描 | |
&
\begin{DUlineblock}{0em}
\item[] |
  |
\end{DUlineblock}
\\
\sphinxhline
\sphinxAtStartPar
XTALSR
&
\sphinxAtStartPar
0x24
&&
\sphinxAtStartPar
0 000000
&
\sphinxAtStartPar
晶体振荡器状态寄存器       |
\\
\sphinxbottomrule
\end{tabular}
\sphinxtableafterendhook\par
\sphinxattableend\end{savenotes}


\begin{savenotes}\sphinxattablestart
\sphinxthistablewithglobalstyle
\centering
\begin{tabular}[t]{\X{12}{96}\X{12}{96}\X{12}{96}\X{12}{96}\X{12}{96}\X{12}{96}\X{12}{96}\X{12}{96}}
\sphinxtoprule
\sphinxtableatstartofbodyhook
\sphinxAtStartPar
31
&
\sphinxAtStartPar
30
&
\sphinxAtStartPar
29
&
\sphinxAtStartPar
28
&
\sphinxAtStartPar
27
&
\sphinxAtStartPar
26
&
\sphinxAtStartPar
25
&
\sphinxAtStartPar
24
\\
\sphinxhline\begin{itemize}
\item {} 
\end{itemize}
&&&&&&&\\
\sphinxhline
\sphinxAtStartPar
23
&
\sphinxAtStartPar
22
&
\sphinxAtStartPar
21
&
\sphinxAtStartPar
20
&
\sphinxAtStartPar
19
&
\sphinxAtStartPar
18
&
\sphinxAtStartPar
17
&
\sphinxAtStartPar
16
\\
\sphinxhline\begin{itemize}
\item {} 
\end{itemize}
&&&&&&&\\
\sphinxhline
\sphinxAtStartPar
15
&
\sphinxAtStartPar
14
&
\sphinxAtStartPar
13
&
\sphinxAtStartPar
12
&
\sphinxAtStartPar
11
&
\sphinxAtStartPar
10
&
\sphinxAtStartPar
9
&
\sphinxAtStartPar
8
\\
\sphinxhline\begin{itemize}
\item {} 
\end{itemize}
&&&&&&&\\
\sphinxhline
\sphinxAtStartPar
7
&
\sphinxAtStartPar
6
&
\sphinxAtStartPar
5
&
\sphinxAtStartPar
4
&
\sphinxAtStartPar
3
&
\sphinxAtStartPar
2
&
\sphinxAtStartPar
1
&
\sphinxAtStartPar
0
\\
\sphinxhline\begin{itemize}
\item {} 
\end{itemize}
&&&&&&&
\sphinxAtStartPar
3 TOP
\\
\sphinxbottomrule
\end{tabular}
\sphinxtableafterendhook\par
\sphinxattableend\end{savenotes}


\begin{savenotes}\sphinxattablestart
\sphinxthistablewithglobalstyle
\centering
\begin{tabular}[t]{\X{33}{99}\X{33}{99}\X{33}{99}}
\sphinxtoprule
\sphinxtableatstartofbodyhook
\sphinxAtStartPar
位域 |
&
\sphinxAtStartPar
名称     | |
&
\sphinxAtStartPar
描述                                        | |
\\
\sphinxhline
\sphinxAtStartPar
31:2
&\begin{itemize}
\item {} 
\end{itemize}
&\begin{itemize}
\item {} 
\end{itemize}
\\
\sphinxhline
\sphinxAtStartPar
1
&
\sphinxAtStartPar
STOP
&
\sphinxAtStartPar
外接高频晶振状态,写1清0                    |

\sphinxAtStartPar
0:正常                                     |

\sphinxAtStartPar
1:停振,发生停震后将自动切换至HRC          |
\\
\sphinxhline
\sphinxAtStartPar
0
&
\sphinxAtStartPar
32KSTOP
&
\sphinxAtStartPar
外接低频晶振状态,写1清0                    |

\sphinxAtStartPar
0:正常                                     |

\sphinxAtStartPar
1:停振                                     |
\\
\sphinxbottomrule
\end{tabular}
\sphinxtableafterendhook\par
\sphinxattableend\end{savenotes}


\subsubsection{内部低频RC配置寄存器LRCCR}
\label{\detokenize{SWM241/_u529f_u80fd_u63cf_u8ff0/_u7cfb_u7edf_u7ba1_u7406:rclrccr}}

\begin{savenotes}\sphinxattablestart
\sphinxthistablewithglobalstyle
\centering
\begin{tabular}[t]{\X{20}{100}\X{20}{100}\X{20}{100}\X{20}{100}\X{20}{100}}
\sphinxtoprule
\sphinxtableatstartofbodyhook
\sphinxAtStartPar
寄存器 |
&
\begin{DUlineblock}{0em}
\item[] 偏移 |
\end{DUlineblock}
&
\begin{DUlineblock}{0em}
\item[] 
\item[] {\color{red}\bfseries{}|}
\end{DUlineblock}
&
\sphinxAtStartPar
复位值 |    描 | |
&
\begin{DUlineblock}{0em}
\item[] |
  |
\end{DUlineblock}
\\
\sphinxhline
\sphinxAtStartPar
LRCCR
&
\sphinxAtStartPar
0x50
&&
\sphinxAtStartPar
0 000001
&
\sphinxAtStartPar
内部低频RC配置寄存器       |
\\
\sphinxbottomrule
\end{tabular}
\sphinxtableafterendhook\par
\sphinxattableend\end{savenotes}


\begin{savenotes}\sphinxattablestart
\sphinxthistablewithglobalstyle
\centering
\begin{tabular}[t]{\X{12}{96}\X{12}{96}\X{12}{96}\X{12}{96}\X{12}{96}\X{12}{96}\X{12}{96}\X{12}{96}}
\sphinxtoprule
\sphinxtableatstartofbodyhook
\sphinxAtStartPar
31
&
\sphinxAtStartPar
30
&
\sphinxAtStartPar
29
&
\sphinxAtStartPar
28
&
\sphinxAtStartPar
27
&
\sphinxAtStartPar
26
&
\sphinxAtStartPar
25
&
\sphinxAtStartPar
24
\\
\sphinxhline\begin{itemize}
\item {} 
\end{itemize}
&&&&&&&\\
\sphinxhline
\sphinxAtStartPar
23
&
\sphinxAtStartPar
22
&
\sphinxAtStartPar
21
&
\sphinxAtStartPar
20
&
\sphinxAtStartPar
19
&
\sphinxAtStartPar
18
&
\sphinxAtStartPar
17
&
\sphinxAtStartPar
16
\\
\sphinxhline\begin{itemize}
\item {} 
\end{itemize}
&&&&&&&\\
\sphinxhline
\sphinxAtStartPar
15
&
\sphinxAtStartPar
14
&
\sphinxAtStartPar
13
&
\sphinxAtStartPar
12
&
\sphinxAtStartPar
11
&
\sphinxAtStartPar
10
&
\sphinxAtStartPar
9
&
\sphinxAtStartPar
8
\\
\sphinxhline\begin{itemize}
\item {} 
\end{itemize}
&&&&&&&\\
\sphinxhline
\sphinxAtStartPar
7
&
\sphinxAtStartPar
6
&
\sphinxAtStartPar
5
&
\sphinxAtStartPar
4
&
\sphinxAtStartPar
3
&
\sphinxAtStartPar
2
&
\sphinxAtStartPar
1
&
\sphinxAtStartPar
0
\\
\sphinxhline\begin{itemize}
\item {} 
\end{itemize}
&&&&&&&
\sphinxAtStartPar
ON
\\
\sphinxbottomrule
\end{tabular}
\sphinxtableafterendhook\par
\sphinxattableend\end{savenotes}


\begin{savenotes}\sphinxattablestart
\sphinxthistablewithglobalstyle
\centering
\begin{tabular}[t]{\X{33}{99}\X{33}{99}\X{33}{99}}
\sphinxtoprule
\sphinxtableatstartofbodyhook
\sphinxAtStartPar
位域 |
&
\sphinxAtStartPar
名称     | |
&
\sphinxAtStartPar
描述                                        | |
\\
\sphinxhline
\sphinxAtStartPar
31:1
&\begin{itemize}
\item {} 
\end{itemize}
&\begin{itemize}
\item {} 
\end{itemize}
\\
\sphinxhline
\sphinxAtStartPar
0
&
\sphinxAtStartPar
ON
&
\sphinxAtStartPar
内置低频RC使能                              |

\sphinxAtStartPar
0:关闭                                     |

\sphinxAtStartPar
1:开启                                     |
\\
\sphinxbottomrule
\end{tabular}
\sphinxtableafterendhook\par
\sphinxattableend\end{savenotes}

\sphinxstepscope


\section{PORTCON}
\label{\detokenize{SWM241/_u529f_u80fd_u63cf_u8ff0/PORTCON:portcon}}\label{\detokenize{SWM241/_u529f_u80fd_u63cf_u8ff0/PORTCON::doc}}
\sphinxAtStartPar
概述
\textasciitilde{}\textasciitilde{}

\sphinxAtStartPar
端口控制模块主要包括管脚输入使能,管脚功能配置,I/O上拉、下拉、开漏配置。SWM241系列所有型号PORTCON模块操作均相同,部分型号无对应管脚时,对应寄存器位无效。

\sphinxAtStartPar
特性
\textasciitilde{}\textasciitilde{}
\begin{itemize}
\item {} 
\sphinxAtStartPar
配置I/O引脚为特定功能

\item {} 
\sphinxAtStartPar
支持上拉/下拉/推挽/开漏功能

\item {} 
\sphinxAtStartPar
配置管脚输入使能

\end{itemize}


\subsection{模块结构框图}
\label{\detokenize{SWM241/_u529f_u80fd_u63cf_u8ff0/PORTCON:id1}}
\sphinxAtStartPar
\sphinxincludegraphics{{SWM241/功能描述/mediaPORTCO002}.emf}

\sphinxAtStartPar
图 6‑3 PORTCON模块结构框图


\subsection{功能描述}
\label{\detokenize{SWM241/_u529f_u80fd_u63cf_u8ff0/PORTCON:id2}}
\sphinxAtStartPar
端口控制模块主要包括管脚输入使能,管脚功能配置,I/O上拉、下拉、开漏配置。SWM241系列所有型号PORTCON模块操作均相同,部分型号无对应管脚时,对应寄存器位无效。


\subsubsection{引脚输入使能}
\label{\detokenize{SWM241/_u529f_u80fd_u63cf_u8ff0/PORTCON:id3}}
\sphinxAtStartPar
本芯片引脚作为输入或需要输入的外设时,需要打开引脚对应输入使能寄存器(INEN\_x),当引脚所在寄存器对应位设置为1时,输入使能打开,引脚可获取外部状态。


\subsubsection{功能选择配置}
\label{\detokenize{SWM241/_u529f_u80fd_u63cf_u8ff0/PORTCON:id4}}
\sphinxAtStartPar
端口复用通过端口复用寄存器PORTx\_SEL寄存器实现。当指定位配置为对应值时,引脚功能实现切换。

\sphinxAtStartPar
每个端口可能具备以下功能:
\begin{itemize}
\item {} 
\sphinxAtStartPar
通用输入输出接口:引脚作为通用输入输出功能,输入或输出指定数字电平

\item {} 
\sphinxAtStartPar
外设接口:将对应引脚切换至指定数字功能,如TIMER/UART/PWM等

\item {} 
\sphinxAtStartPar
模拟接口:将对应引脚切换至模拟功能,如时钟输入等

\item {} 
\sphinxAtStartPar
下载接口:使用仿真器连接下载程序及单步执行

\end{itemize}


\subsubsection{上拉/下拉/推挽/开漏配置}
\label{\detokenize{SWM241/_u529f_u80fd_u63cf_u8ff0/PORTCON:id5}}
\sphinxAtStartPar
本芯片每个引脚均可配置为以下模式:
\begin{itemize}
\item {} 
\sphinxAtStartPar
上拉输入

\item {} 
\sphinxAtStartPar
下拉输入

\item {} 
\sphinxAtStartPar
推挽输出

\item {} 
\sphinxAtStartPar
开漏输出

\end{itemize}

\sphinxAtStartPar
当对应引脚作为除GPIO之外的功能引脚时,此配置同样生效。

\sphinxAtStartPar
作为输入功能使用时,GPIO DIR寄存器对应位为0,该状态为上电默认状态。此时可以开启内部上拉和下拉功能,通过配置PULLU及PULLD寄存器实现,将引脚所对应寄存器指定位配置为1,即可实现该功能。如图 6‑4所示:

\sphinxAtStartPar
\sphinxincludegraphics{{SWM241/功能描述/mediaPORTCO003}.emf}

\sphinxAtStartPar
图 6‑4 IO输入上拉下拉

\sphinxAtStartPar
作为输出功能使用时,GPIO DIR寄存器对应位为1,此时可配置引脚状态为推挽输出或开漏输出,,通过配置OPEND寄存器实现。

\sphinxAtStartPar
作为推挽输出时,GPIO OPEND寄存器对应位为0,芯片具备拉/灌电流的能力,GPIO DATA寄存器配置值将反映到对应引脚电平。如图 6‑5所示:

\sphinxAtStartPar
\sphinxincludegraphics{{SWM241/功能描述/mediaPORTCO004}.emf}

\sphinxAtStartPar
图 6‑5 推挽输出

\sphinxAtStartPar
作为开漏输出时,GPIO OPEND寄存器对应位为1,芯片只具备灌电流的能力,不具备拉电流能力。GPIO输出配置为0时,对应引脚将输出0,配置为1时,输出高阻。若需要输出1时,需要将外部引脚接上拉电阻,通过外部上拉实现高电平输出。示意图如图 6‑6所示:

\sphinxAtStartPar
\sphinxincludegraphics{{SWM241/功能描述/mediaPORTCO005}.emf}

\sphinxAtStartPar
图 6‑6 开漏输出


\subsection{寄存器映射}
\label{\detokenize{SWM241/_u529f_u80fd_u63cf_u8ff0/PORTCON:id6}}

\begin{savenotes}\sphinxattablestart
\sphinxthistablewithglobalstyle
\centering
\begin{tabular}[t]{\X{20}{100}\X{20}{100}\X{20}{100}\X{20}{100}\X{20}{100}}
\sphinxtoprule
\sphinxtableatstartofbodyhook
\sphinxAtStartPar
名称   |
&
\begin{DUlineblock}{0em}
\item[] 偏移 |
\end{DUlineblock}
&
\begin{DUlineblock}{0em}
\item[] 
\item[] |
|
\end{DUlineblock}
&
\begin{DUlineblock}{0em}
\item[] 
\end{DUlineblock}
\begin{quote}

\begin{DUlineblock}{0em}
\item[] 
\item[] 
\end{DUlineblock}
\end{quote}
&
\sphinxAtStartPar
描述                       | | | |
\\
\sphinxhline
\sphinxAtStartPar
POTRGBASE:0 {\color{red}\bfseries{}|}x400A0000
&
\begin{DUlineblock}{0em}
\item[] 
\end{DUlineblock}
&&&\\
\sphinxhline
\sphinxAtStartPar
PORTA\_FUNC0
&
\sphinxAtStartPar
0x00
&&
\sphinxAtStartPar
0x 00000
&
\sphinxAtStartPar
端口A功能配置寄存器0       |
\\
\sphinxhline
\sphinxAtStartPar
PORTA\_FUNC1
&
\sphinxAtStartPar
0x04
&&
\sphinxAtStartPar
0x 00000
&
\sphinxAtStartPar
端口A功能配置寄存器1       |
\\
\sphinxhline
\sphinxAtStartPar
PORTB\_FUNC0
&
\sphinxAtStartPar
0x10
&&
\sphinxAtStartPar
0x 00000
&
\sphinxAtStartPar
端口B功能配置寄存器0       |
\\
\sphinxhline
\sphinxAtStartPar
PORTB\_FUNC1
&
\sphinxAtStartPar
0x14
&&
\sphinxAtStartPar
0x 00000
&
\sphinxAtStartPar
端口B功能配置寄存器1       |
\\
\sphinxhline
\sphinxAtStartPar
PORTC\_FUNC0
&
\sphinxAtStartPar
0x20
&&
\sphinxAtStartPar
0x 00000
&
\sphinxAtStartPar
端口C功能配置寄存器0       |
\\
\sphinxhline
\sphinxAtStartPar
PORTD\_FUNC0
&
\sphinxAtStartPar
0x30
&&
\sphinxAtStartPar
0x 00000
&
\sphinxAtStartPar
端口D功能配置寄存器0       |
\\
\sphinxhline
\sphinxAtStartPar
PORTD\_FUNC1
&
\sphinxAtStartPar
0x34
&&
\sphinxAtStartPar
0x 00000
&
\sphinxAtStartPar
端口D功能配置寄存器1       |
\\
\sphinxhline
\sphinxAtStartPar
PORTnBASE:0 {\color{red}\bfseries{}|}x400A0100
&
\begin{DUlineblock}{0em}
\item[] 
\end{DUlineblock}
&&&\\
\sphinxhline
\sphinxAtStartPar
PULLU\_A
&
\sphinxAtStartPar
0x00
&&
\sphinxAtStartPar
0x 00000
&
\sphinxAtStartPar
端口A上拉使能控制寄存器    |
\\
\sphinxhline
\sphinxAtStartPar
PULLU\_B
&
\sphinxAtStartPar
0x10
&&
\sphinxAtStartPar
0x 00000
&
\sphinxAtStartPar
端口B上拉使能控制寄存器    |
\\
\sphinxhline
\sphinxAtStartPar
PULLU\_C
&
\sphinxAtStartPar
0x20
&&
\sphinxAtStartPar
0x 00000
&
\sphinxAtStartPar
端口C上拉使能控制寄存器    |
\\
\sphinxhline
\sphinxAtStartPar
PULLU\_D
&
\sphinxAtStartPar
0x30
&&
\sphinxAtStartPar
0x 00030
&
\sphinxAtStartPar
端口D上拉使能控制寄存器    |
\\
\sphinxhline
\sphinxAtStartPar
PULLD\_A
&
\sphinxAtStartPar
0x100
&&
\sphinxAtStartPar
0x 00020
&
\sphinxAtStartPar
端口A下拉使能控制寄存器    |
\\
\sphinxhline
\sphinxAtStartPar
PULLD\_B
&
\sphinxAtStartPar
0x110
&&
\sphinxAtStartPar
0x 00000
&
\sphinxAtStartPar
端口B下拉使能控制寄存器    |
\\
\sphinxhline
\sphinxAtStartPar
PULLD\_C
&
\sphinxAtStartPar
0x120
&&
\sphinxAtStartPar
0x 00000
&
\sphinxAtStartPar
端口C下拉使能控制寄存器    |
\\
\sphinxhline
\sphinxAtStartPar
PULLD\_D
&
\sphinxAtStartPar
0x130
&&
\sphinxAtStartPar
0x 00001
&
\sphinxAtStartPar
端口D下拉使能控制寄存器    |
\\
\sphinxhline
\sphinxAtStartPar
INEN\_A
&
\sphinxAtStartPar
0x200
&&
\sphinxAtStartPar
0x 00C20
&
\sphinxAtStartPar
端口A输入使能控制寄存器    |
\\
\sphinxhline
\sphinxAtStartPar
INEN\_B
&
\sphinxAtStartPar
0x210
&&
\sphinxAtStartPar
0x 00000
&
\sphinxAtStartPar
端口B输入使能控制寄存器    |
\\
\sphinxhline
\sphinxAtStartPar
INEN\_C
&
\sphinxAtStartPar
0x220
&&
\sphinxAtStartPar
0x 00000
&
\sphinxAtStartPar
端口C输入使能控制寄存器    |
\\
\sphinxhline
\sphinxAtStartPar
INEN\_D
&
\sphinxAtStartPar
0x230
&&
\sphinxAtStartPar
0x 00031
&
\sphinxAtStartPar
端口D输入使能控制寄存器    |
\\
\sphinxhline
\sphinxAtStartPar
OPEND\_A
&
\sphinxAtStartPar
0x300
&&
\sphinxAtStartPar
0x 00000
&
\sphinxAtStartPar
端口A开漏使能控制寄存器    |
\\
\sphinxhline
\sphinxAtStartPar
OPEND\_B
&
\sphinxAtStartPar
0x310
&&
\sphinxAtStartPar
0x 00000
&
\sphinxAtStartPar
端口B开漏使能控制寄存器    |
\\
\sphinxhline
\sphinxAtStartPar
OPEND\_C
&
\sphinxAtStartPar
0x320
&&
\sphinxAtStartPar
0x 00000
&
\sphinxAtStartPar
端口C开漏使能控制寄存器    |
\\
\sphinxhline
\sphinxAtStartPar
OPEND\_D
&
\sphinxAtStartPar
0x330
&&
\sphinxAtStartPar
0x 00000
&
\sphinxAtStartPar
端口D开漏使能控制寄存器    |
\\
\sphinxbottomrule
\end{tabular}
\sphinxtableafterendhook\par
\sphinxattableend\end{savenotes}


\subsection{寄存器描述}
\label{\detokenize{SWM241/_u529f_u80fd_u63cf_u8ff0/PORTCON:id11}}

\subsubsection{PORTA\_FUNC0}
\label{\detokenize{SWM241/_u529f_u80fd_u63cf_u8ff0/PORTCON:porta-func0}}

\begin{savenotes}\sphinxattablestart
\sphinxthistablewithglobalstyle
\centering
\begin{tabular}[t]{\X{20}{100}\X{20}{100}\X{20}{100}\X{20}{100}\X{20}{100}}
\sphinxtoprule
\sphinxtableatstartofbodyhook
\sphinxAtStartPar
寄存器 |
&
\begin{DUlineblock}{0em}
\item[] 偏移 |
\end{DUlineblock}
&
\begin{DUlineblock}{0em}
\item[] 
\item[] {\color{red}\bfseries{}|}
\end{DUlineblock}
&
\sphinxAtStartPar
复位值 |    描 | |
&
\begin{DUlineblock}{0em}
\item[] |
  |
\end{DUlineblock}
\\
\sphinxhline
\sphinxAtStartPar
PORTA\_FUNC0
&
\sphinxAtStartPar
0x00
&&
\sphinxAtStartPar
0 000000
&
\sphinxAtStartPar
端口A功能配置寄存器0       |
\\
\sphinxbottomrule
\end{tabular}
\sphinxtableafterendhook\par
\sphinxattableend\end{savenotes}


\begin{savenotes}\sphinxattablestart
\sphinxthistablewithglobalstyle
\centering
\begin{tabular}[t]{\X{12}{96}\X{12}{96}\X{12}{96}\X{12}{96}\X{12}{96}\X{12}{96}\X{12}{96}\X{12}{96}}
\sphinxtoprule
\sphinxtableatstartofbodyhook
\sphinxAtStartPar
31
&
\sphinxAtStartPar
30
&
\sphinxAtStartPar
29
&
\sphinxAtStartPar
28
&
\sphinxAtStartPar
27
&
\sphinxAtStartPar
26
&
\sphinxAtStartPar
25
&
\sphinxAtStartPar
24
\\
\sphinxhline
\sphinxAtStartPar
PIN7
&&&&&&&\\
\sphinxhline
\sphinxAtStartPar
23
&
\sphinxAtStartPar
22
&
\sphinxAtStartPar
21
&
\sphinxAtStartPar
20
&
\sphinxAtStartPar
19
&
\sphinxAtStartPar
18
&
\sphinxAtStartPar
17
&
\sphinxAtStartPar
16
\\
\sphinxhline
\sphinxAtStartPar
PIN5
&&&&&&&\\
\sphinxhline
\sphinxAtStartPar
15
&
\sphinxAtStartPar
14
&
\sphinxAtStartPar
13
&
\sphinxAtStartPar
12
&
\sphinxAtStartPar
11
&
\sphinxAtStartPar
10
&
\sphinxAtStartPar
9
&
\sphinxAtStartPar
8
\\
\sphinxhline
\sphinxAtStartPar
PIN3
&&&&&&&\\
\sphinxhline
\sphinxAtStartPar
7
&
\sphinxAtStartPar
6
&
\sphinxAtStartPar
5
&
\sphinxAtStartPar
4
&
\sphinxAtStartPar
3
&
\sphinxAtStartPar
2
&
\sphinxAtStartPar
1
&
\sphinxAtStartPar
0
\\
\sphinxhline
\sphinxAtStartPar
PIN1
&&&&&&&\\
\sphinxbottomrule
\end{tabular}
\sphinxtableafterendhook\par
\sphinxattableend\end{savenotes}

\sphinxAtStartPar
注:具体复用功能,见表格 5‑1 PORTA复用功能。


\subsubsection{PORTA\_FUNC1}
\label{\detokenize{SWM241/_u529f_u80fd_u63cf_u8ff0/PORTCON:porta-func1}}

\begin{savenotes}\sphinxattablestart
\sphinxthistablewithglobalstyle
\centering
\begin{tabular}[t]{\X{20}{100}\X{20}{100}\X{20}{100}\X{20}{100}\X{20}{100}}
\sphinxtoprule
\sphinxtableatstartofbodyhook
\sphinxAtStartPar
寄存器 |
&
\begin{DUlineblock}{0em}
\item[] 偏移 |
\end{DUlineblock}
&
\begin{DUlineblock}{0em}
\item[] 
\item[] {\color{red}\bfseries{}|}
\end{DUlineblock}
&
\sphinxAtStartPar
复位值 |    描 | |
&
\begin{DUlineblock}{0em}
\item[] |
  |
\end{DUlineblock}
\\
\sphinxhline
\sphinxAtStartPar
PORTA\_FUNC1
&
\sphinxAtStartPar
0x04
&&
\sphinxAtStartPar
0 000000
&
\sphinxAtStartPar
端口A功能配置寄存器1       |
\\
\sphinxbottomrule
\end{tabular}
\sphinxtableafterendhook\par
\sphinxattableend\end{savenotes}


\begin{savenotes}\sphinxattablestart
\sphinxthistablewithglobalstyle
\centering
\begin{tabular}[t]{\X{12}{96}\X{12}{96}\X{12}{96}\X{12}{96}\X{12}{96}\X{12}{96}\X{12}{96}\X{12}{96}}
\sphinxtoprule
\sphinxtableatstartofbodyhook
\sphinxAtStartPar
31
&
\sphinxAtStartPar
30
&
\sphinxAtStartPar
29
&
\sphinxAtStartPar
28
&
\sphinxAtStartPar
27
&
\sphinxAtStartPar
26
&
\sphinxAtStartPar
25
&
\sphinxAtStartPar
24
\\
\sphinxhline
\sphinxAtStartPar
PIN15
&&&&&&&\\
\sphinxhline
\sphinxAtStartPar
23
&
\sphinxAtStartPar
22
&
\sphinxAtStartPar
21
&
\sphinxAtStartPar
20
&
\sphinxAtStartPar
19
&
\sphinxAtStartPar
18
&
\sphinxAtStartPar
17
&
\sphinxAtStartPar
16
\\
\sphinxhline
\sphinxAtStartPar
PIN13
&&&&&&&\\
\sphinxhline
\sphinxAtStartPar
15
&
\sphinxAtStartPar
14
&
\sphinxAtStartPar
13
&
\sphinxAtStartPar
12
&
\sphinxAtStartPar
11
&
\sphinxAtStartPar
10
&
\sphinxAtStartPar
9
&
\sphinxAtStartPar
8
\\
\sphinxhline
\sphinxAtStartPar
PIN11
&&&&&&&\\
\sphinxhline
\sphinxAtStartPar
7
&
\sphinxAtStartPar
6
&
\sphinxAtStartPar
5
&
\sphinxAtStartPar
4
&
\sphinxAtStartPar
3
&
\sphinxAtStartPar
2
&
\sphinxAtStartPar
1
&
\sphinxAtStartPar
0
\\
\sphinxhline
\sphinxAtStartPar
PIN9
&&&&&&&\\
\sphinxbottomrule
\end{tabular}
\sphinxtableafterendhook\par
\sphinxattableend\end{savenotes}

\sphinxAtStartPar
注:具体复用功能,见表格 5‑1 PORTA复用功能表。


\subsubsection{PORTB\_FUNC0}
\label{\detokenize{SWM241/_u529f_u80fd_u63cf_u8ff0/PORTCON:portb-func0}}

\begin{savenotes}\sphinxattablestart
\sphinxthistablewithglobalstyle
\centering
\begin{tabular}[t]{\X{20}{100}\X{20}{100}\X{20}{100}\X{20}{100}\X{20}{100}}
\sphinxtoprule
\sphinxtableatstartofbodyhook
\sphinxAtStartPar
寄存器 |
&
\begin{DUlineblock}{0em}
\item[] 偏移 |
\end{DUlineblock}
&
\begin{DUlineblock}{0em}
\item[] 
\item[] {\color{red}\bfseries{}|}
\end{DUlineblock}
&
\sphinxAtStartPar
复位值 |    描 | |
&
\begin{DUlineblock}{0em}
\item[] |
  |
\end{DUlineblock}
\\
\sphinxhline
\sphinxAtStartPar
PORTB\_FUNC0
&
\sphinxAtStartPar
0x10
&&
\sphinxAtStartPar
0 000000
&
\sphinxAtStartPar
端口B功能配置寄存器0       |
\\
\sphinxbottomrule
\end{tabular}
\sphinxtableafterendhook\par
\sphinxattableend\end{savenotes}


\begin{savenotes}\sphinxattablestart
\sphinxthistablewithglobalstyle
\centering
\begin{tabular}[t]{\X{12}{96}\X{12}{96}\X{12}{96}\X{12}{96}\X{12}{96}\X{12}{96}\X{12}{96}\X{12}{96}}
\sphinxtoprule
\sphinxtableatstartofbodyhook
\sphinxAtStartPar
31
&
\sphinxAtStartPar
30
&
\sphinxAtStartPar
29
&
\sphinxAtStartPar
28
&
\sphinxAtStartPar
27
&
\sphinxAtStartPar
26
&
\sphinxAtStartPar
25
&
\sphinxAtStartPar
24
\\
\sphinxhline
\sphinxAtStartPar
PIN7
&&&&&&&\\
\sphinxhline
\sphinxAtStartPar
23
&
\sphinxAtStartPar
22
&
\sphinxAtStartPar
21
&
\sphinxAtStartPar
20
&
\sphinxAtStartPar
19
&
\sphinxAtStartPar
18
&
\sphinxAtStartPar
17
&
\sphinxAtStartPar
16
\\
\sphinxhline
\sphinxAtStartPar
PIN5
&&&&&&&\\
\sphinxhline
\sphinxAtStartPar
15
&
\sphinxAtStartPar
14
&
\sphinxAtStartPar
13
&
\sphinxAtStartPar
12
&
\sphinxAtStartPar
11
&
\sphinxAtStartPar
10
&
\sphinxAtStartPar
9
&
\sphinxAtStartPar
8
\\
\sphinxhline
\sphinxAtStartPar
PIN3
&&&&&&&\\
\sphinxhline
\sphinxAtStartPar
7
&
\sphinxAtStartPar
6
&
\sphinxAtStartPar
5
&
\sphinxAtStartPar
4
&
\sphinxAtStartPar
3
&
\sphinxAtStartPar
2
&
\sphinxAtStartPar
1
&
\sphinxAtStartPar
0
\\
\sphinxhline
\sphinxAtStartPar
PIN1
&&&&&&&\\
\sphinxbottomrule
\end{tabular}
\sphinxtableafterendhook\par
\sphinxattableend\end{savenotes}

\sphinxAtStartPar
注:具体复用功能,见表格 5‑2 PORTB复用功能。


\subsubsection{PORTB\_FUNC1}
\label{\detokenize{SWM241/_u529f_u80fd_u63cf_u8ff0/PORTCON:portb-func1}}

\begin{savenotes}\sphinxattablestart
\sphinxthistablewithglobalstyle
\centering
\begin{tabular}[t]{\X{20}{100}\X{20}{100}\X{20}{100}\X{20}{100}\X{20}{100}}
\sphinxtoprule
\sphinxtableatstartofbodyhook
\sphinxAtStartPar
寄存器 |
&
\begin{DUlineblock}{0em}
\item[] 偏移 |
\end{DUlineblock}
&
\begin{DUlineblock}{0em}
\item[] 
\item[] {\color{red}\bfseries{}|}
\end{DUlineblock}
&
\sphinxAtStartPar
复位值 |    描 | |
&
\begin{DUlineblock}{0em}
\item[] |
  |
\end{DUlineblock}
\\
\sphinxhline
\sphinxAtStartPar
PORTB\_FUNC1
&
\sphinxAtStartPar
0x14
&&
\sphinxAtStartPar
0 000000
&
\sphinxAtStartPar
端口B功能配置寄存器1       |
\\
\sphinxbottomrule
\end{tabular}
\sphinxtableafterendhook\par
\sphinxattableend\end{savenotes}


\begin{savenotes}\sphinxattablestart
\sphinxthistablewithglobalstyle
\centering
\begin{tabular}[t]{\X{12}{96}\X{12}{96}\X{12}{96}\X{12}{96}\X{12}{96}\X{12}{96}\X{12}{96}\X{12}{96}}
\sphinxtoprule
\sphinxtableatstartofbodyhook
\sphinxAtStartPar
31
&
\sphinxAtStartPar
30
&
\sphinxAtStartPar
29
&
\sphinxAtStartPar
28
&
\sphinxAtStartPar
27
&
\sphinxAtStartPar
26
&
\sphinxAtStartPar
25
&
\sphinxAtStartPar
24
\\
\sphinxhline
\sphinxAtStartPar
PIN15
&&&&&&&\\
\sphinxhline
\sphinxAtStartPar
23
&
\sphinxAtStartPar
22
&
\sphinxAtStartPar
21
&
\sphinxAtStartPar
20
&
\sphinxAtStartPar
19
&
\sphinxAtStartPar
18
&
\sphinxAtStartPar
17
&
\sphinxAtStartPar
16
\\
\sphinxhline
\sphinxAtStartPar
PIN13
&&&&&&&\\
\sphinxhline
\sphinxAtStartPar
15
&
\sphinxAtStartPar
14
&
\sphinxAtStartPar
13
&
\sphinxAtStartPar
12
&
\sphinxAtStartPar
11
&
\sphinxAtStartPar
10
&
\sphinxAtStartPar
9
&
\sphinxAtStartPar
8
\\
\sphinxhline
\sphinxAtStartPar
PIN11
&&&&&&&\\
\sphinxhline
\sphinxAtStartPar
7
&
\sphinxAtStartPar
6
&
\sphinxAtStartPar
5
&
\sphinxAtStartPar
4
&
\sphinxAtStartPar
3
&
\sphinxAtStartPar
2
&
\sphinxAtStartPar
1
&
\sphinxAtStartPar
0
\\
\sphinxhline
\sphinxAtStartPar
PIN9
&&&&&&&\\
\sphinxbottomrule
\end{tabular}
\sphinxtableafterendhook\par
\sphinxattableend\end{savenotes}

\sphinxAtStartPar
注:具体复用功能,见表格 5‑2 PORTB复用功能。


\subsubsection{PORTC\_FUNC0}
\label{\detokenize{SWM241/_u529f_u80fd_u63cf_u8ff0/PORTCON:portc-func0}}

\begin{savenotes}\sphinxattablestart
\sphinxthistablewithglobalstyle
\centering
\begin{tabular}[t]{\X{20}{100}\X{20}{100}\X{20}{100}\X{20}{100}\X{20}{100}}
\sphinxtoprule
\sphinxtableatstartofbodyhook
\sphinxAtStartPar
寄存器 |
&
\begin{DUlineblock}{0em}
\item[] 偏移 |
\end{DUlineblock}
&
\begin{DUlineblock}{0em}
\item[] 
\item[] {\color{red}\bfseries{}|}
\end{DUlineblock}
&
\sphinxAtStartPar
复位值 |    描 | |
&
\begin{DUlineblock}{0em}
\item[] |
  |
\end{DUlineblock}
\\
\sphinxhline
\sphinxAtStartPar
PORTC\_FUNC0
&
\sphinxAtStartPar
0x20
&&
\sphinxAtStartPar
0 000000
&
\sphinxAtStartPar
端口C功能配置寄存器0       |
\\
\sphinxbottomrule
\end{tabular}
\sphinxtableafterendhook\par
\sphinxattableend\end{savenotes}


\begin{savenotes}\sphinxattablestart
\sphinxthistablewithglobalstyle
\centering
\begin{tabular}[t]{\X{12}{96}\X{12}{96}\X{12}{96}\X{12}{96}\X{12}{96}\X{12}{96}\X{12}{96}\X{12}{96}}
\sphinxtoprule
\sphinxtableatstartofbodyhook
\sphinxAtStartPar
31
&
\sphinxAtStartPar
30
&
\sphinxAtStartPar
29
&
\sphinxAtStartPar
28
&
\sphinxAtStartPar
27
&
\sphinxAtStartPar
26
&
\sphinxAtStartPar
25
&
\sphinxAtStartPar
24
\\
\sphinxhline
\sphinxAtStartPar
PIN7
&&&&&&&\\
\sphinxhline
\sphinxAtStartPar
23
&
\sphinxAtStartPar
22
&
\sphinxAtStartPar
21
&
\sphinxAtStartPar
20
&
\sphinxAtStartPar
19
&
\sphinxAtStartPar
18
&
\sphinxAtStartPar
17
&
\sphinxAtStartPar
16
\\
\sphinxhline
\sphinxAtStartPar
PIN5
&&&&&&&\\
\sphinxhline
\sphinxAtStartPar
15
&
\sphinxAtStartPar
14
&
\sphinxAtStartPar
13
&
\sphinxAtStartPar
12
&
\sphinxAtStartPar
11
&
\sphinxAtStartPar
10
&
\sphinxAtStartPar
9
&
\sphinxAtStartPar
8
\\
\sphinxhline
\sphinxAtStartPar
PIN3
&&&&&&&\\
\sphinxhline
\sphinxAtStartPar
7
&
\sphinxAtStartPar
6
&
\sphinxAtStartPar
5
&
\sphinxAtStartPar
4
&
\sphinxAtStartPar
3
&
\sphinxAtStartPar
2
&
\sphinxAtStartPar
1
&
\sphinxAtStartPar
0
\\
\sphinxhline
\sphinxAtStartPar
PIN1
&&&&&&&\\
\sphinxbottomrule
\end{tabular}
\sphinxtableafterendhook\par
\sphinxattableend\end{savenotes}

\sphinxAtStartPar
注:具体复用功能,见表格 5‑3 PORTC复用功能。


\subsubsection{PORTC\_FUNC1}
\label{\detokenize{SWM241/_u529f_u80fd_u63cf_u8ff0/PORTCON:portc-func1}}

\begin{savenotes}\sphinxattablestart
\sphinxthistablewithglobalstyle
\centering
\begin{tabular}[t]{\X{20}{100}\X{20}{100}\X{20}{100}\X{20}{100}\X{20}{100}}
\sphinxtoprule
\sphinxtableatstartofbodyhook
\sphinxAtStartPar
寄存器 |
&
\begin{DUlineblock}{0em}
\item[] 偏移 |
\end{DUlineblock}
&
\begin{DUlineblock}{0em}
\item[] 
\item[] {\color{red}\bfseries{}|}
\end{DUlineblock}
&
\sphinxAtStartPar
复位值 |    描 | |
&
\begin{DUlineblock}{0em}
\item[] |
  |
\end{DUlineblock}
\\
\sphinxhline
\sphinxAtStartPar
PORTC\_FUNC1
&
\sphinxAtStartPar
0x20
&&
\sphinxAtStartPar
0 000000
&
\sphinxAtStartPar
端口C功能配置寄存器1       |
\\
\sphinxbottomrule
\end{tabular}
\sphinxtableafterendhook\par
\sphinxattableend\end{savenotes}


\begin{savenotes}\sphinxattablestart
\sphinxthistablewithglobalstyle
\centering
\begin{tabular}[t]{\X{12}{96}\X{12}{96}\X{12}{96}\X{12}{96}\X{12}{96}\X{12}{96}\X{12}{96}\X{12}{96}}
\sphinxtoprule
\sphinxtableatstartofbodyhook
\sphinxAtStartPar
31
&
\sphinxAtStartPar
30
&
\sphinxAtStartPar
29
&
\sphinxAtStartPar
28
&
\sphinxAtStartPar
27
&
\sphinxAtStartPar
26
&
\sphinxAtStartPar
25
&
\sphinxAtStartPar
24
\\
\sphinxhline
\sphinxAtStartPar
PIN15
&&&&&&&\\
\sphinxhline
\sphinxAtStartPar
23
&
\sphinxAtStartPar
22
&
\sphinxAtStartPar
21
&
\sphinxAtStartPar
20
&
\sphinxAtStartPar
19
&
\sphinxAtStartPar
18
&
\sphinxAtStartPar
17
&
\sphinxAtStartPar
16
\\
\sphinxhline
\sphinxAtStartPar
PIN13
&&&&&&&\\
\sphinxhline
\sphinxAtStartPar
15
&
\sphinxAtStartPar
14
&
\sphinxAtStartPar
13
&
\sphinxAtStartPar
12
&
\sphinxAtStartPar
11
&
\sphinxAtStartPar
10
&
\sphinxAtStartPar
9
&
\sphinxAtStartPar
8
\\
\sphinxhline
\sphinxAtStartPar
PIN11
&&&&&&&\\
\sphinxhline
\sphinxAtStartPar
7
&
\sphinxAtStartPar
6
&
\sphinxAtStartPar
5
&
\sphinxAtStartPar
4
&
\sphinxAtStartPar
3
&
\sphinxAtStartPar
2
&
\sphinxAtStartPar
1
&
\sphinxAtStartPar
0
\\
\sphinxhline
\sphinxAtStartPar
PIN9
&&&&&&&\\
\sphinxbottomrule
\end{tabular}
\sphinxtableafterendhook\par
\sphinxattableend\end{savenotes}

\sphinxAtStartPar
注:具体复用功能,见表格 5‑3 PORTC复用功能。


\subsubsection{PORTD\_FUNC0}
\label{\detokenize{SWM241/_u529f_u80fd_u63cf_u8ff0/PORTCON:portd-func0}}

\begin{savenotes}\sphinxattablestart
\sphinxthistablewithglobalstyle
\centering
\begin{tabular}[t]{\X{20}{100}\X{20}{100}\X{20}{100}\X{20}{100}\X{20}{100}}
\sphinxtoprule
\sphinxtableatstartofbodyhook
\sphinxAtStartPar
寄存器 |
&
\begin{DUlineblock}{0em}
\item[] 偏移 |
\end{DUlineblock}
&
\begin{DUlineblock}{0em}
\item[] 
\item[] {\color{red}\bfseries{}|}
\end{DUlineblock}
&
\sphinxAtStartPar
复位值 |    描 | |
&
\begin{DUlineblock}{0em}
\item[] |
  |
\end{DUlineblock}
\\
\sphinxhline
\sphinxAtStartPar
PORTD\_FUNC0
&
\sphinxAtStartPar
0x30
&&
\sphinxAtStartPar
0 000000
&
\sphinxAtStartPar
端口D功能配置寄存器        |
\\
\sphinxbottomrule
\end{tabular}
\sphinxtableafterendhook\par
\sphinxattableend\end{savenotes}


\begin{savenotes}\sphinxattablestart
\sphinxthistablewithglobalstyle
\centering
\begin{tabular}[t]{\X{12}{96}\X{12}{96}\X{12}{96}\X{12}{96}\X{12}{96}\X{12}{96}\X{12}{96}\X{12}{96}}
\sphinxtoprule
\sphinxtableatstartofbodyhook
\sphinxAtStartPar
31
&
\sphinxAtStartPar
30
&
\sphinxAtStartPar
29
&
\sphinxAtStartPar
28
&
\sphinxAtStartPar
27
&
\sphinxAtStartPar
26
&
\sphinxAtStartPar
25
&
\sphinxAtStartPar
24
\\
\sphinxhline
\sphinxAtStartPar
PIN7
&&&&&&&\\
\sphinxhline
\sphinxAtStartPar
23
&
\sphinxAtStartPar
22
&
\sphinxAtStartPar
21
&
\sphinxAtStartPar
20
&
\sphinxAtStartPar
19
&
\sphinxAtStartPar
18
&
\sphinxAtStartPar
17
&
\sphinxAtStartPar
16
\\
\sphinxhline
\sphinxAtStartPar
PIN5
&&&&&&&\\
\sphinxhline
\sphinxAtStartPar
15
&
\sphinxAtStartPar
14
&
\sphinxAtStartPar
13
&
\sphinxAtStartPar
12
&
\sphinxAtStartPar
11
&
\sphinxAtStartPar
10
&
\sphinxAtStartPar
9
&
\sphinxAtStartPar
8
\\
\sphinxhline
\sphinxAtStartPar
PIN3
&&&&&&&\\
\sphinxhline
\sphinxAtStartPar
7
&
\sphinxAtStartPar
6
&
\sphinxAtStartPar
5
&
\sphinxAtStartPar
4
&
\sphinxAtStartPar
3
&
\sphinxAtStartPar
2
&
\sphinxAtStartPar
1
&
\sphinxAtStartPar
0
\\
\sphinxhline
\sphinxAtStartPar
PIN1
&&&&&&&\\
\sphinxbottomrule
\end{tabular}
\sphinxtableafterendhook\par
\sphinxattableend\end{savenotes}

\sphinxAtStartPar
注:具体复用功能,见表格 5‑4 PORTD复用功能。


\subsubsection{PORTD\_FUNC1}
\label{\detokenize{SWM241/_u529f_u80fd_u63cf_u8ff0/PORTCON:portd-func1}}

\begin{savenotes}\sphinxattablestart
\sphinxthistablewithglobalstyle
\centering
\begin{tabular}[t]{\X{20}{100}\X{20}{100}\X{20}{100}\X{20}{100}\X{20}{100}}
\sphinxtoprule
\sphinxtableatstartofbodyhook
\sphinxAtStartPar
寄存器 |
&
\begin{DUlineblock}{0em}
\item[] 偏移 |
\end{DUlineblock}
&
\begin{DUlineblock}{0em}
\item[] 
\item[] {\color{red}\bfseries{}|}
\end{DUlineblock}
&
\sphinxAtStartPar
复位值 |    描 | |
&
\begin{DUlineblock}{0em}
\item[] |
  |
\end{DUlineblock}
\\
\sphinxhline
\sphinxAtStartPar
PORTD\_FUNC1
&
\sphinxAtStartPar
0x34
&&
\sphinxAtStartPar
0 000000
&
\sphinxAtStartPar
端口D功能配置寄存器1       |
\\
\sphinxbottomrule
\end{tabular}
\sphinxtableafterendhook\par
\sphinxattableend\end{savenotes}


\begin{savenotes}\sphinxattablestart
\sphinxthistablewithglobalstyle
\centering
\begin{tabular}[t]{\X{12}{96}\X{12}{96}\X{12}{96}\X{12}{96}\X{12}{96}\X{12}{96}\X{12}{96}\X{12}{96}}
\sphinxtoprule
\sphinxtableatstartofbodyhook
\sphinxAtStartPar
31
&
\sphinxAtStartPar
30
&
\sphinxAtStartPar
29
&
\sphinxAtStartPar
28
&
\sphinxAtStartPar
27
&
\sphinxAtStartPar
26
&
\sphinxAtStartPar
25
&
\sphinxAtStartPar
24
\\
\sphinxhline
\sphinxAtStartPar
PIN15
&&&&&&&\\
\sphinxhline
\sphinxAtStartPar
23
&
\sphinxAtStartPar
22
&
\sphinxAtStartPar
21
&
\sphinxAtStartPar
20
&
\sphinxAtStartPar
19
&
\sphinxAtStartPar
18
&
\sphinxAtStartPar
17
&
\sphinxAtStartPar
16
\\
\sphinxhline
\sphinxAtStartPar
PIN13
&&&&&&&\\
\sphinxhline
\sphinxAtStartPar
15
&
\sphinxAtStartPar
14
&
\sphinxAtStartPar
13
&
\sphinxAtStartPar
12
&
\sphinxAtStartPar
11
&
\sphinxAtStartPar
10
&
\sphinxAtStartPar
9
&
\sphinxAtStartPar
8
\\
\sphinxhline
\sphinxAtStartPar
PIN11
&&&&&&&\\
\sphinxhline
\sphinxAtStartPar
7
&
\sphinxAtStartPar
6
&
\sphinxAtStartPar
5
&
\sphinxAtStartPar
4
&
\sphinxAtStartPar
3
&
\sphinxAtStartPar
2
&
\sphinxAtStartPar
1
&
\sphinxAtStartPar
0
\\
\sphinxhline
\sphinxAtStartPar
PIN9
&&&&&&&\\
\sphinxbottomrule
\end{tabular}
\sphinxtableafterendhook\par
\sphinxattableend\end{savenotes}

\sphinxAtStartPar
注:具体复用功能,见表格 5‑4 PORTD复用功能。


\subsubsection{PORTA端口上拉功能寄存器PULLU\_A}
\label{\detokenize{SWM241/_u529f_u80fd_u63cf_u8ff0/PORTCON:portapullu-a}}

\begin{savenotes}\sphinxattablestart
\sphinxthistablewithglobalstyle
\centering
\begin{tabular}[t]{\X{20}{100}\X{20}{100}\X{20}{100}\X{20}{100}\X{20}{100}}
\sphinxtoprule
\sphinxtableatstartofbodyhook
\sphinxAtStartPar
寄存器 |
&
\begin{DUlineblock}{0em}
\item[] 偏移 |
\end{DUlineblock}
&
\begin{DUlineblock}{0em}
\item[] 
\item[] {\color{red}\bfseries{}|}
\end{DUlineblock}
&
\sphinxAtStartPar
复位值 |    描 | |
&
\begin{DUlineblock}{0em}
\item[] |
  |
\end{DUlineblock}
\\
\sphinxhline
\sphinxAtStartPar
PULLU\_A
&
\sphinxAtStartPar
0x00
&&
\sphinxAtStartPar
0 000000
&
\sphinxAtStartPar
端口A上拉使能控制寄存器    |
\\
\sphinxbottomrule
\end{tabular}
\sphinxtableafterendhook\par
\sphinxattableend\end{savenotes}


\begin{savenotes}\sphinxattablestart
\sphinxthistablewithglobalstyle
\centering
\begin{tabular}[t]{\X{12}{96}\X{12}{96}\X{12}{96}\X{12}{96}\X{12}{96}\X{12}{96}\X{12}{96}\X{12}{96}}
\sphinxtoprule
\sphinxtableatstartofbodyhook
\sphinxAtStartPar
31
&
\sphinxAtStartPar
30
&
\sphinxAtStartPar
29
&
\sphinxAtStartPar
28
&
\sphinxAtStartPar
27
&
\sphinxAtStartPar
26
&
\sphinxAtStartPar
25
&
\sphinxAtStartPar
24
\\
\sphinxhline\begin{itemize}
\item {} 
\end{itemize}
&&&&&&&\\
\sphinxhline
\sphinxAtStartPar
23
&
\sphinxAtStartPar
22
&
\sphinxAtStartPar
21
&
\sphinxAtStartPar
20
&
\sphinxAtStartPar
19
&
\sphinxAtStartPar
18
&
\sphinxAtStartPar
17
&
\sphinxAtStartPar
16
\\
\sphinxhline\begin{itemize}
\item {} 
\end{itemize}
&&&&&&&\\
\sphinxhline
\sphinxAtStartPar
15
&
\sphinxAtStartPar
14
&
\sphinxAtStartPar
13
&
\sphinxAtStartPar
12
&
\sphinxAtStartPar
11
&
\sphinxAtStartPar
10
&
\sphinxAtStartPar
9
&
\sphinxAtStartPar
8
\\
\sphinxhline
\sphinxAtStartPar
PIN15
&
\sphinxAtStartPar
PIN14
&&&&&&\\
\sphinxhline
\sphinxAtStartPar
7
&
\sphinxAtStartPar
6
&
\sphinxAtStartPar
5
&
\sphinxAtStartPar
4
&
\sphinxAtStartPar
3
&
\sphinxAtStartPar
2
&
\sphinxAtStartPar
1
&
\sphinxAtStartPar
0
\\
\sphinxhline
\sphinxAtStartPar
PIN7
&
\sphinxAtStartPar
PIN6
&&&&&&\\
\sphinxbottomrule
\end{tabular}
\sphinxtableafterendhook\par
\sphinxattableend\end{savenotes}


\begin{savenotes}\sphinxattablestart
\sphinxthistablewithglobalstyle
\centering
\begin{tabular}[t]{\X{33}{99}\X{33}{99}\X{33}{99}}
\sphinxtoprule
\sphinxtableatstartofbodyhook
\sphinxAtStartPar
位域 |
&
\sphinxAtStartPar
名称     | |
&
\sphinxAtStartPar
描述                                        | |
\\
\sphinxhline
\sphinxAtStartPar
31:16
&\begin{itemize}
\item {} 
\end{itemize}
&\begin{itemize}
\item {} 
\end{itemize}
\\
\sphinxhline
\sphinxAtStartPar
15
&
\sphinxAtStartPar
PIN15
&
\sphinxAtStartPar
PIN15 上拉电阻使能                          |

\sphinxAtStartPar
0:禁能 1:使能                             |
\\
\sphinxhline
\sphinxAtStartPar
14
&
\sphinxAtStartPar
PIN14
&
\sphinxAtStartPar
PIN14 上拉电阻使能                          |

\sphinxAtStartPar
0:禁能 1:使能                             |
\\
\sphinxhline
\sphinxAtStartPar
13
&
\sphinxAtStartPar
PIN13
&
\sphinxAtStartPar
PIN13 上拉电阻使能                          |

\sphinxAtStartPar
0:禁能 1:使能                             |
\\
\sphinxhline
\sphinxAtStartPar
12
&
\sphinxAtStartPar
PIN12
&
\sphinxAtStartPar
PIN12上拉电阻使能                           |

\sphinxAtStartPar
0:禁能 1:使能                             |
\\
\sphinxhline
\sphinxAtStartPar
11
&
\sphinxAtStartPar
PIN11
&
\sphinxAtStartPar
PIN11 上拉电阻使能                          |

\sphinxAtStartPar
0:禁能 1:使能                             |
\\
\sphinxhline
\sphinxAtStartPar
10
&
\sphinxAtStartPar
PIN10
&
\sphinxAtStartPar
PIN10 上拉电阻使能                          |

\sphinxAtStartPar
0:禁能 1:使能                             |
\\
\sphinxhline
\sphinxAtStartPar
9
&
\sphinxAtStartPar
PIN9
&
\sphinxAtStartPar
PIN9 上拉电阻使能                           |

\sphinxAtStartPar
0:禁能 1:使能                             |
\\
\sphinxhline
\sphinxAtStartPar
8
&
\sphinxAtStartPar
PIN8
&
\sphinxAtStartPar
PIN8 上拉电阻使能                           |

\sphinxAtStartPar
0:禁能 1:使能                             |
\\
\sphinxhline
\sphinxAtStartPar
7
&
\sphinxAtStartPar
PIN7
&
\sphinxAtStartPar
PIN7上拉电阻使能                            |

\sphinxAtStartPar
0:禁能 1:使能                             |
\\
\sphinxhline
\sphinxAtStartPar
6
&
\sphinxAtStartPar
PIN6
&
\sphinxAtStartPar
PIN6上拉电阻使能                            |

\sphinxAtStartPar
0:禁能 1:使能                             |
\\
\sphinxhline
\sphinxAtStartPar
5
&
\sphinxAtStartPar
PIN5
&
\sphinxAtStartPar
PIN5上拉电阻使能                            |

\sphinxAtStartPar
0:禁能 1:使能                             |
\\
\sphinxhline
\sphinxAtStartPar
4
&
\sphinxAtStartPar
PIN4
&
\sphinxAtStartPar
PIN4上拉电阻使能                            |

\sphinxAtStartPar
0:禁能 1:使能                             |
\\
\sphinxhline
\sphinxAtStartPar
3
&
\sphinxAtStartPar
PIN3
&
\sphinxAtStartPar
PIN3上拉电阻使能                            |

\sphinxAtStartPar
0:禁能 1:使能                             |
\\
\sphinxhline
\sphinxAtStartPar
2
&
\sphinxAtStartPar
PIN2
&
\sphinxAtStartPar
PIN2上拉电阻使能                            |

\sphinxAtStartPar
0:禁能 1:使能                             |
\\
\sphinxhline
\sphinxAtStartPar
1
&
\sphinxAtStartPar
PIN1
&
\sphinxAtStartPar
PIN1上拉电阻使能                            |

\sphinxAtStartPar
0:禁能 1:使能                             |
\\
\sphinxhline
\sphinxAtStartPar
0
&
\sphinxAtStartPar
PIN0
&
\sphinxAtStartPar
PIN0上拉电阻使能                            |

\sphinxAtStartPar
0:禁能 1:使能                             |
\\
\sphinxbottomrule
\end{tabular}
\sphinxtableafterendhook\par
\sphinxattableend\end{savenotes}


\subsubsection{PORTB端口上拉功能寄存器PULLU\_B}
\label{\detokenize{SWM241/_u529f_u80fd_u63cf_u8ff0/PORTCON:portbpullu-b}}

\begin{savenotes}\sphinxattablestart
\sphinxthistablewithglobalstyle
\centering
\begin{tabular}[t]{\X{20}{100}\X{20}{100}\X{20}{100}\X{20}{100}\X{20}{100}}
\sphinxtoprule
\sphinxtableatstartofbodyhook
\sphinxAtStartPar
寄存器 |
&
\begin{DUlineblock}{0em}
\item[] 偏移 |
\end{DUlineblock}
&
\begin{DUlineblock}{0em}
\item[] 
\item[] {\color{red}\bfseries{}|}
\end{DUlineblock}
&
\sphinxAtStartPar
复位值 |    描 | |
&
\begin{DUlineblock}{0em}
\item[] |
  |
\end{DUlineblock}
\\
\sphinxhline
\sphinxAtStartPar
PULLU\_B
&
\sphinxAtStartPar
0x10
&&
\sphinxAtStartPar
0 000000
&
\sphinxAtStartPar
端口B上拉使能控制寄存器    |
\\
\sphinxbottomrule
\end{tabular}
\sphinxtableafterendhook\par
\sphinxattableend\end{savenotes}


\begin{savenotes}\sphinxattablestart
\sphinxthistablewithglobalstyle
\centering
\begin{tabular}[t]{\X{12}{96}\X{12}{96}\X{12}{96}\X{12}{96}\X{12}{96}\X{12}{96}\X{12}{96}\X{12}{96}}
\sphinxtoprule
\sphinxtableatstartofbodyhook
\sphinxAtStartPar
31
&
\sphinxAtStartPar
30
&
\sphinxAtStartPar
29
&
\sphinxAtStartPar
28
&
\sphinxAtStartPar
27
&
\sphinxAtStartPar
26
&
\sphinxAtStartPar
25
&
\sphinxAtStartPar
24
\\
\sphinxhline\begin{itemize}
\item {} 
\end{itemize}
&&&&&&&\\
\sphinxhline
\sphinxAtStartPar
23
&
\sphinxAtStartPar
22
&
\sphinxAtStartPar
21
&
\sphinxAtStartPar
20
&
\sphinxAtStartPar
19
&
\sphinxAtStartPar
18
&
\sphinxAtStartPar
17
&
\sphinxAtStartPar
16
\\
\sphinxhline\begin{itemize}
\item {} 
\end{itemize}
&&&&&&&\\
\sphinxhline
\sphinxAtStartPar
15
&
\sphinxAtStartPar
14
&
\sphinxAtStartPar
13
&
\sphinxAtStartPar
12
&
\sphinxAtStartPar
11
&
\sphinxAtStartPar
10
&
\sphinxAtStartPar
9
&
\sphinxAtStartPar
8
\\
\sphinxhline\begin{itemize}
\item {} 
\end{itemize}
&&&&&&&\\
\sphinxhline
\sphinxAtStartPar
7
&
\sphinxAtStartPar
6
&
\sphinxAtStartPar
5
&
\sphinxAtStartPar
4
&
\sphinxAtStartPar
3
&
\sphinxAtStartPar
2
&
\sphinxAtStartPar
1
&
\sphinxAtStartPar
0
\\
\sphinxhline
\sphinxAtStartPar
PIN7
&
\sphinxAtStartPar
PIN6
&&&&&&\\
\sphinxbottomrule
\end{tabular}
\sphinxtableafterendhook\par
\sphinxattableend\end{savenotes}


\begin{savenotes}\sphinxattablestart
\sphinxthistablewithglobalstyle
\centering
\begin{tabular}[t]{\X{33}{99}\X{33}{99}\X{33}{99}}
\sphinxtoprule
\sphinxtableatstartofbodyhook
\sphinxAtStartPar
位域 |
&
\sphinxAtStartPar
名称     | |
&
\sphinxAtStartPar
描述                                        | |
\\
\sphinxhline
\sphinxAtStartPar
31:10
&\begin{itemize}
\item {} 
\end{itemize}
&\begin{itemize}
\item {} 
\end{itemize}
\\
\sphinxhline
\sphinxAtStartPar
9
&
\sphinxAtStartPar
PIN9
&
\sphinxAtStartPar
PIN9 上拉电阻使能                           |

\sphinxAtStartPar
0:禁能 1:使能                             |
\\
\sphinxhline
\sphinxAtStartPar
8
&
\sphinxAtStartPar
PIN8
&
\sphinxAtStartPar
PIN8 上拉电阻使能                           |

\sphinxAtStartPar
0:禁能 1:使能                             |
\\
\sphinxhline
\sphinxAtStartPar
7
&
\sphinxAtStartPar
PIN7
&
\sphinxAtStartPar
PIN7上拉电阻使能                            |

\sphinxAtStartPar
0:禁能 1:使能                             |
\\
\sphinxhline
\sphinxAtStartPar
6
&
\sphinxAtStartPar
PIN6
&
\sphinxAtStartPar
PIN6上拉电阻使能                            |

\sphinxAtStartPar
0:禁能 1:使能                             |
\\
\sphinxhline
\sphinxAtStartPar
5
&
\sphinxAtStartPar
PIN5
&
\sphinxAtStartPar
PIN5上拉电阻使能                            |

\sphinxAtStartPar
0:禁能 1:使能                             |
\\
\sphinxhline
\sphinxAtStartPar
4
&
\sphinxAtStartPar
PIN4
&
\sphinxAtStartPar
PIN4上拉电阻使能                            |

\sphinxAtStartPar
0:禁能 1:使能                             |
\\
\sphinxhline
\sphinxAtStartPar
3
&
\sphinxAtStartPar
PIN3
&
\sphinxAtStartPar
PIN3上拉电阻使能                            |

\sphinxAtStartPar
0:禁能 1:使能                             |
\\
\sphinxhline
\sphinxAtStartPar
2
&
\sphinxAtStartPar
PIN2
&
\sphinxAtStartPar
PIN2上拉电阻使能                            |

\sphinxAtStartPar
0:禁能 1:使能                             |
\\
\sphinxhline
\sphinxAtStartPar
1
&
\sphinxAtStartPar
PIN1
&
\sphinxAtStartPar
PIN1上拉电阻使能                            |

\sphinxAtStartPar
0:禁能 1:使能                             |
\\
\sphinxhline
\sphinxAtStartPar
0
&
\sphinxAtStartPar
PIN0
&
\sphinxAtStartPar
PIN0上拉电阻使能                            |

\sphinxAtStartPar
0:禁能 1:使能                             |
\\
\sphinxbottomrule
\end{tabular}
\sphinxtableafterendhook\par
\sphinxattableend\end{savenotes}


\subsubsection{PORTC端口上拉功能寄存器PULLU\_C}
\label{\detokenize{SWM241/_u529f_u80fd_u63cf_u8ff0/PORTCON:portcpullu-c}}

\begin{savenotes}\sphinxattablestart
\sphinxthistablewithglobalstyle
\centering
\begin{tabular}[t]{\X{20}{100}\X{20}{100}\X{20}{100}\X{20}{100}\X{20}{100}}
\sphinxtoprule
\sphinxtableatstartofbodyhook
\sphinxAtStartPar
寄存器 |
&
\begin{DUlineblock}{0em}
\item[] 偏移 |
\end{DUlineblock}
&
\begin{DUlineblock}{0em}
\item[] 
\item[] {\color{red}\bfseries{}|}
\end{DUlineblock}
&
\sphinxAtStartPar
复位值 |    描 | |
&
\begin{DUlineblock}{0em}
\item[] |
  |
\end{DUlineblock}
\\
\sphinxhline
\sphinxAtStartPar
PULLU\_C
&
\sphinxAtStartPar
0x20
&&
\sphinxAtStartPar
0 000000
&
\sphinxAtStartPar
端口C上拉使能控制寄存器    |
\\
\sphinxbottomrule
\end{tabular}
\sphinxtableafterendhook\par
\sphinxattableend\end{savenotes}


\begin{savenotes}\sphinxattablestart
\sphinxthistablewithglobalstyle
\centering
\begin{tabular}[t]{\X{12}{96}\X{12}{96}\X{12}{96}\X{12}{96}\X{12}{96}\X{12}{96}\X{12}{96}\X{12}{96}}
\sphinxtoprule
\sphinxtableatstartofbodyhook
\sphinxAtStartPar
31
&
\sphinxAtStartPar
30
&
\sphinxAtStartPar
29
&
\sphinxAtStartPar
28
&
\sphinxAtStartPar
27
&
\sphinxAtStartPar
26
&
\sphinxAtStartPar
25
&
\sphinxAtStartPar
24
\\
\sphinxhline\begin{itemize}
\item {} 
\end{itemize}
&&&&&&&\\
\sphinxhline
\sphinxAtStartPar
23
&
\sphinxAtStartPar
22
&
\sphinxAtStartPar
21
&
\sphinxAtStartPar
20
&
\sphinxAtStartPar
19
&
\sphinxAtStartPar
18
&
\sphinxAtStartPar
17
&
\sphinxAtStartPar
16
\\
\sphinxhline\begin{itemize}
\item {} 
\end{itemize}
&&&&&&&\\
\sphinxhline
\sphinxAtStartPar
15
&
\sphinxAtStartPar
14
&
\sphinxAtStartPar
13
&
\sphinxAtStartPar
12
&
\sphinxAtStartPar
11
&
\sphinxAtStartPar
10
&
\sphinxAtStartPar
9
&
\sphinxAtStartPar
8
\\
\sphinxhline\begin{itemize}
\item {} 
\end{itemize}
&&&&&&&\\
\sphinxhline
\sphinxAtStartPar
7
&
\sphinxAtStartPar
6
&
\sphinxAtStartPar
5
&
\sphinxAtStartPar
4
&
\sphinxAtStartPar
3
&
\sphinxAtStartPar
2
&
\sphinxAtStartPar
1
&
\sphinxAtStartPar
0
\\
\sphinxhline\begin{itemize}
\item {} 
\end{itemize}
&&&&&&&\\
\sphinxbottomrule
\end{tabular}
\sphinxtableafterendhook\par
\sphinxattableend\end{savenotes}


\begin{savenotes}\sphinxattablestart
\sphinxthistablewithglobalstyle
\centering
\begin{tabular}[t]{\X{33}{99}\X{33}{99}\X{33}{99}}
\sphinxtoprule
\sphinxtableatstartofbodyhook
\sphinxAtStartPar
位域 |
&
\sphinxAtStartPar
名称     | |
&
\sphinxAtStartPar
描述                                        | |
\\
\sphinxhline
\sphinxAtStartPar
31:4
&\begin{itemize}
\item {} 
\end{itemize}
&\begin{itemize}
\item {} 
\end{itemize}
\\
\sphinxhline
\sphinxAtStartPar
3
&
\sphinxAtStartPar
PIN3
&
\sphinxAtStartPar
PIN3上拉电阻使能                            |

\sphinxAtStartPar
0:禁能 1:使能                             |
\\
\sphinxhline
\sphinxAtStartPar
2
&
\sphinxAtStartPar
PIN2
&
\sphinxAtStartPar
PIN2上拉电阻使能                            |

\sphinxAtStartPar
0:禁能 1:使能                             |
\\
\sphinxhline
\sphinxAtStartPar
1
&
\sphinxAtStartPar
PIN1
&
\sphinxAtStartPar
PIN1上拉电阻使能                            |

\sphinxAtStartPar
0:禁能 1:使能                             |
\\
\sphinxhline
\sphinxAtStartPar
0
&
\sphinxAtStartPar
PIN0
&
\sphinxAtStartPar
PIN0上拉电阻使能                            |

\sphinxAtStartPar
0:禁能 1:使能                             |
\\
\sphinxbottomrule
\end{tabular}
\sphinxtableafterendhook\par
\sphinxattableend\end{savenotes}


\subsubsection{PORTD端口上拉功能寄存器PULLU\_D}
\label{\detokenize{SWM241/_u529f_u80fd_u63cf_u8ff0/PORTCON:portdpullu-d}}

\begin{savenotes}\sphinxattablestart
\sphinxthistablewithglobalstyle
\centering
\begin{tabular}[t]{\X{20}{100}\X{20}{100}\X{20}{100}\X{20}{100}\X{20}{100}}
\sphinxtoprule
\sphinxtableatstartofbodyhook
\sphinxAtStartPar
寄存器 |
&
\begin{DUlineblock}{0em}
\item[] 偏移 |
\end{DUlineblock}
&
\begin{DUlineblock}{0em}
\item[] 
\item[] {\color{red}\bfseries{}|}
\end{DUlineblock}
&
\sphinxAtStartPar
复位值 |    描 | |
&
\begin{DUlineblock}{0em}
\item[] |
  |
\end{DUlineblock}
\\
\sphinxhline
\sphinxAtStartPar
PULLU\_D
&
\sphinxAtStartPar
0x30
&&
\sphinxAtStartPar
0 000030
&
\sphinxAtStartPar
端口D上拉使能控制寄存器    |
\\
\sphinxbottomrule
\end{tabular}
\sphinxtableafterendhook\par
\sphinxattableend\end{savenotes}


\begin{savenotes}\sphinxattablestart
\sphinxthistablewithglobalstyle
\centering
\begin{tabular}[t]{\X{12}{96}\X{12}{96}\X{12}{96}\X{12}{96}\X{12}{96}\X{12}{96}\X{12}{96}\X{12}{96}}
\sphinxtoprule
\sphinxtableatstartofbodyhook
\sphinxAtStartPar
31
&
\sphinxAtStartPar
30
&
\sphinxAtStartPar
29
&
\sphinxAtStartPar
28
&
\sphinxAtStartPar
27
&
\sphinxAtStartPar
26
&
\sphinxAtStartPar
25
&
\sphinxAtStartPar
24
\\
\sphinxhline\begin{itemize}
\item {} 
\end{itemize}
&&&&&&&\\
\sphinxhline
\sphinxAtStartPar
23
&
\sphinxAtStartPar
22
&
\sphinxAtStartPar
21
&
\sphinxAtStartPar
20
&
\sphinxAtStartPar
19
&
\sphinxAtStartPar
18
&
\sphinxAtStartPar
17
&
\sphinxAtStartPar
16
\\
\sphinxhline\begin{itemize}
\item {} 
\end{itemize}
&&&&&&&\\
\sphinxhline
\sphinxAtStartPar
15
&
\sphinxAtStartPar
14
&
\sphinxAtStartPar
13
&
\sphinxAtStartPar
12
&
\sphinxAtStartPar
11
&
\sphinxAtStartPar
10
&
\sphinxAtStartPar
9
&
\sphinxAtStartPar
8
\\
\sphinxhline
\sphinxAtStartPar
PIN15
&
\sphinxAtStartPar
PIN14
&&&&&&\\
\sphinxhline
\sphinxAtStartPar
7
&
\sphinxAtStartPar
6
&
\sphinxAtStartPar
5
&
\sphinxAtStartPar
4
&
\sphinxAtStartPar
3
&
\sphinxAtStartPar
2
&
\sphinxAtStartPar
1
&
\sphinxAtStartPar
0
\\
\sphinxhline
\sphinxAtStartPar
PIN7
&
\sphinxAtStartPar
PIN6
&&&&&&\\
\sphinxbottomrule
\end{tabular}
\sphinxtableafterendhook\par
\sphinxattableend\end{savenotes}


\begin{savenotes}\sphinxattablestart
\sphinxthistablewithglobalstyle
\centering
\begin{tabular}[t]{\X{33}{99}\X{33}{99}\X{33}{99}}
\sphinxtoprule
\sphinxtableatstartofbodyhook
\sphinxAtStartPar
位域 |
&
\sphinxAtStartPar
名称     | |
&
\sphinxAtStartPar
描述                                        | |
\\
\sphinxhline
\sphinxAtStartPar
31:16
&\begin{itemize}
\item {} 
\end{itemize}
&\begin{itemize}
\item {} 
\end{itemize}
\\
\sphinxhline
\sphinxAtStartPar
15
&
\sphinxAtStartPar
PIN15
&
\sphinxAtStartPar
PIN15 上拉电阻使能                          |

\sphinxAtStartPar
0:禁能 1:使能                             |
\\
\sphinxhline
\sphinxAtStartPar
14
&
\sphinxAtStartPar
PIN14
&
\sphinxAtStartPar
PIN14 上拉电阻使能                          |

\sphinxAtStartPar
0:禁能 1:使能                             |
\\
\sphinxhline
\sphinxAtStartPar
13
&
\sphinxAtStartPar
PIN13
&
\sphinxAtStartPar
PIN13 上拉电阻使能                          |

\sphinxAtStartPar
0:禁能 1:使能                             |
\\
\sphinxhline
\sphinxAtStartPar
12
&
\sphinxAtStartPar
PIN12
&
\sphinxAtStartPar
PIN12上拉电阻使能                           |

\sphinxAtStartPar
0:禁能 1:使能                             |
\\
\sphinxhline
\sphinxAtStartPar
11
&
\sphinxAtStartPar
PIN11
&
\sphinxAtStartPar
PIN11 上拉电阻使能                          |

\sphinxAtStartPar
0:禁能 1:使能                             |
\\
\sphinxhline
\sphinxAtStartPar
10
&
\sphinxAtStartPar
PIN10
&
\sphinxAtStartPar
PIN10 上拉电阻使能                          |

\sphinxAtStartPar
0:禁能 1:使能                             |
\\
\sphinxhline
\sphinxAtStartPar
9
&
\sphinxAtStartPar
PIN9
&
\sphinxAtStartPar
PIN9 上拉电阻使能                           |

\sphinxAtStartPar
0:禁能 1:使能                             |
\\
\sphinxhline
\sphinxAtStartPar
8
&
\sphinxAtStartPar
PIN8
&
\sphinxAtStartPar
PIN8 上拉电阻使能                           |

\sphinxAtStartPar
0:禁能 1:使能                             |
\\
\sphinxhline
\sphinxAtStartPar
7
&
\sphinxAtStartPar
PIN7
&
\sphinxAtStartPar
PIN7上拉电阻使能                            |

\sphinxAtStartPar
0:禁能 1:使能                             |
\\
\sphinxhline
\sphinxAtStartPar
6
&
\sphinxAtStartPar
PIN6
&
\sphinxAtStartPar
PIN6上拉电阻使能                            |

\sphinxAtStartPar
0:禁能 1:使能                             |
\\
\sphinxhline
\sphinxAtStartPar
5
&
\sphinxAtStartPar
PIN5
&
\sphinxAtStartPar
PIN5上拉电阻使能                            |

\sphinxAtStartPar
0:禁能 1:使能                             |
\\
\sphinxhline
\sphinxAtStartPar
4
&
\sphinxAtStartPar
PIN4
&
\sphinxAtStartPar
PIN4上拉电阻使能                            |

\sphinxAtStartPar
0:禁能 1:使能                             |
\\
\sphinxhline
\sphinxAtStartPar
3
&
\sphinxAtStartPar
PIN3
&
\sphinxAtStartPar
PIN3上拉电阻使能                            |

\sphinxAtStartPar
0:禁能 1:使能                             |
\\
\sphinxhline
\sphinxAtStartPar
2
&
\sphinxAtStartPar
PIN2
&
\sphinxAtStartPar
PIN2上拉电阻使能                            |

\sphinxAtStartPar
0:禁能 1:使能                             |
\\
\sphinxhline
\sphinxAtStartPar
1
&
\sphinxAtStartPar
PIN1
&
\sphinxAtStartPar
PIN1上拉电阻使能                            |

\sphinxAtStartPar
0:禁能 1:使能                             |
\\
\sphinxhline
\sphinxAtStartPar
0
&
\sphinxAtStartPar
PIN0
&
\sphinxAtStartPar
PIN0上拉电阻使能                            |

\sphinxAtStartPar
0:禁能 1:使能                             |
\\
\sphinxbottomrule
\end{tabular}
\sphinxtableafterendhook\par
\sphinxattableend\end{savenotes}


\subsubsection{PORTA端口下拉功能寄存器PULLD\_A}
\label{\detokenize{SWM241/_u529f_u80fd_u63cf_u8ff0/PORTCON:portapulld-a}}

\begin{savenotes}\sphinxattablestart
\sphinxthistablewithglobalstyle
\centering
\begin{tabular}[t]{\X{20}{100}\X{20}{100}\X{20}{100}\X{20}{100}\X{20}{100}}
\sphinxtoprule
\sphinxtableatstartofbodyhook
\sphinxAtStartPar
寄存器 |
&
\begin{DUlineblock}{0em}
\item[] 偏移 |
\end{DUlineblock}
&
\begin{DUlineblock}{0em}
\item[] 
\item[] {\color{red}\bfseries{}|}
\end{DUlineblock}
&
\sphinxAtStartPar
复位值 |    描 | |
&
\begin{DUlineblock}{0em}
\item[] |
  |
\end{DUlineblock}
\\
\sphinxhline
\sphinxAtStartPar
PULLD\_A
&
\sphinxAtStartPar
0x100
&&
\sphinxAtStartPar
0 000020
&
\sphinxAtStartPar
端口A下拉使能控制寄存器    |
\\
\sphinxbottomrule
\end{tabular}
\sphinxtableafterendhook\par
\sphinxattableend\end{savenotes}


\begin{savenotes}\sphinxattablestart
\sphinxthistablewithglobalstyle
\centering
\begin{tabular}[t]{\X{12}{96}\X{12}{96}\X{12}{96}\X{12}{96}\X{12}{96}\X{12}{96}\X{12}{96}\X{12}{96}}
\sphinxtoprule
\sphinxtableatstartofbodyhook
\sphinxAtStartPar
31
&
\sphinxAtStartPar
30
&
\sphinxAtStartPar
29
&
\sphinxAtStartPar
28
&
\sphinxAtStartPar
27
&
\sphinxAtStartPar
26
&
\sphinxAtStartPar
25
&
\sphinxAtStartPar
24
\\
\sphinxhline\begin{itemize}
\item {} 
\end{itemize}
&&&&&&&\\
\sphinxhline
\sphinxAtStartPar
23
&
\sphinxAtStartPar
22
&
\sphinxAtStartPar
21
&
\sphinxAtStartPar
20
&
\sphinxAtStartPar
19
&
\sphinxAtStartPar
18
&
\sphinxAtStartPar
17
&
\sphinxAtStartPar
16
\\
\sphinxhline\begin{itemize}
\item {} 
\end{itemize}
&&&&&&&\\
\sphinxhline
\sphinxAtStartPar
15
&
\sphinxAtStartPar
14
&
\sphinxAtStartPar
13
&
\sphinxAtStartPar
12
&
\sphinxAtStartPar
11
&
\sphinxAtStartPar
10
&
\sphinxAtStartPar
9
&
\sphinxAtStartPar
8
\\
\sphinxhline
\sphinxAtStartPar
PIN15
&
\sphinxAtStartPar
PIN14
&&&&&&\\
\sphinxhline
\sphinxAtStartPar
7
&
\sphinxAtStartPar
6
&
\sphinxAtStartPar
5
&
\sphinxAtStartPar
4
&
\sphinxAtStartPar
3
&
\sphinxAtStartPar
2
&
\sphinxAtStartPar
1
&
\sphinxAtStartPar
0
\\
\sphinxhline
\sphinxAtStartPar
PIN7
&
\sphinxAtStartPar
PIN6
&&&&&&\\
\sphinxbottomrule
\end{tabular}
\sphinxtableafterendhook\par
\sphinxattableend\end{savenotes}


\begin{savenotes}\sphinxattablestart
\sphinxthistablewithglobalstyle
\centering
\begin{tabular}[t]{\X{33}{99}\X{33}{99}\X{33}{99}}
\sphinxtoprule
\sphinxtableatstartofbodyhook
\sphinxAtStartPar
位域 |
&
\sphinxAtStartPar
名称     | |
&
\sphinxAtStartPar
描述                                        | |
\\
\sphinxhline
\sphinxAtStartPar
31:16
&\begin{itemize}
\item {} 
\end{itemize}
&\begin{itemize}
\item {} 
\end{itemize}
\\
\sphinxhline
\sphinxAtStartPar
15
&
\sphinxAtStartPar
PIN15
&
\sphinxAtStartPar
PIN15 下拉电阻使能                          |

\sphinxAtStartPar
0:禁能 1:使能                             |
\\
\sphinxhline
\sphinxAtStartPar
14
&
\sphinxAtStartPar
PIN14
&
\sphinxAtStartPar
PIN14下拉电阻使能                           |

\sphinxAtStartPar
0:禁能 1:使能                             |
\\
\sphinxhline
\sphinxAtStartPar
13
&
\sphinxAtStartPar
PIN13
&
\sphinxAtStartPar
PIN13下拉电阻使能                           |

\sphinxAtStartPar
0:禁能 1:使能                             |
\\
\sphinxhline
\sphinxAtStartPar
12
&
\sphinxAtStartPar
PIN12
&
\sphinxAtStartPar
PIN12下拉电阻使能                           |

\sphinxAtStartPar
0:禁能 1:使能                             |
\\
\sphinxhline
\sphinxAtStartPar
11
&
\sphinxAtStartPar
PIN11
&
\sphinxAtStartPar
PIN11下拉电阻使能                           |

\sphinxAtStartPar
0:禁能 1:使能                             |
\\
\sphinxhline
\sphinxAtStartPar
10
&
\sphinxAtStartPar
PIN10
&
\sphinxAtStartPar
PIN10下拉电阻使能                           |

\sphinxAtStartPar
0:禁能 1:使能                             |
\\
\sphinxhline
\sphinxAtStartPar
9
&
\sphinxAtStartPar
PIN9
&
\sphinxAtStartPar
PIN9下拉电阻使能                            |

\sphinxAtStartPar
0:禁能 1:使能                             |
\\
\sphinxhline
\sphinxAtStartPar
8
&
\sphinxAtStartPar
PIN8
&
\sphinxAtStartPar
PIN8下拉电阻使能                            |

\sphinxAtStartPar
0:禁能 1:使能                             |
\\
\sphinxhline
\sphinxAtStartPar
7
&
\sphinxAtStartPar
PIN7
&
\sphinxAtStartPar
PIN7下拉电阻使能                            |

\sphinxAtStartPar
0:禁能 1:使能                             |
\\
\sphinxhline
\sphinxAtStartPar
6
&
\sphinxAtStartPar
PIN6
&
\sphinxAtStartPar
PIN6下拉电阻使能                            |

\sphinxAtStartPar
0:禁能 1:使能                             |
\\
\sphinxhline
\sphinxAtStartPar
5
&
\sphinxAtStartPar
PIN5
&
\sphinxAtStartPar
PIN5下拉电阻使能                            |

\sphinxAtStartPar
0:禁能 1:使能                             |
\\
\sphinxhline
\sphinxAtStartPar
4
&
\sphinxAtStartPar
PIN4
&
\sphinxAtStartPar
PIN4下拉电阻使能                            |

\sphinxAtStartPar
0:禁能 1:使能                             |
\\
\sphinxhline
\sphinxAtStartPar
3
&
\sphinxAtStartPar
PIN3
&
\sphinxAtStartPar
PIN3下拉电阻使能                            |

\sphinxAtStartPar
0:禁能 1:使能                             |
\\
\sphinxhline
\sphinxAtStartPar
2
&
\sphinxAtStartPar
PIN2
&
\sphinxAtStartPar
PIN2下拉电阻使能                            |

\sphinxAtStartPar
0:禁能 1:使能                             |
\\
\sphinxhline
\sphinxAtStartPar
1
&
\sphinxAtStartPar
PIN1
&
\sphinxAtStartPar
PIN1下拉电阻使能                            |

\sphinxAtStartPar
0:禁能 1:使能                             |
\\
\sphinxhline
\sphinxAtStartPar
0
&
\sphinxAtStartPar
PIN0
&
\sphinxAtStartPar
PIN0下拉电阻使能                            |

\sphinxAtStartPar
0:禁能 1:使能                             |
\\
\sphinxbottomrule
\end{tabular}
\sphinxtableafterendhook\par
\sphinxattableend\end{savenotes}


\subsubsection{PORTB端口下拉功能寄存器PULLD\_B}
\label{\detokenize{SWM241/_u529f_u80fd_u63cf_u8ff0/PORTCON:portbpulld-b}}

\begin{savenotes}\sphinxattablestart
\sphinxthistablewithglobalstyle
\centering
\begin{tabular}[t]{\X{20}{100}\X{20}{100}\X{20}{100}\X{20}{100}\X{20}{100}}
\sphinxtoprule
\sphinxtableatstartofbodyhook
\sphinxAtStartPar
寄存器 |
&
\begin{DUlineblock}{0em}
\item[] 偏移 |
\end{DUlineblock}
&
\begin{DUlineblock}{0em}
\item[] 
\item[] {\color{red}\bfseries{}|}
\end{DUlineblock}
&
\sphinxAtStartPar
复位值 |    描 | |
&
\begin{DUlineblock}{0em}
\item[] |
  |
\end{DUlineblock}
\\
\sphinxhline
\sphinxAtStartPar
PULLD\_B
&
\sphinxAtStartPar
0x110
&&
\sphinxAtStartPar
0 000000
&
\sphinxAtStartPar
端口B下拉使能控制寄存器    |
\\
\sphinxbottomrule
\end{tabular}
\sphinxtableafterendhook\par
\sphinxattableend\end{savenotes}


\begin{savenotes}\sphinxattablestart
\sphinxthistablewithglobalstyle
\centering
\begin{tabular}[t]{\X{12}{96}\X{12}{96}\X{12}{96}\X{12}{96}\X{12}{96}\X{12}{96}\X{12}{96}\X{12}{96}}
\sphinxtoprule
\sphinxtableatstartofbodyhook
\sphinxAtStartPar
31
&
\sphinxAtStartPar
30
&
\sphinxAtStartPar
29
&
\sphinxAtStartPar
28
&
\sphinxAtStartPar
27
&
\sphinxAtStartPar
26
&
\sphinxAtStartPar
25
&
\sphinxAtStartPar
24
\\
\sphinxhline\begin{itemize}
\item {} 
\end{itemize}
&&&&&&&\\
\sphinxhline
\sphinxAtStartPar
23
&
\sphinxAtStartPar
22
&
\sphinxAtStartPar
21
&
\sphinxAtStartPar
20
&
\sphinxAtStartPar
19
&
\sphinxAtStartPar
18
&
\sphinxAtStartPar
17
&
\sphinxAtStartPar
16
\\
\sphinxhline\begin{itemize}
\item {} 
\end{itemize}
&&&&&&&\\
\sphinxhline
\sphinxAtStartPar
15
&
\sphinxAtStartPar
14
&
\sphinxAtStartPar
13
&
\sphinxAtStartPar
12
&
\sphinxAtStartPar
11
&
\sphinxAtStartPar
10
&
\sphinxAtStartPar
9
&
\sphinxAtStartPar
8
\\
\sphinxhline\begin{itemize}
\item {} 
\end{itemize}
&&&&&&&\\
\sphinxhline
\sphinxAtStartPar
7
&
\sphinxAtStartPar
6
&
\sphinxAtStartPar
5
&
\sphinxAtStartPar
4
&
\sphinxAtStartPar
3
&
\sphinxAtStartPar
2
&
\sphinxAtStartPar
1
&
\sphinxAtStartPar
0
\\
\sphinxhline
\sphinxAtStartPar
PIN7
&
\sphinxAtStartPar
PIN6
&&&&&&\\
\sphinxbottomrule
\end{tabular}
\sphinxtableafterendhook\par
\sphinxattableend\end{savenotes}


\begin{savenotes}\sphinxattablestart
\sphinxthistablewithglobalstyle
\centering
\begin{tabular}[t]{\X{33}{99}\X{33}{99}\X{33}{99}}
\sphinxtoprule
\sphinxtableatstartofbodyhook
\sphinxAtStartPar
位域 |
&
\sphinxAtStartPar
名称     | |
&
\sphinxAtStartPar
描述                                        | |
\\
\sphinxhline
\sphinxAtStartPar
31:10
&\begin{itemize}
\item {} 
\end{itemize}
&\begin{itemize}
\item {} 
\end{itemize}
\\
\sphinxhline
\sphinxAtStartPar
9
&
\sphinxAtStartPar
PIN9
&
\sphinxAtStartPar
PIN9下拉电阻使能                            |

\sphinxAtStartPar
0:禁能 1:使能                             |
\\
\sphinxhline
\sphinxAtStartPar
8
&
\sphinxAtStartPar
PIN8
&
\sphinxAtStartPar
PIN8下拉电阻使能                            |

\sphinxAtStartPar
0:禁能 1:使能                             |
\\
\sphinxhline
\sphinxAtStartPar
7
&
\sphinxAtStartPar
PIN7
&
\sphinxAtStartPar
PIN7下拉电阻使能                            |

\sphinxAtStartPar
0:禁能 1:使能                             |
\\
\sphinxhline
\sphinxAtStartPar
6
&
\sphinxAtStartPar
PIN6
&
\sphinxAtStartPar
PIN6下拉电阻使能                            |

\sphinxAtStartPar
0:禁能 1:使能                             |
\\
\sphinxhline
\sphinxAtStartPar
5
&
\sphinxAtStartPar
PIN5
&
\sphinxAtStartPar
PIN5下拉电阻使能                            |

\sphinxAtStartPar
0:禁能 1:使能                             |
\\
\sphinxhline
\sphinxAtStartPar
4
&
\sphinxAtStartPar
PIN4
&
\sphinxAtStartPar
PIN4下拉电阻使能                            |

\sphinxAtStartPar
0:禁能 1:使能                             |
\\
\sphinxhline
\sphinxAtStartPar
3
&
\sphinxAtStartPar
PIN3
&
\sphinxAtStartPar
PIN3下拉电阻使能                            |

\sphinxAtStartPar
0:禁能 1:使能                             |
\\
\sphinxhline
\sphinxAtStartPar
2
&
\sphinxAtStartPar
PIN2
&
\sphinxAtStartPar
PIN2下拉电阻使能                            |

\sphinxAtStartPar
0:禁能 1:使能                             |
\\
\sphinxhline
\sphinxAtStartPar
1
&
\sphinxAtStartPar
PIN1
&
\sphinxAtStartPar
PIN1下拉电阻使能                            |

\sphinxAtStartPar
0:禁能 1:使能                             |
\\
\sphinxhline
\sphinxAtStartPar
0
&
\sphinxAtStartPar
PIN0
&
\sphinxAtStartPar
PIN0下拉电阻使能                            |

\sphinxAtStartPar
0:禁能 1:使能                             |
\\
\sphinxbottomrule
\end{tabular}
\sphinxtableafterendhook\par
\sphinxattableend\end{savenotes}


\subsubsection{PORTC端口下拉功能寄存器PULLD\_C}
\label{\detokenize{SWM241/_u529f_u80fd_u63cf_u8ff0/PORTCON:portcpulld-c}}

\begin{savenotes}\sphinxattablestart
\sphinxthistablewithglobalstyle
\centering
\begin{tabular}[t]{\X{20}{100}\X{20}{100}\X{20}{100}\X{20}{100}\X{20}{100}}
\sphinxtoprule
\sphinxtableatstartofbodyhook
\sphinxAtStartPar
寄存器 |
&
\begin{DUlineblock}{0em}
\item[] 偏移 |
\end{DUlineblock}
&
\begin{DUlineblock}{0em}
\item[] 
\item[] {\color{red}\bfseries{}|}
\end{DUlineblock}
&
\sphinxAtStartPar
复位值 |    描 | |
&
\begin{DUlineblock}{0em}
\item[] |
  |
\end{DUlineblock}
\\
\sphinxhline
\sphinxAtStartPar
PULLD\_C
&
\sphinxAtStartPar
0x120
&&
\sphinxAtStartPar
0 000000
&
\sphinxAtStartPar
端口C下拉使能控制寄存器    |
\\
\sphinxbottomrule
\end{tabular}
\sphinxtableafterendhook\par
\sphinxattableend\end{savenotes}


\begin{savenotes}\sphinxattablestart
\sphinxthistablewithglobalstyle
\centering
\begin{tabular}[t]{\X{12}{96}\X{12}{96}\X{12}{96}\X{12}{96}\X{12}{96}\X{12}{96}\X{12}{96}\X{12}{96}}
\sphinxtoprule
\sphinxtableatstartofbodyhook
\sphinxAtStartPar
31
&
\sphinxAtStartPar
30
&
\sphinxAtStartPar
29
&
\sphinxAtStartPar
28
&
\sphinxAtStartPar
27
&
\sphinxAtStartPar
26
&
\sphinxAtStartPar
25
&
\sphinxAtStartPar
24
\\
\sphinxhline\begin{itemize}
\item {} 
\end{itemize}
&&&&&&&\\
\sphinxhline
\sphinxAtStartPar
23
&
\sphinxAtStartPar
22
&
\sphinxAtStartPar
21
&
\sphinxAtStartPar
20
&
\sphinxAtStartPar
19
&
\sphinxAtStartPar
18
&
\sphinxAtStartPar
17
&
\sphinxAtStartPar
16
\\
\sphinxhline\begin{itemize}
\item {} 
\end{itemize}
&&&&&&&\\
\sphinxhline
\sphinxAtStartPar
15
&
\sphinxAtStartPar
14
&
\sphinxAtStartPar
13
&
\sphinxAtStartPar
12
&
\sphinxAtStartPar
11
&
\sphinxAtStartPar
10
&
\sphinxAtStartPar
9
&
\sphinxAtStartPar
8
\\
\sphinxhline\begin{itemize}
\item {} 
\end{itemize}
&&&&&&&\\
\sphinxhline
\sphinxAtStartPar
7
&
\sphinxAtStartPar
6
&
\sphinxAtStartPar
5
&
\sphinxAtStartPar
4
&
\sphinxAtStartPar
3
&
\sphinxAtStartPar
2
&
\sphinxAtStartPar
1
&
\sphinxAtStartPar
0
\\
\sphinxhline\begin{itemize}
\item {} 
\end{itemize}
&&&&&&&\\
\sphinxbottomrule
\end{tabular}
\sphinxtableafterendhook\par
\sphinxattableend\end{savenotes}


\begin{savenotes}\sphinxattablestart
\sphinxthistablewithglobalstyle
\centering
\begin{tabular}[t]{\X{33}{99}\X{33}{99}\X{33}{99}}
\sphinxtoprule
\sphinxtableatstartofbodyhook
\sphinxAtStartPar
位域 |
&
\sphinxAtStartPar
名称     | |
&
\sphinxAtStartPar
描述                                        | |
\\
\sphinxhline
\sphinxAtStartPar
31:4
&\begin{itemize}
\item {} 
\end{itemize}
&\begin{itemize}
\item {} 
\end{itemize}
\\
\sphinxhline
\sphinxAtStartPar
3
&
\sphinxAtStartPar
PIN3
&
\sphinxAtStartPar
PIN3下拉电阻使能                            |

\sphinxAtStartPar
0:禁能 1:使能                             |
\\
\sphinxhline
\sphinxAtStartPar
2
&
\sphinxAtStartPar
PIN2
&
\sphinxAtStartPar
PIN2下拉电阻使能                            |

\sphinxAtStartPar
0:禁能 1:使能                             |
\\
\sphinxhline
\sphinxAtStartPar
1
&
\sphinxAtStartPar
PIN1
&
\sphinxAtStartPar
PIN1下拉电阻使能                            |

\sphinxAtStartPar
0:禁能 1:使能                             |
\\
\sphinxhline
\sphinxAtStartPar
0
&
\sphinxAtStartPar
PIN0
&
\sphinxAtStartPar
PIN0下拉电阻使能                            |

\sphinxAtStartPar
0:禁能 1:使能                             |
\\
\sphinxbottomrule
\end{tabular}
\sphinxtableafterendhook\par
\sphinxattableend\end{savenotes}


\subsubsection{PORTD端口下拉功能寄存器PULLD\_D}
\label{\detokenize{SWM241/_u529f_u80fd_u63cf_u8ff0/PORTCON:portdpulld-d}}

\begin{savenotes}\sphinxattablestart
\sphinxthistablewithglobalstyle
\centering
\begin{tabular}[t]{\X{20}{100}\X{20}{100}\X{20}{100}\X{20}{100}\X{20}{100}}
\sphinxtoprule
\sphinxtableatstartofbodyhook
\sphinxAtStartPar
寄存器 |
&
\begin{DUlineblock}{0em}
\item[] 偏移 |
\end{DUlineblock}
&
\begin{DUlineblock}{0em}
\item[] 
\item[] {\color{red}\bfseries{}|}
\end{DUlineblock}
&
\sphinxAtStartPar
复位值 |    描 | |
&
\begin{DUlineblock}{0em}
\item[] |
  |
\end{DUlineblock}
\\
\sphinxhline
\sphinxAtStartPar
PULLD\_D
&
\sphinxAtStartPar
0x130
&&
\sphinxAtStartPar
0 000001
&
\sphinxAtStartPar
端口D下拉使能控制寄存器    |
\\
\sphinxbottomrule
\end{tabular}
\sphinxtableafterendhook\par
\sphinxattableend\end{savenotes}


\begin{savenotes}\sphinxattablestart
\sphinxthistablewithglobalstyle
\centering
\begin{tabular}[t]{\X{12}{96}\X{12}{96}\X{12}{96}\X{12}{96}\X{12}{96}\X{12}{96}\X{12}{96}\X{12}{96}}
\sphinxtoprule
\sphinxtableatstartofbodyhook
\sphinxAtStartPar
31
&
\sphinxAtStartPar
30
&
\sphinxAtStartPar
29
&
\sphinxAtStartPar
28
&
\sphinxAtStartPar
27
&
\sphinxAtStartPar
26
&
\sphinxAtStartPar
25
&
\sphinxAtStartPar
24
\\
\sphinxhline\begin{itemize}
\item {} 
\end{itemize}
&&&&&&&\\
\sphinxhline
\sphinxAtStartPar
23
&
\sphinxAtStartPar
22
&
\sphinxAtStartPar
21
&
\sphinxAtStartPar
20
&
\sphinxAtStartPar
19
&
\sphinxAtStartPar
18
&
\sphinxAtStartPar
17
&
\sphinxAtStartPar
16
\\
\sphinxhline\begin{itemize}
\item {} 
\end{itemize}
&&&&&&&\\
\sphinxhline
\sphinxAtStartPar
15
&
\sphinxAtStartPar
14
&
\sphinxAtStartPar
13
&
\sphinxAtStartPar
12
&
\sphinxAtStartPar
11
&
\sphinxAtStartPar
10
&
\sphinxAtStartPar
9
&
\sphinxAtStartPar
8
\\
\sphinxhline
\sphinxAtStartPar
PIN15
&
\sphinxAtStartPar
PIN14
&&&&&&\\
\sphinxhline
\sphinxAtStartPar
7
&
\sphinxAtStartPar
6
&
\sphinxAtStartPar
5
&
\sphinxAtStartPar
4
&
\sphinxAtStartPar
3
&
\sphinxAtStartPar
2
&
\sphinxAtStartPar
1
&
\sphinxAtStartPar
0
\\
\sphinxhline
\sphinxAtStartPar
PIN7
&
\sphinxAtStartPar
PIN6
&&&&&&\\
\sphinxbottomrule
\end{tabular}
\sphinxtableafterendhook\par
\sphinxattableend\end{savenotes}


\begin{savenotes}\sphinxattablestart
\sphinxthistablewithglobalstyle
\centering
\begin{tabular}[t]{\X{33}{99}\X{33}{99}\X{33}{99}}
\sphinxtoprule
\sphinxtableatstartofbodyhook
\sphinxAtStartPar
位域 |
&
\sphinxAtStartPar
名称     | |
&
\sphinxAtStartPar
描述                                        | |
\\
\sphinxhline
\sphinxAtStartPar
31:16
&\begin{itemize}
\item {} 
\end{itemize}
&\begin{itemize}
\item {} 
\end{itemize}
\\
\sphinxhline
\sphinxAtStartPar
15
&
\sphinxAtStartPar
PIN15
&
\sphinxAtStartPar
PIN15 下拉电阻使能                          |

\sphinxAtStartPar
0:禁能 1:使能                             |
\\
\sphinxhline
\sphinxAtStartPar
14
&
\sphinxAtStartPar
PIN14
&
\sphinxAtStartPar
PIN14下拉电阻使能                           |

\sphinxAtStartPar
0:禁能 1:使能                             |
\\
\sphinxhline
\sphinxAtStartPar
13
&
\sphinxAtStartPar
PIN13
&
\sphinxAtStartPar
PIN13下拉电阻使能                           |

\sphinxAtStartPar
0:禁能 1:使能                             |
\\
\sphinxhline
\sphinxAtStartPar
12
&
\sphinxAtStartPar
PIN12
&
\sphinxAtStartPar
PIN12下拉电阻使能                           |

\sphinxAtStartPar
0:禁能 1:使能                             |
\\
\sphinxhline
\sphinxAtStartPar
11
&
\sphinxAtStartPar
PIN11
&
\sphinxAtStartPar
PIN11下拉电阻使能                           |

\sphinxAtStartPar
0:禁能 1:使能                             |
\\
\sphinxhline
\sphinxAtStartPar
10
&
\sphinxAtStartPar
PIN10
&
\sphinxAtStartPar
PIN10下拉电阻使能                           |

\sphinxAtStartPar
0:禁能 1:使能                             |
\\
\sphinxhline
\sphinxAtStartPar
9
&
\sphinxAtStartPar
PIN9
&
\sphinxAtStartPar
PIN9下拉电阻使能                            |

\sphinxAtStartPar
0:禁能 1:使能                             |
\\
\sphinxhline
\sphinxAtStartPar
8
&
\sphinxAtStartPar
PIN8
&
\sphinxAtStartPar
PIN8下拉电阻使能                            |

\sphinxAtStartPar
0:禁能 1:使能                             |
\\
\sphinxhline
\sphinxAtStartPar
7
&
\sphinxAtStartPar
PIN7
&
\sphinxAtStartPar
PIN7下拉电阻使能                            |

\sphinxAtStartPar
0:禁能 1:使能                             |
\\
\sphinxhline
\sphinxAtStartPar
6
&
\sphinxAtStartPar
PIN6
&
\sphinxAtStartPar
PIN6下拉电阻使能                            |

\sphinxAtStartPar
0:禁能 1:使能                             |
\\
\sphinxhline
\sphinxAtStartPar
5
&
\sphinxAtStartPar
PIN5
&
\sphinxAtStartPar
PIN5下拉电阻使能                            |

\sphinxAtStartPar
0:禁能 1:使能                             |
\\
\sphinxhline
\sphinxAtStartPar
4
&
\sphinxAtStartPar
PIN4
&
\sphinxAtStartPar
PIN4下拉电阻使能                            |

\sphinxAtStartPar
0:禁能 1:使能                             |
\\
\sphinxhline
\sphinxAtStartPar
3
&
\sphinxAtStartPar
PIN3
&
\sphinxAtStartPar
PIN3下拉电阻使能                            |

\sphinxAtStartPar
0:禁能 1:使能                             |
\\
\sphinxhline
\sphinxAtStartPar
2
&
\sphinxAtStartPar
PIN2
&
\sphinxAtStartPar
PIN2下拉电阻使能                            |

\sphinxAtStartPar
0:禁能 1:使能                             |
\\
\sphinxhline
\sphinxAtStartPar
1
&
\sphinxAtStartPar
PIN1
&
\sphinxAtStartPar
PIN1下拉电阻使能                            |

\sphinxAtStartPar
0:禁能 1:使能                             |
\\
\sphinxhline
\sphinxAtStartPar
0
&
\sphinxAtStartPar
PIN0
&
\sphinxAtStartPar
PIN0下拉电阻使能                            |

\sphinxAtStartPar
0:禁能 1:使能                             |
\\
\sphinxbottomrule
\end{tabular}
\sphinxtableafterendhook\par
\sphinxattableend\end{savenotes}


\subsubsection{PORTA端口输入使能功能寄存器INEN\_A}
\label{\detokenize{SWM241/_u529f_u80fd_u63cf_u8ff0/PORTCON:portainen-a}}

\begin{savenotes}\sphinxattablestart
\sphinxthistablewithglobalstyle
\centering
\begin{tabular}[t]{\X{20}{100}\X{20}{100}\X{20}{100}\X{20}{100}\X{20}{100}}
\sphinxtoprule
\sphinxtableatstartofbodyhook
\sphinxAtStartPar
寄存器 |
&
\begin{DUlineblock}{0em}
\item[] 偏移 |
\end{DUlineblock}
&
\begin{DUlineblock}{0em}
\item[] 
\item[] {\color{red}\bfseries{}|}
\end{DUlineblock}
&
\sphinxAtStartPar
复位值 |    描 | |
&
\begin{DUlineblock}{0em}
\item[] |
  |
\end{DUlineblock}
\\
\sphinxhline
\sphinxAtStartPar
INEN\_A
&
\sphinxAtStartPar
0x200
&&
\sphinxAtStartPar
0 000C20
&
\sphinxAtStartPar
端口A输入使能控制寄存器    |
\\
\sphinxbottomrule
\end{tabular}
\sphinxtableafterendhook\par
\sphinxattableend\end{savenotes}


\begin{savenotes}\sphinxattablestart
\sphinxthistablewithglobalstyle
\centering
\begin{tabular}[t]{\X{12}{96}\X{12}{96}\X{12}{96}\X{12}{96}\X{12}{96}\X{12}{96}\X{12}{96}\X{12}{96}}
\sphinxtoprule
\sphinxtableatstartofbodyhook
\sphinxAtStartPar
31
&
\sphinxAtStartPar
30
&
\sphinxAtStartPar
29
&
\sphinxAtStartPar
28
&
\sphinxAtStartPar
27
&
\sphinxAtStartPar
26
&
\sphinxAtStartPar
25
&
\sphinxAtStartPar
24
\\
\sphinxhline\begin{itemize}
\item {} 
\end{itemize}
&&&&&&&\\
\sphinxhline
\sphinxAtStartPar
23
&
\sphinxAtStartPar
22
&
\sphinxAtStartPar
21
&
\sphinxAtStartPar
20
&
\sphinxAtStartPar
19
&
\sphinxAtStartPar
18
&
\sphinxAtStartPar
17
&
\sphinxAtStartPar
16
\\
\sphinxhline\begin{itemize}
\item {} 
\end{itemize}
&&&&&&&\\
\sphinxhline
\sphinxAtStartPar
15
&
\sphinxAtStartPar
14
&
\sphinxAtStartPar
13
&
\sphinxAtStartPar
12
&
\sphinxAtStartPar
11
&
\sphinxAtStartPar
10
&
\sphinxAtStartPar
9
&
\sphinxAtStartPar
8
\\
\sphinxhline
\sphinxAtStartPar
PIN15
&
\sphinxAtStartPar
PIN14
&&&&&&\\
\sphinxhline
\sphinxAtStartPar
7
&
\sphinxAtStartPar
6
&
\sphinxAtStartPar
5
&
\sphinxAtStartPar
4
&
\sphinxAtStartPar
3
&
\sphinxAtStartPar
2
&
\sphinxAtStartPar
1
&
\sphinxAtStartPar
0
\\
\sphinxhline
\sphinxAtStartPar
PIN7
&
\sphinxAtStartPar
PIN6
&&&&&&\\
\sphinxbottomrule
\end{tabular}
\sphinxtableafterendhook\par
\sphinxattableend\end{savenotes}


\begin{savenotes}\sphinxattablestart
\sphinxthistablewithglobalstyle
\centering
\begin{tabular}[t]{\X{33}{99}\X{33}{99}\X{33}{99}}
\sphinxtoprule
\sphinxtableatstartofbodyhook
\sphinxAtStartPar
位域 |
&
\sphinxAtStartPar
名称     | |
&
\sphinxAtStartPar
描述                                        | |
\\
\sphinxhline
\sphinxAtStartPar
31:16
&\begin{itemize}
\item {} 
\end{itemize}
&\begin{itemize}
\item {} 
\end{itemize}
\\
\sphinxhline
\sphinxAtStartPar
15
&
\sphinxAtStartPar
PIN15
&
\sphinxAtStartPar
PIN15 输入使能                              |

\sphinxAtStartPar
0:禁能 1:使能                             |
\\
\sphinxhline
\sphinxAtStartPar
14
&
\sphinxAtStartPar
PIN14
&
\sphinxAtStartPar
PIN14 输入使能                              |

\sphinxAtStartPar
0:禁能 1:使能                             |
\\
\sphinxhline
\sphinxAtStartPar
13
&
\sphinxAtStartPar
PIN13
&
\sphinxAtStartPar
PIN13 输入使能                              |

\sphinxAtStartPar
0:禁能 1:使能                             |
\\
\sphinxhline
\sphinxAtStartPar
12
&
\sphinxAtStartPar
PIN12
&
\sphinxAtStartPar
PIN12 输入使能                              |

\sphinxAtStartPar
0:禁能 1:使能                             |
\\
\sphinxhline
\sphinxAtStartPar
11
&
\sphinxAtStartPar
PIN11
&
\sphinxAtStartPar
PIN11 输入使能                              |

\sphinxAtStartPar
0:禁能 1:使能                             |
\\
\sphinxhline
\sphinxAtStartPar
10
&
\sphinxAtStartPar
PIN10
&
\sphinxAtStartPar
PIN10 输入使能                              |

\sphinxAtStartPar
0:禁能 1:使能                             |
\\
\sphinxhline
\sphinxAtStartPar
9
&
\sphinxAtStartPar
PIN9
&
\sphinxAtStartPar
PIN9 输入使能                               |

\sphinxAtStartPar
0:禁能 1:使能                             |
\\
\sphinxhline
\sphinxAtStartPar
8
&
\sphinxAtStartPar
PIN8
&
\sphinxAtStartPar
PIN8 输入使能                               |

\sphinxAtStartPar
0:禁能 1:使能                             |
\\
\sphinxhline
\sphinxAtStartPar
7
&
\sphinxAtStartPar
PIN7
&
\sphinxAtStartPar
PIN7输入使能                                |

\sphinxAtStartPar
0:禁能 1:使能                             |
\\
\sphinxhline
\sphinxAtStartPar
6
&
\sphinxAtStartPar
PIN6
&
\sphinxAtStartPar
PIN6输入使能                                |

\sphinxAtStartPar
0:禁能 1:使能                             |
\\
\sphinxhline
\sphinxAtStartPar
5
&
\sphinxAtStartPar
PIN5
&
\sphinxAtStartPar
PIN5输入使能                                |

\sphinxAtStartPar
0:禁能 1:使能                             |
\\
\sphinxhline
\sphinxAtStartPar
4
&
\sphinxAtStartPar
PIN4
&
\sphinxAtStartPar
PIN4输入使能                                |

\sphinxAtStartPar
0:禁能 1:使能                             |
\\
\sphinxhline
\sphinxAtStartPar
3
&
\sphinxAtStartPar
PIN3
&
\sphinxAtStartPar
PIN3输入使能                                |

\sphinxAtStartPar
0:禁能 1:使能                             |
\\
\sphinxhline
\sphinxAtStartPar
2
&
\sphinxAtStartPar
PIN2
&
\sphinxAtStartPar
PIN2输入使能                                |

\sphinxAtStartPar
0:禁能 1:使能                             |
\\
\sphinxhline
\sphinxAtStartPar
1
&
\sphinxAtStartPar
PIN1
&
\sphinxAtStartPar
PIN1输入使能                                |

\sphinxAtStartPar
0:禁能 1:使能                             |
\\
\sphinxhline
\sphinxAtStartPar
0
&
\sphinxAtStartPar
PIN0
&
\sphinxAtStartPar
PIN0输入使能                                |

\sphinxAtStartPar
0:禁能 1:使能                             |
\\
\sphinxbottomrule
\end{tabular}
\sphinxtableafterendhook\par
\sphinxattableend\end{savenotes}


\subsubsection{\sphinxstylestrong{PORTB端口输入使能功能寄存器INEN\_B}}
\label{\detokenize{SWM241/_u529f_u80fd_u63cf_u8ff0/PORTCON:portbinen-b}}

\begin{savenotes}\sphinxattablestart
\sphinxthistablewithglobalstyle
\centering
\begin{tabular}[t]{\X{20}{100}\X{20}{100}\X{20}{100}\X{20}{100}\X{20}{100}}
\sphinxtoprule
\sphinxtableatstartofbodyhook
\sphinxAtStartPar
寄存器 |
&
\begin{DUlineblock}{0em}
\item[] 偏移 |
\end{DUlineblock}
&
\begin{DUlineblock}{0em}
\item[] 
\item[] {\color{red}\bfseries{}|}
\end{DUlineblock}
&
\sphinxAtStartPar
复位值 |    描 | |
&
\begin{DUlineblock}{0em}
\item[] |
  |
\end{DUlineblock}
\\
\sphinxhline
\sphinxAtStartPar
INEN\_B
&
\sphinxAtStartPar
0x210
&&
\sphinxAtStartPar
0 000000
&
\sphinxAtStartPar
端口B输入使能控制寄存器    |
\\
\sphinxbottomrule
\end{tabular}
\sphinxtableafterendhook\par
\sphinxattableend\end{savenotes}


\begin{savenotes}\sphinxattablestart
\sphinxthistablewithglobalstyle
\centering
\begin{tabular}[t]{\X{12}{96}\X{12}{96}\X{12}{96}\X{12}{96}\X{12}{96}\X{12}{96}\X{12}{96}\X{12}{96}}
\sphinxtoprule
\sphinxtableatstartofbodyhook
\sphinxAtStartPar
31
&
\sphinxAtStartPar
30
&
\sphinxAtStartPar
29
&
\sphinxAtStartPar
28
&
\sphinxAtStartPar
27
&
\sphinxAtStartPar
26
&
\sphinxAtStartPar
25
&
\sphinxAtStartPar
24
\\
\sphinxhline\begin{itemize}
\item {} 
\end{itemize}
&&&&&&&\\
\sphinxhline
\sphinxAtStartPar
23
&
\sphinxAtStartPar
22
&
\sphinxAtStartPar
21
&
\sphinxAtStartPar
20
&
\sphinxAtStartPar
19
&
\sphinxAtStartPar
18
&
\sphinxAtStartPar
17
&
\sphinxAtStartPar
16
\\
\sphinxhline\begin{itemize}
\item {} 
\end{itemize}
&&&&&&&\\
\sphinxhline
\sphinxAtStartPar
15
&
\sphinxAtStartPar
14
&
\sphinxAtStartPar
13
&
\sphinxAtStartPar
12
&
\sphinxAtStartPar
11
&
\sphinxAtStartPar
10
&
\sphinxAtStartPar
9
&
\sphinxAtStartPar
8
\\
\sphinxhline\begin{itemize}
\item {} 
\end{itemize}
&&&&&&&\\
\sphinxhline
\sphinxAtStartPar
7
&
\sphinxAtStartPar
6
&
\sphinxAtStartPar
5
&
\sphinxAtStartPar
4
&
\sphinxAtStartPar
3
&
\sphinxAtStartPar
2
&
\sphinxAtStartPar
1
&
\sphinxAtStartPar
0
\\
\sphinxhline
\sphinxAtStartPar
PIN7
&
\sphinxAtStartPar
PIN6
&&&&&&\\
\sphinxbottomrule
\end{tabular}
\sphinxtableafterendhook\par
\sphinxattableend\end{savenotes}


\begin{savenotes}\sphinxattablestart
\sphinxthistablewithglobalstyle
\centering
\begin{tabular}[t]{\X{33}{99}\X{33}{99}\X{33}{99}}
\sphinxtoprule
\sphinxtableatstartofbodyhook
\sphinxAtStartPar
位域 |
&
\sphinxAtStartPar
名称     | |
&
\sphinxAtStartPar
描述                                        | |
\\
\sphinxhline
\sphinxAtStartPar
31:10
&\begin{itemize}
\item {} 
\end{itemize}
&\begin{itemize}
\item {} 
\end{itemize}
\\
\sphinxhline
\sphinxAtStartPar
9
&
\sphinxAtStartPar
PIN9
&
\sphinxAtStartPar
PIN9 输入使能                               |

\sphinxAtStartPar
0:禁能 1:使能                             |
\\
\sphinxhline
\sphinxAtStartPar
8
&
\sphinxAtStartPar
PIN8
&
\sphinxAtStartPar
PIN8 输入使能                               |

\sphinxAtStartPar
0:禁能 1:使能                             |
\\
\sphinxhline
\sphinxAtStartPar
7
&
\sphinxAtStartPar
PIN7
&
\sphinxAtStartPar
PIN7输入使能                                |

\sphinxAtStartPar
0:禁能 1:使能                             |
\\
\sphinxhline
\sphinxAtStartPar
6
&
\sphinxAtStartPar
PIN6
&
\sphinxAtStartPar
PIN6输入使能                                |

\sphinxAtStartPar
0:禁能 1:使能                             |
\\
\sphinxhline
\sphinxAtStartPar
5
&
\sphinxAtStartPar
PIN5
&
\sphinxAtStartPar
PIN5输入使能                                |

\sphinxAtStartPar
0:禁能 1:使能                             |
\\
\sphinxhline
\sphinxAtStartPar
4
&
\sphinxAtStartPar
PIN4
&
\sphinxAtStartPar
PIN4输入使能                                |

\sphinxAtStartPar
0:禁能 1:使能                             |
\\
\sphinxhline
\sphinxAtStartPar
3
&
\sphinxAtStartPar
PIN3
&
\sphinxAtStartPar
PIN3输入使能                                |

\sphinxAtStartPar
0:禁能 1:使能                             |
\\
\sphinxhline
\sphinxAtStartPar
2
&
\sphinxAtStartPar
PIN2
&
\sphinxAtStartPar
PIN2输入使能                                |

\sphinxAtStartPar
0:禁能 1:使能                             |
\\
\sphinxhline
\sphinxAtStartPar
1
&
\sphinxAtStartPar
PIN1
&
\sphinxAtStartPar
PIN1输入使能                                |

\sphinxAtStartPar
0:禁能 1:使能                             |
\\
\sphinxhline
\sphinxAtStartPar
0
&
\sphinxAtStartPar
PIN0
&
\sphinxAtStartPar
PIN0输入使能                                |

\sphinxAtStartPar
0:禁能 1:使能                             |
\\
\sphinxbottomrule
\end{tabular}
\sphinxtableafterendhook\par
\sphinxattableend\end{savenotes}


\subsubsection{PORTC端口输入使能功能寄存器INEN\_C}
\label{\detokenize{SWM241/_u529f_u80fd_u63cf_u8ff0/PORTCON:portcinen-c}}

\begin{savenotes}\sphinxattablestart
\sphinxthistablewithglobalstyle
\centering
\begin{tabular}[t]{\X{20}{100}\X{20}{100}\X{20}{100}\X{20}{100}\X{20}{100}}
\sphinxtoprule
\sphinxtableatstartofbodyhook
\sphinxAtStartPar
寄存器 |
&
\begin{DUlineblock}{0em}
\item[] 偏移 |
\end{DUlineblock}
&
\begin{DUlineblock}{0em}
\item[] 
\item[] {\color{red}\bfseries{}|}
\end{DUlineblock}
&
\sphinxAtStartPar
复位值 |    描 | |
&
\begin{DUlineblock}{0em}
\item[] |
  |
\end{DUlineblock}
\\
\sphinxhline
\sphinxAtStartPar
INEN\_C
&
\sphinxAtStartPar
0x220
&&
\sphinxAtStartPar
0 000000
&
\sphinxAtStartPar
端口C输入使能控制寄存器    |
\\
\sphinxbottomrule
\end{tabular}
\sphinxtableafterendhook\par
\sphinxattableend\end{savenotes}


\begin{savenotes}\sphinxattablestart
\sphinxthistablewithglobalstyle
\centering
\begin{tabular}[t]{\X{12}{96}\X{12}{96}\X{12}{96}\X{12}{96}\X{12}{96}\X{12}{96}\X{12}{96}\X{12}{96}}
\sphinxtoprule
\sphinxtableatstartofbodyhook
\sphinxAtStartPar
31
&
\sphinxAtStartPar
30
&
\sphinxAtStartPar
29
&
\sphinxAtStartPar
28
&
\sphinxAtStartPar
27
&
\sphinxAtStartPar
26
&
\sphinxAtStartPar
25
&
\sphinxAtStartPar
24
\\
\sphinxhline\begin{itemize}
\item {} 
\end{itemize}
&&&&&&&\\
\sphinxhline
\sphinxAtStartPar
23
&
\sphinxAtStartPar
22
&
\sphinxAtStartPar
21
&
\sphinxAtStartPar
20
&
\sphinxAtStartPar
19
&
\sphinxAtStartPar
18
&
\sphinxAtStartPar
17
&
\sphinxAtStartPar
16
\\
\sphinxhline\begin{itemize}
\item {} 
\end{itemize}
&&&&&&&\\
\sphinxhline
\sphinxAtStartPar
15
&
\sphinxAtStartPar
14
&
\sphinxAtStartPar
13
&
\sphinxAtStartPar
12
&
\sphinxAtStartPar
11
&
\sphinxAtStartPar
10
&
\sphinxAtStartPar
9
&
\sphinxAtStartPar
8
\\
\sphinxhline\begin{itemize}
\item {} 
\end{itemize}
&&&&&&&\\
\sphinxhline
\sphinxAtStartPar
7
&
\sphinxAtStartPar
6
&
\sphinxAtStartPar
5
&
\sphinxAtStartPar
4
&
\sphinxAtStartPar
3
&
\sphinxAtStartPar
2
&
\sphinxAtStartPar
1
&
\sphinxAtStartPar
0
\\
\sphinxhline\begin{itemize}
\item {} 
\end{itemize}
&&&&&&&\\
\sphinxbottomrule
\end{tabular}
\sphinxtableafterendhook\par
\sphinxattableend\end{savenotes}


\begin{savenotes}\sphinxattablestart
\sphinxthistablewithglobalstyle
\centering
\begin{tabular}[t]{\X{33}{99}\X{33}{99}\X{33}{99}}
\sphinxtoprule
\sphinxtableatstartofbodyhook
\sphinxAtStartPar
位域 |
&
\sphinxAtStartPar
名称     | |
&
\sphinxAtStartPar
描述                                        | |
\\
\sphinxhline
\sphinxAtStartPar
31:4
&\begin{itemize}
\item {} 
\end{itemize}
&\begin{itemize}
\item {} 
\end{itemize}
\\
\sphinxhline
\sphinxAtStartPar
3
&
\sphinxAtStartPar
PIN3
&
\sphinxAtStartPar
PIN3输入使能                                |

\sphinxAtStartPar
0:禁能 1:使能                             |
\\
\sphinxhline
\sphinxAtStartPar
2
&
\sphinxAtStartPar
PIN2
&
\sphinxAtStartPar
PIN2输入使能                                |

\sphinxAtStartPar
0:禁能 1:使能                             |
\\
\sphinxhline
\sphinxAtStartPar
1
&
\sphinxAtStartPar
PIN1
&
\sphinxAtStartPar
PIN1输入使能                                |

\sphinxAtStartPar
0:禁能 1:使能                             |
\\
\sphinxhline
\sphinxAtStartPar
0
&
\sphinxAtStartPar
PIN0
&
\sphinxAtStartPar
PIN0输入使能                                |

\sphinxAtStartPar
0:禁能 1:使能                             |
\\
\sphinxbottomrule
\end{tabular}
\sphinxtableafterendhook\par
\sphinxattableend\end{savenotes}


\subsubsection{PORTD端口输入使能功能寄存器INEN\_D}
\label{\detokenize{SWM241/_u529f_u80fd_u63cf_u8ff0/PORTCON:portdinen-d}}

\begin{savenotes}\sphinxattablestart
\sphinxthistablewithglobalstyle
\centering
\begin{tabular}[t]{\X{20}{100}\X{20}{100}\X{20}{100}\X{20}{100}\X{20}{100}}
\sphinxtoprule
\sphinxtableatstartofbodyhook
\sphinxAtStartPar
寄存器 |
&
\begin{DUlineblock}{0em}
\item[] 偏移 |
\end{DUlineblock}
&
\begin{DUlineblock}{0em}
\item[] 
\item[] {\color{red}\bfseries{}|}
\end{DUlineblock}
&
\sphinxAtStartPar
复位值 |    描 | |
&
\begin{DUlineblock}{0em}
\item[] |
  |
\end{DUlineblock}
\\
\sphinxhline
\sphinxAtStartPar
INEN\_D
&
\sphinxAtStartPar
0x230
&&
\sphinxAtStartPar
0 000031
&
\sphinxAtStartPar
端口D输入使能控制寄存器    |
\\
\sphinxbottomrule
\end{tabular}
\sphinxtableafterendhook\par
\sphinxattableend\end{savenotes}


\begin{savenotes}\sphinxattablestart
\sphinxthistablewithglobalstyle
\centering
\begin{tabular}[t]{\X{12}{96}\X{12}{96}\X{12}{96}\X{12}{96}\X{12}{96}\X{12}{96}\X{12}{96}\X{12}{96}}
\sphinxtoprule
\sphinxtableatstartofbodyhook
\sphinxAtStartPar
31
&
\sphinxAtStartPar
30
&
\sphinxAtStartPar
29
&
\sphinxAtStartPar
28
&
\sphinxAtStartPar
27
&
\sphinxAtStartPar
26
&
\sphinxAtStartPar
25
&
\sphinxAtStartPar
24
\\
\sphinxhline\begin{itemize}
\item {} 
\end{itemize}
&&&&&&&\\
\sphinxhline
\sphinxAtStartPar
23
&
\sphinxAtStartPar
22
&
\sphinxAtStartPar
21
&
\sphinxAtStartPar
20
&
\sphinxAtStartPar
19
&
\sphinxAtStartPar
18
&
\sphinxAtStartPar
17
&
\sphinxAtStartPar
16
\\
\sphinxhline\begin{itemize}
\item {} 
\end{itemize}
&&&&&&&\\
\sphinxhline
\sphinxAtStartPar
15
&
\sphinxAtStartPar
14
&
\sphinxAtStartPar
13
&
\sphinxAtStartPar
12
&
\sphinxAtStartPar
11
&
\sphinxAtStartPar
10
&
\sphinxAtStartPar
9
&
\sphinxAtStartPar
8
\\
\sphinxhline
\sphinxAtStartPar
PIN15
&
\sphinxAtStartPar
PIN14
&&&&&&\\
\sphinxhline
\sphinxAtStartPar
7
&
\sphinxAtStartPar
6
&
\sphinxAtStartPar
5
&
\sphinxAtStartPar
4
&
\sphinxAtStartPar
3
&
\sphinxAtStartPar
2
&
\sphinxAtStartPar
1
&
\sphinxAtStartPar
0
\\
\sphinxhline
\sphinxAtStartPar
PIN7
&
\sphinxAtStartPar
PIN6
&&&&&&\\
\sphinxbottomrule
\end{tabular}
\sphinxtableafterendhook\par
\sphinxattableend\end{savenotes}


\begin{savenotes}\sphinxattablestart
\sphinxthistablewithglobalstyle
\centering
\begin{tabular}[t]{\X{33}{99}\X{33}{99}\X{33}{99}}
\sphinxtoprule
\sphinxtableatstartofbodyhook
\sphinxAtStartPar
位域 |
&
\sphinxAtStartPar
名称     | |
&
\sphinxAtStartPar
描述                                        | |
\\
\sphinxhline
\sphinxAtStartPar
31:16
&\begin{itemize}
\item {} 
\end{itemize}
&\begin{itemize}
\item {} 
\end{itemize}
\\
\sphinxhline
\sphinxAtStartPar
15
&
\sphinxAtStartPar
PIN15
&
\sphinxAtStartPar
PIN15 输入使能                              |

\sphinxAtStartPar
0:禁能 1:使能                             |
\\
\sphinxhline
\sphinxAtStartPar
14
&
\sphinxAtStartPar
PIN14
&
\sphinxAtStartPar
PIN14 输入使能                              |

\sphinxAtStartPar
0:禁能 1:使能                             |
\\
\sphinxhline
\sphinxAtStartPar
13
&
\sphinxAtStartPar
PIN13
&
\sphinxAtStartPar
PIN13 输入使能                              |

\sphinxAtStartPar
0:禁能 1:使能                             |
\\
\sphinxhline
\sphinxAtStartPar
12
&
\sphinxAtStartPar
PIN12
&
\sphinxAtStartPar
PIN12 输入使能                              |

\sphinxAtStartPar
0:禁能 1:使能                             |
\\
\sphinxhline
\sphinxAtStartPar
11
&
\sphinxAtStartPar
PIN11
&
\sphinxAtStartPar
PIN11 输入使能                              |

\sphinxAtStartPar
0:禁能 1:使能                             |
\\
\sphinxhline
\sphinxAtStartPar
10
&
\sphinxAtStartPar
PIN10
&
\sphinxAtStartPar
PIN10 输入使能                              |

\sphinxAtStartPar
0:禁能 1:使能                             |
\\
\sphinxhline
\sphinxAtStartPar
9
&
\sphinxAtStartPar
PIN9
&
\sphinxAtStartPar
PIN9 输入使能                               |

\sphinxAtStartPar
0:禁能 1:使能                             |
\\
\sphinxhline
\sphinxAtStartPar
8
&
\sphinxAtStartPar
PIN8
&
\sphinxAtStartPar
PIN8 输入使能                               |

\sphinxAtStartPar
0:禁能 1:使能                             |
\\
\sphinxhline
\sphinxAtStartPar
7
&
\sphinxAtStartPar
PIN7
&
\sphinxAtStartPar
PIN7输入使能                                |

\sphinxAtStartPar
0:禁能 1:使能                             |
\\
\sphinxhline
\sphinxAtStartPar
6
&
\sphinxAtStartPar
PIN6
&
\sphinxAtStartPar
PIN6输入使能                                |

\sphinxAtStartPar
0:禁能 1:使能                             |
\\
\sphinxhline
\sphinxAtStartPar
5
&
\sphinxAtStartPar
PIN5
&
\sphinxAtStartPar
PIN5输入使能                                |

\sphinxAtStartPar
0:禁能 1:使能                             |
\\
\sphinxhline
\sphinxAtStartPar
4
&
\sphinxAtStartPar
PIN4
&
\sphinxAtStartPar
PIN4输入使能                                |

\sphinxAtStartPar
0:禁能 1:使能                             |
\\
\sphinxhline
\sphinxAtStartPar
3
&
\sphinxAtStartPar
PIN3
&
\sphinxAtStartPar
PIN3输入使能                                |

\sphinxAtStartPar
0:禁能 1:使能                             |
\\
\sphinxhline
\sphinxAtStartPar
2
&
\sphinxAtStartPar
PIN2
&
\sphinxAtStartPar
PIN2输入使能                                |

\sphinxAtStartPar
0:禁能 1:使能                             |
\\
\sphinxhline
\sphinxAtStartPar
1
&
\sphinxAtStartPar
PIN1
&
\sphinxAtStartPar
PIN1输入使能                                |

\sphinxAtStartPar
0:禁能 1:使能                             |
\\
\sphinxhline
\sphinxAtStartPar
0
&
\sphinxAtStartPar
PIN0
&
\sphinxAtStartPar
PIN0输入使能                                |

\sphinxAtStartPar
0:禁能 1:使能                             |
\\
\sphinxbottomrule
\end{tabular}
\sphinxtableafterendhook\par
\sphinxattableend\end{savenotes}


\subsubsection{PORTA端口开漏功能寄存器OPEND\_A}
\label{\detokenize{SWM241/_u529f_u80fd_u63cf_u8ff0/PORTCON:portaopend-a}}

\begin{savenotes}\sphinxattablestart
\sphinxthistablewithglobalstyle
\centering
\begin{tabular}[t]{\X{20}{100}\X{20}{100}\X{20}{100}\X{20}{100}\X{20}{100}}
\sphinxtoprule
\sphinxtableatstartofbodyhook
\sphinxAtStartPar
寄存器 |
&
\begin{DUlineblock}{0em}
\item[] 偏移 |
\end{DUlineblock}
&
\begin{DUlineblock}{0em}
\item[] 
\item[] {\color{red}\bfseries{}|}
\end{DUlineblock}
&
\sphinxAtStartPar
复位值 |    描 | |
&
\begin{DUlineblock}{0em}
\item[] |
  |
\end{DUlineblock}
\\
\sphinxhline
\sphinxAtStartPar
OPEND\_A
&
\sphinxAtStartPar
0x300
&&
\sphinxAtStartPar
0 000000
&
\sphinxAtStartPar
端口A开漏使能控制寄存器    |
\\
\sphinxbottomrule
\end{tabular}
\sphinxtableafterendhook\par
\sphinxattableend\end{savenotes}


\begin{savenotes}\sphinxattablestart
\sphinxthistablewithglobalstyle
\centering
\begin{tabular}[t]{\X{12}{96}\X{12}{96}\X{12}{96}\X{12}{96}\X{12}{96}\X{12}{96}\X{12}{96}\X{12}{96}}
\sphinxtoprule
\sphinxtableatstartofbodyhook
\sphinxAtStartPar
31
&
\sphinxAtStartPar
30
&
\sphinxAtStartPar
29
&
\sphinxAtStartPar
28
&
\sphinxAtStartPar
27
&
\sphinxAtStartPar
26
&
\sphinxAtStartPar
25
&
\sphinxAtStartPar
24
\\
\sphinxhline\begin{itemize}
\item {} 
\end{itemize}
&&&&&&&\\
\sphinxhline
\sphinxAtStartPar
23
&
\sphinxAtStartPar
22
&
\sphinxAtStartPar
21
&
\sphinxAtStartPar
20
&
\sphinxAtStartPar
19
&
\sphinxAtStartPar
18
&
\sphinxAtStartPar
17
&
\sphinxAtStartPar
16
\\
\sphinxhline\begin{itemize}
\item {} 
\end{itemize}
&&&&&&&\\
\sphinxhline
\sphinxAtStartPar
15
&
\sphinxAtStartPar
14
&
\sphinxAtStartPar
13
&
\sphinxAtStartPar
12
&
\sphinxAtStartPar
11
&
\sphinxAtStartPar
10
&
\sphinxAtStartPar
9
&
\sphinxAtStartPar
8
\\
\sphinxhline
\sphinxAtStartPar
PIN15
&
\sphinxAtStartPar
PIN14
&&&&&&\\
\sphinxhline
\sphinxAtStartPar
7
&
\sphinxAtStartPar
6
&
\sphinxAtStartPar
5
&
\sphinxAtStartPar
4
&
\sphinxAtStartPar
3
&
\sphinxAtStartPar
2
&
\sphinxAtStartPar
1
&
\sphinxAtStartPar
0
\\
\sphinxhline
\sphinxAtStartPar
PIN7
&
\sphinxAtStartPar
PIN6
&&&&&&\\
\sphinxbottomrule
\end{tabular}
\sphinxtableafterendhook\par
\sphinxattableend\end{savenotes}


\begin{savenotes}\sphinxattablestart
\sphinxthistablewithglobalstyle
\centering
\begin{tabular}[t]{\X{33}{99}\X{33}{99}\X{33}{99}}
\sphinxtoprule
\sphinxtableatstartofbodyhook
\sphinxAtStartPar
位域 |
&
\sphinxAtStartPar
名称     | |
&
\sphinxAtStartPar
描述                                        | |
\\
\sphinxhline
\sphinxAtStartPar
31:16
&\begin{itemize}
\item {} 
\end{itemize}
&\begin{itemize}
\item {} 
\end{itemize}
\\
\sphinxhline
\sphinxAtStartPar
15
&
\sphinxAtStartPar
PIN15
&
\sphinxAtStartPar
PIN15 开漏使能                              |

\sphinxAtStartPar
0:推挽模式                                 |

\sphinxAtStartPar
1:开漏模式                                 |
\\
\sphinxhline
\sphinxAtStartPar
14
&
\sphinxAtStartPar
PIN14
&
\sphinxAtStartPar
PIN14 开漏使能                              |

\sphinxAtStartPar
0:推挽模式                                 |

\sphinxAtStartPar
1:开漏模式                                 |
\\
\sphinxhline
\sphinxAtStartPar
13
&
\sphinxAtStartPar
PIN13
&
\sphinxAtStartPar
PIN13 开漏使能                              |

\sphinxAtStartPar
0:推挽模式                                 |

\sphinxAtStartPar
1:开漏模式                                 |
\\
\sphinxhline
\sphinxAtStartPar
12
&
\sphinxAtStartPar
PIN12
&
\sphinxAtStartPar
PIN12 开漏使能                              |

\sphinxAtStartPar
0:推挽模式                                 |

\sphinxAtStartPar
1:开漏模式                                 |
\\
\sphinxhline
\sphinxAtStartPar
11
&
\sphinxAtStartPar
PIN11
&
\sphinxAtStartPar
PIN11 开漏使能                              |

\sphinxAtStartPar
0:推挽模式                                 |

\sphinxAtStartPar
1:开漏模式                                 |
\\
\sphinxhline
\sphinxAtStartPar
10
&
\sphinxAtStartPar
PIN10
&
\sphinxAtStartPar
PIN10 开漏使能                              |

\sphinxAtStartPar
0:推挽模式                                 |

\sphinxAtStartPar
1:开漏模式                                 |
\\
\sphinxhline
\sphinxAtStartPar
9
&
\sphinxAtStartPar
PIN9
&
\sphinxAtStartPar
PIN9 开漏使能                               |

\sphinxAtStartPar
0:推挽模式                                 |

\sphinxAtStartPar
1:开漏模式                                 |
\\
\sphinxhline
\sphinxAtStartPar
8
&
\sphinxAtStartPar
PIN8
&
\sphinxAtStartPar
PIN8 开漏使能                               |

\sphinxAtStartPar
0:推挽模式                                 |

\sphinxAtStartPar
1:开漏模式                                 |
\\
\sphinxhline
\sphinxAtStartPar
7
&
\sphinxAtStartPar
PIN7
&
\sphinxAtStartPar
PIN7开漏使能                                |

\sphinxAtStartPar
0:推挽模式                                 |

\sphinxAtStartPar
1:开漏模式                                 |
\\
\sphinxhline
\sphinxAtStartPar
6
&
\sphinxAtStartPar
PIN6
&
\sphinxAtStartPar
PIN6开漏使能                                |

\sphinxAtStartPar
0:推挽模式                                 |

\sphinxAtStartPar
1:开漏模式                                 |
\\
\sphinxhline
\sphinxAtStartPar
5
&
\sphinxAtStartPar
PIN5
&
\sphinxAtStartPar
PIN5开漏使能                                |

\sphinxAtStartPar
0:推挽模式                                 |

\sphinxAtStartPar
1:开漏模式                                 |
\\
\sphinxhline
\sphinxAtStartPar
4
&
\sphinxAtStartPar
PIN4
&
\sphinxAtStartPar
PIN4开漏使能                                |

\sphinxAtStartPar
0:推挽模式                                 |

\sphinxAtStartPar
1:开漏模式                                 |
\\
\sphinxhline
\sphinxAtStartPar
3
&
\sphinxAtStartPar
PIN3
&
\sphinxAtStartPar
PIN3开漏使能                                |

\sphinxAtStartPar
0:推挽模式                                 |

\sphinxAtStartPar
1:开漏模式                                 |
\\
\sphinxhline
\sphinxAtStartPar
2
&
\sphinxAtStartPar
PIN2
&
\sphinxAtStartPar
PIN2开漏使能                                |

\sphinxAtStartPar
0:推挽模式                                 |

\sphinxAtStartPar
1:开漏模式                                 |
\\
\sphinxhline
\sphinxAtStartPar
1
&
\sphinxAtStartPar
PIN1
&
\sphinxAtStartPar
PIN1开漏使能                                |

\sphinxAtStartPar
0:推挽模式                                 |

\sphinxAtStartPar
1:开漏模式                                 |
\\
\sphinxhline
\sphinxAtStartPar
0
&
\sphinxAtStartPar
PIN0
&
\sphinxAtStartPar
PIN0开漏使能                                |

\sphinxAtStartPar
0:推挽模式                                 |

\sphinxAtStartPar
1:开漏模式                                 |
\\
\sphinxbottomrule
\end{tabular}
\sphinxtableafterendhook\par
\sphinxattableend\end{savenotes}


\subsubsection{PORTB端口开漏功能寄存器OPEND\_B}
\label{\detokenize{SWM241/_u529f_u80fd_u63cf_u8ff0/PORTCON:portbopend-b}}

\begin{savenotes}\sphinxattablestart
\sphinxthistablewithglobalstyle
\centering
\begin{tabular}[t]{\X{20}{100}\X{20}{100}\X{20}{100}\X{20}{100}\X{20}{100}}
\sphinxtoprule
\sphinxtableatstartofbodyhook
\sphinxAtStartPar
寄存器 |
&
\begin{DUlineblock}{0em}
\item[] 偏移 |
\end{DUlineblock}
&
\begin{DUlineblock}{0em}
\item[] 
\item[] {\color{red}\bfseries{}|}
\end{DUlineblock}
&
\sphinxAtStartPar
复位值 |    描 | |
&
\begin{DUlineblock}{0em}
\item[] |
  |
\end{DUlineblock}
\\
\sphinxhline
\sphinxAtStartPar
OPEND\_B
&
\sphinxAtStartPar
0x310
&&
\sphinxAtStartPar
0 000000
&
\sphinxAtStartPar
端口B开漏使能控制寄存器    |
\\
\sphinxbottomrule
\end{tabular}
\sphinxtableafterendhook\par
\sphinxattableend\end{savenotes}


\begin{savenotes}\sphinxattablestart
\sphinxthistablewithglobalstyle
\centering
\begin{tabular}[t]{\X{12}{96}\X{12}{96}\X{12}{96}\X{12}{96}\X{12}{96}\X{12}{96}\X{12}{96}\X{12}{96}}
\sphinxtoprule
\sphinxtableatstartofbodyhook
\sphinxAtStartPar
31
&
\sphinxAtStartPar
30
&
\sphinxAtStartPar
29
&
\sphinxAtStartPar
28
&
\sphinxAtStartPar
27
&
\sphinxAtStartPar
26
&
\sphinxAtStartPar
25
&
\sphinxAtStartPar
24
\\
\sphinxhline\begin{itemize}
\item {} 
\end{itemize}
&&&&&&&\\
\sphinxhline
\sphinxAtStartPar
23
&
\sphinxAtStartPar
22
&
\sphinxAtStartPar
21
&
\sphinxAtStartPar
20
&
\sphinxAtStartPar
19
&
\sphinxAtStartPar
18
&
\sphinxAtStartPar
17
&
\sphinxAtStartPar
16
\\
\sphinxhline\begin{itemize}
\item {} 
\end{itemize}
&&&&&&&\\
\sphinxhline
\sphinxAtStartPar
15
&
\sphinxAtStartPar
14
&
\sphinxAtStartPar
13
&
\sphinxAtStartPar
12
&
\sphinxAtStartPar
11
&
\sphinxAtStartPar
10
&
\sphinxAtStartPar
9
&
\sphinxAtStartPar
8
\\
\sphinxhline\begin{itemize}
\item {} 
\end{itemize}
&&&&&&&\\
\sphinxhline
\sphinxAtStartPar
7
&
\sphinxAtStartPar
6
&
\sphinxAtStartPar
5
&
\sphinxAtStartPar
4
&
\sphinxAtStartPar
3
&
\sphinxAtStartPar
2
&
\sphinxAtStartPar
1
&
\sphinxAtStartPar
0
\\
\sphinxhline
\sphinxAtStartPar
PIN7
&
\sphinxAtStartPar
PIN6
&&&&&&\\
\sphinxbottomrule
\end{tabular}
\sphinxtableafterendhook\par
\sphinxattableend\end{savenotes}


\begin{savenotes}\sphinxattablestart
\sphinxthistablewithglobalstyle
\centering
\begin{tabular}[t]{\X{33}{99}\X{33}{99}\X{33}{99}}
\sphinxtoprule
\sphinxtableatstartofbodyhook
\sphinxAtStartPar
位域 |
&
\sphinxAtStartPar
名称     | |
&
\sphinxAtStartPar
描述                                        | |
\\
\sphinxhline
\sphinxAtStartPar
31:10
&\begin{itemize}
\item {} 
\end{itemize}
&\begin{itemize}
\item {} 
\end{itemize}
\\
\sphinxhline
\sphinxAtStartPar
9
&
\sphinxAtStartPar
PIN9
&
\sphinxAtStartPar
PIN9 开漏使能                               |

\sphinxAtStartPar
0:推挽模式                                 |

\sphinxAtStartPar
1:开漏模式                                 |
\\
\sphinxhline
\sphinxAtStartPar
8
&
\sphinxAtStartPar
PIN8
&
\sphinxAtStartPar
PIN8 开漏使能                               |

\sphinxAtStartPar
0:推挽模式                                 |

\sphinxAtStartPar
1:开漏模式                                 |
\\
\sphinxhline
\sphinxAtStartPar
7
&
\sphinxAtStartPar
PIN7
&
\sphinxAtStartPar
PIN7开漏使能                                |

\sphinxAtStartPar
0:推挽模式                                 |

\sphinxAtStartPar
1:开漏模式                                 |
\\
\sphinxhline
\sphinxAtStartPar
6
&
\sphinxAtStartPar
PIN6
&
\sphinxAtStartPar
PIN6开漏使能                                |

\sphinxAtStartPar
0:推挽模式                                 |

\sphinxAtStartPar
1:开漏模式                                 |
\\
\sphinxhline
\sphinxAtStartPar
5
&
\sphinxAtStartPar
PIN5
&
\sphinxAtStartPar
PIN5开漏使能                                |

\sphinxAtStartPar
0:推挽模式                                 |

\sphinxAtStartPar
1:开漏模式                                 |
\\
\sphinxhline
\sphinxAtStartPar
4
&
\sphinxAtStartPar
PIN4
&
\sphinxAtStartPar
PIN4开漏使能                                |

\sphinxAtStartPar
0:推挽模式                                 |

\sphinxAtStartPar
1:开漏模式                                 |
\\
\sphinxhline
\sphinxAtStartPar
3
&
\sphinxAtStartPar
PIN3
&
\sphinxAtStartPar
PIN3开漏使能                                |

\sphinxAtStartPar
0:推挽模式                                 |

\sphinxAtStartPar
1:开漏模式                                 |
\\
\sphinxhline
\sphinxAtStartPar
2
&
\sphinxAtStartPar
PIN2
&
\sphinxAtStartPar
PIN2开漏使能                                |

\sphinxAtStartPar
0:推挽模式                                 |

\sphinxAtStartPar
1:开漏模式                                 |
\\
\sphinxhline
\sphinxAtStartPar
1
&
\sphinxAtStartPar
PIN1
&
\sphinxAtStartPar
PIN1开漏使能                                |

\sphinxAtStartPar
0:推挽模式                                 |

\sphinxAtStartPar
1:开漏模式                                 |
\\
\sphinxhline
\sphinxAtStartPar
0
&
\sphinxAtStartPar
PIN0
&
\sphinxAtStartPar
PIN0开漏使能                                |

\sphinxAtStartPar
0:推挽模式                                 |

\sphinxAtStartPar
1:开漏模式                                 |
\\
\sphinxbottomrule
\end{tabular}
\sphinxtableafterendhook\par
\sphinxattableend\end{savenotes}


\subsubsection{PORTC端口开漏功能寄存器OPEND\_C}
\label{\detokenize{SWM241/_u529f_u80fd_u63cf_u8ff0/PORTCON:portcopend-c}}

\begin{savenotes}\sphinxattablestart
\sphinxthistablewithglobalstyle
\centering
\begin{tabular}[t]{\X{20}{100}\X{20}{100}\X{20}{100}\X{20}{100}\X{20}{100}}
\sphinxtoprule
\sphinxtableatstartofbodyhook
\sphinxAtStartPar
寄存器 |
&
\begin{DUlineblock}{0em}
\item[] 偏移 |
\end{DUlineblock}
&
\begin{DUlineblock}{0em}
\item[] 
\item[] {\color{red}\bfseries{}|}
\end{DUlineblock}
&
\sphinxAtStartPar
复位值 |    描 | |
&
\begin{DUlineblock}{0em}
\item[] |
  |
\end{DUlineblock}
\\
\sphinxhline
\sphinxAtStartPar
OPEND\_C
&
\sphinxAtStartPar
0x320
&&
\sphinxAtStartPar
0 000000
&
\sphinxAtStartPar
端口C开漏使能控制寄存器    |
\\
\sphinxbottomrule
\end{tabular}
\sphinxtableafterendhook\par
\sphinxattableend\end{savenotes}


\begin{savenotes}\sphinxattablestart
\sphinxthistablewithglobalstyle
\centering
\begin{tabular}[t]{\X{12}{96}\X{12}{96}\X{12}{96}\X{12}{96}\X{12}{96}\X{12}{96}\X{12}{96}\X{12}{96}}
\sphinxtoprule
\sphinxtableatstartofbodyhook
\sphinxAtStartPar
31
&
\sphinxAtStartPar
30
&
\sphinxAtStartPar
29
&
\sphinxAtStartPar
28
&
\sphinxAtStartPar
27
&
\sphinxAtStartPar
26
&
\sphinxAtStartPar
25
&
\sphinxAtStartPar
24
\\
\sphinxhline\begin{itemize}
\item {} 
\end{itemize}
&&&&&&&\\
\sphinxhline
\sphinxAtStartPar
23
&
\sphinxAtStartPar
22
&
\sphinxAtStartPar
21
&
\sphinxAtStartPar
20
&
\sphinxAtStartPar
19
&
\sphinxAtStartPar
18
&
\sphinxAtStartPar
17
&
\sphinxAtStartPar
16
\\
\sphinxhline\begin{itemize}
\item {} 
\end{itemize}
&&&&&&&\\
\sphinxhline
\sphinxAtStartPar
15
&
\sphinxAtStartPar
14
&
\sphinxAtStartPar
13
&
\sphinxAtStartPar
12
&
\sphinxAtStartPar
11
&
\sphinxAtStartPar
10
&
\sphinxAtStartPar
9
&
\sphinxAtStartPar
8
\\
\sphinxhline\begin{itemize}
\item {} 
\end{itemize}
&&&&&&&\\
\sphinxhline
\sphinxAtStartPar
7
&
\sphinxAtStartPar
6
&
\sphinxAtStartPar
5
&
\sphinxAtStartPar
4
&
\sphinxAtStartPar
3
&
\sphinxAtStartPar
2
&
\sphinxAtStartPar
1
&
\sphinxAtStartPar
0
\\
\sphinxhline\begin{itemize}
\item {} 
\end{itemize}
&&&&&&&\\
\sphinxbottomrule
\end{tabular}
\sphinxtableafterendhook\par
\sphinxattableend\end{savenotes}


\begin{savenotes}\sphinxattablestart
\sphinxthistablewithglobalstyle
\centering
\begin{tabular}[t]{\X{33}{99}\X{33}{99}\X{33}{99}}
\sphinxtoprule
\sphinxtableatstartofbodyhook
\sphinxAtStartPar
位域 |
&
\sphinxAtStartPar
名称     | |
&
\sphinxAtStartPar
描述                                        | |
\\
\sphinxhline
\sphinxAtStartPar
31:4
&\begin{itemize}
\item {} 
\end{itemize}
&\begin{itemize}
\item {} 
\end{itemize}
\\
\sphinxhline
\sphinxAtStartPar
3
&
\sphinxAtStartPar
PIN3
&
\sphinxAtStartPar
PIN3开漏使能                                |

\sphinxAtStartPar
0:推挽模式                                 |

\sphinxAtStartPar
1:开漏模式                                 |
\\
\sphinxhline
\sphinxAtStartPar
2
&
\sphinxAtStartPar
PIN2
&
\sphinxAtStartPar
PIN2开漏使能                                |

\sphinxAtStartPar
0:推挽模式                                 |

\sphinxAtStartPar
1:开漏模式                                 |
\\
\sphinxhline
\sphinxAtStartPar
1
&
\sphinxAtStartPar
PIN1
&
\sphinxAtStartPar
PIN1开漏使能                                |

\sphinxAtStartPar
0:推挽模式                                 |

\sphinxAtStartPar
1:开漏模式                                 |
\\
\sphinxhline
\sphinxAtStartPar
0
&
\sphinxAtStartPar
PIN0
&
\sphinxAtStartPar
PIN0开漏使能                                |

\sphinxAtStartPar
0:推挽模式                                 |

\sphinxAtStartPar
1:开漏模式                                 |
\\
\sphinxbottomrule
\end{tabular}
\sphinxtableafterendhook\par
\sphinxattableend\end{savenotes}


\subsubsection{PORTD端口开漏功能寄存器OPEND\_D}
\label{\detokenize{SWM241/_u529f_u80fd_u63cf_u8ff0/PORTCON:portdopend-d}}

\begin{savenotes}\sphinxattablestart
\sphinxthistablewithglobalstyle
\centering
\begin{tabular}[t]{\X{20}{100}\X{20}{100}\X{20}{100}\X{20}{100}\X{20}{100}}
\sphinxtoprule
\sphinxtableatstartofbodyhook
\sphinxAtStartPar
寄存器 |
&
\begin{DUlineblock}{0em}
\item[] 偏移 |
\end{DUlineblock}
&
\begin{DUlineblock}{0em}
\item[] 
\item[] {\color{red}\bfseries{}|}
\end{DUlineblock}
&
\sphinxAtStartPar
复位值 |    描 | |
&
\begin{DUlineblock}{0em}
\item[] |
  |
\end{DUlineblock}
\\
\sphinxhline
\sphinxAtStartPar
OPEND\_D
&
\sphinxAtStartPar
0x330
&&
\sphinxAtStartPar
0 000000
&
\sphinxAtStartPar
端口D开漏使能控制寄存器    |
\\
\sphinxbottomrule
\end{tabular}
\sphinxtableafterendhook\par
\sphinxattableend\end{savenotes}


\begin{savenotes}\sphinxattablestart
\sphinxthistablewithglobalstyle
\centering
\begin{tabular}[t]{\X{12}{96}\X{12}{96}\X{12}{96}\X{12}{96}\X{12}{96}\X{12}{96}\X{12}{96}\X{12}{96}}
\sphinxtoprule
\sphinxtableatstartofbodyhook
\sphinxAtStartPar
31
&
\sphinxAtStartPar
30
&
\sphinxAtStartPar
29
&
\sphinxAtStartPar
28
&
\sphinxAtStartPar
27
&
\sphinxAtStartPar
26
&
\sphinxAtStartPar
25
&
\sphinxAtStartPar
24
\\
\sphinxhline\begin{itemize}
\item {} 
\end{itemize}
&&&&&&&\\
\sphinxhline
\sphinxAtStartPar
23
&
\sphinxAtStartPar
22
&
\sphinxAtStartPar
21
&
\sphinxAtStartPar
20
&
\sphinxAtStartPar
19
&
\sphinxAtStartPar
18
&
\sphinxAtStartPar
17
&
\sphinxAtStartPar
16
\\
\sphinxhline\begin{itemize}
\item {} 
\end{itemize}
&&&&&&&\\
\sphinxhline
\sphinxAtStartPar
15
&
\sphinxAtStartPar
14
&
\sphinxAtStartPar
13
&
\sphinxAtStartPar
12
&
\sphinxAtStartPar
11
&
\sphinxAtStartPar
10
&
\sphinxAtStartPar
9
&
\sphinxAtStartPar
8
\\
\sphinxhline
\sphinxAtStartPar
PIN15
&
\sphinxAtStartPar
PIN14
&&&&&&\\
\sphinxhline
\sphinxAtStartPar
7
&
\sphinxAtStartPar
6
&
\sphinxAtStartPar
5
&
\sphinxAtStartPar
4
&
\sphinxAtStartPar
3
&
\sphinxAtStartPar
2
&
\sphinxAtStartPar
1
&
\sphinxAtStartPar
0
\\
\sphinxhline
\sphinxAtStartPar
PIN7
&
\sphinxAtStartPar
PIN6
&&&&&&\\
\sphinxbottomrule
\end{tabular}
\sphinxtableafterendhook\par
\sphinxattableend\end{savenotes}


\begin{savenotes}\sphinxattablestart
\sphinxthistablewithglobalstyle
\centering
\begin{tabular}[t]{\X{33}{99}\X{33}{99}\X{33}{99}}
\sphinxtoprule
\sphinxtableatstartofbodyhook
\sphinxAtStartPar
位域 |
&
\sphinxAtStartPar
名称     | |
&
\sphinxAtStartPar
描述                                        | |
\\
\sphinxhline
\sphinxAtStartPar
31:16
&\begin{itemize}
\item {} 
\end{itemize}
&\begin{itemize}
\item {} 
\end{itemize}
\\
\sphinxhline
\sphinxAtStartPar
15
&
\sphinxAtStartPar
PIN15
&
\sphinxAtStartPar
PIN15 开漏使能                              |

\sphinxAtStartPar
0:推挽模式                                 |

\sphinxAtStartPar
1:开漏模式                                 |
\\
\sphinxhline
\sphinxAtStartPar
14
&
\sphinxAtStartPar
PIN14
&
\sphinxAtStartPar
PIN14 开漏使能                              |

\sphinxAtStartPar
0:推挽模式                                 |

\sphinxAtStartPar
1:开漏模式                                 |
\\
\sphinxhline
\sphinxAtStartPar
13
&
\sphinxAtStartPar
PIN13
&
\sphinxAtStartPar
PIN13 开漏使能                              |

\sphinxAtStartPar
0:推挽模式                                 |

\sphinxAtStartPar
1:开漏模式                                 |
\\
\sphinxhline
\sphinxAtStartPar
12
&
\sphinxAtStartPar
PIN12
&
\sphinxAtStartPar
PIN12 开漏使能                              |

\sphinxAtStartPar
0:推挽模式                                 |

\sphinxAtStartPar
1:开漏模式                                 |
\\
\sphinxhline
\sphinxAtStartPar
11
&
\sphinxAtStartPar
PIN11
&
\sphinxAtStartPar
PIN11 开漏使能                              |

\sphinxAtStartPar
0:推挽模式                                 |

\sphinxAtStartPar
1:开漏模式                                 |
\\
\sphinxhline
\sphinxAtStartPar
10
&
\sphinxAtStartPar
PIN10
&
\sphinxAtStartPar
PIN10 开漏使能                              |

\sphinxAtStartPar
0:推挽模式                                 |

\sphinxAtStartPar
1:开漏模式                                 |
\\
\sphinxhline
\sphinxAtStartPar
9
&
\sphinxAtStartPar
PIN9
&
\sphinxAtStartPar
PIN9 开漏使能                               |

\sphinxAtStartPar
0:推挽模式                                 |

\sphinxAtStartPar
1:开漏模式                                 |
\\
\sphinxhline
\sphinxAtStartPar
8
&
\sphinxAtStartPar
PIN8
&
\sphinxAtStartPar
PIN8 开漏使能                               |

\sphinxAtStartPar
0:推挽模式                                 |

\sphinxAtStartPar
1:开漏模式                                 |
\\
\sphinxhline
\sphinxAtStartPar
7
&
\sphinxAtStartPar
PIN7
&
\sphinxAtStartPar
PIN7开漏使能                                |

\sphinxAtStartPar
0:推挽模式                                 |

\sphinxAtStartPar
1:开漏模式                                 |
\\
\sphinxhline
\sphinxAtStartPar
6
&
\sphinxAtStartPar
PIN6
&
\sphinxAtStartPar
PIN6开漏使能                                |

\sphinxAtStartPar
0:推挽模式                                 |

\sphinxAtStartPar
1:开漏模式                                 |
\\
\sphinxhline
\sphinxAtStartPar
5
&
\sphinxAtStartPar
PIN5
&
\sphinxAtStartPar
PIN5开漏使能                                |

\sphinxAtStartPar
0:推挽模式                                 |

\sphinxAtStartPar
1:开漏模式                                 |
\\
\sphinxhline
\sphinxAtStartPar
4
&
\sphinxAtStartPar
PIN4
&
\sphinxAtStartPar
PIN4开漏使能                                |

\sphinxAtStartPar
0:推挽模式                                 |

\sphinxAtStartPar
1:开漏模式                                 |
\\
\sphinxhline
\sphinxAtStartPar
3
&
\sphinxAtStartPar
PIN3
&
\sphinxAtStartPar
PIN3开漏使能                                |

\sphinxAtStartPar
0:推挽模式                                 |

\sphinxAtStartPar
1:开漏模式                                 |
\\
\sphinxhline
\sphinxAtStartPar
2
&
\sphinxAtStartPar
PIN2
&
\sphinxAtStartPar
PIN2开漏使能                                |

\sphinxAtStartPar
0:推挽模式                                 |

\sphinxAtStartPar
1:开漏模式                                 |
\\
\sphinxhline
\sphinxAtStartPar
1
&
\sphinxAtStartPar
PIN1
&
\sphinxAtStartPar
PIN1开漏使能                                |

\sphinxAtStartPar
0:推挽模式                                 |

\sphinxAtStartPar
1:开漏模式                                 |
\\
\sphinxhline
\sphinxAtStartPar
0
&
\sphinxAtStartPar
PIN0
&
\sphinxAtStartPar
PIN0开漏使能                                |

\sphinxAtStartPar
0:推挽模式                                 |

\sphinxAtStartPar
1:开漏模式                                 |
\\
\sphinxbottomrule
\end{tabular}
\sphinxtableafterendhook\par
\sphinxattableend\end{savenotes}

\sphinxstepscope


\section{通用I/O (GPIO)}
\label{\detokenize{SWM241/_u529f_u80fd_u63cf_u8ff0/_u901a_u7528IO:i-o-gpio}}\label{\detokenize{SWM241/_u529f_u80fd_u63cf_u8ff0/_u901a_u7528IO::doc}}
\sphinxAtStartPar
概述
\textasciitilde{}\textasciitilde{}

\sphinxAtStartPar
通用输入输出模块主要功能包括数据控制、中断控制功能。SWM241系列所有型号GPIO操作均相同。使用前需使能对应GPIO模块时钟。

\sphinxAtStartPar
特性
\textasciitilde{}\textasciitilde{}
\begin{itemize}
\item {} 
\sphinxAtStartPar
最高58个独立IO。

\item {} 
\sphinxAtStartPar
每个IO均支持位带功能

\item {} 
\sphinxAtStartPar
每个IO均可触发中断。

\item {} 
\sphinxAtStartPar
中断触发条件可配置,支持电平触发/边沿触发。
\begin{itemize}
\item {} 
\sphinxAtStartPar
电平触发支持高电平/低电平

\item {} 
\sphinxAtStartPar
边沿触发中断可配置为上升沿/下降沿/双边沿触发。

\end{itemize}

\item {} 
\sphinxAtStartPar
每个IO均支持上拉/下拉/推挽/开漏功能。

\end{itemize}


\subsection{功能描述}
\label{\detokenize{SWM241/_u529f_u80fd_u63cf_u8ff0/_u901a_u7528IO:id1}}

\subsubsection{数据控制}
\label{\detokenize{SWM241/_u529f_u80fd_u63cf_u8ff0/_u901a_u7528IO:id2}}
\sphinxAtStartPar
除SWD引脚与ISP引脚外,所有引脚上电后默认状态均为GPIO浮空输入(DIR = 0)。SWD引脚可在加密章节进行修改,ISP引脚默认下拉使能,保证浮空状态不会进入ISP模式。PA15专用于SLCD电源域控制,使用SLCD时需要设置输出高电平,不使用SLCD时需要设置为输出低电平。

\sphinxAtStartPar
GPIO方向寄存器(DIRx)用来将每个独立的管脚配置为输入模式或者输出模式:
\begin{itemize}
\item {} 
\sphinxAtStartPar
当数据方向设为0时,GPIO对应引脚配置为输入

\end{itemize}
\begin{quote}

\sphinxAtStartPar
通过读取相应数据寄存器(IDRx)对应位或对应DATAPINx寄存器获取指定GPIO端口当前状态值
\end{quote}
\begin{itemize}
\item {} 
\sphinxAtStartPar
当数据方向设为1时,GPIO对应引脚配置为输出

\end{itemize}
\begin{quote}

\sphinxAtStartPar
通过向对应端口数据寄存器(ODRx)对应位或对应DATAPINx寄存器写入值改变指定引脚输出,0输出低电平,1输出高电平。
\end{quote}


\subsubsection{中断控制与清除}
\label{\detokenize{SWM241/_u529f_u80fd_u63cf_u8ff0/_u901a_u7528IO:id3}}
\sphinxAtStartPar
可根据需求将GPIO端口对应引脚配置为中断模式,并通过相关寄存器配置中断极性及触发方式。触发方式分为边沿触发和电平触发两种模式。
\begin{itemize}
\item {} 
\sphinxAtStartPar
对于边沿触发中断,可以设置为上升沿触发,下降沿触发或双边沿触发。中断发生后,标志位具备保持特性,必须通过软件对中断标志位进行清除

\item {} 
\sphinxAtStartPar
对于电平触发中断,当外部引脚输入为指定电平时,中断发生。当电平翻转后,中断信号消失,无需软件进行清除。使用电平触发中断,需保证外部信号源保持电平稳定,以便有效中断电平能被端口识别

\end{itemize}

\sphinxAtStartPar
使用以下寄存器来对产生中断触发方式和极性进行定义:
\begin{itemize}
\item {} 
\sphinxAtStartPar
GPIO中断触发条件寄存器(INTLVLTRG),用于配置电平触发或边沿触发

\item {} 
\sphinxAtStartPar
GPIO中断触发极性寄存器(INTRISEEN),用于配置电平或边沿触发极性

\item {} 
\sphinxAtStartPar
GPIO中断边沿触发配置寄存器(INTBE),选择为边沿触发后,用于配置单边沿触发或双边沿触发

\end{itemize}

\sphinxAtStartPar
通过GPIO中断使能寄存器(INTEN)可以使能或者禁止相应端口对应位中断,GPIO原始中断状态(INTRAWSTAUS)不受使能位影响。当产生中断时,可以在GPIO原始中断状态(RAWINTSTAUS)获取中断信号的状态。当中断使能寄存器(INTEN)对应位为1时,中断状态(INTSTAUS)寄存
器可读取到对应中断信号,且中断信号会进入中断配置模块及NVIC模块,执行中断程序。

\sphinxAtStartPar
通过写1到GPIO中断清除寄存器(INTCLR)指定位可以清除相应位中断。


\subsection{寄存器映射}
\label{\detokenize{SWM241/_u529f_u80fd_u63cf_u8ff0/_u901a_u7528IO:id4}}

\begin{savenotes}\sphinxattablestart
\sphinxthistablewithglobalstyle
\centering
\begin{tabular}[t]{\X{20}{100}\X{20}{100}\X{20}{100}\X{20}{100}\X{20}{100}}
\sphinxtoprule
\sphinxtableatstartofbodyhook
\sphinxAtStartPar
名称   |
&
\begin{DUlineblock}{0em}
\item[] 偏移 |
\end{DUlineblock}
&
\begin{DUlineblock}{0em}
\item[] 
\item[] |
|
\end{DUlineblock}
&
\begin{DUlineblock}{0em}
\item[] 
\end{DUlineblock}
\begin{quote}

\begin{DUlineblock}{0em}
\item[] 
\item[] 
\end{DUlineblock}
\end{quote}
&
\sphinxAtStartPar
描述                       | | | |
\\
\sphinxhline
\sphinxAtStartPar
GPIOABASE:0 {\color{red}\bfseries{}|}x40040000GPIOBBASE:0 {\color{red}\bfseries{}|}x40040800GPIOCBASE:0 {\color{red}\bfseries{}|}x40041000GPIODBASE:0 {\color{red}\bfseries{}|}x40041800
&
\begin{DUlineblock}{0em}
\item[] |     |     |
\end{DUlineblock}
&&&\\
\sphinxhline
\sphinxAtStartPar
ODR
&
\sphinxAtStartPar
0x00
&&
\sphinxAtStartPar
0x 00000
&
\sphinxAtStartPar
GPIO写数据寄存器           |
\\
\sphinxhline
\sphinxAtStartPar
DIR
&
\sphinxAtStartPar
0x04
&&
\sphinxAtStartPar
0x 00000
&
\sphinxAtStartPar
GPIO方向寄存器             |
\\
\sphinxhline
\sphinxAtStartPar
INTLVLTRG
&
\sphinxAtStartPar
0x08
&&
\sphinxAtStartPar
0x 00000
&
\sphinxAtStartPar
GPIO中断触发条件           |
\\
\sphinxhline
\sphinxAtStartPar
INTBE
&
\sphinxAtStartPar
0x0c
&&
\sphinxAtStartPar
0x 00000
&
\sphinxAtStartPar
GPIO中断沿触发配置寄存器   |
\\
\sphinxhline
\sphinxAtStartPar
INTRISEEN
&
\sphinxAtStartPar
0x10
&&
\sphinxAtStartPar
0x 00000
&
\sphinxAtStartPar
GPIO 中断触发极性          |
\\
\sphinxhline
\sphinxAtStartPar
INTEN
&
\sphinxAtStartPar
0x14
&&
\sphinxAtStartPar
0x 00000
&
\sphinxAtStartPar
GPIO中断使能               |
\\
\sphinxhline
\sphinxAtStartPar
INTRAWSTAT
&
\sphinxAtStartPar
0x18
&&
\sphinxAtStartPar
0x 00000
&
\sphinxAtStartPar
GPIO中断原始状态           |
\\
\sphinxhline
\sphinxAtStartPar
INTSTAT
&
\sphinxAtStartPar
0x1c
&&
\sphinxAtStartPar
0x 00000
&
\sphinxAtStartPar
GPIO中断状态               |
\\
\sphinxhline
\sphinxAtStartPar
INTCLR
&
\sphinxAtStartPar
0x20
&&
\sphinxAtStartPar
0x 00000
&
\sphinxAtStartPar
GPIO中断清除               |
\\
\sphinxhline
\sphinxAtStartPar
DMAEN
&
\sphinxAtStartPar
0x24
&&
\sphinxAtStartPar
0x 00000
&
\sphinxAtStartPar
GPIO DMA 使能              |
\\
\sphinxhline
\sphinxAtStartPar
IDR
&
\sphinxAtStartPar
0x30
&&
\sphinxAtStartPar
0x 00000
&
\sphinxAtStartPar
GPIO读数据寄存器           |
\\
\sphinxhline
\sphinxAtStartPar
DATAPIN0
&
\sphinxAtStartPar
0x40
&&
\sphinxAtStartPar
0x 00000
&
\sphinxAtStartPar
GPIO PIN0数据寄存器        |
\\
\sphinxhline
\sphinxAtStartPar
DATAPIN1
&
\sphinxAtStartPar
0x44
&&
\sphinxAtStartPar
0x 00000
&
\sphinxAtStartPar
GPIO PIN1数据寄存器        |
\\
\sphinxhline
\sphinxAtStartPar
DATAPIN2
&
\sphinxAtStartPar
0x48
&&
\sphinxAtStartPar
0x 00000
&
\sphinxAtStartPar
GPIO PIN2数据寄存器        |
\\
\sphinxhline
\sphinxAtStartPar
DATAPIN3
&
\sphinxAtStartPar
0x4c
&&
\sphinxAtStartPar
0x 00000
&
\sphinxAtStartPar
GPIO PIN3数据寄存器        |
\\
\sphinxhline
\sphinxAtStartPar
DATAPIN4
&
\sphinxAtStartPar
0x50
&&
\sphinxAtStartPar
0x 00000
&
\sphinxAtStartPar
GPIO PIN4数据寄存器        |
\\
\sphinxhline
\sphinxAtStartPar
DATAPIN5
&
\sphinxAtStartPar
0x54
&&
\sphinxAtStartPar
0x 00000
&
\sphinxAtStartPar
GPIO PIN5数据寄存器        |
\\
\sphinxhline
\sphinxAtStartPar
DATAPIN6
&
\sphinxAtStartPar
0x58
&&
\sphinxAtStartPar
0x 00000
&
\sphinxAtStartPar
GPIO PIN6数据寄存器        |
\\
\sphinxhline
\sphinxAtStartPar
DATAPIN7
&
\sphinxAtStartPar
0x5c
&&
\sphinxAtStartPar
0x 00000
&
\sphinxAtStartPar
GPIO PIN7数据寄存器        |
\\
\sphinxhline
\sphinxAtStartPar
DATAPIN8
&
\sphinxAtStartPar
0x60
&&
\sphinxAtStartPar
0x 00000
&
\sphinxAtStartPar
GPIO PIN8数据寄存器        |
\\
\sphinxhline
\sphinxAtStartPar
DATAPIN9
&
\sphinxAtStartPar
0x64
&&
\sphinxAtStartPar
0x 00000
&
\sphinxAtStartPar
GPIO PIN9数据寄存器        |
\\
\sphinxhline
\sphinxAtStartPar
DATAPIN10
&
\sphinxAtStartPar
0x68
&&
\sphinxAtStartPar
0x 00000
&
\sphinxAtStartPar
GPIO PIN10数据寄存器       |
\\
\sphinxhline
\sphinxAtStartPar
DATAPIN11
&
\sphinxAtStartPar
0x6c
&&
\sphinxAtStartPar
0x 00000
&
\sphinxAtStartPar
GPIO PIN11数据寄存器       |
\\
\sphinxhline
\sphinxAtStartPar
DATAPIN12
&
\sphinxAtStartPar
0x70
&&
\sphinxAtStartPar
0x 00000
&
\sphinxAtStartPar
GPIO PIN12数据寄存器       |
\\
\sphinxhline
\sphinxAtStartPar
DATAPIN13
&
\sphinxAtStartPar
0x74
&&
\sphinxAtStartPar
0x 00000
&
\sphinxAtStartPar
GPIO PIN13数据寄存器       |
\\
\sphinxhline
\sphinxAtStartPar
DATAPIN14
&
\sphinxAtStartPar
0x78
&&
\sphinxAtStartPar
0x 00000
&
\sphinxAtStartPar
GPIO PIN14数据寄存器       |
\\
\sphinxhline
\sphinxAtStartPar
DATAPIN15
&
\sphinxAtStartPar
0x7c
&&
\sphinxAtStartPar
0x 00000
&
\sphinxAtStartPar
GPIO PIN15数据寄存器       |
\\
\sphinxbottomrule
\end{tabular}
\sphinxtableafterendhook\par
\sphinxattableend\end{savenotes}


\subsection{寄存器描述}
\label{\detokenize{SWM241/_u529f_u80fd_u63cf_u8ff0/_u901a_u7528IO:id13}}

\subsubsection{GPIOx写数据寄存器ODR}
\label{\detokenize{SWM241/_u529f_u80fd_u63cf_u8ff0/_u901a_u7528IO:gpioxodr}}

\begin{savenotes}\sphinxattablestart
\sphinxthistablewithglobalstyle
\centering
\begin{tabular}[t]{\X{20}{100}\X{20}{100}\X{20}{100}\X{20}{100}\X{20}{100}}
\sphinxtoprule
\sphinxtableatstartofbodyhook
\sphinxAtStartPar
寄存器 |
&
\begin{DUlineblock}{0em}
\item[] 偏移 |
\end{DUlineblock}
&
\begin{DUlineblock}{0em}
\item[] 
\item[] {\color{red}\bfseries{}|}
\end{DUlineblock}
&
\sphinxAtStartPar
复位值 |    描 | |
&
\begin{DUlineblock}{0em}
\item[] |
  |
\end{DUlineblock}
\\
\sphinxhline
\sphinxAtStartPar
ODR
&
\sphinxAtStartPar
0x00
&&
\sphinxAtStartPar
0 000000
&
\sphinxAtStartPar
GPIO写数据寄存器           |
\\
\sphinxbottomrule
\end{tabular}
\sphinxtableafterendhook\par
\sphinxattableend\end{savenotes}


\begin{savenotes}\sphinxattablestart
\sphinxthistablewithglobalstyle
\centering
\begin{tabular}[t]{\X{12}{96}\X{12}{96}\X{12}{96}\X{12}{96}\X{12}{96}\X{12}{96}\X{12}{96}\X{12}{96}}
\sphinxtoprule
\sphinxtableatstartofbodyhook
\sphinxAtStartPar
31
&
\sphinxAtStartPar
30
&
\sphinxAtStartPar
29
&
\sphinxAtStartPar
28
&
\sphinxAtStartPar
27
&
\sphinxAtStartPar
26
&
\sphinxAtStartPar
25
&
\sphinxAtStartPar
24
\\
\sphinxhline\begin{itemize}
\item {} 
\end{itemize}
&&&&&&&\\
\sphinxhline
\sphinxAtStartPar
23
&
\sphinxAtStartPar
22
&
\sphinxAtStartPar
21
&
\sphinxAtStartPar
20
&
\sphinxAtStartPar
19
&
\sphinxAtStartPar
18
&
\sphinxAtStartPar
17
&
\sphinxAtStartPar
16
\\
\sphinxhline\begin{itemize}
\item {} 
\end{itemize}
&&&&&&&\\
\sphinxhline
\sphinxAtStartPar
15
&
\sphinxAtStartPar
14
&
\sphinxAtStartPar
13
&
\sphinxAtStartPar
12
&
\sphinxAtStartPar
11
&
\sphinxAtStartPar
10
&
\sphinxAtStartPar
9
&
\sphinxAtStartPar
8
\\
\sphinxhline
\sphinxAtStartPar
PIN15
&
\sphinxAtStartPar
PIN14
&&&&&&\\
\sphinxhline
\sphinxAtStartPar
7
&
\sphinxAtStartPar
6
&
\sphinxAtStartPar
5
&
\sphinxAtStartPar
4
&
\sphinxAtStartPar
3
&
\sphinxAtStartPar
2
&
\sphinxAtStartPar
1
&
\sphinxAtStartPar
0
\\
\sphinxhline
\sphinxAtStartPar
PIN7
&
\sphinxAtStartPar
PIN6
&&&&&&\\
\sphinxbottomrule
\end{tabular}
\sphinxtableafterendhook\par
\sphinxattableend\end{savenotes}


\begin{savenotes}\sphinxattablestart
\sphinxthistablewithglobalstyle
\centering
\begin{tabular}[t]{\X{33}{99}\X{33}{99}\X{33}{99}}
\sphinxtoprule
\sphinxtableatstartofbodyhook
\sphinxAtStartPar
位域 |
&
\sphinxAtStartPar
名称     | |
&
\sphinxAtStartPar
描述                                        | |
\\
\sphinxhline
\sphinxAtStartPar
31:16
&\begin{itemize}
\item {} 
\end{itemize}
&\begin{itemize}
\item {} 
\end{itemize}
\\
\sphinxhline
\sphinxAtStartPar
15
&
\sphinxAtStartPar
PIN15
&
\sphinxAtStartPar
Px15引脚数据写寄存器位                      |
\\
\sphinxhline
\sphinxAtStartPar
14
&
\sphinxAtStartPar
PIN14
&
\sphinxAtStartPar
Px14引脚数据写寄存器位                      |
\\
\sphinxhline
\sphinxAtStartPar
13
&
\sphinxAtStartPar
PIN13
&
\sphinxAtStartPar
Px13引脚数据写寄存器位                      |
\\
\sphinxhline
\sphinxAtStartPar
12
&
\sphinxAtStartPar
PIN12
&
\sphinxAtStartPar
Px12引脚数据写寄存器位                      |
\\
\sphinxhline
\sphinxAtStartPar
11
&
\sphinxAtStartPar
PIN11
&
\sphinxAtStartPar
Px11引脚数据写寄存器位                      |
\\
\sphinxhline
\sphinxAtStartPar
10
&
\sphinxAtStartPar
PIN10
&
\sphinxAtStartPar
Px10引脚数据写寄存器位                      |
\\
\sphinxhline
\sphinxAtStartPar
9
&
\sphinxAtStartPar
PIN9
&
\sphinxAtStartPar
Px9引脚数据写寄存器位                       |
\\
\sphinxhline
\sphinxAtStartPar
8
&
\sphinxAtStartPar
PIN8
&
\sphinxAtStartPar
Px8引脚数据写寄存器位                       |
\\
\sphinxhline
\sphinxAtStartPar
7
&
\sphinxAtStartPar
PIN7
&
\sphinxAtStartPar
Px7引脚数据写寄存器位                       |
\\
\sphinxhline
\sphinxAtStartPar
6
&
\sphinxAtStartPar
PIN6
&
\sphinxAtStartPar
Px6引脚数据写寄存器位                       |
\\
\sphinxhline
\sphinxAtStartPar
5
&
\sphinxAtStartPar
PIN5
&
\sphinxAtStartPar
Px5引脚数据写寄存器位                       |
\\
\sphinxhline
\sphinxAtStartPar
4
&
\sphinxAtStartPar
PIN4
&
\sphinxAtStartPar
Px4引脚数据写寄存器位                       |
\\
\sphinxhline
\sphinxAtStartPar
3
&
\sphinxAtStartPar
PIN3
&
\sphinxAtStartPar
Px3引脚数据写寄存器位                       |
\\
\sphinxhline
\sphinxAtStartPar
2
&
\sphinxAtStartPar
PIN2
&
\sphinxAtStartPar
Px2引脚数据写寄存器位                       |
\\
\sphinxhline
\sphinxAtStartPar
1
&
\sphinxAtStartPar
PIN1
&
\sphinxAtStartPar
Px1引脚数据写寄存器位                       |
\\
\sphinxhline
\sphinxAtStartPar
0
&
\sphinxAtStartPar
PIN0
&
\sphinxAtStartPar
Px0引脚数据写寄存器位                       |
\\
\sphinxbottomrule
\end{tabular}
\sphinxtableafterendhook\par
\sphinxattableend\end{savenotes}


\subsubsection{GPIOx方向寄存器DIR}
\label{\detokenize{SWM241/_u529f_u80fd_u63cf_u8ff0/_u901a_u7528IO:gpioxdir}}

\begin{savenotes}\sphinxattablestart
\sphinxthistablewithglobalstyle
\centering
\begin{tabular}[t]{\X{20}{100}\X{20}{100}\X{20}{100}\X{20}{100}\X{20}{100}}
\sphinxtoprule
\sphinxtableatstartofbodyhook
\sphinxAtStartPar
寄存器 |
&
\begin{DUlineblock}{0em}
\item[] 偏移 |
\end{DUlineblock}
&
\begin{DUlineblock}{0em}
\item[] 
\item[] {\color{red}\bfseries{}|}
\end{DUlineblock}
&
\sphinxAtStartPar
复位值 |    描 | |
&
\begin{DUlineblock}{0em}
\item[] |
  |
\end{DUlineblock}
\\
\sphinxhline
\sphinxAtStartPar
DIR
&
\sphinxAtStartPar
0x04
&&
\sphinxAtStartPar
0 000000
&
\sphinxAtStartPar
GPIO方向寄存器             |
\\
\sphinxbottomrule
\end{tabular}
\sphinxtableafterendhook\par
\sphinxattableend\end{savenotes}


\begin{savenotes}\sphinxattablestart
\sphinxthistablewithglobalstyle
\centering
\begin{tabular}[t]{\X{12}{96}\X{12}{96}\X{12}{96}\X{12}{96}\X{12}{96}\X{12}{96}\X{12}{96}\X{12}{96}}
\sphinxtoprule
\sphinxtableatstartofbodyhook
\sphinxAtStartPar
31
&
\sphinxAtStartPar
30
&
\sphinxAtStartPar
29
&
\sphinxAtStartPar
28
&
\sphinxAtStartPar
27
&
\sphinxAtStartPar
26
&
\sphinxAtStartPar
25
&
\sphinxAtStartPar
24
\\
\sphinxhline\begin{itemize}
\item {} 
\end{itemize}
&&&&&&&\\
\sphinxhline
\sphinxAtStartPar
23
&
\sphinxAtStartPar
22
&
\sphinxAtStartPar
21
&
\sphinxAtStartPar
20
&
\sphinxAtStartPar
19
&
\sphinxAtStartPar
18
&
\sphinxAtStartPar
17
&
\sphinxAtStartPar
16
\\
\sphinxhline\begin{itemize}
\item {} 
\end{itemize}
&&&&&&&\\
\sphinxhline
\sphinxAtStartPar
15
&
\sphinxAtStartPar
14
&
\sphinxAtStartPar
13
&
\sphinxAtStartPar
12
&
\sphinxAtStartPar
11
&
\sphinxAtStartPar
10
&
\sphinxAtStartPar
9
&
\sphinxAtStartPar
8
\\
\sphinxhline
\sphinxAtStartPar
PIN15
&
\sphinxAtStartPar
PIN14
&&&&&&\\
\sphinxhline
\sphinxAtStartPar
7
&
\sphinxAtStartPar
6
&
\sphinxAtStartPar
5
&
\sphinxAtStartPar
4
&
\sphinxAtStartPar
3
&
\sphinxAtStartPar
2
&
\sphinxAtStartPar
1
&
\sphinxAtStartPar
0
\\
\sphinxhline
\sphinxAtStartPar
PIN7
&
\sphinxAtStartPar
PIN6
&&&&&&\\
\sphinxbottomrule
\end{tabular}
\sphinxtableafterendhook\par
\sphinxattableend\end{savenotes}


\begin{savenotes}\sphinxattablestart
\sphinxthistablewithglobalstyle
\centering
\begin{tabular}[t]{\X{33}{99}\X{33}{99}\X{33}{99}}
\sphinxtoprule
\sphinxtableatstartofbodyhook
\sphinxAtStartPar
位域 |
&
\sphinxAtStartPar
名称     | |
&
\sphinxAtStartPar
描述                                        | |
\\
\sphinxhline
\sphinxAtStartPar
31:16
&\begin{itemize}
\item {} 
\end{itemize}
&\begin{itemize}
\item {} 
\end{itemize}
\\
\sphinxhline
\sphinxAtStartPar
15
&
\sphinxAtStartPar
PIN15
&
\sphinxAtStartPar
Px15引脚方向寄存器位                        |

\sphinxAtStartPar
1:输出                                     |

\sphinxAtStartPar
0:输入                                     |
\\
\sphinxhline
\sphinxAtStartPar
14
&
\sphinxAtStartPar
PIN14
&
\sphinxAtStartPar
Px14引脚方向寄存器位                        |

\sphinxAtStartPar
1:输出                                     |

\sphinxAtStartPar
0:输入                                     |
\\
\sphinxhline
\sphinxAtStartPar
13
&
\sphinxAtStartPar
PIN13
&
\sphinxAtStartPar
Px13引脚方向寄存器位                        |

\sphinxAtStartPar
1:输出                                     |

\sphinxAtStartPar
0:输入                                     |
\\
\sphinxhline
\sphinxAtStartPar
12
&
\sphinxAtStartPar
PIN12
&
\sphinxAtStartPar
Px12引脚方向寄存器位                        |

\sphinxAtStartPar
1:输出                                     |

\sphinxAtStartPar
0:输入                                     |
\\
\sphinxhline
\sphinxAtStartPar
11
&
\sphinxAtStartPar
PIN11
&
\sphinxAtStartPar
Px11引脚方向寄存器位                        |

\sphinxAtStartPar
1:输出                                     |

\sphinxAtStartPar
0:输入                                     |
\\
\sphinxhline
\sphinxAtStartPar
10
&
\sphinxAtStartPar
PIN10
&
\sphinxAtStartPar
Px10引脚方向寄存器位                        |

\sphinxAtStartPar
1:输出                                     |

\sphinxAtStartPar
0:输入                                     |
\\
\sphinxhline
\sphinxAtStartPar
9
&
\sphinxAtStartPar
PIN9
&
\sphinxAtStartPar
Px9引脚方向寄存器位                         |

\sphinxAtStartPar
1:输出                                     |

\sphinxAtStartPar
0:输入                                     |
\\
\sphinxhline
\sphinxAtStartPar
8
&
\sphinxAtStartPar
PIN8
&
\sphinxAtStartPar
Px8引脚方向寄存器位                         |

\sphinxAtStartPar
1:输出                                     |

\sphinxAtStartPar
0:输入                                     |
\\
\sphinxhline
\sphinxAtStartPar
7
&
\sphinxAtStartPar
PIN7
&
\sphinxAtStartPar
Px7引脚方向寄存器位                         |

\sphinxAtStartPar
1:输出                                     |

\sphinxAtStartPar
0:输入                                     |
\\
\sphinxhline
\sphinxAtStartPar
6
&
\sphinxAtStartPar
PIN6
&
\sphinxAtStartPar
Px6引脚方向寄存器位                         |

\sphinxAtStartPar
1:输出                                     |

\sphinxAtStartPar
0:输入                                     |
\\
\sphinxhline
\sphinxAtStartPar
5
&
\sphinxAtStartPar
PIN5
&
\sphinxAtStartPar
Px5引脚方向寄存器位                         |

\sphinxAtStartPar
1:输出                                     |

\sphinxAtStartPar
0:输入                                     |
\\
\sphinxhline
\sphinxAtStartPar
4
&
\sphinxAtStartPar
PIN4
&
\sphinxAtStartPar
Px4引脚方向寄存器位                         |

\sphinxAtStartPar
1:输出                                     |

\sphinxAtStartPar
0:输入                                     |
\\
\sphinxhline
\sphinxAtStartPar
3
&
\sphinxAtStartPar
PIN3
&
\sphinxAtStartPar
Px3引脚方向寄存器位                         |

\sphinxAtStartPar
1:输出                                     |

\sphinxAtStartPar
0:输入                                     |
\\
\sphinxhline
\sphinxAtStartPar
2
&
\sphinxAtStartPar
PIN2
&
\sphinxAtStartPar
Px2引脚方向寄存器位                         |

\sphinxAtStartPar
1:输出                                     |

\sphinxAtStartPar
0:输入                                     |
\\
\sphinxhline
\sphinxAtStartPar
1
&
\sphinxAtStartPar
PIN1
&
\sphinxAtStartPar
Px1引脚方向寄存器位                         |

\sphinxAtStartPar
1:输出                                     |

\sphinxAtStartPar
0:输入                                     |
\\
\sphinxhline
\sphinxAtStartPar
0
&
\sphinxAtStartPar
PIN0
&
\sphinxAtStartPar
Px0引脚方向寄存器位                         |

\sphinxAtStartPar
1:输出                                     |

\sphinxAtStartPar
0:输入                                     |
\\
\sphinxbottomrule
\end{tabular}
\sphinxtableafterendhook\par
\sphinxattableend\end{savenotes}


\subsubsection{GPIOx中断触发条件寄存器INTLVLTRG}
\label{\detokenize{SWM241/_u529f_u80fd_u63cf_u8ff0/_u901a_u7528IO:gpioxintlvltrg}}

\begin{savenotes}\sphinxattablestart
\sphinxthistablewithglobalstyle
\centering
\begin{tabular}[t]{\X{20}{100}\X{20}{100}\X{20}{100}\X{20}{100}\X{20}{100}}
\sphinxtoprule
\sphinxtableatstartofbodyhook
\sphinxAtStartPar
寄存器 |
&
\begin{DUlineblock}{0em}
\item[] 偏移 |
\end{DUlineblock}
&
\begin{DUlineblock}{0em}
\item[] 
\item[] {\color{red}\bfseries{}|}
\end{DUlineblock}
&
\sphinxAtStartPar
复位值 |    描 | |
&
\begin{DUlineblock}{0em}
\item[] |
  |
\end{DUlineblock}
\\
\sphinxhline
\sphinxAtStartPar
INTLVLTRG
&
\sphinxAtStartPar
0x08
&&
\sphinxAtStartPar
0 000000
&
\sphinxAtStartPar
GPIO中断触发方式           |
\\
\sphinxbottomrule
\end{tabular}
\sphinxtableafterendhook\par
\sphinxattableend\end{savenotes}


\begin{savenotes}\sphinxattablestart
\sphinxthistablewithglobalstyle
\centering
\begin{tabular}[t]{\X{12}{96}\X{12}{96}\X{12}{96}\X{12}{96}\X{12}{96}\X{12}{96}\X{12}{96}\X{12}{96}}
\sphinxtoprule
\sphinxtableatstartofbodyhook
\sphinxAtStartPar
31
&
\sphinxAtStartPar
30
&
\sphinxAtStartPar
29
&
\sphinxAtStartPar
28
&
\sphinxAtStartPar
27
&
\sphinxAtStartPar
26
&
\sphinxAtStartPar
25
&
\sphinxAtStartPar
24
\\
\sphinxhline\begin{itemize}
\item {} 
\end{itemize}
&&&&&&&\\
\sphinxhline
\sphinxAtStartPar
23
&
\sphinxAtStartPar
22
&
\sphinxAtStartPar
21
&
\sphinxAtStartPar
20
&
\sphinxAtStartPar
19
&
\sphinxAtStartPar
18
&
\sphinxAtStartPar
17
&
\sphinxAtStartPar
16
\\
\sphinxhline\begin{itemize}
\item {} 
\end{itemize}
&&&&&&&\\
\sphinxhline
\sphinxAtStartPar
15
&
\sphinxAtStartPar
14
&
\sphinxAtStartPar
13
&
\sphinxAtStartPar
12
&
\sphinxAtStartPar
11
&
\sphinxAtStartPar
10
&
\sphinxAtStartPar
9
&
\sphinxAtStartPar
8
\\
\sphinxhline
\sphinxAtStartPar
PIN15
&
\sphinxAtStartPar
PIN14
&&&&&&\\
\sphinxhline
\sphinxAtStartPar
7
&
\sphinxAtStartPar
6
&
\sphinxAtStartPar
5
&
\sphinxAtStartPar
4
&
\sphinxAtStartPar
3
&
\sphinxAtStartPar
2
&
\sphinxAtStartPar
1
&
\sphinxAtStartPar
0
\\
\sphinxhline
\sphinxAtStartPar
PIN7
&
\sphinxAtStartPar
PIN6
&&&&&&\\
\sphinxbottomrule
\end{tabular}
\sphinxtableafterendhook\par
\sphinxattableend\end{savenotes}


\begin{savenotes}\sphinxattablestart
\sphinxthistablewithglobalstyle
\centering
\begin{tabular}[t]{\X{33}{99}\X{33}{99}\X{33}{99}}
\sphinxtoprule
\sphinxtableatstartofbodyhook
\sphinxAtStartPar
位域 |
&
\sphinxAtStartPar
名称     | |
&
\sphinxAtStartPar
描述                                        | |
\\
\sphinxhline
\sphinxAtStartPar
31:16
&\begin{itemize}
\item {} 
\end{itemize}
&\begin{itemize}
\item {} 
\end{itemize}
\\
\sphinxhline
\sphinxAtStartPar
15
&
\sphinxAtStartPar
PIN15
&
\sphinxAtStartPar
Px15引脚中断敏感条件寄存器位                |

\sphinxAtStartPar
1:电平检测                                 |

\sphinxAtStartPar
0:边沿检测                                 |
\\
\sphinxhline
\sphinxAtStartPar
14
&
\sphinxAtStartPar
PIN14
&
\sphinxAtStartPar
Px14引脚中断敏感条件寄存器位                |

\sphinxAtStartPar
1:电平检测                                 |

\sphinxAtStartPar
0:边沿检测                                 |
\\
\sphinxhline
\sphinxAtStartPar
13
&
\sphinxAtStartPar
PIN13
&
\sphinxAtStartPar
Px13引脚中断敏感条件寄存器位                |

\sphinxAtStartPar
1:电平检测                                 |

\sphinxAtStartPar
0:边沿检测                                 |
\\
\sphinxhline
\sphinxAtStartPar
12
&
\sphinxAtStartPar
PIN12
&
\sphinxAtStartPar
Px12引脚中断敏感条件寄存器位                |

\sphinxAtStartPar
1:电平检测                                 |

\sphinxAtStartPar
0:边沿检测                                 |
\\
\sphinxhline
\sphinxAtStartPar
11
&
\sphinxAtStartPar
PIN11
&
\sphinxAtStartPar
Px11引脚中断敏感条件寄存器位                |

\sphinxAtStartPar
1:电平检测                                 |

\sphinxAtStartPar
0:边沿检测                                 |
\\
\sphinxhline
\sphinxAtStartPar
10
&
\sphinxAtStartPar
PIN10
&
\sphinxAtStartPar
Px10引脚中断敏感条件寄存器位                |

\sphinxAtStartPar
1:电平检测                                 |

\sphinxAtStartPar
0:边沿检测                                 |
\\
\sphinxhline
\sphinxAtStartPar
9
&
\sphinxAtStartPar
PIN9
&
\sphinxAtStartPar
Px9引脚中断敏感条件寄存器位                 |

\sphinxAtStartPar
1:电平检测                                 |

\sphinxAtStartPar
0:边沿检测                                 |
\\
\sphinxhline
\sphinxAtStartPar
8
&
\sphinxAtStartPar
PIN8
&
\sphinxAtStartPar
Px8引脚中断敏感条件寄存器位                 |

\sphinxAtStartPar
1:电平检测                                 |

\sphinxAtStartPar
0:边沿检测                                 |
\\
\sphinxhline
\sphinxAtStartPar
7
&
\sphinxAtStartPar
PIN7
&
\sphinxAtStartPar
Px7引脚中断敏感条件寄存器位                 |

\sphinxAtStartPar
1:电平检测                                 |

\sphinxAtStartPar
0:边沿检测                                 |
\\
\sphinxhline
\sphinxAtStartPar
6
&
\sphinxAtStartPar
PIN6
&
\sphinxAtStartPar
Px6引脚中断敏感条件寄存器位                 |

\sphinxAtStartPar
1:电平检测                                 |

\sphinxAtStartPar
0:边沿检测                                 |
\\
\sphinxhline
\sphinxAtStartPar
5
&
\sphinxAtStartPar
PIN5
&
\sphinxAtStartPar
Px5引脚中断敏感条件寄存器位                 |

\sphinxAtStartPar
1:电平检测                                 |

\sphinxAtStartPar
0:边沿检测                                 |
\\
\sphinxhline
\sphinxAtStartPar
4
&
\sphinxAtStartPar
PIN4
&
\sphinxAtStartPar
Px4引脚中断敏感条件寄存器位                 |

\sphinxAtStartPar
1:电平检测                                 |

\sphinxAtStartPar
0:边沿检测                                 |
\\
\sphinxhline
\sphinxAtStartPar
3
&
\sphinxAtStartPar
PIN3
&
\sphinxAtStartPar
Px3引脚中断敏感条件寄存器位                 |

\sphinxAtStartPar
1:电平检测                                 |

\sphinxAtStartPar
0:边沿检测                                 |
\\
\sphinxhline
\sphinxAtStartPar
2
&
\sphinxAtStartPar
PIN2
&
\sphinxAtStartPar
Px2引脚中断敏感条件寄存器位                 |

\sphinxAtStartPar
1:电平检测                                 |

\sphinxAtStartPar
0:边沿检测                                 |
\\
\sphinxhline
\sphinxAtStartPar
1
&
\sphinxAtStartPar
PIN1
&
\sphinxAtStartPar
Px1引脚中断敏感条件寄存器位                 |

\sphinxAtStartPar
1:电平检测                                 |

\sphinxAtStartPar
0:边沿检测                                 |
\\
\sphinxhline
\sphinxAtStartPar
0
&
\sphinxAtStartPar
PIN0
&
\sphinxAtStartPar
Px0引脚中断敏感条件寄存器位                 |

\sphinxAtStartPar
1:电平检测                                 |

\sphinxAtStartPar
0:边沿检测                                 |
\\
\sphinxbottomrule
\end{tabular}
\sphinxtableafterendhook\par
\sphinxattableend\end{savenotes}


\subsubsection{GPIOx中断沿触发配置寄存器INTBE}
\label{\detokenize{SWM241/_u529f_u80fd_u63cf_u8ff0/_u901a_u7528IO:gpioxintbe}}

\begin{savenotes}\sphinxattablestart
\sphinxthistablewithglobalstyle
\centering
\begin{tabular}[t]{\X{20}{100}\X{20}{100}\X{20}{100}\X{20}{100}\X{20}{100}}
\sphinxtoprule
\sphinxtableatstartofbodyhook
\sphinxAtStartPar
寄存器 |
&
\begin{DUlineblock}{0em}
\item[] 偏移 |
\end{DUlineblock}
&
\begin{DUlineblock}{0em}
\item[] 
\item[] {\color{red}\bfseries{}|}
\end{DUlineblock}
&
\sphinxAtStartPar
复位值 |    描 | |
&
\begin{DUlineblock}{0em}
\item[] |
  |
\end{DUlineblock}
\\
\sphinxhline
\sphinxAtStartPar
INTBE
&
\sphinxAtStartPar
0x0c
&&
\sphinxAtStartPar
0 000000
&
\sphinxAtStartPar
GPIOx中断沿触发配置寄存器  |
\\
\sphinxbottomrule
\end{tabular}
\sphinxtableafterendhook\par
\sphinxattableend\end{savenotes}


\begin{savenotes}\sphinxattablestart
\sphinxthistablewithglobalstyle
\centering
\begin{tabular}[t]{\X{12}{96}\X{12}{96}\X{12}{96}\X{12}{96}\X{12}{96}\X{12}{96}\X{12}{96}\X{12}{96}}
\sphinxtoprule
\sphinxtableatstartofbodyhook
\sphinxAtStartPar
31
&
\sphinxAtStartPar
30
&
\sphinxAtStartPar
29
&
\sphinxAtStartPar
28
&
\sphinxAtStartPar
27
&
\sphinxAtStartPar
26
&
\sphinxAtStartPar
25
&
\sphinxAtStartPar
24
\\
\sphinxhline\begin{itemize}
\item {} 
\end{itemize}
&&&&&&&\\
\sphinxhline
\sphinxAtStartPar
23
&
\sphinxAtStartPar
22
&
\sphinxAtStartPar
21
&
\sphinxAtStartPar
20
&
\sphinxAtStartPar
19
&
\sphinxAtStartPar
18
&
\sphinxAtStartPar
17
&
\sphinxAtStartPar
16
\\
\sphinxhline\begin{itemize}
\item {} 
\end{itemize}
&&&&&&&\\
\sphinxhline
\sphinxAtStartPar
15
&
\sphinxAtStartPar
14
&
\sphinxAtStartPar
13
&
\sphinxAtStartPar
12
&
\sphinxAtStartPar
11
&
\sphinxAtStartPar
10
&
\sphinxAtStartPar
9
&
\sphinxAtStartPar
8
\\
\sphinxhline
\sphinxAtStartPar
PIN15
&
\sphinxAtStartPar
PIN14
&&&&&&\\
\sphinxhline
\sphinxAtStartPar
7
&
\sphinxAtStartPar
6
&
\sphinxAtStartPar
5
&
\sphinxAtStartPar
4
&
\sphinxAtStartPar
3
&
\sphinxAtStartPar
2
&
\sphinxAtStartPar
1
&
\sphinxAtStartPar
0
\\
\sphinxhline
\sphinxAtStartPar
PIN7
&
\sphinxAtStartPar
PIN6
&&&&&&\\
\sphinxbottomrule
\end{tabular}
\sphinxtableafterendhook\par
\sphinxattableend\end{savenotes}


\begin{savenotes}\sphinxattablestart
\sphinxthistablewithglobalstyle
\centering
\begin{tabular}[t]{\X{33}{99}\X{33}{99}\X{33}{99}}
\sphinxtoprule
\sphinxtableatstartofbodyhook
\sphinxAtStartPar
位域 |
&
\sphinxAtStartPar
名称     | |
&
\sphinxAtStartPar
描述                                        | |
\\
\sphinxhline
\sphinxAtStartPar
31:16
&\begin{itemize}
\item {} 
\end{itemize}
&\begin{itemize}
\item {} 
\end{itemize}
\\
\sphinxhline
\sphinxAtStartPar
15
&
\sphinxAtStartPar
PIN15
&
\sphinxAtStartPar
Px15引脚中断沿触发配置寄存器位              |

\sphinxAtStartPar
1:相应位                                   | 沿触发中断,即上升沿和下降沿都会触发中断 |

\sphinxAtStartPar
0:相应位为单边沿触发中断,                 | TRISEEN寄存器相应位确定是上升沿/下降沿触发 |
\\
\sphinxhline
\sphinxAtStartPar
14
&
\sphinxAtStartPar
PIN14
&
\sphinxAtStartPar
Px14引脚中断沿触发配置寄存器位              |

\sphinxAtStartPar
1:相应位                                   | 沿触发中断,即上升沿和下降沿都会触发中断 |

\sphinxAtStartPar
0:相应位为单边沿触发中断,                 | TRISEEN寄存器相应位确定是上升沿/下降沿触发 |
\\
\sphinxhline
\sphinxAtStartPar
13
&
\sphinxAtStartPar
PIN13
&
\sphinxAtStartPar
Px13引脚中断沿触发配置寄存器位              |

\sphinxAtStartPar
1:相应位                                   | 沿触发中断,即上升沿和下降沿都会触发中断 |

\sphinxAtStartPar
0:相应位为单边沿触发中断,                 | TRISEEN寄存器相应位确定是上升沿/下降沿触发 |
\\
\sphinxhline
\sphinxAtStartPar
12
&
\sphinxAtStartPar
PIN12
&
\sphinxAtStartPar
Px12引脚中断沿触发配置寄存器位              |

\sphinxAtStartPar
1:相应位                                   | 沿触发中断,即上升沿和下降沿都会触发中断 |

\sphinxAtStartPar
0:相应位为单边沿触发中断,                 | TRISEEN寄存器相应位确定是上升沿/下降沿触发 |
\\
\sphinxhline
\sphinxAtStartPar
11
&
\sphinxAtStartPar
PIN11
&
\sphinxAtStartPar
Px11引脚中断沿触发配置寄存器位              |

\sphinxAtStartPar
1:相应位                                   | 沿触发中断,即上升沿和下降沿都会触发中断 |

\sphinxAtStartPar
0:相应位为单边沿触发中断,                 | TRISEEN寄存器相应位确定是上升沿/下降沿触发 |
\\
\sphinxhline
\sphinxAtStartPar
10
&
\sphinxAtStartPar
PIN10
&
\sphinxAtStartPar
Px10引脚中断沿触发配置寄存器位              |

\sphinxAtStartPar
1:相应位                                   | 沿触发中断,即上升沿和下降沿都会触发中断 |

\sphinxAtStartPar
0:相应位为单边沿触发中断,                 | TRISEEN寄存器相应位确定是上升沿/下降沿触发 |
\\
\sphinxhline
\sphinxAtStartPar
9
&
\sphinxAtStartPar
PIN9
&
\sphinxAtStartPar
Px9引脚中断沿触发配置寄存器位               |

\sphinxAtStartPar
1:相应位                                   | 沿触发中断,即上升沿和下降沿都会触发中断 |

\sphinxAtStartPar
0:相应位为单边沿触发中断,                 | TRISEEN寄存器相应位确定是上升沿/下降沿触发 |
\\
\sphinxhline
\sphinxAtStartPar
8
&
\sphinxAtStartPar
PIN8
&
\sphinxAtStartPar
Px8引脚中断沿触发配置寄存器位               |

\sphinxAtStartPar
1:相应位                                   | 沿触发中断,即上升沿和下降沿都会触发中断 |

\sphinxAtStartPar
0:相应位为单边沿触发中断,                 | TRISEEN寄存器相应位确定是上升沿/下降沿触发 |
\\
\sphinxhline
\sphinxAtStartPar
7
&
\sphinxAtStartPar
PIN7
&
\sphinxAtStartPar
Px7引脚中断沿触发配置寄存器位               |

\sphinxAtStartPar
1:相应位                                   | 沿触发中断,即上升沿和下降沿都会触发中断 |

\sphinxAtStartPar
0:相应位为单边沿触发中断,                 | TRISEEN寄存器相应位确定是上升沿/下降沿触发 |
\\
\sphinxhline
\sphinxAtStartPar
6
&
\sphinxAtStartPar
PIN6
&
\sphinxAtStartPar
Px6引脚中断沿触发配置寄存器位               |

\sphinxAtStartPar
1:相应位                                   | 沿触发中断,即上升沿和下降沿都会触发中断 |

\sphinxAtStartPar
0:相应位为单边沿触发中断,                 | TRISEEN寄存器相应位确定是上升沿/下降沿触发 |
\\
\sphinxhline
\sphinxAtStartPar
5
&
\sphinxAtStartPar
PIN5
&
\sphinxAtStartPar
Px5引脚中断沿触发配置寄存器位               |

\sphinxAtStartPar
1:相应位                                   | 沿触发中断,即上升沿和下降沿都会触发中断 |

\sphinxAtStartPar
0:相应位为单边沿触发中断,                 | TRISEEN寄存器相应位确定是上升沿/下降沿触发 |
\\
\sphinxhline
\sphinxAtStartPar
4
&
\sphinxAtStartPar
PIN4
&
\sphinxAtStartPar
Px4引脚中断沿触发配置寄存器位               |

\sphinxAtStartPar
1:相应位                                   | 沿触发中断,即上升沿和下降沿都会触发中断 |

\sphinxAtStartPar
0:相应位为单边沿触发中断,                 | TRISEEN寄存器相应位确定是上升沿/下降沿触发 |
\\
\sphinxhline
\sphinxAtStartPar
3
&
\sphinxAtStartPar
PIN3
&
\sphinxAtStartPar
Px3引脚中断沿触发配置寄存器位               |

\sphinxAtStartPar
1:相应位                                   | 沿触发中断,即上升沿和下降沿都会触发中断 |

\sphinxAtStartPar
0:相应位为单边沿触发中断,                 | TRISEEN寄存器相应位确定是上升沿/下降沿触发 |
\\
\sphinxhline
\sphinxAtStartPar
2
&
\sphinxAtStartPar
PIN2
&
\sphinxAtStartPar
Px2引脚中断沿触发配置寄存器位               |

\sphinxAtStartPar
1:相应位                                   | 沿触发中断,即上升沿和下降沿都会触发中断 |

\sphinxAtStartPar
0:相应位为单边沿触发中断,                 | TRISEEN寄存器相应位确定是上升沿/下降沿触发 |
\\
\sphinxhline
\sphinxAtStartPar
1
&
\sphinxAtStartPar
PIN1
&
\sphinxAtStartPar
Px1引脚中断沿触发配置寄存器位               |

\sphinxAtStartPar
1:相应位                                   | 沿触发中断,即上升沿和下降沿都会触发中断 |

\sphinxAtStartPar
0:相应位为单边沿触发中断,                 | TRISEEN寄存器相应位确定是上升沿/下降沿触发 |
\\
\sphinxhline
\sphinxAtStartPar
0
&
\sphinxAtStartPar
PIN0
&
\sphinxAtStartPar
Px0引脚中断沿触发配置寄存器位               |

\sphinxAtStartPar
1:相应位                                   | 沿触发中断,即上升沿和下降沿都会触发中断 |

\sphinxAtStartPar
0:相应位为单边沿触发中断,                 | TRISEEN寄存器相应位确定是上升沿/下降沿触发 |
\\
\sphinxbottomrule
\end{tabular}
\sphinxtableafterendhook\par
\sphinxattableend\end{savenotes}


\subsubsection{GPIOx中断触发极性寄存器INTRISEEN}
\label{\detokenize{SWM241/_u529f_u80fd_u63cf_u8ff0/_u901a_u7528IO:gpioxintriseen}}

\begin{savenotes}\sphinxattablestart
\sphinxthistablewithglobalstyle
\centering
\begin{tabular}[t]{\X{20}{100}\X{20}{100}\X{20}{100}\X{20}{100}\X{20}{100}}
\sphinxtoprule
\sphinxtableatstartofbodyhook
\sphinxAtStartPar
寄存器 |
&
\begin{DUlineblock}{0em}
\item[] 偏移 |
\end{DUlineblock}
&
\begin{DUlineblock}{0em}
\item[] 
\item[] {\color{red}\bfseries{}|}
\end{DUlineblock}
&
\sphinxAtStartPar
复位值 |    描 | |
&
\begin{DUlineblock}{0em}
\item[] |
  |
\end{DUlineblock}
\\
\sphinxhline
\sphinxAtStartPar
INTRISEEN
&
\sphinxAtStartPar
0x10
&&
\sphinxAtStartPar
0 000000
&
\sphinxAtStartPar
GPIO 中断触发极性          |
\\
\sphinxbottomrule
\end{tabular}
\sphinxtableafterendhook\par
\sphinxattableend\end{savenotes}


\begin{savenotes}\sphinxattablestart
\sphinxthistablewithglobalstyle
\centering
\begin{tabular}[t]{\X{12}{96}\X{12}{96}\X{12}{96}\X{12}{96}\X{12}{96}\X{12}{96}\X{12}{96}\X{12}{96}}
\sphinxtoprule
\sphinxtableatstartofbodyhook
\sphinxAtStartPar
31
&
\sphinxAtStartPar
30
&
\sphinxAtStartPar
29
&
\sphinxAtStartPar
28
&
\sphinxAtStartPar
27
&
\sphinxAtStartPar
26
&
\sphinxAtStartPar
25
&
\sphinxAtStartPar
24
\\
\sphinxhline\begin{itemize}
\item {} 
\end{itemize}
&&&&&&&\\
\sphinxhline
\sphinxAtStartPar
23
&
\sphinxAtStartPar
22
&
\sphinxAtStartPar
21
&
\sphinxAtStartPar
20
&
\sphinxAtStartPar
19
&
\sphinxAtStartPar
18
&
\sphinxAtStartPar
17
&
\sphinxAtStartPar
16
\\
\sphinxhline\begin{itemize}
\item {} 
\end{itemize}
&&&&&&&\\
\sphinxhline
\sphinxAtStartPar
15
&
\sphinxAtStartPar
14
&
\sphinxAtStartPar
13
&
\sphinxAtStartPar
12
&
\sphinxAtStartPar
11
&
\sphinxAtStartPar
10
&
\sphinxAtStartPar
9
&
\sphinxAtStartPar
8
\\
\sphinxhline
\sphinxAtStartPar
PIN15
&
\sphinxAtStartPar
PIN14
&&&&&&\\
\sphinxhline
\sphinxAtStartPar
7
&
\sphinxAtStartPar
6
&
\sphinxAtStartPar
5
&
\sphinxAtStartPar
4
&
\sphinxAtStartPar
3
&
\sphinxAtStartPar
2
&
\sphinxAtStartPar
1
&
\sphinxAtStartPar
0
\\
\sphinxhline
\sphinxAtStartPar
PIN7
&
\sphinxAtStartPar
PIN6
&&&&&&\\
\sphinxbottomrule
\end{tabular}
\sphinxtableafterendhook\par
\sphinxattableend\end{savenotes}


\begin{savenotes}\sphinxattablestart
\sphinxthistablewithglobalstyle
\centering
\begin{tabular}[t]{\X{33}{99}\X{33}{99}\X{33}{99}}
\sphinxtoprule
\sphinxtableatstartofbodyhook
\sphinxAtStartPar
位域 |
&
\sphinxAtStartPar
名称     | |
&
\sphinxAtStartPar
描述                                        | |
\\
\sphinxhline
\sphinxAtStartPar
31:16
&\begin{itemize}
\item {} 
\end{itemize}
&\begin{itemize}
\item {} 
\end{itemize}
\\
\sphinxhline
\sphinxAtStartPar
15
&
\sphinxAtStartPar
PIN15
&
\sphinxAtStartPar
Px15引脚中断事件寄存器位                    |

\sphinxAtStartPar
1:上升沿/高电平触发中断                    |

\sphinxAtStartPar
0:下降沿/低电平触发中断                    |
\\
\sphinxhline
\sphinxAtStartPar
14
&
\sphinxAtStartPar
PIN14
&
\sphinxAtStartPar
Px14引脚中断事件寄存器位                    |

\sphinxAtStartPar
1:上升沿/高电平触发中断                    |

\sphinxAtStartPar
0:下降沿/低电平触发中断                    |
\\
\sphinxhline
\sphinxAtStartPar
13
&
\sphinxAtStartPar
PIN13
&
\sphinxAtStartPar
Px13引脚中断事件寄存器位                    |

\sphinxAtStartPar
1:上升沿/高电平触发中断                    |

\sphinxAtStartPar
0:下降沿/低电平触发中断                    |
\\
\sphinxhline
\sphinxAtStartPar
12
&
\sphinxAtStartPar
PIN12
&
\sphinxAtStartPar
Px12引脚中断事件寄存器位                    |

\sphinxAtStartPar
1:上升沿/高电平触发中断                    |

\sphinxAtStartPar
0:下降沿/低电平触发中断                    |
\\
\sphinxhline
\sphinxAtStartPar
11
&
\sphinxAtStartPar
PIN11
&
\sphinxAtStartPar
Px11引脚中断事件寄存器位                    |

\sphinxAtStartPar
1:上升沿/高电平触发中断                    |

\sphinxAtStartPar
0:下降沿/低电平触发中断                    |
\\
\sphinxhline
\sphinxAtStartPar
10
&
\sphinxAtStartPar
PIN10
&
\sphinxAtStartPar
Px10引脚中断事件寄存器位                    |

\sphinxAtStartPar
1:上升沿/高电平触发中断                    |

\sphinxAtStartPar
0:下降沿/低电平触发中断                    |
\\
\sphinxhline
\sphinxAtStartPar
9
&
\sphinxAtStartPar
PIN9
&
\sphinxAtStartPar
Px9引脚中断事件寄存器位                     |

\sphinxAtStartPar
1:上升沿/高电平触发中断                    |

\sphinxAtStartPar
0:下降沿/低电平触发中断                    |
\\
\sphinxhline
\sphinxAtStartPar
8
&
\sphinxAtStartPar
PIN8
&
\sphinxAtStartPar
Px8引脚中断事件寄存器位                     |

\sphinxAtStartPar
1:上升沿/高电平触发中断                    |

\sphinxAtStartPar
0:下降沿/低电平触发中断                    |
\\
\sphinxhline
\sphinxAtStartPar
7
&
\sphinxAtStartPar
PIN7
&
\sphinxAtStartPar
Px7引脚中断事件寄存器位                     |

\sphinxAtStartPar
1:上升沿/高电平触发中断                    |

\sphinxAtStartPar
0:下降沿/低电平触发中断                    |
\\
\sphinxhline
\sphinxAtStartPar
6
&
\sphinxAtStartPar
PIN6
&
\sphinxAtStartPar
Px6引脚中断事件寄存器位                     |

\sphinxAtStartPar
1:上升沿/高电平触发中断                    |

\sphinxAtStartPar
0:下降沿/低电平触发中断                    |
\\
\sphinxhline
\sphinxAtStartPar
5
&
\sphinxAtStartPar
PIN5
&
\sphinxAtStartPar
Px5引脚中断事件寄存器位                     |

\sphinxAtStartPar
1:上升沿/高电平触发中断                    |

\sphinxAtStartPar
0:下降沿/低电平触发中断                    |
\\
\sphinxhline
\sphinxAtStartPar
4
&
\sphinxAtStartPar
PIN4
&
\sphinxAtStartPar
Px4引脚中断事件寄存器位                     |

\sphinxAtStartPar
1:上升沿/高电平触发中断                    |

\sphinxAtStartPar
0:下降沿/低电平触发中断                    |
\\
\sphinxhline
\sphinxAtStartPar
3
&
\sphinxAtStartPar
PIN3
&
\sphinxAtStartPar
Px3引脚中断事件寄存器位                     |

\sphinxAtStartPar
1:上升沿/高电平触发中断                    |

\sphinxAtStartPar
0:下降沿/低电平触发中断                    |
\\
\sphinxhline
\sphinxAtStartPar
2
&
\sphinxAtStartPar
PIN2
&
\sphinxAtStartPar
Px2引脚中断事件寄存器位                     |

\sphinxAtStartPar
1:上升沿/高电平触发中断                    |

\sphinxAtStartPar
0:下降沿/低电平触发中断                    |
\\
\sphinxhline
\sphinxAtStartPar
1
&
\sphinxAtStartPar
PIN1
&
\sphinxAtStartPar
Px1引脚中断事件寄存器位                     |

\sphinxAtStartPar
1:上升沿/高电平触发中断                    |

\sphinxAtStartPar
0:下降沿/低电平触发中断                    |
\\
\sphinxhline
\sphinxAtStartPar
0
&
\sphinxAtStartPar
PIN0
&
\sphinxAtStartPar
Px0引脚中断事件寄存器位                     |

\sphinxAtStartPar
1:上升沿/高电平触发中断                    |

\sphinxAtStartPar
0:下降沿/低电平触发中断                    |
\\
\sphinxbottomrule
\end{tabular}
\sphinxtableafterendhook\par
\sphinxattableend\end{savenotes}


\subsubsection{GPIOx中断使能寄存器INTEN}
\label{\detokenize{SWM241/_u529f_u80fd_u63cf_u8ff0/_u901a_u7528IO:gpioxinten}}

\begin{savenotes}\sphinxattablestart
\sphinxthistablewithglobalstyle
\centering
\begin{tabular}[t]{\X{20}{100}\X{20}{100}\X{20}{100}\X{20}{100}\X{20}{100}}
\sphinxtoprule
\sphinxtableatstartofbodyhook
\sphinxAtStartPar
寄存器 |
&
\begin{DUlineblock}{0em}
\item[] 偏移 |
\end{DUlineblock}
&
\begin{DUlineblock}{0em}
\item[] 
\item[] {\color{red}\bfseries{}|}
\end{DUlineblock}
&
\sphinxAtStartPar
复位值 |    描 | |
&
\begin{DUlineblock}{0em}
\item[] |
  |
\end{DUlineblock}
\\
\sphinxhline
\sphinxAtStartPar
INTEN
&
\sphinxAtStartPar
0x14
&&
\sphinxAtStartPar
0 000000
&
\sphinxAtStartPar
GPIO中断使能               |
\\
\sphinxbottomrule
\end{tabular}
\sphinxtableafterendhook\par
\sphinxattableend\end{savenotes}


\begin{savenotes}\sphinxattablestart
\sphinxthistablewithglobalstyle
\centering
\begin{tabular}[t]{\X{12}{96}\X{12}{96}\X{12}{96}\X{12}{96}\X{12}{96}\X{12}{96}\X{12}{96}\X{12}{96}}
\sphinxtoprule
\sphinxtableatstartofbodyhook
\sphinxAtStartPar
31
&
\sphinxAtStartPar
30
&
\sphinxAtStartPar
29
&
\sphinxAtStartPar
28
&
\sphinxAtStartPar
27
&
\sphinxAtStartPar
26
&
\sphinxAtStartPar
25
&
\sphinxAtStartPar
24
\\
\sphinxhline\begin{itemize}
\item {} 
\end{itemize}
&&&&&&&\\
\sphinxhline
\sphinxAtStartPar
23
&
\sphinxAtStartPar
22
&
\sphinxAtStartPar
21
&
\sphinxAtStartPar
20
&
\sphinxAtStartPar
19
&
\sphinxAtStartPar
18
&
\sphinxAtStartPar
17
&
\sphinxAtStartPar
16
\\
\sphinxhline\begin{itemize}
\item {} 
\end{itemize}
&&&&&&&\\
\sphinxhline
\sphinxAtStartPar
15
&
\sphinxAtStartPar
14
&
\sphinxAtStartPar
13
&
\sphinxAtStartPar
12
&
\sphinxAtStartPar
11
&
\sphinxAtStartPar
10
&
\sphinxAtStartPar
9
&
\sphinxAtStartPar
8
\\
\sphinxhline
\sphinxAtStartPar
PIN15
&
\sphinxAtStartPar
PIN14
&&&&&&\\
\sphinxhline
\sphinxAtStartPar
7
&
\sphinxAtStartPar
6
&
\sphinxAtStartPar
5
&
\sphinxAtStartPar
4
&
\sphinxAtStartPar
3
&
\sphinxAtStartPar
2
&
\sphinxAtStartPar
1
&
\sphinxAtStartPar
0
\\
\sphinxhline
\sphinxAtStartPar
PIN7
&
\sphinxAtStartPar
PIN6
&&&&&&\\
\sphinxbottomrule
\end{tabular}
\sphinxtableafterendhook\par
\sphinxattableend\end{savenotes}


\begin{savenotes}\sphinxattablestart
\sphinxthistablewithglobalstyle
\centering
\begin{tabular}[t]{\X{33}{99}\X{33}{99}\X{33}{99}}
\sphinxtoprule
\sphinxtableatstartofbodyhook
\sphinxAtStartPar
位域 |
&
\sphinxAtStartPar
名称     | |
&
\sphinxAtStartPar
描述                                        | |
\\
\sphinxhline
\sphinxAtStartPar
31:16
&\begin{itemize}
\item {} 
\end{itemize}
&\begin{itemize}
\item {} 
\end{itemize}
\\
\sphinxhline
\sphinxAtStartPar
15
&
\sphinxAtStartPar
PIN15
&
\sphinxAtStartPar
Px15引脚中断使能寄存器位                    |

\sphinxAtStartPar
1:相应位为中断使能                         |

\sphinxAtStartPar
0:相应位为中断禁止                         |
\\
\sphinxhline
\sphinxAtStartPar
14
&
\sphinxAtStartPar
PIN14
&
\sphinxAtStartPar
Px14引脚中断使能寄存器位                    |

\sphinxAtStartPar
1:相应位为中断使能                         |

\sphinxAtStartPar
0:相应位为中断禁止                         |
\\
\sphinxhline
\sphinxAtStartPar
13
&
\sphinxAtStartPar
PIN13
&
\sphinxAtStartPar
Px13引脚中断使能寄存器位                    |

\sphinxAtStartPar
1:相应位为中断使能                         |

\sphinxAtStartPar
0:相应位为中断禁止                         |
\\
\sphinxhline
\sphinxAtStartPar
12
&
\sphinxAtStartPar
PIN12
&
\sphinxAtStartPar
Px12引脚中断使能寄存器位                    |

\sphinxAtStartPar
1:相应位为中断使能                         |

\sphinxAtStartPar
0:相应位为中断禁止                         |
\\
\sphinxhline
\sphinxAtStartPar
11
&
\sphinxAtStartPar
PIN11
&
\sphinxAtStartPar
Px11引脚中断使能寄存器位                    |

\sphinxAtStartPar
1:相应位为中断使能                         |

\sphinxAtStartPar
0:相应位为中断禁止                         |
\\
\sphinxhline
\sphinxAtStartPar
10
&
\sphinxAtStartPar
PIN10
&
\sphinxAtStartPar
Px10引脚中断使能寄存器位                    |

\sphinxAtStartPar
1:相应位为中断使能                         |

\sphinxAtStartPar
0:相应位为中断禁止                         |
\\
\sphinxhline
\sphinxAtStartPar
9
&
\sphinxAtStartPar
PIN9
&
\sphinxAtStartPar
Px9引脚中断使能寄存器位                     |

\sphinxAtStartPar
1:相应位为中断使能                         |

\sphinxAtStartPar
0:相应位为中断禁止                         |
\\
\sphinxhline
\sphinxAtStartPar
8
&
\sphinxAtStartPar
PIN8
&
\sphinxAtStartPar
Px8引脚中断使能寄存器位                     |

\sphinxAtStartPar
1:相应位为中断使能                         |

\sphinxAtStartPar
0:相应位为中断禁止                         |
\\
\sphinxhline
\sphinxAtStartPar
7
&
\sphinxAtStartPar
PIN7
&
\sphinxAtStartPar
Px7引脚中断使能寄存器位                     |

\sphinxAtStartPar
1:相应位为中断使能                         |

\sphinxAtStartPar
0:相应位为中断禁止                         |
\\
\sphinxhline
\sphinxAtStartPar
6
&
\sphinxAtStartPar
PIN6
&
\sphinxAtStartPar
Px6引脚中断使能寄存器位                     |

\sphinxAtStartPar
1:相应位为中断使能                         |

\sphinxAtStartPar
0:相应位为中断禁止                         |
\\
\sphinxhline
\sphinxAtStartPar
5
&
\sphinxAtStartPar
PIN5
&
\sphinxAtStartPar
Px5引脚中断使能寄存器位                     |

\sphinxAtStartPar
1:相应位为中断使能                         |

\sphinxAtStartPar
0:相应位为中断禁止                         |
\\
\sphinxhline
\sphinxAtStartPar
4
&
\sphinxAtStartPar
PIN4
&
\sphinxAtStartPar
Px4引脚中断使能寄存器位                     |

\sphinxAtStartPar
1:相应位为中断使能                         |

\sphinxAtStartPar
0:相应位为中断禁止                         |
\\
\sphinxhline
\sphinxAtStartPar
3
&
\sphinxAtStartPar
PIN3
&
\sphinxAtStartPar
Px3引脚中断使能寄存器位                     |

\sphinxAtStartPar
1:相应位为中断使能                         |

\sphinxAtStartPar
0:相应位为中断禁止                         |
\\
\sphinxhline
\sphinxAtStartPar
2
&
\sphinxAtStartPar
PIN2
&
\sphinxAtStartPar
Px2引脚中断使能寄存器位                     |

\sphinxAtStartPar
1:相应位为中断使能                         |

\sphinxAtStartPar
0:相应位为中断禁止                         |
\\
\sphinxhline
\sphinxAtStartPar
1
&
\sphinxAtStartPar
PIN1
&
\sphinxAtStartPar
Px1引脚中断使能寄存器位                     |

\sphinxAtStartPar
1:相应位为中断使能                         |

\sphinxAtStartPar
0:相应位为中断禁止                         |
\\
\sphinxhline
\sphinxAtStartPar
0
&
\sphinxAtStartPar
PIN0
&
\sphinxAtStartPar
Px0引脚中断使能寄存器位                     |

\sphinxAtStartPar
1:相应位为中断使能                         |

\sphinxAtStartPar
0:相应位为中断禁止                         |
\\
\sphinxbottomrule
\end{tabular}
\sphinxtableafterendhook\par
\sphinxattableend\end{savenotes}


\subsubsection{GPIOx原始中断状态寄存器INTRAWSTAT}
\label{\detokenize{SWM241/_u529f_u80fd_u63cf_u8ff0/_u901a_u7528IO:gpioxintrawstat}}

\begin{savenotes}\sphinxattablestart
\sphinxthistablewithglobalstyle
\centering
\begin{tabular}[t]{\X{20}{100}\X{20}{100}\X{20}{100}\X{20}{100}\X{20}{100}}
\sphinxtoprule
\sphinxtableatstartofbodyhook
\sphinxAtStartPar
寄存器 |
&
\begin{DUlineblock}{0em}
\item[] 偏移 |
\end{DUlineblock}
&
\begin{DUlineblock}{0em}
\item[] 
\item[] {\color{red}\bfseries{}|}
\end{DUlineblock}
&
\sphinxAtStartPar
复位值 |    描 | |
&
\begin{DUlineblock}{0em}
\item[] |
  |
\end{DUlineblock}
\\
\sphinxhline
\sphinxAtStartPar
INTRAWSTAT
&
\sphinxAtStartPar
0x18
&&
\sphinxAtStartPar
0 000000
&
\sphinxAtStartPar
GPIO中断原始状态           |
\\
\sphinxbottomrule
\end{tabular}
\sphinxtableafterendhook\par
\sphinxattableend\end{savenotes}


\begin{savenotes}\sphinxattablestart
\sphinxthistablewithglobalstyle
\centering
\begin{tabular}[t]{\X{12}{96}\X{12}{96}\X{12}{96}\X{12}{96}\X{12}{96}\X{12}{96}\X{12}{96}\X{12}{96}}
\sphinxtoprule
\sphinxtableatstartofbodyhook
\sphinxAtStartPar
31
&
\sphinxAtStartPar
30
&
\sphinxAtStartPar
29
&
\sphinxAtStartPar
28
&
\sphinxAtStartPar
27
&
\sphinxAtStartPar
26
&
\sphinxAtStartPar
25
&
\sphinxAtStartPar
24
\\
\sphinxhline\begin{itemize}
\item {} 
\end{itemize}
&&&&&&&\\
\sphinxhline
\sphinxAtStartPar
23
&
\sphinxAtStartPar
22
&
\sphinxAtStartPar
21
&
\sphinxAtStartPar
20
&
\sphinxAtStartPar
19
&
\sphinxAtStartPar
18
&
\sphinxAtStartPar
17
&
\sphinxAtStartPar
16
\\
\sphinxhline\begin{itemize}
\item {} 
\end{itemize}
&&&&&&&\\
\sphinxhline
\sphinxAtStartPar
15
&
\sphinxAtStartPar
14
&
\sphinxAtStartPar
13
&
\sphinxAtStartPar
12
&
\sphinxAtStartPar
11
&
\sphinxAtStartPar
10
&
\sphinxAtStartPar
9
&
\sphinxAtStartPar
8
\\
\sphinxhline
\sphinxAtStartPar
PIN15
&
\sphinxAtStartPar
PIN14
&&&&&&\\
\sphinxhline
\sphinxAtStartPar
7
&
\sphinxAtStartPar
6
&
\sphinxAtStartPar
5
&
\sphinxAtStartPar
4
&
\sphinxAtStartPar
3
&
\sphinxAtStartPar
2
&
\sphinxAtStartPar
1
&
\sphinxAtStartPar
0
\\
\sphinxhline
\sphinxAtStartPar
PIN7
&
\sphinxAtStartPar
PIN6
&&&&&&\\
\sphinxbottomrule
\end{tabular}
\sphinxtableafterendhook\par
\sphinxattableend\end{savenotes}


\begin{savenotes}\sphinxattablestart
\sphinxthistablewithglobalstyle
\centering
\begin{tabular}[t]{\X{33}{99}\X{33}{99}\X{33}{99}}
\sphinxtoprule
\sphinxtableatstartofbodyhook
\sphinxAtStartPar
位域 |
&
\sphinxAtStartPar
名称     | |
&
\sphinxAtStartPar
描述                                        | |
\\
\sphinxhline
\sphinxAtStartPar
31:16
&\begin{itemize}
\item {} 
\end{itemize}
&\begin{itemize}
\item {} 
\end{itemize}
\\
\sphinxhline
\sphinxAtStartPar
15
&
\sphinxAtStartPar
PIN15
&
\sphinxAtStartPar
Px15引脚原始中断状态寄存器位                |

\sphinxAtStartPar
1:检测到中断触发条件(不受使能影响)         |

\sphinxAtStartPar
0:没有检测到中断触发条件(不受使能影响)     |
\\
\sphinxhline
\sphinxAtStartPar
14
&
\sphinxAtStartPar
PIN14
&
\sphinxAtStartPar
Px14引脚原始中断状态寄存器位                |

\sphinxAtStartPar
1:检测到中断触发条件(不受使能影响)         |

\sphinxAtStartPar
0:没有检测到中断触发条件(不受使能影响)     |
\\
\sphinxhline
\sphinxAtStartPar
13
&
\sphinxAtStartPar
PIN13
&
\sphinxAtStartPar
Px13引脚原始中断状态寄存器位                |

\sphinxAtStartPar
1:检测到中断触发条件(不受使能影响)         |

\sphinxAtStartPar
0:没有检测到中断触发条件(不受使能影响)     |
\\
\sphinxhline
\sphinxAtStartPar
12
&
\sphinxAtStartPar
PIN12
&
\sphinxAtStartPar
Px12引脚原始中断状态寄存器位                |

\sphinxAtStartPar
1:检测到中断触发条件(不受使能影响)         |

\sphinxAtStartPar
0:没有检测到中断触发条件(不受使能影响)     |
\\
\sphinxhline
\sphinxAtStartPar
11
&
\sphinxAtStartPar
PIN11
&
\sphinxAtStartPar
Px11引脚原始中断状态寄存器位                |

\sphinxAtStartPar
1:检测到中断触发条件(不受使能影响)         |

\sphinxAtStartPar
0:没有检测到中断触发条件(不受使能影响)     |
\\
\sphinxhline
\sphinxAtStartPar
10
&
\sphinxAtStartPar
PIN10
&
\sphinxAtStartPar
Px10引脚原始中断状态寄存器位                |

\sphinxAtStartPar
1:检测到中断触发条件(不受使能影响)         |

\sphinxAtStartPar
0:没有检测到中断触发条件(不受使能影响)     |
\\
\sphinxhline
\sphinxAtStartPar
9
&
\sphinxAtStartPar
PIN9
&
\sphinxAtStartPar
Px9引脚原始中断状态寄存器位                 |

\sphinxAtStartPar
1:检测到中断触发条件(不受使能影响)         |

\sphinxAtStartPar
0:没有检测到中断触发条件(不受使能影响)     |
\\
\sphinxhline
\sphinxAtStartPar
8
&
\sphinxAtStartPar
PIN8
&
\sphinxAtStartPar
Px8引脚原始中断状态寄存器位                 |

\sphinxAtStartPar
1:检测到中断触发条件(不受使能影响)         |

\sphinxAtStartPar
0:没有检测到中断触发条件(不受使能影响)     |
\\
\sphinxhline
\sphinxAtStartPar
7
&
\sphinxAtStartPar
PIN7
&
\sphinxAtStartPar
Px7引脚原始中断状态寄存器位                 |

\sphinxAtStartPar
1:检测到中断触发条件(不受使能影响)         |

\sphinxAtStartPar
0:没有检测到中断触发条件(不受使能影响)     |
\\
\sphinxhline
\sphinxAtStartPar
6
&
\sphinxAtStartPar
PIN6
&
\sphinxAtStartPar
Px6引脚原始中断状态寄存器位                 |

\sphinxAtStartPar
1:检测到中断触发条件(不受使能影响)         |

\sphinxAtStartPar
0:没有检测到中断触发条件(不受使能影响)     |
\\
\sphinxhline
\sphinxAtStartPar
5
&
\sphinxAtStartPar
PIN5
&
\sphinxAtStartPar
Px5引脚原始中断状态寄存器位                 |

\sphinxAtStartPar
1:检测到中断触发条件(不受使能影响)         |

\sphinxAtStartPar
0:没有检测到中断触发条件(不受使能影响)     |
\\
\sphinxhline
\sphinxAtStartPar
4
&
\sphinxAtStartPar
PIN4
&
\sphinxAtStartPar
Px4引脚原始中断状态寄存器位                 |

\sphinxAtStartPar
1:检测到中断触发条件(不受使能影响)         |

\sphinxAtStartPar
0:没有检测到中断触发条件(不受使能影响)     |
\\
\sphinxhline
\sphinxAtStartPar
3
&
\sphinxAtStartPar
PIN3
&
\sphinxAtStartPar
Px3引脚原始中断状态寄存器位                 |

\sphinxAtStartPar
1:检测到中断触发条件(不受使能影响)         |

\sphinxAtStartPar
0:没有检测到中断触发条件(不受使能影响)     |
\\
\sphinxhline
\sphinxAtStartPar
2
&
\sphinxAtStartPar
PIN2
&
\sphinxAtStartPar
Px2引脚原始中断状态寄存器位                 |

\sphinxAtStartPar
1:检测到中断触发条件(不受使能影响)         |

\sphinxAtStartPar
0:没有检测到中断触发条件(不受使能影响)     |
\\
\sphinxhline
\sphinxAtStartPar
1
&
\sphinxAtStartPar
PIN1
&
\sphinxAtStartPar
Px1引脚原始中断状态寄存器位                 |

\sphinxAtStartPar
1:检测到中断触发条件(不受使能影响)         |

\sphinxAtStartPar
0:没有检测到中断触发条件(不受使能影响)     |
\\
\sphinxhline
\sphinxAtStartPar
0
&
\sphinxAtStartPar
PIN0
&
\sphinxAtStartPar
Px0引脚原始中断状态寄存器位                 |

\sphinxAtStartPar
1:检测到中断触发条件(不受使能影响)         |

\sphinxAtStartPar
0:没有检测到中断触发条件(不受使能影响)     |
\\
\sphinxbottomrule
\end{tabular}
\sphinxtableafterendhook\par
\sphinxattableend\end{savenotes}


\subsubsection{GPIOx中断状态寄存器INTSTAT}
\label{\detokenize{SWM241/_u529f_u80fd_u63cf_u8ff0/_u901a_u7528IO:gpioxintstat}}

\begin{savenotes}\sphinxattablestart
\sphinxthistablewithglobalstyle
\centering
\begin{tabular}[t]{\X{20}{100}\X{20}{100}\X{20}{100}\X{20}{100}\X{20}{100}}
\sphinxtoprule
\sphinxtableatstartofbodyhook
\sphinxAtStartPar
寄存器 |
&
\begin{DUlineblock}{0em}
\item[] 偏移 |
\end{DUlineblock}
&
\begin{DUlineblock}{0em}
\item[] 
\item[] {\color{red}\bfseries{}|}
\end{DUlineblock}
&
\sphinxAtStartPar
复位值 |    描 | |
&
\begin{DUlineblock}{0em}
\item[] |
  |
\end{DUlineblock}
\\
\sphinxhline
\sphinxAtStartPar
INTSTAT
&
\sphinxAtStartPar
0x1c
&&
\sphinxAtStartPar
0 000000
&
\sphinxAtStartPar
GPIO中断状态               |
\\
\sphinxbottomrule
\end{tabular}
\sphinxtableafterendhook\par
\sphinxattableend\end{savenotes}


\begin{savenotes}\sphinxattablestart
\sphinxthistablewithglobalstyle
\centering
\begin{tabular}[t]{\X{12}{96}\X{12}{96}\X{12}{96}\X{12}{96}\X{12}{96}\X{12}{96}\X{12}{96}\X{12}{96}}
\sphinxtoprule
\sphinxtableatstartofbodyhook
\sphinxAtStartPar
31
&
\sphinxAtStartPar
30
&
\sphinxAtStartPar
29
&
\sphinxAtStartPar
28
&
\sphinxAtStartPar
27
&
\sphinxAtStartPar
26
&
\sphinxAtStartPar
25
&
\sphinxAtStartPar
24
\\
\sphinxhline\begin{itemize}
\item {} 
\end{itemize}
&&&&&&&\\
\sphinxhline
\sphinxAtStartPar
23
&
\sphinxAtStartPar
22
&
\sphinxAtStartPar
21
&
\sphinxAtStartPar
20
&
\sphinxAtStartPar
19
&
\sphinxAtStartPar
18
&
\sphinxAtStartPar
17
&
\sphinxAtStartPar
16
\\
\sphinxhline\begin{itemize}
\item {} 
\end{itemize}
&&&&&&&\\
\sphinxhline
\sphinxAtStartPar
15
&
\sphinxAtStartPar
14
&
\sphinxAtStartPar
13
&
\sphinxAtStartPar
12
&
\sphinxAtStartPar
11
&
\sphinxAtStartPar
10
&
\sphinxAtStartPar
9
&
\sphinxAtStartPar
8
\\
\sphinxhline
\sphinxAtStartPar
PIN15
&
\sphinxAtStartPar
PIN14
&&&&&&\\
\sphinxhline
\sphinxAtStartPar
7
&
\sphinxAtStartPar
6
&
\sphinxAtStartPar
5
&
\sphinxAtStartPar
4
&
\sphinxAtStartPar
3
&
\sphinxAtStartPar
2
&
\sphinxAtStartPar
1
&
\sphinxAtStartPar
0
\\
\sphinxhline
\sphinxAtStartPar
PIN7
&
\sphinxAtStartPar
PIN6
&&&&&&\\
\sphinxbottomrule
\end{tabular}
\sphinxtableafterendhook\par
\sphinxattableend\end{savenotes}


\begin{savenotes}\sphinxattablestart
\sphinxthistablewithglobalstyle
\centering
\begin{tabular}[t]{\X{33}{99}\X{33}{99}\X{33}{99}}
\sphinxtoprule
\sphinxtableatstartofbodyhook
\sphinxAtStartPar
位域 |
&
\sphinxAtStartPar
名称     | |
&
\sphinxAtStartPar
描述                                        | |
\\
\sphinxhline
\sphinxAtStartPar
31:16
&\begin{itemize}
\item {} 
\end{itemize}
&\begin{itemize}
\item {} 
\end{itemize}
\\
\sphinxhline
\sphinxAtStartPar
15
&
\sphinxAtStartPar
PIN15
&
\sphinxAtStartPar
Px15引脚中断状态寄存器位                    |

\sphinxAtStartPar
1:检测到了中断                             |

\sphinxAtStartPar
0:没有检测到中断                           |

\sphinxAtStartPar
INTSTAT.PINx = INTRAWSTAT.PINx \& INTEN.PINx
\\
\sphinxhline
\sphinxAtStartPar
14
&
\sphinxAtStartPar
PIN14
&
\sphinxAtStartPar
Px14引脚中断状态寄存器位                    |

\sphinxAtStartPar
1:检测到了中断                             |

\sphinxAtStartPar
0:没有检测到中断                           |

\sphinxAtStartPar
INTSTAT.PINx = INTRAWSTAT.PINx \& INTEN.PINx
\\
\sphinxhline
\sphinxAtStartPar
13
&
\sphinxAtStartPar
PIN13
&
\sphinxAtStartPar
Px13引脚中断状态寄存器位                    |

\sphinxAtStartPar
1:检测到了中断                             |

\sphinxAtStartPar
0:没有检测到中断                           |

\sphinxAtStartPar
INTSTAT.PINx = INTRAWSTAT.PINx \& INTEN.PINx
\\
\sphinxhline
\sphinxAtStartPar
12
&
\sphinxAtStartPar
PIN12
&
\sphinxAtStartPar
Px12引脚中断状态寄存器位                    |

\sphinxAtStartPar
1:检测到了中断                             |

\sphinxAtStartPar
0:没有检测到中断                           |

\sphinxAtStartPar
INTSTAT.PINx = INTRAWSTAT.PINx \& INTEN.PINx
\\
\sphinxhline
\sphinxAtStartPar
11
&
\sphinxAtStartPar
PIN11
&
\sphinxAtStartPar
Px11引脚中断状态寄存器位                    |

\sphinxAtStartPar
1:检测到了中断                             |

\sphinxAtStartPar
0:没有检测到中断                           |

\sphinxAtStartPar
INTSTAT.PINx = INTRAWSTAT.PINx \& INTEN.PINx
\\
\sphinxhline
\sphinxAtStartPar
10
&
\sphinxAtStartPar
PIN10
&
\sphinxAtStartPar
Px10引脚中断状态寄存器位                    |

\sphinxAtStartPar
1:检测到了中断                             |

\sphinxAtStartPar
0:没有检测到中断                           |

\sphinxAtStartPar
INTSTAT.PINx = INTRAWSTAT.PINx \& INTEN.PINx
\\
\sphinxhline
\sphinxAtStartPar
9
&
\sphinxAtStartPar
PIN9
&
\sphinxAtStartPar
Px9引脚中断状态寄存器位                     |

\sphinxAtStartPar
1:检测到了中断                             |

\sphinxAtStartPar
0:没有检测到中断                           |

\sphinxAtStartPar
INTSTAT.PINx = INTRAWSTAT.PINx \& INTEN.PINx
\\
\sphinxhline
\sphinxAtStartPar
8
&
\sphinxAtStartPar
PIN8
&
\sphinxAtStartPar
Px8引脚中断状态寄存器位                     |

\sphinxAtStartPar
1:检测到了中断                             |

\sphinxAtStartPar
0:没有检测到中断                           |

\sphinxAtStartPar
INTSTAT.PINx = INTRAWSTAT.PINx \& INTEN.PINx
\\
\sphinxhline
\sphinxAtStartPar
7
&
\sphinxAtStartPar
PIN7
&
\sphinxAtStartPar
Px7引脚中断状态寄存器位                     |

\sphinxAtStartPar
1:检测到了中断                             |

\sphinxAtStartPar
0:没有检测到中断                           |

\sphinxAtStartPar
INTSTAT.PINx = INTRAWSTAT.PINx \& INTEN.PINx
\\
\sphinxhline
\sphinxAtStartPar
6
&
\sphinxAtStartPar
PIN6
&
\sphinxAtStartPar
Px6引脚中断状态寄存器位                     |

\sphinxAtStartPar
1:检测到了中断                             |

\sphinxAtStartPar
0:没有检测到中断                           |

\sphinxAtStartPar
INTSTAT.PINx = INTRAWSTAT.PINx \& INTEN.PINx
\\
\sphinxhline
\sphinxAtStartPar
5
&
\sphinxAtStartPar
PIN5
&
\sphinxAtStartPar
Px5引脚中断状态寄存器位                     |

\sphinxAtStartPar
1:检测到了中断                             |

\sphinxAtStartPar
0:没有检测到中断                           |

\sphinxAtStartPar
INTSTAT.PINx = INTRAWSTAT.PINx \& INTEN.PINx
\\
\sphinxhline
\sphinxAtStartPar
4
&
\sphinxAtStartPar
PIN4
&
\sphinxAtStartPar
Px4引脚中断状态寄存器位                     |

\sphinxAtStartPar
1:检测到了中断                             |

\sphinxAtStartPar
0:没有检测到中断                           |

\sphinxAtStartPar
INTSTAT.PINx = INTRAWSTAT.PINx \& INTEN.PINx
\\
\sphinxhline
\sphinxAtStartPar
3
&
\sphinxAtStartPar
PIN3
&
\sphinxAtStartPar
Px3引脚中断状态寄存器位                     |

\sphinxAtStartPar
1:检测到了中断                             |

\sphinxAtStartPar
0:没有检测到中断                           |

\sphinxAtStartPar
INTSTAT.PINx = INTRAWSTAT.PINx \& INTEN.PINx
\\
\sphinxhline
\sphinxAtStartPar
2
&
\sphinxAtStartPar
PIN2
&
\sphinxAtStartPar
Px2引脚中断状态寄存器位                     |

\sphinxAtStartPar
1:检测到了中断                             |

\sphinxAtStartPar
0:没有检测到中断                           |

\sphinxAtStartPar
INTSTAT.PINx = INTRAWSTAT.PINx \& INTEN.PINx
\\
\sphinxhline
\sphinxAtStartPar
1
&
\sphinxAtStartPar
PIN1
&
\sphinxAtStartPar
Px1引脚中断状态寄存器位                     |

\sphinxAtStartPar
1:检测到了中断                             |

\sphinxAtStartPar
0:没有检测到中断                           |

\sphinxAtStartPar
INTSTAT.PINx = INTRAWSTAT.PINx \& INTEN.PINx
\\
\sphinxhline
\sphinxAtStartPar
0
&
\sphinxAtStartPar
PIN0
&
\sphinxAtStartPar
Px0引脚中断状态寄存器位                     |

\sphinxAtStartPar
1:检测到了中断                             |

\sphinxAtStartPar
0:没有检测到中断                           |

\sphinxAtStartPar
INTSTAT.PINx = INTRAWSTAT.PINx \& INTEN.PINx
\\
\sphinxbottomrule
\end{tabular}
\sphinxtableafterendhook\par
\sphinxattableend\end{savenotes}


\subsubsection{GPIOx中断清除寄存器INTCLR}
\label{\detokenize{SWM241/_u529f_u80fd_u63cf_u8ff0/_u901a_u7528IO:gpioxintclr}}

\begin{savenotes}\sphinxattablestart
\sphinxthistablewithglobalstyle
\centering
\begin{tabular}[t]{\X{20}{100}\X{20}{100}\X{20}{100}\X{20}{100}\X{20}{100}}
\sphinxtoprule
\sphinxtableatstartofbodyhook
\sphinxAtStartPar
寄存器 |
&
\begin{DUlineblock}{0em}
\item[] 偏移 |
\end{DUlineblock}
&
\begin{DUlineblock}{0em}
\item[] 
\item[] {\color{red}\bfseries{}|}
\end{DUlineblock}
&
\sphinxAtStartPar
复位值 |    描 | |
&
\begin{DUlineblock}{0em}
\item[] |
  |
\end{DUlineblock}
\\
\sphinxhline
\sphinxAtStartPar
INTCLR
&
\sphinxAtStartPar
0x20
&&
\sphinxAtStartPar
0 000000
&
\sphinxAtStartPar
GPIO中断清除               |
\\
\sphinxbottomrule
\end{tabular}
\sphinxtableafterendhook\par
\sphinxattableend\end{savenotes}


\begin{savenotes}\sphinxattablestart
\sphinxthistablewithglobalstyle
\centering
\begin{tabular}[t]{\X{12}{96}\X{12}{96}\X{12}{96}\X{12}{96}\X{12}{96}\X{12}{96}\X{12}{96}\X{12}{96}}
\sphinxtoprule
\sphinxtableatstartofbodyhook
\sphinxAtStartPar
31
&
\sphinxAtStartPar
30
&
\sphinxAtStartPar
29
&
\sphinxAtStartPar
28
&
\sphinxAtStartPar
27
&
\sphinxAtStartPar
26
&
\sphinxAtStartPar
25
&
\sphinxAtStartPar
24
\\
\sphinxhline\begin{itemize}
\item {} 
\end{itemize}
&&&&&&&\\
\sphinxhline
\sphinxAtStartPar
23
&
\sphinxAtStartPar
22
&
\sphinxAtStartPar
21
&
\sphinxAtStartPar
20
&
\sphinxAtStartPar
19
&
\sphinxAtStartPar
18
&
\sphinxAtStartPar
17
&
\sphinxAtStartPar
16
\\
\sphinxhline\begin{itemize}
\item {} 
\end{itemize}
&&&&&&&\\
\sphinxhline
\sphinxAtStartPar
15
&
\sphinxAtStartPar
14
&
\sphinxAtStartPar
13
&
\sphinxAtStartPar
12
&
\sphinxAtStartPar
11
&
\sphinxAtStartPar
10
&
\sphinxAtStartPar
9
&
\sphinxAtStartPar
8
\\
\sphinxhline
\sphinxAtStartPar
PIN15
&
\sphinxAtStartPar
PIN14
&&&&&&\\
\sphinxhline
\sphinxAtStartPar
7
&
\sphinxAtStartPar
6
&
\sphinxAtStartPar
5
&
\sphinxAtStartPar
4
&
\sphinxAtStartPar
3
&
\sphinxAtStartPar
2
&
\sphinxAtStartPar
1
&
\sphinxAtStartPar
0
\\
\sphinxhline
\sphinxAtStartPar
PIN7
&
\sphinxAtStartPar
PIN6
&&&&&&\\
\sphinxbottomrule
\end{tabular}
\sphinxtableafterendhook\par
\sphinxattableend\end{savenotes}


\begin{savenotes}\sphinxattablestart
\sphinxthistablewithglobalstyle
\centering
\begin{tabular}[t]{\X{33}{99}\X{33}{99}\X{33}{99}}
\sphinxtoprule
\sphinxtableatstartofbodyhook
\sphinxAtStartPar
位域 |
&
\sphinxAtStartPar
名称     | |
&
\sphinxAtStartPar
描述                                        | |
\\
\sphinxhline
\sphinxAtStartPar
31:16
&\begin{itemize}
\item {} 
\end{itemize}
&\begin{itemize}
\item {} 
\end{itemize}
\\
\sphinxhline
\sphinxAtStartPar
15
&
\sphinxAtStartPar
PIN15
&
\sphinxAtStartPar
Px15引脚中断清除寄存器位,写1清除中断       |
\\
\sphinxhline
\sphinxAtStartPar
14
&
\sphinxAtStartPar
PIN14
&
\sphinxAtStartPar
Px14引脚中断清除寄存器位,写1清除中断       |
\\
\sphinxhline
\sphinxAtStartPar
13
&
\sphinxAtStartPar
PIN13
&
\sphinxAtStartPar
Px13引脚中断清除寄存器位,写1清除中断       |
\\
\sphinxhline
\sphinxAtStartPar
12
&
\sphinxAtStartPar
PIN12
&
\sphinxAtStartPar
Px12引脚中断清除寄存器位,写1清除中断       |
\\
\sphinxhline
\sphinxAtStartPar
11
&
\sphinxAtStartPar
PIN11
&
\sphinxAtStartPar
Px11引脚中断清除寄存器位,写1清除中断       |
\\
\sphinxhline
\sphinxAtStartPar
10
&
\sphinxAtStartPar
PIN10
&
\sphinxAtStartPar
Px10引脚中断清除寄存器位,写1清除中断       |
\\
\sphinxhline
\sphinxAtStartPar
9
&
\sphinxAtStartPar
PIN9
&
\sphinxAtStartPar
Px9引脚中断清除寄存器位,写1清除中断        |
\\
\sphinxhline
\sphinxAtStartPar
8
&
\sphinxAtStartPar
PIN8
&
\sphinxAtStartPar
Px8引脚中断清除寄存器位,写1清除中断        |
\\
\sphinxhline
\sphinxAtStartPar
7
&
\sphinxAtStartPar
PIN7
&
\sphinxAtStartPar
Px7引脚中断清除寄存器位,写1清除中断        |
\\
\sphinxhline
\sphinxAtStartPar
6
&
\sphinxAtStartPar
PIN6
&
\sphinxAtStartPar
Px6引脚中断清除寄存器位,写1清除中断        |
\\
\sphinxhline
\sphinxAtStartPar
5
&
\sphinxAtStartPar
PIN5
&
\sphinxAtStartPar
Px5引脚中断清除寄存器位,写1清除中断        |
\\
\sphinxhline
\sphinxAtStartPar
4
&
\sphinxAtStartPar
PIN4
&
\sphinxAtStartPar
Px4引脚中断清除寄存器位,写1清除中断        |
\\
\sphinxhline
\sphinxAtStartPar
3
&
\sphinxAtStartPar
PIN3
&
\sphinxAtStartPar
Px3引脚中断清除寄存器位,写1清除中断        |
\\
\sphinxhline
\sphinxAtStartPar
2
&
\sphinxAtStartPar
PIN2
&
\sphinxAtStartPar
Px2引脚中断清除寄存器位,写1清除中断        |
\\
\sphinxhline
\sphinxAtStartPar
1
&
\sphinxAtStartPar
PIN1
&
\sphinxAtStartPar
Px1引脚中断清除寄存器位,写1清除中断        |
\\
\sphinxhline
\sphinxAtStartPar
0
&
\sphinxAtStartPar
PIN0
&
\sphinxAtStartPar
Px0引脚中断清除寄存器位,写1清除中断        |
\\
\sphinxbottomrule
\end{tabular}
\sphinxtableafterendhook\par
\sphinxattableend\end{savenotes}


\subsubsection{GPIO DMA触发使能寄存器DMAEN}
\label{\detokenize{SWM241/_u529f_u80fd_u63cf_u8ff0/_u901a_u7528IO:gpio-dmadmaen}}

\begin{savenotes}\sphinxattablestart
\sphinxthistablewithglobalstyle
\centering
\begin{tabular}[t]{\X{20}{100}\X{20}{100}\X{20}{100}\X{20}{100}\X{20}{100}}
\sphinxtoprule
\sphinxtableatstartofbodyhook
\sphinxAtStartPar
寄存器 |
&
\begin{DUlineblock}{0em}
\item[] 偏移 |
\end{DUlineblock}
&
\begin{DUlineblock}{0em}
\item[] 
\item[] {\color{red}\bfseries{}|}
\end{DUlineblock}
&
\sphinxAtStartPar
复位值 |    描 | |
&
\begin{DUlineblock}{0em}
\item[] |
  |
\end{DUlineblock}
\\
\sphinxhline
\sphinxAtStartPar
DMAEN
&
\sphinxAtStartPar
0x24
&&
\sphinxAtStartPar
0 000000
&
\sphinxAtStartPar
GPIO DMA使能寄存器         |
\\
\sphinxbottomrule
\end{tabular}
\sphinxtableafterendhook\par
\sphinxattableend\end{savenotes}


\begin{savenotes}\sphinxattablestart
\sphinxthistablewithglobalstyle
\centering
\begin{tabular}[t]{\X{12}{96}\X{12}{96}\X{12}{96}\X{12}{96}\X{12}{96}\X{12}{96}\X{12}{96}\X{12}{96}}
\sphinxtoprule
\sphinxtableatstartofbodyhook
\sphinxAtStartPar
31
&
\sphinxAtStartPar
30
&
\sphinxAtStartPar
29
&
\sphinxAtStartPar
28
&
\sphinxAtStartPar
27
&
\sphinxAtStartPar
26
&
\sphinxAtStartPar
25
&
\sphinxAtStartPar
24
\\
\sphinxhline\begin{itemize}
\item {} 
\end{itemize}
&&&&&&&\\
\sphinxhline
\sphinxAtStartPar
23
&
\sphinxAtStartPar
22
&
\sphinxAtStartPar
21
&
\sphinxAtStartPar
20
&
\sphinxAtStartPar
19
&
\sphinxAtStartPar
18
&
\sphinxAtStartPar
17
&
\sphinxAtStartPar
16
\\
\sphinxhline\begin{itemize}
\item {} 
\end{itemize}
&&&&&&&\\
\sphinxhline
\sphinxAtStartPar
15
&
\sphinxAtStartPar
14
&
\sphinxAtStartPar
13
&
\sphinxAtStartPar
12
&
\sphinxAtStartPar
11
&
\sphinxAtStartPar
10
&
\sphinxAtStartPar
9
&
\sphinxAtStartPar
8
\\
\sphinxhline
\sphinxAtStartPar
DMAEN
&&&&&&&\\
\sphinxhline
\sphinxAtStartPar
7
&
\sphinxAtStartPar
6
&
\sphinxAtStartPar
5
&
\sphinxAtStartPar
4
&
\sphinxAtStartPar
3
&
\sphinxAtStartPar
2
&
\sphinxAtStartPar
1
&
\sphinxAtStartPar
0
\\
\sphinxhline
\sphinxAtStartPar
DMAEN
&&&&&&&\\
\sphinxbottomrule
\end{tabular}
\sphinxtableafterendhook\par
\sphinxattableend\end{savenotes}


\begin{savenotes}\sphinxattablestart
\sphinxthistablewithglobalstyle
\centering
\begin{tabular}[t]{\X{33}{99}\X{33}{99}\X{33}{99}}
\sphinxtoprule
\sphinxtableatstartofbodyhook
\sphinxAtStartPar
位域 |
&
\sphinxAtStartPar
名称     | |
&
\sphinxAtStartPar
描述                                        | |
\\
\sphinxhline
\sphinxAtStartPar
31:16
&\begin{itemize}
\item {} 
\end{itemize}
&\begin{itemize}
\item {} 
\end{itemize}
\\
\sphinxhline
\sphinxAtStartPar
15:0
&
\sphinxAtStartPar
DMAEN
&
\sphinxAtStartPar
1:通过对应位的GPIO沿                       | 上升沿、下降沿、双沿)来触发DMA读取GPIO  | 数据寄存器                                  |

\sphinxAtStartPar
0:CPU读取GPIODATA寄存器                    |

\sphinxAtStartPar
注:沿状态通过INTBE和INTRISEEN寄存器来配置  |
\\
\sphinxbottomrule
\end{tabular}
\sphinxtableafterendhook\par
\sphinxattableend\end{savenotes}


\subsubsection{GPIOx读数据寄存器IDR}
\label{\detokenize{SWM241/_u529f_u80fd_u63cf_u8ff0/_u901a_u7528IO:gpioxidr}}

\begin{savenotes}\sphinxattablestart
\sphinxthistablewithglobalstyle
\centering
\begin{tabular}[t]{\X{20}{100}\X{20}{100}\X{20}{100}\X{20}{100}\X{20}{100}}
\sphinxtoprule
\sphinxtableatstartofbodyhook
\sphinxAtStartPar
寄存器 |
&
\begin{DUlineblock}{0em}
\item[] 偏移 |
\end{DUlineblock}
&
\begin{DUlineblock}{0em}
\item[] 
\item[] {\color{red}\bfseries{}|}
\end{DUlineblock}
&
\sphinxAtStartPar
复位值 |    描 | |
&
\begin{DUlineblock}{0em}
\item[] |
  |
\end{DUlineblock}
\\
\sphinxhline
\sphinxAtStartPar
IDR
&
\sphinxAtStartPar
0x30
&&
\sphinxAtStartPar
0 000000
&
\sphinxAtStartPar
GPIO读数据寄存器           |
\\
\sphinxbottomrule
\end{tabular}
\sphinxtableafterendhook\par
\sphinxattableend\end{savenotes}


\begin{savenotes}\sphinxattablestart
\sphinxthistablewithglobalstyle
\centering
\begin{tabular}[t]{\X{12}{96}\X{12}{96}\X{12}{96}\X{12}{96}\X{12}{96}\X{12}{96}\X{12}{96}\X{12}{96}}
\sphinxtoprule
\sphinxtableatstartofbodyhook
\sphinxAtStartPar
31
&
\sphinxAtStartPar
30
&
\sphinxAtStartPar
29
&
\sphinxAtStartPar
28
&
\sphinxAtStartPar
27
&
\sphinxAtStartPar
26
&
\sphinxAtStartPar
25
&
\sphinxAtStartPar
24
\\
\sphinxhline\begin{itemize}
\item {} 
\end{itemize}
&&&&&&&\\
\sphinxhline
\sphinxAtStartPar
23
&
\sphinxAtStartPar
22
&
\sphinxAtStartPar
21
&
\sphinxAtStartPar
20
&
\sphinxAtStartPar
19
&
\sphinxAtStartPar
18
&
\sphinxAtStartPar
17
&
\sphinxAtStartPar
16
\\
\sphinxhline\begin{itemize}
\item {} 
\end{itemize}
&&&&&&&\\
\sphinxhline
\sphinxAtStartPar
15
&
\sphinxAtStartPar
14
&
\sphinxAtStartPar
13
&
\sphinxAtStartPar
12
&
\sphinxAtStartPar
11
&
\sphinxAtStartPar
10
&
\sphinxAtStartPar
9
&
\sphinxAtStartPar
8
\\
\sphinxhline
\sphinxAtStartPar
PIN15
&
\sphinxAtStartPar
PIN14
&&&&&&\\
\sphinxhline
\sphinxAtStartPar
7
&
\sphinxAtStartPar
6
&
\sphinxAtStartPar
5
&
\sphinxAtStartPar
4
&
\sphinxAtStartPar
3
&
\sphinxAtStartPar
2
&
\sphinxAtStartPar
1
&
\sphinxAtStartPar
0
\\
\sphinxhline
\sphinxAtStartPar
PIN7
&
\sphinxAtStartPar
PIN6
&&&&&&\\
\sphinxbottomrule
\end{tabular}
\sphinxtableafterendhook\par
\sphinxattableend\end{savenotes}


\begin{savenotes}\sphinxattablestart
\sphinxthistablewithglobalstyle
\centering
\begin{tabular}[t]{\X{33}{99}\X{33}{99}\X{33}{99}}
\sphinxtoprule
\sphinxtableatstartofbodyhook
\sphinxAtStartPar
位域 |
&
\sphinxAtStartPar
名称     | |
&
\sphinxAtStartPar
描述                                        | |
\\
\sphinxhline
\sphinxAtStartPar
31:16
&\begin{itemize}
\item {} 
\end{itemize}
&\begin{itemize}
\item {} 
\end{itemize}
\\
\sphinxhline
\sphinxAtStartPar
15
&
\sphinxAtStartPar
PIN15
&
\sphinxAtStartPar
Px15引脚数据读寄存器位                      |
\\
\sphinxhline
\sphinxAtStartPar
14
&
\sphinxAtStartPar
PIN14
&
\sphinxAtStartPar
Px14引脚数据读寄存器位                      |
\\
\sphinxhline
\sphinxAtStartPar
13
&
\sphinxAtStartPar
PIN13
&
\sphinxAtStartPar
Px13引脚数据读寄存器位                      |
\\
\sphinxhline
\sphinxAtStartPar
12
&
\sphinxAtStartPar
PIN12
&
\sphinxAtStartPar
Px12引脚数据读寄存器位                      |
\\
\sphinxhline
\sphinxAtStartPar
11
&
\sphinxAtStartPar
PIN11
&
\sphinxAtStartPar
Px11引脚数据读寄存器位                      |
\\
\sphinxhline
\sphinxAtStartPar
10
&
\sphinxAtStartPar
PIN10
&
\sphinxAtStartPar
Px10引脚数据读寄存器位                      |
\\
\sphinxhline
\sphinxAtStartPar
9
&
\sphinxAtStartPar
PIN9
&
\sphinxAtStartPar
Px9引脚数据读寄存器位                       |
\\
\sphinxhline
\sphinxAtStartPar
8
&
\sphinxAtStartPar
PIN8
&
\sphinxAtStartPar
Px8引脚数据读寄存器位                       |
\\
\sphinxhline
\sphinxAtStartPar
7
&
\sphinxAtStartPar
PIN7
&
\sphinxAtStartPar
Px7引脚数据读寄存器位                       |
\\
\sphinxhline
\sphinxAtStartPar
6
&
\sphinxAtStartPar
PIN6
&
\sphinxAtStartPar
Px6引脚数据读寄存器位                       |
\\
\sphinxhline
\sphinxAtStartPar
5
&
\sphinxAtStartPar
PIN5
&
\sphinxAtStartPar
Px5引脚数据读寄存器位                       |
\\
\sphinxhline
\sphinxAtStartPar
4
&
\sphinxAtStartPar
PIN4
&
\sphinxAtStartPar
Px4引脚数据读寄存器位                       |
\\
\sphinxhline
\sphinxAtStartPar
3
&
\sphinxAtStartPar
PIN3
&
\sphinxAtStartPar
Px3引脚数据读寄存器位                       |
\\
\sphinxhline
\sphinxAtStartPar
2
&
\sphinxAtStartPar
PIN2
&
\sphinxAtStartPar
Px2引脚数据读寄存器位                       |
\\
\sphinxhline
\sphinxAtStartPar
1
&
\sphinxAtStartPar
PIN1
&
\sphinxAtStartPar
Px1引脚数据读寄存器位                       |
\\
\sphinxhline
\sphinxAtStartPar
0
&
\sphinxAtStartPar
PIN0
&
\sphinxAtStartPar
Px0引脚数据读寄存器位                       |
\\
\sphinxbottomrule
\end{tabular}
\sphinxtableafterendhook\par
\sphinxattableend\end{savenotes}


\subsubsection{GPIOx PINn数据寄存器DATAPINx(x = 0\textasciitilde{}15)}
\label{\detokenize{SWM241/_u529f_u80fd_u63cf_u8ff0/_u901a_u7528IO:gpiox-pinndatapinx-x-0-15}}

\begin{savenotes}\sphinxattablestart
\sphinxthistablewithglobalstyle
\centering
\begin{tabular}[t]{\X{20}{100}\X{20}{100}\X{20}{100}\X{20}{100}\X{20}{100}}
\sphinxtoprule
\sphinxtableatstartofbodyhook
\sphinxAtStartPar
寄存器 |
&
\begin{DUlineblock}{0em}
\item[] 偏移 |
\end{DUlineblock}
&
\begin{DUlineblock}{0em}
\item[] 
\item[] {\color{red}\bfseries{}|}
\end{DUlineblock}
&
\sphinxAtStartPar
复位值 |    描 | |
&
\begin{DUlineblock}{0em}
\item[] |
  |
\end{DUlineblock}
\\
\sphinxhline
\sphinxAtStartPar
DATAPIN0
&
\sphinxAtStartPar
0x40
&&
\sphinxAtStartPar
0 000000
&
\sphinxAtStartPar
GPIO PIN0数据寄存器        |
\\
\sphinxbottomrule
\end{tabular}
\sphinxtableafterendhook\par
\sphinxattableend\end{savenotes}


\begin{savenotes}\sphinxattablestart
\sphinxthistablewithglobalstyle
\centering
\begin{tabular}[t]{\X{20}{100}\X{20}{100}\X{20}{100}\X{20}{100}\X{20}{100}}
\sphinxtoprule
\sphinxtableatstartofbodyhook
\sphinxAtStartPar
寄存器 |
&
\begin{DUlineblock}{0em}
\item[] 偏移 |
\end{DUlineblock}
&
\begin{DUlineblock}{0em}
\item[] 
\item[] {\color{red}\bfseries{}|}
\end{DUlineblock}
&
\sphinxAtStartPar
复位值 |    描 | |
&
\begin{DUlineblock}{0em}
\item[] |
  |
\end{DUlineblock}
\\
\sphinxhline
\sphinxAtStartPar
DATAPIN1
&
\sphinxAtStartPar
0x44
&&
\sphinxAtStartPar
0 000000
&
\sphinxAtStartPar
GPIO PIN1数据寄存器        |
\\
\sphinxbottomrule
\end{tabular}
\sphinxtableafterendhook\par
\sphinxattableend\end{savenotes}


\begin{savenotes}\sphinxattablestart
\sphinxthistablewithglobalstyle
\centering
\begin{tabular}[t]{\X{20}{100}\X{20}{100}\X{20}{100}\X{20}{100}\X{20}{100}}
\sphinxtoprule
\sphinxtableatstartofbodyhook
\sphinxAtStartPar
寄存器 |
&
\begin{DUlineblock}{0em}
\item[] 偏移 |
\end{DUlineblock}
&
\begin{DUlineblock}{0em}
\item[] 
\item[] {\color{red}\bfseries{}|}
\end{DUlineblock}
&
\sphinxAtStartPar
复位值 |    描 | |
&
\begin{DUlineblock}{0em}
\item[] |
  |
\end{DUlineblock}
\\
\sphinxhline
\sphinxAtStartPar
DATAPIN2
&
\sphinxAtStartPar
0x48
&&
\sphinxAtStartPar
0 000000
&
\sphinxAtStartPar
GPIO PIN2数据寄存器        |
\\
\sphinxbottomrule
\end{tabular}
\sphinxtableafterendhook\par
\sphinxattableend\end{savenotes}


\begin{savenotes}\sphinxattablestart
\sphinxthistablewithglobalstyle
\centering
\begin{tabular}[t]{\X{20}{100}\X{20}{100}\X{20}{100}\X{20}{100}\X{20}{100}}
\sphinxtoprule
\sphinxtableatstartofbodyhook
\sphinxAtStartPar
寄存器 |
&
\begin{DUlineblock}{0em}
\item[] 偏移 |
\end{DUlineblock}
&
\begin{DUlineblock}{0em}
\item[] 
\item[] {\color{red}\bfseries{}|}
\end{DUlineblock}
&
\sphinxAtStartPar
复位值 |    描 | |
&
\begin{DUlineblock}{0em}
\item[] |
  |
\end{DUlineblock}
\\
\sphinxhline
\sphinxAtStartPar
DATAPIN3
&
\sphinxAtStartPar
0x4C
&&
\sphinxAtStartPar
0 000000
&
\sphinxAtStartPar
GPIO PIN3数据寄存器        |
\\
\sphinxbottomrule
\end{tabular}
\sphinxtableafterendhook\par
\sphinxattableend\end{savenotes}


\begin{savenotes}\sphinxattablestart
\sphinxthistablewithglobalstyle
\centering
\begin{tabular}[t]{\X{20}{100}\X{20}{100}\X{20}{100}\X{20}{100}\X{20}{100}}
\sphinxtoprule
\sphinxtableatstartofbodyhook
\sphinxAtStartPar
寄存器 |
&
\begin{DUlineblock}{0em}
\item[] 偏移 |
\end{DUlineblock}
&
\begin{DUlineblock}{0em}
\item[] 
\item[] {\color{red}\bfseries{}|}
\end{DUlineblock}
&
\sphinxAtStartPar
复位值 |    描 | |
&
\begin{DUlineblock}{0em}
\item[] |
  |
\end{DUlineblock}
\\
\sphinxhline
\sphinxAtStartPar
DATAPIN4
&
\sphinxAtStartPar
0x50
&&
\sphinxAtStartPar
0 000000
&
\sphinxAtStartPar
GPIO PIN4数据寄存器        |
\\
\sphinxbottomrule
\end{tabular}
\sphinxtableafterendhook\par
\sphinxattableend\end{savenotes}


\begin{savenotes}\sphinxattablestart
\sphinxthistablewithglobalstyle
\centering
\begin{tabular}[t]{\X{20}{100}\X{20}{100}\X{20}{100}\X{20}{100}\X{20}{100}}
\sphinxtoprule
\sphinxtableatstartofbodyhook
\sphinxAtStartPar
寄存器 |
&
\begin{DUlineblock}{0em}
\item[] 偏移 |
\end{DUlineblock}
&
\begin{DUlineblock}{0em}
\item[] 
\item[] {\color{red}\bfseries{}|}
\end{DUlineblock}
&
\sphinxAtStartPar
复位值 |    描 | |
&
\begin{DUlineblock}{0em}
\item[] |
  |
\end{DUlineblock}
\\
\sphinxhline
\sphinxAtStartPar
DATAPIN5
&
\sphinxAtStartPar
0x54
&&
\sphinxAtStartPar
0 000000
&
\sphinxAtStartPar
GPIO PIN5数据寄存器        |
\\
\sphinxbottomrule
\end{tabular}
\sphinxtableafterendhook\par
\sphinxattableend\end{savenotes}


\begin{savenotes}\sphinxattablestart
\sphinxthistablewithglobalstyle
\centering
\begin{tabular}[t]{\X{20}{100}\X{20}{100}\X{20}{100}\X{20}{100}\X{20}{100}}
\sphinxtoprule
\sphinxtableatstartofbodyhook
\sphinxAtStartPar
寄存器 |
&
\begin{DUlineblock}{0em}
\item[] 偏移 |
\end{DUlineblock}
&
\begin{DUlineblock}{0em}
\item[] 
\item[] {\color{red}\bfseries{}|}
\end{DUlineblock}
&
\sphinxAtStartPar
复位值 |    描 | |
&
\begin{DUlineblock}{0em}
\item[] |
  |
\end{DUlineblock}
\\
\sphinxhline
\sphinxAtStartPar
DATAPIN6
&
\sphinxAtStartPar
0x58
&&
\sphinxAtStartPar
0 000000
&
\sphinxAtStartPar
GPIO PIN6数据寄存器        |
\\
\sphinxbottomrule
\end{tabular}
\sphinxtableafterendhook\par
\sphinxattableend\end{savenotes}


\begin{savenotes}\sphinxattablestart
\sphinxthistablewithglobalstyle
\centering
\begin{tabular}[t]{\X{20}{100}\X{20}{100}\X{20}{100}\X{20}{100}\X{20}{100}}
\sphinxtoprule
\sphinxtableatstartofbodyhook
\sphinxAtStartPar
寄存器 |
&
\begin{DUlineblock}{0em}
\item[] 偏移 |
\end{DUlineblock}
&
\begin{DUlineblock}{0em}
\item[] 
\item[] {\color{red}\bfseries{}|}
\end{DUlineblock}
&
\sphinxAtStartPar
复位值 |    描 | |
&
\begin{DUlineblock}{0em}
\item[] |
  |
\end{DUlineblock}
\\
\sphinxhline
\sphinxAtStartPar
DATAPIN7
&
\sphinxAtStartPar
0x5C
&&
\sphinxAtStartPar
0 000000
&
\sphinxAtStartPar
GPIO PIN7数据寄存器        |
\\
\sphinxbottomrule
\end{tabular}
\sphinxtableafterendhook\par
\sphinxattableend\end{savenotes}


\begin{savenotes}\sphinxattablestart
\sphinxthistablewithglobalstyle
\centering
\begin{tabular}[t]{\X{20}{100}\X{20}{100}\X{20}{100}\X{20}{100}\X{20}{100}}
\sphinxtoprule
\sphinxtableatstartofbodyhook
\sphinxAtStartPar
寄存器 |
&
\begin{DUlineblock}{0em}
\item[] 偏移 |
\end{DUlineblock}
&
\begin{DUlineblock}{0em}
\item[] 
\item[] {\color{red}\bfseries{}|}
\end{DUlineblock}
&
\sphinxAtStartPar
复位值 |    描 | |
&
\begin{DUlineblock}{0em}
\item[] |
  |
\end{DUlineblock}
\\
\sphinxhline
\sphinxAtStartPar
DATAPIN8
&
\sphinxAtStartPar
0x60
&&
\sphinxAtStartPar
0 000000
&
\sphinxAtStartPar
GPIO PIN8数据寄存器        |
\\
\sphinxbottomrule
\end{tabular}
\sphinxtableafterendhook\par
\sphinxattableend\end{savenotes}


\begin{savenotes}\sphinxattablestart
\sphinxthistablewithglobalstyle
\centering
\begin{tabular}[t]{\X{20}{100}\X{20}{100}\X{20}{100}\X{20}{100}\X{20}{100}}
\sphinxtoprule
\sphinxtableatstartofbodyhook
\sphinxAtStartPar
寄存器 |
&
\begin{DUlineblock}{0em}
\item[] 偏移 |
\end{DUlineblock}
&
\begin{DUlineblock}{0em}
\item[] 
\item[] {\color{red}\bfseries{}|}
\end{DUlineblock}
&
\sphinxAtStartPar
复位值 |    描 | |
&
\begin{DUlineblock}{0em}
\item[] |
  |
\end{DUlineblock}
\\
\sphinxhline
\sphinxAtStartPar
DATAPIN9
&
\sphinxAtStartPar
0x64
&&
\sphinxAtStartPar
0 000000
&
\sphinxAtStartPar
GPIO PIN9数据寄存器        |
\\
\sphinxbottomrule
\end{tabular}
\sphinxtableafterendhook\par
\sphinxattableend\end{savenotes}


\begin{savenotes}\sphinxattablestart
\sphinxthistablewithglobalstyle
\centering
\begin{tabular}[t]{\X{20}{100}\X{20}{100}\X{20}{100}\X{20}{100}\X{20}{100}}
\sphinxtoprule
\sphinxtableatstartofbodyhook
\sphinxAtStartPar
寄存器 |
&
\begin{DUlineblock}{0em}
\item[] 偏移 |
\end{DUlineblock}
&
\begin{DUlineblock}{0em}
\item[] 
\item[] {\color{red}\bfseries{}|}
\end{DUlineblock}
&
\sphinxAtStartPar
复位值 |    描 | |
&
\begin{DUlineblock}{0em}
\item[] |
  |
\end{DUlineblock}
\\
\sphinxhline
\sphinxAtStartPar
DATAPIN10
&
\sphinxAtStartPar
0x68
&&
\sphinxAtStartPar
0 000000
&
\sphinxAtStartPar
GPIO PIN10数据寄存器       |
\\
\sphinxbottomrule
\end{tabular}
\sphinxtableafterendhook\par
\sphinxattableend\end{savenotes}


\begin{savenotes}\sphinxattablestart
\sphinxthistablewithglobalstyle
\centering
\begin{tabular}[t]{\X{20}{100}\X{20}{100}\X{20}{100}\X{20}{100}\X{20}{100}}
\sphinxtoprule
\sphinxtableatstartofbodyhook
\sphinxAtStartPar
寄存器 |
&
\begin{DUlineblock}{0em}
\item[] 偏移 |
\end{DUlineblock}
&
\begin{DUlineblock}{0em}
\item[] 
\item[] {\color{red}\bfseries{}|}
\end{DUlineblock}
&
\sphinxAtStartPar
复位值 |    描 | |
&
\begin{DUlineblock}{0em}
\item[] |
  |
\end{DUlineblock}
\\
\sphinxhline
\sphinxAtStartPar
DATAPIN11
&
\sphinxAtStartPar
0x6C
&&
\sphinxAtStartPar
0 000000
&
\sphinxAtStartPar
GPIO PIN11数据寄存器       |
\\
\sphinxbottomrule
\end{tabular}
\sphinxtableafterendhook\par
\sphinxattableend\end{savenotes}


\begin{savenotes}\sphinxattablestart
\sphinxthistablewithglobalstyle
\centering
\begin{tabular}[t]{\X{20}{100}\X{20}{100}\X{20}{100}\X{20}{100}\X{20}{100}}
\sphinxtoprule
\sphinxtableatstartofbodyhook
\sphinxAtStartPar
寄存器 |
&
\begin{DUlineblock}{0em}
\item[] 偏移 |
\end{DUlineblock}
&
\begin{DUlineblock}{0em}
\item[] 
\item[] {\color{red}\bfseries{}|}
\end{DUlineblock}
&
\sphinxAtStartPar
复位值 |    描 | |
&
\begin{DUlineblock}{0em}
\item[] |
  |
\end{DUlineblock}
\\
\sphinxhline
\sphinxAtStartPar
DATAPIN12
&
\sphinxAtStartPar
0x70
&&
\sphinxAtStartPar
0 000000
&
\sphinxAtStartPar
GPIO PIN12数据寄存器       |
\\
\sphinxbottomrule
\end{tabular}
\sphinxtableafterendhook\par
\sphinxattableend\end{savenotes}


\begin{savenotes}\sphinxattablestart
\sphinxthistablewithglobalstyle
\centering
\begin{tabular}[t]{\X{20}{100}\X{20}{100}\X{20}{100}\X{20}{100}\X{20}{100}}
\sphinxtoprule
\sphinxtableatstartofbodyhook
\sphinxAtStartPar
寄存器 |
&
\begin{DUlineblock}{0em}
\item[] 偏移 |
\end{DUlineblock}
&
\begin{DUlineblock}{0em}
\item[] 
\item[] {\color{red}\bfseries{}|}
\end{DUlineblock}
&
\sphinxAtStartPar
复位值 |    描 | |
&
\begin{DUlineblock}{0em}
\item[] |
  |
\end{DUlineblock}
\\
\sphinxhline
\sphinxAtStartPar
DATAPIN13
&
\sphinxAtStartPar
0x74
&&
\sphinxAtStartPar
0 000000
&
\sphinxAtStartPar
GPIO PIN13数据寄存器       |
\\
\sphinxbottomrule
\end{tabular}
\sphinxtableafterendhook\par
\sphinxattableend\end{savenotes}


\begin{savenotes}\sphinxattablestart
\sphinxthistablewithglobalstyle
\centering
\begin{tabular}[t]{\X{20}{100}\X{20}{100}\X{20}{100}\X{20}{100}\X{20}{100}}
\sphinxtoprule
\sphinxtableatstartofbodyhook
\sphinxAtStartPar
寄存器 |
&
\begin{DUlineblock}{0em}
\item[] 偏移 |
\end{DUlineblock}
&
\begin{DUlineblock}{0em}
\item[] 
\item[] {\color{red}\bfseries{}|}
\end{DUlineblock}
&
\sphinxAtStartPar
复位值 |    描 | |
&
\begin{DUlineblock}{0em}
\item[] |
  |
\end{DUlineblock}
\\
\sphinxhline
\sphinxAtStartPar
DATAPIN14
&
\sphinxAtStartPar
0x78
&&
\sphinxAtStartPar
0 000000
&
\sphinxAtStartPar
GPIO PIN14数据寄存器       |
\\
\sphinxbottomrule
\end{tabular}
\sphinxtableafterendhook\par
\sphinxattableend\end{savenotes}


\begin{savenotes}\sphinxattablestart
\sphinxthistablewithglobalstyle
\centering
\begin{tabular}[t]{\X{20}{100}\X{20}{100}\X{20}{100}\X{20}{100}\X{20}{100}}
\sphinxtoprule
\sphinxtableatstartofbodyhook
\sphinxAtStartPar
寄存器 |
&
\begin{DUlineblock}{0em}
\item[] 偏移 |
\end{DUlineblock}
&
\begin{DUlineblock}{0em}
\item[] 
\item[] {\color{red}\bfseries{}|}
\end{DUlineblock}
&
\sphinxAtStartPar
复位值 |    描 | |
&
\begin{DUlineblock}{0em}
\item[] |
  |
\end{DUlineblock}
\\
\sphinxhline
\sphinxAtStartPar
DATAPIN15
&
\sphinxAtStartPar
0x7C
&&
\sphinxAtStartPar
0 000000
&
\sphinxAtStartPar
GPIO PIN15数据寄存器       |
\\
\sphinxbottomrule
\end{tabular}
\sphinxtableafterendhook\par
\sphinxattableend\end{savenotes}


\begin{savenotes}\sphinxattablestart
\sphinxthistablewithglobalstyle
\centering
\begin{tabular}[t]{\X{12}{96}\X{12}{96}\X{12}{96}\X{12}{96}\X{12}{96}\X{12}{96}\X{12}{96}\X{12}{96}}
\sphinxtoprule
\sphinxtableatstartofbodyhook
\sphinxAtStartPar
31
&
\sphinxAtStartPar
30
&
\sphinxAtStartPar
29
&
\sphinxAtStartPar
28
&
\sphinxAtStartPar
27
&
\sphinxAtStartPar
26
&
\sphinxAtStartPar
25
&
\sphinxAtStartPar
24
\\
\sphinxhline\begin{itemize}
\item {} 
\end{itemize}
&&&&&&&\\
\sphinxhline
\sphinxAtStartPar
23
&
\sphinxAtStartPar
22
&
\sphinxAtStartPar
21
&
\sphinxAtStartPar
20
&
\sphinxAtStartPar
19
&
\sphinxAtStartPar
18
&
\sphinxAtStartPar
17
&
\sphinxAtStartPar
16
\\
\sphinxhline\begin{itemize}
\item {} 
\end{itemize}
&&&&&&&\\
\sphinxhline
\sphinxAtStartPar
15
&
\sphinxAtStartPar
14
&
\sphinxAtStartPar
13
&
\sphinxAtStartPar
12
&
\sphinxAtStartPar
11
&
\sphinxAtStartPar
10
&
\sphinxAtStartPar
9
&
\sphinxAtStartPar
8
\\
\sphinxhline\begin{itemize}
\item {} 
\end{itemize}
&&&&&&&\\
\sphinxhline
\sphinxAtStartPar
7
&
\sphinxAtStartPar
6
&
\sphinxAtStartPar
5
&
\sphinxAtStartPar
4
&
\sphinxAtStartPar
3
&
\sphinxAtStartPar
2
&
\sphinxAtStartPar
1
&
\sphinxAtStartPar
0
\\
\sphinxhline\begin{itemize}
\item {} 
\end{itemize}
&&&&&&&
\sphinxAtStartPar
DA INi
\\
\sphinxbottomrule
\end{tabular}
\sphinxtableafterendhook\par
\sphinxattableend\end{savenotes}


\begin{savenotes}\sphinxattablestart
\sphinxthistablewithglobalstyle
\centering
\begin{tabular}[t]{\X{33}{99}\X{33}{99}\X{33}{99}}
\sphinxtoprule
\sphinxtableatstartofbodyhook
\sphinxAtStartPar
位域 |
&
\sphinxAtStartPar
名称     | |
&
\sphinxAtStartPar
描述                                        | |
\\
\sphinxhline
\sphinxAtStartPar
31:1
&\begin{itemize}
\item {} 
\end{itemize}
&\begin{itemize}
\item {} 
\end{itemize}
\\
\sphinxhline
\sphinxAtStartPar
0
&
\sphinxAtStartPar
DATAPINi
&
\sphinxAtStartPar
GPIOx PINn数据寄存器。                      |

\sphinxAtStartPar
读:GPIOx PINn的输入数据                    |

\sphinxAtStartPar
写:GPIOx PINn的输出数据                    |
\\
\sphinxbottomrule
\end{tabular}
\sphinxtableafterendhook\par
\sphinxattableend\end{savenotes}

\sphinxstepscope


\section{加强型定时器(TIMER)}
\label{\detokenize{SWM241/_u529f_u80fd_u63cf_u8ff0/_u52a0_u5f3a_u578b_u5b9a_u65f6_u5668:timer}}\label{\detokenize{SWM241/_u529f_u80fd_u63cf_u8ff0/_u52a0_u5f3a_u578b_u5b9a_u65f6_u5668::doc}}
\sphinxAtStartPar
概述
\textasciitilde{}\textasciitilde{}

\sphinxAtStartPar
SWM241系列所有型号TIMER操作均相同,不同型号具备TIMER数量可能不同。使用前需使能TIMER模块时钟。

\sphinxAtStartPar
每个TIMER模块均具备定时器功能(使用片内时钟作为计数基准)和计数器功能(使用片外时钟作为计数基准)、输出比较及输入捕获功能。

\sphinxAtStartPar
TIMER0支持Hall功能及连续脉宽捕捉功能。

\sphinxAtStartPar
特性
\textasciitilde{}\textasciitilde{}
\begin{itemize}
\item {} 
\sphinxAtStartPar
8路32位通用定时器
\begin{itemize}
\item {} 
\sphinxAtStartPar
24位计数器

\item {} 
\sphinxAtStartPar
8位预分频

\end{itemize}

\item {} 
\sphinxAtStartPar
可单独配置计时触发条件为内部时钟或者外部输入

\item {} 
\sphinxAtStartPar
支持脉冲捕获及宽度测量,检测脉冲极性可配

\item {} 
\sphinxAtStartPar
支持脉冲发送功能,可作为PWM使用

\item {} 
\sphinxAtStartPar
TIMER0支持HALL功能,可采集霍尔传感器信号

\item {} 
\sphinxAtStartPar
TIMER0\textasciitilde{}1输出可作为外部触发事件信号

\item {} 
\sphinxAtStartPar
定时器溢出脉冲输出,可用于触发ADC

\end{itemize}


\subsection{模块结构框图}
\label{\detokenize{SWM241/_u529f_u80fd_u63cf_u8ff0/_u52a0_u5f3a_u578b_u5b9a_u65f6_u5668:id1}}
\sphinxAtStartPar
\sphinxincludegraphics{{SWM241/功能描述/media加强型定时器002}.emf}

\sphinxAtStartPar
图 6‑7 TIMER 模块结构框图


\subsection{功能描述}
\label{\detokenize{SWM241/_u529f_u80fd_u63cf_u8ff0/_u52a0_u5f3a_u578b_u5b9a_u65f6_u5668:id2}}

\subsubsection{定时器}
\label{\detokenize{SWM241/_u529f_u80fd_u63cf_u8ff0/_u52a0_u5f3a_u578b_u5b9a_u65f6_u5668:id3}}
\sphinxAtStartPar
使用TIMERx作为定时器时,为递减计数。流程如下:
\begin{itemize}
\item {} 
\sphinxAtStartPar
将控制寄存器(CRx)中MODx位配置为定时器,CLKSRCx位配置计数源选择,配置为使用系统时钟作为计数源。

\item {} 
\sphinxAtStartPar
通过预分频寄存器(PSCx)配置定时器时钟分频值,装载值寄存器(LOADx)设置计数起始值。

\item {} 
\sphinxAtStartPar
使能寄存器(EN)对应位使能为1。

\item {} 
\sphinxAtStartPar
对应TIMERx开始递减计数,计数到0时,产生中断,同时重新装载计数值,进行下一周期计数。

\end{itemize}

\sphinxAtStartPar
在计数过程中,可通过对当前值寄存器(VALUEx)进行读取,获取当前计数值。

\sphinxAtStartPar
定时器计数过程中改变装载值寄存器(LOADx)值,将在下个计数周期(计数到0重新装载)生效,不会改变本周期计数值。

\sphinxAtStartPar
定时器计数过程中,可以通过BRK寄存器控制位置1暂停指定通道计数,置0后继续计数。

\sphinxAtStartPar
如图 6‑8所示。

\sphinxAtStartPar
\sphinxincludegraphics{{SWM241/功能描述/media加强型定时器003}.emf}

\sphinxAtStartPar
图 6‑8定时器工作示意图

\sphinxAtStartPar
注:如图 6‑8中CLK不是系统时钟,是系统时钟经过PSC分频之后的时钟

\sphinxAtStartPar
计数器
\textasciicircum{}\textasciicircum{}\textasciicircum{}

\sphinxAtStartPar
使用TIMERx作为计数器时,为递减计数。流程如下:
\begin{itemize}
\item {} 
\sphinxAtStartPar
将控制寄存器(CRx)中MODEx位配置为定时器模式,CLKSRCx位配置计数源选择,配置为使用外部的cntsrc的上升沿。此时,对应TIMER将以配置为CNT引脚外部输入的上升沿作为计数目标。

\item {} 
\sphinxAtStartPar
针对外部信号输入引脚进行如下操作:
\begin{itemize}
\item {} 
\sphinxAtStartPar
配置PORTCON模块中INEN寄存器使能引脚输入功能。

\item {} 
\sphinxAtStartPar
通过PORT\_FUNC寄存器将引脚切换为指定数字功能。

\end{itemize}

\item {} 
\sphinxAtStartPar
通过装载值寄存器(LOADx)设置计数目标值。

\item {} 
\sphinxAtStartPar
使能寄存器(EN)对应位使能为1,对应TIMERx开始递减计数,计数到0时,产生中断,同时重新装载计数值,进行下一周期计数。

\end{itemize}

\sphinxAtStartPar
在计数过程中,可通过对当前值寄存器(CVALx)进行读取,获取当前计数值。

\sphinxAtStartPar
定时器计数过程中改变装载值寄存器(LOADx)值,将在下个计数周期(计数到0重新装载)生效,不会改变本周期计数值。

\sphinxAtStartPar
计数器使用过程中,可以通过HALT寄存器控制位置1暂停指定通道计数,置0后继续计数。

\sphinxAtStartPar
示意图如图 6‑9所示。

\sphinxAtStartPar
\sphinxincludegraphics{{SWM241/功能描述/media加强型定时器004}.emf}

\sphinxAtStartPar
图 6‑9计数器工作示意图


\subsubsection{级联}
\label{\detokenize{SWM241/_u529f_u80fd_u63cf_u8ff0/_u52a0_u5f3a_u578b_u5b9a_u65f6_u5668:id4}}
\sphinxAtStartPar
当TIMER无法满足计数宽度或时间长度时,可以通过级联方式,使计数周期为TIMER位宽相乘的关系。最高支持两级级联。

\sphinxAtStartPar
使用方式如下:
\begin{itemize}
\item {} 
\sphinxAtStartPar
TIMERn根据需要设置为定时器或计数器模式

\item {} 
\sphinxAtStartPar
TIMERn+1设置为级联模式(CLKSRCx位配置为使用上一路计数器的进位标志)

\item {} 
\sphinxAtStartPar
LOADn = 目标计数值A

\item {} 
\sphinxAtStartPar
LOADn+1 = 目标计数值B,总计数周期为A*B

\item {} 
\sphinxAtStartPar
使能TIMERn+1中断

\item {} 
\sphinxAtStartPar
使能TIMERn+1

\item {} 
\sphinxAtStartPar
使能TIMERn

\item {} 
\sphinxAtStartPar
TIMERn+1中断产生,在中断程序中使能TIMERn中断

\item {} 
\sphinxAtStartPar
TIMERn中断产生,计数周期完成

\end{itemize}

\sphinxAtStartPar
示意图如图 6‑10所示:

\sphinxAtStartPar
\sphinxincludegraphics{{SWM241/功能描述/media加强型定时器005}.emf}

\sphinxAtStartPar
图 6‑10级联模式工作示意图


\subsubsection{脉冲发送}
\label{\detokenize{SWM241/_u529f_u80fd_u63cf_u8ff0/_u52a0_u5f3a_u578b_u5b9a_u65f6_u5668:id5}}
\sphinxAtStartPar
所有TIMER模块均支持脉冲发送功能,可发送指定脉宽的方波。该计数器为向下计数。使用方式如下:
\begin{itemize}
\item {} 
\sphinxAtStartPar
针对外部信号输入引脚进行如下操作
\begin{itemize}
\item {} 
\sphinxAtStartPar
配置PORTCON模块中使能引脚输出功能

\item {} 
\sphinxAtStartPar
通过PORT\_FUNC寄存器将引脚切换为TIMER对应数字功能

\end{itemize}

\item {} 
\sphinxAtStartPar
TIMER初始化
\begin{itemize}
\item {} 
\sphinxAtStartPar
指定要被设置的定时器

\item {} 
\sphinxAtStartPar
设置TIMER的工作模式为OC(输出比较)模式

\item {} 
\sphinxAtStartPar
设置定时周期

\end{itemize}

\item {} 
\sphinxAtStartPar
输出比较功能初始化
\begin{itemize}
\item {} 
\sphinxAtStartPar
指定要被设置的定时器

\item {} 
\sphinxAtStartPar
设置当计数器的值递减到MATCH时引脚输出电平翻转

\item {} 
\sphinxAtStartPar
设置初始输出电平

\end{itemize}

\item {} 
\sphinxAtStartPar
启动定时器

\end{itemize}

\sphinxAtStartPar
示意图如图 6‑11所示:

\sphinxAtStartPar
\sphinxincludegraphics{{SWM241/功能描述/media加强型定时器006}.emf}

\sphinxAtStartPar
图 6‑11 脉冲发送示意图


\subsubsection{脉冲捕捉}
\label{\detokenize{SWM241/_u529f_u80fd_u63cf_u8ff0/_u52a0_u5f3a_u578b_u5b9a_u65f6_u5668:id6}}
\sphinxAtStartPar
所有TIMER模块均支持用于捕捉外部脉宽,可记录外部单个脉冲宽度。

\sphinxAtStartPar
使用方式如下:
\begin{itemize}
\item {} 
\sphinxAtStartPar
针对外部信号输入引脚进行如下操作
\begin{itemize}
\item {} 
\sphinxAtStartPar
配置PORTCON模块中INEN寄存器使能引脚输入功能

\item {} 
\sphinxAtStartPar
通过PORT\_FUNC寄存器将引脚切换为TIMER对应数字功能

\end{itemize}

\item {} 
\sphinxAtStartPar
设置中断使能寄存器(IEx),使能中断

\item {} 
\sphinxAtStartPar
使能寄存器(EN)对应位使能,启动捕捉功能

\item {} 
\sphinxAtStartPar
当指定引脚出现变化沿时,开始记录宽度,再次出现沿时,将对应种类的脉宽长度记录于ICLOWx或ICHIGHx寄存器,并产生中断。

\item {} 
\sphinxAtStartPar
进入中断,读取脉冲长度寄存器,获取指定种类的脉冲宽度

\item {} 
\sphinxAtStartPar
如果不操作EN位,则持续记录电平宽度,直至EN位关闭。

\end{itemize}

\sphinxAtStartPar
捕捉高电平示意图如图 6‑12所示。

\sphinxAtStartPar
\sphinxincludegraphics{{SWM241/功能描述/media加强型定时器007}.emf}

\sphinxAtStartPar
图 6‑12单次高电平捕捉示意图

\sphinxAtStartPar
低电平示意图如图 6‑13所示。

\sphinxAtStartPar
\sphinxincludegraphics{{SWM241/功能描述/media加强型定时器008}.emf}

\sphinxAtStartPar
图 6‑13单次低电平捕捉示意图


\subsubsection{霍尔接口}
\label{\detokenize{SWM241/_u529f_u80fd_u63cf_u8ff0/_u52a0_u5f3a_u578b_u5b9a_u65f6_u5668:id7}}
\sphinxAtStartPar
TIMER0模块提供了HALL接口功能,能够自动记录脉冲变化间隔,并产生中断,使用方式如下:
\begin{itemize}
\item {} 
\sphinxAtStartPar
HALL功能为指定引脚,且不同封装可能有所差异,具体引脚详见管脚排布:
\begin{itemize}
\item {} 
\sphinxAtStartPar
配置PORTCON模块中INEN寄存器使能引脚输入功能

\item {} 
\sphinxAtStartPar
通过PORT\_FUNC寄存器将引脚切换为HALL功能

\end{itemize}

\item {} 
\sphinxAtStartPar
配置HALLCR寄存器,设置对应管脚计数及中断产生条件,支持上升沿/下降沿/双沿产生中断

\item {} 
\sphinxAtStartPar
配置TIMER0装载值寄存器(LOADx)为0xFFFFFFFF

\item {} 
\sphinxAtStartPar
使能控制寄存器使能位(EN)

\item {} 
\sphinxAtStartPar
当外部HALL\_X引脚产生指定电平变化时,TIMER0计数值自动装载至HALL\_X(本次覆盖上次),并产生TIMER中断。同时HALLIF寄存器IFx将产生对应标示位,标识对应引脚产生电平变化。

\item {} 
\sphinxAtStartPar
当TIMER0记载至0时,将重新从0xFFFFFFFF计数

\end{itemize}

\sphinxAtStartPar
双边沿记录示意图如图 6‑14所示。

\sphinxAtStartPar
\sphinxincludegraphics{{SWM241/功能描述/media加强型定时器009}.emf}

\sphinxAtStartPar
图 6‑14 Hall双边沿记录示意图


\subsubsection{ADC采样触发功能}
\label{\detokenize{SWM241/_u529f_u80fd_u63cf_u8ff0/_u52a0_u5f3a_u578b_u5b9a_u65f6_u5668:adc}}
\sphinxAtStartPar
TIMER2/3支持SAR ADC触发功能

\sphinxAtStartPar
对于SAR ADC,配置ADC完成后,将寄存器(CTRL)中TRIG设置为TIMER2触发或TIMER3触发,则当对应TIMER计数值减至0时,将触发SAR ADC配置寄存器(CTRL)中选中的通道进行采样。可以通过ADC采样完成中断进行结果获取。

\sphinxAtStartPar
此功能配置为定时器或脉冲发送均有效。


\subsubsection{中断配置与清除}
\label{\detokenize{SWM241/_u529f_u80fd_u63cf_u8ff0/_u52a0_u5f3a_u578b_u5b9a_u65f6_u5668:id8}}
\sphinxAtStartPar
每路TIMER均具备独立中断,通过中断使能寄存器IE进行各TIMER中断使能。通过中断状态寄存器IF进行中断查询及清除。

\sphinxAtStartPar
TIMER中断

\sphinxAtStartPar
可通过配置中断使能寄存器IEx相应位使能中断。当检测到中断触发条件时,中断标志寄存器IFx相应位中置1。如需清除此标志,需在相应标志位中写1清零(R/W1C),否则中断在开启状态下会一直进入。

\sphinxAtStartPar
HALL中断

\sphinxAtStartPar
可通过配置HALL模式控制寄存器对应位设置输入HALLx信号触发中断的条件:上升沿、下降沿、上升沿和下降沿。可通过配置HALL中断使能寄存器HALLIE相应位使能中断。当检测到中断触发条件时,HALL中断标志寄存器HALLIF相应位中置1。如需清除此标志,需在相应标志位中写1清零(R/W1C),否则
中断在开启状态下会一直进入。


\subsection{寄存器映射}
\label{\detokenize{SWM241/_u529f_u80fd_u63cf_u8ff0/_u52a0_u5f3a_u578b_u5b9a_u65f6_u5668:id9}}

\begin{savenotes}\sphinxattablestart
\sphinxthistablewithglobalstyle
\centering
\begin{tabular}[t]{\X{20}{100}\X{20}{100}\X{20}{100}\X{20}{100}\X{20}{100}}
\sphinxtoprule
\sphinxtableatstartofbodyhook
\sphinxAtStartPar
名称   |
&
\begin{DUlineblock}{0em}
\item[] 偏移 |
\end{DUlineblock}
&
\begin{DUlineblock}{0em}
\item[] 
\end{DUlineblock}
&
\begin{DUlineblock}{0em}
\item[] 
\end{DUlineblock}
&
\sphinxAtStartPar
描述
\\
\sphinxhline
\sphinxAtStartPar
TIMER0
&&&&
\sphinxAtStartPar
BASE:0x40046800
\\
\sphinxhline
\sphinxAtStartPar
TIMER1   BASE:0x40046840
&&&&\\
\sphinxhline
\sphinxAtStartPar
TIMER2    BASE:0x40046880
&&&&\\
\sphinxhline
\sphinxAtStartPar
TIMER3    BASE:0x400468C0
&&&&\\
\sphinxhline
\sphinxAtStartPar
TIMER4   BASE:0x40046900
&&&&\\
\sphinxhline
\sphinxAtStartPar
TIMER5   BASE:0x40046940
&&&&\\
\sphinxhline
\sphinxAtStartPar
TIMER6   BASE:0x40046980
&&&&\\
\sphinxhline
\sphinxAtStartPar
TIMER7   BASE:0x400469C0
&&&&\\
\sphinxhline
\sphinxAtStartPar
LOADx
&
\sphinxAtStartPar
0x0
&&
\sphinxAtStartPar
0x 00000
&
\sphinxAtStartPar
TIMERx装载值寄存器         |
\\
\sphinxhline
\sphinxAtStartPar
VALUEx
&
\sphinxAtStartPar
0x4
&&
\sphinxAtStartPar
0x FFFFF
&
\sphinxAtStartPar
TIMERx当前计数值寄存器     |
\\
\sphinxhline
\sphinxAtStartPar
CRx
&
\sphinxAtStartPar
0x8
&&
\sphinxAtStartPar
0x 00000
&
\sphinxAtStartPar
TIMERx控制寄存器           |
\\
\sphinxhline
\sphinxAtStartPar
IEx
&
\sphinxAtStartPar
0x10
&&
\sphinxAtStartPar
0x 00000
&
\sphinxAtStartPar
TIMERx中断使能寄存器       |
\\
\sphinxhline
\sphinxAtStartPar
IFx
&
\sphinxAtStartPar
0x14
&&
\sphinxAtStartPar
0x 00000
&
\sphinxAtStartPar
TIMERx中断状态。写1清零。  |
\\
\sphinxhline
\sphinxAtStartPar
HALTx
&
\sphinxAtStartPar
0x18
&&
\sphinxAtStartPar
0x 00000
&
\sphinxAtStartPar
TIMERx暂停控制             |
\\
\sphinxhline
\sphinxAtStartPar
OCCRx
&
\sphinxAtStartPar
0x1C
&&
\sphinxAtStartPar
0x 00000
&
\sphinxAtStartPar
TIMER发送脉冲控制信号      |
\\
\sphinxhline
\sphinxAtStartPar
OCMATx
&
\sphinxAtStartPar
0x20
&&
\sphinxAtStartPar
0x 00000
&
\sphinxAtStartPar
PWM输出脉冲反转值          |
\\
\sphinxhline
\sphinxAtStartPar
ICLOWx
&
\sphinxAtStartPar
0x28
&&
\sphinxAtStartPar
0x 00000
&
\sphinxAtStartPar
输入脉冲低电平长度         |
\\
\sphinxhline
\sphinxAtStartPar
ICHIGHx
&
\sphinxAtStartPar
0x2C
&&
\sphinxAtStartPar
0x 00000
&
\sphinxAtStartPar
输入脉冲高电平长度         |
\\
\sphinxhline
\sphinxAtStartPar
PSCx
&
\sphinxAtStartPar
0x30
&&
\sphinxAtStartPar
0x 00000
&
\sphinxAtStartPar
TIMERx预分频器装载值寄存器 |
\\
\sphinxhline
\sphinxAtStartPar
HALLIE
&
\sphinxAtStartPar
0x400
&&
\sphinxAtStartPar
0x 00000
&
\sphinxAtStartPar
HALL中断使能               |
\\
\sphinxhline
\sphinxAtStartPar
HALLIF
&
\sphinxAtStartPar
0x408
&&
\sphinxAtStartPar
0x 00000
&
\sphinxAtStartPar
HALL中断状态               |
\\
\sphinxhline
\sphinxAtStartPar
HALLEN
&
\sphinxAtStartPar
0x40C
&&
\sphinxAtStartPar
0x 00000
&
\sphinxAtStartPar
HALL触发使能寄存器         |
\\
\sphinxhline
\sphinxAtStartPar
HALLDR
&
\sphinxAtStartPar
0x410
&&
\sphinxAtStartPar
0x 00000
&\\
\sphinxhline
\sphinxAtStartPar
HALLSR
&
\sphinxAtStartPar
0x41C
&&
\sphinxAtStartPar
0x 00000
&\\
\sphinxhline
\sphinxAtStartPar
EN
&
\sphinxAtStartPar
0x440
&&
\sphinxAtStartPar
0x 00000
&
\sphinxAtStartPar
TIMER使能寄存器            |
\\
\sphinxbottomrule
\end{tabular}
\sphinxtableafterendhook\par
\sphinxattableend\end{savenotes}


\subsection{寄存器描述}
\label{\detokenize{SWM241/_u529f_u80fd_u63cf_u8ff0/_u52a0_u5f3a_u578b_u5b9a_u65f6_u5668:id10}}

\subsubsection{装载值寄存器LOADx}
\label{\detokenize{SWM241/_u529f_u80fd_u63cf_u8ff0/_u52a0_u5f3a_u578b_u5b9a_u65f6_u5668:loadx}}

\begin{savenotes}\sphinxattablestart
\sphinxthistablewithglobalstyle
\centering
\begin{tabular}[t]{\X{20}{100}\X{20}{100}\X{20}{100}\X{20}{100}\X{20}{100}}
\sphinxtoprule
\sphinxtableatstartofbodyhook
\sphinxAtStartPar
寄存器 |
&
\begin{DUlineblock}{0em}
\item[] 偏移 |
\end{DUlineblock}
&
\begin{DUlineblock}{0em}
\item[] 
\end{DUlineblock}
&
\sphinxAtStartPar
复位值 |    描 | |
&
\begin{DUlineblock}{0em}
\item[] 
\end{DUlineblock}
\\
\sphinxhline
\sphinxAtStartPar
LOADx
&
\sphinxAtStartPar
0x0
&&
\sphinxAtStartPar
0x00
&
\sphinxAtStartPar
TIMERx装载值寄存器         |
\\
\sphinxbottomrule
\end{tabular}
\sphinxtableafterendhook\par
\sphinxattableend\end{savenotes}


\begin{savenotes}\sphinxattablestart
\sphinxthistablewithglobalstyle
\centering
\begin{tabular}[t]{\X{12}{96}\X{12}{96}\X{12}{96}\X{12}{96}\X{12}{96}\X{12}{96}\X{12}{96}\X{12}{96}}
\sphinxtoprule
\sphinxtableatstartofbodyhook
\sphinxAtStartPar
31
&
\sphinxAtStartPar
30
&
\sphinxAtStartPar
29
&
\sphinxAtStartPar
28
&
\sphinxAtStartPar
27
&
\sphinxAtStartPar
26
&
\sphinxAtStartPar
25
&
\sphinxAtStartPar
24
\\
\sphinxhline\begin{itemize}
\item {} 
\end{itemize}
&&&&&&&\\
\sphinxhline
\sphinxAtStartPar
23
&
\sphinxAtStartPar
22
&
\sphinxAtStartPar
21
&
\sphinxAtStartPar
20
&
\sphinxAtStartPar
19
&
\sphinxAtStartPar
18
&
\sphinxAtStartPar
17
&
\sphinxAtStartPar
16
\\
\sphinxhline
\sphinxAtStartPar
LOADx
&&&&&&&\\
\sphinxhline
\sphinxAtStartPar
15
&
\sphinxAtStartPar
14
&
\sphinxAtStartPar
13
&
\sphinxAtStartPar
12
&
\sphinxAtStartPar
11
&
\sphinxAtStartPar
10
&
\sphinxAtStartPar
9
&
\sphinxAtStartPar
8
\\
\sphinxhline
\sphinxAtStartPar
LOADx
&&&&&&&\\
\sphinxhline
\sphinxAtStartPar
7
&
\sphinxAtStartPar
6
&
\sphinxAtStartPar
5
&
\sphinxAtStartPar
4
&
\sphinxAtStartPar
3
&
\sphinxAtStartPar
2
&
\sphinxAtStartPar
1
&
\sphinxAtStartPar
0
\\
\sphinxhline
\sphinxAtStartPar
LOADx
&&&&&&&\\
\sphinxbottomrule
\end{tabular}
\sphinxtableafterendhook\par
\sphinxattableend\end{savenotes}


\begin{savenotes}\sphinxattablestart
\sphinxthistablewithglobalstyle
\centering
\begin{tabular}[t]{\X{33}{99}\X{33}{99}\X{33}{99}}
\sphinxtoprule
\sphinxtableatstartofbodyhook
\sphinxAtStartPar
位域 |
&
\sphinxAtStartPar
名称     | |
&
\sphinxAtStartPar
描述                                        | |
\\
\sphinxhline
\sphinxAtStartPar
31:24
&\begin{itemize}
\item {} 
\end{itemize}
&\begin{itemize}
\item {} 
\end{itemize}
\\
\sphinxhline
\sphinxAtStartPar
23:0
&
\sphinxAtStartPar
LOADx
&
\sphinxAtStartPar
定时器通道x的装载值                         |
\\
\sphinxbottomrule
\end{tabular}
\sphinxtableafterendhook\par
\sphinxattableend\end{savenotes}


\subsubsection{当前值寄存器VALUEx}
\label{\detokenize{SWM241/_u529f_u80fd_u63cf_u8ff0/_u52a0_u5f3a_u578b_u5b9a_u65f6_u5668:valuex}}

\begin{savenotes}\sphinxattablestart
\sphinxthistablewithglobalstyle
\centering
\begin{tabular}[t]{\X{20}{100}\X{20}{100}\X{20}{100}\X{20}{100}\X{20}{100}}
\sphinxtoprule
\sphinxtableatstartofbodyhook
\sphinxAtStartPar
寄存器 |
&
\begin{DUlineblock}{0em}
\item[] 偏移 |
\end{DUlineblock}
&
\begin{DUlineblock}{0em}
\item[] 
\end{DUlineblock}
&
\sphinxAtStartPar
复位值 |    描 | |
&
\begin{DUlineblock}{0em}
\item[] 
\end{DUlineblock}
\\
\sphinxhline
\sphinxAtStartPar
VALUEx
&
\sphinxAtStartPar
0x4
&&&
\sphinxAtStartPar
TIMERx当前计数值寄存器     |
\\
\sphinxbottomrule
\end{tabular}
\sphinxtableafterendhook\par
\sphinxattableend\end{savenotes}


\begin{savenotes}\sphinxattablestart
\sphinxthistablewithglobalstyle
\centering
\begin{tabular}[t]{\X{12}{96}\X{12}{96}\X{12}{96}\X{12}{96}\X{12}{96}\X{12}{96}\X{12}{96}\X{12}{96}}
\sphinxtoprule
\sphinxtableatstartofbodyhook
\sphinxAtStartPar
31
&
\sphinxAtStartPar
30
&
\sphinxAtStartPar
29
&
\sphinxAtStartPar
28
&
\sphinxAtStartPar
27
&
\sphinxAtStartPar
26
&
\sphinxAtStartPar
25
&
\sphinxAtStartPar
24
\\
\sphinxhline\begin{itemize}
\item {} 
\end{itemize}
&&&&&&&\\
\sphinxhline
\sphinxAtStartPar
23
&
\sphinxAtStartPar
22
&
\sphinxAtStartPar
21
&
\sphinxAtStartPar
20
&
\sphinxAtStartPar
19
&
\sphinxAtStartPar
18
&
\sphinxAtStartPar
17
&
\sphinxAtStartPar
16
\\
\sphinxhline
\sphinxAtStartPar
VALUEx
&&&&&&&\\
\sphinxhline
\sphinxAtStartPar
15
&
\sphinxAtStartPar
14
&
\sphinxAtStartPar
13
&
\sphinxAtStartPar
12
&
\sphinxAtStartPar
11
&
\sphinxAtStartPar
10
&
\sphinxAtStartPar
9
&
\sphinxAtStartPar
8
\\
\sphinxhline
\sphinxAtStartPar
VALUEx
&&&&&&&\\
\sphinxhline
\sphinxAtStartPar
7
&
\sphinxAtStartPar
6
&
\sphinxAtStartPar
5
&
\sphinxAtStartPar
4
&
\sphinxAtStartPar
3
&
\sphinxAtStartPar
2
&
\sphinxAtStartPar
1
&
\sphinxAtStartPar
0
\\
\sphinxhline
\sphinxAtStartPar
VALUEx
&&&&&&&\\
\sphinxbottomrule
\end{tabular}
\sphinxtableafterendhook\par
\sphinxattableend\end{savenotes}


\begin{savenotes}\sphinxattablestart
\sphinxthistablewithglobalstyle
\centering
\begin{tabular}[t]{\X{33}{99}\X{33}{99}\X{33}{99}}
\sphinxtoprule
\sphinxtableatstartofbodyhook
\sphinxAtStartPar
位域 |
&
\sphinxAtStartPar
名称     | |
&
\sphinxAtStartPar
描述                                        | |
\\
\sphinxhline
\sphinxAtStartPar
31:24
&\begin{itemize}
\item {} 
\end{itemize}
&\begin{itemize}
\item {} 
\end{itemize}
\\
\sphinxhline
\sphinxAtStartPar
23:0
&
\sphinxAtStartPar
VALUEx
&
\sphinxAtStartPar
定时器通道x的计数器当前值                   |
\\
\sphinxbottomrule
\end{tabular}
\sphinxtableafterendhook\par
\sphinxattableend\end{savenotes}


\subsubsection{控制寄存器CRx}
\label{\detokenize{SWM241/_u529f_u80fd_u63cf_u8ff0/_u52a0_u5f3a_u578b_u5b9a_u65f6_u5668:crx}}

\begin{savenotes}\sphinxattablestart
\sphinxthistablewithglobalstyle
\centering
\begin{tabular}[t]{\X{20}{100}\X{20}{100}\X{20}{100}\X{20}{100}\X{20}{100}}
\sphinxtoprule
\sphinxtableatstartofbodyhook
\sphinxAtStartPar
寄存器 |
&
\begin{DUlineblock}{0em}
\item[] 偏移 |
\end{DUlineblock}
&
\begin{DUlineblock}{0em}
\item[] 
\end{DUlineblock}
&
\sphinxAtStartPar
复位值 |    描 | |
&
\begin{DUlineblock}{0em}
\item[] 
\end{DUlineblock}
\\
\sphinxhline
\sphinxAtStartPar
CRx
&
\sphinxAtStartPar
0x8
&&
\sphinxAtStartPar
0x00
&
\sphinxAtStartPar
TIMERx控制寄存器           |
\\
\sphinxbottomrule
\end{tabular}
\sphinxtableafterendhook\par
\sphinxattableend\end{savenotes}


\begin{savenotes}\sphinxattablestart
\sphinxthistablewithglobalstyle
\centering
\begin{tabular}[t]{\X{12}{96}\X{12}{96}\X{12}{96}\X{12}{96}\X{12}{96}\X{12}{96}\X{12}{96}\X{12}{96}}
\sphinxtoprule
\sphinxtableatstartofbodyhook
\sphinxAtStartPar
31
&
\sphinxAtStartPar
30
&
\sphinxAtStartPar
29
&
\sphinxAtStartPar
28
&
\sphinxAtStartPar
27
&
\sphinxAtStartPar
26
&
\sphinxAtStartPar
25
&
\sphinxAtStartPar
24
\\
\sphinxhline\begin{itemize}
\item {} 
\end{itemize}
&&&&&&&\\
\sphinxhline
\sphinxAtStartPar
23
&
\sphinxAtStartPar
22
&
\sphinxAtStartPar
21
&
\sphinxAtStartPar
20
&
\sphinxAtStartPar
19
&
\sphinxAtStartPar
18
&
\sphinxAtStartPar
17
&
\sphinxAtStartPar
16
\\
\sphinxhline\begin{itemize}
\item {} 
\end{itemize}
&&&&&&&\\
\sphinxhline
\sphinxAtStartPar
15
&
\sphinxAtStartPar
14
&
\sphinxAtStartPar
13
&
\sphinxAtStartPar
12
&
\sphinxAtStartPar
11
&
\sphinxAtStartPar
10
&
\sphinxAtStartPar
9
&
\sphinxAtStartPar
8
\\
\sphinxhline\begin{itemize}
\item {} 
\end{itemize}
&&&&&&&\\
\sphinxhline
\sphinxAtStartPar
7
&
\sphinxAtStartPar
6
&
\sphinxAtStartPar
5
&
\sphinxAtStartPar
4
&
\sphinxAtStartPar
3
&
\sphinxAtStartPar
2
&
\sphinxAtStartPar
1
&
\sphinxAtStartPar
0
\\
\sphinxhline\begin{itemize}
\item {} 
\end{itemize}
&&
\sphinxAtStartPar
I GEx
&&&&
\sphinxAtStartPar
C RCx
&\\
\sphinxbottomrule
\end{tabular}
\sphinxtableafterendhook\par
\sphinxattableend\end{savenotes}


\begin{savenotes}\sphinxattablestart
\sphinxthistablewithglobalstyle
\centering
\begin{tabular}[t]{\X{33}{99}\X{33}{99}\X{33}{99}}
\sphinxtoprule
\sphinxtableatstartofbodyhook
\sphinxAtStartPar
位域 |
&
\sphinxAtStartPar
名称     | |
&
\sphinxAtStartPar
描述                                        | |
\\
\sphinxhline
\sphinxAtStartPar
31:6
&\begin{itemize}
\item {} 
\end{itemize}
&\begin{itemize}
\item {} 
\end{itemize}
\\
\sphinxhline
\sphinxAtStartPar
5:4
&
\sphinxAtStartPar
ICEDGEx
&
\sphinxAtStartPar
输入脉冲测量模式下,计数模式:              |

\sphinxAtStartPar
00:检测到上升沿或者下降沿后开始计数        |

\sphinxAtStartPar
01:检测到上升沿开始计数                    |

\sphinxAtStartPar
10:检测到下降沿开始计数                    |
\\
\sphinxhline
\sphinxAtStartPar
3:2
&
\sphinxAtStartPar
MODEx
&
\sphinxAtStartPar
定时器工作模式:                            |

\sphinxAtStartPar
00:普通定时器模式                          |

\sphinxAtStartPar
01:输入脉冲测量模式                        |

\sphinxAtStartPar
10:输出PWM模式                             |
\\
\sphinxhline
\sphinxAtStartPar
1:0
&
\sphinxAtStartPar
CLKSRCx
&
\sphinxAtStartPar
定时器计数源选择:                          |

\sphinxAtStartPar
00:使用内部系统时钟上升沿                  |

\sphinxAtStartPar
01:使用上一路计数器的进位标                | =1时,使用第0路;x=2时,使用第1路;依次类 | =0时,使用第TM\_NO\sphinxhyphen{}1路,最多支持两级级联) |

\sphinxAtStartPar
10:使用外部的cntsrc的上升沿                |
\\
\sphinxbottomrule
\end{tabular}
\sphinxtableafterendhook\par
\sphinxattableend\end{savenotes}


\subsubsection{中断使能寄存器IEx}
\label{\detokenize{SWM241/_u529f_u80fd_u63cf_u8ff0/_u52a0_u5f3a_u578b_u5b9a_u65f6_u5668:iex}}

\begin{savenotes}\sphinxattablestart
\sphinxthistablewithglobalstyle
\centering
\begin{tabular}[t]{\X{20}{100}\X{20}{100}\X{20}{100}\X{20}{100}\X{20}{100}}
\sphinxtoprule
\sphinxtableatstartofbodyhook
\sphinxAtStartPar
寄存器 |
&
\begin{DUlineblock}{0em}
\item[] 偏移 |
\end{DUlineblock}
&
\begin{DUlineblock}{0em}
\item[] 
\end{DUlineblock}
&
\sphinxAtStartPar
复位值 |    描 | |
&
\begin{DUlineblock}{0em}
\item[] 
\end{DUlineblock}
\\
\sphinxhline
\sphinxAtStartPar
IEx
&
\sphinxAtStartPar
0x10
&&
\sphinxAtStartPar
0x00
&
\sphinxAtStartPar
TIMERx中断使能寄存器       |
\\
\sphinxbottomrule
\end{tabular}
\sphinxtableafterendhook\par
\sphinxattableend\end{savenotes}


\begin{savenotes}\sphinxattablestart
\sphinxthistablewithglobalstyle
\centering
\begin{tabular}[t]{\X{12}{96}\X{12}{96}\X{12}{96}\X{12}{96}\X{12}{96}\X{12}{96}\X{12}{96}\X{12}{96}}
\sphinxtoprule
\sphinxtableatstartofbodyhook
\sphinxAtStartPar
31
&
\sphinxAtStartPar
30
&
\sphinxAtStartPar
29
&
\sphinxAtStartPar
28
&
\sphinxAtStartPar
27
&
\sphinxAtStartPar
26
&
\sphinxAtStartPar
25
&
\sphinxAtStartPar
24
\\
\sphinxhline\begin{itemize}
\item {} 
\end{itemize}
&&&&&&&\\
\sphinxhline
\sphinxAtStartPar
23
&
\sphinxAtStartPar
22
&
\sphinxAtStartPar
21
&
\sphinxAtStartPar
20
&
\sphinxAtStartPar
19
&
\sphinxAtStartPar
18
&
\sphinxAtStartPar
17
&
\sphinxAtStartPar
16
\\
\sphinxhline\begin{itemize}
\item {} 
\end{itemize}
&&&&&&&\\
\sphinxhline
\sphinxAtStartPar
15
&
\sphinxAtStartPar
14
&
\sphinxAtStartPar
13
&
\sphinxAtStartPar
12
&
\sphinxAtStartPar
11
&
\sphinxAtStartPar
10
&
\sphinxAtStartPar
9
&
\sphinxAtStartPar
8
\\
\sphinxhline\begin{itemize}
\item {} 
\end{itemize}
&&&&&&&\\
\sphinxhline
\sphinxAtStartPar
7
&
\sphinxAtStartPar
6
&
\sphinxAtStartPar
5
&
\sphinxAtStartPar
4
&
\sphinxAtStartPar
3
&
\sphinxAtStartPar
2
&
\sphinxAtStartPar
1
&
\sphinxAtStartPar
0
\\
\sphinxhline\begin{itemize}
\item {} 
\end{itemize}
&&&
\sphinxAtStartPar
ICF
&&
\sphinxAtStartPar
OC1
&&
\sphinxAtStartPar
TO
\\
\sphinxbottomrule
\end{tabular}
\sphinxtableafterendhook\par
\sphinxattableend\end{savenotes}


\begin{savenotes}\sphinxattablestart
\sphinxthistablewithglobalstyle
\centering
\begin{tabular}[t]{\X{33}{99}\X{33}{99}\X{33}{99}}
\sphinxtoprule
\sphinxtableatstartofbodyhook
\sphinxAtStartPar
位域 |
&
\sphinxAtStartPar
名称     | |
&
\sphinxAtStartPar
描述                                        | |
\\
\sphinxhline
\sphinxAtStartPar
31:5
&\begin{itemize}
\item {} 
\end{itemize}
&\begin{itemize}
\item {} 
\end{itemize}
\\
\sphinxhline
\sphinxAtStartPar
4
&
\sphinxAtStartPar
ICF
&
\sphinxAtStartPar
输入脉冲下降沿中断使能                      |

\sphinxAtStartPar
1:使能                                     |

\sphinxAtStartPar
0:禁能                                     |
\\
\sphinxhline
\sphinxAtStartPar
3
&
\sphinxAtStartPar
ICR
&
\sphinxAtStartPar
输入脉冲上升沿中断使能                      |

\sphinxAtStartPar
1:使能                                     |

\sphinxAtStartPar
0:禁能                                     |
\\
\sphinxhline
\sphinxAtStartPar
2
&
\sphinxAtStartPar
OC1
&
\sphinxAtStartPar
输出PWM翻转点1中断使能                      |

\sphinxAtStartPar
1:使能                                     |

\sphinxAtStartPar
0:禁能                                     |
\\
\sphinxhline
\sphinxAtStartPar
1
&
\sphinxAtStartPar
OC0
&
\sphinxAtStartPar
输出PWM翻转点0中断使能                      |

\sphinxAtStartPar
1:使能                                     |

\sphinxAtStartPar
0:禁能                                     |
\\
\sphinxhline
\sphinxAtStartPar
0
&
\sphinxAtStartPar
TO
&
\sphinxAtStartPar
计数器溢出中断                              |

\sphinxAtStartPar
1:使能                                     |

\sphinxAtStartPar
0:禁能                                     |
\\
\sphinxbottomrule
\end{tabular}
\sphinxtableafterendhook\par
\sphinxattableend\end{savenotes}


\subsubsection{中断状态寄存器IFx}
\label{\detokenize{SWM241/_u529f_u80fd_u63cf_u8ff0/_u52a0_u5f3a_u578b_u5b9a_u65f6_u5668:ifx}}

\begin{savenotes}\sphinxattablestart
\sphinxthistablewithglobalstyle
\centering
\begin{tabular}[t]{\X{20}{100}\X{20}{100}\X{20}{100}\X{20}{100}\X{20}{100}}
\sphinxtoprule
\sphinxtableatstartofbodyhook
\sphinxAtStartPar
寄存器 |
&
\begin{DUlineblock}{0em}
\item[] 偏移 |
\end{DUlineblock}
&
\begin{DUlineblock}{0em}
\item[] 
\end{DUlineblock}
&
\sphinxAtStartPar
复位值 |    描 | |
&
\begin{DUlineblock}{0em}
\item[] 
\end{DUlineblock}
\\
\sphinxhline
\sphinxAtStartPar
IFx
&
\sphinxAtStartPar
0x14
&&
\sphinxAtStartPar
0x00
&
\sphinxAtStartPar
TIMERx中断状态。写1清零。  |
\\
\sphinxbottomrule
\end{tabular}
\sphinxtableafterendhook\par
\sphinxattableend\end{savenotes}


\begin{savenotes}\sphinxattablestart
\sphinxthistablewithglobalstyle
\centering
\begin{tabular}[t]{\X{12}{96}\X{12}{96}\X{12}{96}\X{12}{96}\X{12}{96}\X{12}{96}\X{12}{96}\X{12}{96}}
\sphinxtoprule
\sphinxtableatstartofbodyhook
\sphinxAtStartPar
31
&
\sphinxAtStartPar
30
&
\sphinxAtStartPar
29
&
\sphinxAtStartPar
28
&
\sphinxAtStartPar
27
&
\sphinxAtStartPar
26
&
\sphinxAtStartPar
25
&
\sphinxAtStartPar
24
\\
\sphinxhline\begin{itemize}
\item {} 
\end{itemize}
&&&&&&&\\
\sphinxhline
\sphinxAtStartPar
23
&
\sphinxAtStartPar
22
&
\sphinxAtStartPar
21
&
\sphinxAtStartPar
20
&
\sphinxAtStartPar
19
&
\sphinxAtStartPar
18
&
\sphinxAtStartPar
17
&
\sphinxAtStartPar
16
\\
\sphinxhline\begin{itemize}
\item {} 
\end{itemize}
&&&&&&&\\
\sphinxhline
\sphinxAtStartPar
15
&
\sphinxAtStartPar
14
&
\sphinxAtStartPar
13
&
\sphinxAtStartPar
12
&
\sphinxAtStartPar
11
&
\sphinxAtStartPar
10
&
\sphinxAtStartPar
9
&
\sphinxAtStartPar
8
\\
\sphinxhline\begin{itemize}
\item {} 
\end{itemize}
&&&&&&&\\
\sphinxhline
\sphinxAtStartPar
7
&
\sphinxAtStartPar
6
&
\sphinxAtStartPar
5
&
\sphinxAtStartPar
4
&
\sphinxAtStartPar
3
&
\sphinxAtStartPar
2
&
\sphinxAtStartPar
1
&
\sphinxAtStartPar
0
\\
\sphinxhline\begin{itemize}
\item {} 
\end{itemize}
&&&
\sphinxAtStartPar
ICF
&&
\sphinxAtStartPar
OC1
&&
\sphinxAtStartPar
TO
\\
\sphinxbottomrule
\end{tabular}
\sphinxtableafterendhook\par
\sphinxattableend\end{savenotes}


\begin{savenotes}\sphinxattablestart
\sphinxthistablewithglobalstyle
\centering
\begin{tabular}[t]{\X{33}{99}\X{33}{99}\X{33}{99}}
\sphinxtoprule
\sphinxtableatstartofbodyhook
\sphinxAtStartPar
位域 |
&
\sphinxAtStartPar
名称     | |
&
\sphinxAtStartPar
描述                                        | |
\\
\sphinxhline
\sphinxAtStartPar
31:5
&\begin{itemize}
\item {} 
\end{itemize}
&\begin{itemize}
\item {} 
\end{itemize}
\\
\sphinxhline
\sphinxAtStartPar
4
&
\sphinxAtStartPar
ICF
&
\sphinxAtStartPar
输入脉冲下降沿中断状态,R/W1C               |

\sphinxAtStartPar
1:中断发生                                 |

\sphinxAtStartPar
0:中断未发生                               |
\\
\sphinxhline
\sphinxAtStartPar
3
&
\sphinxAtStartPar
ICR
&
\sphinxAtStartPar
输入脉冲上升沿中断状态,R/W1C               |

\sphinxAtStartPar
1:中断发生                                 |

\sphinxAtStartPar
0:中断未发生                               |
\\
\sphinxhline
\sphinxAtStartPar
2
&
\sphinxAtStartPar
OC1
&
\sphinxAtStartPar
输出PWM翻转点1中断状态,R/W1C               |

\sphinxAtStartPar
1:中断发生                                 |

\sphinxAtStartPar
0:中断未发生                               |
\\
\sphinxhline
\sphinxAtStartPar
1
&
\sphinxAtStartPar
OC0
&
\sphinxAtStartPar
输出PWM翻转点0中断状态,R/W1C               |

\sphinxAtStartPar
1:中断发生                                 |

\sphinxAtStartPar
0:中断未发生                               |
\\
\sphinxhline
\sphinxAtStartPar
0
&
\sphinxAtStartPar
TO
&
\sphinxAtStartPar
计数器溢出中断状态,R/W1C                   |

\sphinxAtStartPar
1:中断发生                                 |

\sphinxAtStartPar
0:中断未发生                               |

\sphinxAtStartPar
使用级联功能时,高一级的中断不会触发,低一 | 断在全部计数结束后触发。比如通道0和通道  | ,当通道1和通道0的计数都到0时,通道0的溢  | 才会触发,通道1的溢出中断始终不会触发。  |
\\
\sphinxbottomrule
\end{tabular}
\sphinxtableafterendhook\par
\sphinxattableend\end{savenotes}


\subsubsection{暂停控制寄存器HALTx}
\label{\detokenize{SWM241/_u529f_u80fd_u63cf_u8ff0/_u52a0_u5f3a_u578b_u5b9a_u65f6_u5668:haltx}}

\begin{savenotes}\sphinxattablestart
\sphinxthistablewithglobalstyle
\centering
\begin{tabular}[t]{\X{20}{100}\X{20}{100}\X{20}{100}\X{20}{100}\X{20}{100}}
\sphinxtoprule
\sphinxtableatstartofbodyhook
\sphinxAtStartPar
寄存器 |
&
\begin{DUlineblock}{0em}
\item[] 偏移 |
\end{DUlineblock}
&
\begin{DUlineblock}{0em}
\item[] 
\end{DUlineblock}
&
\sphinxAtStartPar
复位值 |    描 | |
&
\begin{DUlineblock}{0em}
\item[] 
\end{DUlineblock}
\\
\sphinxhline
\sphinxAtStartPar
BRKx
&
\sphinxAtStartPar
0x18
&&
\sphinxAtStartPar
0x00
&
\sphinxAtStartPar
TIMERx暂停控制             |
\\
\sphinxbottomrule
\end{tabular}
\sphinxtableafterendhook\par
\sphinxattableend\end{savenotes}


\begin{savenotes}\sphinxattablestart
\sphinxthistablewithglobalstyle
\centering
\begin{tabular}[t]{\X{12}{96}\X{12}{96}\X{12}{96}\X{12}{96}\X{12}{96}\X{12}{96}\X{12}{96}\X{12}{96}}
\sphinxtoprule
\sphinxtableatstartofbodyhook
\sphinxAtStartPar
31
&
\sphinxAtStartPar
30
&
\sphinxAtStartPar
29
&
\sphinxAtStartPar
28
&
\sphinxAtStartPar
27
&
\sphinxAtStartPar
26
&
\sphinxAtStartPar
25
&
\sphinxAtStartPar
24
\\
\sphinxhline\begin{itemize}
\item {} 
\end{itemize}
&&&&&&&\\
\sphinxhline
\sphinxAtStartPar
23
&
\sphinxAtStartPar
22
&
\sphinxAtStartPar
21
&
\sphinxAtStartPar
20
&
\sphinxAtStartPar
19
&
\sphinxAtStartPar
18
&
\sphinxAtStartPar
17
&
\sphinxAtStartPar
16
\\
\sphinxhline\begin{itemize}
\item {} 
\end{itemize}
&&&&&&&\\
\sphinxhline
\sphinxAtStartPar
15
&
\sphinxAtStartPar
14
&
\sphinxAtStartPar
13
&
\sphinxAtStartPar
12
&
\sphinxAtStartPar
11
&
\sphinxAtStartPar
10
&
\sphinxAtStartPar
9
&
\sphinxAtStartPar
8
\\
\sphinxhline\begin{itemize}
\item {} 
\end{itemize}
&&&&&&&\\
\sphinxhline
\sphinxAtStartPar
7
&
\sphinxAtStartPar
6
&
\sphinxAtStartPar
5
&
\sphinxAtStartPar
4
&
\sphinxAtStartPar
3
&
\sphinxAtStartPar
2
&
\sphinxAtStartPar
1
&
\sphinxAtStartPar
0
\\
\sphinxhline\begin{itemize}
\item {} 
\end{itemize}
&&&&&&&\\
\sphinxbottomrule
\end{tabular}
\sphinxtableafterendhook\par
\sphinxattableend\end{savenotes}


\begin{savenotes}\sphinxattablestart
\sphinxthistablewithglobalstyle
\centering
\begin{tabular}[t]{\X{33}{99}\X{33}{99}\X{33}{99}}
\sphinxtoprule
\sphinxtableatstartofbodyhook
\sphinxAtStartPar
位域 |
&
\sphinxAtStartPar
名称     | |
&
\sphinxAtStartPar
描述                                        | |
\\
\sphinxhline
\sphinxAtStartPar
31:1
&\begin{itemize}
\item {} 
\end{itemize}
&\begin{itemize}
\item {} 
\end{itemize}
\\
\sphinxhline
\sphinxAtStartPar
0
&
\sphinxAtStartPar
HALTx
&
\sphinxAtStartPar
定时器暂停控制                              |

\sphinxAtStartPar
1:暂停当前定时器的计数                     |

\sphinxAtStartPar
0:当前定时器正常减计数                     |
\\
\sphinxbottomrule
\end{tabular}
\sphinxtableafterendhook\par
\sphinxattableend\end{savenotes}


\subsubsection{发送脉冲控制信号寄存器OCCRx}
\label{\detokenize{SWM241/_u529f_u80fd_u63cf_u8ff0/_u52a0_u5f3a_u578b_u5b9a_u65f6_u5668:occrx}}

\begin{savenotes}\sphinxattablestart
\sphinxthistablewithglobalstyle
\centering
\begin{tabular}[t]{\X{20}{100}\X{20}{100}\X{20}{100}\X{20}{100}\X{20}{100}}
\sphinxtoprule
\sphinxtableatstartofbodyhook
\sphinxAtStartPar
寄存器 |
&
\begin{DUlineblock}{0em}
\item[] 偏移 |
\end{DUlineblock}
&
\begin{DUlineblock}{0em}
\item[] 
\end{DUlineblock}
&
\sphinxAtStartPar
复位值 |    描 | |
&
\begin{DUlineblock}{0em}
\item[] 
\end{DUlineblock}
\\
\sphinxhline
\sphinxAtStartPar
OCCRx
&
\sphinxAtStartPar
0x1C
&&
\sphinxAtStartPar
0x00
&
\sphinxAtStartPar
TIMER发送脉冲控制信号      |
\\
\sphinxbottomrule
\end{tabular}
\sphinxtableafterendhook\par
\sphinxattableend\end{savenotes}


\begin{savenotes}\sphinxattablestart
\sphinxthistablewithglobalstyle
\centering
\begin{tabular}[t]{\X{12}{96}\X{12}{96}\X{12}{96}\X{12}{96}\X{12}{96}\X{12}{96}\X{12}{96}\X{12}{96}}
\sphinxtoprule
\sphinxtableatstartofbodyhook
\sphinxAtStartPar
31
&
\sphinxAtStartPar
30
&
\sphinxAtStartPar
29
&
\sphinxAtStartPar
28
&
\sphinxAtStartPar
27
&
\sphinxAtStartPar
26
&
\sphinxAtStartPar
25
&
\sphinxAtStartPar
24
\\
\sphinxhline\begin{itemize}
\item {} 
\end{itemize}
&&&&&&&\\
\sphinxhline
\sphinxAtStartPar
23
&
\sphinxAtStartPar
22
&
\sphinxAtStartPar
21
&
\sphinxAtStartPar
20
&
\sphinxAtStartPar
19
&
\sphinxAtStartPar
18
&
\sphinxAtStartPar
17
&
\sphinxAtStartPar
16
\\
\sphinxhline\begin{itemize}
\item {} 
\end{itemize}
&&&&&&&\\
\sphinxhline
\sphinxAtStartPar
15
&
\sphinxAtStartPar
14
&
\sphinxAtStartPar
13
&
\sphinxAtStartPar
12
&
\sphinxAtStartPar
11
&
\sphinxAtStartPar
10
&
\sphinxAtStartPar
9
&
\sphinxAtStartPar
8
\\
\sphinxhline\begin{itemize}
\item {} 
\end{itemize}
&&&&&&&\\
\sphinxhline
\sphinxAtStartPar
7
&
\sphinxAtStartPar
6
&
\sphinxAtStartPar
5
&
\sphinxAtStartPar
4
&
\sphinxAtStartPar
3
&
\sphinxAtStartPar
2
&
\sphinxAtStartPar
1
&
\sphinxAtStartPar
0
\\
\sphinxhline\begin{itemize}
\item {} 
\end{itemize}
&&&&&
\sphinxAtStartPar
F EEN
&
\sphinxAtStartPar
IN VL
&
\sphinxAtStartPar
FO LVL
\\
\sphinxbottomrule
\end{tabular}
\sphinxtableafterendhook\par
\sphinxattableend\end{savenotes}


\begin{savenotes}\sphinxattablestart
\sphinxthistablewithglobalstyle
\centering
\begin{tabular}[t]{\X{33}{99}\X{33}{99}\X{33}{99}}
\sphinxtoprule
\sphinxtableatstartofbodyhook
\sphinxAtStartPar
位域 |
&
\sphinxAtStartPar
名称     | |
&
\sphinxAtStartPar
描述                                        | |
\\
\sphinxhline
\sphinxAtStartPar
31:3
&\begin{itemize}
\item {} 
\end{itemize}
&\begin{itemize}
\item {} 
\end{itemize}
\\
\sphinxhline
\sphinxAtStartPar
2
&
\sphinxAtStartPar
FORCEEN
&
\sphinxAtStartPar
Force Level,强制输出使能                   |
\\
\sphinxhline
\sphinxAtStartPar
1
&
\sphinxAtStartPar
INITLVL
&
\sphinxAtStartPar
Initial Level,初始输出电平                 |
\\
\sphinxhline
\sphinxAtStartPar
0
&
\sphinxAtStartPar
FORCELVL
&
\sphinxAtStartPar
Force Level,强制输出电平                   |
\\
\sphinxbottomrule
\end{tabular}
\sphinxtableafterendhook\par
\sphinxattableend\end{savenotes}


\subsubsection{输出脉冲反转值寄存器OCMATx}
\label{\detokenize{SWM241/_u529f_u80fd_u63cf_u8ff0/_u52a0_u5f3a_u578b_u5b9a_u65f6_u5668:ocmatx}}

\begin{savenotes}\sphinxattablestart
\sphinxthistablewithglobalstyle
\centering
\begin{tabular}[t]{\X{20}{100}\X{20}{100}\X{20}{100}\X{20}{100}\X{20}{100}}
\sphinxtoprule
\sphinxtableatstartofbodyhook
\sphinxAtStartPar
寄存器 |
&
\begin{DUlineblock}{0em}
\item[] 偏移 |
\end{DUlineblock}
&
\begin{DUlineblock}{0em}
\item[] 
\end{DUlineblock}
&
\sphinxAtStartPar
复位值 |    描 | |
&
\begin{DUlineblock}{0em}
\item[] 
\end{DUlineblock}
\\
\sphinxhline
\sphinxAtStartPar
OCMATx
&
\sphinxAtStartPar
0x20
&&
\sphinxAtStartPar
0x00
&
\sphinxAtStartPar
PWM输出脉冲反转值          |
\\
\sphinxbottomrule
\end{tabular}
\sphinxtableafterendhook\par
\sphinxattableend\end{savenotes}


\begin{savenotes}\sphinxattablestart
\sphinxthistablewithglobalstyle
\centering
\begin{tabular}[t]{\X{12}{96}\X{12}{96}\X{12}{96}\X{12}{96}\X{12}{96}\X{12}{96}\X{12}{96}\X{12}{96}}
\sphinxtoprule
\sphinxtableatstartofbodyhook
\sphinxAtStartPar
31
&
\sphinxAtStartPar
30
&
\sphinxAtStartPar
29
&
\sphinxAtStartPar
28
&
\sphinxAtStartPar
27
&
\sphinxAtStartPar
26
&
\sphinxAtStartPar
25
&
\sphinxAtStartPar
24
\\
\sphinxhline\begin{itemize}
\item {} 
\end{itemize}
&&&&&&&\\
\sphinxhline
\sphinxAtStartPar
23
&
\sphinxAtStartPar
22
&
\sphinxAtStartPar
21
&
\sphinxAtStartPar
20
&
\sphinxAtStartPar
19
&
\sphinxAtStartPar
18
&
\sphinxAtStartPar
17
&
\sphinxAtStartPar
16
\\
\sphinxhline
\sphinxAtStartPar
OCMATx
&&&&&&&\\
\sphinxhline
\sphinxAtStartPar
15
&
\sphinxAtStartPar
14
&
\sphinxAtStartPar
13
&
\sphinxAtStartPar
12
&
\sphinxAtStartPar
11
&
\sphinxAtStartPar
10
&
\sphinxAtStartPar
9
&
\sphinxAtStartPar
8
\\
\sphinxhline
\sphinxAtStartPar
OCMATx
&&&&&&&\\
\sphinxhline
\sphinxAtStartPar
7
&
\sphinxAtStartPar
6
&
\sphinxAtStartPar
5
&
\sphinxAtStartPar
4
&
\sphinxAtStartPar
3
&
\sphinxAtStartPar
2
&
\sphinxAtStartPar
1
&
\sphinxAtStartPar
0
\\
\sphinxhline
\sphinxAtStartPar
OCMATx
&&&&&&&\\
\sphinxbottomrule
\end{tabular}
\sphinxtableafterendhook\par
\sphinxattableend\end{savenotes}


\begin{savenotes}\sphinxattablestart
\sphinxthistablewithglobalstyle
\centering
\begin{tabular}[t]{\X{33}{99}\X{33}{99}\X{33}{99}}
\sphinxtoprule
\sphinxtableatstartofbodyhook
\sphinxAtStartPar
位域 |
&
\sphinxAtStartPar
名称     | |
&
\sphinxAtStartPar
描述                                        | |
\\
\sphinxhline
\sphinxAtStartPar
31:24
&\begin{itemize}
\item {} 
\end{itemize}
&\begin{itemize}
\item {} 
\end{itemize}
\\
\sphinxhline
\sphinxAtStartPar
23:0
&
\sphinxAtStartPar
OCMATx
&
\sphinxAtStartPar
PWM输出脉冲反转值                           |
\\
\sphinxbottomrule
\end{tabular}
\sphinxtableafterendhook\par
\sphinxattableend\end{savenotes}


\subsubsection{输入脉冲低电平长度寄存器ICLOWx}
\label{\detokenize{SWM241/_u529f_u80fd_u63cf_u8ff0/_u52a0_u5f3a_u578b_u5b9a_u65f6_u5668:iclowx}}

\begin{savenotes}\sphinxattablestart
\sphinxthistablewithglobalstyle
\centering
\begin{tabular}[t]{\X{20}{100}\X{20}{100}\X{20}{100}\X{20}{100}\X{20}{100}}
\sphinxtoprule
\sphinxtableatstartofbodyhook
\sphinxAtStartPar
寄存器 |
&
\begin{DUlineblock}{0em}
\item[] 偏移 |
\end{DUlineblock}
&
\begin{DUlineblock}{0em}
\item[] 
\end{DUlineblock}
&
\sphinxAtStartPar
复位值 |    描 | |
&
\begin{DUlineblock}{0em}
\item[] 
\end{DUlineblock}
\\
\sphinxhline
\sphinxAtStartPar
ICLOWx
&
\sphinxAtStartPar
0x28
&&
\sphinxAtStartPar
0x00
&
\sphinxAtStartPar
输入脉冲低电平长度         |
\\
\sphinxbottomrule
\end{tabular}
\sphinxtableafterendhook\par
\sphinxattableend\end{savenotes}


\begin{savenotes}\sphinxattablestart
\sphinxthistablewithglobalstyle
\centering
\begin{tabular}[t]{\X{12}{96}\X{12}{96}\X{12}{96}\X{12}{96}\X{12}{96}\X{12}{96}\X{12}{96}\X{12}{96}}
\sphinxtoprule
\sphinxtableatstartofbodyhook
\sphinxAtStartPar
31
&
\sphinxAtStartPar
30
&
\sphinxAtStartPar
29
&
\sphinxAtStartPar
28
&
\sphinxAtStartPar
27
&
\sphinxAtStartPar
26
&
\sphinxAtStartPar
25
&
\sphinxAtStartPar
24
\\
\sphinxhline\begin{itemize}
\item {} 
\end{itemize}
&&&&&&&\\
\sphinxhline
\sphinxAtStartPar
23
&
\sphinxAtStartPar
22
&
\sphinxAtStartPar
21
&
\sphinxAtStartPar
20
&
\sphinxAtStartPar
19
&
\sphinxAtStartPar
18
&
\sphinxAtStartPar
17
&
\sphinxAtStartPar
16
\\
\sphinxhline
\sphinxAtStartPar
ICLOWx
&&&&&&&\\
\sphinxhline
\sphinxAtStartPar
15
&
\sphinxAtStartPar
14
&
\sphinxAtStartPar
13
&
\sphinxAtStartPar
12
&
\sphinxAtStartPar
11
&
\sphinxAtStartPar
10
&
\sphinxAtStartPar
9
&
\sphinxAtStartPar
8
\\
\sphinxhline
\sphinxAtStartPar
ICLOWx
&&&&&&&\\
\sphinxhline
\sphinxAtStartPar
7
&
\sphinxAtStartPar
6
&
\sphinxAtStartPar
5
&
\sphinxAtStartPar
4
&
\sphinxAtStartPar
3
&
\sphinxAtStartPar
2
&
\sphinxAtStartPar
1
&
\sphinxAtStartPar
0
\\
\sphinxhline
\sphinxAtStartPar
ICLOWx
&&&&&&&\\
\sphinxbottomrule
\end{tabular}
\sphinxtableafterendhook\par
\sphinxattableend\end{savenotes}


\begin{savenotes}\sphinxattablestart
\sphinxthistablewithglobalstyle
\centering
\begin{tabular}[t]{\X{33}{99}\X{33}{99}\X{33}{99}}
\sphinxtoprule
\sphinxtableatstartofbodyhook
\sphinxAtStartPar
位域 |
&
\sphinxAtStartPar
名称     | |
&
\sphinxAtStartPar
描述                                        | |
\\
\sphinxhline
\sphinxAtStartPar
31:24
&\begin{itemize}
\item {} 
\end{itemize}
&\begin{itemize}
\item {} 
\end{itemize}
\\
\sphinxhline
\sphinxAtStartPar
23:0
&
\sphinxAtStartPar
ICLOWx
&
\sphinxAtStartPar
输入脉冲低电平长度                          |
\\
\sphinxbottomrule
\end{tabular}
\sphinxtableafterendhook\par
\sphinxattableend\end{savenotes}


\subsubsection{输入脉冲高电平长度寄存器ICHIGHx}
\label{\detokenize{SWM241/_u529f_u80fd_u63cf_u8ff0/_u52a0_u5f3a_u578b_u5b9a_u65f6_u5668:ichighx}}

\begin{savenotes}\sphinxattablestart
\sphinxthistablewithglobalstyle
\centering
\begin{tabular}[t]{\X{20}{100}\X{20}{100}\X{20}{100}\X{20}{100}\X{20}{100}}
\sphinxtoprule
\sphinxtableatstartofbodyhook
\sphinxAtStartPar
寄存器 |
&
\begin{DUlineblock}{0em}
\item[] 偏移 |
\end{DUlineblock}
&
\begin{DUlineblock}{0em}
\item[] 
\end{DUlineblock}
&
\sphinxAtStartPar
复位值 |    描 | |
&
\begin{DUlineblock}{0em}
\item[] |
  |
\end{DUlineblock}
\\
\sphinxhline
\sphinxAtStartPar
ICHIGHx
&
\sphinxAtStartPar
0x2C
&&
\sphinxAtStartPar
0x00
&
\sphinxAtStartPar
输入脉冲高电平长度         |
\\
\sphinxbottomrule
\end{tabular}
\sphinxtableafterendhook\par
\sphinxattableend\end{savenotes}


\begin{savenotes}\sphinxattablestart
\sphinxthistablewithglobalstyle
\centering
\begin{tabular}[t]{\X{12}{96}\X{12}{96}\X{12}{96}\X{12}{96}\X{12}{96}\X{12}{96}\X{12}{96}\X{12}{96}}
\sphinxtoprule
\sphinxtableatstartofbodyhook
\sphinxAtStartPar
31
&
\sphinxAtStartPar
30
&
\sphinxAtStartPar
29
&
\sphinxAtStartPar
28
&
\sphinxAtStartPar
27
&
\sphinxAtStartPar
26
&
\sphinxAtStartPar
25
&
\sphinxAtStartPar
24
\\
\sphinxhline\begin{itemize}
\item {} 
\end{itemize}
&&&&&&&\\
\sphinxhline
\sphinxAtStartPar
23
&
\sphinxAtStartPar
22
&
\sphinxAtStartPar
21
&
\sphinxAtStartPar
20
&
\sphinxAtStartPar
19
&
\sphinxAtStartPar
18
&
\sphinxAtStartPar
17
&
\sphinxAtStartPar
16
\\
\sphinxhline
\sphinxAtStartPar
ICHIGHx
&&&&&&&\\
\sphinxhline
\sphinxAtStartPar
15
&
\sphinxAtStartPar
14
&
\sphinxAtStartPar
13
&
\sphinxAtStartPar
12
&
\sphinxAtStartPar
11
&
\sphinxAtStartPar
10
&
\sphinxAtStartPar
9
&
\sphinxAtStartPar
8
\\
\sphinxhline
\sphinxAtStartPar
ICHIGHx
&&&&&&&\\
\sphinxhline
\sphinxAtStartPar
7
&
\sphinxAtStartPar
6
&
\sphinxAtStartPar
5
&
\sphinxAtStartPar
4
&
\sphinxAtStartPar
3
&
\sphinxAtStartPar
2
&
\sphinxAtStartPar
1
&
\sphinxAtStartPar
0
\\
\sphinxhline
\sphinxAtStartPar
ICHIGHx
&&&&&&&\\
\sphinxbottomrule
\end{tabular}
\sphinxtableafterendhook\par
\sphinxattableend\end{savenotes}


\begin{savenotes}\sphinxattablestart
\sphinxthistablewithglobalstyle
\centering
\begin{tabular}[t]{\X{33}{99}\X{33}{99}\X{33}{99}}
\sphinxtoprule
\sphinxtableatstartofbodyhook
\sphinxAtStartPar
位域 |
&
\sphinxAtStartPar
名称     | |
&
\sphinxAtStartPar
描述                                        | |
\\
\sphinxhline
\sphinxAtStartPar
31:24
&\begin{itemize}
\item {} 
\end{itemize}
&\begin{itemize}
\item {} 
\end{itemize}
\\
\sphinxhline
\sphinxAtStartPar
23:0
&
\sphinxAtStartPar
ICHIGHx
&
\sphinxAtStartPar
输入脉冲高电平长度                          |
\\
\sphinxbottomrule
\end{tabular}
\sphinxtableafterendhook\par
\sphinxattableend\end{savenotes}


\subsubsection{预分频器装载值寄存器PSCx}
\label{\detokenize{SWM241/_u529f_u80fd_u63cf_u8ff0/_u52a0_u5f3a_u578b_u5b9a_u65f6_u5668:pscx}}

\begin{savenotes}\sphinxattablestart
\sphinxthistablewithglobalstyle
\centering
\begin{tabular}[t]{\X{20}{100}\X{20}{100}\X{20}{100}\X{20}{100}\X{20}{100}}
\sphinxtoprule
\sphinxtableatstartofbodyhook
\sphinxAtStartPar
寄存器 |
&
\begin{DUlineblock}{0em}
\item[] 偏移 |
\end{DUlineblock}
&
\begin{DUlineblock}{0em}
\item[] 
\end{DUlineblock}
&
\sphinxAtStartPar
复位值 |    描 | |
&
\begin{DUlineblock}{0em}
\item[] 
\end{DUlineblock}
\\
\sphinxhline
\sphinxAtStartPar
PSCx
&
\sphinxAtStartPar
0x30
&&
\sphinxAtStartPar
0x00
&
\sphinxAtStartPar
TIMERx预分频器装载值寄存器 |
\\
\sphinxbottomrule
\end{tabular}
\sphinxtableafterendhook\par
\sphinxattableend\end{savenotes}


\begin{savenotes}\sphinxattablestart
\sphinxthistablewithglobalstyle
\centering
\begin{tabular}[t]{\X{12}{96}\X{12}{96}\X{12}{96}\X{12}{96}\X{12}{96}\X{12}{96}\X{12}{96}\X{12}{96}}
\sphinxtoprule
\sphinxtableatstartofbodyhook
\sphinxAtStartPar
31
&
\sphinxAtStartPar
30
&
\sphinxAtStartPar
29
&
\sphinxAtStartPar
28
&
\sphinxAtStartPar
27
&
\sphinxAtStartPar
26
&
\sphinxAtStartPar
25
&
\sphinxAtStartPar
24
\\
\sphinxhline\begin{itemize}
\item {} 
\end{itemize}
&&&&&&&\\
\sphinxhline
\sphinxAtStartPar
23
&
\sphinxAtStartPar
22
&
\sphinxAtStartPar
21
&
\sphinxAtStartPar
20
&
\sphinxAtStartPar
19
&
\sphinxAtStartPar
18
&
\sphinxAtStartPar
17
&
\sphinxAtStartPar
16
\\
\sphinxhline\begin{itemize}
\item {} 
\end{itemize}
&&&&&&&\\
\sphinxhline
\sphinxAtStartPar
15
&
\sphinxAtStartPar
14
&
\sphinxAtStartPar
13
&
\sphinxAtStartPar
12
&
\sphinxAtStartPar
11
&
\sphinxAtStartPar
10
&
\sphinxAtStartPar
9
&
\sphinxAtStartPar
8
\\
\sphinxhline\begin{itemize}
\item {} 
\end{itemize}
&&&&&&&\\
\sphinxhline
\sphinxAtStartPar
7
&
\sphinxAtStartPar
6
&
\sphinxAtStartPar
5
&
\sphinxAtStartPar
4
&
\sphinxAtStartPar
3
&
\sphinxAtStartPar
2
&
\sphinxAtStartPar
1
&
\sphinxAtStartPar
0
\\
\sphinxhline
\sphinxAtStartPar
PSCx
&&&&&&&\\
\sphinxbottomrule
\end{tabular}
\sphinxtableafterendhook\par
\sphinxattableend\end{savenotes}


\begin{savenotes}\sphinxattablestart
\sphinxthistablewithglobalstyle
\centering
\begin{tabular}[t]{\X{33}{99}\X{33}{99}\X{33}{99}}
\sphinxtoprule
\sphinxtableatstartofbodyhook
\sphinxAtStartPar
位域 |
&
\sphinxAtStartPar
名称     | |
&
\sphinxAtStartPar
描述                                        | |
\\
\sphinxhline
\sphinxAtStartPar
31:8
&\begin{itemize}
\item {} 
\end{itemize}
&\begin{itemize}
\item {} 
\end{itemize}
\\
\sphinxhline
\sphinxAtStartPar
7:0
&
\sphinxAtStartPar
PSCx
&
\sphinxAtStartPar
定时器时钟分频                              |

\sphinxAtStartPar
0:1分频                                    |

\sphinxAtStartPar
1:2分频                                    |

\sphinxAtStartPar
……

\sphinxAtStartPar
254:255分频                                |

\sphinxAtStartPar
255:256分频                                |

\sphinxAtStartPar
注:在级联模                                | 除了第一级,其它级的这个字段必须设置为0  |
\\
\sphinxbottomrule
\end{tabular}
\sphinxtableafterendhook\par
\sphinxattableend\end{savenotes}


\subsubsection{HALL中断使能寄存器HALLIE}
\label{\detokenize{SWM241/_u529f_u80fd_u63cf_u8ff0/_u52a0_u5f3a_u578b_u5b9a_u65f6_u5668:hallhallie}}

\begin{savenotes}\sphinxattablestart
\sphinxthistablewithglobalstyle
\centering
\begin{tabular}[t]{\X{20}{100}\X{20}{100}\X{20}{100}\X{20}{100}\X{20}{100}}
\sphinxtoprule
\sphinxtableatstartofbodyhook
\sphinxAtStartPar
寄存器 |
&
\begin{DUlineblock}{0em}
\item[] 偏移 |
\end{DUlineblock}
&
\begin{DUlineblock}{0em}
\item[] 
\end{DUlineblock}
&
\sphinxAtStartPar
复位值 |    描 | |
&
\begin{DUlineblock}{0em}
\item[] 
\end{DUlineblock}
\\
\sphinxhline
\sphinxAtStartPar
HALLIE
&
\sphinxAtStartPar
0x400
&&
\sphinxAtStartPar
0x00
&
\sphinxAtStartPar
HALL中断使能               |
\\
\sphinxbottomrule
\end{tabular}
\sphinxtableafterendhook\par
\sphinxattableend\end{savenotes}


\begin{savenotes}\sphinxattablestart
\sphinxthistablewithglobalstyle
\centering
\begin{tabular}[t]{\X{12}{96}\X{12}{96}\X{12}{96}\X{12}{96}\X{12}{96}\X{12}{96}\X{12}{96}\X{12}{96}}
\sphinxtoprule
\sphinxtableatstartofbodyhook
\sphinxAtStartPar
31
&
\sphinxAtStartPar
30
&
\sphinxAtStartPar
29
&
\sphinxAtStartPar
28
&
\sphinxAtStartPar
27
&
\sphinxAtStartPar
26
&
\sphinxAtStartPar
25
&
\sphinxAtStartPar
24
\\
\sphinxhline\begin{itemize}
\item {} 
\end{itemize}
&&&&&&&\\
\sphinxhline
\sphinxAtStartPar
23
&
\sphinxAtStartPar
22
&
\sphinxAtStartPar
21
&
\sphinxAtStartPar
20
&
\sphinxAtStartPar
19
&
\sphinxAtStartPar
18
&
\sphinxAtStartPar
17
&
\sphinxAtStartPar
16
\\
\sphinxhline\begin{itemize}
\item {} 
\end{itemize}
&&&&&&&\\
\sphinxhline
\sphinxAtStartPar
15
&
\sphinxAtStartPar
14
&
\sphinxAtStartPar
13
&
\sphinxAtStartPar
12
&
\sphinxAtStartPar
11
&
\sphinxAtStartPar
10
&
\sphinxAtStartPar
9
&
\sphinxAtStartPar
8
\\
\sphinxhline\begin{itemize}
\item {} 
\end{itemize}
&&&&&&&\\
\sphinxhline
\sphinxAtStartPar
7
&
\sphinxAtStartPar
6
&
\sphinxAtStartPar
5
&
\sphinxAtStartPar
4
&
\sphinxAtStartPar
3
&
\sphinxAtStartPar
2
&
\sphinxAtStartPar
1
&
\sphinxAtStartPar
0
\\
\sphinxhline\begin{itemize}
\item {} 
\end{itemize}
&&&&&&&\\
\sphinxbottomrule
\end{tabular}
\sphinxtableafterendhook\par
\sphinxattableend\end{savenotes}


\begin{savenotes}\sphinxattablestart
\sphinxthistablewithglobalstyle
\centering
\begin{tabular}[t]{\X{33}{99}\X{33}{99}\X{33}{99}}
\sphinxtoprule
\sphinxtableatstartofbodyhook
\sphinxAtStartPar
位域 |
&
\sphinxAtStartPar
名称     | |
&
\sphinxAtStartPar
描述                                        | |
\\
\sphinxhline
\sphinxAtStartPar
31:1
&\begin{itemize}
\item {} 
\end{itemize}
&\begin{itemize}
\item {} 
\end{itemize}
\\
\sphinxhline
\sphinxAtStartPar
0
&
\sphinxAtStartPar
HALL0
&
\sphinxAtStartPar
HALL0中断使能。                             |

\sphinxAtStartPar
1:HALL中断使能                             |

\sphinxAtStartPar
0:HALL中断禁能                             |

\sphinxAtStartPar
注:此功能对应Timer0                        |
\\
\sphinxbottomrule
\end{tabular}
\sphinxtableafterendhook\par
\sphinxattableend\end{savenotes}


\subsubsection{HALL中断状态寄存器HALLIF}
\label{\detokenize{SWM241/_u529f_u80fd_u63cf_u8ff0/_u52a0_u5f3a_u578b_u5b9a_u65f6_u5668:hallhallif}}

\begin{savenotes}\sphinxattablestart
\sphinxthistablewithglobalstyle
\centering
\begin{tabular}[t]{\X{20}{100}\X{20}{100}\X{20}{100}\X{20}{100}\X{20}{100}}
\sphinxtoprule
\sphinxtableatstartofbodyhook
\sphinxAtStartPar
寄存器 |
&
\begin{DUlineblock}{0em}
\item[] 偏移 |
\end{DUlineblock}
&
\begin{DUlineblock}{0em}
\item[] 
\end{DUlineblock}
&
\sphinxAtStartPar
复位值 |    描 | |
&
\begin{DUlineblock}{0em}
\item[] 
\end{DUlineblock}
\\
\sphinxhline
\sphinxAtStartPar
HALLIF
&
\sphinxAtStartPar
0x408
&&
\sphinxAtStartPar
0x00
&
\sphinxAtStartPar
HALL中断状态               |
\\
\sphinxbottomrule
\end{tabular}
\sphinxtableafterendhook\par
\sphinxattableend\end{savenotes}


\begin{savenotes}\sphinxattablestart
\sphinxthistablewithglobalstyle
\centering
\begin{tabular}[t]{\X{12}{96}\X{12}{96}\X{12}{96}\X{12}{96}\X{12}{96}\X{12}{96}\X{12}{96}\X{12}{96}}
\sphinxtoprule
\sphinxtableatstartofbodyhook
\sphinxAtStartPar
31
&
\sphinxAtStartPar
30
&
\sphinxAtStartPar
29
&
\sphinxAtStartPar
28
&
\sphinxAtStartPar
27
&
\sphinxAtStartPar
26
&
\sphinxAtStartPar
25
&
\sphinxAtStartPar
24
\\
\sphinxhline\begin{itemize}
\item {} 
\end{itemize}
&&&&&&&\\
\sphinxhline
\sphinxAtStartPar
23
&
\sphinxAtStartPar
22
&
\sphinxAtStartPar
21
&
\sphinxAtStartPar
20
&
\sphinxAtStartPar
19
&
\sphinxAtStartPar
18
&
\sphinxAtStartPar
17
&
\sphinxAtStartPar
16
\\
\sphinxhline\begin{itemize}
\item {} 
\end{itemize}
&&&&&&&\\
\sphinxhline
\sphinxAtStartPar
15
&
\sphinxAtStartPar
14
&
\sphinxAtStartPar
13
&
\sphinxAtStartPar
12
&
\sphinxAtStartPar
11
&
\sphinxAtStartPar
10
&
\sphinxAtStartPar
9
&
\sphinxAtStartPar
8
\\
\sphinxhline\begin{itemize}
\item {} 
\end{itemize}
&&&&&&&\\
\sphinxhline
\sphinxAtStartPar
7
&
\sphinxAtStartPar
6
&
\sphinxAtStartPar
5
&
\sphinxAtStartPar
4
&
\sphinxAtStartPar
3
&
\sphinxAtStartPar
2
&
\sphinxAtStartPar
1
&
\sphinxAtStartPar
0
\\
\sphinxhline\begin{itemize}
\item {} 
\end{itemize}
&&&&&
\sphinxAtStartPar
IN2
&
\sphinxAtStartPar
IN1
&
\sphinxAtStartPar
IN0
\\
\sphinxbottomrule
\end{tabular}
\sphinxtableafterendhook\par
\sphinxattableend\end{savenotes}


\begin{savenotes}\sphinxattablestart
\sphinxthistablewithglobalstyle
\centering
\begin{tabular}[t]{\X{33}{99}\X{33}{99}\X{33}{99}}
\sphinxtoprule
\sphinxtableatstartofbodyhook
\sphinxAtStartPar
位域 |
&
\sphinxAtStartPar
名称     | |
&
\sphinxAtStartPar
描述                                        | |
\\
\sphinxhline
\sphinxAtStartPar
31:6
&\begin{itemize}
\item {} 
\end{itemize}
&\begin{itemize}
\item {} 
\end{itemize}
\\
\sphinxhline
\sphinxAtStartPar
2
&
\sphinxAtStartPar
IN2
&
\sphinxAtStartPar
输入HALL.0信号2触发中断的状态,R/W1C        |

\sphinxAtStartPar
1:中断已发生                               |

\sphinxAtStartPar
0:中断未发生                               |
\\
\sphinxhline
\sphinxAtStartPar
1
&
\sphinxAtStartPar
IN1
&
\sphinxAtStartPar
输入HALL0信号1触发中断的状态,R/W1C         |

\sphinxAtStartPar
1:中断已发生                               |

\sphinxAtStartPar
0:中断未发生                               |
\\
\sphinxhline
\sphinxAtStartPar
0
&
\sphinxAtStartPar
IN0
&
\sphinxAtStartPar
输入HALL0信号0触发中断的状态,R/W1C         |

\sphinxAtStartPar
1:中断已发生                               |

\sphinxAtStartPar
0:中断未发生                               |
\\
\sphinxbottomrule
\end{tabular}
\sphinxtableafterendhook\par
\sphinxattableend\end{savenotes}


\subsubsection{HALL触发使能寄存器HALLEN}
\label{\detokenize{SWM241/_u529f_u80fd_u63cf_u8ff0/_u52a0_u5f3a_u578b_u5b9a_u65f6_u5668:hallhallen}}

\begin{savenotes}\sphinxattablestart
\sphinxthistablewithglobalstyle
\centering
\begin{tabular}[t]{\X{20}{100}\X{20}{100}\X{20}{100}\X{20}{100}\X{20}{100}}
\sphinxtoprule
\sphinxtableatstartofbodyhook
\sphinxAtStartPar
寄存器 |
&
\begin{DUlineblock}{0em}
\item[] 偏移 |
\end{DUlineblock}
&
\begin{DUlineblock}{0em}
\item[] 
\end{DUlineblock}
&
\sphinxAtStartPar
复位值 |    描 | |
&
\begin{DUlineblock}{0em}
\item[] 
\end{DUlineblock}
\\
\sphinxhline
\sphinxAtStartPar
HALLEN
&
\sphinxAtStartPar
0x40C
&&
\sphinxAtStartPar
0x00
&
\sphinxAtStartPar
HALL触发使能寄存器         |
\\
\sphinxbottomrule
\end{tabular}
\sphinxtableafterendhook\par
\sphinxattableend\end{savenotes}


\begin{savenotes}\sphinxattablestart
\sphinxthistablewithglobalstyle
\centering
\begin{tabular}[t]{\X{12}{96}\X{12}{96}\X{12}{96}\X{12}{96}\X{12}{96}\X{12}{96}\X{12}{96}\X{12}{96}}
\sphinxtoprule
\sphinxtableatstartofbodyhook
\sphinxAtStartPar
31
&
\sphinxAtStartPar
30
&
\sphinxAtStartPar
29
&
\sphinxAtStartPar
28
&
\sphinxAtStartPar
27
&
\sphinxAtStartPar
26
&
\sphinxAtStartPar
25
&
\sphinxAtStartPar
24
\\
\sphinxhline\begin{itemize}
\item {} 
\end{itemize}
&&&&&&&\\
\sphinxhline
\sphinxAtStartPar
23
&
\sphinxAtStartPar
22
&
\sphinxAtStartPar
21
&
\sphinxAtStartPar
20
&
\sphinxAtStartPar
19
&
\sphinxAtStartPar
18
&
\sphinxAtStartPar
17
&
\sphinxAtStartPar
16
\\
\sphinxhline\begin{itemize}
\item {} 
\end{itemize}
&&&&&&&\\
\sphinxhline
\sphinxAtStartPar
15
&
\sphinxAtStartPar
14
&
\sphinxAtStartPar
13
&
\sphinxAtStartPar
12
&
\sphinxAtStartPar
11
&
\sphinxAtStartPar
10
&
\sphinxAtStartPar
9
&
\sphinxAtStartPar
8
\\
\sphinxhline\begin{itemize}
\item {} 
\end{itemize}
&&&&&&&\\
\sphinxhline
\sphinxAtStartPar
7
&
\sphinxAtStartPar
6
&
\sphinxAtStartPar
5
&
\sphinxAtStartPar
4
&
\sphinxAtStartPar
3
&
\sphinxAtStartPar
2
&
\sphinxAtStartPar
1
&
\sphinxAtStartPar
0
\\
\sphinxhline\begin{itemize}
\item {} 
\end{itemize}
&&&&&&&\\
\sphinxbottomrule
\end{tabular}
\sphinxtableafterendhook\par
\sphinxattableend\end{savenotes}


\begin{savenotes}\sphinxattablestart
\sphinxthistablewithglobalstyle
\centering
\begin{tabular}[t]{\X{33}{99}\X{33}{99}\X{33}{99}}
\sphinxtoprule
\sphinxtableatstartofbodyhook
\sphinxAtStartPar
位域 |
&
\sphinxAtStartPar
名称     | |
&
\sphinxAtStartPar
描述                                        | |
\\
\sphinxhline
\sphinxAtStartPar
31:1
&\begin{itemize}
\item {} 
\end{itemize}
&\begin{itemize}
\item {} 
\end{itemize}
\\
\sphinxhline
\sphinxAtStartPar
0
&
\sphinxAtStartPar
HALL0
&
\sphinxAtStartPar
输入HALL0信号触发使能                       |

\sphinxAtStartPar
0:不触发                                   |

\sphinxAtStartPar
1:触发                                     |
\\
\sphinxbottomrule
\end{tabular}
\sphinxtableafterendhook\par
\sphinxattableend\end{savenotes}


\subsubsection{HALL0信号触发时,Timer0计数值寄存器HALLDR}
\label{\detokenize{SWM241/_u529f_u80fd_u63cf_u8ff0/_u52a0_u5f3a_u578b_u5b9a_u65f6_u5668:hall0-timer0halldr}}

\begin{savenotes}\sphinxattablestart
\sphinxthistablewithglobalstyle
\centering
\begin{tabular}[t]{\X{20}{100}\X{20}{100}\X{20}{100}\X{20}{100}\X{20}{100}}
\sphinxtoprule
\sphinxtableatstartofbodyhook
\sphinxAtStartPar
寄存器 |
&
\begin{DUlineblock}{0em}
\item[] 偏移 |
\end{DUlineblock}
&
\begin{DUlineblock}{0em}
\item[] 
\end{DUlineblock}
&
\sphinxAtStartPar
复位值 |    描 | |
&
\begin{DUlineblock}{0em}
\item[] 
\end{DUlineblock}
\\
\sphinxhline
\sphinxAtStartPar
HALLDR
&
\sphinxAtStartPar
0x410
&&
\sphinxAtStartPar
0x00
&\\
\sphinxbottomrule
\end{tabular}
\sphinxtableafterendhook\par
\sphinxattableend\end{savenotes}


\begin{savenotes}\sphinxattablestart
\sphinxthistablewithglobalstyle
\centering
\begin{tabular}[t]{\X{12}{96}\X{12}{96}\X{12}{96}\X{12}{96}\X{12}{96}\X{12}{96}\X{12}{96}\X{12}{96}}
\sphinxtoprule
\sphinxtableatstartofbodyhook
\sphinxAtStartPar
31
&
\sphinxAtStartPar
30
&
\sphinxAtStartPar
29
&
\sphinxAtStartPar
28
&
\sphinxAtStartPar
27
&
\sphinxAtStartPar
26
&
\sphinxAtStartPar
25
&
\sphinxAtStartPar
24
\\
\sphinxhline\begin{itemize}
\item {} 
\end{itemize}
&&&&&&&\\
\sphinxhline
\sphinxAtStartPar
23
&
\sphinxAtStartPar
22
&
\sphinxAtStartPar
21
&
\sphinxAtStartPar
20
&
\sphinxAtStartPar
19
&
\sphinxAtStartPar
18
&
\sphinxAtStartPar
17
&
\sphinxAtStartPar
16
\\
\sphinxhline
\sphinxAtStartPar
HALLDR
&&&&&&&\\
\sphinxhline
\sphinxAtStartPar
15
&
\sphinxAtStartPar
14
&
\sphinxAtStartPar
13
&
\sphinxAtStartPar
12
&
\sphinxAtStartPar
11
&
\sphinxAtStartPar
10
&
\sphinxAtStartPar
9
&
\sphinxAtStartPar
8
\\
\sphinxhline
\sphinxAtStartPar
HALLDR
&&&&&&&\\
\sphinxhline
\sphinxAtStartPar
7
&
\sphinxAtStartPar
6
&
\sphinxAtStartPar
5
&
\sphinxAtStartPar
4
&
\sphinxAtStartPar
3
&
\sphinxAtStartPar
2
&
\sphinxAtStartPar
1
&
\sphinxAtStartPar
0
\\
\sphinxhline
\sphinxAtStartPar
HALLDR
&&&&&&&\\
\sphinxbottomrule
\end{tabular}
\sphinxtableafterendhook\par
\sphinxattableend\end{savenotes}


\begin{savenotes}\sphinxattablestart
\sphinxthistablewithglobalstyle
\centering
\begin{tabular}[t]{\X{33}{99}\X{33}{99}\X{33}{99}}
\sphinxtoprule
\sphinxtableatstartofbodyhook
\sphinxAtStartPar
位域 |
&
\sphinxAtStartPar
名称     | |
&
\sphinxAtStartPar
描述                                        | |
\\
\sphinxhline
\sphinxAtStartPar
31:24
&\begin{itemize}
\item {} 
\end{itemize}
&\begin{itemize}
\item {} 
\end{itemize}
\\
\sphinxhline
\sphinxAtStartPar
23:0
&
\sphinxAtStartPar
HALLDR
&
\sphinxAtStartPar
HALL0信号触发时,计数器0的计数值。          |

\sphinxAtStartPar
HALL0输                                     | 沿将Timer0(加载值\sphinxhyphen{}当前值)存入此寄存器  |
\\
\sphinxbottomrule
\end{tabular}
\sphinxtableafterendhook\par
\sphinxattableend\end{savenotes}


\subsubsection{外部HALL输入信号的状态HALLSR}
\label{\detokenize{SWM241/_u529f_u80fd_u63cf_u8ff0/_u52a0_u5f3a_u578b_u5b9a_u65f6_u5668:hallhallsr}}

\begin{savenotes}\sphinxattablestart
\sphinxthistablewithglobalstyle
\centering
\begin{tabular}[t]{\X{20}{100}\X{20}{100}\X{20}{100}\X{20}{100}\X{20}{100}}
\sphinxtoprule
\sphinxtableatstartofbodyhook
\sphinxAtStartPar
寄存器 |
&
\begin{DUlineblock}{0em}
\item[] 偏移 |
\end{DUlineblock}
&
\begin{DUlineblock}{0em}
\item[] 
\end{DUlineblock}
&
\sphinxAtStartPar
复位值 |    描 | |
&
\begin{DUlineblock}{0em}
\item[] 
\end{DUlineblock}
\\
\sphinxhline
\sphinxAtStartPar
HALLSR
&
\sphinxAtStartPar
0x41C
&&
\sphinxAtStartPar
0x00
&
\sphinxAtStartPar
外部HALL输入信号的状态     |
\\
\sphinxbottomrule
\end{tabular}
\sphinxtableafterendhook\par
\sphinxattableend\end{savenotes}


\begin{savenotes}\sphinxattablestart
\sphinxthistablewithglobalstyle
\centering
\begin{tabular}[t]{\X{12}{96}\X{12}{96}\X{12}{96}\X{12}{96}\X{12}{96}\X{12}{96}\X{12}{96}\X{12}{96}}
\sphinxtoprule
\sphinxtableatstartofbodyhook
\sphinxAtStartPar
31
&
\sphinxAtStartPar
30
&
\sphinxAtStartPar
29
&
\sphinxAtStartPar
28
&
\sphinxAtStartPar
27
&
\sphinxAtStartPar
26
&
\sphinxAtStartPar
25
&
\sphinxAtStartPar
24
\\
\sphinxhline\begin{itemize}
\item {} 
\end{itemize}
&&&&&&&\\
\sphinxhline
\sphinxAtStartPar
23
&
\sphinxAtStartPar
22
&
\sphinxAtStartPar
21
&
\sphinxAtStartPar
20
&
\sphinxAtStartPar
19
&
\sphinxAtStartPar
18
&
\sphinxAtStartPar
17
&
\sphinxAtStartPar
16
\\
\sphinxhline\begin{itemize}
\item {} 
\end{itemize}
&&&&&&&\\
\sphinxhline
\sphinxAtStartPar
15
&
\sphinxAtStartPar
14
&
\sphinxAtStartPar
13
&
\sphinxAtStartPar
12
&
\sphinxAtStartPar
11
&
\sphinxAtStartPar
10
&
\sphinxAtStartPar
9
&
\sphinxAtStartPar
8
\\
\sphinxhline\begin{itemize}
\item {} 
\end{itemize}
&&&&&&&\\
\sphinxhline
\sphinxAtStartPar
7
&
\sphinxAtStartPar
6
&
\sphinxAtStartPar
5
&
\sphinxAtStartPar
4
&
\sphinxAtStartPar
3
&
\sphinxAtStartPar
2
&
\sphinxAtStartPar
1
&
\sphinxAtStartPar
0
\\
\sphinxhline\begin{itemize}
\item {} 
\end{itemize}
&&&&&
\sphinxAtStartPar
IN2
&
\sphinxAtStartPar
IN1
&
\sphinxAtStartPar
IN0
\\
\sphinxbottomrule
\end{tabular}
\sphinxtableafterendhook\par
\sphinxattableend\end{savenotes}


\begin{savenotes}\sphinxattablestart
\sphinxthistablewithglobalstyle
\centering
\begin{tabular}[t]{\X{33}{99}\X{33}{99}\X{33}{99}}
\sphinxtoprule
\sphinxtableatstartofbodyhook
\sphinxAtStartPar
位域 |
&
\sphinxAtStartPar
名称     | |
&
\sphinxAtStartPar
描述                                        | |
\\
\sphinxhline
\sphinxAtStartPar
31:3
&\begin{itemize}
\item {} 
\end{itemize}
&\begin{itemize}
\item {} 
\end{itemize}
\\
\sphinxhline
\sphinxAtStartPar
2
&
\sphinxAtStartPar
IN2
&
\sphinxAtStartPar
输入HALL0输入信号2当前状态,R/W             |

\sphinxAtStartPar
1:中断已发生                               |

\sphinxAtStartPar
0:中断未发生                               |
\\
\sphinxhline
\sphinxAtStartPar
1
&
\sphinxAtStartPar
IN1
&
\sphinxAtStartPar
输入HALL0输入信号1当前状态,R/W             |

\sphinxAtStartPar
1:中断已发生                               |

\sphinxAtStartPar
0:中断未发生                               |
\\
\sphinxhline
\sphinxAtStartPar
0
&
\sphinxAtStartPar
IN0
&
\sphinxAtStartPar
输入HALL0输入信号0当前状态,R/W             |

\sphinxAtStartPar
1:中断已发生                               |

\sphinxAtStartPar
0:中断未发生                               |
\\
\sphinxbottomrule
\end{tabular}
\sphinxtableafterendhook\par
\sphinxattableend\end{savenotes}


\subsubsection{使能寄存器EN}
\label{\detokenize{SWM241/_u529f_u80fd_u63cf_u8ff0/_u52a0_u5f3a_u578b_u5b9a_u65f6_u5668:en}}

\begin{savenotes}\sphinxattablestart
\sphinxthistablewithglobalstyle
\centering
\begin{tabular}[t]{\X{20}{100}\X{20}{100}\X{20}{100}\X{20}{100}\X{20}{100}}
\sphinxtoprule
\sphinxtableatstartofbodyhook
\sphinxAtStartPar
寄存器 |
&
\begin{DUlineblock}{0em}
\item[] 偏移 |
\end{DUlineblock}
&
\begin{DUlineblock}{0em}
\item[] 
\end{DUlineblock}
&
\sphinxAtStartPar
复位值 |    描 | |
&
\begin{DUlineblock}{0em}
\item[] 
\end{DUlineblock}
\\
\sphinxhline
\sphinxAtStartPar
EN
&
\sphinxAtStartPar
0x440
&&
\sphinxAtStartPar
0x00
&
\sphinxAtStartPar
TIMER使能寄存器            |
\\
\sphinxbottomrule
\end{tabular}
\sphinxtableafterendhook\par
\sphinxattableend\end{savenotes}


\begin{savenotes}\sphinxattablestart
\sphinxthistablewithglobalstyle
\centering
\begin{tabular}[t]{\X{12}{96}\X{12}{96}\X{12}{96}\X{12}{96}\X{12}{96}\X{12}{96}\X{12}{96}\X{12}{96}}
\sphinxtoprule
\sphinxtableatstartofbodyhook
\sphinxAtStartPar
31
&
\sphinxAtStartPar
30
&
\sphinxAtStartPar
29
&
\sphinxAtStartPar
28
&
\sphinxAtStartPar
27
&
\sphinxAtStartPar
26
&
\sphinxAtStartPar
25
&
\sphinxAtStartPar
24
\\
\sphinxhline\begin{itemize}
\item {} 
\end{itemize}
&&&&&&&\\
\sphinxhline
\sphinxAtStartPar
23
&
\sphinxAtStartPar
22
&
\sphinxAtStartPar
21
&
\sphinxAtStartPar
20
&
\sphinxAtStartPar
19
&
\sphinxAtStartPar
18
&
\sphinxAtStartPar
17
&
\sphinxAtStartPar
16
\\
\sphinxhline\begin{itemize}
\item {} 
\end{itemize}
&&&&&&&\\
\sphinxhline
\sphinxAtStartPar
15
&
\sphinxAtStartPar
14
&
\sphinxAtStartPar
13
&
\sphinxAtStartPar
12
&
\sphinxAtStartPar
11
&
\sphinxAtStartPar
10
&
\sphinxAtStartPar
9
&
\sphinxAtStartPar
8
\\
\sphinxhline\begin{itemize}
\item {} 
\end{itemize}
&&&&&&&\\
\sphinxhline
\sphinxAtStartPar
7
&
\sphinxAtStartPar
6
&
\sphinxAtStartPar
5
&
\sphinxAtStartPar
4
&
\sphinxAtStartPar
3
&
\sphinxAtStartPar
2
&
\sphinxAtStartPar
1
&
\sphinxAtStartPar
0
\\
\sphinxhline
\sphinxAtStartPar
EN7
&
\sphinxAtStartPar
EN6
&
\sphinxAtStartPar
EN5
&
\sphinxAtStartPar
EN4
&
\sphinxAtStartPar
EN3
&
\sphinxAtStartPar
EN2
&
\sphinxAtStartPar
EN1
&
\sphinxAtStartPar
EN0
\\
\sphinxbottomrule
\end{tabular}
\sphinxtableafterendhook\par
\sphinxattableend\end{savenotes}


\begin{savenotes}\sphinxattablestart
\sphinxthistablewithglobalstyle
\centering
\begin{tabular}[t]{\X{33}{99}\X{33}{99}\X{33}{99}}
\sphinxtoprule
\sphinxtableatstartofbodyhook
\sphinxAtStartPar
位域 |
&
\sphinxAtStartPar
名称     | |
&
\sphinxAtStartPar
描述                                        | |
\\
\sphinxhline
\sphinxAtStartPar
31:8
&\begin{itemize}
\item {} 
\end{itemize}
&\begin{itemize}
\item {} 
\end{itemize}
\\
\sphinxhline
\sphinxAtStartPar
7
&
\sphinxAtStartPar
EN7
&
\sphinxAtStartPar
TIMER7使能                                  |

\sphinxAtStartPar
1:使能                                     |

\sphinxAtStartPar
0:禁能                                     |
\\
\sphinxhline
\sphinxAtStartPar
6
&
\sphinxAtStartPar
EN6
&
\sphinxAtStartPar
TIMER6使能                                  |

\sphinxAtStartPar
1:使能                                     |

\sphinxAtStartPar
0:禁能                                     |
\\
\sphinxhline
\sphinxAtStartPar
5
&
\sphinxAtStartPar
EN5
&
\sphinxAtStartPar
TIMER5使能                                  |

\sphinxAtStartPar
1:使能                                     |

\sphinxAtStartPar
0:禁能                                     |
\\
\sphinxhline
\sphinxAtStartPar
4
&
\sphinxAtStartPar
EN4
&
\sphinxAtStartPar
TIMER4使能                                  |

\sphinxAtStartPar
1:使能                                     |

\sphinxAtStartPar
0:禁能                                     |
\\
\sphinxhline
\sphinxAtStartPar
3
&
\sphinxAtStartPar
EN3
&
\sphinxAtStartPar
TIMER3使能                                  |

\sphinxAtStartPar
1:使能                                     |

\sphinxAtStartPar
0:禁能                                     |
\\
\sphinxhline
\sphinxAtStartPar
2
&
\sphinxAtStartPar
EN2
&
\sphinxAtStartPar
TIMER2使能                                  |

\sphinxAtStartPar
1:使能                                     |

\sphinxAtStartPar
0:禁能                                     |
\\
\sphinxhline
\sphinxAtStartPar
1
&
\sphinxAtStartPar
EN1
&
\sphinxAtStartPar
TIMER1使能                                  |

\sphinxAtStartPar
1:使能                                     |

\sphinxAtStartPar
0:禁能                                     |
\\
\sphinxhline
\sphinxAtStartPar
0
&
\sphinxAtStartPar
EN0
&
\sphinxAtStartPar
TIMER0使能                                  |

\sphinxAtStartPar
1:使能                                     |

\sphinxAtStartPar
0:禁能                                     |
\\
\sphinxbottomrule
\end{tabular}
\sphinxtableafterendhook\par
\sphinxattableend\end{savenotes}

\sphinxstepscope


\section{看门狗定时器(WDT)}
\label{\detokenize{SWM241/_u529f_u80fd_u63cf_u8ff0/_u770b_u95e8_u72d7_u5b9a_u65f6_u5668:wdt}}\label{\detokenize{SWM241/_u529f_u80fd_u63cf_u8ff0/_u770b_u95e8_u72d7_u5b9a_u65f6_u5668::doc}}
\sphinxAtStartPar
概述
\textasciitilde{}\textasciitilde{}

\sphinxAtStartPar
SWM241系列所有型号WDT操作均相同。使用前需使能对应WDT模块时钟。

\sphinxAtStartPar
看门狗定时器(WDT)主要用于控制程序流程正确,在程序流长时间未按既定流程执行指定程序的情况下产生中断或复位芯片。

\sphinxAtStartPar
特性
\textasciitilde{}\textasciitilde{}
\begin{itemize}
\item {} 
\sphinxAtStartPar
产生计数器溢出复位信号,复位信号使能可配

\item {} 
\sphinxAtStartPar
具有16位计数位宽,可配置灵活、宽范围的溢出周期

\item {} 
\sphinxAtStartPar
具有中断功能

\item {} 
\sphinxAtStartPar
时钟源为32K

\end{itemize}


\subsection{模块结构框图}
\label{\detokenize{SWM241/_u529f_u80fd_u63cf_u8ff0/_u770b_u95e8_u72d7_u5b9a_u65f6_u5668:id1}}
\sphinxAtStartPar
\sphinxincludegraphics{{SWM241/功能描述/media看门狗定时器002}.emf}

\sphinxAtStartPar
图 6‑15 WDT模块结构框图


\subsection{功能描述}
\label{\detokenize{SWM241/_u529f_u80fd_u63cf_u8ff0/_u770b_u95e8_u72d7_u5b9a_u65f6_u5668:id2}}

\subsubsection{配置方式}
\label{\detokenize{SWM241/_u529f_u80fd_u63cf_u8ff0/_u770b_u95e8_u72d7_u5b9a_u65f6_u5668:id3}}
\sphinxAtStartPar
看门狗定时器(WDT)主要用于控制程序流程正确,在程序流程时间未按既定流程执行指定程序的情况下产生中断或复位芯片。

\sphinxAtStartPar
看门狗发生中断及复位与计数值之间的关系示意图如图 6‑16所示:

\sphinxAtStartPar
\sphinxincludegraphics{{SWM241/功能描述/media看门狗定时器003}.emf}

\sphinxAtStartPar
图 6‑16 门狗发生中断及复位与计数值之间的关系示意图

\sphinxAtStartPar
配置方式如下:
\begin{itemize}
\item {} 
\sphinxAtStartPar
配置复位值寄存器RSTVAL,设置复位值,WDT为递增计数

\item {} 
\sphinxAtStartPar
配置控制寄存器CR中RSTEN位,设置以系统时钟为单位递增时产生中断或产生复位

\item {} 
\sphinxAtStartPar
将控制寄存器CR中EN位置1,使能WDT模块

\item {} 
\sphinxAtStartPar
程序执行过程中通过向FEED寄存器写入0x55喂狗,重启计数

\item {} 
\sphinxAtStartPar
若当VALUE寄存器加至RSTVAL,依然未执行喂狗操作,则根据CR寄存器设置,产生中断或复位信号

\end{itemize}

\sphinxAtStartPar
工作示意图如图 6‑17所示:

\sphinxAtStartPar
\sphinxincludegraphics{{SWM241/功能描述/media看门狗定时器004}.emf}

\sphinxAtStartPar
图 6‑17 WDT工作示意图

\sphinxAtStartPar
控制寄存器CR中RSTEN位配置为复位使能时,使能后波形如图 6‑18所示:

\sphinxAtStartPar
\sphinxincludegraphics{{SWM241/功能描述/media看门狗定时器005}.emf}

\sphinxAtStartPar
图 6‑18 WDT配置为RESET模式波形图

\sphinxAtStartPar
控制寄存器CR中RSTEN位配置为复位失能时,使能后波形如图 6‑19所示,中断产生后,通过IF寄存器进行清除。

\sphinxAtStartPar
\sphinxincludegraphics{{SWM241/功能描述/media看门狗定时器006}.emf}

\sphinxAtStartPar
图 6‑19 WDT配置为中断模式波形图


\subsubsection{中断配置与清除}
\label{\detokenize{SWM241/_u529f_u80fd_u63cf_u8ff0/_u770b_u95e8_u72d7_u5b9a_u65f6_u5668:id4}}
\sphinxAtStartPar
可通过配置WDT控制寄存器CR设置以系统时钟为单位递增时产生中断,并使能中断,启动WDT,当VALUE寄存器加至RSTVAL,依然未执行喂狗操作时,中断标志寄存器IF位置1。如需清除此标志,需在标志位中写1清零(R/W1C),否则中断在开启状态下会一直进入。


\subsection{寄存器映射}
\label{\detokenize{SWM241/_u529f_u80fd_u63cf_u8ff0/_u770b_u95e8_u72d7_u5b9a_u65f6_u5668:id5}}
\sphinxAtStartPar
下表列出了WDT模块的相关寄存器,所列偏移量为寄存器相对于WDT模块基址的16进制增量:


\begin{savenotes}\sphinxattablestart
\sphinxthistablewithglobalstyle
\centering
\begin{tabular}[t]{\X{20}{100}\X{20}{100}\X{20}{100}\X{20}{100}\X{20}{100}}
\sphinxtoprule
\sphinxtableatstartofbodyhook
\sphinxAtStartPar
名称   |
&
\begin{DUlineblock}{0em}
\item[] 偏移 |
\end{DUlineblock}
&
\begin{DUlineblock}{0em}
\item[] 
\item[] |
|
\end{DUlineblock}
&
\begin{DUlineblock}{0em}
\item[] 
\end{DUlineblock}
\begin{quote}

\begin{DUlineblock}{0em}
\item[] 
\item[] 
\end{DUlineblock}
\end{quote}
&
\sphinxAtStartPar
描述                       | | | |
\\
\sphinxhline
\sphinxAtStartPar
WDTBASE:0 {\color{red}\bfseries{}|}x400A0800
&
\begin{DUlineblock}{0em}
\item[] 
\end{DUlineblock}
&&&\\
\sphinxhline
\sphinxAtStartPar
RSTVAL
&
\sphinxAtStartPar
0x00
&&
\sphinxAtStartPar
0x 0FFFF
&
\sphinxAtStartPar
WDT复位值寄存器            |
\\
\sphinxhline
\sphinxAtStartPar
INTVAL
&
\sphinxAtStartPar
0x04
&&
\sphinxAtStartPar
0x 0FFFF
&
\sphinxAtStartPar
WDT中断值寄存器            |
\\
\sphinxhline
\sphinxAtStartPar
CR
&
\sphinxAtStartPar
0x08
&&
\sphinxAtStartPar
0x 00000
&
\sphinxAtStartPar
WDT控制寄存器              |
\\
\sphinxhline
\sphinxAtStartPar
IF
&
\sphinxAtStartPar
0x0C
&&
\sphinxAtStartPar
0x 00000
&
\sphinxAtStartPar
WDT中断状态寄存器          |
\\
\sphinxhline
\sphinxAtStartPar
FEED
&
\sphinxAtStartPar
0x10
&&
\sphinxAtStartPar
0x 00000
&
\sphinxAtStartPar
WDT重启计数器寄存器        |
\\
\sphinxbottomrule
\end{tabular}
\sphinxtableafterendhook\par
\sphinxattableend\end{savenotes}


\subsection{寄存器描述}
\label{\detokenize{SWM241/_u529f_u80fd_u63cf_u8ff0/_u770b_u95e8_u72d7_u5b9a_u65f6_u5668:id8}}

\subsubsection{WDT复位值寄存器RSTVAL}
\label{\detokenize{SWM241/_u529f_u80fd_u63cf_u8ff0/_u770b_u95e8_u72d7_u5b9a_u65f6_u5668:wdtrstval}}

\begin{savenotes}\sphinxattablestart
\sphinxthistablewithglobalstyle
\centering
\begin{tabular}[t]{\X{20}{100}\X{20}{100}\X{20}{100}\X{20}{100}\X{20}{100}}
\sphinxtoprule
\sphinxtableatstartofbodyhook
\sphinxAtStartPar
寄存器 |
&
\begin{DUlineblock}{0em}
\item[] 偏移 |
\end{DUlineblock}
&
\begin{DUlineblock}{0em}
\item[] 
\item[] {\color{red}\bfseries{}|}
\end{DUlineblock}
&
\sphinxAtStartPar
复位值 |    描 | |
&
\begin{DUlineblock}{0em}
\item[] |
  |
\end{DUlineblock}
\\
\sphinxhline
\sphinxAtStartPar
RSTVAL
&
\sphinxAtStartPar
0x00
&&
\sphinxAtStartPar
0 00FFFF
&
\sphinxAtStartPar
WDT 复位值寄存器           |
\\
\sphinxbottomrule
\end{tabular}
\sphinxtableafterendhook\par
\sphinxattableend\end{savenotes}


\begin{savenotes}\sphinxattablestart
\sphinxthistablewithglobalstyle
\centering
\begin{tabular}[t]{\X{12}{96}\X{12}{96}\X{12}{96}\X{12}{96}\X{12}{96}\X{12}{96}\X{12}{96}\X{12}{96}}
\sphinxtoprule
\sphinxtableatstartofbodyhook
\sphinxAtStartPar
31
&
\sphinxAtStartPar
30
&
\sphinxAtStartPar
29
&
\sphinxAtStartPar
28
&
\sphinxAtStartPar
27
&
\sphinxAtStartPar
26
&
\sphinxAtStartPar
25
&
\sphinxAtStartPar
24
\\
\sphinxhline\begin{itemize}
\item {} 
\end{itemize}
&&&&&&&\\
\sphinxhline
\sphinxAtStartPar
23
&
\sphinxAtStartPar
22
&
\sphinxAtStartPar
21
&
\sphinxAtStartPar
20
&
\sphinxAtStartPar
19
&
\sphinxAtStartPar
18
&
\sphinxAtStartPar
17
&
\sphinxAtStartPar
16
\\
\sphinxhline\begin{itemize}
\item {} 
\end{itemize}
&&&&&&&\\
\sphinxhline
\sphinxAtStartPar
15
&
\sphinxAtStartPar
14
&
\sphinxAtStartPar
13
&
\sphinxAtStartPar
12
&
\sphinxAtStartPar
11
&
\sphinxAtStartPar
10
&
\sphinxAtStartPar
9
&
\sphinxAtStartPar
8
\\
\sphinxhline
\sphinxAtStartPar
RSTVAL
&&&&&&&\\
\sphinxhline
\sphinxAtStartPar
7
&
\sphinxAtStartPar
6
&
\sphinxAtStartPar
5
&
\sphinxAtStartPar
4
&
\sphinxAtStartPar
3
&
\sphinxAtStartPar
2
&
\sphinxAtStartPar
1
&
\sphinxAtStartPar
0
\\
\sphinxhline
\sphinxAtStartPar
RSTVAL
&&&&&&&\\
\sphinxbottomrule
\end{tabular}
\sphinxtableafterendhook\par
\sphinxattableend\end{savenotes}


\begin{savenotes}\sphinxattablestart
\sphinxthistablewithglobalstyle
\centering
\begin{tabular}[t]{\X{33}{99}\X{33}{99}\X{33}{99}}
\sphinxtoprule
\sphinxtableatstartofbodyhook
\sphinxAtStartPar
位域 |
&
\sphinxAtStartPar
名称     | |
&
\sphinxAtStartPar
描述                                        | |
\\
\sphinxhline
\sphinxAtStartPar
31:16
&\begin{itemize}
\item {} 
\end{itemize}
&\begin{itemize}
\item {} 
\end{itemize}
\\
\sphinxhline
\sphinxAtStartPar
15:0
&
\sphinxAtStartPar
RSTVAL
&
\sphinxAtStartPar
WDT计数器的复位计数初始值。                 |

\sphinxAtStartPar
DT计数值计数到该寄存器设置值时,产生复位。 |

\sphinxAtStartPar
使能后配置无效                              |
\\
\sphinxbottomrule
\end{tabular}
\sphinxtableafterendhook\par
\sphinxattableend\end{savenotes}


\subsubsection{WDT中断值寄存器INTVAL}
\label{\detokenize{SWM241/_u529f_u80fd_u63cf_u8ff0/_u770b_u95e8_u72d7_u5b9a_u65f6_u5668:wdtintval}}

\begin{savenotes}\sphinxattablestart
\sphinxthistablewithglobalstyle
\centering
\begin{tabular}[t]{\X{20}{100}\X{20}{100}\X{20}{100}\X{20}{100}\X{20}{100}}
\sphinxtoprule
\sphinxtableatstartofbodyhook
\sphinxAtStartPar
寄存器 |
&
\begin{DUlineblock}{0em}
\item[] 偏移 |
\end{DUlineblock}
&
\begin{DUlineblock}{0em}
\item[] 
\item[] {\color{red}\bfseries{}|}
\end{DUlineblock}
&
\sphinxAtStartPar
复位值 |    描 | |
&
\begin{DUlineblock}{0em}
\item[] |
  |
\end{DUlineblock}
\\
\sphinxhline
\sphinxAtStartPar
INTVAL
&
\sphinxAtStartPar
0x04
&&
\sphinxAtStartPar
0 00FFFF
&
\sphinxAtStartPar
WDT中断值寄存器            |
\\
\sphinxbottomrule
\end{tabular}
\sphinxtableafterendhook\par
\sphinxattableend\end{savenotes}


\begin{savenotes}\sphinxattablestart
\sphinxthistablewithglobalstyle
\centering
\begin{tabular}[t]{\X{12}{96}\X{12}{96}\X{12}{96}\X{12}{96}\X{12}{96}\X{12}{96}\X{12}{96}\X{12}{96}}
\sphinxtoprule
\sphinxtableatstartofbodyhook
\sphinxAtStartPar
31
&
\sphinxAtStartPar
30
&
\sphinxAtStartPar
29
&
\sphinxAtStartPar
28
&
\sphinxAtStartPar
27
&
\sphinxAtStartPar
26
&
\sphinxAtStartPar
25
&
\sphinxAtStartPar
24
\\
\sphinxhline\begin{itemize}
\item {} 
\end{itemize}
&&&&&&&\\
\sphinxhline
\sphinxAtStartPar
23
&
\sphinxAtStartPar
22
&
\sphinxAtStartPar
21
&
\sphinxAtStartPar
20
&
\sphinxAtStartPar
19
&
\sphinxAtStartPar
18
&
\sphinxAtStartPar
17
&
\sphinxAtStartPar
16
\\
\sphinxhline\begin{itemize}
\item {} 
\end{itemize}
&&&&&&&\\
\sphinxhline
\sphinxAtStartPar
15
&
\sphinxAtStartPar
14
&
\sphinxAtStartPar
13
&
\sphinxAtStartPar
12
&
\sphinxAtStartPar
11
&
\sphinxAtStartPar
10
&
\sphinxAtStartPar
9
&
\sphinxAtStartPar
8
\\
\sphinxhline
\sphinxAtStartPar
INTVAL
&&&&&&&\\
\sphinxhline
\sphinxAtStartPar
7
&
\sphinxAtStartPar
6
&
\sphinxAtStartPar
5
&
\sphinxAtStartPar
4
&
\sphinxAtStartPar
3
&
\sphinxAtStartPar
2
&
\sphinxAtStartPar
1
&
\sphinxAtStartPar
0
\\
\sphinxhline
\sphinxAtStartPar
INTVAL
&&&&&&&\\
\sphinxbottomrule
\end{tabular}
\sphinxtableafterendhook\par
\sphinxattableend\end{savenotes}


\begin{savenotes}\sphinxattablestart
\sphinxthistablewithglobalstyle
\centering
\begin{tabular}[t]{\X{33}{99}\X{33}{99}\X{33}{99}}
\sphinxtoprule
\sphinxtableatstartofbodyhook
\sphinxAtStartPar
位域 |
&
\sphinxAtStartPar
名称     | |
&
\sphinxAtStartPar
描述                                        | |
\\
\sphinxhline
\sphinxAtStartPar
31:16
&\begin{itemize}
\item {} 
\end{itemize}
&\begin{itemize}
\item {} 
\end{itemize}
\\
\sphinxhline
\sphinxAtStartPar
15:0
&
\sphinxAtStartPar
INTVAL
&
\sphinxAtStartPar
WDT计数器中断目标值                         |

\sphinxAtStartPar
当W                                         | 数值递增计数到该寄存器设置值时,产生中断; |

\sphinxAtStartPar
产生中断后,若未设置复位值则重              | 计数,若设置复位值,则继续计数直至复位;  |

\sphinxAtStartPar
当中断与复位同时                            | ,INTVAL需要小于RSTVAL,产生中断后,若未 | 狗操作,则计数器继续计数,直至产生复位; |

\sphinxAtStartPar
使能后配置无效;                            |

\sphinxAtStartPar
当CR寄存器WINEN位为1                        | 未发生中断时喂狗,则直接发生看门狗复位。 |

\sphinxAtStartPar
当CR寄存                                    | NEN位为0时,发生看门狗复位跟喂狗没有关系; |
\\
\sphinxbottomrule
\end{tabular}
\sphinxtableafterendhook\par
\sphinxattableend\end{savenotes}


\subsubsection{WDT控制寄存器CR}
\label{\detokenize{SWM241/_u529f_u80fd_u63cf_u8ff0/_u770b_u95e8_u72d7_u5b9a_u65f6_u5668:wdtcr}}

\begin{savenotes}\sphinxattablestart
\sphinxthistablewithglobalstyle
\centering
\begin{tabular}[t]{\X{20}{100}\X{20}{100}\X{20}{100}\X{20}{100}\X{20}{100}}
\sphinxtoprule
\sphinxtableatstartofbodyhook
\sphinxAtStartPar
寄存器 |
&
\begin{DUlineblock}{0em}
\item[] 偏移 |
\end{DUlineblock}
&
\begin{DUlineblock}{0em}
\item[] 
\item[] {\color{red}\bfseries{}|}
\end{DUlineblock}
&
\sphinxAtStartPar
复位值 |    描 | |
&
\begin{DUlineblock}{0em}
\item[] |
  |
\end{DUlineblock}
\\
\sphinxhline
\sphinxAtStartPar
CR
&
\sphinxAtStartPar
0x08
&&
\sphinxAtStartPar
0 000000
&
\sphinxAtStartPar
WDT控制寄存器              |
\\
\sphinxbottomrule
\end{tabular}
\sphinxtableafterendhook\par
\sphinxattableend\end{savenotes}


\begin{savenotes}\sphinxattablestart
\sphinxthistablewithglobalstyle
\centering
\begin{tabular}[t]{\X{12}{96}\X{12}{96}\X{12}{96}\X{12}{96}\X{12}{96}\X{12}{96}\X{12}{96}\X{12}{96}}
\sphinxtoprule
\sphinxtableatstartofbodyhook
\sphinxAtStartPar
31
&
\sphinxAtStartPar
30
&
\sphinxAtStartPar
29
&
\sphinxAtStartPar
28
&
\sphinxAtStartPar
27
&
\sphinxAtStartPar
26
&
\sphinxAtStartPar
25
&
\sphinxAtStartPar
24
\\
\sphinxhline\begin{itemize}
\item {} 
\end{itemize}
&&&&&&&\\
\sphinxhline
\sphinxAtStartPar
23
&
\sphinxAtStartPar
22
&
\sphinxAtStartPar
21
&
\sphinxAtStartPar
20
&
\sphinxAtStartPar
19
&
\sphinxAtStartPar
18
&
\sphinxAtStartPar
17
&
\sphinxAtStartPar
16
\\
\sphinxhline\begin{itemize}
\item {} 
\end{itemize}
&&&&&&&\\
\sphinxhline
\sphinxAtStartPar
15
&
\sphinxAtStartPar
14
&
\sphinxAtStartPar
13
&
\sphinxAtStartPar
12
&
\sphinxAtStartPar
11
&
\sphinxAtStartPar
10
&
\sphinxAtStartPar
9
&
\sphinxAtStartPar
8
\\
\sphinxhline\begin{itemize}
\item {} 
\end{itemize}
&&&&
\sphinxAtStartPar
CLKDIV
&&&\\
\sphinxhline
\sphinxAtStartPar
7
&
\sphinxAtStartPar
6
&
\sphinxAtStartPar
5
&
\sphinxAtStartPar
4
&
\sphinxAtStartPar
3
&
\sphinxAtStartPar
2
&
\sphinxAtStartPar
1
&
\sphinxAtStartPar
0
\\
\sphinxhline\begin{itemize}
\item {} 
\end{itemize}
&&&&
\sphinxAtStartPar
WINEN
&
\sphinxAtStartPar
INTEN
&
\sphinxAtStartPar
RSTEN
&
\sphinxAtStartPar
EN
\\
\sphinxbottomrule
\end{tabular}
\sphinxtableafterendhook\par
\sphinxattableend\end{savenotes}


\begin{savenotes}\sphinxattablestart
\sphinxthistablewithglobalstyle
\centering
\begin{tabular}[t]{\X{33}{99}\X{33}{99}\X{33}{99}}
\sphinxtoprule
\sphinxtableatstartofbodyhook
\sphinxAtStartPar
位域 |
&
\sphinxAtStartPar
名称     | |
&
\sphinxAtStartPar
描述                                        | |
\\
\sphinxhline
\sphinxAtStartPar
31:12
&\begin{itemize}
\item {} 
\end{itemize}
&\begin{itemize}
\item {} 
\end{itemize}
\\
\sphinxhline
\sphinxAtStartPar
11:8
&
\sphinxAtStartPar
CLKDIV
&
\sphinxAtStartPar
看门狗计数时钟预分频寄存器                  |

\sphinxAtStartPar
0000:2                                     |

\sphinxAtStartPar
0001:4                                     |

\sphinxAtStartPar
0010:8                                     |

\sphinxAtStartPar
0011:16                                    |

\sphinxAtStartPar
0100:32                                    |

\sphinxAtStartPar
0101:64                                    |

\sphinxAtStartPar
0110:128                                   |

\sphinxAtStartPar
0111:256                                   |

\sphinxAtStartPar
1000:512                                   |

\sphinxAtStartPar
1001:1024                                  |

\sphinxAtStartPar
1010:2048                                  |

\sphinxAtStartPar
1011:4096                                  |

\sphinxAtStartPar
1100:8192                                  |

\sphinxAtStartPar
1101:16384                                 |

\sphinxAtStartPar
1110:32768                                 |

\sphinxAtStartPar
1111:65536                                 |
\\
\sphinxhline
\sphinxAtStartPar
7:4
&\begin{itemize}
\item {} 
\end{itemize}
&\begin{itemize}
\item {} 
\end{itemize}
\\
\sphinxhline
\sphinxAtStartPar
3
&
\sphinxAtStartPar
WINEN
&
\sphinxAtStartPar
WDT窗口功能使能                             |

\sphinxAtStartPar
1:使能窗口功能                             |

\sphinxAtStartPar
0:禁止窗口功能                             |
\\
\sphinxhline
\sphinxAtStartPar
2
&
\sphinxAtStartPar
INTEN
&
\sphinxAtStartPar
WDT中断输出使能位                           |

\sphinxAtStartPar
1:使能中断                                 |

\sphinxAtStartPar
0:禁止中断                                 |
\\
\sphinxhline
\sphinxAtStartPar
1
&
\sphinxAtStartPar
RSTEN
&
\sphinxAtStartPar
WDT复位输出使能位                           |

\sphinxAtStartPar
1:使能复位                                 |

\sphinxAtStartPar
0:禁止复位                                 |
\\
\sphinxhline
\sphinxAtStartPar
0
&
\sphinxAtStartPar
EN
&
\sphinxAtStartPar
WDT启动位                                   |

\sphinxAtStartPar
1:启动WDT计数                              |

\sphinxAtStartPar
0:停止计数                                 |
\\
\sphinxbottomrule
\end{tabular}
\sphinxtableafterendhook\par
\sphinxattableend\end{savenotes}


\subsubsection{WDT中断状态寄存器IF}
\label{\detokenize{SWM241/_u529f_u80fd_u63cf_u8ff0/_u770b_u95e8_u72d7_u5b9a_u65f6_u5668:wdtif}}

\begin{savenotes}\sphinxattablestart
\sphinxthistablewithglobalstyle
\centering
\begin{tabular}[t]{\X{20}{100}\X{20}{100}\X{20}{100}\X{20}{100}\X{20}{100}}
\sphinxtoprule
\sphinxtableatstartofbodyhook
\sphinxAtStartPar
寄存器 |
&
\begin{DUlineblock}{0em}
\item[] 偏移 |
\end{DUlineblock}
&
\begin{DUlineblock}{0em}
\item[] 
\item[] {\color{red}\bfseries{}|}
\end{DUlineblock}
&
\sphinxAtStartPar
复位值 |    描 | |
&
\begin{DUlineblock}{0em}
\item[] |
  |
\end{DUlineblock}
\\
\sphinxhline
\sphinxAtStartPar
IF
&
\sphinxAtStartPar
0x0C
&&
\sphinxAtStartPar
0 000000
&
\sphinxAtStartPar
WDT中断状态寄存器          |
\\
\sphinxbottomrule
\end{tabular}
\sphinxtableafterendhook\par
\sphinxattableend\end{savenotes}


\begin{savenotes}\sphinxattablestart
\sphinxthistablewithglobalstyle
\centering
\begin{tabular}[t]{\X{12}{96}\X{12}{96}\X{12}{96}\X{12}{96}\X{12}{96}\X{12}{96}\X{12}{96}\X{12}{96}}
\sphinxtoprule
\sphinxtableatstartofbodyhook
\sphinxAtStartPar
31
&
\sphinxAtStartPar
30
&
\sphinxAtStartPar
29
&
\sphinxAtStartPar
28
&
\sphinxAtStartPar
27
&
\sphinxAtStartPar
26
&
\sphinxAtStartPar
25
&
\sphinxAtStartPar
24
\\
\sphinxhline\begin{itemize}
\item {} 
\end{itemize}
&&&&&&&\\
\sphinxhline
\sphinxAtStartPar
23
&
\sphinxAtStartPar
22
&
\sphinxAtStartPar
21
&
\sphinxAtStartPar
20
&
\sphinxAtStartPar
19
&
\sphinxAtStartPar
18
&
\sphinxAtStartPar
17
&
\sphinxAtStartPar
16
\\
\sphinxhline\begin{itemize}
\item {} 
\end{itemize}
&&&&&&&\\
\sphinxhline
\sphinxAtStartPar
15
&
\sphinxAtStartPar
14
&
\sphinxAtStartPar
13
&
\sphinxAtStartPar
12
&
\sphinxAtStartPar
11
&
\sphinxAtStartPar
10
&
\sphinxAtStartPar
9
&
\sphinxAtStartPar
8
\\
\sphinxhline\begin{itemize}
\item {} 
\end{itemize}
&&&&&&&\\
\sphinxhline
\sphinxAtStartPar
7
&
\sphinxAtStartPar
6
&
\sphinxAtStartPar
5
&
\sphinxAtStartPar
4
&
\sphinxAtStartPar
3
&
\sphinxAtStartPar
2
&
\sphinxAtStartPar
1
&
\sphinxAtStartPar
0
\\
\sphinxhline\begin{itemize}
\item {} 
\end{itemize}
&&&&&&&
\sphinxAtStartPar
IF
\\
\sphinxbottomrule
\end{tabular}
\sphinxtableafterendhook\par
\sphinxattableend\end{savenotes}


\begin{savenotes}\sphinxattablestart
\sphinxthistablewithglobalstyle
\centering
\begin{tabular}[t]{\X{33}{99}\X{33}{99}\X{33}{99}}
\sphinxtoprule
\sphinxtableatstartofbodyhook
\sphinxAtStartPar
位域 |
&
\sphinxAtStartPar
名称     | |
&
\sphinxAtStartPar
描述                                        | |
\\
\sphinxhline
\sphinxAtStartPar
31:1
&\begin{itemize}
\item {} 
\end{itemize}
&\begin{itemize}
\item {} 
\end{itemize}
\\
\sphinxhline
\sphinxAtStartPar
0
&
\sphinxAtStartPar
IF
&
\sphinxAtStartPar
WDT中断位,高有效,R/W1C                    |

\sphinxAtStartPar
硬件置位,写1清零                           |
\\
\sphinxbottomrule
\end{tabular}
\sphinxtableafterendhook\par
\sphinxattableend\end{savenotes}


\subsubsection{WDT重启寄存器FEED}
\label{\detokenize{SWM241/_u529f_u80fd_u63cf_u8ff0/_u770b_u95e8_u72d7_u5b9a_u65f6_u5668:wdtfeed}}

\begin{savenotes}\sphinxattablestart
\sphinxthistablewithglobalstyle
\centering
\begin{tabular}[t]{\X{20}{100}\X{20}{100}\X{20}{100}\X{20}{100}\X{20}{100}}
\sphinxtoprule
\sphinxtableatstartofbodyhook
\sphinxAtStartPar
寄存器 |
&
\begin{DUlineblock}{0em}
\item[] 偏移 |
\end{DUlineblock}
&
\begin{DUlineblock}{0em}
\item[] 
\item[] {\color{red}\bfseries{}|}
\end{DUlineblock}
&
\sphinxAtStartPar
复位值 |    描 | |
&
\begin{DUlineblock}{0em}
\item[] |
  |
\end{DUlineblock}
\\
\sphinxhline
\sphinxAtStartPar
FEED
&
\sphinxAtStartPar
0x10
&&
\sphinxAtStartPar
0 000000
&
\sphinxAtStartPar
WDT重启计数器寄存器        |
\\
\sphinxbottomrule
\end{tabular}
\sphinxtableafterendhook\par
\sphinxattableend\end{savenotes}


\begin{savenotes}\sphinxattablestart
\sphinxthistablewithglobalstyle
\centering
\begin{tabular}[t]{\X{12}{96}\X{12}{96}\X{12}{96}\X{12}{96}\X{12}{96}\X{12}{96}\X{12}{96}\X{12}{96}}
\sphinxtoprule
\sphinxtableatstartofbodyhook
\sphinxAtStartPar
31
&
\sphinxAtStartPar
30
&
\sphinxAtStartPar
29
&
\sphinxAtStartPar
28
&
\sphinxAtStartPar
27
&
\sphinxAtStartPar
26
&
\sphinxAtStartPar
25
&
\sphinxAtStartPar
24
\\
\sphinxhline\begin{itemize}
\item {} 
\end{itemize}
&&&&&&&\\
\sphinxhline
\sphinxAtStartPar
23
&
\sphinxAtStartPar
22
&
\sphinxAtStartPar
21
&
\sphinxAtStartPar
20
&
\sphinxAtStartPar
19
&
\sphinxAtStartPar
18
&
\sphinxAtStartPar
17
&
\sphinxAtStartPar
16
\\
\sphinxhline\begin{itemize}
\item {} 
\end{itemize}
&&&&&&&\\
\sphinxhline
\sphinxAtStartPar
15
&
\sphinxAtStartPar
14
&
\sphinxAtStartPar
13
&
\sphinxAtStartPar
12
&
\sphinxAtStartPar
11
&
\sphinxAtStartPar
10
&
\sphinxAtStartPar
9
&
\sphinxAtStartPar
8
\\
\sphinxhline\begin{itemize}
\item {} 
\end{itemize}
&&&&&&&\\
\sphinxhline
\sphinxAtStartPar
7
&
\sphinxAtStartPar
6
&
\sphinxAtStartPar
5
&
\sphinxAtStartPar
4
&
\sphinxAtStartPar
3
&
\sphinxAtStartPar
2
&
\sphinxAtStartPar
1
&
\sphinxAtStartPar
0
\\
\sphinxhline
\sphinxAtStartPar
FEED
&&&&&&&\\
\sphinxbottomrule
\end{tabular}
\sphinxtableafterendhook\par
\sphinxattableend\end{savenotes}


\begin{savenotes}\sphinxattablestart
\sphinxthistablewithglobalstyle
\centering
\begin{tabular}[t]{\X{33}{99}\X{33}{99}\X{33}{99}}
\sphinxtoprule
\sphinxtableatstartofbodyhook
\sphinxAtStartPar
位域 |
&
\sphinxAtStartPar
名称     | |
&
\sphinxAtStartPar
描述                                        | |
\\
\sphinxhline
\sphinxAtStartPar
31:8
&\begin{itemize}
\item {} 
\end{itemize}
&\begin{itemize}
\item {} 
\end{itemize}
\\
\sphinxhline
\sphinxAtStartPar
7:0
&
\sphinxAtStartPar
FEED
&
\sphinxAtStartPar
看门狗重启计数器寄存器                      |

\sphinxAtStartPar
当向该                                      | 写入0x55后会重启看门狗计数器(喂狗操作) |
\\
\sphinxbottomrule
\end{tabular}
\sphinxtableafterendhook\par
\sphinxattableend\end{savenotes}

\sphinxstepscope


\section{实时时钟(RTC)}
\label{\detokenize{SWM241/_u529f_u80fd_u63cf_u8ff0/_u5b9e_u65f6_u65f6_u949f:rtc}}\label{\detokenize{SWM241/_u529f_u80fd_u63cf_u8ff0/_u5b9e_u65f6_u65f6_u949f::doc}}\begin{enumerate}
\sphinxsetlistlabels{\arabic}{enumi}{enumii}{}{.}%
\item {} 
\sphinxAtStartPar
\sphinxstylestrong{概述}

\end{enumerate}

\sphinxAtStartPar
使用前需使能RTC模块时钟。RTC控制器用于提供给用户实时的时间信息与日期信息。
\begin{enumerate}
\sphinxsetlistlabels{\arabic}{enumi}{enumii}{}{.}%
\setcounter{enumi}{1}
\item {} 
\sphinxAtStartPar
\sphinxstylestrong{特性}

\end{enumerate}
\begin{itemize}
\item {} 
\sphinxAtStartPar
可自由设置日期(年、月、周、日)和时间(时、分、秒)

\item {} 
\sphinxAtStartPar
可自由设置闹钟(周、时、分、秒)

\item {} 
\sphinxAtStartPar
自动识别当前设置年份是否为闰年

\item {} 
\sphinxAtStartPar
支持RTC时钟校正功能

\item {} 
\sphinxAtStartPar
支持RTC中断从SLEEP模式下唤醒芯片
\begin{enumerate}
\sphinxsetlistlabels{\arabic}{enumi}{enumii}{}{.}%
\item {} 
\sphinxAtStartPar
\sphinxstylestrong{模块结构框图}

\end{enumerate}

\end{itemize}

\sphinxAtStartPar
\sphinxincludegraphics{{SWM241/功能描述/media实时时钟002}.emf}

\sphinxAtStartPar
图 6‑20 RTC模块结构框图
\begin{enumerate}
\sphinxsetlistlabels{\arabic}{enumi}{enumii}{}{.}%
\setcounter{enumi}{1}
\item {} 
\sphinxAtStartPar
\sphinxstylestrong{功能描述}

\end{enumerate}

\sphinxAtStartPar
使用RTC前,需进行如下操作:
\begin{itemize}
\item {} 
\sphinxAtStartPar
通过寄存器EN禁能RTC;

\item {} 
\sphinxAtStartPar
读取配置状态寄存器CFGABLE,当该寄存器 = 1时,分别通过寄存器MINSEC、DATHUR、MONDAY、YEAR配置RTC的初始计数值 ,通过寄存器MINSECAL、DATHURAL设置RTC闹钟时间;

\item {} 
\sphinxAtStartPar
通过寄存器LOAD加载各项初始值;

\item {} 
\sphinxAtStartPar
用户根据需要,通过寄存器IE使能天/时/分/秒等中断或闹钟中断;

\item {} 
\sphinxAtStartPar
通过寄存器EN使能RTC;

\item {} 
\sphinxAtStartPar
若使能了天/时/分/秒等中断,计数到规定时间后进入中断;若使能了闹钟中断,计数到闹钟设定时间后将芯片从休眠状态唤醒(进入闹钟中断前芯片需处于休眠状态)。

\end{itemize}

\sphinxAtStartPar
\sphinxstylestrong{备份寄存器}

\sphinxAtStartPar
在SYSCON模块中,提供了3个32位RTC电源域备份寄存器,用于存储数据,RTC电源域备份寄存器处在备份域,由于此芯片RTC没有单独额外供电,当系统在待机模式下被唤醒,或系统复位时,该寄存器不会被复位;只有在断电复位时,该寄存器才会复位。使用流程如下所示:
\begin{itemize}
\item {} 
\sphinxAtStartPar
确认RCLF(32KHz时钟)为使能状态

\item {} 
\sphinxAtStartPar
配置PxWKEN寄存器指定位,使能相应端口对应位唤醒功能

\item {} 
\sphinxAtStartPar
SLEEP寄存器BIT{[}0{]} = 1后,芯片进入睡眠模式

\item {} 
\sphinxAtStartPar
当配置端口对应位产生下降沿时,芯片被唤醒,继续执行程序

\item {} 
\sphinxAtStartPar
唤醒后,端口对应PxWKSR寄存器对应位被至1,可通过对该位写1进行清除(该位对进入休眠无影响)

\end{itemize}

\sphinxAtStartPar
\sphinxstyleemphasis{注:内部的NVIC\_RESET对RTC域没有影响}

\sphinxAtStartPar
\sphinxstylestrong{RTC唤醒}

\sphinxAtStartPar
浅睡眠模式下,通过SYSCON模块中RTCWKSR寄存器及RTCWKCR寄存器进行定时器唤醒操作。

\sphinxAtStartPar
在sleep之前,需要将时钟切换为内部高频。

\sphinxAtStartPar
流程如下:

\sphinxAtStartPar
关闭所有不需要唤醒功能的IO输入使能(PORTCON模块中INEN\_x寄存器)

\sphinxAtStartPar
配置RTC时钟源及唤醒时间

\sphinxAtStartPar
使能唤醒源,设置RTCWKCR寄存器EN位为1 (使能前需通过写1清除TWGFLG寄存器FLG位)

\sphinxAtStartPar
使能RTC,RTC开始计数

\sphinxAtStartPar
SLEEP寄存器SLEEP位置1后,芯片进入浅睡眠模式,RTC计到设置值后唤醒芯片

\sphinxAtStartPar
唤醒后,RTCWKSR寄存器FLG位为1(可通过对该位写1进行清除)

\sphinxAtStartPar
\sphinxstylestrong{时钟源}

\sphinxAtStartPar
RTC时钟源可选择片外低频晶体振荡器(32.768KHz)或内部低频RC振荡器(32KHz)两个时钟源,可通过SYSCON模块中CLKSET寄存器中RTC(32K时钟选择)设置RTC时钟源。


\subsection{中断配置与清除}
\label{\detokenize{SWM241/_u529f_u80fd_u63cf_u8ff0/_u5b9e_u65f6_u65f6_u949f:id1}}
\sphinxAtStartPar
可通过配置中断使能寄存器IE相应位使能中断。当计数到设定时间时,中断标志寄存器IF对应位置1。如需清除此标志,需在对应标志位中写1清零(R/W1C),否则中断在开启状态下会一直进入。
\begin{enumerate}
\sphinxsetlistlabels{\arabic}{enumi}{enumii}{}{.}%
\item {} 
\sphinxAtStartPar
\sphinxstylestrong{寄存器映射}

\end{enumerate}


\begin{savenotes}\sphinxattablestart
\sphinxthistablewithglobalstyle
\centering
\begin{tabular}[t]{\X{20}{100}\X{20}{100}\X{20}{100}\X{20}{100}\X{20}{100}}
\sphinxtoprule
\sphinxtableatstartofbodyhook
\sphinxAtStartPar
名称   |
&
\begin{DUlineblock}{0em}
\item[] 偏移 |
\end{DUlineblock}
&
\begin{DUlineblock}{0em}
\item[] 
\item[] |
|
\end{DUlineblock}
&
\begin{DUlineblock}{0em}
\item[] 
\end{DUlineblock}
\begin{quote}

\begin{DUlineblock}{0em}
\item[] 
\item[] 
\end{DUlineblock}
\end{quote}
&
\sphinxAtStartPar
描述                       | | | |
\\
\sphinxhline
\sphinxAtStartPar
RTCBASE:0 {\color{red}\bfseries{}|}x4004B800
&
\begin{DUlineblock}{0em}
\item[] 
\end{DUlineblock}
&&&\\
\sphinxhline
\sphinxAtStartPar
MINSEC
&
\sphinxAtStartPar
0x00
&&
\sphinxAtStartPar
0x 00000
&
\sphinxAtStartPar
分秒计数寄存器             |
\\
\sphinxhline
\sphinxAtStartPar
DATHUR
&
\sphinxAtStartPar
0x04
&&
\sphinxAtStartPar
0x 00000
&
\sphinxAtStartPar
日时计数寄存器             |
\\
\sphinxhline
\sphinxAtStartPar
MONDAY
&
\sphinxAtStartPar
0x08
&&
\sphinxAtStartPar
0x 00000
&
\sphinxAtStartPar
月周计数寄存器             |
\\
\sphinxhline
\sphinxAtStartPar
YEAR
&
\sphinxAtStartPar
0x0C
&&
\sphinxAtStartPar
0x 00000
&
\sphinxAtStartPar
年计数寄存器               |
\\
\sphinxhline
\sphinxAtStartPar
MINSECAL
&
\sphinxAtStartPar
0x10
&&
\sphinxAtStartPar
0x 00000
&
\sphinxAtStartPar
分秒闹铃设置寄存器         |
\\
\sphinxhline
\sphinxAtStartPar
DAYHURAL
&
\sphinxAtStartPar
0x14
&&
\sphinxAtStartPar
0x 00000
&
\sphinxAtStartPar
周时闹铃设置寄存器         |
\\
\sphinxhline
\sphinxAtStartPar
LOAD
&
\sphinxAtStartPar
0x18
&&
\sphinxAtStartPar
0x 00000
&
\sphinxAtStartPar
初始化计数器               |
\\
\sphinxhline
\sphinxAtStartPar
IE
&
\sphinxAtStartPar
0x1C
&&
\sphinxAtStartPar
0x 00000
&
\sphinxAtStartPar
中断使能寄存器             |
\\
\sphinxhline
\sphinxAtStartPar
IF
&
\sphinxAtStartPar
0x20
&&
\sphinxAtStartPar
0x 00000
&
\sphinxAtStartPar
中断标志寄存器             |
\\
\sphinxhline
\sphinxAtStartPar
EN
&
\sphinxAtStartPar
0x24
&&
\sphinxAtStartPar
0x 00000
&
\sphinxAtStartPar
RTC使能寄存器              |
\\
\sphinxhline
\sphinxAtStartPar
CFGABLE
&
\sphinxAtStartPar
0x28
&&
\sphinxAtStartPar
0x 00000
&
\sphinxAtStartPar
配置状态寄存器             |
\\
\sphinxhline
\sphinxAtStartPar
TRIM
&
\sphinxAtStartPar
0x2C
&&
\sphinxAtStartPar
0x 00000
&
\sphinxAtStartPar
时钟调整寄存器             |
\\
\sphinxhline
\sphinxAtStartPar
TRIMM
&
\sphinxAtStartPar
0x30
&&
\sphinxAtStartPar
0x 00000
&
\sphinxAtStartPar
时钟微调整寄存器           |
\\
\sphinxhline
\sphinxAtStartPar
CALIBREFCNT
&
\sphinxAtStartPar
0X60
&&
\sphinxAtStartPar
0x 00000
&
\sphinxAtStartPar
Refclk时钟下cnt\_ref\_target |
\\
\sphinxhline
\sphinxAtStartPar
CALIBEN
&
\sphinxAtStartPar
0X64
&&
\sphinxAtStartPar
0x 00000
&
\sphinxAtStartPar
使能rtc校正                |
\\
\sphinxhline
\sphinxAtStartPar
CALIBSR
&
\sphinxAtStartPar
0x68
&&
\sphinxAtStartPar
0x 00000
&
\sphinxAtStartPar
校正状态寄存器             |
\\
\sphinxbottomrule
\end{tabular}
\sphinxtableafterendhook\par
\sphinxattableend\end{savenotes}
\begin{enumerate}
\sphinxsetlistlabels{\arabic}{enumi}{enumii}{}{.}%
\setcounter{enumi}{1}
\item {} 
\sphinxAtStartPar
\sphinxstylestrong{寄存器描述}

\end{enumerate}

\sphinxAtStartPar
\sphinxstylestrong{分秒计数寄存器MINSEC}


\begin{savenotes}\sphinxattablestart
\sphinxthistablewithglobalstyle
\centering
\begin{tabular}[t]{\X{20}{100}\X{20}{100}\X{20}{100}\X{20}{100}\X{20}{100}}
\sphinxtoprule
\sphinxtableatstartofbodyhook
\sphinxAtStartPar
寄存器 |
&
\begin{DUlineblock}{0em}
\item[] 偏移 |
\end{DUlineblock}
&
\begin{DUlineblock}{0em}
\item[] 
\item[] {\color{red}\bfseries{}|}
\end{DUlineblock}
&
\sphinxAtStartPar
复位值 |    描 | |
&
\begin{DUlineblock}{0em}
\item[] |
  |
\end{DUlineblock}
\\
\sphinxhline
\sphinxAtStartPar
MINSEC
&
\sphinxAtStartPar
0x00
&&
\sphinxAtStartPar
0 000000
&
\sphinxAtStartPar
分秒计数寄存器             |
\\
\sphinxbottomrule
\end{tabular}
\sphinxtableafterendhook\par
\sphinxattableend\end{savenotes}


\begin{savenotes}\sphinxattablestart
\sphinxthistablewithglobalstyle
\centering
\begin{tabular}[t]{\X{12}{96}\X{12}{96}\X{12}{96}\X{12}{96}\X{12}{96}\X{12}{96}\X{12}{96}\X{12}{96}}
\sphinxtoprule
\sphinxtableatstartofbodyhook
\sphinxAtStartPar
31
&
\sphinxAtStartPar
30
&
\sphinxAtStartPar
29
&
\sphinxAtStartPar
28
&
\sphinxAtStartPar
27
&
\sphinxAtStartPar
26
&
\sphinxAtStartPar
25
&
\sphinxAtStartPar
24
\\
\sphinxhline\begin{itemize}
\item {} 
\end{itemize}
&&&&&&&\\
\sphinxhline
\sphinxAtStartPar
23
&
\sphinxAtStartPar
22
&
\sphinxAtStartPar
21
&
\sphinxAtStartPar
20
&
\sphinxAtStartPar
19
&
\sphinxAtStartPar
18
&
\sphinxAtStartPar
17
&
\sphinxAtStartPar
16
\\
\sphinxhline\begin{itemize}
\item {} 
\end{itemize}
&&&&&&&\\
\sphinxhline
\sphinxAtStartPar
15
&
\sphinxAtStartPar
14
&
\sphinxAtStartPar
13
&
\sphinxAtStartPar
12
&
\sphinxAtStartPar
11
&
\sphinxAtStartPar
10
&
\sphinxAtStartPar
9
&
\sphinxAtStartPar
8
\\
\sphinxhline\begin{itemize}
\item {} 
\end{itemize}
&&&&&&&\\
\sphinxhline
\sphinxAtStartPar
7
&
\sphinxAtStartPar
6
&
\sphinxAtStartPar
5
&
\sphinxAtStartPar
4
&
\sphinxAtStartPar
3
&
\sphinxAtStartPar
2
&
\sphinxAtStartPar
1
&
\sphinxAtStartPar
0
\\
\sphinxhline
\sphinxAtStartPar
MIN
&&&&&&&\\
\sphinxbottomrule
\end{tabular}
\sphinxtableafterendhook\par
\sphinxattableend\end{savenotes}


\begin{savenotes}\sphinxattablestart
\sphinxthistablewithglobalstyle
\centering
\begin{tabular}[t]{\X{33}{99}\X{33}{99}\X{33}{99}}
\sphinxtoprule
\sphinxtableatstartofbodyhook
\sphinxAtStartPar
位域 |
&
\sphinxAtStartPar
名称     | |
&
\sphinxAtStartPar
描述                                        | |
\\
\sphinxhline
\sphinxAtStartPar
31:12
&\begin{itemize}
\item {} 
\end{itemize}
&\begin{itemize}
\item {} 
\end{itemize}
\\
\sphinxhline
\sphinxAtStartPar
11:6
&
\sphinxAtStartPar
MIN
&
\sphinxAtStartPar
计时器分钟计数                              |
\\
\sphinxhline
\sphinxAtStartPar
5:0
&
\sphinxAtStartPar
SEC
&
\sphinxAtStartPar
计时器秒计数                                |
\\
\sphinxbottomrule
\end{tabular}
\sphinxtableafterendhook\par
\sphinxattableend\end{savenotes}

\sphinxAtStartPar
\sphinxstylestrong{日时计数寄存器DATHUR}


\begin{savenotes}\sphinxattablestart
\sphinxthistablewithglobalstyle
\centering
\begin{tabular}[t]{\X{20}{100}\X{20}{100}\X{20}{100}\X{20}{100}\X{20}{100}}
\sphinxtoprule
\sphinxtableatstartofbodyhook
\sphinxAtStartPar
寄存器 |
&
\begin{DUlineblock}{0em}
\item[] 偏移 |
\end{DUlineblock}
&
\begin{DUlineblock}{0em}
\item[] 
\item[] {\color{red}\bfseries{}|}
\end{DUlineblock}
&
\sphinxAtStartPar
复位值 |    描 | |
&
\begin{DUlineblock}{0em}
\item[] |
  |
\end{DUlineblock}
\\
\sphinxhline
\sphinxAtStartPar
DATHUR
&
\sphinxAtStartPar
0x04
&&
\sphinxAtStartPar
0 000000
&
\sphinxAtStartPar
日时计数寄存器             |
\\
\sphinxbottomrule
\end{tabular}
\sphinxtableafterendhook\par
\sphinxattableend\end{savenotes}


\begin{savenotes}\sphinxattablestart
\sphinxthistablewithglobalstyle
\centering
\begin{tabular}[t]{\X{12}{96}\X{12}{96}\X{12}{96}\X{12}{96}\X{12}{96}\X{12}{96}\X{12}{96}\X{12}{96}}
\sphinxtoprule
\sphinxtableatstartofbodyhook
\sphinxAtStartPar
31
&
\sphinxAtStartPar
30
&
\sphinxAtStartPar
29
&
\sphinxAtStartPar
28
&
\sphinxAtStartPar
27
&
\sphinxAtStartPar
26
&
\sphinxAtStartPar
25
&
\sphinxAtStartPar
24
\\
\sphinxhline\begin{itemize}
\item {} 
\end{itemize}
&&&&&&&\\
\sphinxhline
\sphinxAtStartPar
23
&
\sphinxAtStartPar
22
&
\sphinxAtStartPar
21
&
\sphinxAtStartPar
20
&
\sphinxAtStartPar
19
&
\sphinxAtStartPar
18
&
\sphinxAtStartPar
17
&
\sphinxAtStartPar
16
\\
\sphinxhline\begin{itemize}
\item {} 
\end{itemize}
&&&&&&&\\
\sphinxhline
\sphinxAtStartPar
15
&
\sphinxAtStartPar
14
&
\sphinxAtStartPar
13
&
\sphinxAtStartPar
12
&
\sphinxAtStartPar
11
&
\sphinxAtStartPar
10
&
\sphinxAtStartPar
9
&
\sphinxAtStartPar
8
\\
\sphinxhline\begin{itemize}
\item {} 
\end{itemize}
&&&&&&&\\
\sphinxhline
\sphinxAtStartPar
7
&
\sphinxAtStartPar
6
&
\sphinxAtStartPar
5
&
\sphinxAtStartPar
4
&
\sphinxAtStartPar
3
&
\sphinxAtStartPar
2
&
\sphinxAtStartPar
1
&
\sphinxAtStartPar
0
\\
\sphinxhline
\sphinxAtStartPar
DATE
&&&&&&&\\
\sphinxbottomrule
\end{tabular}
\sphinxtableafterendhook\par
\sphinxattableend\end{savenotes}


\begin{savenotes}\sphinxattablestart
\sphinxthistablewithglobalstyle
\centering
\begin{tabular}[t]{\X{33}{99}\X{33}{99}\X{33}{99}}
\sphinxtoprule
\sphinxtableatstartofbodyhook
\sphinxAtStartPar
位域 |
&
\sphinxAtStartPar
名称     | |
&
\sphinxAtStartPar
描述                                        | |
\\
\sphinxhline
\sphinxAtStartPar
31:10
&\begin{itemize}
\item {} 
\end{itemize}
&\begin{itemize}
\item {} 
\end{itemize}
\\
\sphinxhline
\sphinxAtStartPar
9:5
&
\sphinxAtStartPar
DATE
&
\sphinxAtStartPar
计时器天计数                                |
\\
\sphinxhline
\sphinxAtStartPar
4:0
&
\sphinxAtStartPar
HOUR
&
\sphinxAtStartPar
计时器小时计数                              |
\\
\sphinxbottomrule
\end{tabular}
\sphinxtableafterendhook\par
\sphinxattableend\end{savenotes}

\sphinxAtStartPar
\sphinxstylestrong{月周计数寄存器MONDAY}


\begin{savenotes}\sphinxattablestart
\sphinxthistablewithglobalstyle
\centering
\begin{tabular}[t]{\X{20}{100}\X{20}{100}\X{20}{100}\X{20}{100}\X{20}{100}}
\sphinxtoprule
\sphinxtableatstartofbodyhook
\sphinxAtStartPar
寄存器 |
&
\begin{DUlineblock}{0em}
\item[] 偏移 |
\end{DUlineblock}
&
\begin{DUlineblock}{0em}
\item[] 
\item[] {\color{red}\bfseries{}|}
\end{DUlineblock}
&
\sphinxAtStartPar
复位值 |    描 | |
&
\begin{DUlineblock}{0em}
\item[] |
  |
\end{DUlineblock}
\\
\sphinxhline
\sphinxAtStartPar
MONDAY
&
\sphinxAtStartPar
0x08
&&
\sphinxAtStartPar
0 000000
&
\sphinxAtStartPar
月周计数寄存器             |
\\
\sphinxbottomrule
\end{tabular}
\sphinxtableafterendhook\par
\sphinxattableend\end{savenotes}


\begin{savenotes}\sphinxattablestart
\sphinxthistablewithglobalstyle
\centering
\begin{tabular}[t]{\X{12}{96}\X{12}{96}\X{12}{96}\X{12}{96}\X{12}{96}\X{12}{96}\X{12}{96}\X{12}{96}}
\sphinxtoprule
\sphinxtableatstartofbodyhook
\sphinxAtStartPar
31
&
\sphinxAtStartPar
30
&
\sphinxAtStartPar
29
&
\sphinxAtStartPar
28
&
\sphinxAtStartPar
27
&
\sphinxAtStartPar
26
&
\sphinxAtStartPar
25
&
\sphinxAtStartPar
24
\\
\sphinxhline\begin{itemize}
\item {} 
\end{itemize}
&&&&&&&\\
\sphinxhline
\sphinxAtStartPar
23
&
\sphinxAtStartPar
22
&
\sphinxAtStartPar
21
&
\sphinxAtStartPar
20
&
\sphinxAtStartPar
19
&
\sphinxAtStartPar
18
&
\sphinxAtStartPar
17
&
\sphinxAtStartPar
16
\\
\sphinxhline\begin{itemize}
\item {} 
\end{itemize}
&&&&&&&\\
\sphinxhline
\sphinxAtStartPar
15
&
\sphinxAtStartPar
14
&
\sphinxAtStartPar
13
&
\sphinxAtStartPar
12
&
\sphinxAtStartPar
11
&
\sphinxAtStartPar
10
&
\sphinxAtStartPar
9
&
\sphinxAtStartPar
8
\\
\sphinxhline\begin{itemize}
\item {} 
\end{itemize}
&&&&&&&\\
\sphinxhline
\sphinxAtStartPar
7
&
\sphinxAtStartPar
6
&
\sphinxAtStartPar
5
&
\sphinxAtStartPar
4
&
\sphinxAtStartPar
3
&
\sphinxAtStartPar
2
&
\sphinxAtStartPar
1
&
\sphinxAtStartPar
0
\\
\sphinxhline\begin{itemize}
\item {} 
\end{itemize}
&
\sphinxAtStartPar
MON
&&&&
\sphinxAtStartPar
DAY
&&\\
\sphinxbottomrule
\end{tabular}
\sphinxtableafterendhook\par
\sphinxattableend\end{savenotes}


\begin{savenotes}\sphinxattablestart
\sphinxthistablewithglobalstyle
\centering
\begin{tabular}[t]{\X{33}{99}\X{33}{99}\X{33}{99}}
\sphinxtoprule
\sphinxtableatstartofbodyhook
\sphinxAtStartPar
位域 |
&
\sphinxAtStartPar
名称     | |
&
\sphinxAtStartPar
描述                                        | |
\\
\sphinxhline
\sphinxAtStartPar
31:7
&\begin{itemize}
\item {} 
\end{itemize}
&\begin{itemize}
\item {} 
\end{itemize}
\\
\sphinxhline
\sphinxAtStartPar
6:3
&
\sphinxAtStartPar
MON
&
\sphinxAtStartPar
计时器月计数                                |

\sphinxAtStartPar
0000:保留                                  |

\sphinxAtStartPar
0001:1月                                   |

\sphinxAtStartPar
0010:2月                                   |

\sphinxAtStartPar
0011:3月                                   |

\sphinxAtStartPar
0100:4月                                   |

\sphinxAtStartPar
0101:5月                                   |

\sphinxAtStartPar
0110:6月                                   |

\sphinxAtStartPar
0111:7月                                   |

\sphinxAtStartPar
1000:8月                                   |

\sphinxAtStartPar
1001:9月                                   |

\sphinxAtStartPar
1010:10月                                  |

\sphinxAtStartPar
1011:11月                                  |

\sphinxAtStartPar
1100:12月                                  |

\sphinxAtStartPar
1101:保留                                  |

\sphinxAtStartPar
1110:保留                                  |

\sphinxAtStartPar
1111:保留                                  |
\\
\sphinxhline
\sphinxAtStartPar
2:0
&
\sphinxAtStartPar
DAY
&
\sphinxAtStartPar
计时器周计数                                |

\sphinxAtStartPar
000:表示周日                               |

\sphinxAtStartPar
001:表示周一                               |

\sphinxAtStartPar
010:表示周二                               |

\sphinxAtStartPar
011:表示周三                               |

\sphinxAtStartPar
100:表示周四                               |

\sphinxAtStartPar
101:表示周五                               |

\sphinxAtStartPar
110:表示周六                               |
\\
\sphinxbottomrule
\end{tabular}
\sphinxtableafterendhook\par
\sphinxattableend\end{savenotes}

\sphinxAtStartPar
\sphinxstylestrong{年计数寄存器YEAR}


\begin{savenotes}\sphinxattablestart
\sphinxthistablewithglobalstyle
\centering
\begin{tabular}[t]{\X{20}{100}\X{20}{100}\X{20}{100}\X{20}{100}\X{20}{100}}
\sphinxtoprule
\sphinxtableatstartofbodyhook
\sphinxAtStartPar
寄存器 |
&
\begin{DUlineblock}{0em}
\item[] 偏移 |
\end{DUlineblock}
&
\begin{DUlineblock}{0em}
\item[] 
\item[] {\color{red}\bfseries{}|}
\end{DUlineblock}
&
\sphinxAtStartPar
复位值 |    描 | |
&
\begin{DUlineblock}{0em}
\item[] |
  |
\end{DUlineblock}
\\
\sphinxhline
\sphinxAtStartPar
YEAR
&
\sphinxAtStartPar
0x0C
&&
\sphinxAtStartPar
0 000000
&
\sphinxAtStartPar
年计数寄存器               |
\\
\sphinxbottomrule
\end{tabular}
\sphinxtableafterendhook\par
\sphinxattableend\end{savenotes}


\begin{savenotes}\sphinxattablestart
\sphinxthistablewithglobalstyle
\centering
\begin{tabular}[t]{\X{12}{96}\X{12}{96}\X{12}{96}\X{12}{96}\X{12}{96}\X{12}{96}\X{12}{96}\X{12}{96}}
\sphinxtoprule
\sphinxtableatstartofbodyhook
\sphinxAtStartPar
31
&
\sphinxAtStartPar
30
&
\sphinxAtStartPar
29
&
\sphinxAtStartPar
28
&
\sphinxAtStartPar
27
&
\sphinxAtStartPar
26
&
\sphinxAtStartPar
25
&
\sphinxAtStartPar
24
\\
\sphinxhline\begin{itemize}
\item {} 
\end{itemize}
&&&&&&&\\
\sphinxhline
\sphinxAtStartPar
23
&
\sphinxAtStartPar
22
&
\sphinxAtStartPar
21
&
\sphinxAtStartPar
20
&
\sphinxAtStartPar
19
&
\sphinxAtStartPar
18
&
\sphinxAtStartPar
17
&
\sphinxAtStartPar
16
\\
\sphinxhline\begin{itemize}
\item {} 
\end{itemize}
&&&&&&&\\
\sphinxhline
\sphinxAtStartPar
15
&
\sphinxAtStartPar
14
&
\sphinxAtStartPar
13
&
\sphinxAtStartPar
12
&
\sphinxAtStartPar
11
&
\sphinxAtStartPar
10
&
\sphinxAtStartPar
9
&
\sphinxAtStartPar
8
\\
\sphinxhline\begin{itemize}
\item {} 
\end{itemize}
&&&&&&&\\
\sphinxhline
\sphinxAtStartPar
7
&
\sphinxAtStartPar
6
&
\sphinxAtStartPar
5
&
\sphinxAtStartPar
4
&
\sphinxAtStartPar
3
&
\sphinxAtStartPar
2
&
\sphinxAtStartPar
1
&
\sphinxAtStartPar
0
\\
\sphinxhline
\sphinxAtStartPar
YEAR
&&&&&&&\\
\sphinxbottomrule
\end{tabular}
\sphinxtableafterendhook\par
\sphinxattableend\end{savenotes}


\begin{savenotes}\sphinxattablestart
\sphinxthistablewithglobalstyle
\centering
\begin{tabular}[t]{\X{33}{99}\X{33}{99}\X{33}{99}}
\sphinxtoprule
\sphinxtableatstartofbodyhook
\sphinxAtStartPar
位域 |
&
\sphinxAtStartPar
名称     | |
&
\sphinxAtStartPar
描述                                        | |
\\
\sphinxhline
\sphinxAtStartPar
31:12
&\begin{itemize}
\item {} 
\end{itemize}
&\begin{itemize}
\item {} 
\end{itemize}
\\
\sphinxhline
\sphinxAtStartPar
11:0
&
\sphinxAtStartPar
YEAR
&
\sphinxAtStartPar
计时器年计数。支持1901\sphinxhyphen{}2199                 |
\\
\sphinxbottomrule
\end{tabular}
\sphinxtableafterendhook\par
\sphinxattableend\end{savenotes}

\sphinxAtStartPar
\sphinxstylestrong{分秒闹铃设置寄存器MINSECAL}


\begin{savenotes}\sphinxattablestart
\sphinxthistablewithglobalstyle
\centering
\begin{tabular}[t]{\X{20}{100}\X{20}{100}\X{20}{100}\X{20}{100}\X{20}{100}}
\sphinxtoprule
\sphinxtableatstartofbodyhook
\sphinxAtStartPar
寄存器 |
&
\begin{DUlineblock}{0em}
\item[] 偏移 |
\end{DUlineblock}
&
\begin{DUlineblock}{0em}
\item[] 
\item[] {\color{red}\bfseries{}|}
\end{DUlineblock}
&
\sphinxAtStartPar
复位值 |    描 | |
&
\begin{DUlineblock}{0em}
\item[] |
  |
\end{DUlineblock}
\\
\sphinxhline
\sphinxAtStartPar
MINSECAL
&
\sphinxAtStartPar
0x10
&&
\sphinxAtStartPar
0 000000
&
\sphinxAtStartPar
分秒闹铃设置寄存器         |
\\
\sphinxbottomrule
\end{tabular}
\sphinxtableafterendhook\par
\sphinxattableend\end{savenotes}


\begin{savenotes}\sphinxattablestart
\sphinxthistablewithglobalstyle
\centering
\begin{tabular}[t]{\X{12}{96}\X{12}{96}\X{12}{96}\X{12}{96}\X{12}{96}\X{12}{96}\X{12}{96}\X{12}{96}}
\sphinxtoprule
\sphinxtableatstartofbodyhook
\sphinxAtStartPar
31
&
\sphinxAtStartPar
30
&
\sphinxAtStartPar
29
&
\sphinxAtStartPar
28
&
\sphinxAtStartPar
27
&
\sphinxAtStartPar
26
&
\sphinxAtStartPar
25
&
\sphinxAtStartPar
24
\\
\sphinxhline\begin{itemize}
\item {} 
\end{itemize}
&&&&&&&\\
\sphinxhline
\sphinxAtStartPar
23
&
\sphinxAtStartPar
22
&
\sphinxAtStartPar
21
&
\sphinxAtStartPar
20
&
\sphinxAtStartPar
19
&
\sphinxAtStartPar
18
&
\sphinxAtStartPar
17
&
\sphinxAtStartPar
16
\\
\sphinxhline\begin{itemize}
\item {} 
\end{itemize}
&&&&&&&\\
\sphinxhline
\sphinxAtStartPar
15
&
\sphinxAtStartPar
14
&
\sphinxAtStartPar
13
&
\sphinxAtStartPar
12
&
\sphinxAtStartPar
11
&
\sphinxAtStartPar
10
&
\sphinxAtStartPar
9
&
\sphinxAtStartPar
8
\\
\sphinxhline\begin{itemize}
\item {} 
\end{itemize}
&&&&&&&\\
\sphinxhline
\sphinxAtStartPar
7
&
\sphinxAtStartPar
6
&
\sphinxAtStartPar
5
&
\sphinxAtStartPar
4
&
\sphinxAtStartPar
3
&
\sphinxAtStartPar
2
&
\sphinxAtStartPar
1
&
\sphinxAtStartPar
0
\\
\sphinxhline
\sphinxAtStartPar
MIN
&&&&&&&\\
\sphinxbottomrule
\end{tabular}
\sphinxtableafterendhook\par
\sphinxattableend\end{savenotes}


\begin{savenotes}\sphinxattablestart
\sphinxthistablewithglobalstyle
\centering
\begin{tabular}[t]{\X{33}{99}\X{33}{99}\X{33}{99}}
\sphinxtoprule
\sphinxtableatstartofbodyhook
\sphinxAtStartPar
位域 |
&
\sphinxAtStartPar
名称     | |
&
\sphinxAtStartPar
描述                                        | |
\\
\sphinxhline
\sphinxAtStartPar
31:12
&\begin{itemize}
\item {} 
\end{itemize}
&\begin{itemize}
\item {} 
\end{itemize}
\\
\sphinxhline
\sphinxAtStartPar
11:6
&
\sphinxAtStartPar
MIN
&
\sphinxAtStartPar
定时器分钟设置                              |
\\
\sphinxhline
\sphinxAtStartPar
5:0
&
\sphinxAtStartPar
SEC
&
\sphinxAtStartPar
定时器秒设置                                |
\\
\sphinxbottomrule
\end{tabular}
\sphinxtableafterendhook\par
\sphinxattableend\end{savenotes}

\sphinxAtStartPar
\sphinxstylestrong{周时闹铃设置寄存器DATHURAL}


\begin{savenotes}\sphinxattablestart
\sphinxthistablewithglobalstyle
\centering
\begin{tabular}[t]{\X{20}{100}\X{20}{100}\X{20}{100}\X{20}{100}\X{20}{100}}
\sphinxtoprule
\sphinxtableatstartofbodyhook
\sphinxAtStartPar
寄存器 |
&
\begin{DUlineblock}{0em}
\item[] 偏移 |
\end{DUlineblock}
&
\begin{DUlineblock}{0em}
\item[] 
\item[] {\color{red}\bfseries{}|}
\end{DUlineblock}
&
\sphinxAtStartPar
复位值 |    描 | |
&
\begin{DUlineblock}{0em}
\item[] |
  |
\end{DUlineblock}
\\
\sphinxhline
\sphinxAtStartPar
DAYHURAL
&
\sphinxAtStartPar
0x14
&&
\sphinxAtStartPar
0 000000
&
\sphinxAtStartPar
周时闹铃设置寄存器         |
\\
\sphinxbottomrule
\end{tabular}
\sphinxtableafterendhook\par
\sphinxattableend\end{savenotes}


\begin{savenotes}\sphinxattablestart
\sphinxthistablewithglobalstyle
\centering
\begin{tabular}[t]{\X{12}{96}\X{12}{96}\X{12}{96}\X{12}{96}\X{12}{96}\X{12}{96}\X{12}{96}\X{12}{96}}
\sphinxtoprule
\sphinxtableatstartofbodyhook
\sphinxAtStartPar
31
&
\sphinxAtStartPar
30
&
\sphinxAtStartPar
29
&
\sphinxAtStartPar
28
&
\sphinxAtStartPar
27
&
\sphinxAtStartPar
26
&
\sphinxAtStartPar
25
&
\sphinxAtStartPar
24
\\
\sphinxhline\begin{itemize}
\item {} 
\end{itemize}
&&&&&&&\\
\sphinxhline
\sphinxAtStartPar
23
&
\sphinxAtStartPar
22
&
\sphinxAtStartPar
21
&
\sphinxAtStartPar
20
&
\sphinxAtStartPar
19
&
\sphinxAtStartPar
18
&
\sphinxAtStartPar
17
&
\sphinxAtStartPar
16
\\
\sphinxhline\begin{itemize}
\item {} 
\end{itemize}
&&&&&&&\\
\sphinxhline
\sphinxAtStartPar
15
&
\sphinxAtStartPar
14
&
\sphinxAtStartPar
13
&
\sphinxAtStartPar
12
&
\sphinxAtStartPar
11
&
\sphinxAtStartPar
10
&
\sphinxAtStartPar
9
&
\sphinxAtStartPar
8
\\
\sphinxhline\begin{itemize}
\item {} 
\end{itemize}
&&&&&
\sphinxAtStartPar
FRI
&&
\sphinxAtStartPar
WED
\\
\sphinxhline
\sphinxAtStartPar
7
&
\sphinxAtStartPar
6
&
\sphinxAtStartPar
5
&
\sphinxAtStartPar
4
&
\sphinxAtStartPar
3
&
\sphinxAtStartPar
2
&
\sphinxAtStartPar
1
&
\sphinxAtStartPar
0
\\
\sphinxhline
\sphinxAtStartPar
TUE
&
\sphinxAtStartPar
MON
&&&&&&\\
\sphinxbottomrule
\end{tabular}
\sphinxtableafterendhook\par
\sphinxattableend\end{savenotes}


\begin{savenotes}\sphinxattablestart
\sphinxthistablewithglobalstyle
\centering
\begin{tabular}[t]{\X{33}{99}\X{33}{99}\X{33}{99}}
\sphinxtoprule
\sphinxtableatstartofbodyhook
\sphinxAtStartPar
位域 |
&
\sphinxAtStartPar
名称     | |
&
\sphinxAtStartPar
描述                                        | |
\\
\sphinxhline
\sphinxAtStartPar
31:12
&\begin{itemize}
\item {} 
\end{itemize}
&\begin{itemize}
\item {} 
\end{itemize}
\\
\sphinxhline
\sphinxAtStartPar
11
&
\sphinxAtStartPar
SAT
&
\sphinxAtStartPar
定时器周设置,设置为周六                     |
\\
\sphinxhline
\sphinxAtStartPar
10
&
\sphinxAtStartPar
FRI
&
\sphinxAtStartPar
定时器周设置,设置为周五                     |
\\
\sphinxhline
\sphinxAtStartPar
9
&
\sphinxAtStartPar
THU
&
\sphinxAtStartPar
定时器周设置,设置为周四                     |
\\
\sphinxhline
\sphinxAtStartPar
8
&
\sphinxAtStartPar
WED
&
\sphinxAtStartPar
定时器周设置,设置为周三                     |
\\
\sphinxhline
\sphinxAtStartPar
7
&
\sphinxAtStartPar
TUE
&
\sphinxAtStartPar
定时器周设置,设置为周二                     |
\\
\sphinxhline
\sphinxAtStartPar
6
&
\sphinxAtStartPar
MON
&
\sphinxAtStartPar
定时器周设置,设置为周一                     |
\\
\sphinxhline
\sphinxAtStartPar
5
&
\sphinxAtStartPar
SUN
&
\sphinxAtStartPar
定时器周设置,设置为周日                     |
\\
\sphinxhline
\sphinxAtStartPar
4:0
&
\sphinxAtStartPar
HOUR
&
\sphinxAtStartPar
定时器小时设置                              |
\\
\sphinxbottomrule
\end{tabular}
\sphinxtableafterendhook\par
\sphinxattableend\end{savenotes}

\sphinxAtStartPar
\sphinxstylestrong{初始化寄存器LOAD}


\begin{savenotes}\sphinxattablestart
\sphinxthistablewithglobalstyle
\centering
\begin{tabular}[t]{\X{20}{100}\X{20}{100}\X{20}{100}\X{20}{100}\X{20}{100}}
\sphinxtoprule
\sphinxtableatstartofbodyhook
\sphinxAtStartPar
寄存器 |
&
\begin{DUlineblock}{0em}
\item[] 偏移 |
\end{DUlineblock}
&
\begin{DUlineblock}{0em}
\item[] 
\item[] {\color{red}\bfseries{}|}
\end{DUlineblock}
&
\sphinxAtStartPar
复位值 |    描 | |
&
\begin{DUlineblock}{0em}
\item[] |
  |
\end{DUlineblock}
\\
\sphinxhline
\sphinxAtStartPar
LOAD
&
\sphinxAtStartPar
0x18
&&
\sphinxAtStartPar
0 000000
&
\sphinxAtStartPar
初始化计数器               |
\\
\sphinxbottomrule
\end{tabular}
\sphinxtableafterendhook\par
\sphinxattableend\end{savenotes}


\begin{savenotes}\sphinxattablestart
\sphinxthistablewithglobalstyle
\centering
\begin{tabular}[t]{\X{12}{96}\X{12}{96}\X{12}{96}\X{12}{96}\X{12}{96}\X{12}{96}\X{12}{96}\X{12}{96}}
\sphinxtoprule
\sphinxtableatstartofbodyhook
\sphinxAtStartPar
31
&
\sphinxAtStartPar
30
&
\sphinxAtStartPar
29
&
\sphinxAtStartPar
28
&
\sphinxAtStartPar
27
&
\sphinxAtStartPar
26
&
\sphinxAtStartPar
25
&
\sphinxAtStartPar
24
\\
\sphinxhline\begin{itemize}
\item {} 
\end{itemize}
&&&&&&&\\
\sphinxhline
\sphinxAtStartPar
23
&
\sphinxAtStartPar
22
&
\sphinxAtStartPar
21
&
\sphinxAtStartPar
20
&
\sphinxAtStartPar
19
&
\sphinxAtStartPar
18
&
\sphinxAtStartPar
17
&
\sphinxAtStartPar
16
\\
\sphinxhline\begin{itemize}
\item {} 
\end{itemize}
&&&&&&&\\
\sphinxhline
\sphinxAtStartPar
15
&
\sphinxAtStartPar
14
&
\sphinxAtStartPar
13
&
\sphinxAtStartPar
12
&
\sphinxAtStartPar
11
&
\sphinxAtStartPar
10
&
\sphinxAtStartPar
9
&
\sphinxAtStartPar
8
\\
\sphinxhline\begin{itemize}
\item {} 
\end{itemize}
&&&&&&&\\
\sphinxhline
\sphinxAtStartPar
7
&
\sphinxAtStartPar
6
&
\sphinxAtStartPar
5
&
\sphinxAtStartPar
4
&
\sphinxAtStartPar
3
&
\sphinxAtStartPar
2
&
\sphinxAtStartPar
1
&
\sphinxAtStartPar
0
\\
\sphinxhline\begin{itemize}
\item {} 
\end{itemize}
&&&&&&&\\
\sphinxbottomrule
\end{tabular}
\sphinxtableafterendhook\par
\sphinxattableend\end{savenotes}


\begin{savenotes}\sphinxattablestart
\sphinxthistablewithglobalstyle
\centering
\begin{tabular}[t]{\X{33}{99}\X{33}{99}\X{33}{99}}
\sphinxtoprule
\sphinxtableatstartofbodyhook
\sphinxAtStartPar
位域 |
&
\sphinxAtStartPar
名称     | |
&
\sphinxAtStartPar
描述                                        | |
\\
\sphinxhline
\sphinxAtStartPar
31:1
&\begin{itemize}
\item {} 
\end{itemize}
&\begin{itemize}
\item {} 
\end{itemize}
\\
\sphinxhline
\sphinxAtStartPar
1
&
\sphinxAtStartPar
ALARM
&
\sphinxAtStartPar
将MISEAL和WEHOAL寄存器装载到alarm同步       | ,持续到rtcclk的上升沿来临,自动清零,AC |
\\
\sphinxhline
\sphinxAtStartPar
0
&
\sphinxAtStartPar
TIME
&
\sphinxAtStartPar
将MINSEC、DATHUR、MONDAY、YEAR              | 载到相关cnt计数器,将TRM和TRMM的值装载到 | ecnt中,持续到rtcclk的上升沿来临,自动清零  |
\\
\sphinxbottomrule
\end{tabular}
\sphinxtableafterendhook\par
\sphinxattableend\end{savenotes}

\sphinxAtStartPar
\sphinxstyleemphasis{注}:

\sphinxAtStartPar
需要在MINSEC、DATHUR、MONDAY、YEAR、TRIM和TRIMM配置完成后,再配置TIME信号;

\sphinxAtStartPar
需要在MINSECAL和DATHURAL配置完成后,再配置ALARM信号。

\sphinxAtStartPar
若配置TIME之后,需要关闭pclk,则只需要等待TIME=0之后,再关闭pclk。

\sphinxAtStartPar
\sphinxstylestrong{中断使能寄存器IE}


\begin{savenotes}\sphinxattablestart
\sphinxthistablewithglobalstyle
\centering
\begin{tabular}[t]{\X{20}{100}\X{20}{100}\X{20}{100}\X{20}{100}\X{20}{100}}
\sphinxtoprule
\sphinxtableatstartofbodyhook
\sphinxAtStartPar
寄存器 |
&
\begin{DUlineblock}{0em}
\item[] 偏移 |
\end{DUlineblock}
&
\begin{DUlineblock}{0em}
\item[] 
\item[] {\color{red}\bfseries{}|}
\end{DUlineblock}
&
\sphinxAtStartPar
复位值 |    描 | |
&
\begin{DUlineblock}{0em}
\item[] |
  |
\end{DUlineblock}
\\
\sphinxhline
\sphinxAtStartPar
IE
&
\sphinxAtStartPar
0x1C
&&
\sphinxAtStartPar
0 000000
&
\sphinxAtStartPar
中断使能寄存器             |
\\
\sphinxbottomrule
\end{tabular}
\sphinxtableafterendhook\par
\sphinxattableend\end{savenotes}


\begin{savenotes}\sphinxattablestart
\sphinxthistablewithglobalstyle
\centering
\begin{tabular}[t]{\X{12}{96}\X{12}{96}\X{12}{96}\X{12}{96}\X{12}{96}\X{12}{96}\X{12}{96}\X{12}{96}}
\sphinxtoprule
\sphinxtableatstartofbodyhook
\sphinxAtStartPar
31
&
\sphinxAtStartPar
30
&
\sphinxAtStartPar
29
&
\sphinxAtStartPar
28
&
\sphinxAtStartPar
27
&
\sphinxAtStartPar
26
&
\sphinxAtStartPar
25
&
\sphinxAtStartPar
24
\\
\sphinxhline\begin{itemize}
\item {} 
\end{itemize}
&&&&&&&\\
\sphinxhline
\sphinxAtStartPar
23
&
\sphinxAtStartPar
22
&
\sphinxAtStartPar
21
&
\sphinxAtStartPar
20
&
\sphinxAtStartPar
19
&
\sphinxAtStartPar
18
&
\sphinxAtStartPar
17
&
\sphinxAtStartPar
16
\\
\sphinxhline\begin{itemize}
\item {} 
\end{itemize}
&&&&&&&\\
\sphinxhline
\sphinxAtStartPar
15
&
\sphinxAtStartPar
14
&
\sphinxAtStartPar
13
&
\sphinxAtStartPar
12
&
\sphinxAtStartPar
11
&
\sphinxAtStartPar
10
&
\sphinxAtStartPar
9
&
\sphinxAtStartPar
8
\\
\sphinxhline\begin{itemize}
\item {} 
\end{itemize}
&&&&&&&\\
\sphinxhline
\sphinxAtStartPar
7
&
\sphinxAtStartPar
6
&
\sphinxAtStartPar
5
&
\sphinxAtStartPar
4
&
\sphinxAtStartPar
3
&
\sphinxAtStartPar
2
&
\sphinxAtStartPar
1
&
\sphinxAtStartPar
0
\\
\sphinxhline
\sphinxAtStartPar
QSEC
&
\sphinxAtStartPar
HSEC
&&&&&&
\sphinxAtStartPar
SEC
\\
\sphinxbottomrule
\end{tabular}
\sphinxtableafterendhook\par
\sphinxattableend\end{savenotes}


\begin{savenotes}\sphinxattablestart
\sphinxthistablewithglobalstyle
\centering
\begin{tabular}[t]{\X{33}{99}\X{33}{99}\X{33}{99}}
\sphinxtoprule
\sphinxtableatstartofbodyhook
\sphinxAtStartPar
位域 |
&
\sphinxAtStartPar
名称     | |
&
\sphinxAtStartPar
描述                                        | |
\\
\sphinxhline
\sphinxAtStartPar
31:8
&\begin{itemize}
\item {} 
\end{itemize}
&\begin{itemize}
\item {} 
\end{itemize}
\\
\sphinxhline
\sphinxAtStartPar
7
&
\sphinxAtStartPar
QSEC
&
\sphinxAtStartPar
四分之一秒中断使能                          |

\sphinxAtStartPar
1:使能                                     |

\sphinxAtStartPar
0:禁能                                     |
\\
\sphinxhline
\sphinxAtStartPar
6
&
\sphinxAtStartPar
HSEC
&
\sphinxAtStartPar
半秒中断使能                                |

\sphinxAtStartPar
1:使能                                     |

\sphinxAtStartPar
0:禁能                                     |
\\
\sphinxhline
\sphinxAtStartPar
5
&
\sphinxAtStartPar
TRIM
&
\sphinxAtStartPar
Rtc\_calib中断使能                           |

\sphinxAtStartPar
1:使能                                     |

\sphinxAtStartPar
0:禁能                                     |
\\
\sphinxhline
\sphinxAtStartPar
4
&
\sphinxAtStartPar
ALARM
&
\sphinxAtStartPar
闹钟中断使能                                |

\sphinxAtStartPar
1:使能                                     |

\sphinxAtStartPar
0:禁能                                     |
\\
\sphinxhline
\sphinxAtStartPar
3
&
\sphinxAtStartPar
DATE
&
\sphinxAtStartPar
天中断使能                                  |

\sphinxAtStartPar
1:使能                                     |

\sphinxAtStartPar
0:禁能                                     |
\\
\sphinxhline
\sphinxAtStartPar
2
&
\sphinxAtStartPar
HOUR
&
\sphinxAtStartPar
小时中断使能                                |

\sphinxAtStartPar
1:使能                                     |

\sphinxAtStartPar
0:禁能                                     |
\\
\sphinxhline
\sphinxAtStartPar
1
&
\sphinxAtStartPar
MIN
&
\sphinxAtStartPar
分钟中断使能                                |

\sphinxAtStartPar
1:使能                                     |

\sphinxAtStartPar
0:禁能                                     |
\\
\sphinxhline
\sphinxAtStartPar
0
&
\sphinxAtStartPar
SEC
&
\sphinxAtStartPar
秒中断使能                                  |

\sphinxAtStartPar
1:使能                                     |

\sphinxAtStartPar
0:禁能                                     |
\\
\sphinxbottomrule
\end{tabular}
\sphinxtableafterendhook\par
\sphinxattableend\end{savenotes}

\sphinxAtStartPar
\sphinxstylestrong{中断标志寄存器IF}


\begin{savenotes}\sphinxattablestart
\sphinxthistablewithglobalstyle
\centering
\begin{tabular}[t]{\X{20}{100}\X{20}{100}\X{20}{100}\X{20}{100}\X{20}{100}}
\sphinxtoprule
\sphinxtableatstartofbodyhook
\sphinxAtStartPar
寄存器 |
&
\begin{DUlineblock}{0em}
\item[] 偏移 |
\end{DUlineblock}
&
\begin{DUlineblock}{0em}
\item[] 
\item[] {\color{red}\bfseries{}|}
\end{DUlineblock}
&
\sphinxAtStartPar
复位值 |    描 | |
&
\begin{DUlineblock}{0em}
\item[] |
  |
\end{DUlineblock}
\\
\sphinxhline
\sphinxAtStartPar
IF
&
\sphinxAtStartPar
0x20
&&
\sphinxAtStartPar
0 000000
&
\sphinxAtStartPar
中断标志寄存器             |
\\
\sphinxbottomrule
\end{tabular}
\sphinxtableafterendhook\par
\sphinxattableend\end{savenotes}


\begin{savenotes}\sphinxattablestart
\sphinxthistablewithglobalstyle
\centering
\begin{tabular}[t]{\X{12}{96}\X{12}{96}\X{12}{96}\X{12}{96}\X{12}{96}\X{12}{96}\X{12}{96}\X{12}{96}}
\sphinxtoprule
\sphinxtableatstartofbodyhook
\sphinxAtStartPar
31
&
\sphinxAtStartPar
30
&
\sphinxAtStartPar
29
&
\sphinxAtStartPar
28
&
\sphinxAtStartPar
27
&
\sphinxAtStartPar
26
&
\sphinxAtStartPar
25
&
\sphinxAtStartPar
24
\\
\sphinxhline\begin{itemize}
\item {} 
\end{itemize}
&&&&&&&\\
\sphinxhline
\sphinxAtStartPar
23
&
\sphinxAtStartPar
22
&
\sphinxAtStartPar
21
&
\sphinxAtStartPar
20
&
\sphinxAtStartPar
19
&
\sphinxAtStartPar
18
&
\sphinxAtStartPar
17
&
\sphinxAtStartPar
16
\\
\sphinxhline\begin{itemize}
\item {} 
\end{itemize}
&&&&&&&\\
\sphinxhline
\sphinxAtStartPar
15
&
\sphinxAtStartPar
14
&
\sphinxAtStartPar
13
&
\sphinxAtStartPar
12
&
\sphinxAtStartPar
11
&
\sphinxAtStartPar
10
&
\sphinxAtStartPar
9
&
\sphinxAtStartPar
8
\\
\sphinxhline\begin{itemize}
\item {} 
\end{itemize}
&&&&&&&\\
\sphinxhline
\sphinxAtStartPar
7
&
\sphinxAtStartPar
6
&
\sphinxAtStartPar
5
&
\sphinxAtStartPar
4
&
\sphinxAtStartPar
3
&
\sphinxAtStartPar
2
&
\sphinxAtStartPar
1
&
\sphinxAtStartPar
0
\\
\sphinxhline
\sphinxAtStartPar
QSEC
&
\sphinxAtStartPar
HSEC
&&&&&&
\sphinxAtStartPar
SEC
\\
\sphinxbottomrule
\end{tabular}
\sphinxtableafterendhook\par
\sphinxattableend\end{savenotes}


\begin{savenotes}\sphinxattablestart
\sphinxthistablewithglobalstyle
\centering
\begin{tabular}[t]{\X{33}{99}\X{33}{99}\X{33}{99}}
\sphinxtoprule
\sphinxtableatstartofbodyhook
\sphinxAtStartPar
位域 |
&
\sphinxAtStartPar
名称     | |
&
\sphinxAtStartPar
描述                                        | |
\\
\sphinxhline
\sphinxAtStartPar
31:8
&\begin{itemize}
\item {} 
\end{itemize}
&\begin{itemize}
\item {} 
\end{itemize}
\\
\sphinxhline
\sphinxAtStartPar
7
&
\sphinxAtStartPar
QSEC
&
\sphinxAtStartPar
四分之一秒中断状态,R/W1C                   |

\sphinxAtStartPar
1:中断已发生                               |

\sphinxAtStartPar
0:中断未发生                               |
\\
\sphinxhline
\sphinxAtStartPar
6
&
\sphinxAtStartPar
HSEC
&
\sphinxAtStartPar
半秒中断状态,R/W1C                         |

\sphinxAtStartPar
1:中断已发生                               |

\sphinxAtStartPar
0:中断未发生                               |
\\
\sphinxhline
\sphinxAtStartPar
5
&
\sphinxAtStartPar
TRIM
&
\sphinxAtStartPar
Rtc\_calib中断状态,写1清零                  |

\sphinxAtStartPar
1:中断已发生                               |

\sphinxAtStartPar
0:中断未发生                               |
\\
\sphinxhline
\sphinxAtStartPar
4
&
\sphinxAtStartPar
ALARM
&
\sphinxAtStartPar
闹钟中断状态,写1清零                       |

\sphinxAtStartPar
1:中断已发生                               |

\sphinxAtStartPar
0:中断未发生                               |
\\
\sphinxhline
\sphinxAtStartPar
3
&
\sphinxAtStartPar
DATE
&
\sphinxAtStartPar
天中断状态,写1清零                         |

\sphinxAtStartPar
1:中断已发生                               |

\sphinxAtStartPar
0:中断未发生                               |
\\
\sphinxhline
\sphinxAtStartPar
2
&
\sphinxAtStartPar
HOUR
&
\sphinxAtStartPar
小时中断状态,写1清零                       |

\sphinxAtStartPar
1:中断已发生                               |

\sphinxAtStartPar
0:中断未发生                               |
\\
\sphinxhline
\sphinxAtStartPar
1
&
\sphinxAtStartPar
MIN
&
\sphinxAtStartPar
分钟中断状态,写1清零                       |

\sphinxAtStartPar
1:中断已发生                               |

\sphinxAtStartPar
0:中断未发生                               |
\\
\sphinxhline
\sphinxAtStartPar
0
&
\sphinxAtStartPar
SEC
&
\sphinxAtStartPar
秒中断状态,写1清零                         |

\sphinxAtStartPar
1:中断已发生                               |

\sphinxAtStartPar
0:中断未发生                               |
\\
\sphinxbottomrule
\end{tabular}
\sphinxtableafterendhook\par
\sphinxattableend\end{savenotes}

\sphinxAtStartPar
\sphinxstylestrong{RTC使能寄存器EN}


\begin{savenotes}\sphinxattablestart
\sphinxthistablewithglobalstyle
\centering
\begin{tabular}[t]{\X{20}{100}\X{20}{100}\X{20}{100}\X{20}{100}\X{20}{100}}
\sphinxtoprule
\sphinxtableatstartofbodyhook
\sphinxAtStartPar
寄存器 |
&
\begin{DUlineblock}{0em}
\item[] 偏移 |
\end{DUlineblock}
&
\begin{DUlineblock}{0em}
\item[] 
\item[] {\color{red}\bfseries{}|}
\end{DUlineblock}
&
\sphinxAtStartPar
复位值 |    描 | |
&
\begin{DUlineblock}{0em}
\item[] |
  |
\end{DUlineblock}
\\
\sphinxhline
\sphinxAtStartPar
EN
&
\sphinxAtStartPar
0x24
&&
\sphinxAtStartPar
0 000000
&
\sphinxAtStartPar
RTC使能寄存器              |
\\
\sphinxbottomrule
\end{tabular}
\sphinxtableafterendhook\par
\sphinxattableend\end{savenotes}


\begin{savenotes}\sphinxattablestart
\sphinxthistablewithglobalstyle
\centering
\begin{tabular}[t]{\X{12}{96}\X{12}{96}\X{12}{96}\X{12}{96}\X{12}{96}\X{12}{96}\X{12}{96}\X{12}{96}}
\sphinxtoprule
\sphinxtableatstartofbodyhook
\sphinxAtStartPar
31
&
\sphinxAtStartPar
30
&
\sphinxAtStartPar
29
&
\sphinxAtStartPar
28
&
\sphinxAtStartPar
27
&
\sphinxAtStartPar
26
&
\sphinxAtStartPar
25
&
\sphinxAtStartPar
24
\\
\sphinxhline\begin{itemize}
\item {} 
\end{itemize}
&&&&&&&\\
\sphinxhline
\sphinxAtStartPar
23
&
\sphinxAtStartPar
22
&
\sphinxAtStartPar
21
&
\sphinxAtStartPar
20
&
\sphinxAtStartPar
19
&
\sphinxAtStartPar
18
&
\sphinxAtStartPar
17
&
\sphinxAtStartPar
16
\\
\sphinxhline\begin{itemize}
\item {} 
\end{itemize}
&&&&&&&\\
\sphinxhline
\sphinxAtStartPar
15
&
\sphinxAtStartPar
14
&
\sphinxAtStartPar
13
&
\sphinxAtStartPar
12
&
\sphinxAtStartPar
11
&
\sphinxAtStartPar
10
&
\sphinxAtStartPar
9
&
\sphinxAtStartPar
8
\\
\sphinxhline\begin{itemize}
\item {} 
\end{itemize}
&&&&&&&\\
\sphinxhline
\sphinxAtStartPar
7
&
\sphinxAtStartPar
6
&
\sphinxAtStartPar
5
&
\sphinxAtStartPar
4
&
\sphinxAtStartPar
3
&
\sphinxAtStartPar
2
&
\sphinxAtStartPar
1
&
\sphinxAtStartPar
0
\\
\sphinxhline\begin{itemize}
\item {} 
\end{itemize}
&&&&&&&
\sphinxAtStartPar
EN
\\
\sphinxbottomrule
\end{tabular}
\sphinxtableafterendhook\par
\sphinxattableend\end{savenotes}


\begin{savenotes}\sphinxattablestart
\sphinxthistablewithglobalstyle
\centering
\begin{tabular}[t]{\X{33}{99}\X{33}{99}\X{33}{99}}
\sphinxtoprule
\sphinxtableatstartofbodyhook
\sphinxAtStartPar
位域 |
&
\sphinxAtStartPar
名称     | |
&
\sphinxAtStartPar
描述                                        | |
\\
\sphinxhline
\sphinxAtStartPar
31:1
&\begin{itemize}
\item {} 
\end{itemize}
&\begin{itemize}
\item {} 
\end{itemize}
\\
\sphinxhline
\sphinxAtStartPar
0
&
\sphinxAtStartPar
EN
&
\sphinxAtStartPar
RTC使能寄存器                               |

\sphinxAtStartPar
1:使能                                     |

\sphinxAtStartPar
0:禁能                                     |
\\
\sphinxbottomrule
\end{tabular}
\sphinxtableafterendhook\par
\sphinxattableend\end{savenotes}

\sphinxAtStartPar
\sphinxstylestrong{配置状态寄存器CFGABLE}


\begin{savenotes}\sphinxattablestart
\sphinxthistablewithglobalstyle
\centering
\begin{tabular}[t]{\X{20}{100}\X{20}{100}\X{20}{100}\X{20}{100}\X{20}{100}}
\sphinxtoprule
\sphinxtableatstartofbodyhook
\sphinxAtStartPar
寄存器 |
&
\begin{DUlineblock}{0em}
\item[] 偏移 |
\end{DUlineblock}
&
\begin{DUlineblock}{0em}
\item[] 
\item[] {\color{red}\bfseries{}|}
\end{DUlineblock}
&
\sphinxAtStartPar
复位值 |    描 | |
&
\begin{DUlineblock}{0em}
\item[] |
  |
\end{DUlineblock}
\\
\sphinxhline
\sphinxAtStartPar
CFGABLE
&
\sphinxAtStartPar
0x28
&&
\sphinxAtStartPar
0 000000
&
\sphinxAtStartPar
配置状态寄存器             |
\\
\sphinxbottomrule
\end{tabular}
\sphinxtableafterendhook\par
\sphinxattableend\end{savenotes}


\begin{savenotes}\sphinxattablestart
\sphinxthistablewithglobalstyle
\centering
\begin{tabular}[t]{\X{12}{96}\X{12}{96}\X{12}{96}\X{12}{96}\X{12}{96}\X{12}{96}\X{12}{96}\X{12}{96}}
\sphinxtoprule
\sphinxtableatstartofbodyhook
\sphinxAtStartPar
31
&
\sphinxAtStartPar
30
&
\sphinxAtStartPar
29
&
\sphinxAtStartPar
28
&
\sphinxAtStartPar
27
&
\sphinxAtStartPar
26
&
\sphinxAtStartPar
25
&
\sphinxAtStartPar
24
\\
\sphinxhline\begin{itemize}
\item {} 
\end{itemize}
&&&&&&&\\
\sphinxhline
\sphinxAtStartPar
23
&
\sphinxAtStartPar
22
&
\sphinxAtStartPar
21
&
\sphinxAtStartPar
20
&
\sphinxAtStartPar
19
&
\sphinxAtStartPar
18
&
\sphinxAtStartPar
17
&
\sphinxAtStartPar
16
\\
\sphinxhline\begin{itemize}
\item {} 
\end{itemize}
&&&&&&&\\
\sphinxhline
\sphinxAtStartPar
15
&
\sphinxAtStartPar
14
&
\sphinxAtStartPar
13
&
\sphinxAtStartPar
12
&
\sphinxAtStartPar
11
&
\sphinxAtStartPar
10
&
\sphinxAtStartPar
9
&
\sphinxAtStartPar
8
\\
\sphinxhline\begin{itemize}
\item {} 
\end{itemize}
&&&&&&&\\
\sphinxhline
\sphinxAtStartPar
7
&
\sphinxAtStartPar
6
&
\sphinxAtStartPar
5
&
\sphinxAtStartPar
4
&
\sphinxAtStartPar
3
&
\sphinxAtStartPar
2
&
\sphinxAtStartPar
1
&
\sphinxAtStartPar
0
\\
\sphinxhline\begin{itemize}
\item {} 
\end{itemize}
&&&&&&&
\sphinxAtStartPar
C BLE
\\
\sphinxbottomrule
\end{tabular}
\sphinxtableafterendhook\par
\sphinxattableend\end{savenotes}


\begin{savenotes}\sphinxattablestart
\sphinxthistablewithglobalstyle
\centering
\begin{tabular}[t]{\X{33}{99}\X{33}{99}\X{33}{99}}
\sphinxtoprule
\sphinxtableatstartofbodyhook
\sphinxAtStartPar
位域 |
&
\sphinxAtStartPar
名称     | |
&
\sphinxAtStartPar
描述                                        | |
\\
\sphinxhline
\sphinxAtStartPar
31:1
&\begin{itemize}
\item {} 
\end{itemize}
&\begin{itemize}
\item {} 
\end{itemize}
\\
\sphinxhline
\sphinxAtStartPar
0
&
\sphinxAtStartPar
CFGABLE
&
\sphinxAtStartPar
寄存器可配置指示。                          |

\sphinxAtStartPar
如果需要更改RTC的寄存器时                   | 先查询此寄存器,当CFGABLE为1时,尽快配置 | 的寄存器(IE和IF的配置不需要关注这一位) |
\\
\sphinxbottomrule
\end{tabular}
\sphinxtableafterendhook\par
\sphinxattableend\end{savenotes}

\sphinxAtStartPar
\sphinxstylestrong{时钟调整寄存器TRIM}


\begin{savenotes}\sphinxattablestart
\sphinxthistablewithglobalstyle
\centering
\begin{tabular}[t]{\X{20}{100}\X{20}{100}\X{20}{100}\X{20}{100}\X{20}{100}}
\sphinxtoprule
\sphinxtableatstartofbodyhook
\sphinxAtStartPar
寄存器 |
&
\begin{DUlineblock}{0em}
\item[] 偏移 |
\end{DUlineblock}
&
\begin{DUlineblock}{0em}
\item[] 
\item[] {\color{red}\bfseries{}|}
\end{DUlineblock}
&
\sphinxAtStartPar
复位值 |    描 | |
&
\begin{DUlineblock}{0em}
\item[] |
  |
\end{DUlineblock}
\\
\sphinxhline
\sphinxAtStartPar
TRIM
&
\sphinxAtStartPar
0x2C
&&
\sphinxAtStartPar
0 000000
&
\sphinxAtStartPar
时钟调整寄存器             |
\\
\sphinxbottomrule
\end{tabular}
\sphinxtableafterendhook\par
\sphinxattableend\end{savenotes}


\begin{savenotes}\sphinxattablestart
\sphinxthistablewithglobalstyle
\centering
\begin{tabular}[t]{\X{12}{96}\X{12}{96}\X{12}{96}\X{12}{96}\X{12}{96}\X{12}{96}\X{12}{96}\X{12}{96}}
\sphinxtoprule
\sphinxtableatstartofbodyhook
\sphinxAtStartPar
31
&
\sphinxAtStartPar
30
&
\sphinxAtStartPar
29
&
\sphinxAtStartPar
28
&
\sphinxAtStartPar
27
&
\sphinxAtStartPar
26
&
\sphinxAtStartPar
25
&
\sphinxAtStartPar
24
\\
\sphinxhline\begin{itemize}
\item {} 
\end{itemize}
&&&&&&&\\
\sphinxhline
\sphinxAtStartPar
23
&
\sphinxAtStartPar
22
&
\sphinxAtStartPar
21
&
\sphinxAtStartPar
20
&
\sphinxAtStartPar
19
&
\sphinxAtStartPar
18
&
\sphinxAtStartPar
17
&
\sphinxAtStartPar
16
\\
\sphinxhline\begin{itemize}
\item {} 
\end{itemize}
&&&&&&&\\
\sphinxhline
\sphinxAtStartPar
15
&
\sphinxAtStartPar
14
&
\sphinxAtStartPar
13
&
\sphinxAtStartPar
12
&
\sphinxAtStartPar
11
&
\sphinxAtStartPar
10
&
\sphinxAtStartPar
9
&
\sphinxAtStartPar
8
\\
\sphinxhline\begin{itemize}
\item {} 
\end{itemize}
&&&&&&&
\sphinxAtStartPar
DEC
\\
\sphinxhline
\sphinxAtStartPar
7
&
\sphinxAtStartPar
6
&
\sphinxAtStartPar
5
&
\sphinxAtStartPar
4
&
\sphinxAtStartPar
3
&
\sphinxAtStartPar
2
&
\sphinxAtStartPar
1
&
\sphinxAtStartPar
0
\\
\sphinxhline
\sphinxAtStartPar
ADJ
&&&&&&&\\
\sphinxbottomrule
\end{tabular}
\sphinxtableafterendhook\par
\sphinxattableend\end{savenotes}


\begin{savenotes}\sphinxattablestart
\sphinxthistablewithglobalstyle
\centering
\begin{tabular}[t]{\X{33}{99}\X{33}{99}\X{33}{99}}
\sphinxtoprule
\sphinxtableatstartofbodyhook
\sphinxAtStartPar
位域 |
&
\sphinxAtStartPar
名称     | |
&
\sphinxAtStartPar
描述                                        | |
\\
\sphinxhline
\sphinxAtStartPar
31:9
&\begin{itemize}
\item {} 
\end{itemize}
&\begin{itemize}
\item {} 
\end{itemize}
\\
\sphinxhline
\sphinxAtStartPar
8
&
\sphinxAtStartPar
DEC
&
\sphinxAtStartPar
用于调                                      | SECNT的计数周期,默认为32768,如果DEC为1, | 周期调整为32768\sphinxhyphen{}ADJ,否则调整为32768+ADJ |
\\
\sphinxhline
\sphinxAtStartPar
7:0
&
\sphinxAtStartPar
ADJ
&\\
\sphinxbottomrule
\end{tabular}
\sphinxtableafterendhook\par
\sphinxattableend\end{savenotes}

\sphinxAtStartPar
\sphinxstylestrong{时钟微调整寄存器TRIMM}


\begin{savenotes}\sphinxattablestart
\sphinxthistablewithglobalstyle
\centering
\begin{tabular}[t]{\X{20}{100}\X{20}{100}\X{20}{100}\X{20}{100}\X{20}{100}}
\sphinxtoprule
\sphinxtableatstartofbodyhook
\sphinxAtStartPar
寄存器 |
&
\begin{DUlineblock}{0em}
\item[] 偏移 |
\end{DUlineblock}
&
\begin{DUlineblock}{0em}
\item[] 
\item[] {\color{red}\bfseries{}|}
\end{DUlineblock}
&
\sphinxAtStartPar
复位值 |    描 | |
&
\begin{DUlineblock}{0em}
\item[] |
  |
\end{DUlineblock}
\\
\sphinxhline
\sphinxAtStartPar
TRIMM
&
\sphinxAtStartPar
0x30
&&
\sphinxAtStartPar
0 000000
&
\sphinxAtStartPar
时钟微调整寄存器           |
\\
\sphinxbottomrule
\end{tabular}
\sphinxtableafterendhook\par
\sphinxattableend\end{savenotes}


\begin{savenotes}\sphinxattablestart
\sphinxthistablewithglobalstyle
\centering
\begin{tabular}[t]{\X{12}{96}\X{12}{96}\X{12}{96}\X{12}{96}\X{12}{96}\X{12}{96}\X{12}{96}\X{12}{96}}
\sphinxtoprule
\sphinxtableatstartofbodyhook
\sphinxAtStartPar
31
&
\sphinxAtStartPar
30
&
\sphinxAtStartPar
29
&
\sphinxAtStartPar
28
&
\sphinxAtStartPar
27
&
\sphinxAtStartPar
26
&
\sphinxAtStartPar
25
&
\sphinxAtStartPar
24
\\
\sphinxhline\begin{itemize}
\item {} 
\end{itemize}
&&&&&&&\\
\sphinxhline
\sphinxAtStartPar
23
&
\sphinxAtStartPar
22
&
\sphinxAtStartPar
21
&
\sphinxAtStartPar
20
&
\sphinxAtStartPar
19
&
\sphinxAtStartPar
18
&
\sphinxAtStartPar
17
&
\sphinxAtStartPar
16
\\
\sphinxhline\begin{itemize}
\item {} 
\end{itemize}
&&&&&&&\\
\sphinxhline
\sphinxAtStartPar
15
&
\sphinxAtStartPar
14
&
\sphinxAtStartPar
13
&
\sphinxAtStartPar
12
&
\sphinxAtStartPar
11
&
\sphinxAtStartPar
10
&
\sphinxAtStartPar
9
&
\sphinxAtStartPar
8
\\
\sphinxhline\begin{itemize}
\item {} 
\end{itemize}
&&&&&&&\\
\sphinxhline
\sphinxAtStartPar
7
&
\sphinxAtStartPar
6
&
\sphinxAtStartPar
5
&
\sphinxAtStartPar
4
&
\sphinxAtStartPar
3
&
\sphinxAtStartPar
2
&
\sphinxAtStartPar
1
&
\sphinxAtStartPar
0
\\
\sphinxhline\begin{itemize}
\item {} 
\end{itemize}
&&&&&&&\\
\sphinxbottomrule
\end{tabular}
\sphinxtableafterendhook\par
\sphinxattableend\end{savenotes}


\begin{savenotes}\sphinxattablestart
\sphinxthistablewithglobalstyle
\centering
\begin{tabular}[t]{\X{33}{99}\X{33}{99}\X{33}{99}}
\sphinxtoprule
\sphinxtableatstartofbodyhook
\sphinxAtStartPar
位域 |
&
\sphinxAtStartPar
名称     | |
&
\sphinxAtStartPar
描述                                        | |
\\
\sphinxhline
\sphinxAtStartPar
31:4
&\begin{itemize}
\item {} 
\end{itemize}
&\begin{itemize}
\item {} 
\end{itemize}
\\
\sphinxhline
\sphinxAtStartPar
3
&
\sphinxAtStartPar
INRC
&
\sphinxAtStartPar
用于计数周期                                | (n个周期调整一次\textless{}n=2\sphinxhyphen{}8\textgreater{}),如果inrc为1  | ,则第n个计数周期                           | (32768\(\pm\)ADJ)\sphinxhyphen{}1,否则调整为(32768\(\pm\)ADJ)+1;  |

\sphinxAtStartPar
(cycles=0时,不进行微调整;                 | les=1,则n为2;cycles=7,则n为8;以此类推)  |
\\
\sphinxhline
\sphinxAtStartPar
2:0
&
\sphinxAtStartPar
CYCLES
&\\
\sphinxbottomrule
\end{tabular}
\sphinxtableafterendhook\par
\sphinxattableend\end{savenotes}

\sphinxAtStartPar
\sphinxstylestrong{目标值寄存器CALIBREFCNT}


\begin{savenotes}\sphinxattablestart
\sphinxthistablewithglobalstyle
\centering
\begin{tabular}[t]{\X{20}{100}\X{20}{100}\X{20}{100}\X{20}{100}\X{20}{100}}
\sphinxtoprule
\sphinxtableatstartofbodyhook
\sphinxAtStartPar
寄存器 |
&
\begin{DUlineblock}{0em}
\item[] 偏移 |
\end{DUlineblock}
&
\begin{DUlineblock}{0em}
\item[] 
\item[] {\color{red}\bfseries{}|}
\end{DUlineblock}
&
\sphinxAtStartPar
复位值 |    描 | |
&
\begin{DUlineblock}{0em}
\item[] |
  |
\end{DUlineblock}
\\
\sphinxhline
\sphinxAtStartPar
CALIBREFCNT
&
\sphinxAtStartPar
0X60
&&
\sphinxAtStartPar
0 000000
&
\sphinxAtStartPar
Refclk时钟下cnt\_ref\_target |
\\
\sphinxbottomrule
\end{tabular}
\sphinxtableafterendhook\par
\sphinxattableend\end{savenotes}


\begin{savenotes}\sphinxattablestart
\sphinxthistablewithglobalstyle
\centering
\begin{tabular}[t]{\X{12}{96}\X{12}{96}\X{12}{96}\X{12}{96}\X{12}{96}\X{12}{96}\X{12}{96}\X{12}{96}}
\sphinxtoprule
\sphinxtableatstartofbodyhook
\sphinxAtStartPar
31
&
\sphinxAtStartPar
30
&
\sphinxAtStartPar
29
&
\sphinxAtStartPar
28
&
\sphinxAtStartPar
27
&
\sphinxAtStartPar
26
&
\sphinxAtStartPar
25
&
\sphinxAtStartPar
24
\\
\sphinxhline\begin{itemize}
\item {} 
\end{itemize}
&&&&&&&\\
\sphinxhline
\sphinxAtStartPar
23
&
\sphinxAtStartPar
22
&
\sphinxAtStartPar
21
&
\sphinxAtStartPar
20
&
\sphinxAtStartPar
19
&
\sphinxAtStartPar
18
&
\sphinxAtStartPar
17
&
\sphinxAtStartPar
16
\\
\sphinxhline\begin{itemize}
\item {} 
\end{itemize}
&&&&
\sphinxAtStartPar
C BR NT
&&&\\
\sphinxhline
\sphinxAtStartPar
15
&
\sphinxAtStartPar
14
&
\sphinxAtStartPar
13
&
\sphinxAtStartPar
12
&
\sphinxAtStartPar
11
&
\sphinxAtStartPar
10
&
\sphinxAtStartPar
9
&
\sphinxAtStartPar
8
\\
\sphinxhline
\sphinxAtStartPar
CALIBREFCNT
&&&&&&&\\
\sphinxhline
\sphinxAtStartPar
7
&
\sphinxAtStartPar
6
&
\sphinxAtStartPar
5
&
\sphinxAtStartPar
4
&
\sphinxAtStartPar
3
&
\sphinxAtStartPar
2
&
\sphinxAtStartPar
1
&
\sphinxAtStartPar
0
\\
\sphinxhline
\sphinxAtStartPar
CALIBREFCNT
&&&&&&&\\
\sphinxbottomrule
\end{tabular}
\sphinxtableafterendhook\par
\sphinxattableend\end{savenotes}


\begin{savenotes}\sphinxattablestart
\sphinxthistablewithglobalstyle
\centering
\begin{tabular}[t]{\X{33}{99}\X{33}{99}\X{33}{99}}
\sphinxtoprule
\sphinxtableatstartofbodyhook
\sphinxAtStartPar
位域 |
&
\sphinxAtStartPar
名称     | |
&
\sphinxAtStartPar
描述                                        | |
\\
\sphinxhline
\sphinxAtStartPar
31:21
&\begin{itemize}
\item {} 
\end{itemize}
&\begin{itemize}
\item {} 
\end{itemize}
\\
\sphinxhline
\sphinxAtStartPar
20:0
&
\sphinxAtStartPar
CALIBREFCNT
&
\sphinxAtStartPar
参考时钟下,cnt\_ref计时500ms,应计目标值    |
\\
\sphinxbottomrule
\end{tabular}
\sphinxtableafterendhook\par
\sphinxattableend\end{savenotes}

\sphinxAtStartPar
\sphinxstylestrong{使能rtc校正寄存器CALIBEN}


\begin{savenotes}\sphinxattablestart
\sphinxthistablewithglobalstyle
\centering
\begin{tabular}[t]{\X{20}{100}\X{20}{100}\X{20}{100}\X{20}{100}\X{20}{100}}
\sphinxtoprule
\sphinxtableatstartofbodyhook
\sphinxAtStartPar
寄存器 |
&
\begin{DUlineblock}{0em}
\item[] 偏移 |
\end{DUlineblock}
&
\begin{DUlineblock}{0em}
\item[] 
\item[] {\color{red}\bfseries{}|}
\end{DUlineblock}
&
\sphinxAtStartPar
复位值 |    描 | |
&
\begin{DUlineblock}{0em}
\item[] |
  |
\end{DUlineblock}
\\
\sphinxhline
\sphinxAtStartPar
CALIBEN
&
\sphinxAtStartPar
0X64
&&
\sphinxAtStartPar
0 000000
&
\sphinxAtStartPar
使能rtc校正                |
\\
\sphinxbottomrule
\end{tabular}
\sphinxtableafterendhook\par
\sphinxattableend\end{savenotes}


\begin{savenotes}\sphinxattablestart
\sphinxthistablewithglobalstyle
\centering
\begin{tabular}[t]{\X{12}{96}\X{12}{96}\X{12}{96}\X{12}{96}\X{12}{96}\X{12}{96}\X{12}{96}\X{12}{96}}
\sphinxtoprule
\sphinxtableatstartofbodyhook
\sphinxAtStartPar
31
&
\sphinxAtStartPar
30
&
\sphinxAtStartPar
29
&
\sphinxAtStartPar
28
&
\sphinxAtStartPar
27
&
\sphinxAtStartPar
26
&
\sphinxAtStartPar
25
&
\sphinxAtStartPar
24
\\
\sphinxhline\begin{itemize}
\item {} 
\end{itemize}
&&&&&&&\\
\sphinxhline
\sphinxAtStartPar
23
&
\sphinxAtStartPar
22
&
\sphinxAtStartPar
21
&
\sphinxAtStartPar
20
&
\sphinxAtStartPar
19
&
\sphinxAtStartPar
18
&
\sphinxAtStartPar
17
&
\sphinxAtStartPar
16
\\
\sphinxhline\begin{itemize}
\item {} 
\end{itemize}
&&&&&&&\\
\sphinxhline
\sphinxAtStartPar
15
&
\sphinxAtStartPar
14
&
\sphinxAtStartPar
13
&
\sphinxAtStartPar
12
&
\sphinxAtStartPar
11
&
\sphinxAtStartPar
10
&
\sphinxAtStartPar
9
&
\sphinxAtStartPar
8
\\
\sphinxhline\begin{itemize}
\item {} 
\end{itemize}
&&&&&&&\\
\sphinxhline
\sphinxAtStartPar
7
&
\sphinxAtStartPar
6
&
\sphinxAtStartPar
5
&
\sphinxAtStartPar
4
&
\sphinxAtStartPar
3
&
\sphinxAtStartPar
2
&
\sphinxAtStartPar
1
&
\sphinxAtStartPar
0
\\
\sphinxhline\begin{itemize}
\item {} 
\end{itemize}
&&&&&&&
\sphinxAtStartPar
c ble
\\
\sphinxbottomrule
\end{tabular}
\sphinxtableafterendhook\par
\sphinxattableend\end{savenotes}


\begin{savenotes}\sphinxattablestart
\sphinxthistablewithglobalstyle
\centering
\begin{tabular}[t]{\X{33}{99}\X{33}{99}\X{33}{99}}
\sphinxtoprule
\sphinxtableatstartofbodyhook
\sphinxAtStartPar
位域 |
&
\sphinxAtStartPar
名称     | |
&
\sphinxAtStartPar
描述                                        | |
\\
\sphinxhline
\sphinxAtStartPar
31:1
&\begin{itemize}
\item {} 
\end{itemize}
&\begin{itemize}
\item {} 
\end{itemize}
\\
\sphinxhline
\sphinxAtStartPar
0
&
\sphinxAtStartPar
cenable
&
\sphinxAtStartPar
Rtc时钟校正                                 |

\sphinxAtStartPar
1:使能                                     |

\sphinxAtStartPar
0:失能                                     |
\\
\sphinxbottomrule
\end{tabular}
\sphinxtableafterendhook\par
\sphinxattableend\end{savenotes}

\sphinxAtStartPar
\sphinxstylestrong{校正状态寄存器CALIBST}


\begin{savenotes}\sphinxattablestart
\sphinxthistablewithglobalstyle
\centering
\begin{tabular}[t]{\X{20}{100}\X{20}{100}\X{20}{100}\X{20}{100}\X{20}{100}}
\sphinxtoprule
\sphinxtableatstartofbodyhook
\sphinxAtStartPar
寄存器 |
&
\begin{DUlineblock}{0em}
\item[] 偏移 |
\end{DUlineblock}
&
\begin{DUlineblock}{0em}
\item[] 
\item[] {\color{red}\bfseries{}|}
\end{DUlineblock}
&
\sphinxAtStartPar
复位值 |    描 | |
&
\begin{DUlineblock}{0em}
\item[] |
  |
\end{DUlineblock}
\\
\sphinxhline
\sphinxAtStartPar
CALIBSR
&
\sphinxAtStartPar
0x68
&&
\sphinxAtStartPar
0 000000
&
\sphinxAtStartPar
校正状态寄存器             |
\\
\sphinxbottomrule
\end{tabular}
\sphinxtableafterendhook\par
\sphinxattableend\end{savenotes}


\begin{savenotes}\sphinxattablestart
\sphinxthistablewithglobalstyle
\centering
\begin{tabular}[t]{\X{12}{96}\X{12}{96}\X{12}{96}\X{12}{96}\X{12}{96}\X{12}{96}\X{12}{96}\X{12}{96}}
\sphinxtoprule
\sphinxtableatstartofbodyhook
\sphinxAtStartPar
31
&
\sphinxAtStartPar
30
&
\sphinxAtStartPar
29
&
\sphinxAtStartPar
28
&
\sphinxAtStartPar
27
&
\sphinxAtStartPar
26
&
\sphinxAtStartPar
25
&
\sphinxAtStartPar
24
\\
\sphinxhline\begin{itemize}
\item {} 
\end{itemize}
&&&&&&&\\
\sphinxhline
\sphinxAtStartPar
23
&
\sphinxAtStartPar
22
&
\sphinxAtStartPar
21
&
\sphinxAtStartPar
20
&
\sphinxAtStartPar
19
&
\sphinxAtStartPar
18
&
\sphinxAtStartPar
17
&
\sphinxAtStartPar
16
\\
\sphinxhline\begin{itemize}
\item {} 
\end{itemize}
&&&&&&&\\
\sphinxhline
\sphinxAtStartPar
15
&
\sphinxAtStartPar
14
&
\sphinxAtStartPar
13
&
\sphinxAtStartPar
12
&
\sphinxAtStartPar
11
&
\sphinxAtStartPar
10
&
\sphinxAtStartPar
9
&
\sphinxAtStartPar
8
\\
\sphinxhline\begin{itemize}
\item {} 
\end{itemize}
&&&&&&&\\
\sphinxhline
\sphinxAtStartPar
7
&
\sphinxAtStartPar
6
&
\sphinxAtStartPar
5
&
\sphinxAtStartPar
4
&
\sphinxAtStartPar
3
&
\sphinxAtStartPar
2
&
\sphinxAtStartPar
1
&
\sphinxAtStartPar
0
\\
\sphinxhline\begin{itemize}
\item {} 
\end{itemize}
&&&&&&&\\
\sphinxbottomrule
\end{tabular}
\sphinxtableafterendhook\par
\sphinxattableend\end{savenotes}


\begin{savenotes}\sphinxattablestart
\sphinxthistablewithglobalstyle
\centering
\begin{tabular}[t]{\X{33}{99}\X{33}{99}\X{33}{99}}
\sphinxtoprule
\sphinxtableatstartofbodyhook
\sphinxAtStartPar
位域 |
&
\sphinxAtStartPar
名称     | |
&
\sphinxAtStartPar
描述                                        | |
\\
\sphinxhline
\sphinxAtStartPar
31:2
&\begin{itemize}
\item {} 
\end{itemize}
&\begin{itemize}
\item {} 
\end{itemize}
\\
\sphinxhline
\sphinxAtStartPar
2
&
\sphinxAtStartPar
flag
&
\sphinxAtStartPar
Rtc正在校正                                 |
\\
\sphinxhline
\sphinxAtStartPar
1
&
\sphinxAtStartPar
fail
&
\sphinxAtStartPar
rtc校正失败                                 |
\\
\sphinxhline
\sphinxAtStartPar
0
&
\sphinxAtStartPar
done
&
\sphinxAtStartPar
rtc校正完成                                 |
\\
\sphinxbottomrule
\end{tabular}
\sphinxtableafterendhook\par
\sphinxattableend\end{savenotes}

\sphinxstepscope


\section{UART接口控制器(UART)}
\label{\detokenize{SWM241/_u529f_u80fd_u63cf_u8ff0/UART_u63a5_u53e3_u63a7_u5236_u5668:uart-uart}}\label{\detokenize{SWM241/_u529f_u80fd_u63cf_u8ff0/UART_u63a5_u53e3_u63a7_u5236_u5668::doc}}
\sphinxAtStartPar
概述
\textasciitilde{}\textasciitilde{}

\sphinxAtStartPar
不同型号具备UART数量可能不同。使用前需使能对应UART模块时钟。

\sphinxAtStartPar
UART模块支持波特率配置,最高速度可达到模块时钟16分频。具备深度为8的FIFO,同时提供了多种中断供选择。
\begin{quote}

\sphinxAtStartPar
特性
\textasciitilde{}\textasciitilde{}
\end{quote}
\begin{itemize}
\item {} 
\sphinxAtStartPar
支持标准的UART协议

\item {} 
\sphinxAtStartPar
支持全双工模式

\item {} 
\sphinxAtStartPar
支持波特率可配置

\item {} 
\sphinxAtStartPar
支持8位/9位数据格式选择

\item {} 
\sphinxAtStartPar
可配置的奇偶校验位

\item {} 
\sphinxAtStartPar
支持1位/2位停止位选择

\item {} 
\sphinxAtStartPar
支持波特率自动调整

\item {} 
\sphinxAtStartPar
深度为8字节的发送和接收FIFO

\item {} 
\sphinxAtStartPar
支持break操作自动检测

\item {} 
\sphinxAtStartPar
支持接收超时中断

\item {} 
\sphinxAtStartPar
支持LIN模式

\item {} 
\sphinxAtStartPar
支持发送/接收数据LSB/MSB选择

\item {} 
\sphinxAtStartPar
支持发送/接收数据电平反向

\end{itemize}


\subsection{模块结构框图}
\label{\detokenize{SWM241/_u529f_u80fd_u63cf_u8ff0/UART_u63a5_u53e3_u63a7_u5236_u5668:id1}}
\sphinxAtStartPar
\sphinxincludegraphics{{SWM241/功能描述/mediaUART接口002}.emf}

\sphinxAtStartPar
图 6‑21 UART模块结构图


\subsection{功能描述}
\label{\detokenize{SWM241/_u529f_u80fd_u63cf_u8ff0/UART_u63a5_u53e3_u63a7_u5236_u5668:id2}}

\subsubsection{数据格式及波特率配置}
\label{\detokenize{SWM241/_u529f_u80fd_u63cf_u8ff0/UART_u63a5_u53e3_u63a7_u5236_u5668:id3}}
\sphinxAtStartPar
数据位

\sphinxAtStartPar
可以通过向CTRL寄存器的NINEBIT位写1,选择支持9位数据模式。该位默认为0,即8位数据模式

\sphinxAtStartPar
奇偶校验位

\sphinxAtStartPar
CTRL寄存器PAREN位使能奇偶校验,PARMD 位选择奇偶校验模式,分别有奇校验、偶校验、常1、常0等四种校验格式,根据需求可以灵活选择配置具体看下表:


\begin{savenotes}\sphinxattablestart
\sphinxthistablewithglobalstyle
\centering
\begin{tabular}[t]{\X{25}{100}\X{25}{100}\X{25}{100}\X{25}{100}}
\sphinxtoprule
\sphinxtableatstartofbodyhook
\sphinxAtStartPar
校验类型     |
&
\sphinxAtStartPar
CTRL{[}21{]}     |
&
\sphinxAtStartPar
CTRL{[}20{]}     |
&
\sphinxAtStartPar
CTRL{[}19{]}   |
\\
\sphinxhline
\sphinxAtStartPar
无校验       |
&
\sphinxAtStartPar
x            |
&
\sphinxAtStartPar
x            |
&
\sphinxAtStartPar
0          |
\\
\sphinxhline
\sphinxAtStartPar
奇校验       |
&
\sphinxAtStartPar
0            |
&
\sphinxAtStartPar
0            |
&
\sphinxAtStartPar
1          |
\\
\sphinxhline
\sphinxAtStartPar
偶校验       |
&
\sphinxAtStartPar
0            |
&
\sphinxAtStartPar
1            |
&
\sphinxAtStartPar
1          |
\\
\sphinxhline
\sphinxAtStartPar
校验位常为1  |
&
\begin{DUlineblock}{0em}
\item[] 
\end{DUlineblock}
&
\begin{DUlineblock}{0em}
\item[] 
\end{DUlineblock}
&
\begin{DUlineblock}{0em}
\item[] 
\end{DUlineblock}
\\
\sphinxhline
\sphinxAtStartPar
校验位常为0  |
&
\begin{DUlineblock}{0em}
\item[] 
\end{DUlineblock}
&
\begin{DUlineblock}{0em}
\item[] 
\end{DUlineblock}
&
\begin{DUlineblock}{0em}
\item[] 
\end{DUlineblock}
\\
\sphinxbottomrule
\end{tabular}
\sphinxtableafterendhook\par
\sphinxattableend\end{savenotes}

\sphinxAtStartPar
停止位

\sphinxAtStartPar
停止位位数默认为1位,可通过向CTRL寄存器STOP2B位选择停止位位数为2位。

\sphinxAtStartPar
字符格式如图 6‑22所示:

\sphinxAtStartPar
\sphinxincludegraphics{{SWM241/功能描述/mediaUART接口003}.emf}

\sphinxAtStartPar
图 6‑22 UART字符格式

\sphinxAtStartPar
使能波特率配置后,对BAUD寄存器BAUD位写入特定值,配置波特率。

\sphinxAtStartPar
配置方式如下:

\sphinxAtStartPar
目标波特率 = 系统主时钟 / (16 * (BAUD+ FRAC + 1))

\sphinxAtStartPar
波特率配置完成后,需将CTRL寄存器EN位使能,使能UART模块,使波特率配置生效。


\subsubsection{自动波特率功能}
\label{\detokenize{SWM241/_u529f_u80fd_u63cf_u8ff0/UART_u63a5_u53e3_u63a7_u5236_u5668:id4}}
\sphinxAtStartPar
UART自动波特率功能可以自动测量UART\_RX脚输入数据的波特率。当自动波特率测量完成后,测量的结果保存在BAUD寄存器的BAUD位。

\sphinxAtStartPar
自动波特率的检测时间,从UART\_RX数据的起始位到第一个上升沿的时间,通过配置BAUD寄存器ABRBIT位设定即2 $^{\text{ABRDBITS}}$位时间。配置BAUD寄存器ABREN位,使能自动波特率检测功能。初始阶段,RXD保持为1,一旦检测到下降沿,即为接收到起始位,自动波特率计数器开始计数,当检测到第一个上升沿时,自动波特率计数器停止计数。

\sphinxAtStartPar
自动波特率计数值除以检测时间长度的结果保存在BAUD位,ABREN位清零。

\sphinxAtStartPar
当自动波特率计数器溢出,BAUD寄存器的ABRERR置1,调节失败,写1清零。

\sphinxAtStartPar
配置流程:
\begin{itemize}
\item {} 
\sphinxAtStartPar
选择检测时间的长度,配置BAUD寄存器ABRBIT位

\item {} 
\sphinxAtStartPar
配置BAUD寄存器ABREN位,使能自动波特率检测功能

\item {} 
\sphinxAtStartPar
等待接收调节的数据,查看BAUD寄存器ABREN位,清零表示波特率检测完成

\item {} 
\sphinxAtStartPar
查看BAUD寄存器ABRERR位,查看自动调节波特率时计数器是否溢出

\item {} 
\sphinxAtStartPar
如果数据未溢出,则表示成功

\end{itemize}

\sphinxAtStartPar
\sphinxincludegraphics{{SWM241/功能描述/mediaUART接口004}.emf}

\sphinxAtStartPar
图 6‑23 自动波特率示意图


\subsubsection{FIFO及中断设置}
\label{\detokenize{SWM241/_u529f_u80fd_u63cf_u8ff0/UART_u63a5_u53e3_u63a7_u5236_u5668:fifo}}
\sphinxAtStartPar
UART模块包含深度为8的接收FIFO及发送FIFO,同时提供了与FIFO相配合的状态位中断,供操作使用。使用方式如下:
\begin{itemize}
\item {} 
\sphinxAtStartPar
通过FIFO寄存器配置中断触发条件,并获取FIFO内部数据数量
\begin{itemize}
\item {} 
\sphinxAtStartPar
TXTHR位设置发送FIFO阈值,当TXFIFO中数据量不超过设置值时,触发中断。当TXTHR位配置为0且使能CTRL中TXIE发送端FIFO中断时,UART使能后即触发发送中断

\item {} 
\sphinxAtStartPar
RXTHR位设置接收FIFO阈值,当RXFIFO中数据量不小于设置值时,触发中断。当RXTHR位配置为0且使能CTRL中RXIE发送端FIFO中断时,UART使能后接收到1个数据值即触发接收中断

\end{itemize}

\item {} 
\sphinxAtStartPar
通过CTRL寄存器RXIE位及TXIE位,使能FIFO中断

\item {} 
\sphinxAtStartPar
通过查询BAUD寄存器RXTHRF或TXTHRF位获取FIFO状态

\end{itemize}


\subsubsection{数据发送及接收}
\label{\detokenize{SWM241/_u529f_u80fd_u63cf_u8ff0/UART_u63a5_u53e3_u63a7_u5236_u5668:id5}}
\sphinxAtStartPar
将控制及状态寄存器(CTRL)EN位置1后,对应UART模块使能

\sphinxAtStartPar
对于发送操作:
\begin{itemize}
\item {} 
\sphinxAtStartPar
向DATA寄存器写入数据,数据发送至UART\_TX线

\item {} 
\sphinxAtStartPar
通过读取CTRL寄存器TXIDLE位状态,获取当前发送状态

\item {} 
\sphinxAtStartPar
可通过读取BAUD寄存器TXD位,获取当前TX线实时状态

\end{itemize}

\sphinxAtStartPar
对于接收操作:
\begin{itemize}
\item {} 
\sphinxAtStartPar
通过判断DATA寄存器中VAILD位,判断是否接收到有效数据

\item {} 
\sphinxAtStartPar
读取DATA寄存器,可获得UART\_RX线接收的数据

\item {} 
\sphinxAtStartPar
可通过读取BAUD寄存器RXD位,获取当前RX线实时状态

\item {} 
\sphinxAtStartPar
可设置接收超时中断。使能后,当接收相邻两个数据间隔时长超过设置时长时,将触发中断

\end{itemize}


\subsubsection{电平反向}
\label{\detokenize{SWM241/_u529f_u80fd_u63cf_u8ff0/UART_u63a5_u53e3_u63a7_u5236_u5668:id6}}
\sphinxAtStartPar
通过设置CFG寄存器的TXINV位及RXINV位,分别对TX和RX线设置取反,设置后电平立刻生效


\subsubsection{大小端控制}
\label{\detokenize{SWM241/_u529f_u80fd_u63cf_u8ff0/UART_u63a5_u53e3_u63a7_u5236_u5668:id7}}
\sphinxAtStartPar
通过CFG寄存器的MSBF位进行配置,设置数据传输是从高位(MSB)开始传输还是从低位(LSB)开始传输。


\subsubsection{LIN Fram}
\label{\detokenize{SWM241/_u529f_u80fd_u63cf_u8ff0/UART_u63a5_u53e3_u63a7_u5236_u5668:lin-fram}}
\sphinxAtStartPar
UART支持LIN功能。在主机模式下,支持LIN\_BREAK产生,在 从机模式下,支持LIN\_BREAK检测。报文是以报文帧的格式传输和发送。报文帧 由主机节点发送的报文头和从机发送的应答组成。报文帧的报头包括break域,同步域和帧识别码(帧ID)。帧
ID仅作为定义帧的用途,从机负责响应相关的帧ID,响应由数据域和校验域组成。

\sphinxAtStartPar
\sphinxincludegraphics{{SWM241/功能描述/mediaUART接口005}.emf}

\sphinxAtStartPar
图 6‑24 LIN Fram示意图

\sphinxAtStartPar
当使用LIN Fram时,可通过LINCR寄存器进行相关设置。

\sphinxAtStartPar
发送操作:

\sphinxAtStartPar
与正常的UART发送相比,选用LIN Fram发送时,除了基本操作步骤外,还需:
\begin{itemize}
\item {} 
\sphinxAtStartPar
通过CTRL寄存器将GENBRK位置1,拉低TX线。该位保持为1时,TX将持续保持低电平,直至该位清除

\item {} 
\sphinxAtStartPar
通过LINCR寄存器将GENBRKIE位置1,使能Break信号发送完成中断

\item {} 
\sphinxAtStartPar
设置LINCR寄存器GENBRK,发送Break信号至总线

\item {} 
\sphinxAtStartPar
Break信号发送完成后,中断产生,LINCR寄存器GENBRKIF位置1。可通过读此寄存器判断是否发送完成

\end{itemize}

\sphinxAtStartPar
注意:发送BREAK信号时,向DATA寄存器写入数据,数据同样会执行发送操作,但数据电平不会体现到TX线上,除非发送数据期间清除CTRL寄存器GENBRK位。

\sphinxAtStartPar
接收操作:

\sphinxAtStartPar
与正常的UART接收相比,选用LIN Fram接收时,除了基本操作步骤外,还需:
\begin{itemize}
\item {} 
\sphinxAtStartPar
通过LINCR寄存器将BRKIE位置1,使能检测到Break信号中断

\item {} 
\sphinxAtStartPar
设置CTRL寄存器BRKIE位及BRKDET位,当RX线接收到Break信号时,将触发中断

\item {} 
\sphinxAtStartPar
检测到Break信号并产生中断后,LINCR寄存器BRKDETIE位置1。可通过读此寄存器判断是否检测到Break信号

\end{itemize}

\sphinxAtStartPar
当Break信号不够长时,丢弃Break,BRKDETIF不置1,如图 6‑25所示:

\sphinxAtStartPar
\sphinxincludegraphics{{SWM241/功能描述/mediaUART接口006}.emf}

\sphinxAtStartPar
图 6‑25 Break信号不够长示意图

\sphinxAtStartPar
当Break信号恰好够长时,等接收线上收到高电平后,检测到Break,BRKDETIF置1,如图 6‑26所示:

\sphinxAtStartPar
\sphinxincludegraphics{{SWM241/功能描述/mediaUART接口007}.emf}

\sphinxAtStartPar
图 6‑26 Break信号恰好够长示意图

\sphinxAtStartPar
当Break信号足够长时,等接收线上收到高电平后,检测到Break,LINBRKST置1,如图 6‑27所示:

\sphinxAtStartPar
\sphinxincludegraphics{{SWM241/功能描述/mediaUART接口008}.emf}

\sphinxAtStartPar
图 6‑27 Break信号足够长示意图

\sphinxAtStartPar
\sphinxstylestrong{硬件流控}

\sphinxAtStartPar
硬件流控(RTS/CTS)制主要功能为防止串口传输时出现丢失数据的现象,使用流控制功能时需将通信两端的RTS和CTS对应相连,通过RTS和CTS可以控制两个串口设备间的串行数据流。

\sphinxAtStartPar
\sphinxstylestrong{RTS流控制}

\sphinxAtStartPar
RTS为输出信号,通过自动流控控制寄存器使能该信号并设置有效极性(高电平/低电平)以及触发阈值,当RTS为有效电平时表示可以接收数据,当接收数据达到所设置的阈值时,RTS无效。

\sphinxAtStartPar
\sphinxstylestrong{CTS流控制}

\sphinxAtStartPar
CTS为输入信号,通过自动流控控制寄存器使能该信号并设置有效极性(高电平/低电平),当RTS为有效电平时表示可以发送数据。

\sphinxAtStartPar
\sphinxincludegraphics{{SWM241/功能描述/mediaUART接口009}.emf}

\sphinxAtStartPar
图 6‑28硬件流控


\subsubsection{接收中断与超时中断}
\label{\detokenize{SWM241/_u529f_u80fd_u63cf_u8ff0/UART_u63a5_u53e3_u63a7_u5236_u5668:id8}}
\sphinxAtStartPar
以如下配置为例:

\sphinxAtStartPar
方式一:FIFO清空后,不产生超时中断
\begin{itemize}
\item {} 
\sphinxAtStartPar
配置FIFO寄存器RXLVL位为3,即RXThreshold=3,接收FIFO取值3

\item {} 
\sphinxAtStartPar
配置CTRL寄存器RXIE位为1,即RXThresholdIEn=1,配置接收FIFO中的个数\textgreater{} RXThreshold时触发中断

\item {} 
\sphinxAtStartPar
配置TOCR寄存器TIME位为10,即TimeoutTime = 10,超时时长 = TimeoutTime/(Baudrate/10) 秒

\item {} 
\sphinxAtStartPar
配置UARTx.TOCR寄存器MODE位为0,FIFO清空后,不产生超时中断

\item {} 
\sphinxAtStartPar
配置CTRL寄存器TOIE位为1,即TimeoutIEn = 1,超时中断,超过TimeoutTime/(Baudrate/10) 秒没有在RX线上接收到数据且接收FIFO中数据个数不为零时可触发中断

\end{itemize}

\sphinxAtStartPar
对方发送8个数据

\sphinxAtStartPar
\sphinxincludegraphics{{SWM241/功能描述/mediaUART接口010}.emf}

\sphinxAtStartPar
图 6‑29 对方发送8个数据接收FIFO示意图

\sphinxAtStartPar
每接收到一个数据,RX FIFO中数据个数加一,当RX FIFO中数据个数大于RXThreshold时,触发接收中断。
\begin{quote}

\sphinxAtStartPar
\sphinxstylestrong{对方发送9个数据}

\sphinxAtStartPar
\sphinxincludegraphics{{SWM241/功能描述/mediaUART接口011}.emf}
\end{quote}

\sphinxAtStartPar
图 6‑30 对方发送9个数据接收FIFO示意图

\sphinxAtStartPar
只有当接收FIFO中有数据,且在指定时间内未接收到新的数据时,才会触发超时中断。

\sphinxAtStartPar
若应用中希望通过数据间时间间隔作为帧间隔依据,即不管对方发送过来多少个数据,最后都能产生超时中断,可以通过在接收ISR中从RX FIFO中读取数据时总是少读一个(即让一个数据留在RX FIFO中)来实现。

\sphinxAtStartPar
\sphinxstylestrong{方式二:无论FIFO是否清空,间隔指定时间后均产生超时中断}
\begin{itemize}
\item {} 
\sphinxAtStartPar
配置FIFO寄存器RXLVL位为3,即RXThreshold=3,接收FIFO取值3

\item {} 
\sphinxAtStartPar
配置CTRL寄存器RXIE位为1,即RXThresholdIEn=1,配置接收FIFO中的个数\textgreater{} RXThreshold时触发中断

\item {} 
\sphinxAtStartPar
配置TOCR寄存器TIME位为10,即TimeoutTime = 10,超时时长 = TimeoutTime/(Baudrate/10) 秒

\item {} 
\sphinxAtStartPar
配置UARTx.TOCR寄存器MODE位为1,无论FIFO是否清空,间隔指定时间后均产生超时中断

\item {} 
\sphinxAtStartPar
配置CTRL寄存器TOIE位为1,即TimeoutIEn = 1,超时中断,超过 TimeoutTime/(Baudrate/10) 秒没有在RX线上接收到数据时可触发中断

\end{itemize}

\sphinxAtStartPar
无论接收FIFO中是否有数据,只要在指定时间内未接收到新的数据时,就会触发超时中断。


\subsubsection{发送中断}
\label{\detokenize{SWM241/_u529f_u80fd_u63cf_u8ff0/UART_u63a5_u53e3_u63a7_u5236_u5668:id9}}
\sphinxAtStartPar
以如下配置为例:
\begin{itemize}
\item {} 
\sphinxAtStartPar
配置FIFO寄存器TXLVL位为3,即TXThreshold = 4,发送FIFO取值4

\item {} 
\sphinxAtStartPar
配置CTRL寄存器TXIE位为1,即TXThresholdIEn = 1,配置发送FIFO中的个数\textgreater{} TXThreshold时触发中断

\end{itemize}

\sphinxAtStartPar
\sphinxincludegraphics{{SWM241/功能描述/mediaUART接口012}.emf}

\sphinxAtStartPar
图 6‑31 发送FIFO示意图

\sphinxAtStartPar
每发送出一个数据,TX FIFO中数据个数减1,当TX FIFO中数据个数小于等于TXThreshold时,触发发送中断。

\sphinxAtStartPar
如果初始化时TX FIFO中数据个数为零,则开启发送中断后会立即触发发送中断。建议在发送FIFO填入数据后再开启发送中断。


\subsubsection{中断清除}
\label{\detokenize{SWM241/_u529f_u80fd_u63cf_u8ff0/UART_u63a5_u53e3_u63a7_u5236_u5668:id10}}
\sphinxAtStartPar
此模块中中断状态位详见寄存器中各个中断标志位属性,当其中断标志位属性为R/W1C时,如需清除此标志,需在对应标志位中写1清零(R/W1C),否则中断在开启状态下会一直进入;当其中断标志位属性为AC时,表示此中断状态位会自动清零;当其中断标志位属性为RO时,表示此标志位会随着水位的变化而改变,标志位只
与其当前状态有关,不需要清除。具体详见寄存器描述。


\subsection{寄存器映射}
\label{\detokenize{SWM241/_u529f_u80fd_u63cf_u8ff0/UART_u63a5_u53e3_u63a7_u5236_u5668:id11}}

\begin{savenotes}\sphinxattablestart
\sphinxthistablewithglobalstyle
\centering
\begin{tabular}[t]{\X{20}{100}\X{20}{100}\X{20}{100}\X{20}{100}\X{20}{100}}
\sphinxtoprule
\sphinxtableatstartofbodyhook
\sphinxAtStartPar
名称   |
&
\begin{DUlineblock}{0em}
\item[] 偏移 |
\end{DUlineblock}
&
\begin{DUlineblock}{0em}
\item[] 
\item[] |
|
\end{DUlineblock}
&
\begin{DUlineblock}{0em}
\item[] 
\end{DUlineblock}
\begin{quote}

\begin{DUlineblock}{0em}
\item[] 
\item[] 
\end{DUlineblock}
\end{quote}
&
\sphinxAtStartPar
描述                       | | | |
\\
\sphinxhline
\sphinxAtStartPar
UART0BASE: {\color{red}\bfseries{}|}0x40042000UART1BASE: {\color{red}\bfseries{}|}0x40042800UART2BASE: {\color{red}\bfseries{}|}0x40043000UART3BASE: {\color{red}\bfseries{}|}0x40043800
&
\begin{DUlineblock}{0em}
\item[] |     |     |
\end{DUlineblock}
&&&\\
\sphinxhline
\sphinxAtStartPar
DATA
&
\sphinxAtStartPar
0x00
&&
\sphinxAtStartPar
0x 00000
&
\sphinxAtStartPar
UART数据寄存器             |
\\
\sphinxhline
\sphinxAtStartPar
CTRL
&
\sphinxAtStartPar
0x04
&&
\sphinxAtStartPar
0x 00001
&
\sphinxAtStartPar
UART控制及状态寄存器       |
\\
\sphinxhline
\sphinxAtStartPar
BAUD
&
\sphinxAtStartPar
0x08
&&
\sphinxAtStartPar
0x 04000
&
\sphinxAtStartPar
UART波特率控制寄存器       |
\\
\sphinxhline
\sphinxAtStartPar
FIFO
&
\sphinxAtStartPar
0x0C
&&
\sphinxAtStartPar
0x 00000
&
\sphinxAtStartPar
UART数据队列寄存器         |
\\
\sphinxhline
\sphinxAtStartPar
LINCR
&
\sphinxAtStartPar
0x10
&&
\sphinxAtStartPar
0x 00000
&
\sphinxAtStartPar
LIN Frame控制寄存器        |
\\
\sphinxhline
\sphinxAtStartPar
CTSCR/RTSCR
&
\sphinxAtStartPar
0x14
&&
\sphinxAtStartPar
0x 00000
&
\sphinxAtStartPar
自动流控控制寄存器         |
\\
\sphinxhline
\sphinxAtStartPar
CFG
&
\sphinxAtStartPar
0x18
&&
\sphinxAtStartPar
0x 00335
&
\sphinxAtStartPar
CFG 寄存器                 |
\\
\sphinxhline
\sphinxAtStartPar
TOCR
&
\sphinxAtStartPar
0x1C
&&
\sphinxAtStartPar
0x 00000
&
\sphinxAtStartPar
接收超时控制寄存器         |
\\
\sphinxbottomrule
\end{tabular}
\sphinxtableafterendhook\par
\sphinxattableend\end{savenotes}


\subsection{寄存器描述}
\label{\detokenize{SWM241/_u529f_u80fd_u63cf_u8ff0/UART_u63a5_u53e3_u63a7_u5236_u5668:id20}}

\subsubsection{数据接口寄存器DATA}
\label{\detokenize{SWM241/_u529f_u80fd_u63cf_u8ff0/UART_u63a5_u53e3_u63a7_u5236_u5668:data}}

\begin{savenotes}\sphinxattablestart
\sphinxthistablewithglobalstyle
\centering
\begin{tabular}[t]{\X{20}{100}\X{20}{100}\X{20}{100}\X{20}{100}\X{20}{100}}
\sphinxtoprule
\sphinxtableatstartofbodyhook
\sphinxAtStartPar
寄存器 |
&
\begin{DUlineblock}{0em}
\item[] 偏移 |
\end{DUlineblock}
&
\begin{DUlineblock}{0em}
\item[] 
\item[] {\color{red}\bfseries{}|}
\end{DUlineblock}
&
\sphinxAtStartPar
复位值 |    描 | |
&
\begin{DUlineblock}{0em}
\item[] |
  |
\end{DUlineblock}
\\
\sphinxhline
\sphinxAtStartPar
DATA
&
\sphinxAtStartPar
0x00
&&
\sphinxAtStartPar
0 000000
&
\sphinxAtStartPar
UART数据寄存器             |
\\
\sphinxbottomrule
\end{tabular}
\sphinxtableafterendhook\par
\sphinxattableend\end{savenotes}


\begin{savenotes}\sphinxattablestart
\sphinxthistablewithglobalstyle
\centering
\begin{tabular}[t]{\X{12}{96}\X{12}{96}\X{12}{96}\X{12}{96}\X{12}{96}\X{12}{96}\X{12}{96}\X{12}{96}}
\sphinxtoprule
\sphinxtableatstartofbodyhook
\sphinxAtStartPar
31
&
\sphinxAtStartPar
30
&
\sphinxAtStartPar
29
&
\sphinxAtStartPar
28
&
\sphinxAtStartPar
27
&
\sphinxAtStartPar
26
&
\sphinxAtStartPar
25
&
\sphinxAtStartPar
24
\\
\sphinxhline\begin{itemize}
\item {} 
\end{itemize}
&&&&&&&\\
\sphinxhline
\sphinxAtStartPar
23
&
\sphinxAtStartPar
22
&
\sphinxAtStartPar
21
&
\sphinxAtStartPar
20
&
\sphinxAtStartPar
19
&
\sphinxAtStartPar
18
&
\sphinxAtStartPar
17
&
\sphinxAtStartPar
16
\\
\sphinxhline\begin{itemize}
\item {} 
\end{itemize}
&&&&&&&\\
\sphinxhline
\sphinxAtStartPar
15
&
\sphinxAtStartPar
14
&
\sphinxAtStartPar
13
&
\sphinxAtStartPar
12
&
\sphinxAtStartPar
11
&
\sphinxAtStartPar
10
&
\sphinxAtStartPar
9
&
\sphinxAtStartPar
8
\\
\sphinxhline\begin{itemize}
\item {} 
\end{itemize}
&&&&&&&\\
\sphinxhline
\sphinxAtStartPar
7
&
\sphinxAtStartPar
6
&
\sphinxAtStartPar
5
&
\sphinxAtStartPar
4
&
\sphinxAtStartPar
3
&
\sphinxAtStartPar
2
&
\sphinxAtStartPar
1
&
\sphinxAtStartPar
0
\\
\sphinxhline
\sphinxAtStartPar
DATA
&&&&&&&\\
\sphinxbottomrule
\end{tabular}
\sphinxtableafterendhook\par
\sphinxattableend\end{savenotes}


\begin{savenotes}\sphinxattablestart
\sphinxthistablewithglobalstyle
\centering
\begin{tabular}[t]{\X{33}{99}\X{33}{99}\X{33}{99}}
\sphinxtoprule
\sphinxtableatstartofbodyhook
\sphinxAtStartPar
位域 |
&
\sphinxAtStartPar
名称     | |
&
\sphinxAtStartPar
描述                                        | |
\\
\sphinxhline
\sphinxAtStartPar
31:11
&\begin{itemize}
\item {} 
\end{itemize}
&\begin{itemize}
\item {} 
\end{itemize}
\\
\sphinxhline
\sphinxAtStartPar
10
&
\sphinxAtStartPar
PARERR
&
\sphinxAtStartPar
当前读回的的数据是否存在校验错误,RO        |

\sphinxAtStartPar
1:存在                                     |

\sphinxAtStartPar
0:不存在                                   |
\\
\sphinxhline
\sphinxAtStartPar
9
&
\sphinxAtStartPar
VALID
&
\sphinxAtStartPar
数据有效位,RO                              |

\sphinxAtStartPar
1:DATA字段有有效的接收数据                 |

\sphinxAtStartPar
0:DATA字段无有效的接收数据                 |

\sphinxAtStartPar
当DATA字段有有效                            | 数据时,该位硬件置1,读取数据后自动清零  |
\\
\sphinxhline
\sphinxAtStartPar
8:0
&
\sphinxAtStartPar
DATA
&
\sphinxAtStartPar
UART数据位                                  |

\sphinxAtStartPar
读操作:返回缓存中接收到的数据              |

\sphinxAtStartPar
写操作:将待发送的数据写入缓存中            |
\\
\sphinxbottomrule
\end{tabular}
\sphinxtableafterendhook\par
\sphinxattableend\end{savenotes}


\subsubsection{控制及状态寄存器CTRL}
\label{\detokenize{SWM241/_u529f_u80fd_u63cf_u8ff0/UART_u63a5_u53e3_u63a7_u5236_u5668:ctrl}}

\begin{savenotes}\sphinxattablestart
\sphinxthistablewithglobalstyle
\centering
\begin{tabular}[t]{\X{20}{100}\X{20}{100}\X{20}{100}\X{20}{100}\X{20}{100}}
\sphinxtoprule
\sphinxtableatstartofbodyhook
\sphinxAtStartPar
寄存器 |
&
\begin{DUlineblock}{0em}
\item[] 偏移 |
\end{DUlineblock}
&
\begin{DUlineblock}{0em}
\item[] 
\item[] {\color{red}\bfseries{}|}
\end{DUlineblock}
&
\sphinxAtStartPar
复位值 |    描 | |
&
\begin{DUlineblock}{0em}
\item[] |
  |
\end{DUlineblock}
\\
\sphinxhline
\sphinxAtStartPar
CTRL
&
\sphinxAtStartPar
0x04
&&
\sphinxAtStartPar
0 000001
&
\sphinxAtStartPar
UART控制及状态寄存器       |
\\
\sphinxbottomrule
\end{tabular}
\sphinxtableafterendhook\par
\sphinxattableend\end{savenotes}


\begin{savenotes}\sphinxattablestart
\sphinxthistablewithglobalstyle
\centering
\begin{tabular}[t]{\X{12}{96}\X{12}{96}\X{12}{96}\X{12}{96}\X{12}{96}\X{12}{96}\X{12}{96}\X{12}{96}}
\sphinxtoprule
\sphinxtableatstartofbodyhook
\sphinxAtStartPar
31
&
\sphinxAtStartPar
30
&
\sphinxAtStartPar
29
&
\sphinxAtStartPar
28
&
\sphinxAtStartPar
27
&
\sphinxAtStartPar
26
&
\sphinxAtStartPar
25
&
\sphinxAtStartPar
24
\\
\sphinxhline\begin{itemize}
\item {} 
\end{itemize}
&&&&&&&\\
\sphinxhline
\sphinxAtStartPar
23
&
\sphinxAtStartPar
22
&
\sphinxAtStartPar
21
&
\sphinxAtStartPar
20
&
\sphinxAtStartPar
19
&
\sphinxAtStartPar
18
&
\sphinxAtStartPar
17
&
\sphinxAtStartPar
16
\\
\sphinxhline
\sphinxAtStartPar
STOP2B
&&&&&
\sphinxAtStartPar
N BIT
&
\sphinxAtStartPar
G RK
&\\
\sphinxhline
\sphinxAtStartPar
15
&
\sphinxAtStartPar
14
&
\sphinxAtStartPar
13
&
\sphinxAtStartPar
12
&
\sphinxAtStartPar
11
&
\sphinxAtStartPar
10
&
\sphinxAtStartPar
9
&
\sphinxAtStartPar
8
\\
\sphinxhline
\sphinxAtStartPar
BRKDET
&
\sphinxAtStartPar
TOIE
&\begin{itemize}
\item {} 
\end{itemize}
&&&&
\sphinxAtStartPar
EN
&\begin{itemize}
\item {} 
\end{itemize}
\\
\sphinxhline
\sphinxAtStartPar
7
&
\sphinxAtStartPar
6
&
\sphinxAtStartPar
5
&
\sphinxAtStartPar
4
&
\sphinxAtStartPar
3
&
\sphinxAtStartPar
2
&
\sphinxAtStartPar
1
&
\sphinxAtStartPar
0
\\
\sphinxhline\begin{itemize}
\item {} 
\end{itemize}
&
\sphinxAtStartPar
TXDOIE
&&&&&&\\
\sphinxbottomrule
\end{tabular}
\sphinxtableafterendhook\par
\sphinxattableend\end{savenotes}


\begin{savenotes}\sphinxattablestart
\sphinxthistablewithglobalstyle
\centering
\begin{tabular}[t]{\X{33}{99}\X{33}{99}\X{33}{99}}
\sphinxtoprule
\sphinxtableatstartofbodyhook
\sphinxAtStartPar
位域 |
&
\sphinxAtStartPar
名称     | |
&
\sphinxAtStartPar
描述                                        | |
\\
\sphinxhline
\sphinxAtStartPar
31:24
&\begin{itemize}
\item {} 
\end{itemize}
&\begin{itemize}
\item {} 
\end{itemize}
\\
\sphinxhline
\sphinxAtStartPar
23:22
&
\sphinxAtStartPar
STOP2B
&
\sphinxAtStartPar
停止位模式                                  |

\sphinxAtStartPar
00:1位                                     |

\sphinxAtStartPar
01:2位                                     |

\sphinxAtStartPar
1x:保留                                    |
\\
\sphinxhline
\sphinxAtStartPar
21:20
&
\sphinxAtStartPar
PARMD
&
\sphinxAtStartPar
奇偶校验位模式                              |

\sphinxAtStartPar
00:奇校验                                  |

\sphinxAtStartPar
01:偶校验                                  |

\sphinxAtStartPar
10:常1                                     |

\sphinxAtStartPar
11:常0                                     |
\\
\sphinxhline
\sphinxAtStartPar
19
&
\sphinxAtStartPar
PAREN
&
\sphinxAtStartPar
奇偶校验使能位                              |

\sphinxAtStartPar
1:使能                                     |

\sphinxAtStartPar
0:禁能                                     |
\\
\sphinxhline
\sphinxAtStartPar
18
&
\sphinxAtStartPar
NINEBIT
&
\sphinxAtStartPar
数据位模式                                  |

\sphinxAtStartPar
1:9位数据位                                |

\sphinxAtStartPar
0:8位数据位                                |
\\
\sphinxhline
\sphinxAtStartPar
17
&
\sphinxAtStartPar
GENBRK
&
\sphinxAtStartPar
Generate LIN Break,发送LIN Break           |

\sphinxAtStartPar
0:正常发送数据                             |

\sphinxAtStartPar
1:将UART\_TX\_OUT管脚拉低                    |
\\
\sphinxhline
\sphinxAtStartPar
16
&
\sphinxAtStartPar
BRKIE
&
\sphinxAtStartPar
LIN Break Detect 中断使能                   |

\sphinxAtStartPar
接收到BREAK时,BREAKDET状态反应到中断输出 |

\sphinxAtStartPar
0:接收到BREAK时,不产生中断信号            |
\\
\sphinxhline
\sphinxAtStartPar
15
&
\sphinxAtStartPar
BRKDET
&
\sphinxAtStartPar
LIN Break Detect,检测到LIN                 | Break,即RX线上检测到连续11位低电平         |

\sphinxAtStartPar
1:接收到BREAK                              |

\sphinxAtStartPar
0:没有接收到BREAK                          |
\\
\sphinxhline
\sphinxAtStartPar
14
&
\sphinxAtStartPar
TOIE
&
\sphinxAtStartPar
1:超时产生中断                             |

\sphinxAtStartPar
0:超时不产生中断                           |
\\
\sphinxhline
\sphinxAtStartPar
13:11
&\begin{itemize}
\item {} 
\end{itemize}
&\begin{itemize}
\item {} 
\end{itemize}
\\
\sphinxhline
\sphinxAtStartPar
10
&
\sphinxAtStartPar
LOOP
&\\
\sphinxhline
\sphinxAtStartPar
9
&
\sphinxAtStartPar
EN
&
\sphinxAtStartPar
UART模块使能位                              |

\sphinxAtStartPar
1:使能                                     |

\sphinxAtStartPar
0:禁能                                     |
\\
\sphinxhline
\sphinxAtStartPar
8:7
&\begin{itemize}
\item {} 
\end{itemize}
&\begin{itemize}
\item {} 
\end{itemize}
\\
\sphinxhline
\sphinxAtStartPar
6
&
\sphinxAtStartPar
TXDOIE
&
\sphinxAtStartPar
发送完成中断使能位                          |

\sphinxAtStartPar
1:使能                                     |

\sphinxAtStartPar
0:禁能                                     |
\\
\sphinxhline
\sphinxAtStartPar
5
&
\sphinxAtStartPar
RXOV
&
\sphinxAtStartPar
接收端FIFO溢出标志位,W1C                   |

\sphinxAtStartPar
1:接收FIFO溢出                             |

\sphinxAtStartPar
0:接收FIFO没有溢出                         |
\\
\sphinxhline
\sphinxAtStartPar
4
&
\sphinxAtStartPar
RXIE
&
\sphinxAtStartPar
接收端FIFO中断使能位                        |

\sphinxAtStartPar
1:接收FIFO达到预定的数量时产生中断         |

\sphinxAtStartPar
0:接收FIFO达到预定的数量时不产生中断       |

\sphinxAtStartPar
注                                          | FIFO中此位为0表示接收到1个数据,依次类推 |
\\
\sphinxhline
\sphinxAtStartPar
3
&
\sphinxAtStartPar
RXNE
&
\sphinxAtStartPar
接收端FIFO非空标志位,RO                    |

\sphinxAtStartPar
1:非空                                     |

\sphinxAtStartPar
0:空                                       |
\\
\sphinxhline
\sphinxAtStartPar
2
&
\sphinxAtStartPar
TXIE
&
\sphinxAtStartPar
发送端FIFO中断使能位                        |

\sphinxAtStartPar
当发送FIFO内的数据少于预定的数量时产生中断 |

\sphinxAtStartPar
0 送FIFO内的数据少于预定的数量时不产生中断 |

\sphinxAtStartPar
送FIFO中此位为0表示发送0个数据,依次类推 |
\\
\sphinxhline
\sphinxAtStartPar
1
&
\sphinxAtStartPar
TXFF
&
\sphinxAtStartPar
发送端FIFO满标志位,RO                      |

\sphinxAtStartPar
1:发送FIFO内的数据满                       |

\sphinxAtStartPar
0:发送FIFO内的数据不满                     |
\\
\sphinxhline
\sphinxAtStartPar
0
&
\sphinxAtStartPar
TXIDLE
&
\sphinxAtStartPar
发送线空闲标志位,RO                        |

\sphinxAtStartPar
1:发送线空闲                               |

\sphinxAtStartPar
0:发送线忙,正在发送数据                   |
\\
\sphinxbottomrule
\end{tabular}
\sphinxtableafterendhook\par
\sphinxattableend\end{savenotes}


\subsubsection{波特率寄存器BAUD}
\label{\detokenize{SWM241/_u529f_u80fd_u63cf_u8ff0/UART_u63a5_u53e3_u63a7_u5236_u5668:baud}}

\begin{savenotes}\sphinxattablestart
\sphinxthistablewithglobalstyle
\centering
\begin{tabular}[t]{\X{20}{100}\X{20}{100}\X{20}{100}\X{20}{100}\X{20}{100}}
\sphinxtoprule
\sphinxtableatstartofbodyhook
\sphinxAtStartPar
寄存器 |
&
\begin{DUlineblock}{0em}
\item[] 偏移 |
\end{DUlineblock}
&
\begin{DUlineblock}{0em}
\item[] 
\item[] {\color{red}\bfseries{}|}
\end{DUlineblock}
&
\sphinxAtStartPar
复位值 |    描 | |
&
\begin{DUlineblock}{0em}
\item[] |
  |
\end{DUlineblock}
\\
\sphinxhline
\sphinxAtStartPar
BAUD
&
\sphinxAtStartPar
0x08
&&
\sphinxAtStartPar
0 104000
&
\sphinxAtStartPar
UART波特率控制寄存器       |
\\
\sphinxbottomrule
\end{tabular}
\sphinxtableafterendhook\par
\sphinxattableend\end{savenotes}


\begin{savenotes}\sphinxattablestart
\sphinxthistablewithglobalstyle
\centering
\begin{tabular}[t]{\X{12}{96}\X{12}{96}\X{12}{96}\X{12}{96}\X{12}{96}\X{12}{96}\X{12}{96}\X{12}{96}}
\sphinxtoprule
\sphinxtableatstartofbodyhook
\sphinxAtStartPar
31
&
\sphinxAtStartPar
30
&
\sphinxAtStartPar
29
&
\sphinxAtStartPar
28
&
\sphinxAtStartPar
27
&
\sphinxAtStartPar
26
&
\sphinxAtStartPar
25
&
\sphinxAtStartPar
24
\\
\sphinxhline
\sphinxAtStartPar
FRAC
&&&&&&
\sphinxAtStartPar
A IT
&\\
\sphinxhline
\sphinxAtStartPar
23
&
\sphinxAtStartPar
22
&
\sphinxAtStartPar
21
&
\sphinxAtStartPar
20
&
\sphinxAtStartPar
19
&
\sphinxAtStartPar
18
&
\sphinxAtStartPar
17
&
\sphinxAtStartPar
16
\\
\sphinxhline
\sphinxAtStartPar
ABREN
&
\sphinxAtStartPar
RXIF
&&&&&&\\
\sphinxhline
\sphinxAtStartPar
15
&
\sphinxAtStartPar
14
&
\sphinxAtStartPar
13
&
\sphinxAtStartPar
12
&
\sphinxAtStartPar
11
&
\sphinxAtStartPar
10
&
\sphinxAtStartPar
9
&
\sphinxAtStartPar
8
\\
\sphinxhline
\sphinxAtStartPar
RXD
&
\sphinxAtStartPar
TXD
&&&&&&\\
\sphinxhline
\sphinxAtStartPar
7
&
\sphinxAtStartPar
6
&
\sphinxAtStartPar
5
&
\sphinxAtStartPar
4
&
\sphinxAtStartPar
3
&
\sphinxAtStartPar
2
&
\sphinxAtStartPar
1
&
\sphinxAtStartPar
0
\\
\sphinxhline
\sphinxAtStartPar
BAUD
&&&&&&&\\
\sphinxbottomrule
\end{tabular}
\sphinxtableafterendhook\par
\sphinxattableend\end{savenotes}


\begin{savenotes}\sphinxattablestart
\sphinxthistablewithglobalstyle
\centering
\begin{tabular}[t]{\X{33}{99}\X{33}{99}\X{33}{99}}
\sphinxtoprule
\sphinxtableatstartofbodyhook
\sphinxAtStartPar
位域 |
&
\sphinxAtStartPar
名称     | |
&
\sphinxAtStartPar
描述                                        | |
\\
\sphinxhline
\sphinxAtStartPar
31:28
&
\sphinxAtStartPar
FRAC
&
\sphinxAtStartPar
波特率设置                                  | 波特率分频值的小数部分),参考BAUD的设置  |
\\
\sphinxhline
\sphinxAtStartPar
27
&
\sphinxAtStartPar
TXDOIF
&
\sphinxAtStartPar
发送完成中断状态位                          |

\sphinxAtStartPar
1:中断已产生                               |

\sphinxAtStartPar
0:中断未产生                               |

\sphinxAtStartPar
RO,表示此标志位会随着水位的变              | 变,标志位只与其当前状态有关,不需要清除 |
\\
\sphinxhline
\sphinxAtStartPar
26
&
\sphinxAtStartPar
ABRERR
&
\sphinxAtStartPar
自动调节波特率时,计数器溢出中断标志,R/W1C |

\sphinxAtStartPar
1:自动调节波特率时,计数器溢出,调节失败。 |

\sphinxAtStartPar
0:自动调节波特率时,计数器没有溢出。       |
\\
\sphinxhline
\sphinxAtStartPar
25:24
&
\sphinxAtStartPar
ABRBIT
&
\sphinxAtStartPar
自动调节波特率时,检测的时间长度            |

\sphinxAtStartPar
00:1位长度                                 |

\sphinxAtStartPar
01:2位长度                                 |

\sphinxAtStartPar
10:4位长度                                 |

\sphinxAtStartPar
11:8位长度                                 |
\\
\sphinxhline
\sphinxAtStartPar
23
&
\sphinxAtStartPar
ABREN
&
\sphinxAtStartPar
1:打开波特率自动调节功能。                 |

\sphinxAtStartPar
0:关闭波特率自动调节功能。                 |

\sphinxAtStartPar
调节完成自动清零,R/W,AC                   |
\\
\sphinxhline
\sphinxAtStartPar
22
&
\sphinxAtStartPar
RXIF
&
\sphinxAtStartPar
1:接收数据缓存达到预定数量                 |

\sphinxAtStartPar
0:接收数据缓存未达到预定数量               |

\sphinxAtStartPar
RO,表示此标志位会随着水位的变              | 变,标志位只与其当前状态有关,不需要清除 |
\\
\sphinxhline
\sphinxAtStartPar
21
&
\sphinxAtStartPar
TOIF
&
\sphinxAtStartPar
1:接收数据超出TIME确定的时间               |

\sphinxAtStartPar
0:接收数据未超出TIME确定的时间             |

\sphinxAtStartPar
RO,表示此标志位会随着水位的变              | 变,标志位只与其当前状态有关,不需要清除 |

\sphinxAtStartPar
超过 TOTIME/BAUDRAUD                        |

\sphinxAtStartPar
接收到新的数据时若TOIE=1,此位由硬件置位 |
\\
\sphinxhline
\sphinxAtStartPar
20
&
\sphinxAtStartPar
TXTHRF
&
\sphinxAtStartPar
TX FIFO Threshold Flag,TX                  | FIFO中数据少于设定个数(TXLVL \textless{}=            | TXTHR)时硬件置1                            |

\sphinxAtStartPar
1:发送数据缓存达到预定数量                 |

\sphinxAtStartPar
0:发送数据缓存未达到预定数量               |

\sphinxAtStartPar
RO,表示此标志位会随着水位的变              | 变,标志位只与其当前状态有关,不需要清除 |
\\
\sphinxhline
\sphinxAtStartPar
19
&
\sphinxAtStartPar
RXTHRF
&
\sphinxAtStartPar
RX FIFO Threshold Flag,RX                  | FIFO中数据达到设定个数(RXLVL \textgreater{}             | RXTHR)时硬件置1                            |

\sphinxAtStartPar
1:接收数据缓存达到预定数量                 |

\sphinxAtStartPar
0:接收数据缓存未达到预定数量               |

\sphinxAtStartPar
RO,表示此标志位会随着水位的变              | 变,标志位只与其当前状态有关,不需要清除 |
\\
\sphinxhline
\sphinxAtStartPar
18
&
\sphinxAtStartPar
BRKIF
&
\sphinxAtStartPar
LIN Break Detect 中断标志,检测到LIN        | Break时若BRKIE=1,此位由硬件置位            |

\sphinxAtStartPar
当                                          | BREAK字符时,如果BREAKIRQON为1,该位为1  |
\\
\sphinxhline
\sphinxAtStartPar
17
&
\sphinxAtStartPar
TXIF
&
\sphinxAtStartPar
1:发送数据缓存内的数据少于预定的数量       |

\sphinxAtStartPar
0:发送数据缓存内的数据大于预定的数量       |

\sphinxAtStartPar
RO,表示此标志位会随着水位的变              | 变,标志位只与其当前状态有关,不需要清除 |
\\
\sphinxhline
\sphinxAtStartPar
16
&
\sphinxAtStartPar
RXTOIF
&
\sphinxAtStartPar
接收或超时中断标志                          |

\sphinxAtStartPar
11:中断已产生                              |

\sphinxAtStartPar
0:中断未产生                               |

\sphinxAtStartPar
RO,表示此标志位会随着水位的变              | 变,标志位只与其当前状态有关,不需要清除 |
\\
\sphinxhline
\sphinxAtStartPar
15
&
\sphinxAtStartPar
RXD
&
\sphinxAtStartPar
直接读取接收线状态,RO                      |
\\
\sphinxhline
\sphinxAtStartPar
14
&
\sphinxAtStartPar
TXD
&
\sphinxAtStartPar
直接读取发送线状态,RO                      |
\\
\sphinxhline
\sphinxAtStartPar
13:0
&
\sphinxAtStartPar
BAUD
&
\sphinxAtStartPar
用于控制UART工作的波特率                    |

\sphinxAtStartPar
得到的波特率为:系统主时钟 / (BAUD.BAUD     | *16 +BAUD.FRAC + 1)

\sphinxAtStartPar
可通过BAUD寄存器                            | C位进行波特率微调,使波特率的误差在5\%以内。 |
\\
\sphinxbottomrule
\end{tabular}
\sphinxtableafterendhook\par
\sphinxattableend\end{savenotes}


\subsubsection{数据队列寄存器FIFO}
\label{\detokenize{SWM241/_u529f_u80fd_u63cf_u8ff0/UART_u63a5_u53e3_u63a7_u5236_u5668:id27}}

\begin{savenotes}\sphinxattablestart
\sphinxthistablewithglobalstyle
\centering
\begin{tabular}[t]{\X{20}{100}\X{20}{100}\X{20}{100}\X{20}{100}\X{20}{100}}
\sphinxtoprule
\sphinxtableatstartofbodyhook
\sphinxAtStartPar
寄存器 |
&
\begin{DUlineblock}{0em}
\item[] 偏移 |
\end{DUlineblock}
&
\begin{DUlineblock}{0em}
\item[] 
\item[] {\color{red}\bfseries{}|}
\end{DUlineblock}
&
\sphinxAtStartPar
复位值 |    描 | |
&
\begin{DUlineblock}{0em}
\item[] |
  |
\end{DUlineblock}
\\
\sphinxhline
\sphinxAtStartPar
FIFO
&
\sphinxAtStartPar
0x0C
&&
\sphinxAtStartPar
0 000000
&
\sphinxAtStartPar
UART数据队列寄存器         |
\\
\sphinxbottomrule
\end{tabular}
\sphinxtableafterendhook\par
\sphinxattableend\end{savenotes}


\begin{savenotes}\sphinxattablestart
\sphinxthistablewithglobalstyle
\centering
\begin{tabular}[t]{\X{12}{96}\X{12}{96}\X{12}{96}\X{12}{96}\X{12}{96}\X{12}{96}\X{12}{96}\X{12}{96}}
\sphinxtoprule
\sphinxtableatstartofbodyhook
\sphinxAtStartPar
31
&
\sphinxAtStartPar
30
&
\sphinxAtStartPar
29
&
\sphinxAtStartPar
28
&
\sphinxAtStartPar
27
&
\sphinxAtStartPar
26
&
\sphinxAtStartPar
25
&
\sphinxAtStartPar
24
\\
\sphinxhline\begin{itemize}
\item {} 
\end{itemize}
&&&&&&&\\
\sphinxhline
\sphinxAtStartPar
23
&
\sphinxAtStartPar
22
&
\sphinxAtStartPar
21
&
\sphinxAtStartPar
20
&
\sphinxAtStartPar
19
&
\sphinxAtStartPar
18
&
\sphinxAtStartPar
17
&
\sphinxAtStartPar
16
\\
\sphinxhline\begin{itemize}
\item {} 
\end{itemize}
&&&&&&&\\
\sphinxhline
\sphinxAtStartPar
15
&
\sphinxAtStartPar
14
&
\sphinxAtStartPar
13
&
\sphinxAtStartPar
12
&
\sphinxAtStartPar
11
&
\sphinxAtStartPar
10
&
\sphinxAtStartPar
9
&
\sphinxAtStartPar
8
\\
\sphinxhline\begin{itemize}
\item {} 
\end{itemize}
&&&&&&&\\
\sphinxhline
\sphinxAtStartPar
7
&
\sphinxAtStartPar
6
&
\sphinxAtStartPar
5
&
\sphinxAtStartPar
4
&
\sphinxAtStartPar
3
&
\sphinxAtStartPar
2
&
\sphinxAtStartPar
1
&
\sphinxAtStartPar
0
\\
\sphinxhline\begin{itemize}
\item {} 
\end{itemize}
&&&&&&&\\
\sphinxbottomrule
\end{tabular}
\sphinxtableafterendhook\par
\sphinxattableend\end{savenotes}


\begin{savenotes}\sphinxattablestart
\sphinxthistablewithglobalstyle
\centering
\begin{tabular}[t]{\X{33}{99}\X{33}{99}\X{33}{99}}
\sphinxtoprule
\sphinxtableatstartofbodyhook
\sphinxAtStartPar
位域 |
&
\sphinxAtStartPar
名称     | |
&
\sphinxAtStartPar
描述                                        | |
\\
\sphinxhline
\sphinxAtStartPar
31:28
&\begin{itemize}
\item {} 
\end{itemize}
&\begin{itemize}
\item {} 
\end{itemize}
\\
\sphinxhline
\sphinxAtStartPar
27:24
&
\sphinxAtStartPar
TXTHR
&
\sphinxAtStartPar
设置发送FIFO中断(TXIF)阈值                |

\sphinxAtStartPar
当发送FIFO里的水位小于等于设置值时产生中断 |

\sphinxAtStartPar
0 送FIFO里的水位小于等于设置值时不产生中断 |
\\
\sphinxhline
\sphinxAtStartPar
23:20
&\begin{itemize}
\item {} 
\end{itemize}
&\begin{itemize}
\item {} 
\end{itemize}
\\
\sphinxhline
\sphinxAtStartPar
19:16
&
\sphinxAtStartPar
RXTHR
&
\sphinxAtStartPar
设置接收FIFO中断(RXIF)阈值                |

\sphinxAtStartPar
1:当接收FIFO里的水位大于设置值时产生中断   |

\sphinxAtStartPar
0:当接收FIFO里的水位大于设置值时不产生中断 |
\\
\sphinxhline
\sphinxAtStartPar
15:12
&\begin{itemize}
\item {} 
\end{itemize}
&\begin{itemize}
\item {} 
\end{itemize}
\\
\sphinxhline
\sphinxAtStartPar
11:8
&
\sphinxAtStartPar
TXLVL
&
\sphinxAtStartPar
发送缓存的实际水位                          |
\\
\sphinxhline
\sphinxAtStartPar
7:4
&\begin{itemize}
\item {} 
\end{itemize}
&\begin{itemize}
\item {} 
\end{itemize}
\\
\sphinxhline
\sphinxAtStartPar
3:0
&
\sphinxAtStartPar
RXLVL
&
\sphinxAtStartPar
接收缓存的实际水位                          |
\\
\sphinxbottomrule
\end{tabular}
\sphinxtableafterendhook\par
\sphinxattableend\end{savenotes}


\subsubsection{LIN Frame控制寄存器LINCR}
\label{\detokenize{SWM241/_u529f_u80fd_u63cf_u8ff0/UART_u63a5_u53e3_u63a7_u5236_u5668:lin-framelincr}}

\begin{savenotes}\sphinxattablestart
\sphinxthistablewithglobalstyle
\centering
\begin{tabular}[t]{\X{20}{100}\X{20}{100}\X{20}{100}\X{20}{100}\X{20}{100}}
\sphinxtoprule
\sphinxtableatstartofbodyhook
\sphinxAtStartPar
寄存器 |
&
\begin{DUlineblock}{0em}
\item[] 偏移 |
\end{DUlineblock}
&
\begin{DUlineblock}{0em}
\item[] 
\item[] {\color{red}\bfseries{}|}
\end{DUlineblock}
&
\sphinxAtStartPar
复位值 |    描 | |
&
\begin{DUlineblock}{0em}
\item[] |
  |
\end{DUlineblock}
\\
\sphinxhline
\sphinxAtStartPar
LINCR
&
\sphinxAtStartPar
0x10
&&
\sphinxAtStartPar
0 000000
&
\sphinxAtStartPar
LIN Frame控制寄存器        |
\\
\sphinxbottomrule
\end{tabular}
\sphinxtableafterendhook\par
\sphinxattableend\end{savenotes}


\begin{savenotes}\sphinxattablestart
\sphinxthistablewithglobalstyle
\centering
\begin{tabular}[t]{\X{12}{96}\X{12}{96}\X{12}{96}\X{12}{96}\X{12}{96}\X{12}{96}\X{12}{96}\X{12}{96}}
\sphinxtoprule
\sphinxtableatstartofbodyhook
\sphinxAtStartPar
31
&
\sphinxAtStartPar
30
&
\sphinxAtStartPar
29
&
\sphinxAtStartPar
28
&
\sphinxAtStartPar
27
&
\sphinxAtStartPar
26
&
\sphinxAtStartPar
25
&
\sphinxAtStartPar
24
\\
\sphinxhline\begin{itemize}
\item {} 
\end{itemize}
&&&&&&&\\
\sphinxhline
\sphinxAtStartPar
23
&
\sphinxAtStartPar
22
&
\sphinxAtStartPar
21
&
\sphinxAtStartPar
20
&
\sphinxAtStartPar
19
&
\sphinxAtStartPar
18
&
\sphinxAtStartPar
17
&
\sphinxAtStartPar
16
\\
\sphinxhline\begin{itemize}
\item {} 
\end{itemize}
&&&&&&&\\
\sphinxhline
\sphinxAtStartPar
15
&
\sphinxAtStartPar
14
&
\sphinxAtStartPar
13
&
\sphinxAtStartPar
12
&
\sphinxAtStartPar
11
&
\sphinxAtStartPar
10
&
\sphinxAtStartPar
9
&
\sphinxAtStartPar
8
\\
\sphinxhline\begin{itemize}
\item {} 
\end{itemize}
&&&&&&&\\
\sphinxhline
\sphinxAtStartPar
7
&
\sphinxAtStartPar
6
&
\sphinxAtStartPar
5
&
\sphinxAtStartPar
4
&
\sphinxAtStartPar
3
&
\sphinxAtStartPar
2
&
\sphinxAtStartPar
1
&
\sphinxAtStartPar
0
\\
\sphinxhline\begin{itemize}
\item {} 
\end{itemize}
&&&&
\sphinxAtStartPar
GE KIF
&
\sphinxAtStartPar
GE KIE
&
\sphinxAtStartPar
BR TIF
&
\sphinxAtStartPar
BR TIE
\\
\sphinxbottomrule
\end{tabular}
\sphinxtableafterendhook\par
\sphinxattableend\end{savenotes}


\begin{savenotes}\sphinxattablestart
\sphinxthistablewithglobalstyle
\centering
\begin{tabular}[t]{\X{33}{99}\X{33}{99}\X{33}{99}}
\sphinxtoprule
\sphinxtableatstartofbodyhook
\sphinxAtStartPar
位域 |
&
\sphinxAtStartPar
名称     | |
&
\sphinxAtStartPar
描述                                        | |
\\
\sphinxhline
\sphinxAtStartPar
31:5
&\begin{itemize}
\item {} 
\end{itemize}
&\begin{itemize}
\item {} 
\end{itemize}
\\
\sphinxhline
\sphinxAtStartPar
4
&
\sphinxAtStartPar
GENBRK
&
\sphinxAtStartPar
发送LIN Break                               |

\sphinxAtStartPar
1:发送                                     |

\sphinxAtStartPar
0:不发送                                   |

\sphinxAtStartPar
发送完成自动清零,R/W,AC                   |
\\
\sphinxhline
\sphinxAtStartPar
3
&
\sphinxAtStartPar
GENBRKIF
&
\sphinxAtStartPar
LIN Break发送完成中断状态,R/W1C            |

\sphinxAtStartPar
1:中断已产生                               |

\sphinxAtStartPar
0:中断未产生                               |
\\
\sphinxhline
\sphinxAtStartPar
2
&
\sphinxAtStartPar
GENBRKIE
&
\sphinxAtStartPar
发送LIN Break完成中断的使能                 |

\sphinxAtStartPar
1:使能                                     |

\sphinxAtStartPar
0:禁能                                     |
\\
\sphinxhline
\sphinxAtStartPar
1
&
\sphinxAtStartPar
BRKDETIF
&
\sphinxAtStartPar
检测到LIN Break中断状态,R/W1C              |

\sphinxAtStartPar
1:中断已产生                               |

\sphinxAtStartPar
0:中断未产生                               |
\\
\sphinxhline
\sphinxAtStartPar
0
&
\sphinxAtStartPar
BRKDETIE
&
\sphinxAtStartPar
检测到LIN Break中断的使能                   |

\sphinxAtStartPar
1:使能                                     |

\sphinxAtStartPar
0:禁能                                     |
\\
\sphinxbottomrule
\end{tabular}
\sphinxtableafterendhook\par
\sphinxattableend\end{savenotes}


\subsubsection{自动流控控制寄存器CTSCR/ RTSCR}
\label{\detokenize{SWM241/_u529f_u80fd_u63cf_u8ff0/UART_u63a5_u53e3_u63a7_u5236_u5668:ctscr-rtscr}}

\begin{savenotes}\sphinxattablestart
\sphinxthistablewithglobalstyle
\centering
\begin{tabular}[t]{\X{20}{100}\X{20}{100}\X{20}{100}\X{20}{100}\X{20}{100}}
\sphinxtoprule
\sphinxtableatstartofbodyhook
\sphinxAtStartPar
寄存器 |
&
\begin{DUlineblock}{0em}
\item[] 偏移 |
\end{DUlineblock}
&
\begin{DUlineblock}{0em}
\item[] 
\item[] {\color{red}\bfseries{}|}
\end{DUlineblock}
&
\sphinxAtStartPar
复位值 |    描 | |
&
\begin{DUlineblock}{0em}
\item[] |
  |
\end{DUlineblock}
\\
\sphinxhline
\sphinxAtStartPar
CTSCR/RTSCR
&
\sphinxAtStartPar
0x14
&&
\sphinxAtStartPar
0 000000
&
\sphinxAtStartPar
自动流控控制寄存器         |
\\
\sphinxbottomrule
\end{tabular}
\sphinxtableafterendhook\par
\sphinxattableend\end{savenotes}


\begin{savenotes}\sphinxattablestart
\sphinxthistablewithglobalstyle
\centering
\begin{tabular}[t]{\X{12}{96}\X{12}{96}\X{12}{96}\X{12}{96}\X{12}{96}\X{12}{96}\X{12}{96}\X{12}{96}}
\sphinxtoprule
\sphinxtableatstartofbodyhook
\sphinxAtStartPar
31
&
\sphinxAtStartPar
30
&
\sphinxAtStartPar
29
&
\sphinxAtStartPar
28
&
\sphinxAtStartPar
27
&
\sphinxAtStartPar
26
&
\sphinxAtStartPar
25
&
\sphinxAtStartPar
24
\\
\sphinxhline\begin{itemize}
\item {} 
\end{itemize}
&&&&&&&\\
\sphinxhline
\sphinxAtStartPar
23
&
\sphinxAtStartPar
22
&
\sphinxAtStartPar
21
&
\sphinxAtStartPar
20
&
\sphinxAtStartPar
19
&
\sphinxAtStartPar
18
&
\sphinxAtStartPar
17
&
\sphinxAtStartPar
16
\\
\sphinxhline\begin{itemize}
\item {} 
\end{itemize}
&&&&&&&\\
\sphinxhline
\sphinxAtStartPar
15
&
\sphinxAtStartPar
14
&
\sphinxAtStartPar
13
&
\sphinxAtStartPar
12
&
\sphinxAtStartPar
11
&
\sphinxAtStartPar
10
&
\sphinxAtStartPar
9
&
\sphinxAtStartPar
8
\\
\sphinxhline\begin{itemize}
\item {} 
\end{itemize}
&&&&&&&\\
\sphinxhline
\sphinxAtStartPar
7
&
\sphinxAtStartPar
6
&
\sphinxAtStartPar
5
&
\sphinxAtStartPar
4
&
\sphinxAtStartPar
3
&
\sphinxAtStartPar
2
&
\sphinxAtStartPar
1
&
\sphinxAtStartPar
0
\\
\sphinxhline
\sphinxAtStartPar
CTSCR\_STAT
&
\sphinxAtStartPar
RTS CR\_THR
&&&
\sphinxAtStartPar
RTS POL
&
\sphinxAtStartPar
CTS POL
&
\sphinxAtStartPar
RT \_EN
&
\sphinxAtStartPar
CT \_EN
\\
\sphinxbottomrule
\end{tabular}
\sphinxtableafterendhook\par
\sphinxattableend\end{savenotes}


\begin{savenotes}\sphinxattablestart
\sphinxthistablewithglobalstyle
\centering
\begin{tabular}[t]{\X{33}{99}\X{33}{99}\X{33}{99}}
\sphinxtoprule
\sphinxtableatstartofbodyhook
\sphinxAtStartPar
位域 |
&
\sphinxAtStartPar
名称     | |
&
\sphinxAtStartPar
描述                                        | |
\\
\sphinxhline
\sphinxAtStartPar
31:9
&\begin{itemize}
\item {} 
\end{itemize}
&\begin{itemize}
\item {} 
\end{itemize}
\\
\sphinxhline
\sphinxAtStartPar
8
&
\sphinxAtStartPar
RTSCR\_STAT
&
\sphinxAtStartPar
RTS的当前状态,RO                           |
\\
\sphinxhline
\sphinxAtStartPar
7
&
\sphinxAtStartPar
CTSCR\_STAT
&
\sphinxAtStartPar
CTS的当前状态,RO                           |
\\
\sphinxhline
\sphinxAtStartPar
6:4
&
\sphinxAtStartPar
RTSCR\_THR
&
\sphinxAtStartPar
RTS流控的触发阈值                           |

\sphinxAtStartPar
000:触发                                   | 1byte,内部缓存的剩余空间最多只剩1个BYTE |

\sphinxAtStartPar
001:触发阈                                 | bytes,内部缓存的剩余空间最多只剩2个BYTE  |

\sphinxAtStartPar
010:触发阈                                 | bytes,内部缓存的剩余空间最多只剩4个BYTE  |

\sphinxAtStartPar
011:触发阈                                 | bytes,内部缓存的剩余空间最多只剩6个BYTE  |
\\
\sphinxhline
\sphinxAtStartPar
3
&
\sphinxAtStartPar
RTSCR\_POL
&
\sphinxAtStartPar
RTS信号的极性                               |

\sphinxAtStartPar
1:高有效,rts输出高,可以接收数据          |

\sphinxAtStartPar
0:低有效,rts输出低,可以接收数据          |
\\
\sphinxhline
\sphinxAtStartPar
2
&
\sphinxAtStartPar
CTSCR\_POL
&
\sphinxAtStartPar
CTS信号的极性。                             |

\sphinxAtStartPar
1:高有效,cts输入为高,可以发送数据        |

\sphinxAtStartPar
0:低有效,cts输入为低,可以发送数据        |
\\
\sphinxhline
\sphinxAtStartPar
1
&
\sphinxAtStartPar
RTSCR\_EN
&
\sphinxAtStartPar
RTS流控使能                                 |

\sphinxAtStartPar
1:rts信号发挥流控的作用                    |

\sphinxAtStartPar
0:忽略rts                                  |
\\
\sphinxhline
\sphinxAtStartPar
0
&
\sphinxAtStartPar
CTSCR\_EN
&
\sphinxAtStartPar
CTS流控使能                                 |

\sphinxAtStartPar
1:cts信号发挥流控的作用                    |

\sphinxAtStartPar
0:忽略cts                                  |
\\
\sphinxbottomrule
\end{tabular}
\sphinxtableafterendhook\par
\sphinxattableend\end{savenotes}


\subsubsection{配置寄存器CFG}
\label{\detokenize{SWM241/_u529f_u80fd_u63cf_u8ff0/UART_u63a5_u53e3_u63a7_u5236_u5668:cfg}}

\begin{savenotes}\sphinxattablestart
\sphinxthistablewithglobalstyle
\centering
\begin{tabular}[t]{\X{20}{100}\X{20}{100}\X{20}{100}\X{20}{100}\X{20}{100}}
\sphinxtoprule
\sphinxtableatstartofbodyhook
\sphinxAtStartPar
寄存器 |
&
\begin{DUlineblock}{0em}
\item[] 偏移 |
\end{DUlineblock}
&
\begin{DUlineblock}{0em}
\item[] 
\item[] {\color{red}\bfseries{}|}
\end{DUlineblock}
&
\sphinxAtStartPar
复位值 |    描 | |
&
\begin{DUlineblock}{0em}
\item[] |
  |
\end{DUlineblock}
\\
\sphinxhline
\sphinxAtStartPar
CFG
&
\sphinxAtStartPar
0x18
&&
\sphinxAtStartPar
0 000335
&
\sphinxAtStartPar
CFG 寄存器                 |
\\
\sphinxbottomrule
\end{tabular}
\sphinxtableafterendhook\par
\sphinxattableend\end{savenotes}


\begin{savenotes}\sphinxattablestart
\sphinxthistablewithglobalstyle
\centering
\begin{tabular}[t]{\X{12}{96}\X{12}{96}\X{12}{96}\X{12}{96}\X{12}{96}\X{12}{96}\X{12}{96}\X{12}{96}}
\sphinxtoprule
\sphinxtableatstartofbodyhook
\sphinxAtStartPar
31
&
\sphinxAtStartPar
30
&
\sphinxAtStartPar
29
&
\sphinxAtStartPar
28
&
\sphinxAtStartPar
27
&
\sphinxAtStartPar
26
&
\sphinxAtStartPar
25
&
\sphinxAtStartPar
24
\\
\sphinxhline\begin{itemize}
\item {} 
\end{itemize}
&&&&&&&\\
\sphinxhline
\sphinxAtStartPar
23
&
\sphinxAtStartPar
22
&
\sphinxAtStartPar
21
&
\sphinxAtStartPar
20
&
\sphinxAtStartPar
19
&
\sphinxAtStartPar
18
&
\sphinxAtStartPar
17
&
\sphinxAtStartPar
16
\\
\sphinxhline\begin{itemize}
\item {} 
\end{itemize}
&&&&&&&\\
\sphinxhline
\sphinxAtStartPar
15
&
\sphinxAtStartPar
14
&
\sphinxAtStartPar
13
&
\sphinxAtStartPar
12
&
\sphinxAtStartPar
11
&
\sphinxAtStartPar
10
&
\sphinxAtStartPar
9
&
\sphinxAtStartPar
8
\\
\sphinxhline\begin{itemize}
\item {} 
\end{itemize}
&&&&&&
\sphinxAtStartPar
BR LEN
&\\
\sphinxhline
\sphinxAtStartPar
7
&
\sphinxAtStartPar
6
&
\sphinxAtStartPar
5
&
\sphinxAtStartPar
4
&
\sphinxAtStartPar
3
&
\sphinxAtStartPar
2
&
\sphinxAtStartPar
1
&
\sphinxAtStartPar
0
\\
\sphinxhline
\sphinxAtStartPar
BRKRXLEN
&&&&&&&\\
\sphinxbottomrule
\end{tabular}
\sphinxtableafterendhook\par
\sphinxattableend\end{savenotes}


\begin{savenotes}\sphinxattablestart
\sphinxthistablewithglobalstyle
\centering
\begin{tabular}[t]{\X{33}{99}\X{33}{99}\X{33}{99}}
\sphinxtoprule
\sphinxtableatstartofbodyhook
\sphinxAtStartPar
位域 |
&
\sphinxAtStartPar
名称     | |
&
\sphinxAtStartPar
描述                                        | |
\\
\sphinxhline
\sphinxAtStartPar
31:12
&\begin{itemize}
\item {} 
\end{itemize}
&\begin{itemize}
\item {} 
\end{itemize}
\\
\sphinxhline
\sphinxAtStartPar
11
&
\sphinxAtStartPar
TXINV
&
\sphinxAtStartPar
1:发送时电平取反                           |

\sphinxAtStartPar
0:发送时电平不取反                         |
\\
\sphinxhline
\sphinxAtStartPar
10
&
\sphinxAtStartPar
RXINV
&
\sphinxAtStartPar
1:接收时电平取反                           |

\sphinxAtStartPar
0:接收时电平不取反                         |
\\
\sphinxhline
\sphinxAtStartPar
9:6
&
\sphinxAtStartPar
BRKRXLEN
&
\sphinxAtStartPar
接收BRK的判定长度。                         |

\sphinxAtStartPar
0000:表示收到1 bit的0                      |

\sphinxAtStartPar
0001:表示收到2 bit的0                      |

\sphinxAtStartPar
N:表示收到(n+1)bit的0                    |

\sphinxAtStartPar
1111:表示收到16bit的0                      |
\\
\sphinxhline
\sphinxAtStartPar
5:2
&
\sphinxAtStartPar
BRKTXLEN
&
\sphinxAtStartPar
发送BRK的长度。                             |

\sphinxAtStartPar
0001:表示发送1bit的0                       |

\sphinxAtStartPar
0010:表示发送2bit的0                       |

\sphinxAtStartPar
n:表示发送n bit的0                         |

\sphinxAtStartPar
1111:表示发送15bit的0                      |
\\
\sphinxhline
\sphinxAtStartPar
1
&
\sphinxAtStartPar
MSBF
&
\sphinxAtStartPar
1:发送和接收时MSB在前                      |

\sphinxAtStartPar
0:发送和接收时LSB在前                      |
\\
\sphinxhline
\sphinxAtStartPar
0
&
\sphinxAtStartPar
RXEN
&
\sphinxAtStartPar
接收打开使能                                |

\sphinxAtStartPar
1:接收打开。可接收外来的数据。             |

\sphinxAtStartPar
0:接收                                     | 不能接收外来的数据。接收的数据一直是1。  |
\\
\sphinxbottomrule
\end{tabular}
\sphinxtableafterendhook\par
\sphinxattableend\end{savenotes}


\subsubsection{接收超时控制寄存器TOCR}
\label{\detokenize{SWM241/_u529f_u80fd_u63cf_u8ff0/UART_u63a5_u53e3_u63a7_u5236_u5668:tocr}}

\begin{savenotes}\sphinxattablestart
\sphinxthistablewithglobalstyle
\centering
\begin{tabular}[t]{\X{20}{100}\X{20}{100}\X{20}{100}\X{20}{100}\X{20}{100}}
\sphinxtoprule
\sphinxtableatstartofbodyhook
\sphinxAtStartPar
寄存器 |
&
\begin{DUlineblock}{0em}
\item[] 偏移 |
\end{DUlineblock}
&
\begin{DUlineblock}{0em}
\item[] 
\item[] {\color{red}\bfseries{}|}
\end{DUlineblock}
&
\sphinxAtStartPar
复位值 |    描 | |
&
\begin{DUlineblock}{0em}
\item[] |
  |
\end{DUlineblock}
\\
\sphinxhline
\sphinxAtStartPar
TOCR
&
\sphinxAtStartPar
0x1C
&&
\sphinxAtStartPar
0 000000
&
\sphinxAtStartPar
接收超时控制寄存器         |
\\
\sphinxbottomrule
\end{tabular}
\sphinxtableafterendhook\par
\sphinxattableend\end{savenotes}


\begin{savenotes}\sphinxattablestart
\sphinxthistablewithglobalstyle
\centering
\begin{tabular}[t]{\X{12}{96}\X{12}{96}\X{12}{96}\X{12}{96}\X{12}{96}\X{12}{96}\X{12}{96}\X{12}{96}}
\sphinxtoprule
\sphinxtableatstartofbodyhook
\sphinxAtStartPar
31
&
\sphinxAtStartPar
30
&
\sphinxAtStartPar
29
&
\sphinxAtStartPar
28
&
\sphinxAtStartPar
27
&
\sphinxAtStartPar
26
&
\sphinxAtStartPar
25
&
\sphinxAtStartPar
24
\\
\sphinxhline\begin{itemize}
\item {} 
\end{itemize}
&&&&&&&\\
\sphinxhline
\sphinxAtStartPar
23
&
\sphinxAtStartPar
22
&
\sphinxAtStartPar
21
&
\sphinxAtStartPar
20
&
\sphinxAtStartPar
19
&
\sphinxAtStartPar
18
&
\sphinxAtStartPar
17
&
\sphinxAtStartPar
16
\\
\sphinxhline\begin{itemize}
\item {} 
\end{itemize}
&&&&&&&\\
\sphinxhline
\sphinxAtStartPar
15
&
\sphinxAtStartPar
14
&
\sphinxAtStartPar
13
&
\sphinxAtStartPar
12
&
\sphinxAtStartPar
11
&
\sphinxAtStartPar
10
&
\sphinxAtStartPar
9
&
\sphinxAtStartPar
8
\\
\sphinxhline\begin{itemize}
\item {} 
\end{itemize}
&&&&&&&\\
\sphinxhline
\sphinxAtStartPar
7
&
\sphinxAtStartPar
6
&
\sphinxAtStartPar
5
&
\sphinxAtStartPar
4
&
\sphinxAtStartPar
3
&
\sphinxAtStartPar
2
&
\sphinxAtStartPar
1
&
\sphinxAtStartPar
0
\\
\sphinxhline
\sphinxAtStartPar
TIME
&&&&&&&\\
\sphinxbottomrule
\end{tabular}
\sphinxtableafterendhook\par
\sphinxattableend\end{savenotes}


\begin{savenotes}\sphinxattablestart
\sphinxthistablewithglobalstyle
\centering
\begin{tabular}[t]{\X{33}{99}\X{33}{99}\X{33}{99}}
\sphinxtoprule
\sphinxtableatstartofbodyhook
\sphinxAtStartPar
位域 |
&
\sphinxAtStartPar
名称     | |
&
\sphinxAtStartPar
描述                                        | |
\\
\sphinxhline
\sphinxAtStartPar
31:14
&\begin{itemize}
\item {} 
\end{itemize}
&\begin{itemize}
\item {} 
\end{itemize}
\\
\sphinxhline
\sphinxAtStartPar
13
&
\sphinxAtStartPar
IFCLR
&
\sphinxAtStartPar
超时计数器清零,写1清除中断,R/W1C          |
\\
\sphinxhline
\sphinxAtStartPar
12
&
\sphinxAtStartPar
MODE
&
\sphinxAtStartPar
1:                                         | IFO是否清空,间隔指定时间后均产生超时中断 |

\sphinxAtStartPar
0:FIFO清空后,不产生超时中断               |
\\
\sphinxhline
\sphinxAtStartPar
11:0
&
\sphinxAtStartPar
TIME
&
\sphinxAtStartPar
接收数据超时中断的触发条件。                |

\sphinxAtStartPar
计时单位为10个SYMBOL TIME                   |

\sphinxAtStartPar
具体和实际波特率的设                        | 。如波特率为9600,则计时单位为1/960秒。  |
\\
\sphinxbottomrule
\end{tabular}
\sphinxtableafterendhook\par
\sphinxattableend\end{savenotes}

\sphinxstepscope


\section{I2C总线控制器(I2C)}
\label{\detokenize{SWM241/_u529f_u80fd_u63cf_u8ff0/I2C_u603b_u7ebf_u63a7_u5236_u5668:i2c-i2c}}\label{\detokenize{SWM241/_u529f_u80fd_u63cf_u8ff0/I2C_u603b_u7ebf_u63a7_u5236_u5668::doc}}
\sphinxAtStartPar
概述
\textasciitilde{}\textasciitilde{}

\sphinxAtStartPar
SWM241系列所有型号I2C操作均相同,不同型号I2C模块数量可能不同。使用前需使能对应I2C模块时钟。

\sphinxAtStartPar
I2C模块提供了MASTER模式及SLAVE模式,基本操作及配置详见功能描述章节。

\sphinxAtStartPar
特性
\textasciitilde{}\textasciitilde{}
\begin{itemize}
\item {} 
\sphinxAtStartPar
支持通过APB总线进行配置

\item {} 
\sphinxAtStartPar
支持master、slave两种模式

\item {} 
\sphinxAtStartPar
支持I2C输入信号数字滤波

\item {} 
\sphinxAtStartPar
支持Standard\sphinxhyphen{}mode(100kbps)、Fast\sphinxhyphen{}mode(400kbps)、Fast\sphinxhyphen{}mode Plus(1Mbps)、High\sphinxhyphen{}speed mode(3.4Mbps)

\item {} 
\sphinxAtStartPar
SCL/SDA线上数据可读

\item {} 
\sphinxAtStartPar
Master模式特性:
\begin{itemize}
\item {} 
\sphinxAtStartPar
支持clock synchronization

\item {} 
\sphinxAtStartPar
支持多master总线仲裁

\item {} 
\sphinxAtStartPar
支持clock stretching,slave器件可通过拉低SCL来hold总线

\item {} 
\sphinxAtStartPar
支持SCL LOW超时报警

\item {} 
\sphinxAtStartPar
支持读、写操作

\item {} 
\sphinxAtStartPar
支持发出的SCL时钟周期最大为(2\textasciicircum{}17)*pclk

\item {} 
\sphinxAtStartPar
SCL时钟占空比可配置

\end{itemize}

\item {} 
\sphinxAtStartPar
Slave模式特性:
\begin{itemize}
\item {} 
\sphinxAtStartPar
支持多slave

\item {} 
\sphinxAtStartPar
支持7位、10位两种地址模式

\item {} 
\sphinxAtStartPar
支持地址mask,一个slave器件可以占用多个地址
\begin{itemize}
\item {} 
\sphinxAtStartPar
7位地址模式,一个slave器件最多可占用128个地址

\item {} 
\sphinxAtStartPar
10位地址模式,一个slave器件最多可占用256个地址

\end{itemize}

\item {} 
\sphinxAtStartPar
支持clock stretching,slave器件可通过拉低SCL来hold总线

\item {} 
\sphinxAtStartPar
支持读、写操作

\end{itemize}

\end{itemize}


\subsection{模块结构框图}
\label{\detokenize{SWM241/_u529f_u80fd_u63cf_u8ff0/I2C_u603b_u7ebf_u63a7_u5236_u5668:id1}}
\sphinxAtStartPar
\sphinxincludegraphics{{SWM241/功能描述/mediaI2C总线控002}.emf}

\sphinxAtStartPar
图 6‑32 I2C模块结构框图

\sphinxAtStartPar
\sphinxstyleemphasis{注:I2CxDATA主机模式下是MSTDAT,从机模式下写入、读出时分别是SLVTX、SLVRX}


\subsection{功能描述}
\label{\detokenize{SWM241/_u529f_u80fd_u63cf_u8ff0/I2C_u603b_u7ebf_u63a7_u5236_u5668:id2}}
\sphinxAtStartPar
总线设置

\sphinxAtStartPar
I2C总线采用串行数据线(SDA)和串行时钟线(SCL)传输数据。I2C总线的设备端口为开漏输出,必须在接口外接上拉电阻。

\sphinxAtStartPar
数据在主从设备之间通过SCL时钟信号在SDA数据线上逐字节同步传输。每一个SCL时钟脉冲发送一位数据,高位在前。每发送一个字节的数据产生一个应答信号。在时钟线SCL高电平期间对数据的每一位进行采样。数据线SDA在时钟线SCL为低改变,在时钟线SCL为高电平时保持稳定。

\sphinxAtStartPar
协议介绍

\sphinxAtStartPar
通常情况下,一个标准的通信包含四个部分:开始信号、从机地址、数据传输、停止信号。如图 6‑33所示:

\sphinxAtStartPar
\sphinxincludegraphics{{SWM241/功能描述/mediaI2C总线控003}.emf}

\sphinxAtStartPar
图 6‑33 I2C通信示意图

\sphinxAtStartPar
起始位发送

\sphinxAtStartPar
当总线空闲时,表示没有主机设备占用总线(SCL和SDA都保持高电平),主机可以通过发送一个起始信号启动传输。启动信号,通常被称为S位。SCL为高电平时,SDA由高电平向低电平跳变。启动信号表示开始新的数据传输。

\sphinxAtStartPar
重新启动是没有先产生一个停止信号的启动信号。主机使用此方法与另一个从机或者在不释放总线的情况下与相同的从机改变数据传输方向(例如从写入设备到写入设备的转换)。

\sphinxAtStartPar
当命令寄存器的STA位被置位,同时RD或者WR位被置位时,系统核心产生一个启动信号。根据SCLK的当前的不同状态,生成启动信号或重复启动信号。

\sphinxAtStartPar
地址发送

\sphinxAtStartPar
在开始信号后,由主机传输的第一个字节数据是从机地址。包含7位的从设备地址和1位的RW指示位。RW指示位信号表示与从机的数据传输方向。在系统中的从机不可以具有相同的地址。只有从机地址和主机发送的地址匹配时才能产生一个应答位(在第九个时钟周期拉低SDA)进行响应。对于10位从机地址,模块通过产生两个从机
地址支持。

\sphinxAtStartPar
发送从机地址为一次写操作,在传输寄存器中保存从机地址并对WR位置位,从机地址将被发送到总线上。

\sphinxAtStartPar
数据发送

\sphinxAtStartPar
一旦成功取得了从机地址,主机就可以通过R/W位控制逐字节的发送数据。每传输一个字节都需要在第九个时钟周期产生一个应答位。

\sphinxAtStartPar
如果从机信号无效,主机可以生成一个停止信号中止数据传输或生成重复启动的信号并开始一个新的传输周期。如果从机返回一个NACK信号,主机就会产生一个停止信号放弃数据传输,或者产生一个重新启动信号开始一个新的传输周期。

\sphinxAtStartPar
如果主机作为接收设备,没有应答从机,从机就会释放SDA,主机产生停止信号或者重新启动信号。

\sphinxAtStartPar
向从机写入数据,需把将要发送的数据存入传输寄存器中并设置WR位。从从机中读取数据,需设置RD位。在数据传输过程中系统核心设置TIP提示标志,指示传输正在进行。当传输完成后TIP提示标志会自动清除。当中断使能时,中断标志位IF被置位,并产生中断。当中断标志位IF被置位后,接收寄存器收到有效数据。当TI
P提示标志复位后,用户可以发出新的写入或读取命令。

\sphinxAtStartPar
停止位发送

\sphinxAtStartPar
主机可以通过生成一个停止信号终止通信。停止信号通常被称为P位,被定义为SCL为高电平时,SDA由低电平向高电平跳变。


\subsubsection{Master SCL周期配置}
\label{\detokenize{SWM241/_u529f_u80fd_u63cf_u8ff0/I2C_u603b_u7ebf_u63a7_u5236_u5668:master-scl}}
\sphinxAtStartPar
\sphinxincludegraphics{{SWM241/功能描述/mediaI2C总线控004}.emf}

\sphinxAtStartPar
图 6‑34 Master SCL周期配置示意图


\subsubsection{主机发送模式}
\label{\detokenize{SWM241/_u529f_u80fd_u63cf_u8ff0/I2C_u603b_u7ebf_u63a7_u5236_u5668:id3}}
\sphinxAtStartPar
I2C模块作为主机,初始化配置操作如下:
\begin{itemize}
\item {} 
\sphinxAtStartPar
配置PORTCON模块中端口对应PORTX\_FUNC寄存器,将指定引脚切换为功能复用

\item {} 
\sphinxAtStartPar
配置PORTCON模块中端口对应PULLU\_x上拉使能寄存器,使能端口内部上拉电阻(也可使用外部上拉电阻)

\item {} 
\sphinxAtStartPar
配置PORTCON模块中端口对应INEN\_x输入使能寄存器,使能I2C数据线输入功能

\item {} 
\sphinxAtStartPar
配置CR寄存器的EN位,关闭I2C模块,确保配置寄存器过程中模块未工作

\item {} 
\sphinxAtStartPar
配置CR寄存器的MASTER位,将I2C模块设置为主机模式

\item {} 
\sphinxAtStartPar
配置CR寄存器的EN位,I2C模块总线使能

\item {} 
\sphinxAtStartPar
设置时序配置寄存器CLK,假设pclk=48M,希望I2C工作在Standard\sphinxhyphen{}mode(100kbps)速度下,则每个SCL 480个pclk,可以设置SCLL=0Xa0,SCLH =0x50,DIV=0x01

\item {} 
\sphinxAtStartPar
查询SR.BUSY,如果为1,则等待直至其变为0;如果为0,则进行下一步

\item {} 
\sphinxAtStartPar
发送Start。设置MCR.STA=1,查询该位,直至其变为0

\item {} 
\sphinxAtStartPar
发slave地址字节

\item {} 
\sphinxAtStartPar
设置TXDATA为【7位slave地址字节左移一位】

\item {} 
\sphinxAtStartPar
设置MCR.WR=1,查询该位,直至其变为0(或查询到IF的TXDONE=1(发送成功)或AL=1(仲裁丢失总线),并写1清除)

\item {} 
\sphinxAtStartPar
如果TXDONE=1,读TR.RXACK,如果该位为0,表示slave地址匹配成功

\item {} 
\sphinxAtStartPar
如果AL=1,表示本master失去总线,不能再进行后续操作,需重新查询SR.BUSY位直至1,才可以重新发送Start位,重新申请总线操作
\begin{itemize}
\item {} 
\sphinxAtStartPar
向slave发送待写数据

\end{itemize}

\item {} 
\sphinxAtStartPar
设置TXDATA,准备待写入slave的数据

\item {} 
\sphinxAtStartPar
设置MCR.WR=1,查询该位,直至其变为0(或查询到RIST的TXDONE=1,并写1清除)

\item {} 
\sphinxAtStartPar
读TR.RXACK,如果该位为0,表示写数据成功
\begin{itemize}
\item {} 
\sphinxAtStartPar
发STOP。设置MCR.STO=1,查询该位,直至其变为0

\end{itemize}

\end{itemize}

\sphinxAtStartPar
示意图如图 6‑35所示:

\sphinxAtStartPar
\sphinxincludegraphics{{SWM241/功能描述/mediaI2C总线控005}.emf}

\sphinxAtStartPar
图 6‑35 Master 寄存器时序示意图

\sphinxAtStartPar
\sphinxstyleemphasis{注:}\sphinxstyleemphasis{图中红色部分表示软件操作}


\subsubsection{主机接收模式}
\label{\detokenize{SWM241/_u529f_u80fd_u63cf_u8ff0/I2C_u603b_u7ebf_u63a7_u5236_u5668:id4}}
\sphinxAtStartPar
I2C作为主机接收模式,需将I2C模块设置为MASTER,初始化过程与主发送模式相同。

\sphinxAtStartPar
I2C作为主机从从机接收数据操作流程如下:
\begin{itemize}
\item {} 
\sphinxAtStartPar
配置PORTCON模块中端口对应PORTx\_FUNC寄存器,将指定引脚切换为功能复用

\item {} 
\sphinxAtStartPar
配置PORTCON模块中端口对应PULLU\_x上拉使能寄存器,使能端口内部上拉电阻(也可使用外部上拉电阻)

\item {} 
\sphinxAtStartPar
配置PORTCON模块中端口对应INEN\_x输入使能寄存器,使能I2C数据线输入功能

\item {} 
\sphinxAtStartPar
配置CR寄存器的EN位,关闭I2C模块,确保配置寄存器过程中模块未工作

\item {} 
\sphinxAtStartPar
配置CR寄存器的MASTER位,将I2C模块设置为主机模式

\item {} 
\sphinxAtStartPar
配置CR寄存器的EN位,I2C模块总线使能

\item {} 
\sphinxAtStartPar
设置时序配置寄存器CLK,假设pclk=48M,希望I2C工作在Standard\sphinxhyphen{}mode(100kbps)速度下,则每个SCL 480个pclk,可以设置SCLL=0Xa0,SCLH =0x50,DIV=0x01

\item {} 
\sphinxAtStartPar
查询SR.BUSY,如果为1,则等待直至其变为0;如果为0,则进行下一步

\item {} 
\sphinxAtStartPar
发送Start。设置MCR.STA=1,查询该位,直至其变为0

\item {} 
\sphinxAtStartPar
发slave地址字节

\item {} 
\sphinxAtStartPar
设置TXDATA为【7位slave地址字节地址右移1位】

\item {} 
\sphinxAtStartPar
设置MCR.WR=1,查询该位,直至其变为0(或查询到IF的TXDONE=1(发送成功)或AL=1(仲裁丢失总线),并写1清除)

\item {} 
\sphinxAtStartPar
如果TXDONE=1,读TR.RXACK,如果该位为0,表示slave地址匹配成功

\item {} 
\sphinxAtStartPar
如果AL=1,表示本master失去总线,不能再进行后续操作,需重新查询SR.BUSY位直至1,才可以重新发送Start位,重新申请总线操作
\begin{itemize}
\item {} 
\sphinxAtStartPar
从slave读数据

\end{itemize}

\item {} 
\sphinxAtStartPar
设置TR.TXACK=0

\item {} 
\sphinxAtStartPar
设置MCR.RD=1,查询直到IF.RXNE=1

\item {} 
\sphinxAtStartPar
读取RXDATA,得到slave数据

\item {} 
\sphinxAtStartPar
查询MCR.RD,直至其变为0(或查询到IF.RXDONE=1,并写1清除)
\begin{itemize}
\item {} 
\sphinxAtStartPar
发STOP。设置MCR.STO=1,查询该位,直至其变为0

\end{itemize}

\end{itemize}


\subsubsection{从发送模式}
\label{\detokenize{SWM241/_u529f_u80fd_u63cf_u8ff0/I2C_u603b_u7ebf_u63a7_u5236_u5668:id5}}
\sphinxAtStartPar
I2C作为从发送模式,需将I2C模块设置为SLAVE,具体软件配置操作如下:
\begin{itemize}
\item {} 
\sphinxAtStartPar
配置PORTCON模块中端口对应PORTx\_FUNC寄存器,将指定引脚切换为功能复用

\item {} 
\sphinxAtStartPar
配置PORTCON模块中端口对应PULLU\_x上拉使能寄存器,使能端口内部上拉电阻(也可使用外部上拉电阻)

\item {} 
\sphinxAtStartPar
配置PORTCON模块中端口对应INEN\_x输入使能寄存器,使能I2C数据线输入功能

\item {} 
\sphinxAtStartPar
配置CR寄存器的EN位,关闭I2C模块,确保配置寄存器过程中模块未工作

\item {} 
\sphinxAtStartPar
配置CR寄存器的MASTER位,将I2C模块设置为从机模式

\item {} 
\sphinxAtStartPar
配置CR寄存器的EN位,I2C模块总线使能

\item {} 
\sphinxAtStartPar
设置slave地址模式。SCR.ADDR10=0

\item {} 
\sphinxAtStartPar
设置slave地址SADDR

\item {} 
\sphinxAtStartPar
查询直至IF.RXSTA,表示检测到I2C总线上有start发出

\item {} 
\sphinxAtStartPar
查询直至IF.RXNE=1。表示有master选中本器件

\item {} 
\sphinxAtStartPar
如果SADDR中设置了地址mask,则读取RXDATA,判断master发送的实际地址

\item {} 
\sphinxAtStartPar
如果判断到TR.SLVRD=1,表示master希望从slave读取数据

\item {} 
\sphinxAtStartPar
准备数据,写TXDATA

\item {} 
\sphinxAtStartPar
查询直到RXDONE=1,表示之前地址匹配后,返回ACK结束

\item {} 
\sphinxAtStartPar
查询直到IF.TXE=1,就可以向TXDATA中写入新数据了

\item {} 
\sphinxAtStartPar
查询直到IF.TXDONE=1,表示数据发送完成。然后写1清除

\item {} 
\sphinxAtStartPar
查询TR.RXACK,如果为0,表示master希望继续接收数据,则可重新向TXDATA中写入数据;如果RXACK=1,表示master希望结束读操作,则设置TR.TXCLR,清除之前预准备到TXDATA中的最后一个数据。转入下一步

\item {} 
\sphinxAtStartPar
查询到IF.RXSTO,表示检测到I2C总线上有STOP发出。本次会话结束

\end{itemize}

\sphinxAtStartPar
示意图如图 6‑36所示:

\sphinxAtStartPar
\sphinxincludegraphics{{SWM241/功能描述/mediaI2C总线控006}.emf}

\sphinxAtStartPar
图 6‑36 Slave 寄存器时序示意图

\sphinxAtStartPar
\sphinxstyleemphasis{注1:图中红色部分表示软件操作}

\sphinxAtStartPar
\sphinxstyleemphasis{注2:图中t1= tLOW,由CLK寄存器设置}


\subsubsection{从接收模式}
\label{\detokenize{SWM241/_u529f_u80fd_u63cf_u8ff0/I2C_u603b_u7ebf_u63a7_u5236_u5668:id6}}
\sphinxAtStartPar
I2C作为从接收模式,需将I2C模块设置为SLAVE,操作流程如下:
\begin{itemize}
\item {} 
\sphinxAtStartPar
配置PORTCON模块中端口对应PORTx\_FUNC寄存器,将指定引脚切换为功能复用

\item {} 
\sphinxAtStartPar
配置PORTCON模块中端口对应PULLU\_x上拉使能寄存器,使能端口内部上拉电阻(也可使用外部上拉电阻)

\item {} 
\sphinxAtStartPar
配置PORTCON模块中端口对应INEN\_x输入使能寄存器,使能I2C数据线输入功能

\item {} 
\sphinxAtStartPar
配置CR寄存器的EN位,关闭I2C模块,确保配置寄存器过程中模块未工作

\item {} 
\sphinxAtStartPar
配置CR寄存器的MASTER位,将I2C模块设置为从机模式

\item {} 
\sphinxAtStartPar
配置CR寄存器的EN位,I2C模块总线使能

\item {} 
\sphinxAtStartPar
设置slave地址模式。SCR.ADDR10=0

\item {} 
\sphinxAtStartPar
设置slave地址SADDR

\item {} 
\sphinxAtStartPar
查询直至IF.RXSTA,表示检测到I2C总线上有start发出

\item {} 
\sphinxAtStartPar
查询直至IF.RXNE=1。表示有master选中本器件

\item {} 
\sphinxAtStartPar
如果SADDR中设置了地址mask,则读取RXDATA,判断master发送的实际地址

\item {} 
\sphinxAtStartPar
如果判断到TR.SLVWR=1,表示master希望向slave写入数据

\item {} 
\sphinxAtStartPar
查询直到RXDONE=1,表示之前地址匹配后,返回ACK结束。然后写1清除

\item {} 
\sphinxAtStartPar
设置TR.TXACK=0

\item {} 
\sphinxAtStartPar
查询直到IF.RXNE=1,表示slave接收到新数据,读取RXDATA

\item {} 
\sphinxAtStartPar
查询直到RXDONE=1,表示之前接收数据后,返回ACK结束。然后写1清除

\item {} 
\sphinxAtStartPar
可重复查询IF.RXNE位,继续接收数据,直到查询到IF.RXSTO,表示本次会话结束

\end{itemize}


\subsubsection{时钟延展clock stretching}
\label{\detokenize{SWM241/_u529f_u80fd_u63cf_u8ff0/I2C_u603b_u7ebf_u63a7_u5236_u5668:clock-stretching}}
\sphinxAtStartPar
clock stretching通过将SCL线拉低来暂停一个传输,直到释放SCL线为高电平,传输才继续进行。

\sphinxAtStartPar
以master\sphinxhyphen{}receiver,slave\sphinxhyphen{}transmitter为例,具体软件配置操作如下:
\begin{itemize}
\item {} 
\sphinxAtStartPar
配置PORTCON模块中端口对应PORTx\_FUNC寄存器,将指定引脚切换为功能复用

\item {} 
\sphinxAtStartPar
配置PORTCON模块中端口对应PULLU\_x上拉使能寄存器,使能端口内部上拉电阻(也可使用外部上拉电阻)

\item {} 
\sphinxAtStartPar
配置PORTCON模块中端口对应INEN\_x输入使能寄存器,使能I2C数据线输入功能

\item {} 
\sphinxAtStartPar
配置CR寄存器的EN位,关闭I2C模块,确保配置寄存器过程中模块未工作

\item {} 
\sphinxAtStartPar
配置CR寄存器的MASTER位,将I2C模块设置为主机模式

\item {} 
\sphinxAtStartPar
配置CR寄存器的EN位,I2C模块总线使能

\item {} 
\sphinxAtStartPar
设置时序配置寄存器CLK,假设pclk=48M,希望I2C工作在Standard\sphinxhyphen{}mode(100kbps)速度下,则每个SCL 480个pclk,可以设置SCLL=0Xa0,SCLH =0x50,DIV=0x01

\item {} 
\sphinxAtStartPar
查询SR.BUSY,如果为1,则等待直至其变为0;如果为0,则进行下一步

\item {} 
\sphinxAtStartPar
发送Start。设置MCR.STA=1,查询该位,直至其变为0

\item {} 
\sphinxAtStartPar
发slave地址字节

\item {} 
\sphinxAtStartPar
设置TXDATA为【7位slave地址字节左移一位】

\item {} 
\sphinxAtStartPar
设置MCR.WR=1,查询该位,直至其变为0(或查询到IF的TXDONE=1(发送成功)或AL=1(仲裁丢失总线),并写1清除)

\item {} 
\sphinxAtStartPar
如果TXDONE=1,读TR.RXACK,如果该位为0,表示slave地址匹配成功

\item {} 
\sphinxAtStartPar
如果AL=1,表示本master失去总线,不能再进行后续的步骤6\textasciitilde{}7,需查询直至SR.BUSY=1,才可以回到步骤4,重新发送Start位,重新申请总线操作
\begin{itemize}
\item {} 
\sphinxAtStartPar
向slave发送待写数据

\end{itemize}

\item {} 
\sphinxAtStartPar
设置TXDATA,准备待写入slave的数据

\item {} 
\sphinxAtStartPar
设置MCR.WR=1,查询该位,直至其变为0(或查询到RIST的TXDONE=1,并写1清除)

\item {} 
\sphinxAtStartPar
读TR.RXACK,如果该位为0,表示写数据成功
\begin{itemize}
\item {} 
\sphinxAtStartPar
发STOP。设置MCR.STO=1,查询该位,直至其变为0

\end{itemize}

\end{itemize}


\subsubsection{HS\sphinxhyphen{}MODE}
\label{\detokenize{SWM241/_u529f_u80fd_u63cf_u8ff0/I2C_u603b_u7ebf_u63a7_u5236_u5668:hs-mode}}
\sphinxAtStartPar
以master\sphinxhyphen{}transmitter为例

\sphinxAtStartPar
具体软件配置操作如下:
\begin{itemize}
\item {} 
\sphinxAtStartPar
设置CR.HS=0,以普通模式发第一个字节

\item {} 
\sphinxAtStartPar
以主机发送模式的方式,先在F/S\sphinxhyphen{}mode下发送START和master code。在此过程中,可以进行multi\sphinxhyphen{}master的总线仲裁

\item {} 
\sphinxAtStartPar
如果本master获得了总线控制权。则进行如下步骤

\item {} 
\sphinxAtStartPar
设置CR.HS=1。才可以设置为高速模式

\item {} 
\sphinxAtStartPar
设置CLK寄存器。假设pclk=60M,希望I2C工作在HS\sphinxhyphen{}mode(3.4Mbps)速度下,则每个SCL 14个pclk,可以设置SCLL=0x0A,SCLH=0x05,DIV=0x0

\item {} 
\sphinxAtStartPar
以主机发送模式的方式,以High\sphinxhyphen{}speed发送Sr和slave地址(不需要再判断IF.AL位)、写数据等

\end{itemize}

\sphinxAtStartPar
以slave\sphinxhyphen{}receiver为例

\sphinxAtStartPar
具体软件配置操作如下:
\begin{itemize}
\item {} 
\sphinxAtStartPar
根据F/S\sphinxhyphen{}mode速度设置CLK寄存器

\item {} 
\sphinxAtStartPar
设置CR.MASTER=0(slave),CR.EN=1,CR.HS=0

\item {} 
\sphinxAtStartPar
设置slave SCR.MCDE=1,等待master发送master code

\item {} 
\sphinxAtStartPar
查询直到RXNE=1,表示接收到master code

\item {} 
\sphinxAtStartPar
读取RXDATA中的数据,判断是multi\sphinxhyphen{}master中的哪一个master获得了总线。(对于single\sphinxhyphen{}master情况,可以省略此判断,但RXDATA中的数据需要读走,否则会影响后续地址和数据的接收)

\item {} 
\sphinxAtStartPar
设置HS\sphinxhyphen{}mode,后续操作在HS\sphinxhyphen{}mode下进行。设置CR.HS=1;设置SCR.MCDE=0

\item {} 
\sphinxAtStartPar
根据HS\sphinxhyphen{}mode速度设置CLK寄存器

\item {} 
\sphinxAtStartPar
设置slave地址模式及地址。设置SCR.ADDR10,并相应设置SADDR

\item {} 
\sphinxAtStartPar
查询直到IF.RXSTA=1,表示接收到Sr

\item {} 
\sphinxAtStartPar
查询直到RXNE=1,表示接收到匹配的地址

\item {} 
\sphinxAtStartPar
根据从机接收模式的操作继续后续操作,直至结束本次会话

\end{itemize}


\subsubsection{中断清除}
\label{\detokenize{SWM241/_u529f_u80fd_u63cf_u8ff0/I2C_u603b_u7ebf_u63a7_u5236_u5668:id7}}
\sphinxAtStartPar
此模块中中断状态位详见寄存器中各个中断标志位属性,当其中断标志位属性为R/W1C时,如需清除此标志,需在对应标志位中写1清零(R/W1C),否则中断在开启状态下会一直进入;当其中断标志位属性为AC时,表示此中断状态位会自动清零;当其中断标志位属性为RO时,表示此标志位会随着水位的变化而改变,标志位只
与其当前状态有关,不需要清除。具体详见寄存器描述。


\subsection{寄存器映射}
\label{\detokenize{SWM241/_u529f_u80fd_u63cf_u8ff0/I2C_u603b_u7ebf_u63a7_u5236_u5668:id8}}

\begin{savenotes}\sphinxattablestart
\sphinxthistablewithglobalstyle
\centering
\begin{tabular}[t]{\X{20}{100}\X{20}{100}\X{20}{100}\X{20}{100}\X{20}{100}}
\sphinxtoprule
\sphinxtableatstartofbodyhook
\sphinxAtStartPar
名称   |
&
\begin{DUlineblock}{0em}
\item[] 偏移 |
\end{DUlineblock}
&
\begin{DUlineblock}{0em}
\item[] 
\item[] |
|
\end{DUlineblock}
&
\begin{DUlineblock}{0em}
\item[] 
\end{DUlineblock}
\begin{quote}

\begin{DUlineblock}{0em}
\item[] 
\item[] 
\end{DUlineblock}
\end{quote}
&
\sphinxAtStartPar
描述                       | | | |
\\
\sphinxhline
\sphinxAtStartPar
I2C0BASE: {\color{red}\bfseries{}|}0x400A6000I2C1BASE: {\color{red}\bfseries{}|}0x400A6800
&
\begin{DUlineblock}{0em}
\item[] {\color{red}\bfseries{}|}
\end{DUlineblock}
&&&\\
\sphinxhline
\sphinxAtStartPar
CR
&
\sphinxAtStartPar
0x0
&&
\sphinxAtStartPar
0x 00018
&
\sphinxAtStartPar
通用配置寄存器             |
\\
\sphinxhline
\sphinxAtStartPar
SR
&
\sphinxAtStartPar
0x4
&&
\sphinxAtStartPar
0x 00000
&
\sphinxAtStartPar
通用状态寄存器             |
\\
\sphinxhline
\sphinxAtStartPar
TR
&
\sphinxAtStartPar
0x8
&&
\sphinxAtStartPar
0x 00002
&
\sphinxAtStartPar
通用传输寄存器             |
\\
\sphinxhline
\sphinxAtStartPar
RXDATA
&
\sphinxAtStartPar
0xC
&&
\sphinxAtStartPar
0x 00000
&
\sphinxAtStartPar
接收数据寄存器             |
\\
\sphinxhline
\sphinxAtStartPar
TXDATA
&
\sphinxAtStartPar
0x10
&&
\sphinxAtStartPar
0x0 \_0000
&
\sphinxAtStartPar
发送数据寄存器             |
\\
\sphinxhline
\sphinxAtStartPar
IF
&
\sphinxAtStartPar
0x14
&&
\sphinxAtStartPar
0x 00001
&
\sphinxAtStartPar
中断标志寄存器             |
\\
\sphinxhline
\sphinxAtStartPar
IE
&
\sphinxAtStartPar
0x18
&&
\sphinxAtStartPar
0x 00000
&
\sphinxAtStartPar
中断使能寄存器             |
\\
\sphinxhline
\sphinxAtStartPar
MCR
&
\sphinxAtStartPar
0x20
&&
\sphinxAtStartPar
0x0 \_0000
&
\sphinxAtStartPar
Master控制寄存器           |
\\
\sphinxhline
\sphinxAtStartPar
CLK
&
\sphinxAtStartPar
0x24
&&
\sphinxAtStartPar
0x 33F7F
&
\sphinxAtStartPar
时序配置寄存器             |
\\
\sphinxhline
\sphinxAtStartPar
SCR
&
\sphinxAtStartPar
0x30
&&
\sphinxAtStartPar
0x 00008
&
\sphinxAtStartPar
Slave控制寄存器            |
\\
\sphinxhline
\sphinxAtStartPar
SADDR
&
\sphinxAtStartPar
0x34
&&
\sphinxAtStartPar
0x 00000
&
\sphinxAtStartPar
Slave地址寄存器            |
\\
\sphinxbottomrule
\end{tabular}
\sphinxtableafterendhook\par
\sphinxattableend\end{savenotes}


\subsection{寄存器描述}
\label{\detokenize{SWM241/_u529f_u80fd_u63cf_u8ff0/I2C_u603b_u7ebf_u63a7_u5236_u5668:id15}}

\subsubsection{通用配置寄存器CR}
\label{\detokenize{SWM241/_u529f_u80fd_u63cf_u8ff0/I2C_u603b_u7ebf_u63a7_u5236_u5668:cr}}

\begin{savenotes}\sphinxattablestart
\sphinxthistablewithglobalstyle
\centering
\begin{tabular}[t]{\X{20}{100}\X{20}{100}\X{20}{100}\X{20}{100}\X{20}{100}}
\sphinxtoprule
\sphinxtableatstartofbodyhook
\sphinxAtStartPar
寄存器 |
&
\begin{DUlineblock}{0em}
\item[] 偏移 |
\end{DUlineblock}
&
\begin{DUlineblock}{0em}
\item[] 
\item[] {\color{red}\bfseries{}|}
\end{DUlineblock}
&
\sphinxAtStartPar
复位值 |    描 | |
&
\begin{DUlineblock}{0em}
\item[] |
  |
\end{DUlineblock}
\\
\sphinxhline
\sphinxAtStartPar
CR
&
\sphinxAtStartPar
0x0
&&
\sphinxAtStartPar
0 000018
&
\sphinxAtStartPar
通用配置寄存器             |
\\
\sphinxbottomrule
\end{tabular}
\sphinxtableafterendhook\par
\sphinxattableend\end{savenotes}


\begin{savenotes}\sphinxattablestart
\sphinxthistablewithglobalstyle
\centering
\begin{tabular}[t]{\X{12}{96}\X{12}{96}\X{12}{96}\X{12}{96}\X{12}{96}\X{12}{96}\X{12}{96}\X{12}{96}}
\sphinxtoprule
\sphinxtableatstartofbodyhook
\sphinxAtStartPar
31
&
\sphinxAtStartPar
30
&
\sphinxAtStartPar
29
&
\sphinxAtStartPar
28
&
\sphinxAtStartPar
27
&
\sphinxAtStartPar
26
&
\sphinxAtStartPar
25
&
\sphinxAtStartPar
24
\\
\sphinxhline\begin{itemize}
\item {} 
\end{itemize}
&&&&&&&\\
\sphinxhline
\sphinxAtStartPar
23
&
\sphinxAtStartPar
22
&
\sphinxAtStartPar
21
&
\sphinxAtStartPar
20
&
\sphinxAtStartPar
19
&
\sphinxAtStartPar
18
&
\sphinxAtStartPar
17
&
\sphinxAtStartPar
16
\\
\sphinxhline\begin{itemize}
\item {} 
\end{itemize}
&&&&&&&\\
\sphinxhline
\sphinxAtStartPar
15
&
\sphinxAtStartPar
14
&
\sphinxAtStartPar
13
&
\sphinxAtStartPar
12
&
\sphinxAtStartPar
11
&
\sphinxAtStartPar
10
&
\sphinxAtStartPar
9
&
\sphinxAtStartPar
8
\\
\sphinxhline\begin{itemize}
\item {} 
\end{itemize}
&&&&&&&\\
\sphinxhline
\sphinxAtStartPar
7
&
\sphinxAtStartPar
6
&
\sphinxAtStartPar
5
&
\sphinxAtStartPar
4
&
\sphinxAtStartPar
3
&
\sphinxAtStartPar
2
&
\sphinxAtStartPar
1
&
\sphinxAtStartPar
0
\\
\sphinxhline\begin{itemize}
\item {} 
\end{itemize}
&
\sphinxAtStartPar
DNF
&&&&
\sphinxAtStartPar
HS
&&
\sphinxAtStartPar
EN
\\
\sphinxbottomrule
\end{tabular}
\sphinxtableafterendhook\par
\sphinxattableend\end{savenotes}


\begin{savenotes}\sphinxattablestart
\sphinxthistablewithglobalstyle
\centering
\begin{tabular}[t]{\X{33}{99}\X{33}{99}\X{33}{99}}
\sphinxtoprule
\sphinxtableatstartofbodyhook
\sphinxAtStartPar
位域 |
&
\sphinxAtStartPar
名称     | |
&
\sphinxAtStartPar
描述                                        | |
\\
\sphinxhline
\sphinxAtStartPar
31:7
&\begin{itemize}
\item {} 
\end{itemize}
&\begin{itemize}
\item {} 
\end{itemize}
\\
\sphinxhline
\sphinxAtStartPar
6:3
&
\sphinxAtStartPar
DNF
&
\sphinxAtStartPar
Receive SDA、SCL数字噪声滤波(Digital Noise | Filter)                                    |

\sphinxAtStartPar
0000:滤波不使能                            |

\sphinxAtStartPar
0001:滤波使能,且滤波能力最大1个系统时钟   |

\sphinxAtStartPar
……

\sphinxAtStartPar
1111:滤波使能,且滤波能力最大15个系统时钟  |
\\
\sphinxhline
\sphinxAtStartPar
2
&
\sphinxAtStartPar
HS
&
\sphinxAtStartPar
High\sphinxhyphen{}Speed mode,仅在master模式下有效       |

\sphinxAtStartPar
0:Standard\sphinxhyphen{}mode, Fast\sphinxhyphen{}mode, Fast\sphinxhyphen{}mode      | Plus。SCL为open\sphinxhyphen{}drain输出                   |

\sphinxAtStartPar
1:High\sphinxhyphen{}Speed                               | mode。SCL为电流源上                         | 输出。Master发送STOP后,硬件自动清除本位 |
\\
\sphinxhline
\sphinxAtStartPar
1
&
\sphinxAtStartPar
MASTER
&
\sphinxAtStartPar
模式控制                                    |

\sphinxAtStartPar
0:slave模式                                |

\sphinxAtStartPar
1:master模式                               |
\\
\sphinxhline
\sphinxAtStartPar
0
&
\sphinxAtStartPar
EN
&
\sphinxAtStartPar
i2c总线使能                                 |

\sphinxAtStartPar
0:不使能                                   |

\sphinxAtStartPar
1:使能                                     |
\\
\sphinxbottomrule
\end{tabular}
\sphinxtableafterendhook\par
\sphinxattableend\end{savenotes}


\subsubsection{通用状态寄存器SR}
\label{\detokenize{SWM241/_u529f_u80fd_u63cf_u8ff0/I2C_u603b_u7ebf_u63a7_u5236_u5668:sr}}

\begin{savenotes}\sphinxattablestart
\sphinxthistablewithglobalstyle
\centering
\begin{tabular}[t]{\X{20}{100}\X{20}{100}\X{20}{100}\X{20}{100}\X{20}{100}}
\sphinxtoprule
\sphinxtableatstartofbodyhook
\sphinxAtStartPar
寄存器 |
&
\begin{DUlineblock}{0em}
\item[] 偏移 |
\end{DUlineblock}
&
\begin{DUlineblock}{0em}
\item[] 
\item[] {\color{red}\bfseries{}|}
\end{DUlineblock}
&
\sphinxAtStartPar
复位值 |    描 | |
&
\begin{DUlineblock}{0em}
\item[] |
  |
\end{DUlineblock}
\\
\sphinxhline
\sphinxAtStartPar
SR
&
\sphinxAtStartPar
0x4
&&
\sphinxAtStartPar
0 000000
&
\sphinxAtStartPar
通用状态寄存器             |
\\
\sphinxbottomrule
\end{tabular}
\sphinxtableafterendhook\par
\sphinxattableend\end{savenotes}


\begin{savenotes}\sphinxattablestart
\sphinxthistablewithglobalstyle
\centering
\begin{tabular}[t]{\X{12}{96}\X{12}{96}\X{12}{96}\X{12}{96}\X{12}{96}\X{12}{96}\X{12}{96}\X{12}{96}}
\sphinxtoprule
\sphinxtableatstartofbodyhook
\sphinxAtStartPar
31
&
\sphinxAtStartPar
30
&
\sphinxAtStartPar
29
&
\sphinxAtStartPar
28
&
\sphinxAtStartPar
27
&
\sphinxAtStartPar
26
&
\sphinxAtStartPar
25
&
\sphinxAtStartPar
24
\\
\sphinxhline\begin{itemize}
\item {} 
\end{itemize}
&&&&&&&\\
\sphinxhline
\sphinxAtStartPar
23
&
\sphinxAtStartPar
22
&
\sphinxAtStartPar
21
&
\sphinxAtStartPar
20
&
\sphinxAtStartPar
19
&
\sphinxAtStartPar
18
&
\sphinxAtStartPar
17
&
\sphinxAtStartPar
16
\\
\sphinxhline\begin{itemize}
\item {} 
\end{itemize}
&&&&&&&\\
\sphinxhline
\sphinxAtStartPar
15
&
\sphinxAtStartPar
14
&
\sphinxAtStartPar
13
&
\sphinxAtStartPar
12
&
\sphinxAtStartPar
11
&
\sphinxAtStartPar
10
&
\sphinxAtStartPar
9
&
\sphinxAtStartPar
8
\\
\sphinxhline\begin{itemize}
\item {} 
\end{itemize}
&&&&&&&\\
\sphinxhline
\sphinxAtStartPar
7
&
\sphinxAtStartPar
6
&
\sphinxAtStartPar
5
&
\sphinxAtStartPar
4
&
\sphinxAtStartPar
3
&
\sphinxAtStartPar
2
&
\sphinxAtStartPar
1
&
\sphinxAtStartPar
0
\\
\sphinxhline\begin{itemize}
\item {} 
\end{itemize}
&&&&&
\sphinxAtStartPar
SDA
&
\sphinxAtStartPar
SCL
&\\
\sphinxbottomrule
\end{tabular}
\sphinxtableafterendhook\par
\sphinxattableend\end{savenotes}


\begin{savenotes}\sphinxattablestart
\sphinxthistablewithglobalstyle
\centering
\begin{tabular}[t]{\X{33}{99}\X{33}{99}\X{33}{99}}
\sphinxtoprule
\sphinxtableatstartofbodyhook
\sphinxAtStartPar
位域 |
&
\sphinxAtStartPar
名称     | |
&
\sphinxAtStartPar
描述                                        | |
\\
\sphinxhline
\sphinxAtStartPar
31:3
&\begin{itemize}
\item {} 
\end{itemize}
&\begin{itemize}
\item {} 
\end{itemize}
\\
\sphinxhline
\sphinxAtStartPar
2
&
\sphinxAtStartPar
SDA
&
\sphinxAtStartPar
I2C SDA状态。不受I2C总线使能影响。          |

\sphinxAtStartPar
0:I2C SDA为低。                            |

\sphinxAtStartPar
1:I2C SDA为高。                            |
\\
\sphinxhline
\sphinxAtStartPar
1
&
\sphinxAtStartPar
SCL
&
\sphinxAtStartPar
I2C SCL状态。不受I2C总线使能影响。          |

\sphinxAtStartPar
0:I2C SCL为低。                            |

\sphinxAtStartPar
1:I2C SCL为高。                            |
\\
\sphinxhline
\sphinxAtStartPar
0
&
\sphinxAtStartPar
BUSY
&
\sphinxAtStartPar
总线忙状态。本位不受C                       | N位控制,当EN不使能时,仍然检测总线忙状态。 |

\sphinxAtStartPar
0:总线不忙。                               |

\sphinxAtStartPar
1:总线忙,I2C总线START至STOP期间有效。     |
\\
\sphinxbottomrule
\end{tabular}
\sphinxtableafterendhook\par
\sphinxattableend\end{savenotes}


\subsubsection{通用传输寄存器TR}
\label{\detokenize{SWM241/_u529f_u80fd_u63cf_u8ff0/I2C_u603b_u7ebf_u63a7_u5236_u5668:tr}}

\begin{savenotes}\sphinxattablestart
\sphinxthistablewithglobalstyle
\centering
\begin{tabular}[t]{\X{20}{100}\X{20}{100}\X{20}{100}\X{20}{100}\X{20}{100}}
\sphinxtoprule
\sphinxtableatstartofbodyhook
\sphinxAtStartPar
寄存器 |
&
\begin{DUlineblock}{0em}
\item[] 偏移 |
\end{DUlineblock}
&
\begin{DUlineblock}{0em}
\item[] 
\item[] {\color{red}\bfseries{}|}
\end{DUlineblock}
&
\sphinxAtStartPar
复位值 |    描 | |
&
\begin{DUlineblock}{0em}
\item[] |
  |
\end{DUlineblock}
\\
\sphinxhline
\sphinxAtStartPar
TR
&
\sphinxAtStartPar
0x8
&&
\sphinxAtStartPar
0 000002
&
\sphinxAtStartPar
通用传输寄存器             |
\\
\sphinxbottomrule
\end{tabular}
\sphinxtableafterendhook\par
\sphinxattableend\end{savenotes}


\begin{savenotes}\sphinxattablestart
\sphinxthistablewithglobalstyle
\centering
\begin{tabular}[t]{\X{12}{96}\X{12}{96}\X{12}{96}\X{12}{96}\X{12}{96}\X{12}{96}\X{12}{96}\X{12}{96}}
\sphinxtoprule
\sphinxtableatstartofbodyhook
\sphinxAtStartPar
31
&
\sphinxAtStartPar
30
&
\sphinxAtStartPar
29
&
\sphinxAtStartPar
28
&
\sphinxAtStartPar
27
&
\sphinxAtStartPar
26
&
\sphinxAtStartPar
25
&
\sphinxAtStartPar
24
\\
\sphinxhline\begin{itemize}
\item {} 
\end{itemize}
&&&&&&&\\
\sphinxhline
\sphinxAtStartPar
23
&
\sphinxAtStartPar
22
&
\sphinxAtStartPar
21
&
\sphinxAtStartPar
20
&
\sphinxAtStartPar
19
&
\sphinxAtStartPar
18
&
\sphinxAtStartPar
17
&
\sphinxAtStartPar
16
\\
\sphinxhline\begin{itemize}
\item {} 
\end{itemize}
&&&&&&&\\
\sphinxhline
\sphinxAtStartPar
15
&
\sphinxAtStartPar
14
&
\sphinxAtStartPar
13
&
\sphinxAtStartPar
12
&
\sphinxAtStartPar
11
&
\sphinxAtStartPar
10
&
\sphinxAtStartPar
9
&
\sphinxAtStartPar
8
\\
\sphinxhline\begin{itemize}
\item {} 
\end{itemize}
&&
\sphinxAtStartPar
S DS
&&&&&\\
\sphinxhline
\sphinxAtStartPar
7
&
\sphinxAtStartPar
6
&
\sphinxAtStartPar
5
&
\sphinxAtStartPar
4
&
\sphinxAtStartPar
3
&
\sphinxAtStartPar
2
&
\sphinxAtStartPar
1
&
\sphinxAtStartPar
0
\\
\sphinxhline\begin{itemize}
\item {} 
\end{itemize}
&&&&&&&\\
\sphinxbottomrule
\end{tabular}
\sphinxtableafterendhook\par
\sphinxattableend\end{savenotes}


\begin{savenotes}\sphinxattablestart
\sphinxthistablewithglobalstyle
\centering
\begin{tabular}[t]{\X{33}{99}\X{33}{99}\X{33}{99}}
\sphinxtoprule
\sphinxtableatstartofbodyhook
\sphinxAtStartPar
位域 |
&
\sphinxAtStartPar
名称     | |
&
\sphinxAtStartPar
描述                                        | |
\\
\sphinxhline
\sphinxAtStartPar
31:14
&\begin{itemize}
\item {} 
\end{itemize}
&\begin{itemize}
\item {} 
\end{itemize}
\\
\sphinxhline
\sphinxAtStartPar
13:12
&
\sphinxAtStartPar
SLVRDS
&
\sphinxAtStartPar
Slave接收到的数据类型。仅在Slave模式有效。  |

\sphinxAtStartPar
00:RXDATA为空。                            |

\sphinxAtStartPar
01:接收到的是地址。                        |

\sphinxAtStartPar
10:接收到的是数据。                        |

\sphinxAtStartPar
11:接收到的是master                        | code。仅当MCDE=1时有效。                    |
\\
\sphinxhline
\sphinxAtStartPar
11
&
\sphinxAtStartPar
SLVSTR
&
\sphinxAtStartPar
Slave clock stretching忙状态。仅在slave模式有效。       |

\sphinxAtStartPar
0:无clock stretching。                     |

\sphinxAtStartPar
1:有clock stretching。                     |
\\
\sphinxhline
\sphinxAtStartPar
10
&
\sphinxAtStartPar
SLVWR
&
\sphinxAtStartPar
Slave写状态。仅在slave模式有效。            |

\sphinxAtStartPar
1:Slave接收到master的写请求后有效。        |

\sphinxAtStartPar
0:                                         | ve接收到master的读请求或STOP后,自动清除。  |
\\
\sphinxhline
\sphinxAtStartPar
9
&
\sphinxAtStartPar
SLVRD
&
\sphinxAtStartPar
Slave读状态。仅在slave模式有效。            |

\sphinxAtStartPar
1:Slave接收到master的读请求后有效。        |

\sphinxAtStartPar
0:                                         | ve接收到master的写请求或STOP后,自动清除。  |
\\
\sphinxhline
\sphinxAtStartPar
8
&
\sphinxAtStartPar
SLVACT
&
\sphinxAtStartPar
Slave活跃状态。仅在slave模式有效。          |

\sphinxAtStartPar
0:slave器件处于非活跃状态                  |

\sphinxAtStartPar
1:s                                        | e器件处于活跃状态。地址匹配成功后本位有效; | STOP,或Sr后的地址匹配不成功,自动清除。 |
\\
\sphinxhline
\sphinxAtStartPar
7:3
&\begin{itemize}
\item {} 
\end{itemize}
&\begin{itemize}
\item {} 
\end{itemize}
\\
\sphinxhline
\sphinxAtStartPar
2
&
\sphinxAtStartPar
TXCLR
&
\sphinxAtStartPar
发送数据寄存器清空。硬件自动清除。          |

\sphinxAtStartPar
0:不清空。                                 |

\sphinxAtStartPar
1:清空TXDATA中的数据,并更新TXE位。        |
\\
\sphinxhline
\sphinxAtStartPar
1
&
\sphinxAtStartPar
RXACK
&
\sphinxAtStartPar
当作为trans                                 | ter时,接收到的ACK/NACK。硬件置位,TXDONE有 | 可查询此位;接收到Sr或STOP会将此位清零。 |

\sphinxAtStartPar
0:接收到ACK                                |

\sphinxAtStartPar
1:接收到NACK                               |
\\
\sphinxhline
\sphinxAtStartPar
0
&
\sphinxAtStartPar
TXACK
&
\sphinxAtStartPar
当作为receiver时,反馈ACK/NACK。            |

\sphinxAtStartPar
0:反馈ACK。                                |

\sphinxAtStartPar
1:反馈NACK。                               |

\sphinxAtStartPar
以下情况,ACK/NACK不由本位决定:            |

\sphinxAtStartPar
slave接收地址时,硬件自动反馈ACK/NACK。     |

\sphinxAtStartPar
slave MCDE有效,接收到master                | code时,硬件自动返回NACK。                  |

\sphinxAtStartPar
slave接收溢出时,硬件自动反馈NACK。         |
\\
\sphinxbottomrule
\end{tabular}
\sphinxtableafterendhook\par
\sphinxattableend\end{savenotes}


\subsubsection{接收数据寄存器RXDATA}
\label{\detokenize{SWM241/_u529f_u80fd_u63cf_u8ff0/I2C_u603b_u7ebf_u63a7_u5236_u5668:rxdata}}

\begin{savenotes}\sphinxattablestart
\sphinxthistablewithglobalstyle
\centering
\begin{tabular}[t]{\X{20}{100}\X{20}{100}\X{20}{100}\X{20}{100}\X{20}{100}}
\sphinxtoprule
\sphinxtableatstartofbodyhook
\sphinxAtStartPar
寄存器 |
&
\begin{DUlineblock}{0em}
\item[] 偏移 |
\end{DUlineblock}
&
\begin{DUlineblock}{0em}
\item[] 
\item[] {\color{red}\bfseries{}|}
\end{DUlineblock}
&
\sphinxAtStartPar
复位值 |    描 | |
&
\begin{DUlineblock}{0em}
\item[] |
  |
\end{DUlineblock}
\\
\sphinxhline
\sphinxAtStartPar
RXDATA
&
\sphinxAtStartPar
0xC
&&
\sphinxAtStartPar
0 000000
&
\sphinxAtStartPar
接收数据寄存器             |
\\
\sphinxbottomrule
\end{tabular}
\sphinxtableafterendhook\par
\sphinxattableend\end{savenotes}


\begin{savenotes}\sphinxattablestart
\sphinxthistablewithglobalstyle
\centering
\begin{tabular}[t]{\X{12}{96}\X{12}{96}\X{12}{96}\X{12}{96}\X{12}{96}\X{12}{96}\X{12}{96}\X{12}{96}}
\sphinxtoprule
\sphinxtableatstartofbodyhook
\sphinxAtStartPar
31
&
\sphinxAtStartPar
30
&
\sphinxAtStartPar
29
&
\sphinxAtStartPar
28
&
\sphinxAtStartPar
27
&
\sphinxAtStartPar
26
&
\sphinxAtStartPar
25
&
\sphinxAtStartPar
24
\\
\sphinxhline\begin{itemize}
\item {} 
\end{itemize}
&&&&&&&\\
\sphinxhline
\sphinxAtStartPar
23
&
\sphinxAtStartPar
22
&
\sphinxAtStartPar
21
&
\sphinxAtStartPar
20
&
\sphinxAtStartPar
19
&
\sphinxAtStartPar
18
&
\sphinxAtStartPar
17
&
\sphinxAtStartPar
16
\\
\sphinxhline\begin{itemize}
\item {} 
\end{itemize}
&&&&&&&\\
\sphinxhline
\sphinxAtStartPar
15
&
\sphinxAtStartPar
14
&
\sphinxAtStartPar
13
&
\sphinxAtStartPar
12
&
\sphinxAtStartPar
11
&
\sphinxAtStartPar
10
&
\sphinxAtStartPar
9
&
\sphinxAtStartPar
8
\\
\sphinxhline\begin{itemize}
\item {} 
\end{itemize}
&&&&&&&\\
\sphinxhline
\sphinxAtStartPar
7
&
\sphinxAtStartPar
6
&
\sphinxAtStartPar
5
&
\sphinxAtStartPar
4
&
\sphinxAtStartPar
3
&
\sphinxAtStartPar
2
&
\sphinxAtStartPar
1
&
\sphinxAtStartPar
0
\\
\sphinxhline
\sphinxAtStartPar
RXDATA
&&&&&&&\\
\sphinxbottomrule
\end{tabular}
\sphinxtableafterendhook\par
\sphinxattableend\end{savenotes}


\begin{savenotes}\sphinxattablestart
\sphinxthistablewithglobalstyle
\centering
\begin{tabular}[t]{\X{33}{99}\X{33}{99}\X{33}{99}}
\sphinxtoprule
\sphinxtableatstartofbodyhook
\sphinxAtStartPar
位域 |
&
\sphinxAtStartPar
名称     | |
&
\sphinxAtStartPar
描述                                        | |
\\
\sphinxhline
\sphinxAtStartPar
31:8
&\begin{itemize}
\item {} 
\end{itemize}
&\begin{itemize}
\item {} 
\end{itemize}
\\
\sphinxhline
\sphinxAtStartPar
7:0
&
\sphinxAtStartPar
RXDATA
&
\sphinxAtStartPar
接收数据                                    | 。RXNE为1,表示本寄存器中存在有效数据。  |

\sphinxAtStartPar
在完成数据接                                | 包含ACK/NACK发送)的时刻,更新此寄存器。 |

\sphinxAtStartPar
slave接收地址字节情况,参见RXDONE位说明。   |
\\
\sphinxbottomrule
\end{tabular}
\sphinxtableafterendhook\par
\sphinxattableend\end{savenotes}


\subsubsection{发送数据寄存器TXDATA}
\label{\detokenize{SWM241/_u529f_u80fd_u63cf_u8ff0/I2C_u603b_u7ebf_u63a7_u5236_u5668:txdata}}

\begin{savenotes}\sphinxattablestart
\sphinxthistablewithglobalstyle
\centering
\begin{tabular}[t]{\X{20}{100}\X{20}{100}\X{20}{100}\X{20}{100}\X{20}{100}}
\sphinxtoprule
\sphinxtableatstartofbodyhook
\sphinxAtStartPar
寄存器 |
&
\begin{DUlineblock}{0em}
\item[] 偏移 |
\end{DUlineblock}
&
\begin{DUlineblock}{0em}
\item[] 
\item[] {\color{red}\bfseries{}|}
\end{DUlineblock}
&
\sphinxAtStartPar
复位值 |    描 | |
&
\begin{DUlineblock}{0em}
\item[] |
  |
\end{DUlineblock}
\\
\sphinxhline
\sphinxAtStartPar
TXDATA
&
\sphinxAtStartPar
0x10
&&
\sphinxAtStartPar
0 000000
&
\sphinxAtStartPar
发送数据寄存器             |
\\
\sphinxbottomrule
\end{tabular}
\sphinxtableafterendhook\par
\sphinxattableend\end{savenotes}


\begin{savenotes}\sphinxattablestart
\sphinxthistablewithglobalstyle
\centering
\begin{tabular}[t]{\X{12}{96}\X{12}{96}\X{12}{96}\X{12}{96}\X{12}{96}\X{12}{96}\X{12}{96}\X{12}{96}}
\sphinxtoprule
\sphinxtableatstartofbodyhook
\sphinxAtStartPar
31
&
\sphinxAtStartPar
30
&
\sphinxAtStartPar
29
&
\sphinxAtStartPar
28
&
\sphinxAtStartPar
27
&
\sphinxAtStartPar
26
&
\sphinxAtStartPar
25
&
\sphinxAtStartPar
24
\\
\sphinxhline\begin{itemize}
\item {} 
\end{itemize}
&&&&&&&\\
\sphinxhline
\sphinxAtStartPar
23
&
\sphinxAtStartPar
22
&
\sphinxAtStartPar
21
&
\sphinxAtStartPar
20
&
\sphinxAtStartPar
19
&
\sphinxAtStartPar
18
&
\sphinxAtStartPar
17
&
\sphinxAtStartPar
16
\\
\sphinxhline\begin{itemize}
\item {} 
\end{itemize}
&&&&&&&\\
\sphinxhline
\sphinxAtStartPar
15
&
\sphinxAtStartPar
14
&
\sphinxAtStartPar
13
&
\sphinxAtStartPar
12
&
\sphinxAtStartPar
11
&
\sphinxAtStartPar
10
&
\sphinxAtStartPar
9
&
\sphinxAtStartPar
8
\\
\sphinxhline\begin{itemize}
\item {} 
\end{itemize}
&&&&&&&\\
\sphinxhline
\sphinxAtStartPar
7
&
\sphinxAtStartPar
6
&
\sphinxAtStartPar
5
&
\sphinxAtStartPar
4
&
\sphinxAtStartPar
3
&
\sphinxAtStartPar
2
&
\sphinxAtStartPar
1
&
\sphinxAtStartPar
0
\\
\sphinxhline
\sphinxAtStartPar
TXDATA
&&&&&&&\\
\sphinxbottomrule
\end{tabular}
\sphinxtableafterendhook\par
\sphinxattableend\end{savenotes}


\begin{savenotes}\sphinxattablestart
\sphinxthistablewithglobalstyle
\centering
\begin{tabular}[t]{\X{33}{99}\X{33}{99}\X{33}{99}}
\sphinxtoprule
\sphinxtableatstartofbodyhook
\sphinxAtStartPar
位域 |
&
\sphinxAtStartPar
名称     | |
&
\sphinxAtStartPar
描述                                        | |
\\
\sphinxhline
\sphinxAtStartPar
31:8
&\begin{itemize}
\item {} 
\end{itemize}
&\begin{itemize}
\item {} 
\end{itemize}
\\
\sphinxhline
\sphinxAtStartPar
7:0
&
\sphinxAtStartPar
TXDATA
&
\sphinxAtStartPar
发送数据                                    | 。TXE为0,表示本寄存器中存在待发送数据。 |
\\
\sphinxbottomrule
\end{tabular}
\sphinxtableafterendhook\par
\sphinxattableend\end{savenotes}


\subsubsection{中断标志寄存器IF}
\label{\detokenize{SWM241/_u529f_u80fd_u63cf_u8ff0/I2C_u603b_u7ebf_u63a7_u5236_u5668:if}}

\begin{savenotes}\sphinxattablestart
\sphinxthistablewithglobalstyle
\centering
\begin{tabular}[t]{\X{20}{100}\X{20}{100}\X{20}{100}\X{20}{100}\X{20}{100}}
\sphinxtoprule
\sphinxtableatstartofbodyhook
\sphinxAtStartPar
寄存器 |
&
\begin{DUlineblock}{0em}
\item[] 偏移 |
\end{DUlineblock}
&
\begin{DUlineblock}{0em}
\item[] 
\item[] {\color{red}\bfseries{}|}
\end{DUlineblock}
&
\sphinxAtStartPar
复位值 |    描 | |
&
\begin{DUlineblock}{0em}
\item[] |
  |
\end{DUlineblock}
\\
\sphinxhline
\sphinxAtStartPar
IF
&
\sphinxAtStartPar
0x14
&&
\sphinxAtStartPar
0 000001
&
\sphinxAtStartPar
中断标志寄存器             |
\\
\sphinxbottomrule
\end{tabular}
\sphinxtableafterendhook\par
\sphinxattableend\end{savenotes}


\begin{savenotes}\sphinxattablestart
\sphinxthistablewithglobalstyle
\centering
\begin{tabular}[t]{\X{12}{96}\X{12}{96}\X{12}{96}\X{12}{96}\X{12}{96}\X{12}{96}\X{12}{96}\X{12}{96}}
\sphinxtoprule
\sphinxtableatstartofbodyhook
\sphinxAtStartPar
31
&
\sphinxAtStartPar
30
&
\sphinxAtStartPar
29
&
\sphinxAtStartPar
28
&
\sphinxAtStartPar
27
&
\sphinxAtStartPar
26
&
\sphinxAtStartPar
25
&
\sphinxAtStartPar
24
\\
\sphinxhline\begin{itemize}
\item {} 
\end{itemize}
&&&&&&&\\
\sphinxhline
\sphinxAtStartPar
23
&
\sphinxAtStartPar
22
&
\sphinxAtStartPar
21
&
\sphinxAtStartPar
20
&
\sphinxAtStartPar
19
&
\sphinxAtStartPar
18
&
\sphinxAtStartPar
17
&
\sphinxAtStartPar
16
\\
\sphinxhline\begin{itemize}
\item {} 
\end{itemize}
&&&&&&&
\sphinxAtStartPar
AL
\\
\sphinxhline
\sphinxAtStartPar
15
&
\sphinxAtStartPar
14
&
\sphinxAtStartPar
13
&
\sphinxAtStartPar
12
&
\sphinxAtStartPar
11
&
\sphinxAtStartPar
10
&
\sphinxAtStartPar
9
&
\sphinxAtStartPar
8
\\
\sphinxhline\begin{itemize}
\item {} 
\end{itemize}
&&&&&&&\\
\sphinxhline
\sphinxAtStartPar
7
&
\sphinxAtStartPar
6
&
\sphinxAtStartPar
5
&
\sphinxAtStartPar
4
&
\sphinxAtStartPar
3
&
\sphinxAtStartPar
2
&
\sphinxAtStartPar
1
&
\sphinxAtStartPar
0
\\
\sphinxhline\begin{itemize}
\item {} 
\end{itemize}
&&&&&&&
\sphinxAtStartPar
TXE
\\
\sphinxbottomrule
\end{tabular}
\sphinxtableafterendhook\par
\sphinxattableend\end{savenotes}


\begin{savenotes}\sphinxattablestart
\sphinxthistablewithglobalstyle
\centering
\begin{tabular}[t]{\X{33}{99}\X{33}{99}\X{33}{99}}
\sphinxtoprule
\sphinxtableatstartofbodyhook
\sphinxAtStartPar
位域 |
&
\sphinxAtStartPar
名称     | |
&
\sphinxAtStartPar
描述                                        | |
\\
\sphinxhline
\sphinxAtStartPar
31:18
&\begin{itemize}
\item {} 
\end{itemize}
&\begin{itemize}
\item {} 
\end{itemize}
\\
\sphinxhline
\sphinxAtStartPar
17
&
\sphinxAtStartPar
MLTO
&
\sphinxAtStartPar
Master SCL LOW超时。写1清除。仅在master模式有效        |

\sphinxAtStartPar
0:未超时                                   |

\sphinxAtStartPar
1:超时。SCL                                | LOW时间超过1024个由CLK寄存器设置的SCL       | LOW时间                                     |

\sphinxAtStartPar
【对于golden model,SCL                     | LOW超时时间由MLTO\_LIM设置】                 |
\\
\sphinxhline
\sphinxAtStartPar
16
&
\sphinxAtStartPar
AL
&
\sphinxAtStartPar
M er仲裁丢失总线。写1清除。仅在master模式有效 |

\sphinxAtStartPar
0:无仲裁丢失总线控制权                     |

\sphinxAtStartPar
1:仲裁丢失总线控制权                       |
\\
\sphinxhline
\sphinxAtStartPar
15:10
&\begin{itemize}
\item {} 
\end{itemize}
&\begin{itemize}
\item {} 
\end{itemize}
\\
\sphinxhline
\sphinxAtStartPar
9
&
\sphinxAtStartPar
RXSTO
&
\sphinxAtStartPar
S e检测到STOP。写1清除。仅在slave模式下有效。 |

\sphinxAtStartPar
0:slave未检测到STOP                        |

\sphinxAtStartPar
1:slave检测到STOP                          |
\\
\sphinxhline
\sphinxAtStartPar
8
&
\sphinxAtStartPar
RXSTA
&\\
\sphinxhline
\sphinxAtStartPar
7:5
&\begin{itemize}
\item {} 
\end{itemize}
&\begin{itemize}
\item {} 
\end{itemize}
\\
\sphinxhline
\sphinxAtStartPar
4
&
\sphinxAtStartPar
RXDONE
&
\sphinxAtStartPar
接收结束。写1清除,包含ACK/NACK时间         |

\sphinxAtStartPar
0:接收未结束                               |

\sphinxAtStartPar
1:接收结束                                 |

\sphinxAtStartPar
Slave接收情况说明                           |

\sphinxAtStartPar
Slave器件7位地址模式下,slave地址字节       | /W位)接收完成,若地址匹配,则生成此中断  |

\sphinxAtStartPar
Slave器件10位地                             | 下,slave地址的第2字节(ADDR{[}7:0{]})接收  | 若10位地址匹配,则生成此中断;跟在repeat | START之后的slave地址第1字节,若地址8、9位   |
则生成此中断;跟在START之后的第1字节接收 | ,即使ADDR{[}9:8{]}匹配,也不会生成此中断。  |

\sphinxAtStartPar
Slave模式,MCDE=1,接收到master             | code时,会生成此中断。                      |
\\
\sphinxhline
\sphinxAtStartPar
3
&
\sphinxAtStartPar
TXDONE
&
\sphinxAtStartPar
发送结束。写1清除,包含ACK/NACK时间         |

\sphinxAtStartPar
0:发送未结束,或没有发送                   |

\sphinxAtStartPar
1:发送结束                                 |

\sphinxAtStartPar
说明:当master                              | 送字节发生仲裁丢失总线时,不产生本中断。 |
\\
\sphinxhline
\sphinxAtStartPar
2
&
\sphinxAtStartPar
RXOV
&
\sphinxAtStartPar
接收数据寄存器溢出。软件                    | 除。(更新的时刻点,不包含ACK/NACK发送)  |

\sphinxAtStartPar
0:无溢出                                   |

\sphinxAtStartPar
1:当RXDATA非空时,又接收到                 | 节,会产生溢出。溢出发生时,新数据丢失。 |

\sphinxAtStartPar
说明:对于slave模式,如果STRE位             | 当接收数据寄存器非空,且又接收到新的字节 | ave器件会拉低SCL信号,直到RXDATA中的旧数被 | 再把新数存到RXDATA中,此情况不会产生溢出 |
\\
\sphinxhline
\sphinxAtStartPar
1
&
\sphinxAtStartPar
RXNE
&
\sphinxAtStartPar
接收数据寄存器非空                          |

\sphinxAtStartPar
0:接收数据寄存器空,不存在未读取的接收数据 |

\sphinxAtStartPar
1:接收数据寄存器非空,存在未读取的接收数据 |

\sphinxAtStartPar
在接收完                                    | 时刻更新此位(不包含ACK/NACK发送时间)。 |

\sphinxAtStartPar
如果新数据接收                              | ,旧数据未及时读取,分如下几种情况处理: |

\sphinxAtStartPar
Master模式:                                |

\sphinxAtStartPar
新数据丢失。同时置位RXD\_OV位。              |

\sphinxAtStartPar
Slave模式:                                 |

\sphinxAtStartPar
A.STRE=0:新                               | 失。同时置位RXD\_OV位,硬件自动反馈NACK。 |

\sphinxAtStartPar
B.STRE=1:正常返回                         | ,然后在master发送下一个字节前,slave将SCL  | hold在低电平,直到旧数据被读走后,          | 数据更新到RXDATA寄存器中。最后释放SCL。  |
\\
\sphinxhline
\sphinxAtStartPar
0
&
\sphinxAtStartPar
TXE
&
\sphinxAtStartPar
发送数据寄存器空                            |

\sphinxAtStartPar
0:发送数据寄存器非空,不允许写TXDATA寄存器 |

\sphinxAtStartPar
1:发送数据寄存器空,允许写TXDATA寄存器     |

\sphinxAtStartPar
在发送数据开始的时刻,发送数据被            | 走后,此位被更新为1(此时TXDONE仍为0)。 |

\sphinxAtStartPar
向TXDATA寄存器写入新数据,可清除此位。      |
\\
\sphinxbottomrule
\end{tabular}
\sphinxtableafterendhook\par
\sphinxattableend\end{savenotes}


\subsubsection{中断使能寄存器IE}
\label{\detokenize{SWM241/_u529f_u80fd_u63cf_u8ff0/I2C_u603b_u7ebf_u63a7_u5236_u5668:ie}}

\begin{savenotes}\sphinxattablestart
\sphinxthistablewithglobalstyle
\centering
\begin{tabular}[t]{\X{20}{100}\X{20}{100}\X{20}{100}\X{20}{100}\X{20}{100}}
\sphinxtoprule
\sphinxtableatstartofbodyhook
\sphinxAtStartPar
寄存器 |
&
\begin{DUlineblock}{0em}
\item[] 偏移 |
\end{DUlineblock}
&
\begin{DUlineblock}{0em}
\item[] 
\item[] {\color{red}\bfseries{}|}
\end{DUlineblock}
&
\sphinxAtStartPar
复位值 |    描 | |
&
\begin{DUlineblock}{0em}
\item[] |
  |
\end{DUlineblock}
\\
\sphinxhline
\sphinxAtStartPar
IE
&
\sphinxAtStartPar
0x18
&&
\sphinxAtStartPar
0 000000
&
\sphinxAtStartPar
中断使能寄存器             |
\\
\sphinxbottomrule
\end{tabular}
\sphinxtableafterendhook\par
\sphinxattableend\end{savenotes}


\begin{savenotes}\sphinxattablestart
\sphinxthistablewithglobalstyle
\centering
\begin{tabular}[t]{\X{12}{96}\X{12}{96}\X{12}{96}\X{12}{96}\X{12}{96}\X{12}{96}\X{12}{96}\X{12}{96}}
\sphinxtoprule
\sphinxtableatstartofbodyhook
\sphinxAtStartPar
31
&
\sphinxAtStartPar
30
&
\sphinxAtStartPar
29
&
\sphinxAtStartPar
28
&
\sphinxAtStartPar
27
&
\sphinxAtStartPar
26
&
\sphinxAtStartPar
25
&
\sphinxAtStartPar
24
\\
\sphinxhline\begin{itemize}
\item {} 
\end{itemize}
&&&&&&&\\
\sphinxhline
\sphinxAtStartPar
23
&
\sphinxAtStartPar
22
&
\sphinxAtStartPar
21
&
\sphinxAtStartPar
20
&
\sphinxAtStartPar
19
&
\sphinxAtStartPar
18
&
\sphinxAtStartPar
17
&
\sphinxAtStartPar
16
\\
\sphinxhline\begin{itemize}
\item {} 
\end{itemize}
&&&&&&&
\sphinxAtStartPar
AL
\\
\sphinxhline
\sphinxAtStartPar
12
&
\sphinxAtStartPar
11
&
\sphinxAtStartPar
10
&
\sphinxAtStartPar
9
&
\sphinxAtStartPar
8
&
\sphinxAtStartPar
10
&
\sphinxAtStartPar
9
&
\sphinxAtStartPar
8
\\
\sphinxhline\begin{itemize}
\item {} 
\end{itemize}
&&&&&&&\\
\sphinxhline
\sphinxAtStartPar
7
&
\sphinxAtStartPar
6
&
\sphinxAtStartPar
5
&
\sphinxAtStartPar
4
&
\sphinxAtStartPar
3
&
\sphinxAtStartPar
2
&
\sphinxAtStartPar
1
&
\sphinxAtStartPar
0
\\
\sphinxhline\begin{itemize}
\item {} 
\end{itemize}
&&&&&&&
\sphinxAtStartPar
TXE
\\
\sphinxbottomrule
\end{tabular}
\sphinxtableafterendhook\par
\sphinxattableend\end{savenotes}


\begin{savenotes}\sphinxattablestart
\sphinxthistablewithglobalstyle
\centering
\begin{tabular}[t]{\X{33}{99}\X{33}{99}\X{33}{99}}
\sphinxtoprule
\sphinxtableatstartofbodyhook
\sphinxAtStartPar
位域 |
&
\sphinxAtStartPar
名称     | |
&
\sphinxAtStartPar
描述                                        | |
\\
\sphinxhline
\sphinxAtStartPar
31:18
&\begin{itemize}
\item {} 
\end{itemize}
&\begin{itemize}
\item {} 
\end{itemize}
\\
\sphinxhline
\sphinxAtStartPar
17
&
\sphinxAtStartPar
MLTO
&
\sphinxAtStartPar
Master SCL LOW超时中断使能                  |

\sphinxAtStartPar
0:不使能                                   |

\sphinxAtStartPar
1:使能                                     |
\\
\sphinxhline
\sphinxAtStartPar
16
&
\sphinxAtStartPar
AL
&
\sphinxAtStartPar
Master仲裁丢失总线中断使能                  |

\sphinxAtStartPar
0:不使能                                   |

\sphinxAtStartPar
1:使能                                     |
\\
\sphinxhline
\sphinxAtStartPar
15:10
&\begin{itemize}
\item {} 
\end{itemize}
&\begin{itemize}
\item {} 
\end{itemize}
\\
\sphinxhline
\sphinxAtStartPar
9
&
\sphinxAtStartPar
RXSTO
&
\sphinxAtStartPar
Slave检测到STOP中断使能                     |

\sphinxAtStartPar
0:不使能                                   |

\sphinxAtStartPar
1:使能                                     |
\\
\sphinxhline
\sphinxAtStartPar
8
&
\sphinxAtStartPar
RXSTA
&
\sphinxAtStartPar
Slave检测到START中断使能                    |

\sphinxAtStartPar
0:不使能                                   |

\sphinxAtStartPar
1:使能                                     |
\\
\sphinxhline
\sphinxAtStartPar
7:5
&\begin{itemize}
\item {} 
\end{itemize}
&\begin{itemize}
\item {} 
\end{itemize}
\\
\sphinxhline
\sphinxAtStartPar
4
&
\sphinxAtStartPar
RXDONE
&
\sphinxAtStartPar
接收数据结束中断使能                        |

\sphinxAtStartPar
0:不使能                                   |

\sphinxAtStartPar
1:使能                                     |
\\
\sphinxhline
\sphinxAtStartPar
3
&
\sphinxAtStartPar
TXDONE
&
\sphinxAtStartPar
发送数据结束中断使能                        |

\sphinxAtStartPar
0:不使能                                   |

\sphinxAtStartPar
1:使能                                     |
\\
\sphinxhline
\sphinxAtStartPar
2
&
\sphinxAtStartPar
RXOV
&
\sphinxAtStartPar
接收数据寄存器溢出中断使能                  |

\sphinxAtStartPar
0:不使能                                   |

\sphinxAtStartPar
1:使能                                     |
\\
\sphinxhline
\sphinxAtStartPar
1
&
\sphinxAtStartPar
RXNE
&
\sphinxAtStartPar
接收数据寄存器非空中断使能                  |

\sphinxAtStartPar
0:不使能                                   |

\sphinxAtStartPar
1:使能                                     |
\\
\sphinxhline
\sphinxAtStartPar
0
&
\sphinxAtStartPar
TXE
&
\sphinxAtStartPar
发送数据寄存器空中断使能                    |

\sphinxAtStartPar
0:不使能                                   |

\sphinxAtStartPar
1:使能                                     |
\\
\sphinxbottomrule
\end{tabular}
\sphinxtableafterendhook\par
\sphinxattableend\end{savenotes}


\subsubsection{Master控制寄存器MCR}
\label{\detokenize{SWM241/_u529f_u80fd_u63cf_u8ff0/I2C_u603b_u7ebf_u63a7_u5236_u5668:mastermcr}}

\begin{savenotes}\sphinxattablestart
\sphinxthistablewithglobalstyle
\centering
\begin{tabular}[t]{\X{20}{100}\X{20}{100}\X{20}{100}\X{20}{100}\X{20}{100}}
\sphinxtoprule
\sphinxtableatstartofbodyhook
\sphinxAtStartPar
寄存器 |
&
\begin{DUlineblock}{0em}
\item[] 偏移 |
\end{DUlineblock}
&
\begin{DUlineblock}{0em}
\item[] 
\item[] {\color{red}\bfseries{}|}
\end{DUlineblock}
&
\sphinxAtStartPar
复位值 |    描 | |
&
\begin{DUlineblock}{0em}
\item[] |
  |
\end{DUlineblock}
\\
\sphinxhline
\sphinxAtStartPar
MCR
&
\sphinxAtStartPar
0x20
&&
\sphinxAtStartPar
0 000000
&
\sphinxAtStartPar
Master控制寄存器           |
\\
\sphinxbottomrule
\end{tabular}
\sphinxtableafterendhook\par
\sphinxattableend\end{savenotes}


\begin{savenotes}\sphinxattablestart
\sphinxthistablewithglobalstyle
\centering
\begin{tabular}[t]{\X{12}{96}\X{12}{96}\X{12}{96}\X{12}{96}\X{12}{96}\X{12}{96}\X{12}{96}\X{12}{96}}
\sphinxtoprule
\sphinxtableatstartofbodyhook
\sphinxAtStartPar
31
&
\sphinxAtStartPar
30
&
\sphinxAtStartPar
29
&
\sphinxAtStartPar
28
&
\sphinxAtStartPar
27
&
\sphinxAtStartPar
26
&
\sphinxAtStartPar
25
&
\sphinxAtStartPar
24
\\
\sphinxhline\begin{itemize}
\item {} 
\end{itemize}
&&&&&&&\\
\sphinxhline
\sphinxAtStartPar
23
&
\sphinxAtStartPar
22
&
\sphinxAtStartPar
21
&
\sphinxAtStartPar
20
&
\sphinxAtStartPar
19
&
\sphinxAtStartPar
18
&
\sphinxAtStartPar
17
&
\sphinxAtStartPar
16
\\
\sphinxhline\begin{itemize}
\item {} 
\end{itemize}
&&&&&&&\\
\sphinxhline
\sphinxAtStartPar
15
&
\sphinxAtStartPar
14
&
\sphinxAtStartPar
13
&
\sphinxAtStartPar
12
&
\sphinxAtStartPar
11
&
\sphinxAtStartPar
10
&
\sphinxAtStartPar
9
&
\sphinxAtStartPar
8
\\
\sphinxhline\begin{itemize}
\item {} 
\end{itemize}
&&&&&&&\\
\sphinxhline
\sphinxAtStartPar
7
&
\sphinxAtStartPar
6
&
\sphinxAtStartPar
5
&
\sphinxAtStartPar
4
&
\sphinxAtStartPar
3
&
\sphinxAtStartPar
2
&
\sphinxAtStartPar
1
&
\sphinxAtStartPar
0
\\
\sphinxhline\begin{itemize}
\item {} 
\end{itemize}
&&&&
\sphinxAtStartPar
STO
&
\sphinxAtStartPar
WR
&
\sphinxAtStartPar
RD
&
\sphinxAtStartPar
STA
\\
\sphinxbottomrule
\end{tabular}
\sphinxtableafterendhook\par
\sphinxattableend\end{savenotes}


\begin{savenotes}\sphinxattablestart
\sphinxthistablewithglobalstyle
\centering
\begin{tabular}[t]{\X{33}{99}\X{33}{99}\X{33}{99}}
\sphinxtoprule
\sphinxtableatstartofbodyhook
\sphinxAtStartPar
位域 |
&
\sphinxAtStartPar
名称     | |
&
\sphinxAtStartPar
描述                                        | |
\\
\sphinxhline
\sphinxAtStartPar
31:5
&\begin{itemize}
\item {} 
\end{itemize}
&\begin{itemize}
\item {} 
\end{itemize}
\\
\sphinxhline
\sphinxAtStartPar
3
&
\sphinxAtStartPar
STO
&
\sphinxAtStartPar
写1,产生STOP,完成后自动清零。             |
\\
\sphinxhline
\sphinxAtStartPar
2
&
\sphinxAtStartPar
WR
&
\sphinxAtStartPar
写1,发送TX                                 | A中数据,完成后(含ACK/NACK时间)自动清零。 |

\sphinxAtStartPar
向本位写                                    | 要求TXDATA不能为空。否则,本位无法设置。  |

\sphinxAtStartPar
注意:WR与RD位不能同时写1.
|
\\
\sphinxhline
\sphinxAtStartPar
1
&
\sphinxAtStartPar
RD
&
\sphinxAtStartPar
写1,接收数据                               | DATA中,完成后(含ACK/NACK时间)自动清零。 |
\\
\sphinxhline
\sphinxAtStartPar
0
&
\sphinxAtStartPar
STA
&
\sphinxAtStartPar
写1,产生START,完成后自动清零。            |

\sphinxAtStartPar
注:允许STA和WR同时置位,优先发送START。    |
\\
\sphinxbottomrule
\end{tabular}
\sphinxtableafterendhook\par
\sphinxattableend\end{savenotes}


\subsubsection{时序配置寄存器CLK}
\label{\detokenize{SWM241/_u529f_u80fd_u63cf_u8ff0/I2C_u603b_u7ebf_u63a7_u5236_u5668:clk}}

\begin{savenotes}\sphinxattablestart
\sphinxthistablewithglobalstyle
\centering
\begin{tabular}[t]{\X{20}{100}\X{20}{100}\X{20}{100}\X{20}{100}\X{20}{100}}
\sphinxtoprule
\sphinxtableatstartofbodyhook
\sphinxAtStartPar
寄存器 |
&
\begin{DUlineblock}{0em}
\item[] 偏移 |
\end{DUlineblock}
&
\begin{DUlineblock}{0em}
\item[] 
\item[] {\color{red}\bfseries{}|}
\end{DUlineblock}
&
\sphinxAtStartPar
复位值 |    描 | |
&
\begin{DUlineblock}{0em}
\item[] |
  |
\end{DUlineblock}
\\
\sphinxhline
\sphinxAtStartPar
CLK
&
\sphinxAtStartPar
0x24
&&
\sphinxAtStartPar
0 033F7F
&
\sphinxAtStartPar
时序配置寄存器             |
\\
\sphinxbottomrule
\end{tabular}
\sphinxtableafterendhook\par
\sphinxattableend\end{savenotes}


\begin{savenotes}\sphinxattablestart
\sphinxthistablewithglobalstyle
\centering
\begin{tabular}[t]{\X{12}{96}\X{12}{96}\X{12}{96}\X{12}{96}\X{12}{96}\X{12}{96}\X{12}{96}\X{12}{96}}
\sphinxtoprule
\sphinxtableatstartofbodyhook
\sphinxAtStartPar
31
&
\sphinxAtStartPar
30
&
\sphinxAtStartPar
29
&
\sphinxAtStartPar
28
&
\sphinxAtStartPar
27
&
\sphinxAtStartPar
26
&
\sphinxAtStartPar
25
&
\sphinxAtStartPar
24
\\
\sphinxhline\begin{itemize}
\item {} 
\end{itemize}
&&&&&&&\\
\sphinxhline
\sphinxAtStartPar
23
&
\sphinxAtStartPar
22
&
\sphinxAtStartPar
21
&
\sphinxAtStartPar
20
&
\sphinxAtStartPar
19
&
\sphinxAtStartPar
18
&
\sphinxAtStartPar
17
&
\sphinxAtStartPar
16
\\
\sphinxhline
\sphinxAtStartPar
DIV
&&&&&&&\\
\sphinxhline
\sphinxAtStartPar
15
&
\sphinxAtStartPar
14
&
\sphinxAtStartPar
13
&
\sphinxAtStartPar
12
&
\sphinxAtStartPar
11
&
\sphinxAtStartPar
10
&
\sphinxAtStartPar
9
&
\sphinxAtStartPar
8
\\
\sphinxhline
\sphinxAtStartPar
SCLH
&&&&&&&\\
\sphinxhline
\sphinxAtStartPar
7
&
\sphinxAtStartPar
6
&
\sphinxAtStartPar
5
&
\sphinxAtStartPar
4
&
\sphinxAtStartPar
3
&
\sphinxAtStartPar
2
&
\sphinxAtStartPar
1
&
\sphinxAtStartPar
0
\\
\sphinxhline
\sphinxAtStartPar
SCLL
&&&&&&&\\
\sphinxbottomrule
\end{tabular}
\sphinxtableafterendhook\par
\sphinxattableend\end{savenotes}


\begin{savenotes}\sphinxattablestart
\sphinxthistablewithglobalstyle
\centering
\begin{tabular}[t]{\X{33}{99}\X{33}{99}\X{33}{99}}
\sphinxtoprule
\sphinxtableatstartofbodyhook
\sphinxAtStartPar
位域 |
&
\sphinxAtStartPar
名称     | |
&
\sphinxAtStartPar
描述                                        | |
\\
\sphinxhline
\sphinxAtStartPar
31:28
&\begin{itemize}
\item {} 
\end{itemize}
&\begin{itemize}
\item {} 
\end{itemize}
\\
\sphinxhline
\sphinxAtStartPar
27:24
&
\sphinxAtStartPar
SDAH
&\\
\sphinxhline
\sphinxAtStartPar
23:16
&
\sphinxAtStartPar
DIV
&
\sphinxAtStartPar
时钟预分                                    | 见SCLH和SCLL描述。(仅对Master模式有效) |

\sphinxAtStartPar
0:1分频                                    |

\sphinxAtStartPar
1:2分频                                    |

\sphinxAtStartPar
2:3分频                                    |

\sphinxAtStartPar
……

\sphinxAtStartPar
255:256分频                                |
\\
\sphinxhline
\sphinxAtStartPar
15:8
&
\sphinxAtStartPar
SCLH
&\\
\sphinxhline
\sphinxAtStartPar
7:0
&
\sphinxAtStartPar
SCLL
&
\sphinxAtStartPar
SCL时钟低电平时间配置。(对Master模         | ;在slave模式下,如果使能了STRETCH功能, | DS配置为0,则需要配置本寄存器。在slave写T  | TA后,延迟本寄存器设置的时间,再释放SCL。) |

\sphinxAtStartPar
tLOW=( (SCLL+1) * (DIV+1) + SDAH + 5) * Tpclk

\sphinxAtStartPar
SCL的周期为tHIGH+tLOW。                     |

\sphinxAtStartPar
推荐SCLH与SCLL的比例为1:2。                 |
\\
\sphinxbottomrule
\end{tabular}
\sphinxtableafterendhook\par
\sphinxattableend\end{savenotes}

\sphinxAtStartPar
\sphinxstyleemphasis{注:示意图如图 6‑34所示}


\subsubsection{Slave控制寄存器SCR}
\label{\detokenize{SWM241/_u529f_u80fd_u63cf_u8ff0/I2C_u603b_u7ebf_u63a7_u5236_u5668:slavescr}}

\begin{savenotes}\sphinxattablestart
\sphinxthistablewithglobalstyle
\centering
\begin{tabular}[t]{\X{20}{100}\X{20}{100}\X{20}{100}\X{20}{100}\X{20}{100}}
\sphinxtoprule
\sphinxtableatstartofbodyhook
\sphinxAtStartPar
寄存器 |
&
\begin{DUlineblock}{0em}
\item[] 偏移 |
\end{DUlineblock}
&
\begin{DUlineblock}{0em}
\item[] 
\item[] {\color{red}\bfseries{}|}
\end{DUlineblock}
&
\sphinxAtStartPar
复位值 |    描 | |
&
\begin{DUlineblock}{0em}
\item[] |
  |
\end{DUlineblock}
\\
\sphinxhline
\sphinxAtStartPar
SCR
&
\sphinxAtStartPar
0x30
&&
\sphinxAtStartPar
0 000008
&
\sphinxAtStartPar
Slave控制寄存器            |
\\
\sphinxbottomrule
\end{tabular}
\sphinxtableafterendhook\par
\sphinxattableend\end{savenotes}


\begin{savenotes}\sphinxattablestart
\sphinxthistablewithglobalstyle
\centering
\begin{tabular}[t]{\X{12}{96}\X{12}{96}\X{12}{96}\X{12}{96}\X{12}{96}\X{12}{96}\X{12}{96}\X{12}{96}}
\sphinxtoprule
\sphinxtableatstartofbodyhook
\sphinxAtStartPar
31
&
\sphinxAtStartPar
30
&
\sphinxAtStartPar
29
&
\sphinxAtStartPar
28
&
\sphinxAtStartPar
27
&
\sphinxAtStartPar
26
&
\sphinxAtStartPar
25
&
\sphinxAtStartPar
24
\\
\sphinxhline\begin{itemize}
\item {} 
\end{itemize}
&&&&&&&\\
\sphinxhline
\sphinxAtStartPar
23
&
\sphinxAtStartPar
22
&
\sphinxAtStartPar
21
&
\sphinxAtStartPar
20
&
\sphinxAtStartPar
19
&
\sphinxAtStartPar
18
&
\sphinxAtStartPar
17
&
\sphinxAtStartPar
16
\\
\sphinxhline\begin{itemize}
\item {} 
\end{itemize}
&&&&&&&\\
\sphinxhline
\sphinxAtStartPar
15
&
\sphinxAtStartPar
14
&
\sphinxAtStartPar
13
&
\sphinxAtStartPar
12
&
\sphinxAtStartPar
11
&
\sphinxAtStartPar
10
&
\sphinxAtStartPar
9
&
\sphinxAtStartPar
8
\\
\sphinxhline\begin{itemize}
\item {} 
\end{itemize}
&&&&&&&\\
\sphinxhline
\sphinxAtStartPar
7
&
\sphinxAtStartPar
6
&
\sphinxAtStartPar
5
&
\sphinxAtStartPar
4
&
\sphinxAtStartPar
3
&
\sphinxAtStartPar
2
&
\sphinxAtStartPar
1
&
\sphinxAtStartPar
0
\\
\sphinxhline\begin{itemize}
\item {} 
\end{itemize}
&&&&&&&\\
\sphinxbottomrule
\end{tabular}
\sphinxtableafterendhook\par
\sphinxattableend\end{savenotes}


\begin{savenotes}\sphinxattablestart
\sphinxthistablewithglobalstyle
\centering
\begin{tabular}[t]{\X{33}{99}\X{33}{99}\X{33}{99}}
\sphinxtoprule
\sphinxtableatstartofbodyhook
\sphinxAtStartPar
位域 |
&
\sphinxAtStartPar
名称     | |
&
\sphinxAtStartPar
描述                                        | |
\\
\sphinxhline
\sphinxAtStartPar
31:4
&\begin{itemize}
\item {} 
\end{itemize}
&\begin{itemize}
\item {} 
\end{itemize}
\\
\sphinxhline
\sphinxAtStartPar
3
&
\sphinxAtStartPar
ASDS
&\\
\sphinxhline
\sphinxAtStartPar
2
&
\sphinxAtStartPar
STRE
&
\sphinxAtStartPar
Clock stretching使能控制。                  |

\sphinxAtStartPar
0:Clock stretching不使能。                 |

\sphinxAtStartPar
1:Clock stretching使能。                   |

\sphinxAtStartPar
(slave作为                                 | eiver时,当接收到新数据,但旧数据未被及时读 | XNE=1):SLVSTR变有效,在返回ACK后,将SCL | hold在低电平,直到                          |
被读取后,把新数据更新到RXDATA中,同时S  | TR变无效,再释放SCL,开始下一个数据的接收。 |

\sphinxAtStartPar
slave作为transmitter时                      | 送结束(TXDONE=1,含接收ACK/NACK时间), | 据未准备好(TXE=1):SLVSTR变有效,将SCL | ho 低电平,直到新数据准备好,延迟SCLL时间后, |
STR变无效,再释放SCL,开始新数据的发送。)  |
\\
\sphinxhline
\sphinxAtStartPar
1
&
\sphinxAtStartPar
MCDE
&
\sphinxAtStartPar
Master Code Detect Enable.

\sphinxAtStartPar
0:不检测master code。                      |

\sphinxAtStartPar
1:检测master code。                        |

\sphinxAtStartPar
本位有效时,slave在START之后检测到master    | code,会生成RXDONE中断,并硬件设置          | RDS为11。软件应保证slave地址设置不与master  | code冲突。                                  |
\\
\sphinxhline
\sphinxAtStartPar
0
&
\sphinxAtStartPar
ADDR10
&
\sphinxAtStartPar
slave地址模式控制。                         |

\sphinxAtStartPar
0:7位地址模式                              |

\sphinxAtStartPar
1:10位地址模式                             |
\\
\sphinxbottomrule
\end{tabular}
\sphinxtableafterendhook\par
\sphinxattableend\end{savenotes}


\subsubsection{Slave地址寄存器SADDR}
\label{\detokenize{SWM241/_u529f_u80fd_u63cf_u8ff0/I2C_u603b_u7ebf_u63a7_u5236_u5668:slavesaddr}}

\begin{savenotes}\sphinxattablestart
\sphinxthistablewithglobalstyle
\centering
\begin{tabular}[t]{\X{20}{100}\X{20}{100}\X{20}{100}\X{20}{100}\X{20}{100}}
\sphinxtoprule
\sphinxtableatstartofbodyhook
\sphinxAtStartPar
寄存器 |
&
\begin{DUlineblock}{0em}
\item[] 偏移 |
\end{DUlineblock}
&
\begin{DUlineblock}{0em}
\item[] 
\item[] {\color{red}\bfseries{}|}
\end{DUlineblock}
&
\sphinxAtStartPar
复位值 |    描 | |
&
\begin{DUlineblock}{0em}
\item[] |
  |
\end{DUlineblock}
\\
\sphinxhline
\sphinxAtStartPar
SADDR
&
\sphinxAtStartPar
0x34
&&
\sphinxAtStartPar
0 000000
&
\sphinxAtStartPar
Slave地址寄存器            |
\\
\sphinxbottomrule
\end{tabular}
\sphinxtableafterendhook\par
\sphinxattableend\end{savenotes}


\begin{savenotes}\sphinxattablestart
\sphinxthistablewithglobalstyle
\centering
\begin{tabular}[t]{\X{12}{96}\X{12}{96}\X{12}{96}\X{12}{96}\X{12}{96}\X{12}{96}\X{12}{96}\X{12}{96}}
\sphinxtoprule
\sphinxtableatstartofbodyhook
\sphinxAtStartPar
31
&
\sphinxAtStartPar
30
&
\sphinxAtStartPar
29
&
\sphinxAtStartPar
28
&
\sphinxAtStartPar
27
&
\sphinxAtStartPar
26
&
\sphinxAtStartPar
25
&
\sphinxAtStartPar
24
\\
\sphinxhline\begin{itemize}
\item {} 
\end{itemize}
&&&&&&&\\
\sphinxhline
\sphinxAtStartPar
23
&
\sphinxAtStartPar
22
&
\sphinxAtStartPar
21
&
\sphinxAtStartPar
20
&
\sphinxAtStartPar
19
&
\sphinxAtStartPar
18
&
\sphinxAtStartPar
17
&
\sphinxAtStartPar
16
\\
\sphinxhline
\sphinxAtStartPar
MASK\_ADDR7
&&&&&&&\\
\sphinxhline
\sphinxAtStartPar
15
&
\sphinxAtStartPar
14
&
\sphinxAtStartPar
13
&
\sphinxAtStartPar
12
&
\sphinxAtStartPar
11
&
\sphinxAtStartPar
10
&
\sphinxAtStartPar
9
&
\sphinxAtStartPar
8
\\
\sphinxhline\begin{itemize}
\item {} 
\end{itemize}
&&&&&&&\\
\sphinxhline
\sphinxAtStartPar
7
&
\sphinxAtStartPar
6
&
\sphinxAtStartPar
5
&
\sphinxAtStartPar
4
&
\sphinxAtStartPar
3
&
\sphinxAtStartPar
2
&
\sphinxAtStartPar
1
&
\sphinxAtStartPar
0
\\
\sphinxhline
\sphinxAtStartPar
ADDR7
&&&&&&&\\
\sphinxbottomrule
\end{tabular}
\sphinxtableafterendhook\par
\sphinxattableend\end{savenotes}


\begin{savenotes}\sphinxattablestart
\sphinxthistablewithglobalstyle
\centering
\begin{tabular}[t]{\X{33}{99}\X{33}{99}\X{33}{99}}
\sphinxtoprule
\sphinxtableatstartofbodyhook
\sphinxAtStartPar
位域 |
&
\sphinxAtStartPar
名称     | |
&
\sphinxAtStartPar
描述                                        | |
\\
\sphinxhline
\sphinxAtStartPar
31:24
&\begin{itemize}
\item {} 
\end{itemize}
&\begin{itemize}
\item {} 
\end{itemize}
\\
\sphinxhline
\sphinxAtStartPar
23:17
&
\sphinxAtStartPar
MASK7
&
\sphinxAtStartPar
Slave对应地址位掩码。                       |

\sphinxAtStartPar
0:不掩码。                                 |

\sphinxAtStartPar
1:掩码对应位地址。掩码                     | 件匹配slave地址时,忽略被掩码的地址位。  |

\sphinxAtStartPar
对于10位地址模式,RXDATA                    | ADDR{[}7:0{]},所以不支持对ADDR{[}9:8{]}的mask。 |
\\
\sphinxhline
\sphinxAtStartPar
16
&
\sphinxAtStartPar
MASK10
&
\sphinxAtStartPar
Slave对应地址位掩码。                       |
\\
\sphinxhline
\sphinxAtStartPar
15:10
&\begin{itemize}
\item {} 
\end{itemize}
&\begin{itemize}
\item {} 
\end{itemize}
\\
\sphinxhline
\sphinxAtStartPar
9:8
&
\sphinxAtStartPar
ADDR10
&
\sphinxAtStartPar
10位地址模式:地址bit9\textasciitilde{}bit8                 |
\\
\sphinxhline
\sphinxAtStartPar
7:1
&
\sphinxAtStartPar
ADDR7
&
\sphinxAtStartPar
地址bit7\textasciitilde{}bit1                               |
\\
\sphinxhline
\sphinxAtStartPar
0
&
\sphinxAtStartPar
ADDR0
&
\sphinxAtStartPar
10位地址模式:地址bit0                      |
\\
\sphinxbottomrule
\end{tabular}
\sphinxtableafterendhook\par
\sphinxattableend\end{savenotes}

\sphinxstepscope


\section{SPI总线控制器(SPI)}
\label{\detokenize{SWM241/_u529f_u80fd_u63cf_u8ff0/SPI_u603b_u7ebf_u63a7_u5236_u5668:spi-spi}}\label{\detokenize{SWM241/_u529f_u80fd_u63cf_u8ff0/SPI_u603b_u7ebf_u63a7_u5236_u5668::doc}}
\sphinxAtStartPar
概述
\textasciitilde{}\textasciitilde{}

\sphinxAtStartPar
SWM241系列所有型号SPI模块操作均相同,不同型号SPI数量可能不同。使用前需使能对应SPI模块时钟。

\sphinxAtStartPar
SPI是一种用于全双工模式的串行同步数据通讯协议。该模块为支持SPI通讯协议的接口控制模块,它支持主/从工作模式。

\sphinxAtStartPar
SPI模块支持SPI模式及SSI模式。SPI模式下支持MASTER模式及SLAVE模式。具备深度为8的FIFO,速率及帧宽度可灵活配置。其结构图如图 6‑37所示。

\sphinxAtStartPar
除了支持SPI协议外,还可支持SSI协议。

\sphinxAtStartPar
特性
\textasciitilde{}\textasciitilde{}
\begin{itemize}
\item {} 
\sphinxAtStartPar
支持主机模式和从机模式

\item {} 
\sphinxAtStartPar
支持SPI和SSI两种帧结构

\item {} 
\sphinxAtStartPar
内置深度为8的FIFO,作为接收和发送数据的缓存

\item {} 
\sphinxAtStartPar
支持DMA

\item {} 
\sphinxAtStartPar
数据位数4\textasciitilde{}16bit可配置

\item {} 
\sphinxAtStartPar
可编程时钟极性和相位

\item {} 
\sphinxAtStartPar
支持LSB和MSB可配置

\end{itemize}


\subsection{模块结构框图}
\label{\detokenize{SWM241/_u529f_u80fd_u63cf_u8ff0/SPI_u603b_u7ebf_u63a7_u5236_u5668:id1}}
\sphinxAtStartPar
\sphinxincludegraphics{{SWM241/功能描述/mediaSPI总线控002}.emf}

\sphinxAtStartPar
图 6‑37 SPI模块结构框图


\subsection{功能描述}
\label{\detokenize{SWM241/_u529f_u80fd_u63cf_u8ff0/SPI_u603b_u7ebf_u63a7_u5236_u5668:id2}}

\subsubsection{位速率的产生}
\label{\detokenize{SWM241/_u529f_u80fd_u63cf_u8ff0/SPI_u603b_u7ebf_u63a7_u5236_u5668:id3}}
\sphinxAtStartPar
SPI模块包含一个可编程的位速率时钟分频器来生成串行输出时钟。串行位速率通过设置CTRL寄存器CLKDIV位对输入时钟进行分频来获得。分频值的范围为4\textasciitilde{}512分频值。计算公式如下F$_{\text{sclk\_out}}$ = F$_{\text{HCLK}}$/SCKDIV。

\sphinxAtStartPar
作为主设备时,SPI\_CLK最高支持模块输入时钟4分频,即当时钟为48MHz时,最高可支持输出12MHz时钟。

\sphinxAtStartPar
作为从设备时,SPI\_CLK最高支持模块输入时钟6分频,即当时钟为48MHz时,最高支持输入8MHz时钟。

\sphinxAtStartPar
帧宽度
\textasciicircum{}\textasciicircum{}\textasciicircum{}

\sphinxAtStartPar
使能SPI模块前,可通过设置CTRL寄存器SIZE位选择数据帧长度,支持4~16位。设置该寄存器位时,需保证SPI处于关闭状态。


\subsubsection{SPI模式}
\label{\detokenize{SWM241/_u529f_u80fd_u63cf_u8ff0/SPI_u603b_u7ebf_u63a7_u5236_u5668:spi}}
\sphinxAtStartPar
使能SPI模块前,可通过设置CTRL寄存器中FFS位选择输出模式。当该位选择SPI模式,此时,可通过CTRL寄存器中CPOL和CPHA配置SPI模块时钟空闲状态极性与数据采样时间点。

\sphinxAtStartPar
当CPOL=0,CPHA=0时,时钟空闲状态为低电平,起始采样点为时钟上升沿。

\sphinxAtStartPar
当CPOL=0,CPHA=1时,时钟空闲状态为低电平,起始采样点为时钟下降沿。

\sphinxAtStartPar
当CPOL=1,CPHA=0时,时钟空闲状态为高电平,起始采样点为时钟下降沿。

\sphinxAtStartPar
当CPOL=1,CPHA=1时,时钟空闲状态为高电平,起始采样点为时钟上升沿。

\sphinxAtStartPar
输出波形如图 6‑38所示:

\sphinxAtStartPar
\sphinxincludegraphics{{SWM241/功能描述/mediaSPI总线控003}.emf}

\sphinxAtStartPar
图 6‑38 SPI模式波形图

\sphinxAtStartPar
所有模式下,片选线均为发送一个数据后自动拉高,第二个数据再次拉低,因此当需要使用连续片选时,需使用GPIO模拟片选线。


\subsubsection{SSI模式}
\label{\detokenize{SWM241/_u529f_u80fd_u63cf_u8ff0/SPI_u603b_u7ebf_u63a7_u5236_u5668:ssi}}
\sphinxAtStartPar
可通过设置CTRL寄存器中FFS位选择输出模式,当该位选择SSI模式,单次输出波形如图 6‑39所示:

\sphinxAtStartPar
\sphinxincludegraphics{{SWM241/功能描述/mediaSPI总线控004}.emf}

\sphinxAtStartPar
图 6‑39 SSI模式单次输出波

\sphinxAtStartPar
连续输出波形如图 6‑40所示:

\sphinxAtStartPar
\sphinxincludegraphics{{SWM241/功能描述/mediaSPI总线控005}.emf}

\sphinxAtStartPar
图 6‑40 SSI模式连续输出波形


\subsubsection{主设备操作}
\label{\detokenize{SWM241/_u529f_u80fd_u63cf_u8ff0/SPI_u603b_u7ebf_u63a7_u5236_u5668:id4}}
\sphinxAtStartPar
当SPI模块作为主模块工作时,操作流程如下:
\begin{itemize}
\item {} 
\sphinxAtStartPar
通过CTRL寄存器CLKDIV位定义串行时钟波特率

\item {} 
\sphinxAtStartPar
设置CTRL寄存器SIZE位来选择数据位数

\item {} 
\sphinxAtStartPar
选择CTRL寄存器CPOL和CPHA位,定义数据传输和串行时钟间的相位关系。主、从设备的CPOL和CPHA位必须一致

\item {} 
\sphinxAtStartPar
配置CTRL寄存器FFS位定义数据帧格式,主、从设备的数据帧格式必须一致。

\item {} 
\sphinxAtStartPar
设置CTRL寄存器MSTR位为主模式

\item {} 
\sphinxAtStartPar
使能CTRL寄存器EN位

\end{itemize}

\sphinxAtStartPar
在主模式配置中,MOSI引脚是数据输出,而MISO引脚是数据输入。

\sphinxAtStartPar
注意:当选择硬件提供的CS引脚作为从设备片选使能时,每传输一个字节的数据,CS引脚均会变高。因此,当从设备需要连续拉低的片选信号时,需要使用GPIO模拟CS信号。


\subsubsection{从设备操作}
\label{\detokenize{SWM241/_u529f_u80fd_u63cf_u8ff0/SPI_u603b_u7ebf_u63a7_u5236_u5668:id5}}
\sphinxAtStartPar
在从模式下,SCK引脚用于接收从主设备来的串行时钟。对CTRL寄存器中CLKDIV的设置不影响数据传输速率。

\sphinxAtStartPar
操作流程:
\begin{itemize}
\item {} 
\sphinxAtStartPar
设置CTRL寄存器SIZE位来定义数据位数选择。

\item {} 
\sphinxAtStartPar
选择CTRL寄存器CPOL和CPHA位,与主设备一致。

\item {} 
\sphinxAtStartPar
配置CTRL寄存器FFS位定义数据帧格式。

\item {} 
\sphinxAtStartPar
设置CTRL寄存器MSTR位为从模式

\end{itemize}

\sphinxAtStartPar
在从设备配置中,MOSI引脚是数据输入,MISO引脚是数据输出。


\subsubsection{FIFO操作}
\label{\detokenize{SWM241/_u529f_u80fd_u63cf_u8ff0/SPI_u603b_u7ebf_u63a7_u5236_u5668:fifo}}
\sphinxAtStartPar
发送FIFO

\sphinxAtStartPar
通用发送FIFO是一个32位宽、8单元深、先进先出的存储缓冲区。通过写DATA寄存器来将数据写入发送FIFO,数据在由发送逻辑读出之前一直保存在发送FIFO中。并行数据在进行串行转换并通过MOSI管脚分别发送到相关的从机之前先写入发送FIFO。

\sphinxAtStartPar
接收FIFO

\sphinxAtStartPar
通用接收FIFO是一个32位宽、8单元深、先进先出的存储缓冲区。从串行接口接收到的数据在读出之前一直保存在缓冲区中,通过读DATA寄存器来访问读FIFO。从MISO管脚接收到的串行数据在分别并行加载到相关的主机接收FIFO之前先进行记录。

\sphinxAtStartPar
可通过中断使能寄存器IE、中断状态寄存器IF、状态寄存器STAT对FIFO状态及中断进行查询与控制。


\subsubsection{中断配置与清除}
\label{\detokenize{SWM241/_u529f_u80fd_u63cf_u8ff0/SPI_u603b_u7ebf_u63a7_u5236_u5668:id6}}
\sphinxAtStartPar
可通过配置中断使能寄存器IE相应位使能中断。当中断触发后,中断标志寄存器IF对应位置1。如需清除此标志,需在对应标志位中写1清零(R/W1C),否则中断在开启状态下会一直进入。


\subsection{寄存器映射}
\label{\detokenize{SWM241/_u529f_u80fd_u63cf_u8ff0/SPI_u603b_u7ebf_u63a7_u5236_u5668:id7}}

\begin{savenotes}\sphinxattablestart
\sphinxthistablewithglobalstyle
\centering
\begin{tabular}[t]{\X{20}{100}\X{20}{100}\X{20}{100}\X{20}{100}\X{20}{100}}
\sphinxtoprule
\sphinxtableatstartofbodyhook
\sphinxAtStartPar
名称   |
&
\begin{DUlineblock}{0em}
\item[] 偏移 |
\end{DUlineblock}
&
\begin{DUlineblock}{0em}
\item[] 
\item[] |
|
\end{DUlineblock}
&
\begin{DUlineblock}{0em}
\item[] 
\end{DUlineblock}
\begin{quote}

\begin{DUlineblock}{0em}
\item[] 
\item[] 
\end{DUlineblock}
\end{quote}
&
\sphinxAtStartPar
描述                       | | | |
\\
\sphinxhline
\sphinxAtStartPar
SPI0BASE: {\color{red}\bfseries{}|}0x40044000SPI1BASE: {\color{red}\bfseries{}|}0x40044800
&
\begin{DUlineblock}{0em}
\item[] {\color{red}\bfseries{}|}
\end{DUlineblock}
&&&\\
\sphinxhline
\sphinxAtStartPar
CTRL
&
\sphinxAtStartPar
0x00
&&
\sphinxAtStartPar
0x E1172
&
\sphinxAtStartPar
SPI控制寄存器              |
\\
\sphinxhline
\sphinxAtStartPar
DATA
&
\sphinxAtStartPar
0x04
&&
\sphinxAtStartPar
0x 00000
&
\sphinxAtStartPar
SPI数据寄存器              |
\\
\sphinxhline
\sphinxAtStartPar
STAT
&
\sphinxAtStartPar
0x08
&&
\sphinxAtStartPar
0x 00006
&
\sphinxAtStartPar
SPI状态寄存器              |
\\
\sphinxhline
\sphinxAtStartPar
IE
&
\sphinxAtStartPar
0x0C
&&
\sphinxAtStartPar
0x 00000
&
\sphinxAtStartPar
SPI中断使能寄存器          |
\\
\sphinxhline
\sphinxAtStartPar
IF
&
\sphinxAtStartPar
0x10
&&
\sphinxAtStartPar
0x 00000
&
\sphinxAtStartPar
SPI中断状态寄存器          |
\\
\sphinxbottomrule
\end{tabular}
\sphinxtableafterendhook\par
\sphinxattableend\end{savenotes}


\subsection{寄存器描述}
\label{\detokenize{SWM241/_u529f_u80fd_u63cf_u8ff0/SPI_u603b_u7ebf_u63a7_u5236_u5668:id14}}

\subsubsection{控制寄存器CTRL}
\label{\detokenize{SWM241/_u529f_u80fd_u63cf_u8ff0/SPI_u603b_u7ebf_u63a7_u5236_u5668:ctrl}}

\begin{savenotes}\sphinxattablestart
\sphinxthistablewithglobalstyle
\centering
\begin{tabular}[t]{\X{20}{100}\X{20}{100}\X{20}{100}\X{20}{100}\X{20}{100}}
\sphinxtoprule
\sphinxtableatstartofbodyhook
\sphinxAtStartPar
寄存器 |
&
\begin{DUlineblock}{0em}
\item[] 偏移 |
\end{DUlineblock}
&
\begin{DUlineblock}{0em}
\item[] 
\item[] {\color{red}\bfseries{}|}
\end{DUlineblock}
&
\sphinxAtStartPar
复位值 |    描 | |
&
\begin{DUlineblock}{0em}
\item[] |
  |
\end{DUlineblock}
\\
\sphinxhline
\sphinxAtStartPar
CTRL
&
\sphinxAtStartPar
0x00
&&
\sphinxAtStartPar
0 9E1172
&
\sphinxAtStartPar
SPI控制寄存器              |
\\
\sphinxbottomrule
\end{tabular}
\sphinxtableafterendhook\par
\sphinxattableend\end{savenotes}


\begin{savenotes}\sphinxattablestart
\sphinxthistablewithglobalstyle
\centering
\begin{tabular}[t]{\X{12}{96}\X{12}{96}\X{12}{96}\X{12}{96}\X{12}{96}\X{12}{96}\X{12}{96}\X{12}{96}}
\sphinxtoprule
\sphinxtableatstartofbodyhook
\sphinxAtStartPar
31
&
\sphinxAtStartPar
30
&
\sphinxAtStartPar
29
&
\sphinxAtStartPar
28
&
\sphinxAtStartPar
27
&
\sphinxAtStartPar
26
&
\sphinxAtStartPar
25
&
\sphinxAtStartPar
24
\\
\sphinxhline\begin{itemize}
\item {} 
\end{itemize}
&&&&\begin{itemize}
\item {} 
\end{itemize}
&&&\\
\sphinxhline
\sphinxAtStartPar
23
&
\sphinxAtStartPar
22
&
\sphinxAtStartPar
21
&
\sphinxAtStartPar
20
&
\sphinxAtStartPar
19
&
\sphinxAtStartPar
18
&
\sphinxAtStartPar
17
&
\sphinxAtStartPar
16
\\
\sphinxhline
\sphinxAtStartPar
TFTHR
&&&&&&&\\
\sphinxhline
\sphinxAtStartPar
15
&
\sphinxAtStartPar
14
&
\sphinxAtStartPar
13
&
\sphinxAtStartPar
12
&
\sphinxAtStartPar
11
&
\sphinxAtStartPar
10
&
\sphinxAtStartPar
9
&
\sphinxAtStartPar
8
\\
\sphinxhline
\sphinxAtStartPar
DMARXEN
&
\sphinxAtStartPar
D MATXEN
&&&
\sphinxAtStartPar
FFS
&&&\\
\sphinxhline
\sphinxAtStartPar
7
&
\sphinxAtStartPar
6
&
\sphinxAtStartPar
5
&
\sphinxAtStartPar
4
&
\sphinxAtStartPar
3
&
\sphinxAtStartPar
2
&
\sphinxAtStartPar
1
&
\sphinxAtStartPar
0
\\
\sphinxhline
\sphinxAtStartPar
SIZE
&&&&
\sphinxAtStartPar
EN
&&&\\
\sphinxbottomrule
\end{tabular}
\sphinxtableafterendhook\par
\sphinxattableend\end{savenotes}


\begin{savenotes}\sphinxattablestart
\sphinxthistablewithglobalstyle
\centering
\begin{tabular}[t]{\X{33}{99}\X{33}{99}\X{33}{99}}
\sphinxtoprule
\sphinxtableatstartofbodyhook
\sphinxAtStartPar
位域 |
&
\sphinxAtStartPar
名称     | |
&
\sphinxAtStartPar
描述                                        | |
\\
\sphinxhline
\sphinxAtStartPar
31:29
&\begin{itemize}
\item {} 
\end{itemize}
&\begin{itemize}
\item {} 
\end{itemize}
\\
\sphinxhline
\sphinxAtStartPar
28
&
\sphinxAtStartPar
LSBF
&
\sphinxAtStartPar
LSB配置寄存器                               |

\sphinxAtStartPar
1:数据按照LSB发送(发                      | TX寄存器数据的bit0位会首先被发出;接收时 | 的第一个bit数据会放到RX寄存器的bit0位)  |

\sphinxAtStartPar
0:数据按照MSB发送(发                      | TX寄存器数据的最高位会首先被发出;接收时 | 的第一个bit数据会放到RX寄存器的最高位)  |
\\
\sphinxhline
\sphinxAtStartPar
27:26
&\begin{itemize}
\item {} 
\end{itemize}
&\begin{itemize}
\item {} 
\end{itemize}
\\
\sphinxhline
\sphinxAtStartPar
25
&
\sphinxAtStartPar
TFCLR
&
\sphinxAtStartPar
发送FIFO清除控制位                          |

\sphinxAtStartPar
1:发送FIFO清除有效                         |

\sphinxAtStartPar
0:发送FIFO清除无效                         |
\\
\sphinxhline
\sphinxAtStartPar
24
&
\sphinxAtStartPar
RFCLR
&
\sphinxAtStartPar
接收FIFO清除控制位                          |

\sphinxAtStartPar
1:接收FIFO清除有效                         |

\sphinxAtStartPar
0:接收FIFO清除无效                         |
\\
\sphinxhline
\sphinxAtStartPar
23:21
&
\sphinxAtStartPar
TFTHR
&
\sphinxAtStartPar
发送FIFO达到设置水位后产生中断配置位        |

\sphinxAtStartPar
000:发送FIFO中最多有0个数据                |

\sphinxAtStartPar
001:发送FIFO中最多有1个数据                |

\sphinxAtStartPar
010:发送FIFO中最多有2个数据                |

\sphinxAtStartPar
011:发送FIFO中最多有3个数据                |

\sphinxAtStartPar
100:发送FIFO中最多有4个数据                |

\sphinxAtStartPar
101:发送FIFO中最多有5个数据                |

\sphinxAtStartPar
110:发送FIFO中最多有6个数据                |

\sphinxAtStartPar
111:发送FIFO中最多有7个数据                |
\\
\sphinxhline
\sphinxAtStartPar
20:18
&
\sphinxAtStartPar
RFTHR
&
\sphinxAtStartPar
接收FIFO达到设置水位后会产生中断配置位      |

\sphinxAtStartPar
000:接收FIFO中至少有1个数据                |

\sphinxAtStartPar
001:接收FIFO中至少有2个数据                |

\sphinxAtStartPar
010:接收FIFO中至少有3个数据                |

\sphinxAtStartPar
011:接收FIFO中至少有4个数据                |

\sphinxAtStartPar
100:接收FIFO中至少有5个数据                |

\sphinxAtStartPar
101:接收FIFO中至少有6个数据                |

\sphinxAtStartPar
110:接收FIFO中至少有7个数据                |

\sphinxAtStartPar
111:接收FIFO中至少有8个数据                |
\\
\sphinxhline
\sphinxAtStartPar
17
&
\sphinxAtStartPar
SSN\_H
&
\sphinxAtStartPar
SSN在传输过程中是否出现控制位。(在数据帧   | I模式下,并且配置主模式工作时,通过该位可  | 传输过程中每帧数据之间是否需要SSN拉高)  |

\sphinxAtStartPar
0:传输过程中SSN始终为0                     |

\sphinxAtStartPar
1:传输过                                   | 一帧数据之间会将SSN至少拉高0.5个SCK周期  |
\\
\sphinxhline
\sphinxAtStartPar
16
&
\sphinxAtStartPar
FILTE
&
\sphinxAtStartPar
输入信号去抖控制                            |

\sphinxAtStartPar
0:对输入信号不进行去抖操作                 |

\sphinxAtStartPar
1:对输入信号进行去抖操作                   |

\sphinxAtStartPar
注:采用去抖功能,会极大提高                | 线上信号的可靠性,但也会极大损失传输速率。  |

\sphinxAtStartPar
抖功能,在全双工模式下最高只能采用波特率 | 分频工作,单工情况下可采用8分频波特率工作  | 能关闭,全双工模式下最高只能采用波特率为 | 工作,在单工情况下可采用4分频波特率工作。 |
\\
\sphinxhline
\sphinxAtStartPar
15
&
\sphinxAtStartPar
DMARXEN
&
\sphinxAtStartPar
DMA读SPI模式选择                            |

\sphinxAtStartPar
1:通过DMA读FIFO                            |

\sphinxAtStartPar
0:通过MCU写FIFO                            |
\\
\sphinxhline
\sphinxAtStartPar
14
&
\sphinxAtStartPar
DMATXEN
&
\sphinxAtStartPar
DMA写SPI模式选择                            |

\sphinxAtStartPar
1:通过DMA写FIFO                            |

\sphinxAtStartPar
0:通过MCU写FIFO                            |
\\
\sphinxhline
\sphinxAtStartPar
13
&
\sphinxAtStartPar
FAST
&
\sphinxAtStartPar
快速模式选择                                |

\sphinxAtStartPar
1:SPI的SCLK为pclk的2分频                   |

\sphinxAtStartPar
0:SPI的SCLK由CLKDIV控制                    |

\sphinxAtStartPar
注:仅适用于SPI模式                         |
\\
\sphinxhline
\sphinxAtStartPar
12
&
\sphinxAtStartPar
MSTR
&
\sphinxAtStartPar
主从模式选择                                |

\sphinxAtStartPar
1:SPI系统配置为主器件模式                  |

\sphinxAtStartPar
0:SPI系统配置为从器件模式                  |

\sphinxAtStartPar
注:仅适用于SPI模式                         |
\\
\sphinxhline
\sphinxAtStartPar
11:10
&
\sphinxAtStartPar
FFS
&
\sphinxAtStartPar
数据帧格式选择                              |

\sphinxAtStartPar
00:SPI模式                                 |

\sphinxAtStartPar
01:SSI模式                                 |

\sphinxAtStartPar
10:保留                                    |

\sphinxAtStartPar
11:保留                                    |
\\
\sphinxhline
\sphinxAtStartPar
9
&
\sphinxAtStartPar
CPOL
&
\sphinxAtStartPar
时钟极性选择                                |

\sphinxAtStartPar
串行时钟空闲状态为低电平,有效电平为高电平 |

\sphinxAtStartPar
串行时钟空闲状态为高电平,有效电平为低电平 |

\sphinxAtStartPar
注:仅适用于SPI模式                         |
\\
\sphinxhline
\sphinxAtStartPar
8
&
\sphinxAtStartPar
CPHA
&
\sphinxAtStartPar
时钟相位选择                                |

\sphinxAtStartPar
0:在串行时钟的第一个跳变沿采样数据         |

\sphinxAtStartPar
1:在串行时钟的第二个跳变沿采样数据         |

\sphinxAtStartPar
注:仅适用于SPI模式                         |
\\
\sphinxhline
\sphinxAtStartPar
7:4
&
\sphinxAtStartPar
SIZE
&
\sphinxAtStartPar
数据位数选择                                |

\sphinxAtStartPar
0000:保留                                  |

\sphinxAtStartPar
0001:保留                                  |

\sphinxAtStartPar
0010:保留                                  |

\sphinxAtStartPar
0011:4bit数据                              |

\sphinxAtStartPar
0100:5bit数据                              |

\sphinxAtStartPar
0101:6bit数据                              |

\sphinxAtStartPar
0110:7bit数据                              |

\sphinxAtStartPar
0111:8bit数据                              |

\sphinxAtStartPar
1000:9bit数据                              |

\sphinxAtStartPar
1001:10bit数据                             |

\sphinxAtStartPar
1010:11bit数据                             |

\sphinxAtStartPar
1011:12bit数据                             |

\sphinxAtStartPar
1100:13bit数据                             |

\sphinxAtStartPar
1101:14bit数据                             |

\sphinxAtStartPar
1110:15bit数据                             |

\sphinxAtStartPar
1111:16bit数据                             |

\sphinxAtStartPar
注:仅适用于SPI模式                         |
\\
\sphinxhline
\sphinxAtStartPar
3
&
\sphinxAtStartPar
EN
&
\sphinxAtStartPar
SPI使能位                                   |

\sphinxAtStartPar
0:关闭                                     |

\sphinxAtStartPar
1:开启                                     |

\sphinxAtStartPar
注1:仅适用于SPI模式                        |

\sphinxAtStartPar
注2:该寄                                   | 能后,若在主模式下,当发送FIFO有数据时则 | 动数据帧传输;在从模式下,等待数据帧传输 |
\\
\sphinxhline
\sphinxAtStartPar
2:0
&
\sphinxAtStartPar
CLKDIV
&
\sphinxAtStartPar
波特率选择                                  |

\sphinxAtStartPar
000:主时钟4分频                            |

\sphinxAtStartPar
001:主时钟8分频                            |

\sphinxAtStartPar
010:主时钟16分频                           |

\sphinxAtStartPar
011:主时钟32分频                           |

\sphinxAtStartPar
100:主时钟64分频                           |

\sphinxAtStartPar
101:主时钟128分频                          |

\sphinxAtStartPar
110:主时钟256分频                          |

\sphinxAtStartPar
111:主时钟512分频                          |

\sphinxAtStartPar
注:仅适用于SPI模式                         |
\\
\sphinxbottomrule
\end{tabular}
\sphinxtableafterendhook\par
\sphinxattableend\end{savenotes}


\subsubsection{数据寄存器DATA}
\label{\detokenize{SWM241/_u529f_u80fd_u63cf_u8ff0/SPI_u603b_u7ebf_u63a7_u5236_u5668:data}}

\begin{savenotes}\sphinxattablestart
\sphinxthistablewithglobalstyle
\centering
\begin{tabular}[t]{\X{20}{100}\X{20}{100}\X{20}{100}\X{20}{100}\X{20}{100}}
\sphinxtoprule
\sphinxtableatstartofbodyhook
\sphinxAtStartPar
寄存器 |
&
\begin{DUlineblock}{0em}
\item[] 偏移 |
\end{DUlineblock}
&
\begin{DUlineblock}{0em}
\item[] 
\item[] {\color{red}\bfseries{}|}
\end{DUlineblock}
&
\sphinxAtStartPar
复位值 |    描 | |
&
\begin{DUlineblock}{0em}
\item[] |
  |
\end{DUlineblock}
\\
\sphinxhline
\sphinxAtStartPar
DATA
&
\sphinxAtStartPar
0x04
&&
\sphinxAtStartPar
0 000000
&
\sphinxAtStartPar
SPI数据寄存器              |
\\
\sphinxbottomrule
\end{tabular}
\sphinxtableafterendhook\par
\sphinxattableend\end{savenotes}


\begin{savenotes}\sphinxattablestart
\sphinxthistablewithglobalstyle
\centering
\begin{tabular}[t]{\X{12}{96}\X{12}{96}\X{12}{96}\X{12}{96}\X{12}{96}\X{12}{96}\X{12}{96}\X{12}{96}}
\sphinxtoprule
\sphinxtableatstartofbodyhook
\sphinxAtStartPar
31
&
\sphinxAtStartPar
30
&
\sphinxAtStartPar
29
&
\sphinxAtStartPar
28
&
\sphinxAtStartPar
27
&
\sphinxAtStartPar
26
&
\sphinxAtStartPar
25
&
\sphinxAtStartPar
24
\\
\sphinxhline
\sphinxAtStartPar
DATA
&&&&&&&\\
\sphinxhline
\sphinxAtStartPar
23
&
\sphinxAtStartPar
22
&
\sphinxAtStartPar
21
&
\sphinxAtStartPar
20
&
\sphinxAtStartPar
19
&
\sphinxAtStartPar
18
&
\sphinxAtStartPar
17
&
\sphinxAtStartPar
16
\\
\sphinxhline
\sphinxAtStartPar
DATA
&&&&&&&\\
\sphinxhline
\sphinxAtStartPar
15
&
\sphinxAtStartPar
14
&
\sphinxAtStartPar
13
&
\sphinxAtStartPar
12
&
\sphinxAtStartPar
11
&
\sphinxAtStartPar
10
&
\sphinxAtStartPar
9
&
\sphinxAtStartPar
8
\\
\sphinxhline
\sphinxAtStartPar
DATA
&&&&&&&\\
\sphinxhline
\sphinxAtStartPar
7
&
\sphinxAtStartPar
6
&
\sphinxAtStartPar
5
&
\sphinxAtStartPar
4
&
\sphinxAtStartPar
3
&
\sphinxAtStartPar
2
&
\sphinxAtStartPar
1
&
\sphinxAtStartPar
0
\\
\sphinxhline
\sphinxAtStartPar
DATA
&&&&&&&\\
\sphinxbottomrule
\end{tabular}
\sphinxtableafterendhook\par
\sphinxattableend\end{savenotes}


\begin{savenotes}\sphinxattablestart
\sphinxthistablewithglobalstyle
\centering
\begin{tabular}[t]{\X{33}{99}\X{33}{99}\X{33}{99}}
\sphinxtoprule
\sphinxtableatstartofbodyhook
\sphinxAtStartPar
位域 |
&
\sphinxAtStartPar
名称     | |
&
\sphinxAtStartPar
描述                                        | |
\\
\sphinxhline
\sphinxAtStartPar
31:0
&
\sphinxAtStartPar
DATA
&
\sphinxAtStartPar
SPI接收/发送数据寄存器                      |

\sphinxAtStartPar
读操作从接收FIFO中读出接收到的数据          |

\sphinxAtStartPar
写操作将数据写入发送FIFO中                  |

\sphinxAtStartPar
注:若数据                                  | 2bit,则按照右对齐进行排列,高位不关心。  |
\\
\sphinxbottomrule
\end{tabular}
\sphinxtableafterendhook\par
\sphinxattableend\end{savenotes}


\subsubsection{状态寄存器STAT}
\label{\detokenize{SWM241/_u529f_u80fd_u63cf_u8ff0/SPI_u603b_u7ebf_u63a7_u5236_u5668:stat}}

\begin{savenotes}\sphinxattablestart
\sphinxthistablewithglobalstyle
\centering
\begin{tabular}[t]{\X{20}{100}\X{20}{100}\X{20}{100}\X{20}{100}\X{20}{100}}
\sphinxtoprule
\sphinxtableatstartofbodyhook
\sphinxAtStartPar
寄存器 |
&
\begin{DUlineblock}{0em}
\item[] 偏移 |
\end{DUlineblock}
&
\begin{DUlineblock}{0em}
\item[] 
\item[] {\color{red}\bfseries{}|}
\end{DUlineblock}
&
\sphinxAtStartPar
复位值 |    描 | |
&
\begin{DUlineblock}{0em}
\item[] |
  |
\end{DUlineblock}
\\
\sphinxhline
\sphinxAtStartPar
STAT
&
\sphinxAtStartPar
0x08
&&
\sphinxAtStartPar
0 000006
&
\sphinxAtStartPar
SPI状态寄存器              |
\\
\sphinxbottomrule
\end{tabular}
\sphinxtableafterendhook\par
\sphinxattableend\end{savenotes}


\begin{savenotes}\sphinxattablestart
\sphinxthistablewithglobalstyle
\centering
\begin{tabular}[t]{\X{12}{96}\X{12}{96}\X{12}{96}\X{12}{96}\X{12}{96}\X{12}{96}\X{12}{96}\X{12}{96}}
\sphinxtoprule
\sphinxtableatstartofbodyhook
\sphinxAtStartPar
31
&
\sphinxAtStartPar
30
&
\sphinxAtStartPar
29
&
\sphinxAtStartPar
28
&
\sphinxAtStartPar
27
&
\sphinxAtStartPar
26
&
\sphinxAtStartPar
25
&
\sphinxAtStartPar
24
\\
\sphinxhline\begin{itemize}
\item {} 
\end{itemize}
&&&&&&&\\
\sphinxhline
\sphinxAtStartPar
23
&
\sphinxAtStartPar
22
&
\sphinxAtStartPar
21
&
\sphinxAtStartPar
20
&
\sphinxAtStartPar
19
&
\sphinxAtStartPar
18
&
\sphinxAtStartPar
17
&
\sphinxAtStartPar
16
\\
\sphinxhline\begin{itemize}
\item {} 
\end{itemize}
&&&&&&&\\
\sphinxhline
\sphinxAtStartPar
15
&
\sphinxAtStartPar
14
&
\sphinxAtStartPar
13
&
\sphinxAtStartPar
12
&
\sphinxAtStartPar
11
&
\sphinxAtStartPar
10
&
\sphinxAtStartPar
9
&
\sphinxAtStartPar
8
\\
\sphinxhline
\sphinxAtStartPar
BUSY
&\begin{itemize}
\item {} 
\end{itemize}
&&&&&&\\
\sphinxhline
\sphinxAtStartPar
7
&
\sphinxAtStartPar
6
&
\sphinxAtStartPar
5
&
\sphinxAtStartPar
4
&
\sphinxAtStartPar
3
&
\sphinxAtStartPar
2
&
\sphinxAtStartPar
1
&
\sphinxAtStartPar
0
\\
\sphinxhline
\sphinxAtStartPar
TFLVL
&&&
\sphinxAtStartPar
RFF
&&&
\sphinxAtStartPar
TFE
&
\sphinxAtStartPar
WTC
\\
\sphinxbottomrule
\end{tabular}
\sphinxtableafterendhook\par
\sphinxattableend\end{savenotes}


\begin{savenotes}\sphinxattablestart
\sphinxthistablewithglobalstyle
\centering
\begin{tabular}[t]{\X{33}{99}\X{33}{99}\X{33}{99}}
\sphinxtoprule
\sphinxtableatstartofbodyhook
\sphinxAtStartPar
位域 |
&
\sphinxAtStartPar
名称     | |
&
\sphinxAtStartPar
描述                                        | |
\\
\sphinxhline
\sphinxAtStartPar
31:16
&\begin{itemize}
\item {} 
\end{itemize}
&\begin{itemize}
\item {} 
\end{itemize}
\\
\sphinxhline
\sphinxAtStartPar
15
&
\sphinxAtStartPar
BUSY
&
\sphinxAtStartPar
SPI传输忙标志位                             |

\sphinxAtStartPar
0:表示SPI未进行传输                        |

\sphinxAtStartPar
1:表示SPI正在进行传输                      |

\sphinxAtStartPar
注:仅适用于SPI模式                         |
\\
\sphinxhline
\sphinxAtStartPar
14:12
&\begin{itemize}
\item {} 
\end{itemize}
&\begin{itemize}
\item {} 
\end{itemize}
\\
\sphinxhline
\sphinxAtStartPar
11:9
&
\sphinxAtStartPar
RFLVL
&
\sphinxAtStartPar
接收FIFO数据深度位标志,RO                  |

\sphinxAtStartPar
000:RFF为1时,表示FIFO内有8组数据          |

\sphinxAtStartPar
RFF为0时,表示FIFO内没有数据;              |

\sphinxAtStartPar
001:表示FIFO内有1组数据;                  |

\sphinxAtStartPar
010:表示FIFO内有2组数据;                  |

\sphinxAtStartPar
011:表示FIFO内有3组数据;                  |

\sphinxAtStartPar
100:表示FIFO内有4组数据;                  |

\sphinxAtStartPar
101:表示FIFO内有5组数据;                  |

\sphinxAtStartPar
110:表示FIFO内有6组数据;                  |

\sphinxAtStartPar
111:表示FIFO内有7组数据;                  |
\\
\sphinxhline
\sphinxAtStartPar
8:6
&
\sphinxAtStartPar
TFLVL
&
\sphinxAtStartPar
发送FIFO数据深度位标志,RO                  |

\sphinxAtStartPar
000:TFNF为0时,表示FIFO内有8组数据         |

\sphinxAtStartPar
TFNF为1时,表示FIFO内没有数据;             |

\sphinxAtStartPar
001:表示FIFO内有1组数据;                  |

\sphinxAtStartPar
010:表示FIFO内有2组数据;                  |

\sphinxAtStartPar
011:表示FIFO内有3组数据;                  |

\sphinxAtStartPar
100:表示FIFO内有4组数据;                  |

\sphinxAtStartPar
101:表示FIFO内有5组数据;                  |

\sphinxAtStartPar
110:表示FIFO内有6组数据;                  |

\sphinxAtStartPar
111:表示FIFO内有7组数据;                  |
\\
\sphinxhline
\sphinxAtStartPar
5
&
\sphinxAtStartPar
RFOV
&
\sphinxAtStartPar
接收FIFO溢出标志,软件清零,写清零          |

\sphinxAtStartPar
0:未溢出                                   |

\sphinxAtStartPar
1:溢出                                     |
\\
\sphinxhline
\sphinxAtStartPar
4
&
\sphinxAtStartPar
RFF
&
\sphinxAtStartPar
接收FIFO满标志                              |

\sphinxAtStartPar
0:非满                                     |

\sphinxAtStartPar
1:满                                       |
\\
\sphinxhline
\sphinxAtStartPar
3
&
\sphinxAtStartPar
RFNE
&
\sphinxAtStartPar
接收FIFO非空标志                            |

\sphinxAtStartPar
0:空                                       |

\sphinxAtStartPar
1:非空                                     |
\\
\sphinxhline
\sphinxAtStartPar
2
&
\sphinxAtStartPar
TFNF
&
\sphinxAtStartPar
发送FIFO非满标志                            |

\sphinxAtStartPar
0:满                                       |

\sphinxAtStartPar
1:非满                                     |
\\
\sphinxhline
\sphinxAtStartPar
1
&
\sphinxAtStartPar
TFE
&
\sphinxAtStartPar
发送FIFO空标志                              |

\sphinxAtStartPar
0:非空                                     |

\sphinxAtStartPar
1:空                                       |
\\
\sphinxhline
\sphinxAtStartPar
0
&
\sphinxAtStartPar
WTC
&
\sphinxAtStartPar
SPI数据帧传输结束标志                       |

\sphinxAtStartPar
每次数据帧传输结束后,该标志会被置位。      |

\sphinxAtStartPar
软件清零,写1清零。                         |

\sphinxAtStartPar
注:仅适用于SPI模式                         |
\\
\sphinxbottomrule
\end{tabular}
\sphinxtableafterendhook\par
\sphinxattableend\end{savenotes}


\subsubsection{中断使能寄存器IE}
\label{\detokenize{SWM241/_u529f_u80fd_u63cf_u8ff0/SPI_u603b_u7ebf_u63a7_u5236_u5668:ie}}

\begin{savenotes}\sphinxattablestart
\sphinxthistablewithglobalstyle
\centering
\begin{tabular}[t]{\X{20}{100}\X{20}{100}\X{20}{100}\X{20}{100}\X{20}{100}}
\sphinxtoprule
\sphinxtableatstartofbodyhook
\sphinxAtStartPar
寄存器 |
&
\begin{DUlineblock}{0em}
\item[] 偏移 |
\end{DUlineblock}
&
\begin{DUlineblock}{0em}
\item[] 
\item[] {\color{red}\bfseries{}|}
\end{DUlineblock}
&
\sphinxAtStartPar
复位值 |    描 | |
&
\begin{DUlineblock}{0em}
\item[] |
  |
\end{DUlineblock}
\\
\sphinxhline
\sphinxAtStartPar
IE
&
\sphinxAtStartPar
0x0C
&&
\sphinxAtStartPar
0 000000
&
\sphinxAtStartPar
SPI中断使能寄存器          |
\\
\sphinxbottomrule
\end{tabular}
\sphinxtableafterendhook\par
\sphinxattableend\end{savenotes}


\begin{savenotes}\sphinxattablestart
\sphinxthistablewithglobalstyle
\centering
\begin{tabular}[t]{\X{12}{96}\X{12}{96}\X{12}{96}\X{12}{96}\X{12}{96}\X{12}{96}\X{12}{96}\X{12}{96}}
\sphinxtoprule
\sphinxtableatstartofbodyhook
\sphinxAtStartPar
31
&
\sphinxAtStartPar
30
&
\sphinxAtStartPar
29
&
\sphinxAtStartPar
28
&
\sphinxAtStartPar
27
&
\sphinxAtStartPar
26
&
\sphinxAtStartPar
25
&
\sphinxAtStartPar
24
\\
\sphinxhline\begin{itemize}
\item {} 
\end{itemize}
&&&&&&&\\
\sphinxhline
\sphinxAtStartPar
23
&
\sphinxAtStartPar
22
&
\sphinxAtStartPar
21
&
\sphinxAtStartPar
20
&
\sphinxAtStartPar
19
&
\sphinxAtStartPar
18
&
\sphinxAtStartPar
17
&
\sphinxAtStartPar
16
\\
\sphinxhline\begin{itemize}
\item {} 
\end{itemize}
&&&&&&&\\
\sphinxhline
\sphinxAtStartPar
15
&
\sphinxAtStartPar
14
&
\sphinxAtStartPar
13
&
\sphinxAtStartPar
12
&
\sphinxAtStartPar
11
&
\sphinxAtStartPar
10
&
\sphinxAtStartPar
9
&
\sphinxAtStartPar
8
\\
\sphinxhline\begin{itemize}
\item {} 
\end{itemize}
&&&&&&
\sphinxAtStartPar
FTC
&
\sphinxAtStartPar
WTC
\\
\sphinxhline
\sphinxAtStartPar
7
&
\sphinxAtStartPar
6
&
\sphinxAtStartPar
5
&
\sphinxAtStartPar
4
&
\sphinxAtStartPar
3
&
\sphinxAtStartPar
2
&
\sphinxAtStartPar
1
&
\sphinxAtStartPar
0
\\
\sphinxhline
\sphinxAtStartPar
­\sphinxhyphen{}
&
\begin{DUlineblock}{0em}
\item[] TFTHR
\end{DUlineblock}
&
\begin{DUlineblock}{0em}
\item[] RFTHR
\end{DUlineblock}
&
\begin{DUlineblock}{0em}
\item[] TFHF
\end{DUlineblock}
&
\begin{DUlineblock}{0em}
\item[] TFE
\end{DUlineblock}
&
\begin{DUlineblock}{0em}
\item[] RFHF
\end{DUlineblock}
&
\begin{DUlineblock}{0em}
\item[] RFF
\end{DUlineblock}
&
\begin{DUlineblock}{0em}
\item[] RFOV
\end{DUlineblock}
\\
\sphinxbottomrule
\end{tabular}
\sphinxtableafterendhook\par
\sphinxattableend\end{savenotes}


\begin{savenotes}\sphinxattablestart
\sphinxthistablewithglobalstyle
\centering
\begin{tabular}[t]{\X{33}{99}\X{33}{99}\X{33}{99}}
\sphinxtoprule
\sphinxtableatstartofbodyhook
\sphinxAtStartPar
位域 |
&
\sphinxAtStartPar
名称     | |
&
\sphinxAtStartPar
描述                                        | |
\\
\sphinxhline
\sphinxAtStartPar
31:12
&\begin{itemize}
\item {} 
\end{itemize}
&\begin{itemize}
\item {} 
\end{itemize}
\\
\sphinxhline
\sphinxAtStartPar
11
&
\sphinxAtStartPar
SSRISE
&
\sphinxAtStartPar
从机模式下,SSN信号上升沿检测中断使能       |
\\
\sphinxhline
\sphinxAtStartPar
10
&
\sphinxAtStartPar
SSFALL
&
\sphinxAtStartPar
从机模式下,SSN信号下降沿检测中断使能       |
\\
\sphinxhline
\sphinxAtStartPar
9
&
\sphinxAtStartPar
FTC
&
\sphinxAtStartPar
SPI传输结束中断使能                         |
\\
\sphinxhline
\sphinxAtStartPar
8
&
\sphinxAtStartPar
WTC
&
\sphinxAtStartPar
SPI数据帧传输结束中断使能                   |
\\
\sphinxhline
\sphinxAtStartPar
7
&\begin{itemize}
\item {} 
\end{itemize}
&\begin{itemize}
\item {} 
\end{itemize}
\\
\sphinxhline
\sphinxAtStartPar
6
&
\sphinxAtStartPar
TFTHR
&
\sphinxAtStartPar
发送FIFO达到设定水位中断使能                |
\\
\sphinxhline
\sphinxAtStartPar
5
&
\sphinxAtStartPar
RFTHR
&
\sphinxAtStartPar
接收FIFO达到设定水位中断使能                |
\\
\sphinxhline
\sphinxAtStartPar
4
&
\sphinxAtStartPar
TFHF
&
\sphinxAtStartPar
发送FIFO半满使能                            |
\\
\sphinxhline
\sphinxAtStartPar
3
&
\sphinxAtStartPar
TFE
&
\sphinxAtStartPar
发送FIFO空中断使能                          |
\\
\sphinxhline
\sphinxAtStartPar
2
&
\sphinxAtStartPar
RFHF
&
\sphinxAtStartPar
接收FIFO半满使能                            |
\\
\sphinxhline
\sphinxAtStartPar
1
&
\sphinxAtStartPar
RFF
&
\sphinxAtStartPar
接收FIFO满中断使能                          |
\\
\sphinxhline
\sphinxAtStartPar
0
&
\sphinxAtStartPar
RFOV
&
\sphinxAtStartPar
接收FIFO溢出中断使能                        |
\\
\sphinxbottomrule
\end{tabular}
\sphinxtableafterendhook\par
\sphinxattableend\end{savenotes}


\subsubsection{中断状态寄存器IF}
\label{\detokenize{SWM241/_u529f_u80fd_u63cf_u8ff0/SPI_u603b_u7ebf_u63a7_u5236_u5668:if}}

\begin{savenotes}\sphinxattablestart
\sphinxthistablewithglobalstyle
\centering
\begin{tabular}[t]{\X{20}{100}\X{20}{100}\X{20}{100}\X{20}{100}\X{20}{100}}
\sphinxtoprule
\sphinxtableatstartofbodyhook
\sphinxAtStartPar
寄存器 |
&
\begin{DUlineblock}{0em}
\item[] 偏移 |
\end{DUlineblock}
&
\begin{DUlineblock}{0em}
\item[] 
\item[] {\color{red}\bfseries{}|}
\end{DUlineblock}
&
\sphinxAtStartPar
复位值 |    描 | |
&
\begin{DUlineblock}{0em}
\item[] |
  |
\end{DUlineblock}
\\
\sphinxhline
\sphinxAtStartPar
IF
&
\sphinxAtStartPar
0x10
&&
\sphinxAtStartPar
0 000000
&
\sphinxAtStartPar
SPI中断状态寄存器          |
\\
\sphinxbottomrule
\end{tabular}
\sphinxtableafterendhook\par
\sphinxattableend\end{savenotes}


\begin{savenotes}\sphinxattablestart
\sphinxthistablewithglobalstyle
\centering
\begin{tabular}[t]{\X{12}{96}\X{12}{96}\X{12}{96}\X{12}{96}\X{12}{96}\X{12}{96}\X{12}{96}\X{12}{96}}
\sphinxtoprule
\sphinxtableatstartofbodyhook
\sphinxAtStartPar
31
&
\sphinxAtStartPar
30
&
\sphinxAtStartPar
29
&
\sphinxAtStartPar
28
&
\sphinxAtStartPar
27
&
\sphinxAtStartPar
26
&
\sphinxAtStartPar
25
&
\sphinxAtStartPar
24
\\
\sphinxhline\begin{itemize}
\item {} 
\end{itemize}
&&&&&&&\\
\sphinxhline
\sphinxAtStartPar
23
&
\sphinxAtStartPar
22
&
\sphinxAtStartPar
21
&
\sphinxAtStartPar
20
&
\sphinxAtStartPar
19
&
\sphinxAtStartPar
18
&
\sphinxAtStartPar
17
&
\sphinxAtStartPar
16
\\
\sphinxhline\begin{itemize}
\item {} 
\end{itemize}
&&&&&&&\\
\sphinxhline
\sphinxAtStartPar
15
&
\sphinxAtStartPar
14
&
\sphinxAtStartPar
13
&
\sphinxAtStartPar
12
&
\sphinxAtStartPar
11
&
\sphinxAtStartPar
10
&
\sphinxAtStartPar
9
&
\sphinxAtStartPar
8
\\
\sphinxhline\begin{itemize}
\item {} 
\end{itemize}
&&&&&&
\sphinxAtStartPar
FTC
&
\sphinxAtStartPar
WTC
\\
\sphinxhline
\sphinxAtStartPar
7
&
\sphinxAtStartPar
6
&
\sphinxAtStartPar
5
&
\sphinxAtStartPar
4
&
\sphinxAtStartPar
3
&
\sphinxAtStartPar
2
&
\sphinxAtStartPar
1
&
\sphinxAtStartPar
0
\\
\sphinxhline
\sphinxAtStartPar
­\sphinxhyphen{}
&
\begin{DUlineblock}{0em}
\item[] TFTHR
\end{DUlineblock}
&
\begin{DUlineblock}{0em}
\item[] RFTHR
\end{DUlineblock}
&
\begin{DUlineblock}{0em}
\item[] TFHF
\end{DUlineblock}
&
\begin{DUlineblock}{0em}
\item[] TFE
\end{DUlineblock}
&
\begin{DUlineblock}{0em}
\item[] RFHF
\end{DUlineblock}
&
\begin{DUlineblock}{0em}
\item[] RFF
\end{DUlineblock}
&
\begin{DUlineblock}{0em}
\item[] RFOV
\end{DUlineblock}
\\
\sphinxbottomrule
\end{tabular}
\sphinxtableafterendhook\par
\sphinxattableend\end{savenotes}


\begin{savenotes}\sphinxattablestart
\sphinxthistablewithglobalstyle
\centering
\begin{tabular}[t]{\X{33}{99}\X{33}{99}\X{33}{99}}
\sphinxtoprule
\sphinxtableatstartofbodyhook
\sphinxAtStartPar
位域 |
&
\sphinxAtStartPar
名称     | |
&
\sphinxAtStartPar
描述                                        | |
\\
\sphinxhline
\sphinxAtStartPar
31:12
&\begin{itemize}
\item {} 
\end{itemize}
&\begin{itemize}
\item {} 
\end{itemize}
\\
\sphinxhline
\sphinxAtStartPar
11
&
\sphinxAtStartPar
SSRISE
&
\sphinxAtStartPar
从机模式SSN上升沿中断,写1清中断            |

\sphinxAtStartPar
1:中断                                     |

\sphinxAtStartPar
0:未中断                                   |
\\
\sphinxhline
\sphinxAtStartPar
10
&
\sphinxAtStartPar
SSFALL
&
\sphinxAtStartPar
从机模式SSN下降沿中断,写1清中断            |

\sphinxAtStartPar
1:中断                                     |

\sphinxAtStartPar
0:未中断                                   |
\\
\sphinxhline
\sphinxAtStartPar
9
&
\sphinxAtStartPar
FTC
&
\sphinxAtStartPar
SPI传输结束中断                             |

\sphinxAtStartPar
1:中断                                     |

\sphinxAtStartPar
0:未中断                                   |

\sphinxAtStartPar
写1清中断                                   |
\\
\sphinxhline
\sphinxAtStartPar
8
&
\sphinxAtStartPar
WTC
&
\sphinxAtStartPar
SPI数据帧传输结束中断                       |

\sphinxAtStartPar
1:中断                                     |

\sphinxAtStartPar
0:未中断                                   |

\sphinxAtStartPar
写1清中断                                   |
\\
\sphinxhline
\sphinxAtStartPar
7
&\begin{itemize}
\item {} 
\end{itemize}
&\begin{itemize}
\item {} 
\end{itemize}
\\
\sphinxhline
\sphinxAtStartPar
6
&
\sphinxAtStartPar
TFTHR
&
\sphinxAtStartPar
发送FIFO达到设定水位中断                    |

\sphinxAtStartPar
1:中断                                     |

\sphinxAtStartPar
0:未中断                                   |

\sphinxAtStartPar
写1清中断                                   |
\\
\sphinxhline
\sphinxAtStartPar
5
&
\sphinxAtStartPar
RFTHR
&
\sphinxAtStartPar
接收FIFO达到设定水位中断                    |

\sphinxAtStartPar
1:中断                                     |

\sphinxAtStartPar
0:未中断                                   |

\sphinxAtStartPar
写1清中断                                   |
\\
\sphinxhline
\sphinxAtStartPar
4
&
\sphinxAtStartPar
TFHF
&
\sphinxAtStartPar
发送FIFO半满中断                            |

\sphinxAtStartPar
1:中断                                     |

\sphinxAtStartPar
0:未中断                                   |

\sphinxAtStartPar
写1清中断状态                               |
\\
\sphinxhline
\sphinxAtStartPar
3
&
\sphinxAtStartPar
TFE
&
\sphinxAtStartPar
发送FIFO空中断                              |

\sphinxAtStartPar
1:中断                                     |

\sphinxAtStartPar
0:未中断                                   |

\sphinxAtStartPar
写1清中断状态                               |
\\
\sphinxhline
\sphinxAtStartPar
2
&
\sphinxAtStartPar
RFHF
&
\sphinxAtStartPar
接收FIFO半满中断                            |

\sphinxAtStartPar
1:中断                                     |

\sphinxAtStartPar
0:未中断                                   |

\sphinxAtStartPar
写1清中断状态                               |
\\
\sphinxhline
\sphinxAtStartPar
1
&
\sphinxAtStartPar
RFF
&
\sphinxAtStartPar
接收FIFO满中断                              |

\sphinxAtStartPar
1:中断                                     |

\sphinxAtStartPar
0:未中断                                   |

\sphinxAtStartPar
写1清中断状态                               |
\\
\sphinxhline
\sphinxAtStartPar
0
&
\sphinxAtStartPar
RFOVF
&
\sphinxAtStartPar
接收FIFO溢出中断                            |

\sphinxAtStartPar
1:中断                                     |

\sphinxAtStartPar
0:未中断                                   |

\sphinxAtStartPar
写1清中断状态                               |
\\
\sphinxbottomrule
\end{tabular}
\sphinxtableafterendhook\par
\sphinxattableend\end{savenotes}

\sphinxstepscope


\section{局域网控制器(CAN)}
\label{\detokenize{SWM241/_u529f_u80fd_u63cf_u8ff0/_u5c40_u57df_u7f51_u63a7_u5236_u5668:can}}\label{\detokenize{SWM241/_u529f_u80fd_u63cf_u8ff0/_u5c40_u57df_u7f51_u63a7_u5236_u5668::doc}}
\sphinxAtStartPar
概述
\textasciitilde{}\textasciitilde{}

\sphinxAtStartPar
本系列所有型号CAN模块操作均相同,不同型号CAN数量可能不同。使用前需使能CAN模块时钟。与物理层相连需要连接额外的硬件收发器。

\sphinxAtStartPar
特性
\textasciitilde{}\textasciitilde{}
\begin{itemize}
\item {} 
\sphinxAtStartPar
支持协议2.0A(11位标识符)和2.0B(29位标识符)

\item {} 
\sphinxAtStartPar
支持最大1 Mbit/s 的比特率

\item {} 
\sphinxAtStartPar
提供64字节的接收FIFO

\item {} 
\sphinxAtStartPar
提供32个16位或16个32位的滤波器

\item {} 
\sphinxAtStartPar
提供可掩蔽中断

\item {} 
\sphinxAtStartPar
为自检操作提供可编程环回模式

\end{itemize}


\subsection{功能描述}
\label{\detokenize{SWM241/_u529f_u80fd_u63cf_u8ff0/_u5c40_u57df_u7f51_u63a7_u5236_u5668:id1}}

\subsubsection{中断配置与清除}
\label{\detokenize{SWM241/_u529f_u80fd_u63cf_u8ff0/_u5c40_u57df_u7f51_u63a7_u5236_u5668:id2}}
\sphinxAtStartPar
CAN模块支持如下中断:
\begin{itemize}
\item {} 
\sphinxAtStartPar
接收中断

\item {} 
\sphinxAtStartPar
发送中断

\item {} 
\sphinxAtStartPar
错误中断

\item {} 
\sphinxAtStartPar
数据溢出中断

\item {} 
\sphinxAtStartPar
唤醒中断

\item {} 
\sphinxAtStartPar
被动错误中断

\item {} 
\sphinxAtStartPar
仲裁丢失中断

\item {} 
\sphinxAtStartPar
总线错误中断

\end{itemize}

\sphinxAtStartPar
触发中断前,首先需要设置相应位的中断使能(IE)。

\sphinxAtStartPar
各中断状态清除(除接收中断),均为读清除。对于接收中断,需要将CMD寄存器RRB位写1清除。


\subsubsection{数据发送}
\label{\detokenize{SWM241/_u529f_u80fd_u63cf_u8ff0/_u5c40_u57df_u7f51_u63a7_u5236_u5668:id3}}
\sphinxAtStartPar
发送报文需要设置发送buffer (寄存器INFO,DATA0—DATA11)。可以是标准帧格式或是扩展帧格式。数据位最大是8个字节,超过8字节,自动按8字节计算。

\sphinxAtStartPar
写数据前,需要查看SR寄存器TXRDY位是否等于1,如果不等于1,则发送的数据将会被丢弃。发送数据请求通过设置CMD寄存器TXREQ位为1(发送请求) 或是CMD.
SRR=1(自接收请求)。发送传输启动时,状态寄存器SR.TXBUSY置1,发送请求位清零。

\sphinxAtStartPar
数据传输没有开始时,可以通过设置命令寄存器(CMD.
ABTTX = 1)中止传输。如果已经开始传输,则不能中止。


\subsubsection{数据接收}
\label{\detokenize{SWM241/_u529f_u80fd_u63cf_u8ff0/_u5c40_u57df_u7f51_u63a7_u5236_u5668:id4}}
\sphinxAtStartPar
数据接收先通过滤波器,符合条件标识符的才可以接收。滤波器的设置详见“接收滤波”章节。

\sphinxAtStartPar
数据接收可以读取内部64字节FIFO。

\sphinxAtStartPar
当CAN模块开始将接收到的数据写入接收FIFO时,状态寄存器SR.RXBUSY = 1,当接收FIFO (寄存器INFO,DATA0—DATA11)接收到完整报文的时候,状态寄存器SR.
RXDA = 1 ,中断状态IF.
RXDA置1(如果中断使能寄存器IE.
RXDA置1)。接收FIFO是64字节,最多允许接收5个完整的扩展帧报文。如果接收FIFO没有足够的内存,状态寄存器SR.
RXOV = 1,数据溢出,(如果中断使能 IE.
RXOV =1),溢出中断置位IF.RXOV = 1。

\sphinxAtStartPar
从接收FIFO中读取数据后,需要释放FIFO(设置CMD.
RRB= 1)。如果没有读取的数据,中断状态位(IE.
RXDA)和接收BUFFER(SR.
RXDA)状态位清除。

\sphinxAtStartPar
自接收
\textasciicircum{}\textasciicircum{}\textasciicircum{}

\sphinxAtStartPar
自接收功能,数据可以自发自收,不发送应答位。通过设置自接收请求(CMD.
SRR = 1),根据配置,可以产生发送和接收中断。

\sphinxAtStartPar
如果自接收请求和发送请求同时设置,则自接收请求设置无效


\subsubsection{接收滤波}
\label{\detokenize{SWM241/_u529f_u80fd_u63cf_u8ff0/_u5c40_u57df_u7f51_u63a7_u5236_u5668:id5}}
\sphinxAtStartPar
验收滤波器有验收代码寄存器(ACR0—ACR3)和验收屏蔽寄存器(AMR0—AMR1)

\sphinxAtStartPar
当接收到的CAN帧的ID满足ACR \& \textasciitilde{}AMR == ID \& \textasciitilde{}AMR时,该帧通过过滤,否则丢弃该帧。


\begin{savenotes}\sphinxattablestart
\sphinxthistablewithglobalstyle
\centering
\begin{tabular}[t]{\X{16}{96}\X{16}{96}\X{16}{96}\X{16}{96}\X{16}{96}\X{16}{96}}
\sphinxtoprule
\sphinxtableatstartofbodyhook
\sphinxAtStartPar
\sphinxstylestrong{标准帧格 |式,单过 |滤模式} |
&
\begin{DUlineblock}{0em}
\item[] | |
\end{DUlineblock}
&&&&\\
\sphinxhline
\sphinxAtStartPar
接       {\color{red}\bfseries{}|}收buffer |
&
\begin{DUlineblock}{0em}
\item[] {\color{red}\bfseries{}|}
\end{DUlineblock}
&
\begin{DUlineblock}{0em}
\item[] 
\item[] 
\end{DUlineblock}
&
\begin{DUlineblock}{0em}
\item[] 
\item[] 
\end{DUlineblock}
&
\begin{DUlineblock}{0em}
\item[] 
\item[] 
\end{DUlineblock}
&
\begin{DUlineblock}{0em}
\item[] 
\item[] 
\end{DUlineblock}
\\
\sphinxhline
\sphinxAtStartPar
地址0x44 |
&
\sphinxAtStartPar
x48     |
&
\begin{DUlineblock}{0em}
\item[] 
\end{DUlineblock}
&
\begin{DUlineblock}{0em}
\item[] 
\end{DUlineblock}
&
\sphinxAtStartPar
x4c        |
&
\sphinxAtStartPar
x50       |
\\
\sphinxhline
\sphinxAtStartPar
ID28….ID21
&
\sphinxAtStartPar
I D20…ID18
&
\sphinxAtStartPar
R TR
&
\sphinxAtStartPar
XXXX(不匹配)  | 数
&
\sphinxAtStartPar
字节1   | 数据字节2
&
\begin{DUlineblock}{0em}
\item[] 
\end{DUlineblock}
\\
\sphinxhline
\sphinxAtStartPar
过滤器   |
&
\begin{DUlineblock}{0em}
\item[] 
\end{DUlineblock}
&
\begin{DUlineblock}{0em}
\item[] 
\end{DUlineblock}
&
\begin{DUlineblock}{0em}
\item[] 
\end{DUlineblock}
&
\begin{DUlineblock}{0em}
\item[] 
\end{DUlineblock}
&
\begin{DUlineblock}{0em}
\item[] 
\end{DUlineblock}
\\
\sphinxhline
\sphinxAtStartPar
ACR0{[}7:0{]}
&
\sphinxAtStartPar
A CR1{[}7:4{]}
&&
\sphinxAtStartPar
(ACR 1{[}3:0{]}不使用) |
&
\sphinxAtStartPar
ACR2{[}7:0{]}  | |
&
\sphinxAtStartPar
ACR3{[}7:0{]} | |
\\
\sphinxhline
\sphinxAtStartPar
AMR0{[}7:0{]}
&
\sphinxAtStartPar
A MR1{[}7:4{]}
&&
\sphinxAtStartPar
(AMR 1{[}3:0{]}不使用) |
&
\sphinxAtStartPar
AMR2{[}7:0{]}  | |
&
\sphinxAtStartPar
AMR3{[}7:0{]} | |
\\
\sphinxhline
\sphinxAtStartPar
\sphinxstyleemphasis{注      |:如果不 |需要数据 |匹配,AM |R2、AMR3 |设置     |0xFF}
&
\begin{DUlineblock}{0em}
\item[] | | | | |
\end{DUlineblock}
&
\begin{DUlineblock}{0em}
\item[] 
\end{DUlineblock}

\begin{DUlineblock}{0em}
\item[] 
\end{DUlineblock}
&
\begin{DUlineblock}{0em}
\item[] 
\end{DUlineblock}

\begin{DUlineblock}{0em}
\item[] 
\end{DUlineblock}
&
\begin{DUlineblock}{0em}
\item[] 
\end{DUlineblock}

\begin{DUlineblock}{0em}
\item[] 
\end{DUlineblock}
&
\begin{DUlineblock}{0em}
\item[] 
\end{DUlineblock}

\begin{DUlineblock}{0em}
\item[] 
\end{DUlineblock}
\\
\sphinxbottomrule
\end{tabular}
\sphinxtableafterendhook\par
\sphinxattableend\end{savenotes}


\begin{savenotes}\sphinxattablestart
\sphinxthistablewithglobalstyle
\centering
\begin{tabular}[t]{\X{11}{99}\X{11}{99}\X{11}{99}\X{11}{99}\X{11}{99}\X{11}{99}\X{11}{99}\X{11}{99}\X{11}{99}}
\sphinxtoprule
\sphinxtableatstartofbodyhook
\sphinxAtStartPar
\sphinxstylestrong{标准帧 |格式,  |双过滤  |模式}  |
&
\begin{DUlineblock}{0em}
\item[] | | |
\end{DUlineblock}
&&&&&&&\\
\sphinxhline
\sphinxAtStartPar
接收    {\color{red}\bfseries{}|}buffer
&
\begin{DUlineblock}{0em}
\item[] 
\end{DUlineblock}
&
\begin{DUlineblock}{0em}
\item[] 
\end{DUlineblock}
&
\begin{DUlineblock}{0em}
\item[] 
\end{DUlineblock}
&
\begin{DUlineblock}{0em}
\item[] 
\end{DUlineblock}
&
\begin{DUlineblock}{0em}
\item[] 
\end{DUlineblock}
&
\begin{DUlineblock}{0em}
\item[] 
\end{DUlineblock}
&
\begin{DUlineblock}{0em}
\item[] 
\end{DUlineblock}
&
\begin{DUlineblock}{0em}
\item[] 
\end{DUlineblock}
\\
\sphinxhline
\sphinxAtStartPar
地      {\color{red}\bfseries{}|}址0x44  |
&
\sphinxAtStartPar
0x48    | |
&
\begin{DUlineblock}{0em}
\item[] 
\item[] 
\end{DUlineblock}
&
\begin{DUlineblock}{0em}
\item[] 
\item[] 
\end{DUlineblock}
&
\sphinxAtStartPar
0 | x | 4 C
&
\begin{DUlineblock}{0em}
\item[] 
\item[] 
\end{DUlineblock}
&
\begin{DUlineblock}{0em}
\item[] 
\item[] 
\end{DUlineblock}
&
\sphinxAtStartPar
0 | x | 5 0
&
\begin{DUlineblock}{0em}
\item[] 
\item[] 
\end{DUlineblock}
\\
\sphinxhline
\sphinxAtStartPar
ID28…ID21
&
\sphinxAtStartPar
ID 20…ID18
&
\sphinxAtStartPar
R TR
&
\sphinxAtStartPar
XX( 不匹配) | 据
&
\sphinxAtStartPar
数

\sphinxAtStartPar
字 节 1 {[} 7 : 4 {]}
&
\begin{DUlineblock}{0em}
\item[] 
\end{DUlineblock}

\begin{DUlineblock}{0em}
\item[] 
\end{DUlineblock}
&
\sphinxAtStartPar
数据字    | 数 3:0{]}  | 据 | | |
&&
\begin{DUlineblock}{0em}
\item[] 
\item[] 
\end{DUlineblock}
\\
\sphinxhline
\sphinxAtStartPar
过      {\color{red}\bfseries{}|}滤器1: |
&
\begin{DUlineblock}{0em}
\item[] {\color{red}\bfseries{}|}
\end{DUlineblock}
&
\begin{DUlineblock}{0em}
\item[] 
\end{DUlineblock}
&
\begin{DUlineblock}{0em}
\item[] 
\end{DUlineblock}
&
\begin{DUlineblock}{0em}
\item[] 
\end{DUlineblock}
&
\begin{DUlineblock}{0em}
\item[] 
\end{DUlineblock}
&
\begin{DUlineblock}{0em}
\item[] 
\end{DUlineblock}
&
\begin{DUlineblock}{0em}
\item[] 
\end{DUlineblock}
&
\begin{DUlineblock}{0em}
\item[] 
\end{DUlineblock}
\\
\sphinxhline
\sphinxAtStartPar
ACR0{[}7:0{]}
&
\sphinxAtStartPar
AC R1{[}7:4{]}
&&&&
\sphinxAtStartPar
AC R1{[}3:0{]}
&
\sphinxAtStartPar
ACR3{[}3:0{]}
&&\\
\sphinxhline
\sphinxAtStartPar
AMR0{[}7:0{]}
&
\sphinxAtStartPar
AM R1{[}7:4{]}
&&&&
\sphinxAtStartPar
AM R1{[}3:0{]}
&
\sphinxAtStartPar
AMR3{[}3:0{]}
&&\\
\sphinxhline
\sphinxAtStartPar
过      {\color{red}\bfseries{}|}滤器2: |
&
\begin{DUlineblock}{0em}
\item[] {\color{red}\bfseries{}|}
\end{DUlineblock}
&
\begin{DUlineblock}{0em}
\item[] 
\end{DUlineblock}
&
\begin{DUlineblock}{0em}
\item[] 
\end{DUlineblock}
&
\begin{DUlineblock}{0em}
\item[] 
\end{DUlineblock}
&
\begin{DUlineblock}{0em}
\item[] 
\end{DUlineblock}
&
\begin{DUlineblock}{0em}
\item[] 
\end{DUlineblock}
&
\begin{DUlineblock}{0em}
\item[] 
\end{DUlineblock}
&
\begin{DUlineblock}{0em}
\item[] 
\end{DUlineblock}
\\
\sphinxhline
\sphinxAtStartPar
ACR2{[}7:0{]}
&
\sphinxAtStartPar
AC R3{[}7:4{]}
&&&&&&&\\
\sphinxhline
\sphinxAtStartPar
AMR2{[}7:0{]}
&
\sphinxAtStartPar
AM R3{[}7:4{]}
&&&&&&&\\
\sphinxbottomrule
\end{tabular}
\sphinxtableafterendhook\par
\sphinxattableend\end{savenotes}


\begin{savenotes}\sphinxattablestart
\sphinxthistablewithglobalstyle
\centering
\begin{tabular}[t]{\X{16}{96}\X{16}{96}\X{16}{96}\X{16}{96}\X{16}{96}\X{16}{96}}
\sphinxtoprule
\sphinxtableatstartofbodyhook
\sphinxAtStartPar
\sphinxstylestrong{扩展帧格 |式,单过  |滤模式}  |
&
\begin{DUlineblock}{0em}
\item[] | |
\end{DUlineblock}
&&&&\\
\sphinxhline
\sphinxAtStartPar
接        {\color{red}\bfseries{}|}收buffer  |
&
\begin{DUlineblock}{0em}
\item[] {\color{red}\bfseries{}|}
\end{DUlineblock}
&
\begin{DUlineblock}{0em}
\item[] 
\item[] 
\end{DUlineblock}
&
\begin{DUlineblock}{0em}
\item[] 
\item[] 
\end{DUlineblock}
&
\begin{DUlineblock}{0em}
\item[] 
\item[] 
\end{DUlineblock}
&
\begin{DUlineblock}{0em}
\item[] 
\item[] 
\end{DUlineblock}
\\
\sphinxhline
\sphinxAtStartPar
地        {\color{red}\bfseries{}|}址:0x44  |
&
\sphinxAtStartPar
0x48       | |
&
\sphinxAtStartPar
0x4c      | |
&
\sphinxAtStartPar
0x50   | |
&
\begin{DUlineblock}{0em}
\item[] 
\end{DUlineblock}
&
\begin{DUlineblock}{0em}
\item[] 
\end{DUlineblock}
\\
\sphinxhline
\sphinxAtStartPar
ID28…ID21
&
\sphinxAtStartPar
ID20…ID13
&
\sphinxAtStartPar
ID12…ID5
&
\sphinxAtStartPar
I D4…ID0
&
\sphinxAtStartPar
RTR
&
\sphinxAtStartPar
XX(不匹配)     |
\\
\sphinxhline
\sphinxAtStartPar
过滤器:  |
&
\begin{DUlineblock}{0em}
\item[] 
\end{DUlineblock}
&
\begin{DUlineblock}{0em}
\item[] 
\end{DUlineblock}
&
\begin{DUlineblock}{0em}
\item[] 
\end{DUlineblock}
&
\begin{DUlineblock}{0em}
\item[] 
\end{DUlineblock}
&
\begin{DUlineblock}{0em}
\item[] 
\end{DUlineblock}
\\
\sphinxhline
\sphinxAtStartPar
ACR0{[}7:0{]}
&
\sphinxAtStartPar
ACR1{[}7:0{]}
&
\sphinxAtStartPar
ACR2{[}7:0{]}
&
\sphinxAtStartPar
ACR 3{[}7:2{]}
&&
\sphinxAtStartPar
A CR3{[}1:0{]}不匹配 |
\\
\sphinxhline
\sphinxAtStartPar
AMR0{[}7:0{]}
&
\sphinxAtStartPar
AMR1{[}7:0{]}
&
\sphinxAtStartPar
AMR2{[}7:0{]}
&
\sphinxAtStartPar
AMR 3{[}7:2{]}
&&
\sphinxAtStartPar
A MR3{[}1:0{]}不匹配 |
\\
\sphinxbottomrule
\end{tabular}
\sphinxtableafterendhook\par
\sphinxattableend\end{savenotes}


\begin{savenotes}\sphinxattablestart
\sphinxthistablewithglobalstyle
\centering
\begin{tabular}[t]{\X{16}{96}\X{16}{96}\X{16}{96}\X{16}{96}\X{16}{96}\X{16}{96}}
\sphinxtoprule
\sphinxtableatstartofbodyhook
\sphinxAtStartPar
\sphinxstylestrong{扩展帧 |格式, |双过滤 |模式} |
&
\begin{DUlineblock}{0em}
\item[] | | |
\end{DUlineblock}
&&&&\\
\sphinxhline
\sphinxAtStartPar
接收   {\color{red}\bfseries{}|}buffer
&
\begin{DUlineblock}{0em}
\item[] 
\end{DUlineblock}
&
\begin{DUlineblock}{0em}
\item[] 
\end{DUlineblock}
&
\begin{DUlineblock}{0em}
\item[] 
\end{DUlineblock}
&
\begin{DUlineblock}{0em}
\item[] 
\end{DUlineblock}
&
\begin{DUlineblock}{0em}
\item[] 
\end{DUlineblock}
\\
\sphinxhline
\sphinxAtStartPar
地址   {\color{red}\bfseries{}|}:0x44 |
&
\sphinxAtStartPar
x48     | |
&
\sphinxAtStartPar
x4C        | |
&
\sphinxAtStartPar
x50       | |
&
\begin{DUlineblock}{0em}
\item[] {\color{red}\bfseries{}|}
\end{DUlineblock}
&
\begin{DUlineblock}{0em}
\item[] {\color{red}\bfseries{}|}
\end{DUlineblock}
\\
\sphinxhline
\sphinxAtStartPar
ID28…ID21
&
\sphinxAtStartPar
I D20…ID13
&
\sphinxAtStartPar
ID12\textasciitilde{} ID5(不匹配) | D
&
\sphinxAtStartPar
ID4\textasciitilde{}I 不匹配) | (不匹配
&
\sphinxAtStartPar
RTR | (不匹配) |
&
\sphinxAtStartPar
XX
\\
\sphinxhline
\sphinxAtStartPar
过滤   {\color{red}\bfseries{}|}器1:  |
&
\begin{DUlineblock}{0em}
\item[] {\color{red}\bfseries{}|}
\end{DUlineblock}
&
\begin{DUlineblock}{0em}
\item[] 
\item[] 
\end{DUlineblock}
&
\begin{DUlineblock}{0em}
\item[] 
\item[] 
\end{DUlineblock}
&
\begin{DUlineblock}{0em}
\item[] 
\item[] 
\end{DUlineblock}
&
\begin{DUlineblock}{0em}
\item[] 
\item[] 
\end{DUlineblock}
\\
\sphinxhline
\sphinxAtStartPar
ACR0{[}7:0{]}
&
\sphinxAtStartPar
A CR1{[}7:0{]}
&&&&\\
\sphinxhline
\sphinxAtStartPar
AMR0{[}7:0{]}
&
\sphinxAtStartPar
A MR1{[}7:0{]}
&&&&\\
\sphinxhline
\sphinxAtStartPar
过滤 {\color{red}\bfseries{}|}器2:  |
&
\begin{DUlineblock}{0em}
\item[] {\color{red}\bfseries{}|}
\end{DUlineblock}
&&&&\\
\sphinxhline
\sphinxAtStartPar
ACR2{[}7:0{]}
&
\sphinxAtStartPar
A CR3{[}7:0{]}
&&&&\\
\sphinxhline
\sphinxAtStartPar
AMR2{[}7:0{]}
&
\sphinxAtStartPar
A MR3{[}7:0{]}
&&&&\\
\sphinxbottomrule
\end{tabular}
\sphinxtableafterendhook\par
\sphinxattableend\end{savenotes}

\sphinxAtStartPar
波特率
\textasciicircum{}\textasciicircum{}\textasciicircum{}

\sphinxAtStartPar
可通过BT0 和BT1寄存器设置波特率,波特率的分频值(BRP)低6bit存入BT0寄存器BRP位,高4bit存入BT2寄存器BRP位。

\sphinxAtStartPar
如BRP=(SystemCoreClock/2)/2/ Baudrate/(1 + (BT1.
TSEG1+ 1) + (BT1.
TSEG2 + 1)) \textendash{} 1

\sphinxAtStartPar
值得注意的是需要确保BRP的值为整数,即(SystemCoreClock/2)/2/ Baudrate为整数,即(1 + (BT1.
TSEG1+ 1) + (BT1.
TSEG2 + 1))能被((SystemCoreClock/2)/2/ Baudrate)整除。

\sphinxAtStartPar
采样点 = ( BT1.
TSEG1 + 1)/((1 + (BT1.
TSEG1+ 1) + (BT1.
TSEG2 + 1))*100\%

\sphinxAtStartPar
如图 6‑41 波特率设置示意图所示。

\sphinxAtStartPar
\sphinxincludegraphics{{SWM241/功能描述/media局域网控制器002}.emf}

\sphinxAtStartPar
图 6‑41 波特率设置示意图


\subsubsection{错误处理}
\label{\detokenize{SWM241/_u529f_u80fd_u63cf_u8ff0/_u5c40_u57df_u7f51_u63a7_u5236_u5668:id44}}
\sphinxAtStartPar
CAN 模块包括两个错误计数器:接收错误计数器RXERR和发送错误计数器TXERR。当发生错误时,它们会根据 CAN 2.0 规范自动递增。

\sphinxAtStartPar
错误的类型(位错误、格式错误、填充错误或是其他错误)和错误在帧中的位置,可以通过错误代码捕捉寄存器ECC查询。

\sphinxAtStartPar
错误报警限制寄存器EWLIM用于设定当接收/发送错误个数达到指定值时触发警告,默认值是96。当发送错误计数器或是接收错误计数器超过错误报警限制寄存器设置的值时,错误状态寄存器(SR.
ERRWARN = 1)置1,如果错误中断使能(IE.
ERRWARN = 1),产生错误中断(IF.
ERRWARN = 1)。

\sphinxAtStartPar
如果任何一个错误计数器超过127是,CAN进入错误被动状态(Error Passive),如果主动错误中断使能(IE.
ERRPASS = 1),产生错误主动中断(IF.
ERRPASS = 1)。

\sphinxAtStartPar
如果发送错误计数器超过了255,总线状态位(SR.
BUSOFF)会被置1,总线关闭,CAN就会进入复位模式。当清除控制寄存器的复位模式(CR.
RST),CAN退出复位模式。


\subsubsection{睡眠模式}
\label{\detokenize{SWM241/_u529f_u80fd_u63cf_u8ff0/_u5c40_u57df_u7f51_u63a7_u5236_u5668:id45}}
\sphinxAtStartPar
CAN可以工作在低功耗的睡眠模式。通过设置控制寄存器CR.SLEEP = 1,进入睡眠模式。

\sphinxAtStartPar
唤醒睡眠模式可以通过以下三种:
\begin{itemize}
\item {} 
\sphinxAtStartPar
总线上有活动

\item {} 
\sphinxAtStartPar
配置睡眠中断使能,触发睡眠唤醒中断

\item {} 
\sphinxAtStartPar
清除睡眠位(CR.SLEEP =0)

\end{itemize}

\sphinxAtStartPar
如果是总线上有活动唤醒睡眠模式,CAN 直到检测到总线空闲,并且接收到11bit后,才接受报文。在复位模式下,CAN不能进入睡眠模式。


\subsubsection{仅听模式}
\label{\detokenize{SWM241/_u529f_u80fd_u63cf_u8ff0/_u5c40_u57df_u7f51_u63a7_u5236_u5668:id46}}
\sphinxAtStartPar
配置CR.
LOM = 1,进入仅听模式。(至少需要三个节点)。

\sphinxAtStartPar
CAN工作在仅听模式,只接收数据,不发送数据。即使接收成功,也不发送应答位。


\subsubsection{初始化和配置}
\label{\detokenize{SWM241/_u529f_u80fd_u63cf_u8ff0/_u5c40_u57df_u7f51_u63a7_u5236_u5668:id47}}
\sphinxAtStartPar
初始化
\begin{itemize}
\item {} 
\sphinxAtStartPar
配置中断使能寄存器

\item {} 
\sphinxAtStartPar
选择单/双过滤模式和复位模式

\item {} 
\sphinxAtStartPar
配置验收寄存器(ACR0—ACR3) 和验收屏蔽寄存器(AMR0—AMR3)

\item {} 
\sphinxAtStartPar
配置总线定时寄存器0(BTR0)和1(BTR1),设置波特率

\item {} 
\sphinxAtStartPar
配置CR寄存器,退出复位模式

\end{itemize}

\sphinxAtStartPar
设置发送数据
\begin{itemize}
\item {} 
\sphinxAtStartPar
查看发送buffer状态位,SR.

\end{itemize}
\begin{quote}

\sphinxAtStartPar
TXBR
\end{quote}
\begin{itemize}
\item {} 
\sphinxAtStartPar
如果可以写入新的报文发送,在发送buffer中写入数据 (配置寄存器INFO,DATA0—DATA11)

\item {} 
\sphinxAtStartPar
配置命令寄存器CMD,设置CMD.

\end{itemize}
\begin{quote}

\sphinxAtStartPar
TXREQ,发送数据请求,或CMD.
SRR,自接收请求
\end{quote}
\begin{itemize}
\item {} 
\sphinxAtStartPar
设置接收数据

\item {} 
\sphinxAtStartPar
查看接收中断状态IF.

\end{itemize}
\begin{quote}

\sphinxAtStartPar
RXDA (使能接收中断)或是接收buffer状态寄存器SR.
RXDA
\end{quote}
\begin{itemize}
\item {} 
\sphinxAtStartPar
当读取接收buffer里的数据后(寄存器INFO,DATA0—DATA11),将CMD.

\end{itemize}
\begin{quote}

\sphinxAtStartPar
RRB置1,释放接收FIFO。
\end{quote}


\subsection{寄存器映射}
\label{\detokenize{SWM241/_u529f_u80fd_u63cf_u8ff0/_u5c40_u57df_u7f51_u63a7_u5236_u5668:id48}}

\begin{savenotes}\sphinxattablestart
\sphinxthistablewithglobalstyle
\centering
\begin{tabular}[t]{\X{20}{100}\X{20}{100}\X{20}{100}\X{20}{100}\X{20}{100}}
\sphinxtoprule
\sphinxtableatstartofbodyhook
\sphinxAtStartPar
名称   |
&
\begin{DUlineblock}{0em}
\item[] 偏移 |
\end{DUlineblock}
&
\begin{DUlineblock}{0em}
\item[] 
\item[] |
|
\end{DUlineblock}
&
\begin{DUlineblock}{0em}
\item[] 
\end{DUlineblock}
\begin{quote}

\begin{DUlineblock}{0em}
\item[] 
\item[] 
\end{DUlineblock}
\end{quote}
&
\sphinxAtStartPar
描述                       | | | |
\\
\sphinxhline
\sphinxAtStartPar
CAN0BASE:0 {\color{red}\bfseries{}|}x400A8000
&
\begin{DUlineblock}{0em}
\item[] 
\end{DUlineblock}
&&&\\
\sphinxhline
\sphinxAtStartPar
CR
&
\sphinxAtStartPar
0x00
&&
\sphinxAtStartPar
0x 00001
&
\sphinxAtStartPar
控制寄存器                 |
\\
\sphinxhline
\sphinxAtStartPar
CMD
&
\sphinxAtStartPar
0x04
&&
\sphinxAtStartPar
0x 00000
&
\sphinxAtStartPar
命令寄存器                 |
\\
\sphinxhline
\sphinxAtStartPar
SR
&
\sphinxAtStartPar
0x08
&&
\sphinxAtStartPar
0x 0003C
&
\sphinxAtStartPar
状态寄存器                 |
\\
\sphinxhline
\sphinxAtStartPar
IF
&
\sphinxAtStartPar
0x0C
&&
\sphinxAtStartPar
0x 00000
&
\sphinxAtStartPar
中断标志寄存器             |
\\
\sphinxhline
\sphinxAtStartPar
IE
&
\sphinxAtStartPar
0x10
&&
\sphinxAtStartPar
0x 00000
&
\sphinxAtStartPar
中断使能寄存器             |
\\
\sphinxhline
\sphinxAtStartPar
BT2
&
\sphinxAtStartPar
0x14
&&
\sphinxAtStartPar
0x 00000
&
\sphinxAtStartPar
总线定时器2                |
\\
\sphinxhline
\sphinxAtStartPar
BT0
&
\sphinxAtStartPar
0x18
&&
\sphinxAtStartPar
0x 00000
&
\sphinxAtStartPar
总线定时器0                |
\\
\sphinxhline
\sphinxAtStartPar
BT1
&
\sphinxAtStartPar
0x1C
&&
\sphinxAtStartPar
0x 00000
&
\sphinxAtStartPar
总线定时器1                |
\\
\sphinxhline
\sphinxAtStartPar
AFM
&
\sphinxAtStartPar
0x24
&&
\sphinxAtStartPar
0x 00000
&
\sphinxAtStartPar
过滤方式选择寄存器         |
\\
\sphinxhline
\sphinxAtStartPar
AFE
&
\sphinxAtStartPar
0x28
&&
\sphinxAtStartPar
0x 00000
&
\sphinxAtStartPar
过滤使能寄存器AFE          |
\\
\sphinxhline
\sphinxAtStartPar
ALC
&
\sphinxAtStartPar
0x2C
&&
\sphinxAtStartPar
0x 00000
&
\sphinxAtStartPar
仲裁丢失捕捉               |
\\
\sphinxhline
\sphinxAtStartPar
ECC
&
\sphinxAtStartPar
0x30
&&
\sphinxAtStartPar
0x 00000
&
\sphinxAtStartPar
错误代码捕捉               |
\\
\sphinxhline
\sphinxAtStartPar
EWLIM
&
\sphinxAtStartPar
0x34
&&
\sphinxAtStartPar
0x 00060
&
\sphinxAtStartPar
错误报警限制               |
\\
\sphinxhline
\sphinxAtStartPar
RXERR
&
\sphinxAtStartPar
0x38
&&
\sphinxAtStartPar
0x 00000
&
\sphinxAtStartPar
接收错误计数               |
\\
\sphinxhline
\sphinxAtStartPar
TXERR
&
\sphinxAtStartPar
0x3C
&&
\sphinxAtStartPar
0x 00000
&
\sphinxAtStartPar
发送错误计数               |
\\
\sphinxhline
\sphinxAtStartPar
INFO
&
\sphinxAtStartPar
0x40
&&
\sphinxAtStartPar
0x 00000
&
\sphinxAtStartPar
帧格式                     |
\\
\sphinxhline
\sphinxAtStartPar
DATA0\textasciitilde{}11
&
\sphinxAtStartPar
0x44 \textasciitilde{}0x70
&&
\sphinxAtStartPar
0x 00000
&
\sphinxAtStartPar
数据0\textasciitilde{}11寄存器             |
\\
\sphinxhline
\sphinxAtStartPar
RMCNT
&
\sphinxAtStartPar
0x74
&&
\sphinxAtStartPar
0x 00000
&
\sphinxAtStartPar
接收数据计数寄存器         |
\\
\sphinxhline
\sphinxAtStartPar
ACR0\textasciitilde{}15
&
\sphinxAtStartPar
0 x300\textasciitilde{} 0x33C
&&
\sphinxAtStartPar
0x 00000
&
\sphinxAtStartPar
验收寄存器0\textasciitilde{}15             |
\\
\sphinxhline
\sphinxAtStartPar
AMR0\textasciitilde{}15
&
\sphinxAtStartPar
0 x380\textasciitilde{} 0x3BC
&&
\sphinxAtStartPar
0x 00000
&
\sphinxAtStartPar
验收屏蔽寄存器0\textasciitilde{}15         |
\\
\sphinxbottomrule
\end{tabular}
\sphinxtableafterendhook\par
\sphinxattableend\end{savenotes}


\subsection{寄存器描述}
\label{\detokenize{SWM241/_u529f_u80fd_u63cf_u8ff0/_u5c40_u57df_u7f51_u63a7_u5236_u5668:id51}}

\subsubsection{控制寄存器CR}
\label{\detokenize{SWM241/_u529f_u80fd_u63cf_u8ff0/_u5c40_u57df_u7f51_u63a7_u5236_u5668:cr}}

\begin{savenotes}\sphinxattablestart
\sphinxthistablewithglobalstyle
\centering
\begin{tabular}[t]{\X{20}{100}\X{20}{100}\X{20}{100}\X{20}{100}\X{20}{100}}
\sphinxtoprule
\sphinxtableatstartofbodyhook
\sphinxAtStartPar
寄存器 |
&
\begin{DUlineblock}{0em}
\item[] 偏移 |
\end{DUlineblock}
&
\begin{DUlineblock}{0em}
\item[] 
\item[] {\color{red}\bfseries{}|}
\end{DUlineblock}
&
\sphinxAtStartPar
复位值 |    描 | |
&
\begin{DUlineblock}{0em}
\item[] |
  |
\end{DUlineblock}
\\
\sphinxhline
\sphinxAtStartPar
CR
&
\sphinxAtStartPar
0x00
&&
\sphinxAtStartPar
0 000001
&
\sphinxAtStartPar
控制寄存器                 |
\\
\sphinxbottomrule
\end{tabular}
\sphinxtableafterendhook\par
\sphinxattableend\end{savenotes}


\begin{savenotes}\sphinxattablestart
\sphinxthistablewithglobalstyle
\centering
\begin{tabular}[t]{\X{12}{96}\X{12}{96}\X{12}{96}\X{12}{96}\X{12}{96}\X{12}{96}\X{12}{96}\X{12}{96}}
\sphinxtoprule
\sphinxtableatstartofbodyhook
\sphinxAtStartPar
31
&
\sphinxAtStartPar
30
&
\sphinxAtStartPar
29
&
\sphinxAtStartPar
28
&
\sphinxAtStartPar
27
&
\sphinxAtStartPar
26
&
\sphinxAtStartPar
25
&
\sphinxAtStartPar
24
\\
\sphinxhline\begin{itemize}
\item {} 
\end{itemize}
&&&&&&&\\
\sphinxhline
\sphinxAtStartPar
23
&
\sphinxAtStartPar
22
&
\sphinxAtStartPar
21
&
\sphinxAtStartPar
20
&
\sphinxAtStartPar
19
&
\sphinxAtStartPar
18
&
\sphinxAtStartPar
17
&
\sphinxAtStartPar
16
\\
\sphinxhline\begin{itemize}
\item {} 
\end{itemize}
&&&&&&&\\
\sphinxhline
\sphinxAtStartPar
15
&
\sphinxAtStartPar
14
&
\sphinxAtStartPar
13
&
\sphinxAtStartPar
12
&
\sphinxAtStartPar
11
&
\sphinxAtStartPar
10
&
\sphinxAtStartPar
9
&
\sphinxAtStartPar
8
\\
\sphinxhline\begin{itemize}
\item {} 
\end{itemize}
&&&&&&&\\
\sphinxhline
\sphinxAtStartPar
7
&
\sphinxAtStartPar
6
&
\sphinxAtStartPar
5
&
\sphinxAtStartPar
4
&
\sphinxAtStartPar
3
&
\sphinxAtStartPar
2
&
\sphinxAtStartPar
1
&
\sphinxAtStartPar
0
\\
\sphinxhline\begin{itemize}
\item {} 
\end{itemize}
&&&&\begin{itemize}
\item {} 
\end{itemize}
&
\sphinxAtStartPar
STM
&
\sphinxAtStartPar
LOM
&
\sphinxAtStartPar
RST
\\
\sphinxbottomrule
\end{tabular}
\sphinxtableafterendhook\par
\sphinxattableend\end{savenotes}


\begin{savenotes}\sphinxattablestart
\sphinxthistablewithglobalstyle
\centering
\begin{tabular}[t]{\X{33}{99}\X{33}{99}\X{33}{99}}
\sphinxtoprule
\sphinxtableatstartofbodyhook
\sphinxAtStartPar
位域 |
&
\sphinxAtStartPar
名称     | |
&
\sphinxAtStartPar
描述                                        | |
\\
\sphinxhline
\sphinxAtStartPar
31:5
&\begin{itemize}
\item {} 
\end{itemize}
&\begin{itemize}
\item {} 
\end{itemize}
\\
\sphinxhline
\sphinxAtStartPar
4
&
\sphinxAtStartPar
SLEEP
&
\sphinxAtStartPar
1:进入                                     | 式,有总线活动或中断时唤醒并自动清零此位 |

\sphinxAtStartPar
0:正常模式                                 |
\\
\sphinxhline
\sphinxAtStartPar
3
&\begin{itemize}
\item {} 
\end{itemize}
&\begin{itemize}
\item {} 
\end{itemize}
\\
\sphinxhline
\sphinxAtStartPar
2
&
\sphinxAtStartPar
STM
&
\sphinxAtStartPar
1 :自                                        | ,即使没有应答,CAN控制器也可以成功发送  |

\sphinxAtStartPar
0 :正常模式,成功发送数据,需要应答信号    |
\\
\sphinxhline
\sphinxAtStartPar
1
&
\sphinxAtStartPar
LOM
&
\sphinxAtStartPar
1 :仅听模式                                |

\sphinxAtStartPar
0 :正常模式                                |
\\
\sphinxhline
\sphinxAtStartPar
0
&
\sphinxAtStartPar
RST
&
\sphinxAtStartPar
1 :复位模式                                |

\sphinxAtStartPar
0 :正常模式                                |

\sphinxAtStartPar
注:复位模                                  | 收到’1’\sphinxhyphen{}’0’跳变后,CAN控制器回到工作模式 |
\\
\sphinxbottomrule
\end{tabular}
\sphinxtableafterendhook\par
\sphinxattableend\end{savenotes}

\sphinxAtStartPar
\sphinxstyleemphasis{注:CR.SLEEP 只能在正常模式下写;CR{[}2:1{]} 在正常模式和复位模式下都可以写}


\subsubsection{命令寄存器 CMD}
\label{\detokenize{SWM241/_u529f_u80fd_u63cf_u8ff0/_u5c40_u57df_u7f51_u63a7_u5236_u5668:cmd}}

\begin{savenotes}\sphinxattablestart
\sphinxthistablewithglobalstyle
\centering
\begin{tabular}[t]{\X{20}{100}\X{20}{100}\X{20}{100}\X{20}{100}\X{20}{100}}
\sphinxtoprule
\sphinxtableatstartofbodyhook
\sphinxAtStartPar
寄存器 |
&
\begin{DUlineblock}{0em}
\item[] 偏移 |
\end{DUlineblock}
&
\begin{DUlineblock}{0em}
\item[] 
\item[] {\color{red}\bfseries{}|}
\end{DUlineblock}
&
\sphinxAtStartPar
复位值 |    描 | |
&
\begin{DUlineblock}{0em}
\item[] |
  |
\end{DUlineblock}
\\
\sphinxhline
\sphinxAtStartPar
CMD
&
\sphinxAtStartPar
0x04
&&
\sphinxAtStartPar
0 000000
&
\sphinxAtStartPar
命令寄存器                 |
\\
\sphinxbottomrule
\end{tabular}
\sphinxtableafterendhook\par
\sphinxattableend\end{savenotes}


\begin{savenotes}\sphinxattablestart
\sphinxthistablewithglobalstyle
\centering
\begin{tabular}[t]{\X{12}{96}\X{12}{96}\X{12}{96}\X{12}{96}\X{12}{96}\X{12}{96}\X{12}{96}\X{12}{96}}
\sphinxtoprule
\sphinxtableatstartofbodyhook
\sphinxAtStartPar
31
&
\sphinxAtStartPar
30
&
\sphinxAtStartPar
29
&
\sphinxAtStartPar
28
&
\sphinxAtStartPar
27
&
\sphinxAtStartPar
26
&
\sphinxAtStartPar
25
&
\sphinxAtStartPar
24
\\
\sphinxhline\begin{itemize}
\item {} 
\end{itemize}
&&&&&&&\\
\sphinxhline
\sphinxAtStartPar
23
&
\sphinxAtStartPar
22
&
\sphinxAtStartPar
21
&
\sphinxAtStartPar
20
&
\sphinxAtStartPar
19
&
\sphinxAtStartPar
18
&
\sphinxAtStartPar
17
&
\sphinxAtStartPar
16
\\
\sphinxhline\begin{itemize}
\item {} 
\end{itemize}
&&&&&&&\\
\sphinxhline
\sphinxAtStartPar
15
&
\sphinxAtStartPar
14
&
\sphinxAtStartPar
13
&
\sphinxAtStartPar
12
&
\sphinxAtStartPar
11
&
\sphinxAtStartPar
10
&
\sphinxAtStartPar
9
&
\sphinxAtStartPar
8
\\
\sphinxhline\begin{itemize}
\item {} 
\end{itemize}
&&&&&&&\\
\sphinxhline
\sphinxAtStartPar
7
&
\sphinxAtStartPar
6
&
\sphinxAtStartPar
5
&
\sphinxAtStartPar
4
&
\sphinxAtStartPar
3
&
\sphinxAtStartPar
2
&
\sphinxAtStartPar
1
&
\sphinxAtStartPar
0
\\
\sphinxhline\begin{itemize}
\item {} 
\end{itemize}
&&&
\sphinxAtStartPar
SRR
&&
\sphinxAtStartPar
RRB
&&\\
\sphinxbottomrule
\end{tabular}
\sphinxtableafterendhook\par
\sphinxattableend\end{savenotes}


\begin{savenotes}\sphinxattablestart
\sphinxthistablewithglobalstyle
\centering
\begin{tabular}[t]{\X{33}{99}\X{33}{99}\X{33}{99}}
\sphinxtoprule
\sphinxtableatstartofbodyhook
\sphinxAtStartPar
位域 |
&
\sphinxAtStartPar
名称     | |
&
\sphinxAtStartPar
描述                                        | |
\\
\sphinxhline
\sphinxAtStartPar
31:5 |
&\begin{itemize}
\item {} 
\begin{DUlineblock}{0em}
\item[] 
\end{DUlineblock}

\end{itemize}
&\begin{itemize}
\item {} 
\begin{DUlineblock}{0em}
\item[] 
\end{DUlineblock}

\end{itemize}
\\
\sphinxhline
\sphinxAtStartPar
4
&
\sphinxAtStartPar
SRR
&
\sphinxAtStartPar
1:                                         | 式下,自接收请求,数据可以同时发送和接收 |
\\
\sphinxhline
\sphinxAtStartPar
3
&
\sphinxAtStartPar
CLROV
&
\sphinxAtStartPar
1:清除数据溢出状态位                       |
\\
\sphinxhline
\sphinxAtStartPar
2
&
\sphinxAtStartPar
RRB
&
\sphinxAtStartPar
1:释放接收缓冲                             |
\\
\sphinxhline
\sphinxAtStartPar
1
&
\sphinxAtStartPar
ABTTX
&
\sphinxAtStartPar
1:取消下一个发送请求                       |
\\
\sphinxhline
\sphinxAtStartPar
0
&
\sphinxAtStartPar
TXREQ
&
\sphinxAtStartPar
1:工作模式下,发送数据请求                 |
\\
\sphinxbottomrule
\end{tabular}
\sphinxtableafterendhook\par
\sphinxattableend\end{savenotes}

\sphinxAtStartPar
注1:同时设置CMD.
ABTTX =1,CMD.
TXREQ =1,在发生总线错误和丢失仲裁的时候,数据只发送一次

\sphinxAtStartPar
注2:同时设置CMD.
SRR =1,CMD.
TXREQ =1,那么CMD.
SRR =1无效

\sphinxAtStartPar
注3:同时设置CMD ABTTX =1 CMD.
SRR =1,在发生总线错误和丢失仲裁的时候,数据只发送一次

\sphinxAtStartPar
注4:发送请求位(CMD.
TXREQ)不能通过设置CMD.
TXREQ =0 取消发送请求,只能通过设置发送终止命令(CMD.
ABTTX =1)取消

\sphinxAtStartPar
注5:命令寄存器只写,读清零。


\subsubsection{状态寄存器 SR}
\label{\detokenize{SWM241/_u529f_u80fd_u63cf_u8ff0/_u5c40_u57df_u7f51_u63a7_u5236_u5668:sr}}

\begin{savenotes}\sphinxattablestart
\sphinxthistablewithglobalstyle
\centering
\begin{tabular}[t]{\X{20}{100}\X{20}{100}\X{20}{100}\X{20}{100}\X{20}{100}}
\sphinxtoprule
\sphinxtableatstartofbodyhook
\sphinxAtStartPar
寄存器 |
&
\begin{DUlineblock}{0em}
\item[] 偏移 |
\end{DUlineblock}
&
\begin{DUlineblock}{0em}
\item[] 
\item[] {\color{red}\bfseries{}|}
\end{DUlineblock}
&
\sphinxAtStartPar
复位值 |    描 | |
&
\begin{DUlineblock}{0em}
\item[] |
  |
\end{DUlineblock}
\\
\sphinxhline
\sphinxAtStartPar
SR
&
\sphinxAtStartPar
0x08
&&
\sphinxAtStartPar
0 00003C
&
\sphinxAtStartPar
状态寄存器                 |
\\
\sphinxbottomrule
\end{tabular}
\sphinxtableafterendhook\par
\sphinxattableend\end{savenotes}


\begin{savenotes}\sphinxattablestart
\sphinxthistablewithglobalstyle
\centering
\begin{tabular}[t]{\X{12}{96}\X{12}{96}\X{12}{96}\X{12}{96}\X{12}{96}\X{12}{96}\X{12}{96}\X{12}{96}}
\sphinxtoprule
\sphinxtableatstartofbodyhook
\sphinxAtStartPar
31
&
\sphinxAtStartPar
30
&
\sphinxAtStartPar
29
&
\sphinxAtStartPar
28
&
\sphinxAtStartPar
27
&
\sphinxAtStartPar
26
&
\sphinxAtStartPar
25
&
\sphinxAtStartPar
24
\\
\sphinxhline\begin{itemize}
\item {} 
\end{itemize}
&&&&&&&\\
\sphinxhline
\sphinxAtStartPar
23
&
\sphinxAtStartPar
22
&
\sphinxAtStartPar
21
&
\sphinxAtStartPar
20
&
\sphinxAtStartPar
19
&
\sphinxAtStartPar
18
&
\sphinxAtStartPar
17
&
\sphinxAtStartPar
16
\\
\sphinxhline\begin{itemize}
\item {} 
\end{itemize}
&&&&&&&\\
\sphinxhline
\sphinxAtStartPar
15
&
\sphinxAtStartPar
14
&
\sphinxAtStartPar
13
&
\sphinxAtStartPar
12
&
\sphinxAtStartPar
11
&
\sphinxAtStartPar
10
&
\sphinxAtStartPar
9
&
\sphinxAtStartPar
8
\\
\sphinxhline\begin{itemize}
\item {} 
\end{itemize}
&&&&&&&\\
\sphinxhline
\sphinxAtStartPar
7
&
\sphinxAtStartPar
6
&
\sphinxAtStartPar
5
&
\sphinxAtStartPar
4
&
\sphinxAtStartPar
3
&
\sphinxAtStartPar
2
&
\sphinxAtStartPar
1
&
\sphinxAtStartPar
0
\\
\sphinxhline
\sphinxAtStartPar
BUSOFF
&
\sphinxAtStartPar
E RRWARN
&&&&&&\\
\sphinxbottomrule
\end{tabular}
\sphinxtableafterendhook\par
\sphinxattableend\end{savenotes}


\begin{savenotes}\sphinxattablestart
\sphinxthistablewithglobalstyle
\centering
\begin{tabular}[t]{\X{33}{99}\X{33}{99}\X{33}{99}}
\sphinxtoprule
\sphinxtableatstartofbodyhook
\sphinxAtStartPar
位域 |
&
\sphinxAtStartPar
名称     | |
&
\sphinxAtStartPar
描述                                        | |
\\
\sphinxhline
\sphinxAtStartPar
31:8 |
&\begin{itemize}
\item {} 
\begin{DUlineblock}{0em}
\item[] 
\end{DUlineblock}

\end{itemize}
&\begin{itemize}
\item {} 
\begin{DUlineblock}{0em}
\item[] 
\end{DUlineblock}

\end{itemize}
\\
\sphinxhline
\sphinxAtStartPar
7
&
\sphinxAtStartPar
BUSOFF
&
\sphinxAtStartPar
1:CAN                                      | 控制器处于总线关闭状态,没有参与到总线活动  |

\sphinxAtStartPar
0:CAN 控制器处于总线开启状态,参与总线活动 |
\\
\sphinxhline
\sphinxAtStartPar
6
&
\sphinxAtStartPar
ERRWARN
&
\sphinxAtStartPar
1 一个错误计数器达到错误限制寄存器设置的值 |

\sphinxAtStartPar
0:错误计数器的值小于错误限制寄存器设置的值 |
\\
\sphinxhline
\sphinxAtStartPar
5
&
\sphinxAtStartPar
TXBUSY
&
\sphinxAtStartPar
1:正在发送报文                             |

\sphinxAtStartPar
0:空闲                                     |
\\
\sphinxhline
\sphinxAtStartPar
4
&
\sphinxAtStartPar
RXBUSY
&
\sphinxAtStartPar
1:正在接收报文                             |

\sphinxAtStartPar
0:空闲                                     |
\\
\sphinxhline
\sphinxAtStartPar
3
&
\sphinxAtStartPar
TXOK
&
\sphinxAtStartPar
1:上一个报文发送成功完成                   |

\sphinxAtStartPar
0:上一次的报文没有成功发送                 |
\\
\sphinxhline
\sphinxAtStartPar
2
&
\sphinxAtStartPar
TXBR
&
\sphinxAtStartPar
1:可以写入新的报文发送                     |

\sphinxAtStartPar
0:正在处理前面的发送,现在不能写新的报文   |
\\
\sphinxhline
\sphinxAtStartPar
1
&
\sphinxAtStartPar
RXOV
&
\sphinxAtStartPar
1:数据                                     | 在接收FIFO里没有足够的空间导致数据的丢失 |

\sphinxAtStartPar
上一次写入清除数据溢出命令后,没有数据溢出 |
\\
\sphinxhline
\sphinxAtStartPar
0
&
\sphinxAtStartPar
RXDA
&
\sphinxAtStartPar
1:接收                                     | fer满。接收buffer里有一个或多个数据可以读取 |

\sphinxAtStartPar
0:接收buffer空。没有可读数据               |
\\
\sphinxbottomrule
\end{tabular}
\sphinxtableafterendhook\par
\sphinxattableend\end{savenotes}


\subsubsection{中断状态寄存器 IF}
\label{\detokenize{SWM241/_u529f_u80fd_u63cf_u8ff0/_u5c40_u57df_u7f51_u63a7_u5236_u5668:if}}

\begin{savenotes}\sphinxattablestart
\sphinxthistablewithglobalstyle
\centering
\begin{tabular}[t]{\X{20}{100}\X{20}{100}\X{20}{100}\X{20}{100}\X{20}{100}}
\sphinxtoprule
\sphinxtableatstartofbodyhook
\sphinxAtStartPar
寄存器 |
&
\begin{DUlineblock}{0em}
\item[] 偏移 |
\end{DUlineblock}
&
\begin{DUlineblock}{0em}
\item[] 
\item[] {\color{red}\bfseries{}|}
\end{DUlineblock}
&
\sphinxAtStartPar
复位值 |    描 | |
&
\begin{DUlineblock}{0em}
\item[] |
  |
\end{DUlineblock}
\\
\sphinxhline
\sphinxAtStartPar
IF
&
\sphinxAtStartPar
0x0C
&&
\sphinxAtStartPar
0 000000
&
\sphinxAtStartPar
中断标志寄存器             |
\\
\sphinxbottomrule
\end{tabular}
\sphinxtableafterendhook\par
\sphinxattableend\end{savenotes}


\begin{savenotes}\sphinxattablestart
\sphinxthistablewithglobalstyle
\centering
\begin{tabular}[t]{\X{12}{96}\X{12}{96}\X{12}{96}\X{12}{96}\X{12}{96}\X{12}{96}\X{12}{96}\X{12}{96}}
\sphinxtoprule
\sphinxtableatstartofbodyhook
\sphinxAtStartPar
31
&
\sphinxAtStartPar
30
&
\sphinxAtStartPar
29
&
\sphinxAtStartPar
28
&
\sphinxAtStartPar
27
&
\sphinxAtStartPar
26
&
\sphinxAtStartPar
25
&
\sphinxAtStartPar
24
\\
\sphinxhline\begin{itemize}
\item {} 
\end{itemize}
&&&&&&&\\
\sphinxhline
\sphinxAtStartPar
23
&
\sphinxAtStartPar
22
&
\sphinxAtStartPar
21
&
\sphinxAtStartPar
20
&
\sphinxAtStartPar
19
&
\sphinxAtStartPar
18
&
\sphinxAtStartPar
17
&
\sphinxAtStartPar
16
\\
\sphinxhline\begin{itemize}
\item {} 
\end{itemize}
&&&&&&&\\
\sphinxhline
\sphinxAtStartPar
15
&
\sphinxAtStartPar
14
&
\sphinxAtStartPar
13
&
\sphinxAtStartPar
12
&
\sphinxAtStartPar
11
&
\sphinxAtStartPar
10
&
\sphinxAtStartPar
9
&
\sphinxAtStartPar
8
\\
\sphinxhline\begin{itemize}
\item {} 
\end{itemize}
&&&&&&&\\
\sphinxhline
\sphinxAtStartPar
7
&
\sphinxAtStartPar
6
&
\sphinxAtStartPar
5
&
\sphinxAtStartPar
4
&
\sphinxAtStartPar
3
&
\sphinxAtStartPar
2
&
\sphinxAtStartPar
1
&
\sphinxAtStartPar
0
\\
\sphinxhline
\sphinxAtStartPar
BUSERR
&
\sphinxAtStartPar
A RBLOST
&
\sphinxAtStartPar
E ASS
&&&
\sphinxAtStartPar
E ARN
&&\\
\sphinxbottomrule
\end{tabular}
\sphinxtableafterendhook\par
\sphinxattableend\end{savenotes}


\begin{savenotes}\sphinxattablestart
\sphinxthistablewithglobalstyle
\centering
\begin{tabular}[t]{\X{33}{99}\X{33}{99}\X{33}{99}}
\sphinxtoprule
\sphinxtableatstartofbodyhook
\sphinxAtStartPar
位域 |
&
\sphinxAtStartPar
名称     | |
&
\sphinxAtStartPar
描述                                        | |
\\
\sphinxhline
\sphinxAtStartPar
31:8 |
&\begin{itemize}
\item {} 
\begin{DUlineblock}{0em}
\item[] 
\end{DUlineblock}

\end{itemize}
&\begin{itemize}
\item {} 
\begin{DUlineblock}{0em}
\item[] 
\end{DUlineblock}

\end{itemize}
\\
\sphinxhline
\sphinxAtStartPar
7
&
\sphinxAtStartPar
BUSERR
&
\sphinxAtStartPar
CAN控制器检测到总线错误                     |

\sphinxAtStartPar
1:中断已产生                               |

\sphinxAtStartPar
0:中断未产生                               |
\\
\sphinxhline
\sphinxAtStartPar
6
&
\sphinxAtStartPar
ARBLOST
&
\sphinxAtStartPar
CAN控制器丢失仲裁变成接收方                 |

\sphinxAtStartPar
1:中断已产生                               |

\sphinxAtStartPar
0:中断未产生                               |
\\
\sphinxhline
\sphinxAtStartPar
5
&
\sphinxAtStartPar
ERRPASS
&
\sphinxAtStartPar
从被动错                                    | 主动错误,或是至少一个错误计数器超过127  |

\sphinxAtStartPar
1:中断已产生                               |

\sphinxAtStartPar
0:中断未产生                               |
\\
\sphinxhline
\sphinxAtStartPar
4
&
\sphinxAtStartPar
WKUP
&
\sphinxAtStartPar
在睡眠模式下的CAN控制器检测到总线活动       |

\sphinxAtStartPar
1:中断已产生                               |

\sphinxAtStartPar
0:中断未产生                               |
\\
\sphinxhline
\sphinxAtStartPar
3
&
\sphinxAtStartPar
RXOV
&
\sphinxAtStartPar
数据溢出                                    |

\sphinxAtStartPar
1:中断已产生                               |

\sphinxAtStartPar
0:中断未产生                               |
\\
\sphinxhline
\sphinxAtStartPar
2
&
\sphinxAtStartPar
ERRWARN
&
\sphinxAtStartPar
错误(SR.
ERRWARN或SR.BUSOFF 0\sphinxhyphen{}to\sphinxhyphen{}1 或      | 1\sphinxhyphen{}to\sphinxhyphen{}0)                                    |

\sphinxAtStartPar
1:中断已产生                               |

\sphinxAtStartPar
0:中断未产生                               |
\\
\sphinxhline
\sphinxAtStartPar
1
&
\sphinxAtStartPar
TXBR
&
\sphinxAtStartPar
可以写入新的报文,发送buffer状态位(SR.
| TXRDY)从0变成1                              |

\sphinxAtStartPar
1:中断已产生                               |

\sphinxAtStartPar
0:中断未产生                               |
\\
\sphinxhline
\sphinxAtStartPar
0
&
\sphinxAtStartPar
RXDA
&\\
\sphinxbottomrule
\end{tabular}
\sphinxtableafterendhook\par
\sphinxattableend\end{savenotes}

\sphinxAtStartPar
注:各中断状态清除(除接收中断),均为读清除。对于接收中断,需要将CMD寄存器RRB位写1清除。


\subsubsection{中断使能寄存器 IE}
\label{\detokenize{SWM241/_u529f_u80fd_u63cf_u8ff0/_u5c40_u57df_u7f51_u63a7_u5236_u5668:ie}}

\begin{savenotes}\sphinxattablestart
\sphinxthistablewithglobalstyle
\centering
\begin{tabular}[t]{\X{20}{100}\X{20}{100}\X{20}{100}\X{20}{100}\X{20}{100}}
\sphinxtoprule
\sphinxtableatstartofbodyhook
\sphinxAtStartPar
寄存器 |
&
\begin{DUlineblock}{0em}
\item[] 偏移 |
\end{DUlineblock}
&
\begin{DUlineblock}{0em}
\item[] 
\item[] {\color{red}\bfseries{}|}
\end{DUlineblock}
&
\sphinxAtStartPar
复位值 |    描 | |
&
\begin{DUlineblock}{0em}
\item[] |
  |
\end{DUlineblock}
\\
\sphinxhline
\sphinxAtStartPar
IE
&
\sphinxAtStartPar
0x10
&&
\sphinxAtStartPar
0 000000
&
\sphinxAtStartPar
中断使能寄存器             |
\\
\sphinxbottomrule
\end{tabular}
\sphinxtableafterendhook\par
\sphinxattableend\end{savenotes}


\begin{savenotes}\sphinxattablestart
\sphinxthistablewithglobalstyle
\centering
\begin{tabular}[t]{\X{12}{96}\X{12}{96}\X{12}{96}\X{12}{96}\X{12}{96}\X{12}{96}\X{12}{96}\X{12}{96}}
\sphinxtoprule
\sphinxtableatstartofbodyhook
\sphinxAtStartPar
31
&
\sphinxAtStartPar
30
&
\sphinxAtStartPar
29
&
\sphinxAtStartPar
28
&
\sphinxAtStartPar
27
&
\sphinxAtStartPar
26
&
\sphinxAtStartPar
25
&
\sphinxAtStartPar
24
\\
\sphinxhline\begin{itemize}
\item {} 
\end{itemize}
&&&&&&&\\
\sphinxhline
\sphinxAtStartPar
23
&
\sphinxAtStartPar
22
&
\sphinxAtStartPar
21
&
\sphinxAtStartPar
20
&
\sphinxAtStartPar
19
&
\sphinxAtStartPar
18
&
\sphinxAtStartPar
17
&
\sphinxAtStartPar
16
\\
\sphinxhline\begin{itemize}
\item {} 
\end{itemize}
&&&&&&&\\
\sphinxhline
\sphinxAtStartPar
15
&
\sphinxAtStartPar
14
&
\sphinxAtStartPar
13
&
\sphinxAtStartPar
12
&
\sphinxAtStartPar
11
&
\sphinxAtStartPar
10
&
\sphinxAtStartPar
9
&
\sphinxAtStartPar
8
\\
\sphinxhline\begin{itemize}
\item {} 
\end{itemize}
&&&&&&&\\
\sphinxhline
\sphinxAtStartPar
7
&
\sphinxAtStartPar
6
&
\sphinxAtStartPar
5
&
\sphinxAtStartPar
4
&
\sphinxAtStartPar
3
&
\sphinxAtStartPar
2
&
\sphinxAtStartPar
1
&
\sphinxAtStartPar
0
\\
\sphinxhline
\sphinxAtStartPar
BUSERR
&
\sphinxAtStartPar
A RBLOST
&
\sphinxAtStartPar
E ASS
&&&
\sphinxAtStartPar
E ARN
&&\\
\sphinxbottomrule
\end{tabular}
\sphinxtableafterendhook\par
\sphinxattableend\end{savenotes}


\begin{savenotes}\sphinxattablestart
\sphinxthistablewithglobalstyle
\centering
\begin{tabular}[t]{\X{33}{99}\X{33}{99}\X{33}{99}}
\sphinxtoprule
\sphinxtableatstartofbodyhook
\sphinxAtStartPar
位域 |
&
\sphinxAtStartPar
名称     | |
&
\sphinxAtStartPar
描述                                        | |
\\
\sphinxhline
\sphinxAtStartPar
31:8 |
&\begin{itemize}
\item {} 
\begin{DUlineblock}{0em}
\item[] 
\end{DUlineblock}

\end{itemize}
&\begin{itemize}
\item {} 
\begin{DUlineblock}{0em}
\item[] 
\end{DUlineblock}

\end{itemize}
\\
\sphinxhline
\sphinxAtStartPar
7
&
\sphinxAtStartPar
BUSERR
&
\sphinxAtStartPar
总线错误使能                                |

\sphinxAtStartPar
1:使能                                     |

\sphinxAtStartPar
0:禁能                                     |
\\
\sphinxhline
\sphinxAtStartPar
6
&
\sphinxAtStartPar
ARBLOST
&
\sphinxAtStartPar
丢失仲裁使能                                |

\sphinxAtStartPar
1:使能                                     |

\sphinxAtStartPar
0:禁能                                     |
\\
\sphinxhline
\sphinxAtStartPar
5
&
\sphinxAtStartPar
ERRPASS
&
\sphinxAtStartPar
主动错误使能                                |

\sphinxAtStartPar
1:使能                                     |

\sphinxAtStartPar
0:禁能                                     |
\\
\sphinxhline
\sphinxAtStartPar
4
&
\sphinxAtStartPar
WKUP
&
\sphinxAtStartPar
睡眠唤醒使能                                |

\sphinxAtStartPar
1:使能                                     |

\sphinxAtStartPar
0:禁能                                     |
\\
\sphinxhline
\sphinxAtStartPar
3
&
\sphinxAtStartPar
RXOV
&
\sphinxAtStartPar
接收报文溢出使能                            |

\sphinxAtStartPar
1:使能                                     |

\sphinxAtStartPar
0:禁能                                     |
\\
\sphinxhline
\sphinxAtStartPar
2
&
\sphinxAtStartPar
ERRWARN
&
\sphinxAtStartPar
错误使能                                    |

\sphinxAtStartPar
1:使能                                     |

\sphinxAtStartPar
0:禁能                                     |
\\
\sphinxhline
\sphinxAtStartPar
1
&
\sphinxAtStartPar
TXBR
&
\sphinxAtStartPar
可以写入新的报文使能                        |

\sphinxAtStartPar
1:使能                                     |

\sphinxAtStartPar
0:禁能                                     |
\\
\sphinxhline
\sphinxAtStartPar
0
&
\sphinxAtStartPar
RXDA
&
\sphinxAtStartPar
接收中断使能                                |

\sphinxAtStartPar
1:使能                                     |

\sphinxAtStartPar
0:禁能                                     |
\\
\sphinxbottomrule
\end{tabular}
\sphinxtableafterendhook\par
\sphinxattableend\end{savenotes}


\subsubsection{总线定时器高四位寄存器BT2}
\label{\detokenize{SWM241/_u529f_u80fd_u63cf_u8ff0/_u5c40_u57df_u7f51_u63a7_u5236_u5668:bt2}}

\begin{savenotes}\sphinxattablestart
\sphinxthistablewithglobalstyle
\centering
\begin{tabular}[t]{\X{20}{100}\X{20}{100}\X{20}{100}\X{20}{100}\X{20}{100}}
\sphinxtoprule
\sphinxtableatstartofbodyhook
\sphinxAtStartPar
寄存器 |
&
\begin{DUlineblock}{0em}
\item[] 偏移 |
\end{DUlineblock}
&
\begin{DUlineblock}{0em}
\item[] 
\item[] {\color{red}\bfseries{}|}
\end{DUlineblock}
&
\sphinxAtStartPar
复位值 |    描 | |
&
\begin{DUlineblock}{0em}
\item[] |
  |
\end{DUlineblock}
\\
\sphinxhline
\sphinxAtStartPar
BT2
&
\sphinxAtStartPar
0x14
&&
\sphinxAtStartPar
0 000000
&
\sphinxAtStartPar
总线定时器高四位寄存器2    |
\\
\sphinxbottomrule
\end{tabular}
\sphinxtableafterendhook\par
\sphinxattableend\end{savenotes}


\begin{savenotes}\sphinxattablestart
\sphinxthistablewithglobalstyle
\centering
\begin{tabular}[t]{\X{12}{96}\X{12}{96}\X{12}{96}\X{12}{96}\X{12}{96}\X{12}{96}\X{12}{96}\X{12}{96}}
\sphinxtoprule
\sphinxtableatstartofbodyhook
\sphinxAtStartPar
31
&
\sphinxAtStartPar
30
&
\sphinxAtStartPar
29
&
\sphinxAtStartPar
28
&
\sphinxAtStartPar
27
&
\sphinxAtStartPar
26
&
\sphinxAtStartPar
25
&
\sphinxAtStartPar
24
\\
\sphinxhline\begin{itemize}
\item {} 
\end{itemize}
&&&&&&&\\
\sphinxhline
\sphinxAtStartPar
23
&
\sphinxAtStartPar
22
&
\sphinxAtStartPar
21
&
\sphinxAtStartPar
20
&
\sphinxAtStartPar
19
&
\sphinxAtStartPar
18
&
\sphinxAtStartPar
17
&
\sphinxAtStartPar
16
\\
\sphinxhline\begin{itemize}
\item {} 
\end{itemize}
&&&&&&&\\
\sphinxhline
\sphinxAtStartPar
15
&
\sphinxAtStartPar
14
&
\sphinxAtStartPar
13
&
\sphinxAtStartPar
12
&
\sphinxAtStartPar
11
&
\sphinxAtStartPar
10
&
\sphinxAtStartPar
9
&
\sphinxAtStartPar
8
\\
\sphinxhline\begin{itemize}
\item {} 
\end{itemize}
&&&&&&&\\
\sphinxhline
\sphinxAtStartPar
7
&
\sphinxAtStartPar
6
&
\sphinxAtStartPar
5
&
\sphinxAtStartPar
4
&
\sphinxAtStartPar
3
&
\sphinxAtStartPar
2
&
\sphinxAtStartPar
1
&
\sphinxAtStartPar
0
\\
\sphinxhline\begin{itemize}
\item {} 
\end{itemize}
&&&&
\sphinxAtStartPar
BRP
&&&\\
\sphinxbottomrule
\end{tabular}
\sphinxtableafterendhook\par
\sphinxattableend\end{savenotes}


\begin{savenotes}\sphinxattablestart
\sphinxthistablewithglobalstyle
\centering
\begin{tabular}[t]{\X{33}{99}\X{33}{99}\X{33}{99}}
\sphinxtoprule
\sphinxtableatstartofbodyhook
\sphinxAtStartPar
位域 |
&
\sphinxAtStartPar
名称     | |
&
\sphinxAtStartPar
描述                                        | |
\\
\sphinxhline
\sphinxAtStartPar
31:4
&\begin{itemize}
\item {} 
\end{itemize}
&\begin{itemize}
\item {} 
\end{itemize}
\\
\sphinxhline
\sphinxAtStartPar
3:0
&
\sphinxAtStartPar
BRP
&
\sphinxAtStartPar
Baudrate Prescale,波特率预分频值高四位     |

\sphinxAtStartPar
CAN时间单位=2*Tsysclk*(BT2.BRP\textless{}\textless{}6+ BT0.BRP  | +1)
\\
\sphinxbottomrule
\end{tabular}
\sphinxtableafterendhook\par
\sphinxattableend\end{savenotes}


\subsubsection{总线定时器BT0}
\label{\detokenize{SWM241/_u529f_u80fd_u63cf_u8ff0/_u5c40_u57df_u7f51_u63a7_u5236_u5668:bt0}}

\begin{savenotes}\sphinxattablestart
\sphinxthistablewithglobalstyle
\centering
\begin{tabular}[t]{\X{20}{100}\X{20}{100}\X{20}{100}\X{20}{100}\X{20}{100}}
\sphinxtoprule
\sphinxtableatstartofbodyhook
\sphinxAtStartPar
寄存器 |
&
\begin{DUlineblock}{0em}
\item[] 偏移 |
\end{DUlineblock}
&
\begin{DUlineblock}{0em}
\item[] 
\item[] {\color{red}\bfseries{}|}
\end{DUlineblock}
&
\sphinxAtStartPar
复位值 |    描 | |
&
\begin{DUlineblock}{0em}
\item[] |
  |
\end{DUlineblock}
\\
\sphinxhline
\sphinxAtStartPar
BT0
&
\sphinxAtStartPar
0x18
&&
\sphinxAtStartPar
0 000000
&
\sphinxAtStartPar
总线定时器0                |
\\
\sphinxbottomrule
\end{tabular}
\sphinxtableafterendhook\par
\sphinxattableend\end{savenotes}


\begin{savenotes}\sphinxattablestart
\sphinxthistablewithglobalstyle
\centering
\begin{tabular}[t]{\X{12}{96}\X{12}{96}\X{12}{96}\X{12}{96}\X{12}{96}\X{12}{96}\X{12}{96}\X{12}{96}}
\sphinxtoprule
\sphinxtableatstartofbodyhook
\sphinxAtStartPar
31
&
\sphinxAtStartPar
30
&
\sphinxAtStartPar
29
&
\sphinxAtStartPar
28
&
\sphinxAtStartPar
27
&
\sphinxAtStartPar
26
&
\sphinxAtStartPar
25
&
\sphinxAtStartPar
24
\\
\sphinxhline\begin{itemize}
\item {} 
\end{itemize}
&&&&&&&\\
\sphinxhline
\sphinxAtStartPar
23
&
\sphinxAtStartPar
22
&
\sphinxAtStartPar
21
&
\sphinxAtStartPar
20
&
\sphinxAtStartPar
19
&
\sphinxAtStartPar
18
&
\sphinxAtStartPar
17
&
\sphinxAtStartPar
16
\\
\sphinxhline\begin{itemize}
\item {} 
\end{itemize}
&&&&&&&\\
\sphinxhline
\sphinxAtStartPar
15
&
\sphinxAtStartPar
14
&
\sphinxAtStartPar
13
&
\sphinxAtStartPar
12
&
\sphinxAtStartPar
11
&
\sphinxAtStartPar
10
&
\sphinxAtStartPar
9
&
\sphinxAtStartPar
8
\\
\sphinxhline\begin{itemize}
\item {} 
\end{itemize}
&&&&&&&\\
\sphinxhline
\sphinxAtStartPar
7
&
\sphinxAtStartPar
6
&
\sphinxAtStartPar
5
&
\sphinxAtStartPar
4
&
\sphinxAtStartPar
3
&
\sphinxAtStartPar
2
&
\sphinxAtStartPar
1
&
\sphinxAtStartPar
0
\\
\sphinxhline
\sphinxAtStartPar
SJW
&&
\sphinxAtStartPar
BRP
&&&&&\\
\sphinxbottomrule
\end{tabular}
\sphinxtableafterendhook\par
\sphinxattableend\end{savenotes}


\begin{savenotes}\sphinxattablestart
\sphinxthistablewithglobalstyle
\centering
\begin{tabular}[t]{\X{33}{99}\X{33}{99}\X{33}{99}}
\sphinxtoprule
\sphinxtableatstartofbodyhook
\sphinxAtStartPar
位域 |
&
\sphinxAtStartPar
名称     | |
&
\sphinxAtStartPar
描述                                        | |
\\
\sphinxhline
\sphinxAtStartPar
31:8
&\begin{itemize}
\item {} 
\end{itemize}
&\begin{itemize}
\item {} 
\end{itemize}
\\
\sphinxhline
\sphinxAtStartPar
7:6
&
\sphinxAtStartPar
SJW
&
\sphinxAtStartPar
同步跳变宽度                                |
\\
\sphinxhline
\sphinxAtStartPar
5:0
&
\sphinxAtStartPar
BRP
&
\sphinxAtStartPar
Baudrate Prescale,波特率分频低6位          |

\sphinxAtStartPar
CAN时间单位=2*Tsysclk*(BT2.BRP\textless{}\textless{}6+ BT0.BRP  | +1)
\\
\sphinxbottomrule
\end{tabular}
\sphinxtableafterendhook\par
\sphinxattableend\end{savenotes}


\subsubsection{总线定时器BT1}
\label{\detokenize{SWM241/_u529f_u80fd_u63cf_u8ff0/_u5c40_u57df_u7f51_u63a7_u5236_u5668:bt1}}

\begin{savenotes}\sphinxattablestart
\sphinxthistablewithglobalstyle
\centering
\begin{tabular}[t]{\X{20}{100}\X{20}{100}\X{20}{100}\X{20}{100}\X{20}{100}}
\sphinxtoprule
\sphinxtableatstartofbodyhook
\sphinxAtStartPar
寄存器 |
&
\begin{DUlineblock}{0em}
\item[] 偏移 |
\end{DUlineblock}
&
\begin{DUlineblock}{0em}
\item[] 
\item[] {\color{red}\bfseries{}|}
\end{DUlineblock}
&
\sphinxAtStartPar
复位值 |    描 | |
&
\begin{DUlineblock}{0em}
\item[] |
  |
\end{DUlineblock}
\\
\sphinxhline
\sphinxAtStartPar
BT1
&
\sphinxAtStartPar
0x1C
&&
\sphinxAtStartPar
0 000000
&
\sphinxAtStartPar
总线定时器1                |
\\
\sphinxbottomrule
\end{tabular}
\sphinxtableafterendhook\par
\sphinxattableend\end{savenotes}


\begin{savenotes}\sphinxattablestart
\sphinxthistablewithglobalstyle
\centering
\begin{tabular}[t]{\X{12}{96}\X{12}{96}\X{12}{96}\X{12}{96}\X{12}{96}\X{12}{96}\X{12}{96}\X{12}{96}}
\sphinxtoprule
\sphinxtableatstartofbodyhook
\sphinxAtStartPar
31
&
\sphinxAtStartPar
30
&
\sphinxAtStartPar
29
&
\sphinxAtStartPar
28
&
\sphinxAtStartPar
27
&
\sphinxAtStartPar
26
&
\sphinxAtStartPar
25
&
\sphinxAtStartPar
24
\\
\sphinxhline\begin{itemize}
\item {} 
\end{itemize}
&&&&&&&\\
\sphinxhline
\sphinxAtStartPar
23
&
\sphinxAtStartPar
22
&
\sphinxAtStartPar
21
&
\sphinxAtStartPar
20
&
\sphinxAtStartPar
19
&
\sphinxAtStartPar
18
&
\sphinxAtStartPar
17
&
\sphinxAtStartPar
16
\\
\sphinxhline\begin{itemize}
\item {} 
\end{itemize}
&&&&&&&\\
\sphinxhline
\sphinxAtStartPar
15
&
\sphinxAtStartPar
14
&
\sphinxAtStartPar
13
&
\sphinxAtStartPar
12
&
\sphinxAtStartPar
11
&
\sphinxAtStartPar
10
&
\sphinxAtStartPar
9
&
\sphinxAtStartPar
8
\\
\sphinxhline\begin{itemize}
\item {} 
\end{itemize}
&&&&&&&\\
\sphinxhline
\sphinxAtStartPar
7
&
\sphinxAtStartPar
6
&
\sphinxAtStartPar
5
&
\sphinxAtStartPar
4
&
\sphinxAtStartPar
3
&
\sphinxAtStartPar
2
&
\sphinxAtStartPar
1
&
\sphinxAtStartPar
0
\\
\sphinxhline
\sphinxAtStartPar
SAM
&
\sphinxAtStartPar
TSEG2
&&&&&&\\
\sphinxbottomrule
\end{tabular}
\sphinxtableafterendhook\par
\sphinxattableend\end{savenotes}


\begin{savenotes}\sphinxattablestart
\sphinxthistablewithglobalstyle
\centering
\begin{tabular}[t]{\X{33}{99}\X{33}{99}\X{33}{99}}
\sphinxtoprule
\sphinxtableatstartofbodyhook
\sphinxAtStartPar
位域 |
&
\sphinxAtStartPar
名称     | |
&
\sphinxAtStartPar
描述                                        | |
\\
\sphinxhline
\sphinxAtStartPar
31:8
&\begin{itemize}
\item {} 
\end{itemize}
&\begin{itemize}
\item {} 
\end{itemize}
\\
\sphinxhline
\sphinxAtStartPar
7
&
\sphinxAtStartPar
SAM
&
\sphinxAtStartPar
采样次数                                    |

\sphinxAtStartPar
0:1次                                      |

\sphinxAtStartPar
1:3次                                      |
\\
\sphinxhline
\sphinxAtStartPar
6:4
&
\sphinxAtStartPar
TSEG2
&
\sphinxAtStartPar
t\_tseg2 = CAN时间单位 * (TSEG2+1)          |
\\
\sphinxhline
\sphinxAtStartPar
3:0
&
\sphinxAtStartPar
TSEG1
&
\sphinxAtStartPar
t\_tseg1 = CAN时间单位 * (TSEG1+1)          |
\\
\sphinxbottomrule
\end{tabular}
\sphinxtableafterendhook\par
\sphinxattableend\end{savenotes}


\subsubsection{过滤方式选择寄存器AFM}
\label{\detokenize{SWM241/_u529f_u80fd_u63cf_u8ff0/_u5c40_u57df_u7f51_u63a7_u5236_u5668:afm}}

\begin{savenotes}\sphinxattablestart
\sphinxthistablewithglobalstyle
\centering
\begin{tabular}[t]{\X{20}{100}\X{20}{100}\X{20}{100}\X{20}{100}\X{20}{100}}
\sphinxtoprule
\sphinxtableatstartofbodyhook
\sphinxAtStartPar
寄存器 |
&
\begin{DUlineblock}{0em}
\item[] 偏移 |
\end{DUlineblock}
&
\begin{DUlineblock}{0em}
\item[] 
\item[] {\color{red}\bfseries{}|}
\end{DUlineblock}
&
\sphinxAtStartPar
复位值 |    描 | |
&
\begin{DUlineblock}{0em}
\item[] |
  |
\end{DUlineblock}
\\
\sphinxhline
\sphinxAtStartPar
AFM
&
\sphinxAtStartPar
0x24
&&
\sphinxAtStartPar
0 000000
&
\sphinxAtStartPar
过滤方式选择寄存器         |
\\
\sphinxbottomrule
\end{tabular}
\sphinxtableafterendhook\par
\sphinxattableend\end{savenotes}


\begin{savenotes}\sphinxattablestart
\sphinxthistablewithglobalstyle
\centering
\begin{tabular}[t]{\X{12}{96}\X{12}{96}\X{12}{96}\X{12}{96}\X{12}{96}\X{12}{96}\X{12}{96}\X{12}{96}}
\sphinxtoprule
\sphinxtableatstartofbodyhook
\sphinxAtStartPar
31
&
\sphinxAtStartPar
30
&
\sphinxAtStartPar
29
&
\sphinxAtStartPar
28
&
\sphinxAtStartPar
27
&
\sphinxAtStartPar
26
&
\sphinxAtStartPar
25
&
\sphinxAtStartPar
24
\\
\sphinxhline\begin{itemize}
\item {} 
\end{itemize}
&&&&&&&\\
\sphinxhline
\sphinxAtStartPar
23
&
\sphinxAtStartPar
22
&
\sphinxAtStartPar
21
&
\sphinxAtStartPar
20
&
\sphinxAtStartPar
19
&
\sphinxAtStartPar
18
&
\sphinxAtStartPar
17
&
\sphinxAtStartPar
16
\\
\sphinxhline\begin{itemize}
\item {} 
\end{itemize}
&&&&&&&\\
\sphinxhline
\sphinxAtStartPar
15
&
\sphinxAtStartPar
14
&
\sphinxAtStartPar
13
&
\sphinxAtStartPar
12
&
\sphinxAtStartPar
11
&
\sphinxAtStartPar
10
&
\sphinxAtStartPar
9
&
\sphinxAtStartPar
8
\\
\sphinxhline
\sphinxAtStartPar
AFM15
&
\sphinxAtStartPar
AFM14
&&&&&&\\
\sphinxhline
\sphinxAtStartPar
7
&
\sphinxAtStartPar
6
&
\sphinxAtStartPar
5
&
\sphinxAtStartPar
4
&
\sphinxAtStartPar
3
&
\sphinxAtStartPar
2
&
\sphinxAtStartPar
1
&
\sphinxAtStartPar
0
\\
\sphinxhline
\sphinxAtStartPar
AFM7
&
\sphinxAtStartPar
AFM6
&&&&&&\\
\sphinxbottomrule
\end{tabular}
\sphinxtableafterendhook\par
\sphinxattableend\end{savenotes}


\begin{savenotes}\sphinxattablestart
\sphinxthistablewithglobalstyle
\centering
\begin{tabular}[t]{\X{33}{99}\X{33}{99}\X{33}{99}}
\sphinxtoprule
\sphinxtableatstartofbodyhook
\sphinxAtStartPar
位域 |
&
\sphinxAtStartPar
名称     | |
&
\sphinxAtStartPar
描述                                        | |
\\
\sphinxhline
\sphinxAtStartPar
31:16
&\begin{itemize}
\item {} 
\end{itemize}
&\begin{itemize}
\item {} 
\end{itemize}
\\
\sphinxhline
\sphinxAtStartPar
15
&
\sphinxAtStartPar
AFM15
&
\sphinxAtStartPar
过滤器15的滤波方式                          |

\sphinxAtStartPar
1:单滤波(32)位,只用于扩展帧             |

\sphinxAtStartPar
0:双滤波(16)位,只用于标准帧             |
\\
\sphinxhline
\sphinxAtStartPar
14
&
\sphinxAtStartPar
AFM14
&
\sphinxAtStartPar
过滤器14的滤波方式                          |

\sphinxAtStartPar
1:单滤波(32)位,只用于扩展帧             |

\sphinxAtStartPar
0:双滤波(16)位,只用于标准帧             |
\\
\sphinxhline
\sphinxAtStartPar
13
&
\sphinxAtStartPar
AFM13
&
\sphinxAtStartPar
过滤器13的滤波方式                          |

\sphinxAtStartPar
1:单滤波(32)位,只用于扩展帧             |

\sphinxAtStartPar
0:双滤波(16)位,只用于标准帧             |
\\
\sphinxhline
\sphinxAtStartPar
12
&
\sphinxAtStartPar
AFM12
&
\sphinxAtStartPar
过滤器12的滤波方式                          |

\sphinxAtStartPar
1:单滤波(32)位,只用于扩展帧             |

\sphinxAtStartPar
0:双滤波(16)位,只用于标准帧             |
\\
\sphinxhline
\sphinxAtStartPar
11
&
\sphinxAtStartPar
AFM11
&
\sphinxAtStartPar
过滤器11的滤波方式                          |

\sphinxAtStartPar
1:单滤波(32)位,只用于扩展帧             |

\sphinxAtStartPar
0:双滤波(16)位,只用于标准帧             |
\\
\sphinxhline
\sphinxAtStartPar
10
&
\sphinxAtStartPar
AFM10
&
\sphinxAtStartPar
过滤器10的滤波方式                          |

\sphinxAtStartPar
1:单滤波(32)位,只用于扩展帧             |

\sphinxAtStartPar
0:双滤波(16)位,只用于标准帧             |
\\
\sphinxhline
\sphinxAtStartPar
9
&
\sphinxAtStartPar
AFM9
&
\sphinxAtStartPar
过滤器9的滤波方式                           |

\sphinxAtStartPar
1:单滤波(32)位,只用于扩展帧             |

\sphinxAtStartPar
0:双滤波(16)位,只用于标准帧             |
\\
\sphinxhline
\sphinxAtStartPar
8
&
\sphinxAtStartPar
AFM8
&
\sphinxAtStartPar
过滤器8的滤波方式                           |

\sphinxAtStartPar
1:单滤波(32)位,只用于扩展帧             |

\sphinxAtStartPar
0:双滤波(16)位,只用于标准帧             |
\\
\sphinxhline
\sphinxAtStartPar
7
&
\sphinxAtStartPar
AFM7
&
\sphinxAtStartPar
过滤器7的滤波方式                           |

\sphinxAtStartPar
1:单滤波(32)位,只用于扩展帧             |

\sphinxAtStartPar
0:双滤波(16)位,只用于标准帧             |
\\
\sphinxhline
\sphinxAtStartPar
6
&
\sphinxAtStartPar
AFM6
&
\sphinxAtStartPar
过滤器6的滤波方式                           |

\sphinxAtStartPar
1:单滤波(32)位,只用于扩展帧             |

\sphinxAtStartPar
0:双滤波(16)位,只用于标准帧             |
\\
\sphinxhline
\sphinxAtStartPar
5
&
\sphinxAtStartPar
AFM5
&
\sphinxAtStartPar
过滤器5的滤波方式                           |

\sphinxAtStartPar
1:单滤波(32)位,只用于扩展帧             |

\sphinxAtStartPar
0:双滤波(16)位,只用于标准帧             |
\\
\sphinxhline
\sphinxAtStartPar
4
&
\sphinxAtStartPar
AFM4
&
\sphinxAtStartPar
过滤器4的滤波方式                           |

\sphinxAtStartPar
1:单滤波(32)位,只用于扩展帧             |

\sphinxAtStartPar
0:双滤波(16)位,只用于标准帧             |
\\
\sphinxhline
\sphinxAtStartPar
3
&
\sphinxAtStartPar
AFM3
&
\sphinxAtStartPar
过滤器3的滤波方式                           |

\sphinxAtStartPar
1:单滤波(32)位,只用于扩展帧             |

\sphinxAtStartPar
0:双滤波(16)位,只用于标准帧             |
\\
\sphinxhline
\sphinxAtStartPar
2
&
\sphinxAtStartPar
AFM2
&
\sphinxAtStartPar
过滤器2的滤波方式                           |

\sphinxAtStartPar
1:单滤波(32)位,只用于扩展帧             |

\sphinxAtStartPar
0:双滤波(16)位,只用于标准帧             |
\\
\sphinxhline
\sphinxAtStartPar
1
&
\sphinxAtStartPar
AFM1
&
\sphinxAtStartPar
过滤器1的滤波方式                           |

\sphinxAtStartPar
1:单滤波(32)位,只用于扩展帧             |

\sphinxAtStartPar
0:双滤波(16)位,只用于标准帧             |
\\
\sphinxhline
\sphinxAtStartPar
0
&
\sphinxAtStartPar
AFM0
&
\sphinxAtStartPar
过滤器0的滤波方式                           |

\sphinxAtStartPar
1:单滤波(32)位,只用于扩展帧             |

\sphinxAtStartPar
0:双滤波(16)位,只用于标准帧             |
\\
\sphinxbottomrule
\end{tabular}
\sphinxtableafterendhook\par
\sphinxattableend\end{savenotes}


\subsubsection{过滤使能寄存器AFE}
\label{\detokenize{SWM241/_u529f_u80fd_u63cf_u8ff0/_u5c40_u57df_u7f51_u63a7_u5236_u5668:afe}}

\begin{savenotes}\sphinxattablestart
\sphinxthistablewithglobalstyle
\centering
\begin{tabular}[t]{\X{20}{100}\X{20}{100}\X{20}{100}\X{20}{100}\X{20}{100}}
\sphinxtoprule
\sphinxtableatstartofbodyhook
\sphinxAtStartPar
寄存器 |
&
\begin{DUlineblock}{0em}
\item[] 偏移 |
\end{DUlineblock}
&
\begin{DUlineblock}{0em}
\item[] 
\item[] {\color{red}\bfseries{}|}
\end{DUlineblock}
&
\sphinxAtStartPar
复位值 |    描 | |
&
\begin{DUlineblock}{0em}
\item[] |
  |
\end{DUlineblock}
\\
\sphinxhline
\sphinxAtStartPar
AFE
&
\sphinxAtStartPar
0x28
&&
\sphinxAtStartPar
0 000000
&
\sphinxAtStartPar
过滤使能寄存器AFE          |
\\
\sphinxbottomrule
\end{tabular}
\sphinxtableafterendhook\par
\sphinxattableend\end{savenotes}


\begin{savenotes}\sphinxattablestart
\sphinxthistablewithglobalstyle
\centering
\begin{tabular}[t]{\X{12}{96}\X{12}{96}\X{12}{96}\X{12}{96}\X{12}{96}\X{12}{96}\X{12}{96}\X{12}{96}}
\sphinxtoprule
\sphinxtableatstartofbodyhook
\sphinxAtStartPar
31
&
\sphinxAtStartPar
30
&
\sphinxAtStartPar
29
&
\sphinxAtStartPar
28
&
\sphinxAtStartPar
27
&
\sphinxAtStartPar
26
&
\sphinxAtStartPar
25
&
\sphinxAtStartPar
24
\\
\sphinxhline\begin{itemize}
\item {} 
\end{itemize}
&&&&&&&\\
\sphinxhline
\sphinxAtStartPar
23
&
\sphinxAtStartPar
22
&
\sphinxAtStartPar
21
&
\sphinxAtStartPar
20
&
\sphinxAtStartPar
19
&
\sphinxAtStartPar
18
&
\sphinxAtStartPar
17
&
\sphinxAtStartPar
16
\\
\sphinxhline\begin{itemize}
\item {} 
\end{itemize}
&&&&&&&\\
\sphinxhline
\sphinxAtStartPar
15
&
\sphinxAtStartPar
14
&
\sphinxAtStartPar
13
&
\sphinxAtStartPar
12
&
\sphinxAtStartPar
11
&
\sphinxAtStartPar
10
&
\sphinxAtStartPar
9
&
\sphinxAtStartPar
8
\\
\sphinxhline
\sphinxAtStartPar
AFE15
&
\sphinxAtStartPar
AFE14
&&&&&&\\
\sphinxhline
\sphinxAtStartPar
7
&
\sphinxAtStartPar
6
&
\sphinxAtStartPar
5
&
\sphinxAtStartPar
4
&
\sphinxAtStartPar
3
&
\sphinxAtStartPar
2
&
\sphinxAtStartPar
1
&
\sphinxAtStartPar
0
\\
\sphinxhline
\sphinxAtStartPar
AFE7
&
\sphinxAtStartPar
AFE6
&&&&&&\\
\sphinxbottomrule
\end{tabular}
\sphinxtableafterendhook\par
\sphinxattableend\end{savenotes}


\begin{savenotes}\sphinxattablestart
\sphinxthistablewithglobalstyle
\centering
\begin{tabular}[t]{\X{33}{99}\X{33}{99}\X{33}{99}}
\sphinxtoprule
\sphinxtableatstartofbodyhook
\sphinxAtStartPar
位域 |
&
\sphinxAtStartPar
名称     | |
&
\sphinxAtStartPar
描述                                        | |
\\
\sphinxhline
\sphinxAtStartPar
31:16
&\begin{itemize}
\item {} 
\end{itemize}
&\begin{itemize}
\item {} 
\end{itemize}
\\
\sphinxhline
\sphinxAtStartPar
15
&
\sphinxAtStartPar
AFE15
&
\sphinxAtStartPar
滤波器15的使能                              |

\sphinxAtStartPar
1:使能滤波                                 | 据滤波规则进行包收取,默认所有滤波器使能 |

\sphinxAtStartPar
0:关闭滤滤器,不收取该滤波器对应的包       |
\\
\sphinxhline
\sphinxAtStartPar
14
&
\sphinxAtStartPar
AFE14
&
\sphinxAtStartPar
滤波器14的使能                              |

\sphinxAtStartPar
1:使能滤波                                 | 据滤波规则进行包收取,默认所有滤波器使能 |

\sphinxAtStartPar
0:关闭滤滤器,不收取该滤波器对应的包       |
\\
\sphinxhline
\sphinxAtStartPar
13
&
\sphinxAtStartPar
AFE13
&
\sphinxAtStartPar
滤波器13的使能                              |

\sphinxAtStartPar
1:使能滤波                                 | 据滤波规则进行包收取,默认所有滤波器使能 |

\sphinxAtStartPar
0:关闭滤滤器,不收取该滤波器对应的包       |
\\
\sphinxhline
\sphinxAtStartPar
12
&
\sphinxAtStartPar
AFE12
&
\sphinxAtStartPar
滤波器12的使能                              |

\sphinxAtStartPar
1:使能滤波                                 | 据滤波规则进行包收取,默认所有滤波器使能 |

\sphinxAtStartPar
0:关闭滤滤器,不收取该滤波器对应的包       |
\\
\sphinxhline
\sphinxAtStartPar
11
&
\sphinxAtStartPar
AFE11
&
\sphinxAtStartPar
滤波器11的使能                              |

\sphinxAtStartPar
1:使能滤波                                 | 据滤波规则进行包收取,默认所有滤波器使能 |

\sphinxAtStartPar
0:关闭滤滤器,不收取该滤波器对应的包       |
\\
\sphinxhline
\sphinxAtStartPar
10
&
\sphinxAtStartPar
AFE10
&
\sphinxAtStartPar
滤波器10的使能                              |

\sphinxAtStartPar
1:使能滤波                                 | 据滤波规则进行包收取,默认所有滤波器使能 |

\sphinxAtStartPar
0:关闭滤滤器,不收取该滤波器对应的包       |
\\
\sphinxhline
\sphinxAtStartPar
9
&
\sphinxAtStartPar
AFE9
&
\sphinxAtStartPar
滤波器9的使能                               |

\sphinxAtStartPar
1:使能滤波                                 | 据滤波规则进行包收取,默认所有滤波器使能 |

\sphinxAtStartPar
0:关闭滤滤器,不收取该滤波器对应的包       |
\\
\sphinxhline
\sphinxAtStartPar
8
&
\sphinxAtStartPar
AFE8
&
\sphinxAtStartPar
滤波器8的使能                               |

\sphinxAtStartPar
1:使能滤波                                 | 据滤波规则进行包收取,默认所有滤波器使能 |

\sphinxAtStartPar
0:关闭滤滤器,不收取该滤波器对应的包       |
\\
\sphinxhline
\sphinxAtStartPar
7
&
\sphinxAtStartPar
AFE7
&
\sphinxAtStartPar
滤波器7的使能                               |

\sphinxAtStartPar
1:使能滤波                                 | 据滤波规则进行包收取,默认所有滤波器使能 |

\sphinxAtStartPar
0:关闭滤滤器,不收取该滤波器对应的包       |
\\
\sphinxhline
\sphinxAtStartPar
6
&
\sphinxAtStartPar
AFE6
&
\sphinxAtStartPar
滤波器6的使能                               |

\sphinxAtStartPar
1:使能滤波                                 | 据滤波规则进行包收取,默认所有滤波器使能 |

\sphinxAtStartPar
0:关闭滤滤器,不收取该滤波器对应的包       |
\\
\sphinxhline
\sphinxAtStartPar
5
&
\sphinxAtStartPar
AFE5
&
\sphinxAtStartPar
滤波器5的使能                               |

\sphinxAtStartPar
1:使能滤波                                 | 据滤波规则进行包收取,默认所有滤波器使能 |

\sphinxAtStartPar
0:关闭滤滤器,不收取该滤波器对应的包       |
\\
\sphinxhline
\sphinxAtStartPar
4
&
\sphinxAtStartPar
AFE4
&
\sphinxAtStartPar
滤波器4的使能                               |

\sphinxAtStartPar
1:使能滤波                                 | 据滤波规则进行包收取,默认所有滤波器使能 |

\sphinxAtStartPar
0:关闭滤滤器,不收取该滤波器对应的包       |
\\
\sphinxhline
\sphinxAtStartPar
3
&
\sphinxAtStartPar
AFE3
&
\sphinxAtStartPar
滤波器3的使能                               |

\sphinxAtStartPar
1:使能滤波                                 | 据滤波规则进行包收取,默认所有滤波器使能 |

\sphinxAtStartPar
0:关闭滤滤器,不收取该滤波器对应的包       |
\\
\sphinxhline
\sphinxAtStartPar
2
&
\sphinxAtStartPar
AFE2
&
\sphinxAtStartPar
滤波器2的使能                               |

\sphinxAtStartPar
1:使能滤波                                 | 据滤波规则进行包收取,默认所有滤波器使能 |

\sphinxAtStartPar
0:关闭滤滤器,不收取该滤波器对应的包       |
\\
\sphinxhline
\sphinxAtStartPar
1
&
\sphinxAtStartPar
AFE1
&
\sphinxAtStartPar
滤波器1的使能                               |

\sphinxAtStartPar
1:使能滤波                                 | 据滤波规则进行包收取,默认所有滤波器使能 |

\sphinxAtStartPar
0:关闭滤滤器,不收取该滤波器对应的包       |
\\
\sphinxhline
\sphinxAtStartPar
0
&
\sphinxAtStartPar
AFE0
&
\sphinxAtStartPar
滤波器0的使能                               |

\sphinxAtStartPar
1:使能滤波                                 | 据滤波规则进行包收取,默认所有滤波器使能 |

\sphinxAtStartPar
0:关闭滤滤器,不收取该滤波器对应的包       |
\\
\sphinxbottomrule
\end{tabular}
\sphinxtableafterendhook\par
\sphinxattableend\end{savenotes}


\subsubsection{仲裁丢失捕捉寄存器 ALC}
\label{\detokenize{SWM241/_u529f_u80fd_u63cf_u8ff0/_u5c40_u57df_u7f51_u63a7_u5236_u5668:alc}}

\begin{savenotes}\sphinxattablestart
\sphinxthistablewithglobalstyle
\centering
\begin{tabular}[t]{\X{20}{100}\X{20}{100}\X{20}{100}\X{20}{100}\X{20}{100}}
\sphinxtoprule
\sphinxtableatstartofbodyhook
\sphinxAtStartPar
寄存器 |
&
\begin{DUlineblock}{0em}
\item[] 偏移 |
\end{DUlineblock}
&
\begin{DUlineblock}{0em}
\item[] 
\item[] {\color{red}\bfseries{}|}
\end{DUlineblock}
&
\sphinxAtStartPar
复位值 |    描 | |
&
\begin{DUlineblock}{0em}
\item[] |
  |
\end{DUlineblock}
\\
\sphinxhline
\sphinxAtStartPar
ALC
&
\sphinxAtStartPar
0x2C
&&
\sphinxAtStartPar
0 000000
&
\sphinxAtStartPar
仲裁丢失捕捉               |
\\
\sphinxbottomrule
\end{tabular}
\sphinxtableafterendhook\par
\sphinxattableend\end{savenotes}


\begin{savenotes}\sphinxattablestart
\sphinxthistablewithglobalstyle
\centering
\begin{tabular}[t]{\X{12}{96}\X{12}{96}\X{12}{96}\X{12}{96}\X{12}{96}\X{12}{96}\X{12}{96}\X{12}{96}}
\sphinxtoprule
\sphinxtableatstartofbodyhook
\sphinxAtStartPar
31
&
\sphinxAtStartPar
30
&
\sphinxAtStartPar
29
&
\sphinxAtStartPar
28
&
\sphinxAtStartPar
27
&
\sphinxAtStartPar
26
&
\sphinxAtStartPar
25
&
\sphinxAtStartPar
24
\\
\sphinxhline\begin{itemize}
\item {} 
\end{itemize}
&&&&&&&\\
\sphinxhline
\sphinxAtStartPar
23
&
\sphinxAtStartPar
22
&
\sphinxAtStartPar
21
&
\sphinxAtStartPar
20
&
\sphinxAtStartPar
19
&
\sphinxAtStartPar
18
&
\sphinxAtStartPar
17
&
\sphinxAtStartPar
16
\\
\sphinxhline\begin{itemize}
\item {} 
\end{itemize}
&&&&&&&\\
\sphinxhline
\sphinxAtStartPar
15
&
\sphinxAtStartPar
14
&
\sphinxAtStartPar
13
&
\sphinxAtStartPar
12
&
\sphinxAtStartPar
11
&
\sphinxAtStartPar
10
&
\sphinxAtStartPar
9
&
\sphinxAtStartPar
8
\\
\sphinxhline\begin{itemize}
\item {} 
\end{itemize}
&&&&&&&\\
\sphinxhline
\sphinxAtStartPar
7
&
\sphinxAtStartPar
6
&
\sphinxAtStartPar
5
&
\sphinxAtStartPar
4
&
\sphinxAtStartPar
3
&
\sphinxAtStartPar
2
&
\sphinxAtStartPar
1
&
\sphinxAtStartPar
0
\\
\sphinxhline\begin{itemize}
\item {} 
\end{itemize}
&&&
\sphinxAtStartPar
ER ode
&&&&\\
\sphinxbottomrule
\end{tabular}
\sphinxtableafterendhook\par
\sphinxattableend\end{savenotes}


\begin{savenotes}\sphinxattablestart
\sphinxthistablewithglobalstyle
\centering
\begin{tabular}[t]{\X{33}{99}\X{33}{99}\X{33}{99}}
\sphinxtoprule
\sphinxtableatstartofbodyhook
\sphinxAtStartPar
位域 |
&
\sphinxAtStartPar
名称     | |
&
\sphinxAtStartPar
描述                                        | |
\\
\sphinxhline
\sphinxAtStartPar
31:5
&\begin{itemize}
\item {} 
\end{itemize}
&\begin{itemize}
\item {} 
\end{itemize}
\\
\sphinxhline
\sphinxAtStartPar
4:0
&
\sphinxAtStartPar
ERR\_Code
&
\sphinxAtStartPar
详见下表                                    |
\\
\sphinxbottomrule
\end{tabular}
\sphinxtableafterendhook\par
\sphinxattableend\end{savenotes}


\begin{savenotes}
\sphinxatlongtablestart
\sphinxthistablewithglobalstyle
\makeatletter
  \LTleft \@totalleftmargin plus1fill
  \LTright\dimexpr\columnwidth-\@totalleftmargin-\linewidth\relax plus1fill
\makeatother
\begin{longtable}{\X{33}{99}\X{33}{99}\X{33}{99}}
\sphinxtoprule
\endfirsthead

\multicolumn{3}{c}{\sphinxnorowcolor
    \makebox[0pt]{\sphinxtablecontinued{\tablename\ \thetable{} \textendash{} continued from previous page}}%
}\\
\sphinxtoprule
\endhead

\sphinxbottomrule
\multicolumn{3}{r}{\sphinxnorowcolor
    \makebox[0pt][r]{\sphinxtablecontinued{continues on next page}}%
}\\
\endfoot

\endlastfoot
\sphinxtableatstartofbodyhook

\sphinxAtStartPar
ALC{[}4:0{]} |
&
\sphinxAtStartPar
十进制值 |  |
&
\sphinxAtStartPar
功能                                        |
\begin{quote}

\begin{DUlineblock}{0em}
\item[] 
\end{DUlineblock}
\end{quote}
\\
\sphinxhline
\sphinxAtStartPar
00000
&
\sphinxAtStartPar
00
&
\sphinxAtStartPar
仲裁丢失在识别码的bit1(ID.28)               |
\\
\sphinxhline
\sphinxAtStartPar
00001
&
\sphinxAtStartPar
01
&
\sphinxAtStartPar
仲裁丢失在识别码的bit2(ID.27)               |
\\
\sphinxhline
\sphinxAtStartPar
00010
&
\sphinxAtStartPar
02
&
\sphinxAtStartPar
仲裁丢失在识别码的bit3(ID.26)               |
\\
\sphinxhline
\sphinxAtStartPar
00011
&
\sphinxAtStartPar
03
&
\sphinxAtStartPar
仲裁丢失在识别码的bit4(ID.25)               |
\\
\sphinxhline
\sphinxAtStartPar
00100
&
\sphinxAtStartPar
04
&
\sphinxAtStartPar
仲裁丢失在识别码的bit5(ID.24)               |
\\
\sphinxhline
\sphinxAtStartPar
00101
&
\sphinxAtStartPar
05
&
\sphinxAtStartPar
仲裁丢失在识别码的bit6(ID.23)               |
\\
\sphinxhline
\sphinxAtStartPar
00110
&
\sphinxAtStartPar
06
&
\sphinxAtStartPar
仲裁丢失在识别码的bit7(ID.22)               |
\\
\sphinxhline
\sphinxAtStartPar
00111
&
\sphinxAtStartPar
07
&
\sphinxAtStartPar
仲裁丢失在识别码的bit8(ID.21)               |
\\
\sphinxhline
\sphinxAtStartPar
01000
&
\sphinxAtStartPar
08
&
\sphinxAtStartPar
仲裁丢失在识别码的bit9(ID.20)               |
\\
\sphinxhline
\sphinxAtStartPar
01001
&
\sphinxAtStartPar
09
&
\sphinxAtStartPar
仲裁丢失在识别码的bit10(ID.19)              |
\\
\sphinxhline
\sphinxAtStartPar
01010
&
\sphinxAtStartPar
10
&
\sphinxAtStartPar
仲裁丢失在识别码的bit11(ID.18)              |
\\
\sphinxhline
\sphinxAtStartPar
01011
&
\sphinxAtStartPar
11
&
\sphinxAtStartPar
仲裁丢失在SRTR位                            |
\\
\sphinxhline
\sphinxAtStartPar
01100
&
\sphinxAtStartPar
12
&
\sphinxAtStartPar
仲裁丢失在IDE位                             |
\\
\sphinxhline
\sphinxAtStartPar
01101
&
\sphinxAtStartPar
13
&
\sphinxAtStartPar
仲                                          | 在识别码的bit12(ID.17),只存在扩展帧格式 |
\\
\sphinxhline
\sphinxAtStartPar
01110
&
\sphinxAtStartPar
14
&
\sphinxAtStartPar
仲                                          | 在识别码的bit13(ID.16),只存在扩展帧格式 |
\\
\sphinxhline
\sphinxAtStartPar
01111
&
\sphinxAtStartPar
15
&
\sphinxAtStartPar
仲裁丢失在识别码的bit14(ID.15)              | ,只存在扩展帧格式                          |
\\
\sphinxhline
\sphinxAtStartPar
10000
&
\sphinxAtStartPar
16
&
\sphinxAtStartPar
仲裁丢失在识别码的bit15(ID.14)              | ,只存在扩展帧格式                          |
\\
\sphinxhline
\sphinxAtStartPar
10001
&
\sphinxAtStartPar
17
&
\sphinxAtStartPar
仲裁丢失在识别码的bit16(ID.13)              | ,只存在扩展帧格式                          |
\\
\sphinxhline
\sphinxAtStartPar
10010
&
\sphinxAtStartPar
18
&
\sphinxAtStartPar
仲裁丢失在识别码的bit17(ID.12)              | ,只存在扩展帧格式                          |
\\
\sphinxhline
\sphinxAtStartPar
10011
&
\sphinxAtStartPar
19
&
\sphinxAtStartPar
仲裁丢失在识别码的bit18(ID.11)              | ,只存在扩展帧格式                          |
\\
\sphinxhline
\sphinxAtStartPar
10100
&
\sphinxAtStartPar
20
&
\sphinxAtStartPar
仲裁丢失在识别码的bit19(ID.10)              | ,只存在扩展帧格式                          |
\\
\sphinxhline
\sphinxAtStartPar
10101
&
\sphinxAtStartPar
21
&
\sphinxAtStartPar
仲裁丢失在识别码的bit20(ID.
9)              | ,只存在扩展帧格式                          |
\\
\sphinxhline
\sphinxAtStartPar
10110
&
\sphinxAtStartPar
22
&
\sphinxAtStartPar
仲裁丢失在识别码的bit21(ID.
8)              | ,只存在扩展帧格式                          |
\\
\sphinxhline
\sphinxAtStartPar
10111
&
\sphinxAtStartPar
23
&
\sphinxAtStartPar
仲裁丢失在识别码的bit22(ID.
7)              | ,只存在扩展帧格式                          |
\\
\sphinxhline
\sphinxAtStartPar
11000
&
\sphinxAtStartPar
24
&
\sphinxAtStartPar
仲裁丢失在识别码的bit23(ID.
6)              | ,只存在扩展帧格式                          |
\\
\sphinxhline
\sphinxAtStartPar
11001
&
\sphinxAtStartPar
25
&
\sphinxAtStartPar
仲裁丢失在识别码的bit24(ID.
5)              | ,只存在扩展帧格式                          |
\\
\sphinxhline
\sphinxAtStartPar
11010
&
\sphinxAtStartPar
26
&
\sphinxAtStartPar
仲裁丢失在识别码的bit25(ID.
4)              | ,只存在扩展帧格式                          |
\\
\sphinxhline
\sphinxAtStartPar
11011
&
\sphinxAtStartPar
27
&
\sphinxAtStartPar
仲裁丢失在识别码的bit26(ID.
3)              | ,只存在扩展帧格式                          |
\\
\sphinxhline
\sphinxAtStartPar
11100
&
\sphinxAtStartPar
28
&
\sphinxAtStartPar
仲裁丢失在识别码的bit27(ID.
2)              | ,只存在扩展帧格式                          |
\\
\sphinxhline
\sphinxAtStartPar
11101
&
\sphinxAtStartPar
29
&
\sphinxAtStartPar
仲裁丢失在识别码的bit28(ID.
1)              | ,只存在扩展帧格式                          |
\\
\sphinxhline
\sphinxAtStartPar
11110
&
\sphinxAtStartPar
30
&
\sphinxAtStartPar
仲裁丢失在识别码的bit29(ID.
0)              | ,只存在扩展帧格式                          |
\\
\sphinxhline
\sphinxAtStartPar
11111
&
\sphinxAtStartPar
31
&
\sphinxAtStartPar
仲裁丢失在RTR位,只存在扩展帧格式           |
\\
\sphinxbottomrule
\end{longtable}
\sphinxtableafterendhook
\sphinxatlongtableend
\end{savenotes}


\subsubsection{错误代码 ECC}
\label{\detokenize{SWM241/_u529f_u80fd_u63cf_u8ff0/_u5c40_u57df_u7f51_u63a7_u5236_u5668:ecc}}

\begin{savenotes}\sphinxattablestart
\sphinxthistablewithglobalstyle
\centering
\begin{tabular}[t]{\X{20}{100}\X{20}{100}\X{20}{100}\X{20}{100}\X{20}{100}}
\sphinxtoprule
\sphinxtableatstartofbodyhook
\sphinxAtStartPar
寄存器 |
&
\begin{DUlineblock}{0em}
\item[] 偏移 |
\end{DUlineblock}
&
\begin{DUlineblock}{0em}
\item[] 
\item[] {\color{red}\bfseries{}|}
\end{DUlineblock}
&
\sphinxAtStartPar
复位值 |    描 | |
&
\begin{DUlineblock}{0em}
\item[] |
  |
\end{DUlineblock}
\\
\sphinxhline
\sphinxAtStartPar
ECC
&
\sphinxAtStartPar
0x30
&&
\sphinxAtStartPar
0 000000
&
\sphinxAtStartPar
错误代码捕捉               |
\\
\sphinxbottomrule
\end{tabular}
\sphinxtableafterendhook\par
\sphinxattableend\end{savenotes}


\begin{savenotes}\sphinxattablestart
\sphinxthistablewithglobalstyle
\centering
\begin{tabular}[t]{\X{12}{96}\X{12}{96}\X{12}{96}\X{12}{96}\X{12}{96}\X{12}{96}\X{12}{96}\X{12}{96}}
\sphinxtoprule
\sphinxtableatstartofbodyhook
\sphinxAtStartPar
31
&
\sphinxAtStartPar
30
&
\sphinxAtStartPar
29
&
\sphinxAtStartPar
28
&
\sphinxAtStartPar
27
&
\sphinxAtStartPar
26
&
\sphinxAtStartPar
25
&
\sphinxAtStartPar
24
\\
\sphinxhline\begin{itemize}
\item {} 
\end{itemize}
&&&&&&&\\
\sphinxhline
\sphinxAtStartPar
23
&
\sphinxAtStartPar
22
&
\sphinxAtStartPar
21
&
\sphinxAtStartPar
20
&
\sphinxAtStartPar
19
&
\sphinxAtStartPar
18
&
\sphinxAtStartPar
17
&
\sphinxAtStartPar
16
\\
\sphinxhline\begin{itemize}
\item {} 
\end{itemize}
&&&&&&&\\
\sphinxhline
\sphinxAtStartPar
15
&
\sphinxAtStartPar
14
&
\sphinxAtStartPar
13
&
\sphinxAtStartPar
12
&
\sphinxAtStartPar
11
&
\sphinxAtStartPar
10
&
\sphinxAtStartPar
9
&
\sphinxAtStartPar
8
\\
\sphinxhline\begin{itemize}
\item {} 
\end{itemize}
&&&&&&&\\
\sphinxhline
\sphinxAtStartPar
7
&
\sphinxAtStartPar
6
&
\sphinxAtStartPar
5
&
\sphinxAtStartPar
4
&
\sphinxAtStartPar
3
&
\sphinxAtStartPar
2
&
\sphinxAtStartPar
1
&
\sphinxAtStartPar
0
\\
\sphinxhline
\sphinxAtStartPar
ERRCODE
&&
\sphinxAtStartPar
DIR
&
\sphinxAtStartPar
S ODE
&&&&\\
\sphinxbottomrule
\end{tabular}
\sphinxtableafterendhook\par
\sphinxattableend\end{savenotes}


\begin{savenotes}\sphinxattablestart
\sphinxthistablewithglobalstyle
\centering
\begin{tabular}[t]{\X{33}{99}\X{33}{99}\X{33}{99}}
\sphinxtoprule
\sphinxtableatstartofbodyhook
\sphinxAtStartPar
位域 |
&
\sphinxAtStartPar
名称     | |
&
\sphinxAtStartPar
描述                                        | |
\\
\sphinxhline
\sphinxAtStartPar
31:8
&\begin{itemize}
\item {} 
\end{itemize}
&\begin{itemize}
\item {} 
\end{itemize}
\\
\sphinxhline
\sphinxAtStartPar
7:6
&
\sphinxAtStartPar
ERRCODE
&
\sphinxAtStartPar
错误代码:                                  |

\sphinxAtStartPar
00:位错误                                  |

\sphinxAtStartPar
01:格式错误                                |

\sphinxAtStartPar
10:填充错误                                |

\sphinxAtStartPar
11:其它错误                                |
\\
\sphinxhline
\sphinxAtStartPar
5
&
\sphinxAtStartPar
DIR
&
\sphinxAtStartPar
0 发送时发生错误                            |

\sphinxAtStartPar
1 接收时发生错误                            |
\\
\sphinxhline
\sphinxAtStartPar
4:0
&
\sphinxAtStartPar
SEGCODE
&
\sphinxAtStartPar
错误段码,见下表                            |
\\
\sphinxbottomrule
\end{tabular}
\sphinxtableafterendhook\par
\sphinxattableend\end{savenotes}


\begin{savenotes}\sphinxattablestart
\sphinxthistablewithglobalstyle
\centering
\begin{tabular}[t]{\X{33}{99}\X{33}{99}\X{33}{99}}
\sphinxtoprule
\sphinxtableatstartofbodyhook
\sphinxAtStartPar
ECC{[}4:0{]} |
&
\sphinxAtStartPar
十进制值 |  |
&
\sphinxAtStartPar
功能                                        |
\begin{quote}

\begin{DUlineblock}{0em}
\item[] 
\end{DUlineblock}
\end{quote}
\\
\sphinxhline
\sphinxAtStartPar
00000
&
\sphinxAtStartPar
00
&\begin{itemize}
\item {} 
\end{itemize}
\\
\sphinxhline
\sphinxAtStartPar
00001
&
\sphinxAtStartPar
01
&\begin{itemize}
\item {} 
\end{itemize}
\\
\sphinxhline
\sphinxAtStartPar
00010
&
\sphinxAtStartPar
02
&
\sphinxAtStartPar
ID28—ID21
\\
\sphinxhline
\sphinxAtStartPar
00011
&
\sphinxAtStartPar
03
&
\sphinxAtStartPar
帧开始                                      |
\\
\sphinxhline
\sphinxAtStartPar
00100
&
\sphinxAtStartPar
04
&
\sphinxAtStartPar
SRTR位                                      |
\\
\sphinxhline
\sphinxAtStartPar
00101
&
\sphinxAtStartPar
05
&
\sphinxAtStartPar
IDE位                                       |
\\
\sphinxhline
\sphinxAtStartPar
00110
&
\sphinxAtStartPar
06
&
\sphinxAtStartPar
ID20—ID18
\\
\sphinxhline
\sphinxAtStartPar
00111
&
\sphinxAtStartPar
07
&
\sphinxAtStartPar
ID17—ID13
\\
\sphinxhline
\sphinxAtStartPar
01000
&
\sphinxAtStartPar
08
&
\sphinxAtStartPar
CRC序列                                     |
\\
\sphinxhline
\sphinxAtStartPar
01001
&
\sphinxAtStartPar
09
&
\sphinxAtStartPar
保留位0                                     |
\\
\sphinxhline
\sphinxAtStartPar
01010
&
\sphinxAtStartPar
10
&
\sphinxAtStartPar
数据区                                      |
\\
\sphinxhline
\sphinxAtStartPar
01011
&
\sphinxAtStartPar
11
&
\sphinxAtStartPar
数据长度代码                                |
\\
\sphinxhline
\sphinxAtStartPar
01100
&
\sphinxAtStartPar
12
&
\sphinxAtStartPar
RTR位                                       |
\\
\sphinxhline
\sphinxAtStartPar
01101
&
\sphinxAtStartPar
13
&
\sphinxAtStartPar
保留位1                                     |
\\
\sphinxhline
\sphinxAtStartPar
01110
&
\sphinxAtStartPar
14
&
\sphinxAtStartPar
ID.4 \textendash{} ID.0
\\
\sphinxhline
\sphinxAtStartPar
01111
&
\sphinxAtStartPar
15
&
\sphinxAtStartPar
ID.12 \textendash{} ID.5
\\
\sphinxhline
\sphinxAtStartPar
10000
&
\sphinxAtStartPar
16
&\begin{itemize}
\item {} 
\end{itemize}
\\
\sphinxhline
\sphinxAtStartPar
10001
&
\sphinxAtStartPar
17
&
\sphinxAtStartPar
积极错误标志                                |
\\
\sphinxhline
\sphinxAtStartPar
10010
&
\sphinxAtStartPar
18
&\begin{itemize}
\item {} 
\end{itemize}
\\
\sphinxhline
\sphinxAtStartPar
10011
&
\sphinxAtStartPar
19
&
\sphinxAtStartPar
支配(控制)位误差                          |
\\
\sphinxhline
\sphinxAtStartPar
10100
&
\sphinxAtStartPar
20
&\begin{itemize}
\item {} 
\end{itemize}
\\
\sphinxhline
\sphinxAtStartPar
10101
&
\sphinxAtStartPar
21
&\begin{itemize}
\item {} 
\end{itemize}
\\
\sphinxhline
\sphinxAtStartPar
10110
&
\sphinxAtStartPar
22
&
\sphinxAtStartPar
消极错误标志                                |
\\
\sphinxhline
\sphinxAtStartPar
10111
&
\sphinxAtStartPar
23
&
\sphinxAtStartPar
错误定义符                                  |
\\
\sphinxhline
\sphinxAtStartPar
11000
&
\sphinxAtStartPar
24
&
\sphinxAtStartPar
CRC定义符                                   |
\\
\sphinxhline
\sphinxAtStartPar
11001
&
\sphinxAtStartPar
25
&
\sphinxAtStartPar
应答通道                                    |
\\
\sphinxhline
\sphinxAtStartPar
11010
&
\sphinxAtStartPar
26
&
\sphinxAtStartPar
帧结束                                      |
\\
\sphinxhline
\sphinxAtStartPar
11011
&
\sphinxAtStartPar
27
&
\sphinxAtStartPar
应答定义符                                  |
\\
\sphinxhline
\sphinxAtStartPar
11100
&
\sphinxAtStartPar
28
&
\sphinxAtStartPar
溢出标志                                    |
\\
\sphinxbottomrule
\end{tabular}
\sphinxtableafterendhook\par
\sphinxattableend\end{savenotes}


\subsubsection{错误报警限制EWLIM}
\label{\detokenize{SWM241/_u529f_u80fd_u63cf_u8ff0/_u5c40_u57df_u7f51_u63a7_u5236_u5668:ewlim}}

\begin{savenotes}\sphinxattablestart
\sphinxthistablewithglobalstyle
\centering
\begin{tabular}[t]{\X{20}{100}\X{20}{100}\X{20}{100}\X{20}{100}\X{20}{100}}
\sphinxtoprule
\sphinxtableatstartofbodyhook
\sphinxAtStartPar
寄存器 |
&
\begin{DUlineblock}{0em}
\item[] 偏移 |
\end{DUlineblock}
&
\begin{DUlineblock}{0em}
\item[] 
\item[] {\color{red}\bfseries{}|}
\end{DUlineblock}
&
\sphinxAtStartPar
复位值 |    描 | |
&
\begin{DUlineblock}{0em}
\item[] |
  |
\end{DUlineblock}
\\
\sphinxhline
\sphinxAtStartPar
EWLIM
&
\sphinxAtStartPar
0x34
&&
\sphinxAtStartPar
0 000060
&
\sphinxAtStartPar
错误报警限制               |
\\
\sphinxbottomrule
\end{tabular}
\sphinxtableafterendhook\par
\sphinxattableend\end{savenotes}


\begin{savenotes}\sphinxattablestart
\sphinxthistablewithglobalstyle
\centering
\begin{tabular}[t]{\X{12}{96}\X{12}{96}\X{12}{96}\X{12}{96}\X{12}{96}\X{12}{96}\X{12}{96}\X{12}{96}}
\sphinxtoprule
\sphinxtableatstartofbodyhook
\sphinxAtStartPar
31
&
\sphinxAtStartPar
30
&
\sphinxAtStartPar
29
&
\sphinxAtStartPar
28
&
\sphinxAtStartPar
27
&
\sphinxAtStartPar
26
&
\sphinxAtStartPar
25
&
\sphinxAtStartPar
24
\\
\sphinxhline\begin{itemize}
\item {} 
\end{itemize}
&&&&&&&\\
\sphinxhline
\sphinxAtStartPar
23
&
\sphinxAtStartPar
22
&
\sphinxAtStartPar
21
&
\sphinxAtStartPar
20
&
\sphinxAtStartPar
19
&
\sphinxAtStartPar
18
&
\sphinxAtStartPar
17
&
\sphinxAtStartPar
16
\\
\sphinxhline\begin{itemize}
\item {} 
\end{itemize}
&&&&&&&\\
\sphinxhline
\sphinxAtStartPar
15
&
\sphinxAtStartPar
14
&
\sphinxAtStartPar
13
&
\sphinxAtStartPar
12
&
\sphinxAtStartPar
11
&
\sphinxAtStartPar
10
&
\sphinxAtStartPar
9
&
\sphinxAtStartPar
8
\\
\sphinxhline\begin{itemize}
\item {} 
\end{itemize}
&&&&&&&\\
\sphinxhline
\sphinxAtStartPar
7
&
\sphinxAtStartPar
6
&
\sphinxAtStartPar
5
&
\sphinxAtStartPar
4
&
\sphinxAtStartPar
3
&
\sphinxAtStartPar
2
&
\sphinxAtStartPar
1
&
\sphinxAtStartPar
0
\\
\sphinxhline
\sphinxAtStartPar
EWLIM
&&&&&&&\\
\sphinxbottomrule
\end{tabular}
\sphinxtableafterendhook\par
\sphinxattableend\end{savenotes}


\begin{savenotes}\sphinxattablestart
\sphinxthistablewithglobalstyle
\centering
\begin{tabular}[t]{\X{33}{99}\X{33}{99}\X{33}{99}}
\sphinxtoprule
\sphinxtableatstartofbodyhook
\sphinxAtStartPar
位域 |
&
\sphinxAtStartPar
名称     | |
&
\sphinxAtStartPar
描述                                        | |
\\
\sphinxhline
\sphinxAtStartPar
31:8
&\begin{itemize}
\item {} 
\end{itemize}
&\begin{itemize}
\item {} 
\end{itemize}
\\
\sphinxhline
\sphinxAtStartPar
7:0
&
\sphinxAtStartPar
EWLIM
&
\sphinxAtStartPar
注意:在复位模式下可读可写                  |

\sphinxAtStartPar
在正常模式下只读                            |
\\
\sphinxbottomrule
\end{tabular}
\sphinxtableafterendhook\par
\sphinxattableend\end{savenotes}


\subsubsection{接收错误计数器 RXERR}
\label{\detokenize{SWM241/_u529f_u80fd_u63cf_u8ff0/_u5c40_u57df_u7f51_u63a7_u5236_u5668:rxerr}}

\begin{savenotes}\sphinxattablestart
\sphinxthistablewithglobalstyle
\centering
\begin{tabular}[t]{\X{20}{100}\X{20}{100}\X{20}{100}\X{20}{100}\X{20}{100}}
\sphinxtoprule
\sphinxtableatstartofbodyhook
\sphinxAtStartPar
寄存器 |
&
\begin{DUlineblock}{0em}
\item[] 偏移 |
\end{DUlineblock}
&
\begin{DUlineblock}{0em}
\item[] 
\item[] {\color{red}\bfseries{}|}
\end{DUlineblock}
&
\sphinxAtStartPar
复位值 |    描 | |
&
\begin{DUlineblock}{0em}
\item[] |
  |
\end{DUlineblock}
\\
\sphinxhline
\sphinxAtStartPar
RXERR
&
\sphinxAtStartPar
0x38
&&
\sphinxAtStartPar
0 000000
&
\sphinxAtStartPar
接收错误计数               |
\\
\sphinxbottomrule
\end{tabular}
\sphinxtableafterendhook\par
\sphinxattableend\end{savenotes}


\begin{savenotes}\sphinxattablestart
\sphinxthistablewithglobalstyle
\centering
\begin{tabular}[t]{\X{12}{96}\X{12}{96}\X{12}{96}\X{12}{96}\X{12}{96}\X{12}{96}\X{12}{96}\X{12}{96}}
\sphinxtoprule
\sphinxtableatstartofbodyhook
\sphinxAtStartPar
31
&
\sphinxAtStartPar
30
&
\sphinxAtStartPar
29
&
\sphinxAtStartPar
28
&
\sphinxAtStartPar
27
&
\sphinxAtStartPar
26
&
\sphinxAtStartPar
25
&
\sphinxAtStartPar
24
\\
\sphinxhline\begin{itemize}
\item {} 
\end{itemize}
&&&&&&&\\
\sphinxhline
\sphinxAtStartPar
23
&
\sphinxAtStartPar
22
&
\sphinxAtStartPar
21
&
\sphinxAtStartPar
20
&
\sphinxAtStartPar
19
&
\sphinxAtStartPar
18
&
\sphinxAtStartPar
17
&
\sphinxAtStartPar
16
\\
\sphinxhline\begin{itemize}
\item {} 
\end{itemize}
&&&&&&&\\
\sphinxhline
\sphinxAtStartPar
15
&
\sphinxAtStartPar
14
&
\sphinxAtStartPar
13
&
\sphinxAtStartPar
12
&
\sphinxAtStartPar
11
&
\sphinxAtStartPar
10
&
\sphinxAtStartPar
9
&
\sphinxAtStartPar
8
\\
\sphinxhline\begin{itemize}
\item {} 
\end{itemize}
&&&&&&&\\
\sphinxhline
\sphinxAtStartPar
7
&
\sphinxAtStartPar
6
&
\sphinxAtStartPar
5
&
\sphinxAtStartPar
4
&
\sphinxAtStartPar
3
&
\sphinxAtStartPar
2
&
\sphinxAtStartPar
1
&
\sphinxAtStartPar
0
\\
\sphinxhline
\sphinxAtStartPar
ERRCNT
&&&&&&&\\
\sphinxbottomrule
\end{tabular}
\sphinxtableafterendhook\par
\sphinxattableend\end{savenotes}


\begin{savenotes}\sphinxattablestart
\sphinxthistablewithglobalstyle
\centering
\begin{tabular}[t]{\X{33}{99}\X{33}{99}\X{33}{99}}
\sphinxtoprule
\sphinxtableatstartofbodyhook
\sphinxAtStartPar
位域 |
&
\sphinxAtStartPar
名称     | |
&
\sphinxAtStartPar
描述                                        | |
\\
\sphinxhline
\sphinxAtStartPar
31:8
&\begin{itemize}
\item {} 
\end{itemize}
&\begin{itemize}
\item {} 
\end{itemize}
\\
\sphinxhline
\sphinxAtStartPar
7:0
&
\sphinxAtStartPar
ERRCNT
&
\sphinxAtStartPar
当前接收错误计数器的值                      |

\sphinxAtStartPar
注意:在复位模式下可读可写                  |

\sphinxAtStartPar
在正常模式下只读                            |
\\
\sphinxbottomrule
\end{tabular}
\sphinxtableafterendhook\par
\sphinxattableend\end{savenotes}


\subsubsection{发送错误计数器 TXERR}
\label{\detokenize{SWM241/_u529f_u80fd_u63cf_u8ff0/_u5c40_u57df_u7f51_u63a7_u5236_u5668:txerr}}

\begin{savenotes}\sphinxattablestart
\sphinxthistablewithglobalstyle
\centering
\begin{tabular}[t]{\X{20}{100}\X{20}{100}\X{20}{100}\X{20}{100}\X{20}{100}}
\sphinxtoprule
\sphinxtableatstartofbodyhook
\sphinxAtStartPar
寄存器 |
&
\begin{DUlineblock}{0em}
\item[] 偏移 |
\end{DUlineblock}
&
\begin{DUlineblock}{0em}
\item[] 
\item[] {\color{red}\bfseries{}|}
\end{DUlineblock}
&
\sphinxAtStartPar
复位值 |    描 | |
&
\begin{DUlineblock}{0em}
\item[] |
  |
\end{DUlineblock}
\\
\sphinxhline
\sphinxAtStartPar
TXERR
&
\sphinxAtStartPar
0x3C
&&
\sphinxAtStartPar
0 000000
&
\sphinxAtStartPar
发送错误计数               |
\\
\sphinxbottomrule
\end{tabular}
\sphinxtableafterendhook\par
\sphinxattableend\end{savenotes}


\begin{savenotes}\sphinxattablestart
\sphinxthistablewithglobalstyle
\centering
\begin{tabular}[t]{\X{12}{96}\X{12}{96}\X{12}{96}\X{12}{96}\X{12}{96}\X{12}{96}\X{12}{96}\X{12}{96}}
\sphinxtoprule
\sphinxtableatstartofbodyhook
\sphinxAtStartPar
31
&
\sphinxAtStartPar
30
&
\sphinxAtStartPar
29
&
\sphinxAtStartPar
28
&
\sphinxAtStartPar
27
&
\sphinxAtStartPar
26
&
\sphinxAtStartPar
25
&
\sphinxAtStartPar
24
\\
\sphinxhline\begin{itemize}
\item {} 
\end{itemize}
&&&&&&&\\
\sphinxhline
\sphinxAtStartPar
23
&
\sphinxAtStartPar
22
&
\sphinxAtStartPar
21
&
\sphinxAtStartPar
20
&
\sphinxAtStartPar
19
&
\sphinxAtStartPar
18
&
\sphinxAtStartPar
17
&
\sphinxAtStartPar
16
\\
\sphinxhline\begin{itemize}
\item {} 
\end{itemize}
&&&&&&&\\
\sphinxhline
\sphinxAtStartPar
15
&
\sphinxAtStartPar
14
&
\sphinxAtStartPar
13
&
\sphinxAtStartPar
12
&
\sphinxAtStartPar
11
&
\sphinxAtStartPar
10
&
\sphinxAtStartPar
9
&
\sphinxAtStartPar
8
\\
\sphinxhline\begin{itemize}
\item {} 
\end{itemize}
&&&&&&&\\
\sphinxhline
\sphinxAtStartPar
7
&
\sphinxAtStartPar
6
&
\sphinxAtStartPar
5
&
\sphinxAtStartPar
4
&
\sphinxAtStartPar
3
&
\sphinxAtStartPar
2
&
\sphinxAtStartPar
1
&
\sphinxAtStartPar
0
\\
\sphinxhline
\sphinxAtStartPar
ERRCNT
&&&&&&&\\
\sphinxbottomrule
\end{tabular}
\sphinxtableafterendhook\par
\sphinxattableend\end{savenotes}


\begin{savenotes}\sphinxattablestart
\sphinxthistablewithglobalstyle
\centering
\begin{tabular}[t]{\X{33}{99}\X{33}{99}\X{33}{99}}
\sphinxtoprule
\sphinxtableatstartofbodyhook
\sphinxAtStartPar
位域 |
&
\sphinxAtStartPar
名称     | |
&
\sphinxAtStartPar
描述                                        | |
\\
\sphinxhline
\sphinxAtStartPar
31:8
&\begin{itemize}
\item {} 
\end{itemize}
&\begin{itemize}
\item {} 
\end{itemize}
\\
\sphinxhline
\sphinxAtStartPar
7:0
&
\sphinxAtStartPar
ERRCNT
&
\sphinxAtStartPar
发送错误计数器当前值                        |

\sphinxAtStartPar
注意:在复位模式下可读可写                  |

\sphinxAtStartPar
在正常模式下只读                            |
\\
\sphinxbottomrule
\end{tabular}
\sphinxtableafterendhook\par
\sphinxattableend\end{savenotes}


\subsubsection{帧信息寄存器INFO}
\label{\detokenize{SWM241/_u529f_u80fd_u63cf_u8ff0/_u5c40_u57df_u7f51_u63a7_u5236_u5668:info}}

\begin{savenotes}\sphinxattablestart
\sphinxthistablewithglobalstyle
\centering
\begin{tabular}[t]{\X{20}{100}\X{20}{100}\X{20}{100}\X{20}{100}\X{20}{100}}
\sphinxtoprule
\sphinxtableatstartofbodyhook
\sphinxAtStartPar
寄存器 |
&
\begin{DUlineblock}{0em}
\item[] 偏移 |
\end{DUlineblock}
&
\begin{DUlineblock}{0em}
\item[] 
\item[] {\color{red}\bfseries{}|}
\end{DUlineblock}
&
\sphinxAtStartPar
复位值 |    描 | |
&
\begin{DUlineblock}{0em}
\item[] |
  |
\end{DUlineblock}
\\
\sphinxhline
\sphinxAtStartPar
INFO
&
\sphinxAtStartPar
0x40
&&
\sphinxAtStartPar
0 000000
&
\sphinxAtStartPar
帧格式                     |
\\
\sphinxbottomrule
\end{tabular}
\sphinxtableafterendhook\par
\sphinxattableend\end{savenotes}


\begin{savenotes}\sphinxattablestart
\sphinxthistablewithglobalstyle
\centering
\begin{tabular}[t]{\X{12}{96}\X{12}{96}\X{12}{96}\X{12}{96}\X{12}{96}\X{12}{96}\X{12}{96}\X{12}{96}}
\sphinxtoprule
\sphinxtableatstartofbodyhook
\sphinxAtStartPar
31
&
\sphinxAtStartPar
30
&
\sphinxAtStartPar
29
&
\sphinxAtStartPar
28
&
\sphinxAtStartPar
27
&
\sphinxAtStartPar
26
&
\sphinxAtStartPar
25
&
\sphinxAtStartPar
24
\\
\sphinxhline\begin{itemize}
\item {} 
\end{itemize}
&&&&&&&\\
\sphinxhline
\sphinxAtStartPar
23
&
\sphinxAtStartPar
22
&
\sphinxAtStartPar
21
&
\sphinxAtStartPar
20
&
\sphinxAtStartPar
19
&
\sphinxAtStartPar
18
&
\sphinxAtStartPar
17
&
\sphinxAtStartPar
16
\\
\sphinxhline\begin{itemize}
\item {} 
\end{itemize}
&&&&&&&\\
\sphinxhline
\sphinxAtStartPar
15
&
\sphinxAtStartPar
14
&
\sphinxAtStartPar
13
&
\sphinxAtStartPar
12
&
\sphinxAtStartPar
11
&
\sphinxAtStartPar
10
&
\sphinxAtStartPar
9
&
\sphinxAtStartPar
8
\\
\sphinxhline\begin{itemize}
\item {} 
\end{itemize}
&&&&&&&\\
\sphinxhline
\sphinxAtStartPar
7
&
\sphinxAtStartPar
6
&
\sphinxAtStartPar
5
&
\sphinxAtStartPar
4
&
\sphinxAtStartPar
3
&
\sphinxAtStartPar
2
&
\sphinxAtStartPar
1
&
\sphinxAtStartPar
0
\\
\sphinxhline
\sphinxAtStartPar
FF
&
\sphinxAtStartPar
RTR
&\begin{itemize}
\item {} 
\end{itemize}
&&
\sphinxAtStartPar
DLC
&&&\\
\sphinxbottomrule
\end{tabular}
\sphinxtableafterendhook\par
\sphinxattableend\end{savenotes}


\begin{savenotes}\sphinxattablestart
\sphinxthistablewithglobalstyle
\centering
\begin{tabular}[t]{\X{33}{99}\X{33}{99}\X{33}{99}}
\sphinxtoprule
\sphinxtableatstartofbodyhook
\sphinxAtStartPar
位域 |
&
\sphinxAtStartPar
名称     | |
&
\sphinxAtStartPar
描述                                        | |
\\
\sphinxhline
\sphinxAtStartPar
31:8
&\begin{itemize}
\item {} 
\end{itemize}
&\begin{itemize}
\item {} 
\end{itemize}
\\
\sphinxhline
\sphinxAtStartPar
7
&
\sphinxAtStartPar
FF
&
\sphinxAtStartPar
帧格式                                      |

\sphinxAtStartPar
0 标准帧格式                                |

\sphinxAtStartPar
1 扩展帧格式                                |
\\
\sphinxhline
\sphinxAtStartPar
6
&
\sphinxAtStartPar
RTR
&
\sphinxAtStartPar
帧格式                                      |

\sphinxAtStartPar
0 数据帧                                    |

\sphinxAtStartPar
1 远程帧                                    |
\\
\sphinxhline
\sphinxAtStartPar
5:4
&\begin{itemize}
\item {} 
\end{itemize}
&\begin{itemize}
\item {} 
\end{itemize}
\\
\sphinxhline
\sphinxAtStartPar
3:0
&
\sphinxAtStartPar
DLC
&
\sphinxAtStartPar
数据长度                                    |
\\
\sphinxbottomrule
\end{tabular}
\sphinxtableafterendhook\par
\sphinxattableend\end{savenotes}


\subsubsection{\textless{}标准帧格式\textgreater{}数据寄存器0 DATA0}
\label{\detokenize{SWM241/_u529f_u80fd_u63cf_u8ff0/_u5c40_u57df_u7f51_u63a7_u5236_u5668:data0}}

\begin{savenotes}\sphinxattablestart
\sphinxthistablewithglobalstyle
\centering
\begin{tabular}[t]{\X{20}{100}\X{20}{100}\X{20}{100}\X{20}{100}\X{20}{100}}
\sphinxtoprule
\sphinxtableatstartofbodyhook
\sphinxAtStartPar
寄存器 |
&
\begin{DUlineblock}{0em}
\item[] 偏移 |
\end{DUlineblock}
&
\begin{DUlineblock}{0em}
\item[] 
\item[] {\color{red}\bfseries{}|}
\end{DUlineblock}
&
\sphinxAtStartPar
复位值 |    描 | |
&
\begin{DUlineblock}{0em}
\item[] |
  |
\end{DUlineblock}
\\
\sphinxhline
\sphinxAtStartPar
DATA0
&
\sphinxAtStartPar
0x44
&&
\sphinxAtStartPar
0 000000
&
\sphinxAtStartPar
数据0寄存器                |
\\
\sphinxbottomrule
\end{tabular}
\sphinxtableafterendhook\par
\sphinxattableend\end{savenotes}


\begin{savenotes}\sphinxattablestart
\sphinxthistablewithglobalstyle
\centering
\begin{tabular}[t]{\X{12}{96}\X{12}{96}\X{12}{96}\X{12}{96}\X{12}{96}\X{12}{96}\X{12}{96}\X{12}{96}}
\sphinxtoprule
\sphinxtableatstartofbodyhook
\sphinxAtStartPar
31
&
\sphinxAtStartPar
30
&
\sphinxAtStartPar
29
&
\sphinxAtStartPar
28
&
\sphinxAtStartPar
27
&
\sphinxAtStartPar
26
&
\sphinxAtStartPar
25
&
\sphinxAtStartPar
24
\\
\sphinxhline\begin{itemize}
\item {} 
\end{itemize}
&&&&&&&\\
\sphinxhline
\sphinxAtStartPar
23
&
\sphinxAtStartPar
22
&
\sphinxAtStartPar
21
&
\sphinxAtStartPar
20
&
\sphinxAtStartPar
19
&
\sphinxAtStartPar
18
&
\sphinxAtStartPar
17
&
\sphinxAtStartPar
16
\\
\sphinxhline\begin{itemize}
\item {} 
\end{itemize}
&&&&&&&\\
\sphinxhline
\sphinxAtStartPar
15
&
\sphinxAtStartPar
14
&
\sphinxAtStartPar
13
&
\sphinxAtStartPar
12
&
\sphinxAtStartPar
11
&
\sphinxAtStartPar
10
&
\sphinxAtStartPar
9
&
\sphinxAtStartPar
8
\\
\sphinxhline\begin{itemize}
\item {} 
\end{itemize}
&&&&&&&\\
\sphinxhline
\sphinxAtStartPar
7
&
\sphinxAtStartPar
6
&
\sphinxAtStartPar
5
&
\sphinxAtStartPar
4
&
\sphinxAtStartPar
3
&
\sphinxAtStartPar
2
&
\sphinxAtStartPar
1
&
\sphinxAtStartPar
0
\\
\sphinxhline
\sphinxAtStartPar
ID
&&&&&&&\\
\sphinxbottomrule
\end{tabular}
\sphinxtableafterendhook\par
\sphinxattableend\end{savenotes}


\begin{savenotes}\sphinxattablestart
\sphinxthistablewithglobalstyle
\centering
\begin{tabular}[t]{\X{33}{99}\X{33}{99}\X{33}{99}}
\sphinxtoprule
\sphinxtableatstartofbodyhook
\sphinxAtStartPar
位域 |
&
\sphinxAtStartPar
名称     | |
&
\sphinxAtStartPar
描述                                        | |
\\
\sphinxhline
\sphinxAtStartPar
31:8
&\begin{itemize}
\item {} 
\end{itemize}
&\begin{itemize}
\item {} 
\end{itemize}
\\
\sphinxhline
\sphinxAtStartPar
7:0
&
\sphinxAtStartPar
ID
&
\sphinxAtStartPar
标识符ID{[}28:21{]}                            |
\\
\sphinxbottomrule
\end{tabular}
\sphinxtableafterendhook\par
\sphinxattableend\end{savenotes}


\subsubsection{\textless{}标准帧格式\textgreater{}数据寄存器1 DATA1}
\label{\detokenize{SWM241/_u529f_u80fd_u63cf_u8ff0/_u5c40_u57df_u7f51_u63a7_u5236_u5668:data1}}

\begin{savenotes}\sphinxattablestart
\sphinxthistablewithglobalstyle
\centering
\begin{tabular}[t]{\X{20}{100}\X{20}{100}\X{20}{100}\X{20}{100}\X{20}{100}}
\sphinxtoprule
\sphinxtableatstartofbodyhook
\sphinxAtStartPar
寄存器 |
&
\begin{DUlineblock}{0em}
\item[] 偏移 |
\end{DUlineblock}
&
\begin{DUlineblock}{0em}
\item[] 
\item[] {\color{red}\bfseries{}|}
\end{DUlineblock}
&
\sphinxAtStartPar
复位值 |    描 | |
&
\begin{DUlineblock}{0em}
\item[] |
  |
\end{DUlineblock}
\\
\sphinxhline
\sphinxAtStartPar
DATA1
&
\sphinxAtStartPar
0x48
&&
\sphinxAtStartPar
0 000000
&
\sphinxAtStartPar
数据1寄存器                |
\\
\sphinxbottomrule
\end{tabular}
\sphinxtableafterendhook\par
\sphinxattableend\end{savenotes}


\begin{savenotes}\sphinxattablestart
\sphinxthistablewithglobalstyle
\centering
\begin{tabular}[t]{\X{12}{96}\X{12}{96}\X{12}{96}\X{12}{96}\X{12}{96}\X{12}{96}\X{12}{96}\X{12}{96}}
\sphinxtoprule
\sphinxtableatstartofbodyhook
\sphinxAtStartPar
31
&
\sphinxAtStartPar
30
&
\sphinxAtStartPar
29
&
\sphinxAtStartPar
28
&
\sphinxAtStartPar
27
&
\sphinxAtStartPar
26
&
\sphinxAtStartPar
25
&
\sphinxAtStartPar
24
\\
\sphinxhline\begin{itemize}
\item {} 
\end{itemize}
&&&&&&&\\
\sphinxhline
\sphinxAtStartPar
23
&
\sphinxAtStartPar
22
&
\sphinxAtStartPar
21
&
\sphinxAtStartPar
20
&
\sphinxAtStartPar
19
&
\sphinxAtStartPar
18
&
\sphinxAtStartPar
17
&
\sphinxAtStartPar
16
\\
\sphinxhline\begin{itemize}
\item {} 
\end{itemize}
&&&&&&&\\
\sphinxhline
\sphinxAtStartPar
15
&
\sphinxAtStartPar
14
&
\sphinxAtStartPar
13
&
\sphinxAtStartPar
12
&
\sphinxAtStartPar
11
&
\sphinxAtStartPar
10
&
\sphinxAtStartPar
9
&
\sphinxAtStartPar
8
\\
\sphinxhline\begin{itemize}
\item {} 
\end{itemize}
&&&&&&&\\
\sphinxhline
\sphinxAtStartPar
7
&
\sphinxAtStartPar
6
&
\sphinxAtStartPar
5
&
\sphinxAtStartPar
4
&
\sphinxAtStartPar
3
&
\sphinxAtStartPar
2
&
\sphinxAtStartPar
1
&
\sphinxAtStartPar
0
\\
\sphinxhline
\sphinxAtStartPar
ID
&&&\begin{itemize}
\item {} 
\end{itemize}
&&&&\\
\sphinxbottomrule
\end{tabular}
\sphinxtableafterendhook\par
\sphinxattableend\end{savenotes}


\begin{savenotes}\sphinxattablestart
\sphinxthistablewithglobalstyle
\centering
\begin{tabular}[t]{\X{33}{99}\X{33}{99}\X{33}{99}}
\sphinxtoprule
\sphinxtableatstartofbodyhook
\sphinxAtStartPar
位域 |
&
\sphinxAtStartPar
名称     | |
&
\sphinxAtStartPar
描述                                        | |
\\
\sphinxhline
\sphinxAtStartPar
31:8
&\begin{itemize}
\item {} 
\end{itemize}
&\begin{itemize}
\item {} 
\end{itemize}
\\
\sphinxhline
\sphinxAtStartPar
7:5
&
\sphinxAtStartPar
ID
&
\sphinxAtStartPar
标识符ID{[}20:18{]}                            |
\\
\sphinxhline
\sphinxAtStartPar
4:0
&\begin{itemize}
\item {} 
\end{itemize}
&\begin{itemize}
\item {} 
\end{itemize}
\\
\sphinxbottomrule
\end{tabular}
\sphinxtableafterendhook\par
\sphinxattableend\end{savenotes}


\subsubsection{\textless{}标准帧格式\textgreater{}数据寄存器2 DATA2}
\label{\detokenize{SWM241/_u529f_u80fd_u63cf_u8ff0/_u5c40_u57df_u7f51_u63a7_u5236_u5668:data2}}

\begin{savenotes}\sphinxattablestart
\sphinxthistablewithglobalstyle
\centering
\begin{tabular}[t]{\X{20}{100}\X{20}{100}\X{20}{100}\X{20}{100}\X{20}{100}}
\sphinxtoprule
\sphinxtableatstartofbodyhook
\sphinxAtStartPar
寄存器 |
&
\begin{DUlineblock}{0em}
\item[] 偏移 |
\end{DUlineblock}
&
\begin{DUlineblock}{0em}
\item[] 
\item[] {\color{red}\bfseries{}|}
\end{DUlineblock}
&
\sphinxAtStartPar
复位值 |    描 | |
&
\begin{DUlineblock}{0em}
\item[] |
  |
\end{DUlineblock}
\\
\sphinxhline
\sphinxAtStartPar
DATA2
&
\sphinxAtStartPar
0x4C
&&
\sphinxAtStartPar
0 000000
&
\sphinxAtStartPar
数据2寄存器                |
\\
\sphinxbottomrule
\end{tabular}
\sphinxtableafterendhook\par
\sphinxattableend\end{savenotes}


\begin{savenotes}\sphinxattablestart
\sphinxthistablewithglobalstyle
\centering
\begin{tabular}[t]{\X{12}{96}\X{12}{96}\X{12}{96}\X{12}{96}\X{12}{96}\X{12}{96}\X{12}{96}\X{12}{96}}
\sphinxtoprule
\sphinxtableatstartofbodyhook
\sphinxAtStartPar
31
&
\sphinxAtStartPar
30
&
\sphinxAtStartPar
29
&
\sphinxAtStartPar
28
&
\sphinxAtStartPar
27
&
\sphinxAtStartPar
26
&
\sphinxAtStartPar
25
&
\sphinxAtStartPar
24
\\
\sphinxhline\begin{itemize}
\item {} 
\end{itemize}
&&&&&&&\\
\sphinxhline
\sphinxAtStartPar
23
&
\sphinxAtStartPar
22
&
\sphinxAtStartPar
21
&
\sphinxAtStartPar
20
&
\sphinxAtStartPar
19
&
\sphinxAtStartPar
18
&
\sphinxAtStartPar
17
&
\sphinxAtStartPar
16
\\
\sphinxhline\begin{itemize}
\item {} 
\end{itemize}
&&&&&&&\\
\sphinxhline
\sphinxAtStartPar
15
&
\sphinxAtStartPar
14
&
\sphinxAtStartPar
13
&
\sphinxAtStartPar
12
&
\sphinxAtStartPar
11
&
\sphinxAtStartPar
10
&
\sphinxAtStartPar
9
&
\sphinxAtStartPar
8
\\
\sphinxhline\begin{itemize}
\item {} 
\end{itemize}
&&&&&&&\\
\sphinxhline
\sphinxAtStartPar
7
&
\sphinxAtStartPar
6
&
\sphinxAtStartPar
5
&
\sphinxAtStartPar
4
&
\sphinxAtStartPar
3
&
\sphinxAtStartPar
2
&
\sphinxAtStartPar
1
&
\sphinxAtStartPar
0
\\
\sphinxhline
\sphinxAtStartPar
DATA
&&&&&&&\\
\sphinxbottomrule
\end{tabular}
\sphinxtableafterendhook\par
\sphinxattableend\end{savenotes}


\begin{savenotes}\sphinxattablestart
\sphinxthistablewithglobalstyle
\centering
\begin{tabular}[t]{\X{33}{99}\X{33}{99}\X{33}{99}}
\sphinxtoprule
\sphinxtableatstartofbodyhook
\sphinxAtStartPar
位域 |
&
\sphinxAtStartPar
名称     | |
&
\sphinxAtStartPar
描述                                        | |
\\
\sphinxhline
\sphinxAtStartPar
31:8
&\begin{itemize}
\item {} 
\end{itemize}
&\begin{itemize}
\item {} 
\end{itemize}
\\
\sphinxhline
\sphinxAtStartPar
7:0
&
\sphinxAtStartPar
DATA
&
\sphinxAtStartPar
数据字节0                                   |
\\
\sphinxbottomrule
\end{tabular}
\sphinxtableafterendhook\par
\sphinxattableend\end{savenotes}


\subsubsection{\textless{}标准帧格式\textgreater{}数据寄存器3 DATA3}
\label{\detokenize{SWM241/_u529f_u80fd_u63cf_u8ff0/_u5c40_u57df_u7f51_u63a7_u5236_u5668:data3}}

\begin{savenotes}\sphinxattablestart
\sphinxthistablewithglobalstyle
\centering
\begin{tabular}[t]{\X{20}{100}\X{20}{100}\X{20}{100}\X{20}{100}\X{20}{100}}
\sphinxtoprule
\sphinxtableatstartofbodyhook
\sphinxAtStartPar
寄存器 |
&
\begin{DUlineblock}{0em}
\item[] 偏移 |
\end{DUlineblock}
&
\begin{DUlineblock}{0em}
\item[] 
\item[] {\color{red}\bfseries{}|}
\end{DUlineblock}
&
\sphinxAtStartPar
复位值 |    描 | |
&
\begin{DUlineblock}{0em}
\item[] |
  |
\end{DUlineblock}
\\
\sphinxhline
\sphinxAtStartPar
DATA3
&
\sphinxAtStartPar
0x50
&&
\sphinxAtStartPar
0 000000
&
\sphinxAtStartPar
数据3寄存器                |
\\
\sphinxbottomrule
\end{tabular}
\sphinxtableafterendhook\par
\sphinxattableend\end{savenotes}


\begin{savenotes}\sphinxattablestart
\sphinxthistablewithglobalstyle
\centering
\begin{tabular}[t]{\X{12}{96}\X{12}{96}\X{12}{96}\X{12}{96}\X{12}{96}\X{12}{96}\X{12}{96}\X{12}{96}}
\sphinxtoprule
\sphinxtableatstartofbodyhook
\sphinxAtStartPar
31
&
\sphinxAtStartPar
30
&
\sphinxAtStartPar
29
&
\sphinxAtStartPar
28
&
\sphinxAtStartPar
27
&
\sphinxAtStartPar
26
&
\sphinxAtStartPar
25
&
\sphinxAtStartPar
24
\\
\sphinxhline\begin{itemize}
\item {} 
\end{itemize}
&&&&&&&\\
\sphinxhline
\sphinxAtStartPar
23
&
\sphinxAtStartPar
22
&
\sphinxAtStartPar
21
&
\sphinxAtStartPar
20
&
\sphinxAtStartPar
19
&
\sphinxAtStartPar
18
&
\sphinxAtStartPar
17
&
\sphinxAtStartPar
16
\\
\sphinxhline\begin{itemize}
\item {} 
\end{itemize}
&&&&&&&\\
\sphinxhline
\sphinxAtStartPar
15
&
\sphinxAtStartPar
14
&
\sphinxAtStartPar
13
&
\sphinxAtStartPar
12
&
\sphinxAtStartPar
11
&
\sphinxAtStartPar
10
&
\sphinxAtStartPar
9
&
\sphinxAtStartPar
8
\\
\sphinxhline\begin{itemize}
\item {} 
\end{itemize}
&&&&&&&\\
\sphinxhline
\sphinxAtStartPar
7
&
\sphinxAtStartPar
6
&
\sphinxAtStartPar
5
&
\sphinxAtStartPar
4
&
\sphinxAtStartPar
3
&
\sphinxAtStartPar
2
&
\sphinxAtStartPar
1
&
\sphinxAtStartPar
0
\\
\sphinxhline
\sphinxAtStartPar
DATA
&&&&&&&\\
\sphinxbottomrule
\end{tabular}
\sphinxtableafterendhook\par
\sphinxattableend\end{savenotes}


\begin{savenotes}\sphinxattablestart
\sphinxthistablewithglobalstyle
\centering
\begin{tabular}[t]{\X{33}{99}\X{33}{99}\X{33}{99}}
\sphinxtoprule
\sphinxtableatstartofbodyhook
\sphinxAtStartPar
位域 |
&
\sphinxAtStartPar
名称     | |
&
\sphinxAtStartPar
描述                                        | |
\\
\sphinxhline
\sphinxAtStartPar
31:8
&\begin{itemize}
\item {} 
\end{itemize}
&\begin{itemize}
\item {} 
\end{itemize}
\\
\sphinxhline
\sphinxAtStartPar
7:0
&
\sphinxAtStartPar
DATA
&
\sphinxAtStartPar
数据字节1                                   |
\\
\sphinxbottomrule
\end{tabular}
\sphinxtableafterendhook\par
\sphinxattableend\end{savenotes}


\subsubsection{\textless{}标准帧格式\textgreater{}数据寄存器4 DATA4}
\label{\detokenize{SWM241/_u529f_u80fd_u63cf_u8ff0/_u5c40_u57df_u7f51_u63a7_u5236_u5668:data4}}

\begin{savenotes}\sphinxattablestart
\sphinxthistablewithglobalstyle
\centering
\begin{tabular}[t]{\X{20}{100}\X{20}{100}\X{20}{100}\X{20}{100}\X{20}{100}}
\sphinxtoprule
\sphinxtableatstartofbodyhook
\sphinxAtStartPar
寄存器 |
&
\begin{DUlineblock}{0em}
\item[] 偏移 |
\end{DUlineblock}
&
\begin{DUlineblock}{0em}
\item[] 
\item[] {\color{red}\bfseries{}|}
\end{DUlineblock}
&
\sphinxAtStartPar
复位值 |    描 | |
&
\begin{DUlineblock}{0em}
\item[] |
  |
\end{DUlineblock}
\\
\sphinxhline
\sphinxAtStartPar
DATA4
&
\sphinxAtStartPar
0x54
&&
\sphinxAtStartPar
0 000000
&
\sphinxAtStartPar
数据4寄存器                |
\\
\sphinxbottomrule
\end{tabular}
\sphinxtableafterendhook\par
\sphinxattableend\end{savenotes}


\begin{savenotes}\sphinxattablestart
\sphinxthistablewithglobalstyle
\centering
\begin{tabular}[t]{\X{12}{96}\X{12}{96}\X{12}{96}\X{12}{96}\X{12}{96}\X{12}{96}\X{12}{96}\X{12}{96}}
\sphinxtoprule
\sphinxtableatstartofbodyhook
\sphinxAtStartPar
31
&
\sphinxAtStartPar
30
&
\sphinxAtStartPar
29
&
\sphinxAtStartPar
28
&
\sphinxAtStartPar
27
&
\sphinxAtStartPar
26
&
\sphinxAtStartPar
25
&
\sphinxAtStartPar
24
\\
\sphinxhline\begin{itemize}
\item {} 
\end{itemize}
&&&&&&&\\
\sphinxhline
\sphinxAtStartPar
23
&
\sphinxAtStartPar
22
&
\sphinxAtStartPar
21
&
\sphinxAtStartPar
20
&
\sphinxAtStartPar
19
&
\sphinxAtStartPar
18
&
\sphinxAtStartPar
17
&
\sphinxAtStartPar
16
\\
\sphinxhline\begin{itemize}
\item {} 
\end{itemize}
&&&&&&&\\
\sphinxhline
\sphinxAtStartPar
15
&
\sphinxAtStartPar
14
&
\sphinxAtStartPar
13
&
\sphinxAtStartPar
12
&
\sphinxAtStartPar
11
&
\sphinxAtStartPar
10
&
\sphinxAtStartPar
9
&
\sphinxAtStartPar
8
\\
\sphinxhline\begin{itemize}
\item {} 
\end{itemize}
&&&&&&&\\
\sphinxhline
\sphinxAtStartPar
7
&
\sphinxAtStartPar
6
&
\sphinxAtStartPar
5
&
\sphinxAtStartPar
4
&
\sphinxAtStartPar
3
&
\sphinxAtStartPar
2
&
\sphinxAtStartPar
1
&
\sphinxAtStartPar
0
\\
\sphinxhline
\sphinxAtStartPar
DATA
&&&&&&&\\
\sphinxbottomrule
\end{tabular}
\sphinxtableafterendhook\par
\sphinxattableend\end{savenotes}


\begin{savenotes}\sphinxattablestart
\sphinxthistablewithglobalstyle
\centering
\begin{tabular}[t]{\X{33}{99}\X{33}{99}\X{33}{99}}
\sphinxtoprule
\sphinxtableatstartofbodyhook
\sphinxAtStartPar
位域 |
&
\sphinxAtStartPar
名称     | |
&
\sphinxAtStartPar
描述                                        | |
\\
\sphinxhline
\sphinxAtStartPar
31:8
&\begin{itemize}
\item {} 
\end{itemize}
&\begin{itemize}
\item {} 
\end{itemize}
\\
\sphinxhline
\sphinxAtStartPar
7:0
&
\sphinxAtStartPar
DATA
&
\sphinxAtStartPar
数据字节2                                   |
\\
\sphinxbottomrule
\end{tabular}
\sphinxtableafterendhook\par
\sphinxattableend\end{savenotes}


\subsubsection{\textless{}标准帧格式\textgreater{}数据寄存器5 DATA5}
\label{\detokenize{SWM241/_u529f_u80fd_u63cf_u8ff0/_u5c40_u57df_u7f51_u63a7_u5236_u5668:data5}}

\begin{savenotes}\sphinxattablestart
\sphinxthistablewithglobalstyle
\centering
\begin{tabular}[t]{\X{20}{100}\X{20}{100}\X{20}{100}\X{20}{100}\X{20}{100}}
\sphinxtoprule
\sphinxtableatstartofbodyhook
\sphinxAtStartPar
寄存器 |
&
\begin{DUlineblock}{0em}
\item[] 偏移 |
\end{DUlineblock}
&
\begin{DUlineblock}{0em}
\item[] 
\item[] {\color{red}\bfseries{}|}
\end{DUlineblock}
&
\sphinxAtStartPar
复位值 |    描 | |
&
\begin{DUlineblock}{0em}
\item[] |
  |
\end{DUlineblock}
\\
\sphinxhline
\sphinxAtStartPar
DATA5
&
\sphinxAtStartPar
0x58
&&
\sphinxAtStartPar
0 000000
&
\sphinxAtStartPar
数据5寄存器                |
\\
\sphinxbottomrule
\end{tabular}
\sphinxtableafterendhook\par
\sphinxattableend\end{savenotes}


\begin{savenotes}\sphinxattablestart
\sphinxthistablewithglobalstyle
\centering
\begin{tabular}[t]{\X{12}{96}\X{12}{96}\X{12}{96}\X{12}{96}\X{12}{96}\X{12}{96}\X{12}{96}\X{12}{96}}
\sphinxtoprule
\sphinxtableatstartofbodyhook
\sphinxAtStartPar
31
&
\sphinxAtStartPar
30
&
\sphinxAtStartPar
29
&
\sphinxAtStartPar
28
&
\sphinxAtStartPar
27
&
\sphinxAtStartPar
26
&
\sphinxAtStartPar
25
&
\sphinxAtStartPar
24
\\
\sphinxhline\begin{itemize}
\item {} 
\end{itemize}
&&&&&&&\\
\sphinxhline
\sphinxAtStartPar
23
&
\sphinxAtStartPar
22
&
\sphinxAtStartPar
21
&
\sphinxAtStartPar
20
&
\sphinxAtStartPar
19
&
\sphinxAtStartPar
18
&
\sphinxAtStartPar
17
&
\sphinxAtStartPar
16
\\
\sphinxhline\begin{itemize}
\item {} 
\end{itemize}
&&&&&&&\\
\sphinxhline
\sphinxAtStartPar
15
&
\sphinxAtStartPar
14
&
\sphinxAtStartPar
13
&
\sphinxAtStartPar
12
&
\sphinxAtStartPar
11
&
\sphinxAtStartPar
10
&
\sphinxAtStartPar
9
&
\sphinxAtStartPar
8
\\
\sphinxhline\begin{itemize}
\item {} 
\end{itemize}
&&&&&&&\\
\sphinxhline
\sphinxAtStartPar
7
&
\sphinxAtStartPar
6
&
\sphinxAtStartPar
5
&
\sphinxAtStartPar
4
&
\sphinxAtStartPar
3
&
\sphinxAtStartPar
2
&
\sphinxAtStartPar
1
&
\sphinxAtStartPar
0
\\
\sphinxhline
\sphinxAtStartPar
DATA
&&&&&&&\\
\sphinxbottomrule
\end{tabular}
\sphinxtableafterendhook\par
\sphinxattableend\end{savenotes}


\begin{savenotes}\sphinxattablestart
\sphinxthistablewithglobalstyle
\centering
\begin{tabular}[t]{\X{33}{99}\X{33}{99}\X{33}{99}}
\sphinxtoprule
\sphinxtableatstartofbodyhook
\sphinxAtStartPar
位域 |
&
\sphinxAtStartPar
名称     | |
&
\sphinxAtStartPar
描述                                        | |
\\
\sphinxhline
\sphinxAtStartPar
31:8
&\begin{itemize}
\item {} 
\end{itemize}
&\begin{itemize}
\item {} 
\end{itemize}
\\
\sphinxhline
\sphinxAtStartPar
7:0
&
\sphinxAtStartPar
DATA
&
\sphinxAtStartPar
数据字节3                                   |
\\
\sphinxbottomrule
\end{tabular}
\sphinxtableafterendhook\par
\sphinxattableend\end{savenotes}


\subsubsection{\textless{}标准帧格式\textgreater{}数据寄存器6 DATA6}
\label{\detokenize{SWM241/_u529f_u80fd_u63cf_u8ff0/_u5c40_u57df_u7f51_u63a7_u5236_u5668:data6}}

\begin{savenotes}\sphinxattablestart
\sphinxthistablewithglobalstyle
\centering
\begin{tabular}[t]{\X{20}{100}\X{20}{100}\X{20}{100}\X{20}{100}\X{20}{100}}
\sphinxtoprule
\sphinxtableatstartofbodyhook
\sphinxAtStartPar
寄存器 |
&
\begin{DUlineblock}{0em}
\item[] 偏移 |
\end{DUlineblock}
&
\begin{DUlineblock}{0em}
\item[] 
\item[] {\color{red}\bfseries{}|}
\end{DUlineblock}
&
\sphinxAtStartPar
复位值 |    描 | |
&
\begin{DUlineblock}{0em}
\item[] |
  |
\end{DUlineblock}
\\
\sphinxhline
\sphinxAtStartPar
DATA6
&
\sphinxAtStartPar
0x5C
&&
\sphinxAtStartPar
0 000000
&
\sphinxAtStartPar
数据6寄存器                |
\\
\sphinxbottomrule
\end{tabular}
\sphinxtableafterendhook\par
\sphinxattableend\end{savenotes}


\begin{savenotes}\sphinxattablestart
\sphinxthistablewithglobalstyle
\centering
\begin{tabular}[t]{\X{12}{96}\X{12}{96}\X{12}{96}\X{12}{96}\X{12}{96}\X{12}{96}\X{12}{96}\X{12}{96}}
\sphinxtoprule
\sphinxtableatstartofbodyhook
\sphinxAtStartPar
31
&
\sphinxAtStartPar
30
&
\sphinxAtStartPar
29
&
\sphinxAtStartPar
28
&
\sphinxAtStartPar
27
&
\sphinxAtStartPar
26
&
\sphinxAtStartPar
25
&
\sphinxAtStartPar
24
\\
\sphinxhline\begin{itemize}
\item {} 
\end{itemize}
&&&&&&&\\
\sphinxhline
\sphinxAtStartPar
23
&
\sphinxAtStartPar
22
&
\sphinxAtStartPar
21
&
\sphinxAtStartPar
20
&
\sphinxAtStartPar
19
&
\sphinxAtStartPar
18
&
\sphinxAtStartPar
17
&
\sphinxAtStartPar
16
\\
\sphinxhline\begin{itemize}
\item {} 
\end{itemize}
&&&&&&&\\
\sphinxhline
\sphinxAtStartPar
15
&
\sphinxAtStartPar
14
&
\sphinxAtStartPar
13
&
\sphinxAtStartPar
12
&
\sphinxAtStartPar
11
&
\sphinxAtStartPar
10
&
\sphinxAtStartPar
9
&
\sphinxAtStartPar
8
\\
\sphinxhline\begin{itemize}
\item {} 
\end{itemize}
&&&&&&&\\
\sphinxhline
\sphinxAtStartPar
7
&
\sphinxAtStartPar
6
&
\sphinxAtStartPar
5
&
\sphinxAtStartPar
4
&
\sphinxAtStartPar
3
&
\sphinxAtStartPar
2
&
\sphinxAtStartPar
1
&
\sphinxAtStartPar
0
\\
\sphinxhline
\sphinxAtStartPar
DATA
&&&&&&&\\
\sphinxbottomrule
\end{tabular}
\sphinxtableafterendhook\par
\sphinxattableend\end{savenotes}


\begin{savenotes}\sphinxattablestart
\sphinxthistablewithglobalstyle
\centering
\begin{tabular}[t]{\X{33}{99}\X{33}{99}\X{33}{99}}
\sphinxtoprule
\sphinxtableatstartofbodyhook
\sphinxAtStartPar
位域 |
&
\sphinxAtStartPar
名称     | |
&
\sphinxAtStartPar
描述                                        | |
\\
\sphinxhline
\sphinxAtStartPar
31:8
&\begin{itemize}
\item {} 
\end{itemize}
&\begin{itemize}
\item {} 
\end{itemize}
\\
\sphinxhline
\sphinxAtStartPar
7:0
&
\sphinxAtStartPar
DATA
&
\sphinxAtStartPar
数据字节4                                   |
\\
\sphinxbottomrule
\end{tabular}
\sphinxtableafterendhook\par
\sphinxattableend\end{savenotes}


\subsubsection{\textless{}标准帧格式\textgreater{}数据寄存器7 DATA7}
\label{\detokenize{SWM241/_u529f_u80fd_u63cf_u8ff0/_u5c40_u57df_u7f51_u63a7_u5236_u5668:data7}}

\begin{savenotes}\sphinxattablestart
\sphinxthistablewithglobalstyle
\centering
\begin{tabular}[t]{\X{20}{100}\X{20}{100}\X{20}{100}\X{20}{100}\X{20}{100}}
\sphinxtoprule
\sphinxtableatstartofbodyhook
\sphinxAtStartPar
寄存器 |
&
\begin{DUlineblock}{0em}
\item[] 偏移 |
\end{DUlineblock}
&
\begin{DUlineblock}{0em}
\item[] 
\item[] {\color{red}\bfseries{}|}
\end{DUlineblock}
&
\sphinxAtStartPar
复位值 |    描 | |
&
\begin{DUlineblock}{0em}
\item[] |
  |
\end{DUlineblock}
\\
\sphinxhline
\sphinxAtStartPar
DATA7
&
\sphinxAtStartPar
0x60
&&
\sphinxAtStartPar
0 000000
&
\sphinxAtStartPar
数据7寄存器                |
\\
\sphinxbottomrule
\end{tabular}
\sphinxtableafterendhook\par
\sphinxattableend\end{savenotes}


\begin{savenotes}\sphinxattablestart
\sphinxthistablewithglobalstyle
\centering
\begin{tabular}[t]{\X{12}{96}\X{12}{96}\X{12}{96}\X{12}{96}\X{12}{96}\X{12}{96}\X{12}{96}\X{12}{96}}
\sphinxtoprule
\sphinxtableatstartofbodyhook
\sphinxAtStartPar
31
&
\sphinxAtStartPar
30
&
\sphinxAtStartPar
29
&
\sphinxAtStartPar
28
&
\sphinxAtStartPar
27
&
\sphinxAtStartPar
26
&
\sphinxAtStartPar
25
&
\sphinxAtStartPar
24
\\
\sphinxhline\begin{itemize}
\item {} 
\end{itemize}
&&&&&&&\\
\sphinxhline
\sphinxAtStartPar
23
&
\sphinxAtStartPar
22
&
\sphinxAtStartPar
21
&
\sphinxAtStartPar
20
&
\sphinxAtStartPar
19
&
\sphinxAtStartPar
18
&
\sphinxAtStartPar
17
&
\sphinxAtStartPar
16
\\
\sphinxhline\begin{itemize}
\item {} 
\end{itemize}
&&&&&&&\\
\sphinxhline
\sphinxAtStartPar
15
&
\sphinxAtStartPar
14
&
\sphinxAtStartPar
13
&
\sphinxAtStartPar
12
&
\sphinxAtStartPar
11
&
\sphinxAtStartPar
10
&
\sphinxAtStartPar
9
&
\sphinxAtStartPar
8
\\
\sphinxhline\begin{itemize}
\item {} 
\end{itemize}
&&&&&&&\\
\sphinxhline
\sphinxAtStartPar
7
&
\sphinxAtStartPar
6
&
\sphinxAtStartPar
5
&
\sphinxAtStartPar
4
&
\sphinxAtStartPar
3
&
\sphinxAtStartPar
2
&
\sphinxAtStartPar
1
&
\sphinxAtStartPar
0
\\
\sphinxhline
\sphinxAtStartPar
DATA
&&&&&&&\\
\sphinxbottomrule
\end{tabular}
\sphinxtableafterendhook\par
\sphinxattableend\end{savenotes}


\begin{savenotes}\sphinxattablestart
\sphinxthistablewithglobalstyle
\centering
\begin{tabular}[t]{\X{33}{99}\X{33}{99}\X{33}{99}}
\sphinxtoprule
\sphinxtableatstartofbodyhook
\sphinxAtStartPar
位域 |
&
\sphinxAtStartPar
名称     | |
&
\sphinxAtStartPar
描述                                        | |
\\
\sphinxhline
\sphinxAtStartPar
31:8
&\begin{itemize}
\item {} 
\end{itemize}
&\begin{itemize}
\item {} 
\end{itemize}
\\
\sphinxhline
\sphinxAtStartPar
7:0
&
\sphinxAtStartPar
DATA
&
\sphinxAtStartPar
数据字节5                                   |
\\
\sphinxbottomrule
\end{tabular}
\sphinxtableafterendhook\par
\sphinxattableend\end{savenotes}


\subsubsection{\textless{}标准帧格式\textgreater{}数据寄存器8 DATA8}
\label{\detokenize{SWM241/_u529f_u80fd_u63cf_u8ff0/_u5c40_u57df_u7f51_u63a7_u5236_u5668:data8}}

\begin{savenotes}\sphinxattablestart
\sphinxthistablewithglobalstyle
\centering
\begin{tabular}[t]{\X{20}{100}\X{20}{100}\X{20}{100}\X{20}{100}\X{20}{100}}
\sphinxtoprule
\sphinxtableatstartofbodyhook
\sphinxAtStartPar
寄存器 |
&
\begin{DUlineblock}{0em}
\item[] 偏移 |
\end{DUlineblock}
&
\begin{DUlineblock}{0em}
\item[] 
\item[] {\color{red}\bfseries{}|}
\end{DUlineblock}
&
\sphinxAtStartPar
复位值 |    描 | |
&
\begin{DUlineblock}{0em}
\item[] |
  |
\end{DUlineblock}
\\
\sphinxhline
\sphinxAtStartPar
DATA8
&
\sphinxAtStartPar
0x64
&&
\sphinxAtStartPar
0 000000
&
\sphinxAtStartPar
数据8寄存器                |
\\
\sphinxbottomrule
\end{tabular}
\sphinxtableafterendhook\par
\sphinxattableend\end{savenotes}


\begin{savenotes}\sphinxattablestart
\sphinxthistablewithglobalstyle
\centering
\begin{tabular}[t]{\X{12}{96}\X{12}{96}\X{12}{96}\X{12}{96}\X{12}{96}\X{12}{96}\X{12}{96}\X{12}{96}}
\sphinxtoprule
\sphinxtableatstartofbodyhook
\sphinxAtStartPar
31
&
\sphinxAtStartPar
30
&
\sphinxAtStartPar
29
&
\sphinxAtStartPar
28
&
\sphinxAtStartPar
27
&
\sphinxAtStartPar
26
&
\sphinxAtStartPar
25
&
\sphinxAtStartPar
24
\\
\sphinxhline\begin{itemize}
\item {} 
\end{itemize}
&&&&&&&\\
\sphinxhline
\sphinxAtStartPar
23
&
\sphinxAtStartPar
22
&
\sphinxAtStartPar
21
&
\sphinxAtStartPar
20
&
\sphinxAtStartPar
19
&
\sphinxAtStartPar
18
&
\sphinxAtStartPar
17
&
\sphinxAtStartPar
16
\\
\sphinxhline\begin{itemize}
\item {} 
\end{itemize}
&&&&&&&\\
\sphinxhline
\sphinxAtStartPar
15
&
\sphinxAtStartPar
14
&
\sphinxAtStartPar
13
&
\sphinxAtStartPar
12
&
\sphinxAtStartPar
11
&
\sphinxAtStartPar
10
&
\sphinxAtStartPar
9
&
\sphinxAtStartPar
8
\\
\sphinxhline\begin{itemize}
\item {} 
\end{itemize}
&&&&&&&\\
\sphinxhline
\sphinxAtStartPar
7
&
\sphinxAtStartPar
6
&
\sphinxAtStartPar
5
&
\sphinxAtStartPar
4
&
\sphinxAtStartPar
3
&
\sphinxAtStartPar
2
&
\sphinxAtStartPar
1
&
\sphinxAtStartPar
0
\\
\sphinxhline
\sphinxAtStartPar
DATA
&&&&&&&\\
\sphinxbottomrule
\end{tabular}
\sphinxtableafterendhook\par
\sphinxattableend\end{savenotes}


\begin{savenotes}\sphinxattablestart
\sphinxthistablewithglobalstyle
\centering
\begin{tabular}[t]{\X{33}{99}\X{33}{99}\X{33}{99}}
\sphinxtoprule
\sphinxtableatstartofbodyhook
\sphinxAtStartPar
位域 |
&
\sphinxAtStartPar
名称     | |
&
\sphinxAtStartPar
描述                                        | |
\\
\sphinxhline
\sphinxAtStartPar
31:8
&\begin{itemize}
\item {} 
\end{itemize}
&\begin{itemize}
\item {} 
\end{itemize}
\\
\sphinxhline
\sphinxAtStartPar
7:0
&
\sphinxAtStartPar
DATA
&
\sphinxAtStartPar
数据字节6                                   |
\\
\sphinxbottomrule
\end{tabular}
\sphinxtableafterendhook\par
\sphinxattableend\end{savenotes}


\subsubsection{\textless{}标准帧格式\textgreater{}数据寄存器9 DATA9}
\label{\detokenize{SWM241/_u529f_u80fd_u63cf_u8ff0/_u5c40_u57df_u7f51_u63a7_u5236_u5668:data9}}

\begin{savenotes}\sphinxattablestart
\sphinxthistablewithglobalstyle
\centering
\begin{tabular}[t]{\X{20}{100}\X{20}{100}\X{20}{100}\X{20}{100}\X{20}{100}}
\sphinxtoprule
\sphinxtableatstartofbodyhook
\sphinxAtStartPar
寄存器 |
&
\begin{DUlineblock}{0em}
\item[] 偏移 |
\end{DUlineblock}
&
\begin{DUlineblock}{0em}
\item[] 
\item[] {\color{red}\bfseries{}|}
\end{DUlineblock}
&
\sphinxAtStartPar
复位值 |    描 | |
&
\begin{DUlineblock}{0em}
\item[] |
  |
\end{DUlineblock}
\\
\sphinxhline
\sphinxAtStartPar
DATA9
&
\sphinxAtStartPar
0x68
&&
\sphinxAtStartPar
0 000000
&
\sphinxAtStartPar
数据9寄存器                |
\\
\sphinxbottomrule
\end{tabular}
\sphinxtableafterendhook\par
\sphinxattableend\end{savenotes}


\begin{savenotes}\sphinxattablestart
\sphinxthistablewithglobalstyle
\centering
\begin{tabular}[t]{\X{12}{96}\X{12}{96}\X{12}{96}\X{12}{96}\X{12}{96}\X{12}{96}\X{12}{96}\X{12}{96}}
\sphinxtoprule
\sphinxtableatstartofbodyhook
\sphinxAtStartPar
31
&
\sphinxAtStartPar
30
&
\sphinxAtStartPar
29
&
\sphinxAtStartPar
28
&
\sphinxAtStartPar
27
&
\sphinxAtStartPar
26
&
\sphinxAtStartPar
25
&
\sphinxAtStartPar
24
\\
\sphinxhline\begin{itemize}
\item {} 
\end{itemize}
&&&&&&&\\
\sphinxhline
\sphinxAtStartPar
23
&
\sphinxAtStartPar
22
&
\sphinxAtStartPar
21
&
\sphinxAtStartPar
20
&
\sphinxAtStartPar
19
&
\sphinxAtStartPar
18
&
\sphinxAtStartPar
17
&
\sphinxAtStartPar
16
\\
\sphinxhline\begin{itemize}
\item {} 
\end{itemize}
&&&&&&&\\
\sphinxhline
\sphinxAtStartPar
15
&
\sphinxAtStartPar
14
&
\sphinxAtStartPar
13
&
\sphinxAtStartPar
12
&
\sphinxAtStartPar
11
&
\sphinxAtStartPar
10
&
\sphinxAtStartPar
9
&
\sphinxAtStartPar
8
\\
\sphinxhline\begin{itemize}
\item {} 
\end{itemize}
&&&&&&&\\
\sphinxhline
\sphinxAtStartPar
7
&
\sphinxAtStartPar
6
&
\sphinxAtStartPar
5
&
\sphinxAtStartPar
4
&
\sphinxAtStartPar
3
&
\sphinxAtStartPar
2
&
\sphinxAtStartPar
1
&
\sphinxAtStartPar
0
\\
\sphinxhline
\sphinxAtStartPar
DATA
&&&&&&&\\
\sphinxbottomrule
\end{tabular}
\sphinxtableafterendhook\par
\sphinxattableend\end{savenotes}


\begin{savenotes}\sphinxattablestart
\sphinxthistablewithglobalstyle
\centering
\begin{tabular}[t]{\X{33}{99}\X{33}{99}\X{33}{99}}
\sphinxtoprule
\sphinxtableatstartofbodyhook
\sphinxAtStartPar
位域 |
&
\sphinxAtStartPar
名称     | |
&
\sphinxAtStartPar
描述                                        | |
\\
\sphinxhline
\sphinxAtStartPar
31:8
&\begin{itemize}
\item {} 
\end{itemize}
&\begin{itemize}
\item {} 
\end{itemize}
\\
\sphinxhline
\sphinxAtStartPar
7:0
&
\sphinxAtStartPar
DATA
&
\sphinxAtStartPar
数据字节7                                   |
\\
\sphinxbottomrule
\end{tabular}
\sphinxtableafterendhook\par
\sphinxattableend\end{savenotes}


\subsubsection{\textless{}扩展帧格式\textgreater{}数据寄存器0 DATA0}
\label{\detokenize{SWM241/_u529f_u80fd_u63cf_u8ff0/_u5c40_u57df_u7f51_u63a7_u5236_u5668:id104}}

\begin{savenotes}\sphinxattablestart
\sphinxthistablewithglobalstyle
\centering
\begin{tabular}[t]{\X{20}{100}\X{20}{100}\X{20}{100}\X{20}{100}\X{20}{100}}
\sphinxtoprule
\sphinxtableatstartofbodyhook
\sphinxAtStartPar
寄存器 |
&
\begin{DUlineblock}{0em}
\item[] 偏移 |
\end{DUlineblock}
&
\begin{DUlineblock}{0em}
\item[] 
\item[] {\color{red}\bfseries{}|}
\end{DUlineblock}
&
\sphinxAtStartPar
复位值 |    描 | |
&
\begin{DUlineblock}{0em}
\item[] |
  |
\end{DUlineblock}
\\
\sphinxhline
\sphinxAtStartPar
DATA0
&
\sphinxAtStartPar
0x44
&&
\sphinxAtStartPar
0 000000
&
\sphinxAtStartPar
数据0寄存器                |
\\
\sphinxbottomrule
\end{tabular}
\sphinxtableafterendhook\par
\sphinxattableend\end{savenotes}


\begin{savenotes}\sphinxattablestart
\sphinxthistablewithglobalstyle
\centering
\begin{tabular}[t]{\X{12}{96}\X{12}{96}\X{12}{96}\X{12}{96}\X{12}{96}\X{12}{96}\X{12}{96}\X{12}{96}}
\sphinxtoprule
\sphinxtableatstartofbodyhook
\sphinxAtStartPar
31
&
\sphinxAtStartPar
30
&
\sphinxAtStartPar
29
&
\sphinxAtStartPar
28
&
\sphinxAtStartPar
27
&
\sphinxAtStartPar
26
&
\sphinxAtStartPar
25
&
\sphinxAtStartPar
24
\\
\sphinxhline\begin{itemize}
\item {} 
\end{itemize}
&&&&&&&\\
\sphinxhline
\sphinxAtStartPar
23
&
\sphinxAtStartPar
22
&
\sphinxAtStartPar
21
&
\sphinxAtStartPar
20
&
\sphinxAtStartPar
19
&
\sphinxAtStartPar
18
&
\sphinxAtStartPar
17
&
\sphinxAtStartPar
16
\\
\sphinxhline\begin{itemize}
\item {} 
\end{itemize}
&&&&&&&\\
\sphinxhline
\sphinxAtStartPar
15
&
\sphinxAtStartPar
14
&
\sphinxAtStartPar
13
&
\sphinxAtStartPar
12
&
\sphinxAtStartPar
11
&
\sphinxAtStartPar
10
&
\sphinxAtStartPar
9
&
\sphinxAtStartPar
8
\\
\sphinxhline\begin{itemize}
\item {} 
\end{itemize}
&&&&&&&\\
\sphinxhline
\sphinxAtStartPar
7
&
\sphinxAtStartPar
6
&
\sphinxAtStartPar
5
&
\sphinxAtStartPar
4
&
\sphinxAtStartPar
3
&
\sphinxAtStartPar
2
&
\sphinxAtStartPar
1
&
\sphinxAtStartPar
0
\\
\sphinxhline
\sphinxAtStartPar
ID
&&&&&&&\\
\sphinxbottomrule
\end{tabular}
\sphinxtableafterendhook\par
\sphinxattableend\end{savenotes}


\begin{savenotes}\sphinxattablestart
\sphinxthistablewithglobalstyle
\centering
\begin{tabular}[t]{\X{33}{99}\X{33}{99}\X{33}{99}}
\sphinxtoprule
\sphinxtableatstartofbodyhook
\sphinxAtStartPar
位域 |
&
\sphinxAtStartPar
名称     | |
&
\sphinxAtStartPar
描述                                        | |
\\
\sphinxhline
\sphinxAtStartPar
31:8
&\begin{itemize}
\item {} 
\end{itemize}
&\begin{itemize}
\item {} 
\end{itemize}
\\
\sphinxhline
\sphinxAtStartPar
7:0
&
\sphinxAtStartPar
ID
&
\sphinxAtStartPar
标识符ID{[}28:21{]}                            |
\\
\sphinxbottomrule
\end{tabular}
\sphinxtableafterendhook\par
\sphinxattableend\end{savenotes}


\subsubsection{\textless{}扩展帧格式\textgreater{}数据寄存器1 DATA1}
\label{\detokenize{SWM241/_u529f_u80fd_u63cf_u8ff0/_u5c40_u57df_u7f51_u63a7_u5236_u5668:id107}}

\begin{savenotes}\sphinxattablestart
\sphinxthistablewithglobalstyle
\centering
\begin{tabular}[t]{\X{20}{100}\X{20}{100}\X{20}{100}\X{20}{100}\X{20}{100}}
\sphinxtoprule
\sphinxtableatstartofbodyhook
\sphinxAtStartPar
寄存器 |
&
\begin{DUlineblock}{0em}
\item[] 偏移 |
\end{DUlineblock}
&
\begin{DUlineblock}{0em}
\item[] 
\item[] {\color{red}\bfseries{}|}
\end{DUlineblock}
&
\sphinxAtStartPar
复位值 |    描 | |
&
\begin{DUlineblock}{0em}
\item[] |
  |
\end{DUlineblock}
\\
\sphinxhline
\sphinxAtStartPar
DATA1
&
\sphinxAtStartPar
0x48
&&
\sphinxAtStartPar
0 000000
&
\sphinxAtStartPar
数据1寄存器                |
\\
\sphinxbottomrule
\end{tabular}
\sphinxtableafterendhook\par
\sphinxattableend\end{savenotes}


\begin{savenotes}\sphinxattablestart
\sphinxthistablewithglobalstyle
\centering
\begin{tabular}[t]{\X{12}{96}\X{12}{96}\X{12}{96}\X{12}{96}\X{12}{96}\X{12}{96}\X{12}{96}\X{12}{96}}
\sphinxtoprule
\sphinxtableatstartofbodyhook
\sphinxAtStartPar
31
&
\sphinxAtStartPar
30
&
\sphinxAtStartPar
29
&
\sphinxAtStartPar
28
&
\sphinxAtStartPar
27
&
\sphinxAtStartPar
26
&
\sphinxAtStartPar
25
&
\sphinxAtStartPar
24
\\
\sphinxhline\begin{itemize}
\item {} 
\end{itemize}
&&&&&&&\\
\sphinxhline
\sphinxAtStartPar
23
&
\sphinxAtStartPar
22
&
\sphinxAtStartPar
21
&
\sphinxAtStartPar
20
&
\sphinxAtStartPar
19
&
\sphinxAtStartPar
18
&
\sphinxAtStartPar
17
&
\sphinxAtStartPar
16
\\
\sphinxhline\begin{itemize}
\item {} 
\end{itemize}
&&&&&&&\\
\sphinxhline
\sphinxAtStartPar
15
&
\sphinxAtStartPar
14
&
\sphinxAtStartPar
13
&
\sphinxAtStartPar
12
&
\sphinxAtStartPar
11
&
\sphinxAtStartPar
10
&
\sphinxAtStartPar
9
&
\sphinxAtStartPar
8
\\
\sphinxhline\begin{itemize}
\item {} 
\end{itemize}
&&&&&&&\\
\sphinxhline
\sphinxAtStartPar
7
&
\sphinxAtStartPar
6
&
\sphinxAtStartPar
5
&
\sphinxAtStartPar
4
&
\sphinxAtStartPar
3
&
\sphinxAtStartPar
2
&
\sphinxAtStartPar
1
&
\sphinxAtStartPar
0
\\
\sphinxhline
\sphinxAtStartPar
ID
&&&&&&&\\
\sphinxbottomrule
\end{tabular}
\sphinxtableafterendhook\par
\sphinxattableend\end{savenotes}


\begin{savenotes}\sphinxattablestart
\sphinxthistablewithglobalstyle
\centering
\begin{tabular}[t]{\X{33}{99}\X{33}{99}\X{33}{99}}
\sphinxtoprule
\sphinxtableatstartofbodyhook
\sphinxAtStartPar
位域 |
&
\sphinxAtStartPar
名称     | |
&
\sphinxAtStartPar
描述                                        | |
\\
\sphinxhline
\sphinxAtStartPar
31:8
&\begin{itemize}
\item {} 
\end{itemize}
&\begin{itemize}
\item {} 
\end{itemize}
\\
\sphinxhline
\sphinxAtStartPar
7:0
&
\sphinxAtStartPar
ID
&
\sphinxAtStartPar
标识符ID{[}20:13{]}                            |
\\
\sphinxbottomrule
\end{tabular}
\sphinxtableafterendhook\par
\sphinxattableend\end{savenotes}


\subsubsection{\textless{}扩展帧格式\textgreater{}数据寄存器2 DATA2}
\label{\detokenize{SWM241/_u529f_u80fd_u63cf_u8ff0/_u5c40_u57df_u7f51_u63a7_u5236_u5668:id110}}

\begin{savenotes}\sphinxattablestart
\sphinxthistablewithglobalstyle
\centering
\begin{tabular}[t]{\X{20}{100}\X{20}{100}\X{20}{100}\X{20}{100}\X{20}{100}}
\sphinxtoprule
\sphinxtableatstartofbodyhook
\sphinxAtStartPar
寄存器 |
&
\begin{DUlineblock}{0em}
\item[] 偏移 |
\end{DUlineblock}
&
\begin{DUlineblock}{0em}
\item[] 
\item[] {\color{red}\bfseries{}|}
\end{DUlineblock}
&
\sphinxAtStartPar
复位值 |    描 | |
&
\begin{DUlineblock}{0em}
\item[] |
  |
\end{DUlineblock}
\\
\sphinxhline
\sphinxAtStartPar
DATA2
&
\sphinxAtStartPar
0x4C
&&
\sphinxAtStartPar
0 000000
&
\sphinxAtStartPar
数据2寄存器                |
\\
\sphinxbottomrule
\end{tabular}
\sphinxtableafterendhook\par
\sphinxattableend\end{savenotes}


\begin{savenotes}\sphinxattablestart
\sphinxthistablewithglobalstyle
\centering
\begin{tabular}[t]{\X{12}{96}\X{12}{96}\X{12}{96}\X{12}{96}\X{12}{96}\X{12}{96}\X{12}{96}\X{12}{96}}
\sphinxtoprule
\sphinxtableatstartofbodyhook
\sphinxAtStartPar
31
&
\sphinxAtStartPar
30
&
\sphinxAtStartPar
29
&
\sphinxAtStartPar
28
&
\sphinxAtStartPar
27
&
\sphinxAtStartPar
26
&
\sphinxAtStartPar
25
&
\sphinxAtStartPar
24
\\
\sphinxhline\begin{itemize}
\item {} 
\end{itemize}
&&&&&&&\\
\sphinxhline
\sphinxAtStartPar
23
&
\sphinxAtStartPar
22
&
\sphinxAtStartPar
21
&
\sphinxAtStartPar
20
&
\sphinxAtStartPar
19
&
\sphinxAtStartPar
18
&
\sphinxAtStartPar
17
&
\sphinxAtStartPar
16
\\
\sphinxhline\begin{itemize}
\item {} 
\end{itemize}
&&&&&&&\\
\sphinxhline
\sphinxAtStartPar
15
&
\sphinxAtStartPar
14
&
\sphinxAtStartPar
13
&
\sphinxAtStartPar
12
&
\sphinxAtStartPar
11
&
\sphinxAtStartPar
10
&
\sphinxAtStartPar
9
&
\sphinxAtStartPar
8
\\
\sphinxhline\begin{itemize}
\item {} 
\end{itemize}
&&&&&&&\\
\sphinxhline
\sphinxAtStartPar
7
&
\sphinxAtStartPar
6
&
\sphinxAtStartPar
5
&
\sphinxAtStartPar
4
&
\sphinxAtStartPar
3
&
\sphinxAtStartPar
2
&
\sphinxAtStartPar
1
&
\sphinxAtStartPar
0
\\
\sphinxhline
\sphinxAtStartPar
ID
&&&&&&&\\
\sphinxbottomrule
\end{tabular}
\sphinxtableafterendhook\par
\sphinxattableend\end{savenotes}


\begin{savenotes}\sphinxattablestart
\sphinxthistablewithglobalstyle
\centering
\begin{tabular}[t]{\X{33}{99}\X{33}{99}\X{33}{99}}
\sphinxtoprule
\sphinxtableatstartofbodyhook
\sphinxAtStartPar
位域 |
&
\sphinxAtStartPar
名称     | |
&
\sphinxAtStartPar
描述                                        | |
\\
\sphinxhline
\sphinxAtStartPar
31:8
&\begin{itemize}
\item {} 
\end{itemize}
&\begin{itemize}
\item {} 
\end{itemize}
\\
\sphinxhline
\sphinxAtStartPar
7:0
&
\sphinxAtStartPar
ID
&
\sphinxAtStartPar
标识符ID{[}12:5{]}                             |
\\
\sphinxbottomrule
\end{tabular}
\sphinxtableafterendhook\par
\sphinxattableend\end{savenotes}


\subsubsection{\textless{}扩展帧格式\textgreater{}数据寄存器3 DATA3}
\label{\detokenize{SWM241/_u529f_u80fd_u63cf_u8ff0/_u5c40_u57df_u7f51_u63a7_u5236_u5668:id113}}

\begin{savenotes}\sphinxattablestart
\sphinxthistablewithglobalstyle
\centering
\begin{tabular}[t]{\X{20}{100}\X{20}{100}\X{20}{100}\X{20}{100}\X{20}{100}}
\sphinxtoprule
\sphinxtableatstartofbodyhook
\sphinxAtStartPar
寄存器 |
&
\begin{DUlineblock}{0em}
\item[] 偏移 |
\end{DUlineblock}
&
\begin{DUlineblock}{0em}
\item[] 
\item[] {\color{red}\bfseries{}|}
\end{DUlineblock}
&
\sphinxAtStartPar
复位值 |    描 | |
&
\begin{DUlineblock}{0em}
\item[] |
  |
\end{DUlineblock}
\\
\sphinxhline
\sphinxAtStartPar
DATA3
&
\sphinxAtStartPar
0x50
&&
\sphinxAtStartPar
0 000000
&
\sphinxAtStartPar
数据3寄存器                |
\\
\sphinxbottomrule
\end{tabular}
\sphinxtableafterendhook\par
\sphinxattableend\end{savenotes}


\begin{savenotes}\sphinxattablestart
\sphinxthistablewithglobalstyle
\centering
\begin{tabular}[t]{\X{12}{96}\X{12}{96}\X{12}{96}\X{12}{96}\X{12}{96}\X{12}{96}\X{12}{96}\X{12}{96}}
\sphinxtoprule
\sphinxtableatstartofbodyhook
\sphinxAtStartPar
31
&
\sphinxAtStartPar
30
&
\sphinxAtStartPar
29
&
\sphinxAtStartPar
28
&
\sphinxAtStartPar
27
&
\sphinxAtStartPar
26
&
\sphinxAtStartPar
25
&
\sphinxAtStartPar
24
\\
\sphinxhline\begin{itemize}
\item {} 
\end{itemize}
&&&&&&&\\
\sphinxhline
\sphinxAtStartPar
23
&
\sphinxAtStartPar
22
&
\sphinxAtStartPar
21
&
\sphinxAtStartPar
20
&
\sphinxAtStartPar
19
&
\sphinxAtStartPar
18
&
\sphinxAtStartPar
17
&
\sphinxAtStartPar
16
\\
\sphinxhline\begin{itemize}
\item {} 
\end{itemize}
&&&&&&&\\
\sphinxhline
\sphinxAtStartPar
15
&
\sphinxAtStartPar
14
&
\sphinxAtStartPar
13
&
\sphinxAtStartPar
12
&
\sphinxAtStartPar
11
&
\sphinxAtStartPar
10
&
\sphinxAtStartPar
9
&
\sphinxAtStartPar
8
\\
\sphinxhline\begin{itemize}
\item {} 
\end{itemize}
&&&&&&&\\
\sphinxhline
\sphinxAtStartPar
7
&
\sphinxAtStartPar
6
&
\sphinxAtStartPar
5
&
\sphinxAtStartPar
4
&
\sphinxAtStartPar
3
&
\sphinxAtStartPar
2
&
\sphinxAtStartPar
1
&
\sphinxAtStartPar
0
\\
\sphinxhline
\sphinxAtStartPar
ID
&&&&&\begin{itemize}
\item {} 
\end{itemize}
&&\\
\sphinxbottomrule
\end{tabular}
\sphinxtableafterendhook\par
\sphinxattableend\end{savenotes}


\begin{savenotes}\sphinxattablestart
\sphinxthistablewithglobalstyle
\centering
\begin{tabular}[t]{\X{33}{99}\X{33}{99}\X{33}{99}}
\sphinxtoprule
\sphinxtableatstartofbodyhook
\sphinxAtStartPar
位域 |
&
\sphinxAtStartPar
名称     | |
&
\sphinxAtStartPar
描述                                        | |
\\
\sphinxhline
\sphinxAtStartPar
31:8
&\begin{itemize}
\item {} 
\end{itemize}
&\begin{itemize}
\item {} 
\end{itemize}
\\
\sphinxhline
\sphinxAtStartPar
7:3
&
\sphinxAtStartPar
ID
&
\sphinxAtStartPar
标识符ID{[}4:0{]}                              |
\\
\sphinxhline
\sphinxAtStartPar
2:0
&\begin{itemize}
\item {} 
\end{itemize}
&\begin{itemize}
\item {} 
\end{itemize}
\\
\sphinxbottomrule
\end{tabular}
\sphinxtableafterendhook\par
\sphinxattableend\end{savenotes}


\subsubsection{\textless{}扩展帧格式\textgreater{}数据寄存器4 DATA4}
\label{\detokenize{SWM241/_u529f_u80fd_u63cf_u8ff0/_u5c40_u57df_u7f51_u63a7_u5236_u5668:id116}}

\begin{savenotes}\sphinxattablestart
\sphinxthistablewithglobalstyle
\centering
\begin{tabular}[t]{\X{20}{100}\X{20}{100}\X{20}{100}\X{20}{100}\X{20}{100}}
\sphinxtoprule
\sphinxtableatstartofbodyhook
\sphinxAtStartPar
寄存器 |
&
\begin{DUlineblock}{0em}
\item[] 偏移 |
\end{DUlineblock}
&
\begin{DUlineblock}{0em}
\item[] 
\item[] {\color{red}\bfseries{}|}
\end{DUlineblock}
&
\sphinxAtStartPar
复位值 |    描 | |
&
\begin{DUlineblock}{0em}
\item[] |
  |
\end{DUlineblock}
\\
\sphinxhline
\sphinxAtStartPar
DATA4
&
\sphinxAtStartPar
0x54
&&
\sphinxAtStartPar
0 000000
&
\sphinxAtStartPar
数据4寄存器                |
\\
\sphinxbottomrule
\end{tabular}
\sphinxtableafterendhook\par
\sphinxattableend\end{savenotes}


\begin{savenotes}\sphinxattablestart
\sphinxthistablewithglobalstyle
\centering
\begin{tabular}[t]{\X{12}{96}\X{12}{96}\X{12}{96}\X{12}{96}\X{12}{96}\X{12}{96}\X{12}{96}\X{12}{96}}
\sphinxtoprule
\sphinxtableatstartofbodyhook
\sphinxAtStartPar
31
&
\sphinxAtStartPar
30
&
\sphinxAtStartPar
29
&
\sphinxAtStartPar
28
&
\sphinxAtStartPar
27
&
\sphinxAtStartPar
26
&
\sphinxAtStartPar
25
&
\sphinxAtStartPar
24
\\
\sphinxhline\begin{itemize}
\item {} 
\end{itemize}
&&&&&&&\\
\sphinxhline
\sphinxAtStartPar
23
&
\sphinxAtStartPar
22
&
\sphinxAtStartPar
21
&
\sphinxAtStartPar
20
&
\sphinxAtStartPar
19
&
\sphinxAtStartPar
18
&
\sphinxAtStartPar
17
&
\sphinxAtStartPar
16
\\
\sphinxhline\begin{itemize}
\item {} 
\end{itemize}
&&&&&&&\\
\sphinxhline
\sphinxAtStartPar
15
&
\sphinxAtStartPar
14
&
\sphinxAtStartPar
13
&
\sphinxAtStartPar
12
&
\sphinxAtStartPar
11
&
\sphinxAtStartPar
10
&
\sphinxAtStartPar
9
&
\sphinxAtStartPar
8
\\
\sphinxhline\begin{itemize}
\item {} 
\end{itemize}
&&&&&&&\\
\sphinxhline
\sphinxAtStartPar
7
&
\sphinxAtStartPar
6
&
\sphinxAtStartPar
5
&
\sphinxAtStartPar
4
&
\sphinxAtStartPar
3
&
\sphinxAtStartPar
2
&
\sphinxAtStartPar
1
&
\sphinxAtStartPar
0
\\
\sphinxhline
\sphinxAtStartPar
DATA
&&&&&&&\\
\sphinxbottomrule
\end{tabular}
\sphinxtableafterendhook\par
\sphinxattableend\end{savenotes}


\begin{savenotes}\sphinxattablestart
\sphinxthistablewithglobalstyle
\centering
\begin{tabular}[t]{\X{33}{99}\X{33}{99}\X{33}{99}}
\sphinxtoprule
\sphinxtableatstartofbodyhook
\sphinxAtStartPar
位域 |
&
\sphinxAtStartPar
名称     | |
&
\sphinxAtStartPar
描述                                        | |
\\
\sphinxhline
\sphinxAtStartPar
31:8
&\begin{itemize}
\item {} 
\end{itemize}
&\begin{itemize}
\item {} 
\end{itemize}
\\
\sphinxhline
\sphinxAtStartPar
7:0
&
\sphinxAtStartPar
DATA
&
\sphinxAtStartPar
数据字节0                                   |
\\
\sphinxbottomrule
\end{tabular}
\sphinxtableafterendhook\par
\sphinxattableend\end{savenotes}


\subsubsection{\textless{}扩展帧格式\textgreater{}数据寄存器5 DATA5}
\label{\detokenize{SWM241/_u529f_u80fd_u63cf_u8ff0/_u5c40_u57df_u7f51_u63a7_u5236_u5668:id119}}

\begin{savenotes}\sphinxattablestart
\sphinxthistablewithglobalstyle
\centering
\begin{tabular}[t]{\X{20}{100}\X{20}{100}\X{20}{100}\X{20}{100}\X{20}{100}}
\sphinxtoprule
\sphinxtableatstartofbodyhook
\sphinxAtStartPar
寄存器 |
&
\begin{DUlineblock}{0em}
\item[] 偏移 |
\end{DUlineblock}
&
\begin{DUlineblock}{0em}
\item[] 
\item[] {\color{red}\bfseries{}|}
\end{DUlineblock}
&
\sphinxAtStartPar
复位值 |    描 | |
&
\begin{DUlineblock}{0em}
\item[] |
  |
\end{DUlineblock}
\\
\sphinxhline
\sphinxAtStartPar
DATA5
&
\sphinxAtStartPar
0x58
&&
\sphinxAtStartPar
0 000000
&
\sphinxAtStartPar
数据5寄存器                |
\\
\sphinxbottomrule
\end{tabular}
\sphinxtableafterendhook\par
\sphinxattableend\end{savenotes}


\begin{savenotes}\sphinxattablestart
\sphinxthistablewithglobalstyle
\centering
\begin{tabular}[t]{\X{12}{96}\X{12}{96}\X{12}{96}\X{12}{96}\X{12}{96}\X{12}{96}\X{12}{96}\X{12}{96}}
\sphinxtoprule
\sphinxtableatstartofbodyhook
\sphinxAtStartPar
31
&
\sphinxAtStartPar
30
&
\sphinxAtStartPar
29
&
\sphinxAtStartPar
28
&
\sphinxAtStartPar
27
&
\sphinxAtStartPar
26
&
\sphinxAtStartPar
25
&
\sphinxAtStartPar
24
\\
\sphinxhline\begin{itemize}
\item {} 
\end{itemize}
&&&&&&&\\
\sphinxhline
\sphinxAtStartPar
23
&
\sphinxAtStartPar
22
&
\sphinxAtStartPar
21
&
\sphinxAtStartPar
20
&
\sphinxAtStartPar
19
&
\sphinxAtStartPar
18
&
\sphinxAtStartPar
17
&
\sphinxAtStartPar
16
\\
\sphinxhline\begin{itemize}
\item {} 
\end{itemize}
&&&&&&&\\
\sphinxhline
\sphinxAtStartPar
15
&
\sphinxAtStartPar
14
&
\sphinxAtStartPar
13
&
\sphinxAtStartPar
12
&
\sphinxAtStartPar
11
&
\sphinxAtStartPar
10
&
\sphinxAtStartPar
9
&
\sphinxAtStartPar
8
\\
\sphinxhline\begin{itemize}
\item {} 
\end{itemize}
&&&&&&&\\
\sphinxhline
\sphinxAtStartPar
7
&
\sphinxAtStartPar
6
&
\sphinxAtStartPar
5
&
\sphinxAtStartPar
4
&
\sphinxAtStartPar
3
&
\sphinxAtStartPar
2
&
\sphinxAtStartPar
1
&
\sphinxAtStartPar
0
\\
\sphinxhline
\sphinxAtStartPar
DATA
&&&&&&&\\
\sphinxbottomrule
\end{tabular}
\sphinxtableafterendhook\par
\sphinxattableend\end{savenotes}


\begin{savenotes}\sphinxattablestart
\sphinxthistablewithglobalstyle
\centering
\begin{tabular}[t]{\X{33}{99}\X{33}{99}\X{33}{99}}
\sphinxtoprule
\sphinxtableatstartofbodyhook
\sphinxAtStartPar
位域 |
&
\sphinxAtStartPar
名称     | |
&
\sphinxAtStartPar
描述                                        | |
\\
\sphinxhline
\sphinxAtStartPar
31:8
&\begin{itemize}
\item {} 
\end{itemize}
&\begin{itemize}
\item {} 
\end{itemize}
\\
\sphinxhline
\sphinxAtStartPar
7:0
&
\sphinxAtStartPar
DATA
&
\sphinxAtStartPar
数据字节1                                   |
\\
\sphinxbottomrule
\end{tabular}
\sphinxtableafterendhook\par
\sphinxattableend\end{savenotes}


\subsubsection{\textless{}扩展帧格式\textgreater{}数据寄存器6 DATA6}
\label{\detokenize{SWM241/_u529f_u80fd_u63cf_u8ff0/_u5c40_u57df_u7f51_u63a7_u5236_u5668:id122}}

\begin{savenotes}\sphinxattablestart
\sphinxthistablewithglobalstyle
\centering
\begin{tabular}[t]{\X{20}{100}\X{20}{100}\X{20}{100}\X{20}{100}\X{20}{100}}
\sphinxtoprule
\sphinxtableatstartofbodyhook
\sphinxAtStartPar
寄存器 |
&
\begin{DUlineblock}{0em}
\item[] 偏移 |
\end{DUlineblock}
&
\begin{DUlineblock}{0em}
\item[] 
\item[] {\color{red}\bfseries{}|}
\end{DUlineblock}
&
\sphinxAtStartPar
复位值 |    描 | |
&
\begin{DUlineblock}{0em}
\item[] |
  |
\end{DUlineblock}
\\
\sphinxhline
\sphinxAtStartPar
DATA6
&
\sphinxAtStartPar
0x5C
&&
\sphinxAtStartPar
0 000000
&
\sphinxAtStartPar
数据6寄存器                |
\\
\sphinxbottomrule
\end{tabular}
\sphinxtableafterendhook\par
\sphinxattableend\end{savenotes}


\begin{savenotes}\sphinxattablestart
\sphinxthistablewithglobalstyle
\centering
\begin{tabular}[t]{\X{12}{96}\X{12}{96}\X{12}{96}\X{12}{96}\X{12}{96}\X{12}{96}\X{12}{96}\X{12}{96}}
\sphinxtoprule
\sphinxtableatstartofbodyhook
\sphinxAtStartPar
31
&
\sphinxAtStartPar
30
&
\sphinxAtStartPar
29
&
\sphinxAtStartPar
28
&
\sphinxAtStartPar
27
&
\sphinxAtStartPar
26
&
\sphinxAtStartPar
25
&
\sphinxAtStartPar
24
\\
\sphinxhline\begin{itemize}
\item {} 
\end{itemize}
&&&&&&&\\
\sphinxhline
\sphinxAtStartPar
23
&
\sphinxAtStartPar
22
&
\sphinxAtStartPar
21
&
\sphinxAtStartPar
20
&
\sphinxAtStartPar
19
&
\sphinxAtStartPar
18
&
\sphinxAtStartPar
17
&
\sphinxAtStartPar
16
\\
\sphinxhline\begin{itemize}
\item {} 
\end{itemize}
&&&&&&&\\
\sphinxhline
\sphinxAtStartPar
15
&
\sphinxAtStartPar
14
&
\sphinxAtStartPar
13
&
\sphinxAtStartPar
12
&
\sphinxAtStartPar
11
&
\sphinxAtStartPar
10
&
\sphinxAtStartPar
9
&
\sphinxAtStartPar
8
\\
\sphinxhline\begin{itemize}
\item {} 
\end{itemize}
&&&&&&&\\
\sphinxhline
\sphinxAtStartPar
7
&
\sphinxAtStartPar
6
&
\sphinxAtStartPar
5
&
\sphinxAtStartPar
4
&
\sphinxAtStartPar
3
&
\sphinxAtStartPar
2
&
\sphinxAtStartPar
1
&
\sphinxAtStartPar
0
\\
\sphinxhline
\sphinxAtStartPar
DATA
&&&&&&&\\
\sphinxbottomrule
\end{tabular}
\sphinxtableafterendhook\par
\sphinxattableend\end{savenotes}


\begin{savenotes}\sphinxattablestart
\sphinxthistablewithglobalstyle
\centering
\begin{tabular}[t]{\X{33}{99}\X{33}{99}\X{33}{99}}
\sphinxtoprule
\sphinxtableatstartofbodyhook
\sphinxAtStartPar
位域 |
&
\sphinxAtStartPar
名称     | |
&
\sphinxAtStartPar
描述                                        | |
\\
\sphinxhline
\sphinxAtStartPar
31:8
&\begin{itemize}
\item {} 
\end{itemize}
&\begin{itemize}
\item {} 
\end{itemize}
\\
\sphinxhline
\sphinxAtStartPar
7:0
&
\sphinxAtStartPar
DATA
&
\sphinxAtStartPar
数据字节2                                   |
\\
\sphinxbottomrule
\end{tabular}
\sphinxtableafterendhook\par
\sphinxattableend\end{savenotes}


\subsubsection{\textless{}扩展帧格式\textgreater{}数据寄存器7 DATA7}
\label{\detokenize{SWM241/_u529f_u80fd_u63cf_u8ff0/_u5c40_u57df_u7f51_u63a7_u5236_u5668:id125}}

\begin{savenotes}\sphinxattablestart
\sphinxthistablewithglobalstyle
\centering
\begin{tabular}[t]{\X{20}{100}\X{20}{100}\X{20}{100}\X{20}{100}\X{20}{100}}
\sphinxtoprule
\sphinxtableatstartofbodyhook
\sphinxAtStartPar
寄存器 |
&
\begin{DUlineblock}{0em}
\item[] 偏移 |
\end{DUlineblock}
&
\begin{DUlineblock}{0em}
\item[] 
\item[] {\color{red}\bfseries{}|}
\end{DUlineblock}
&
\sphinxAtStartPar
复位值 |    描 | |
&
\begin{DUlineblock}{0em}
\item[] |
  |
\end{DUlineblock}
\\
\sphinxhline
\sphinxAtStartPar
DATA7
&
\sphinxAtStartPar
0x60
&&
\sphinxAtStartPar
0 000000
&
\sphinxAtStartPar
数据7寄存器                |
\\
\sphinxbottomrule
\end{tabular}
\sphinxtableafterendhook\par
\sphinxattableend\end{savenotes}


\begin{savenotes}\sphinxattablestart
\sphinxthistablewithglobalstyle
\centering
\begin{tabular}[t]{\X{12}{96}\X{12}{96}\X{12}{96}\X{12}{96}\X{12}{96}\X{12}{96}\X{12}{96}\X{12}{96}}
\sphinxtoprule
\sphinxtableatstartofbodyhook
\sphinxAtStartPar
31
&
\sphinxAtStartPar
30
&
\sphinxAtStartPar
29
&
\sphinxAtStartPar
28
&
\sphinxAtStartPar
27
&
\sphinxAtStartPar
26
&
\sphinxAtStartPar
25
&
\sphinxAtStartPar
24
\\
\sphinxhline\begin{itemize}
\item {} 
\end{itemize}
&&&&&&&\\
\sphinxhline
\sphinxAtStartPar
23
&
\sphinxAtStartPar
22
&
\sphinxAtStartPar
21
&
\sphinxAtStartPar
20
&
\sphinxAtStartPar
19
&
\sphinxAtStartPar
18
&
\sphinxAtStartPar
17
&
\sphinxAtStartPar
16
\\
\sphinxhline\begin{itemize}
\item {} 
\end{itemize}
&&&&&&&\\
\sphinxhline
\sphinxAtStartPar
15
&
\sphinxAtStartPar
14
&
\sphinxAtStartPar
13
&
\sphinxAtStartPar
12
&
\sphinxAtStartPar
11
&
\sphinxAtStartPar
10
&
\sphinxAtStartPar
9
&
\sphinxAtStartPar
8
\\
\sphinxhline\begin{itemize}
\item {} 
\end{itemize}
&&&&&&&\\
\sphinxhline
\sphinxAtStartPar
7
&
\sphinxAtStartPar
6
&
\sphinxAtStartPar
5
&
\sphinxAtStartPar
4
&
\sphinxAtStartPar
3
&
\sphinxAtStartPar
2
&
\sphinxAtStartPar
1
&
\sphinxAtStartPar
0
\\
\sphinxhline
\sphinxAtStartPar
DATA
&&&&&&&\\
\sphinxbottomrule
\end{tabular}
\sphinxtableafterendhook\par
\sphinxattableend\end{savenotes}


\begin{savenotes}\sphinxattablestart
\sphinxthistablewithglobalstyle
\centering
\begin{tabular}[t]{\X{33}{99}\X{33}{99}\X{33}{99}}
\sphinxtoprule
\sphinxtableatstartofbodyhook
\sphinxAtStartPar
位域 |
&
\sphinxAtStartPar
名称     | |
&
\sphinxAtStartPar
描述                                        | |
\\
\sphinxhline
\sphinxAtStartPar
31:8
&\begin{itemize}
\item {} 
\end{itemize}
&\begin{itemize}
\item {} 
\end{itemize}
\\
\sphinxhline
\sphinxAtStartPar
7:0
&
\sphinxAtStartPar
DATA
&
\sphinxAtStartPar
数据字节3                                   |
\\
\sphinxbottomrule
\end{tabular}
\sphinxtableafterendhook\par
\sphinxattableend\end{savenotes}


\subsubsection{\textless{}扩展帧格式\textgreater{}数据寄存器8 DATA8}
\label{\detokenize{SWM241/_u529f_u80fd_u63cf_u8ff0/_u5c40_u57df_u7f51_u63a7_u5236_u5668:id128}}

\begin{savenotes}\sphinxattablestart
\sphinxthistablewithglobalstyle
\centering
\begin{tabular}[t]{\X{20}{100}\X{20}{100}\X{20}{100}\X{20}{100}\X{20}{100}}
\sphinxtoprule
\sphinxtableatstartofbodyhook
\sphinxAtStartPar
寄存器 |
&
\begin{DUlineblock}{0em}
\item[] 偏移 |
\end{DUlineblock}
&
\begin{DUlineblock}{0em}
\item[] 
\item[] {\color{red}\bfseries{}|}
\end{DUlineblock}
&
\sphinxAtStartPar
复位值 |    描 | |
&
\begin{DUlineblock}{0em}
\item[] |
  |
\end{DUlineblock}
\\
\sphinxhline
\sphinxAtStartPar
DATA8
&
\sphinxAtStartPar
0x64
&&
\sphinxAtStartPar
0 000000
&
\sphinxAtStartPar
数据8寄存器                |
\\
\sphinxbottomrule
\end{tabular}
\sphinxtableafterendhook\par
\sphinxattableend\end{savenotes}


\begin{savenotes}\sphinxattablestart
\sphinxthistablewithglobalstyle
\centering
\begin{tabular}[t]{\X{12}{96}\X{12}{96}\X{12}{96}\X{12}{96}\X{12}{96}\X{12}{96}\X{12}{96}\X{12}{96}}
\sphinxtoprule
\sphinxtableatstartofbodyhook
\sphinxAtStartPar
31
&
\sphinxAtStartPar
30
&
\sphinxAtStartPar
29
&
\sphinxAtStartPar
28
&
\sphinxAtStartPar
27
&
\sphinxAtStartPar
26
&
\sphinxAtStartPar
25
&
\sphinxAtStartPar
24
\\
\sphinxhline\begin{itemize}
\item {} 
\end{itemize}
&&&&&&&\\
\sphinxhline
\sphinxAtStartPar
23
&
\sphinxAtStartPar
22
&
\sphinxAtStartPar
21
&
\sphinxAtStartPar
20
&
\sphinxAtStartPar
19
&
\sphinxAtStartPar
18
&
\sphinxAtStartPar
17
&
\sphinxAtStartPar
16
\\
\sphinxhline\begin{itemize}
\item {} 
\end{itemize}
&&&&&&&\\
\sphinxhline
\sphinxAtStartPar
15
&
\sphinxAtStartPar
14
&
\sphinxAtStartPar
13
&
\sphinxAtStartPar
12
&
\sphinxAtStartPar
11
&
\sphinxAtStartPar
10
&
\sphinxAtStartPar
9
&
\sphinxAtStartPar
8
\\
\sphinxhline\begin{itemize}
\item {} 
\end{itemize}
&&&&&&&\\
\sphinxhline
\sphinxAtStartPar
7
&
\sphinxAtStartPar
6
&
\sphinxAtStartPar
5
&
\sphinxAtStartPar
4
&
\sphinxAtStartPar
3
&
\sphinxAtStartPar
2
&
\sphinxAtStartPar
1
&
\sphinxAtStartPar
0
\\
\sphinxhline
\sphinxAtStartPar
DATA
&&&&&&&\\
\sphinxbottomrule
\end{tabular}
\sphinxtableafterendhook\par
\sphinxattableend\end{savenotes}


\begin{savenotes}\sphinxattablestart
\sphinxthistablewithglobalstyle
\centering
\begin{tabular}[t]{\X{33}{99}\X{33}{99}\X{33}{99}}
\sphinxtoprule
\sphinxtableatstartofbodyhook
\sphinxAtStartPar
位域 |
&
\sphinxAtStartPar
名称     | |
&
\sphinxAtStartPar
描述                                        | |
\\
\sphinxhline
\sphinxAtStartPar
31:8
&\begin{itemize}
\item {} 
\end{itemize}
&\begin{itemize}
\item {} 
\end{itemize}
\\
\sphinxhline
\sphinxAtStartPar
7:0
&
\sphinxAtStartPar
DATA
&
\sphinxAtStartPar
数据字节4                                   |
\\
\sphinxbottomrule
\end{tabular}
\sphinxtableafterendhook\par
\sphinxattableend\end{savenotes}


\subsubsection{\textless{}扩展帧格式\textgreater{}数据寄存器9 DATA9}
\label{\detokenize{SWM241/_u529f_u80fd_u63cf_u8ff0/_u5c40_u57df_u7f51_u63a7_u5236_u5668:id131}}

\begin{savenotes}\sphinxattablestart
\sphinxthistablewithglobalstyle
\centering
\begin{tabular}[t]{\X{20}{100}\X{20}{100}\X{20}{100}\X{20}{100}\X{20}{100}}
\sphinxtoprule
\sphinxtableatstartofbodyhook
\sphinxAtStartPar
寄存器 |
&
\begin{DUlineblock}{0em}
\item[] 偏移 |
\end{DUlineblock}
&
\begin{DUlineblock}{0em}
\item[] 
\item[] {\color{red}\bfseries{}|}
\end{DUlineblock}
&
\sphinxAtStartPar
复位值 |    描 | |
&
\begin{DUlineblock}{0em}
\item[] |
  |
\end{DUlineblock}
\\
\sphinxhline
\sphinxAtStartPar
DATA9
&
\sphinxAtStartPar
0x68
&&
\sphinxAtStartPar
0 000000
&
\sphinxAtStartPar
数据9寄存器                |
\\
\sphinxbottomrule
\end{tabular}
\sphinxtableafterendhook\par
\sphinxattableend\end{savenotes}


\begin{savenotes}\sphinxattablestart
\sphinxthistablewithglobalstyle
\centering
\begin{tabular}[t]{\X{12}{96}\X{12}{96}\X{12}{96}\X{12}{96}\X{12}{96}\X{12}{96}\X{12}{96}\X{12}{96}}
\sphinxtoprule
\sphinxtableatstartofbodyhook
\sphinxAtStartPar
31
&
\sphinxAtStartPar
30
&
\sphinxAtStartPar
29
&
\sphinxAtStartPar
28
&
\sphinxAtStartPar
27
&
\sphinxAtStartPar
26
&
\sphinxAtStartPar
25
&
\sphinxAtStartPar
24
\\
\sphinxhline\begin{itemize}
\item {} 
\end{itemize}
&&&&&&&\\
\sphinxhline
\sphinxAtStartPar
23
&
\sphinxAtStartPar
22
&
\sphinxAtStartPar
21
&
\sphinxAtStartPar
20
&
\sphinxAtStartPar
19
&
\sphinxAtStartPar
18
&
\sphinxAtStartPar
17
&
\sphinxAtStartPar
16
\\
\sphinxhline\begin{itemize}
\item {} 
\end{itemize}
&&&&&&&\\
\sphinxhline
\sphinxAtStartPar
15
&
\sphinxAtStartPar
14
&
\sphinxAtStartPar
13
&
\sphinxAtStartPar
12
&
\sphinxAtStartPar
11
&
\sphinxAtStartPar
10
&
\sphinxAtStartPar
9
&
\sphinxAtStartPar
8
\\
\sphinxhline\begin{itemize}
\item {} 
\end{itemize}
&&&&&&&\\
\sphinxhline
\sphinxAtStartPar
7
&
\sphinxAtStartPar
6
&
\sphinxAtStartPar
5
&
\sphinxAtStartPar
4
&
\sphinxAtStartPar
3
&
\sphinxAtStartPar
2
&
\sphinxAtStartPar
1
&
\sphinxAtStartPar
0
\\
\sphinxhline
\sphinxAtStartPar
DATA
&&&&&&&\\
\sphinxbottomrule
\end{tabular}
\sphinxtableafterendhook\par
\sphinxattableend\end{savenotes}


\begin{savenotes}\sphinxattablestart
\sphinxthistablewithglobalstyle
\centering
\begin{tabular}[t]{\X{33}{99}\X{33}{99}\X{33}{99}}
\sphinxtoprule
\sphinxtableatstartofbodyhook
\sphinxAtStartPar
位域 |
&
\sphinxAtStartPar
名称     | |
&
\sphinxAtStartPar
描述                                        | |
\\
\sphinxhline
\sphinxAtStartPar
31:8
&\begin{itemize}
\item {} 
\end{itemize}
&\begin{itemize}
\item {} 
\end{itemize}
\\
\sphinxhline
\sphinxAtStartPar
7:0
&
\sphinxAtStartPar
DATA
&
\sphinxAtStartPar
数据字节5                                   |
\\
\sphinxbottomrule
\end{tabular}
\sphinxtableafterendhook\par
\sphinxattableend\end{savenotes}


\subsubsection{\textless{}扩展帧格式\textgreater{}数据寄存器10 DATA10}
\label{\detokenize{SWM241/_u529f_u80fd_u63cf_u8ff0/_u5c40_u57df_u7f51_u63a7_u5236_u5668:data10}}

\begin{savenotes}\sphinxattablestart
\sphinxthistablewithglobalstyle
\centering
\begin{tabular}[t]{\X{20}{100}\X{20}{100}\X{20}{100}\X{20}{100}\X{20}{100}}
\sphinxtoprule
\sphinxtableatstartofbodyhook
\sphinxAtStartPar
寄存器 |
&
\begin{DUlineblock}{0em}
\item[] 偏移 |
\end{DUlineblock}
&
\begin{DUlineblock}{0em}
\item[] 
\item[] {\color{red}\bfseries{}|}
\end{DUlineblock}
&
\sphinxAtStartPar
复位值 |    描 | |
&
\begin{DUlineblock}{0em}
\item[] |
  |
\end{DUlineblock}
\\
\sphinxhline
\sphinxAtStartPar
DATA10
&
\sphinxAtStartPar
0x6C
&&
\sphinxAtStartPar
0 000000
&
\sphinxAtStartPar
数据10寄存器               |
\\
\sphinxbottomrule
\end{tabular}
\sphinxtableafterendhook\par
\sphinxattableend\end{savenotes}


\begin{savenotes}\sphinxattablestart
\sphinxthistablewithglobalstyle
\centering
\begin{tabular}[t]{\X{12}{96}\X{12}{96}\X{12}{96}\X{12}{96}\X{12}{96}\X{12}{96}\X{12}{96}\X{12}{96}}
\sphinxtoprule
\sphinxtableatstartofbodyhook
\sphinxAtStartPar
31
&
\sphinxAtStartPar
30
&
\sphinxAtStartPar
29
&
\sphinxAtStartPar
28
&
\sphinxAtStartPar
27
&
\sphinxAtStartPar
26
&
\sphinxAtStartPar
25
&
\sphinxAtStartPar
24
\\
\sphinxhline\begin{itemize}
\item {} 
\end{itemize}
&&&&&&&\\
\sphinxhline
\sphinxAtStartPar
23
&
\sphinxAtStartPar
22
&
\sphinxAtStartPar
21
&
\sphinxAtStartPar
20
&
\sphinxAtStartPar
19
&
\sphinxAtStartPar
18
&
\sphinxAtStartPar
17
&
\sphinxAtStartPar
16
\\
\sphinxhline\begin{itemize}
\item {} 
\end{itemize}
&&&&&&&\\
\sphinxhline
\sphinxAtStartPar
15
&
\sphinxAtStartPar
14
&
\sphinxAtStartPar
13
&
\sphinxAtStartPar
12
&
\sphinxAtStartPar
11
&
\sphinxAtStartPar
10
&
\sphinxAtStartPar
9
&
\sphinxAtStartPar
8
\\
\sphinxhline\begin{itemize}
\item {} 
\end{itemize}
&&&&&&&\\
\sphinxhline
\sphinxAtStartPar
7
&
\sphinxAtStartPar
6
&
\sphinxAtStartPar
5
&
\sphinxAtStartPar
4
&
\sphinxAtStartPar
3
&
\sphinxAtStartPar
2
&
\sphinxAtStartPar
1
&
\sphinxAtStartPar
0
\\
\sphinxhline
\sphinxAtStartPar
DATA
&&&&&&&\\
\sphinxbottomrule
\end{tabular}
\sphinxtableafterendhook\par
\sphinxattableend\end{savenotes}


\begin{savenotes}\sphinxattablestart
\sphinxthistablewithglobalstyle
\centering
\begin{tabular}[t]{\X{33}{99}\X{33}{99}\X{33}{99}}
\sphinxtoprule
\sphinxtableatstartofbodyhook
\sphinxAtStartPar
位域 |
&
\sphinxAtStartPar
名称     | |
&
\sphinxAtStartPar
描述                                        | |
\\
\sphinxhline
\sphinxAtStartPar
31:8
&\begin{itemize}
\item {} 
\end{itemize}
&\begin{itemize}
\item {} 
\end{itemize}
\\
\sphinxhline
\sphinxAtStartPar
7:0
&
\sphinxAtStartPar
DATA
&
\sphinxAtStartPar
数据字节6                                   |
\\
\sphinxbottomrule
\end{tabular}
\sphinxtableafterendhook\par
\sphinxattableend\end{savenotes}


\subsubsection{\textless{}扩展帧格式\textgreater{}数据寄存器11 DATA11}
\label{\detokenize{SWM241/_u529f_u80fd_u63cf_u8ff0/_u5c40_u57df_u7f51_u63a7_u5236_u5668:data11}}

\begin{savenotes}\sphinxattablestart
\sphinxthistablewithglobalstyle
\centering
\begin{tabular}[t]{\X{20}{100}\X{20}{100}\X{20}{100}\X{20}{100}\X{20}{100}}
\sphinxtoprule
\sphinxtableatstartofbodyhook
\sphinxAtStartPar
寄存器 |
&
\begin{DUlineblock}{0em}
\item[] 偏移 |
\end{DUlineblock}
&
\begin{DUlineblock}{0em}
\item[] 
\item[] {\color{red}\bfseries{}|}
\end{DUlineblock}
&
\sphinxAtStartPar
复位值 |    描 | |
&
\begin{DUlineblock}{0em}
\item[] |
  |
\end{DUlineblock}
\\
\sphinxhline
\sphinxAtStartPar
DATA11
&
\sphinxAtStartPar
0x70
&&
\sphinxAtStartPar
0 000000
&
\sphinxAtStartPar
数据11寄存器               |
\\
\sphinxbottomrule
\end{tabular}
\sphinxtableafterendhook\par
\sphinxattableend\end{savenotes}


\begin{savenotes}\sphinxattablestart
\sphinxthistablewithglobalstyle
\centering
\begin{tabular}[t]{\X{12}{96}\X{12}{96}\X{12}{96}\X{12}{96}\X{12}{96}\X{12}{96}\X{12}{96}\X{12}{96}}
\sphinxtoprule
\sphinxtableatstartofbodyhook
\sphinxAtStartPar
31
&
\sphinxAtStartPar
30
&
\sphinxAtStartPar
29
&
\sphinxAtStartPar
28
&
\sphinxAtStartPar
27
&
\sphinxAtStartPar
26
&
\sphinxAtStartPar
25
&
\sphinxAtStartPar
24
\\
\sphinxhline\begin{itemize}
\item {} 
\end{itemize}
&&&&&&&\\
\sphinxhline
\sphinxAtStartPar
23
&
\sphinxAtStartPar
22
&
\sphinxAtStartPar
21
&
\sphinxAtStartPar
20
&
\sphinxAtStartPar
19
&
\sphinxAtStartPar
18
&
\sphinxAtStartPar
17
&
\sphinxAtStartPar
16
\\
\sphinxhline\begin{itemize}
\item {} 
\end{itemize}
&&&&&&&\\
\sphinxhline
\sphinxAtStartPar
15
&
\sphinxAtStartPar
14
&
\sphinxAtStartPar
13
&
\sphinxAtStartPar
12
&
\sphinxAtStartPar
11
&
\sphinxAtStartPar
10
&
\sphinxAtStartPar
9
&
\sphinxAtStartPar
8
\\
\sphinxhline\begin{itemize}
\item {} 
\end{itemize}
&&&&&&&\\
\sphinxhline
\sphinxAtStartPar
7
&
\sphinxAtStartPar
6
&
\sphinxAtStartPar
5
&
\sphinxAtStartPar
4
&
\sphinxAtStartPar
3
&
\sphinxAtStartPar
2
&
\sphinxAtStartPar
1
&
\sphinxAtStartPar
0
\\
\sphinxhline
\sphinxAtStartPar
DATA
&&&&&&&\\
\sphinxbottomrule
\end{tabular}
\sphinxtableafterendhook\par
\sphinxattableend\end{savenotes}


\begin{savenotes}\sphinxattablestart
\sphinxthistablewithglobalstyle
\centering
\begin{tabular}[t]{\X{33}{99}\X{33}{99}\X{33}{99}}
\sphinxtoprule
\sphinxtableatstartofbodyhook
\sphinxAtStartPar
位域 |
&
\sphinxAtStartPar
名称     | |
&
\sphinxAtStartPar
描述                                        | |
\\
\sphinxhline
\sphinxAtStartPar
31:8
&\begin{itemize}
\item {} 
\end{itemize}
&\begin{itemize}
\item {} 
\end{itemize}
\\
\sphinxhline
\sphinxAtStartPar
7:0
&
\sphinxAtStartPar
DATA
&
\sphinxAtStartPar
数据字节7                                   |
\\
\sphinxbottomrule
\end{tabular}
\sphinxtableafterendhook\par
\sphinxattableend\end{savenotes}


\subsubsection{接收报文数目寄存器RMCNT}
\label{\detokenize{SWM241/_u529f_u80fd_u63cf_u8ff0/_u5c40_u57df_u7f51_u63a7_u5236_u5668:rmcnt}}

\begin{savenotes}\sphinxattablestart
\sphinxthistablewithglobalstyle
\centering
\begin{tabular}[t]{\X{20}{100}\X{20}{100}\X{20}{100}\X{20}{100}\X{20}{100}}
\sphinxtoprule
\sphinxtableatstartofbodyhook
\sphinxAtStartPar
寄存器 |
&
\begin{DUlineblock}{0em}
\item[] 偏移 |
\end{DUlineblock}
&
\begin{DUlineblock}{0em}
\item[] 
\item[] {\color{red}\bfseries{}|}
\end{DUlineblock}
&
\sphinxAtStartPar
复位值 |    描 | |
&
\begin{DUlineblock}{0em}
\item[] |
  |
\end{DUlineblock}
\\
\sphinxhline
\sphinxAtStartPar
RMCNT
&
\sphinxAtStartPar
0x74
&&
\sphinxAtStartPar
0 000000
&
\sphinxAtStartPar
接收数据计数寄存器         |
\\
\sphinxbottomrule
\end{tabular}
\sphinxtableafterendhook\par
\sphinxattableend\end{savenotes}


\begin{savenotes}\sphinxattablestart
\sphinxthistablewithglobalstyle
\centering
\begin{tabular}[t]{\X{12}{96}\X{12}{96}\X{12}{96}\X{12}{96}\X{12}{96}\X{12}{96}\X{12}{96}\X{12}{96}}
\sphinxtoprule
\sphinxtableatstartofbodyhook
\sphinxAtStartPar
31
&
\sphinxAtStartPar
30
&
\sphinxAtStartPar
29
&
\sphinxAtStartPar
28
&
\sphinxAtStartPar
27
&
\sphinxAtStartPar
26
&
\sphinxAtStartPar
25
&
\sphinxAtStartPar
24
\\
\sphinxhline\begin{itemize}
\item {} 
\end{itemize}
&&&&&&&\\
\sphinxhline
\sphinxAtStartPar
23
&
\sphinxAtStartPar
22
&
\sphinxAtStartPar
21
&
\sphinxAtStartPar
20
&
\sphinxAtStartPar
19
&
\sphinxAtStartPar
18
&
\sphinxAtStartPar
17
&
\sphinxAtStartPar
16
\\
\sphinxhline\begin{itemize}
\item {} 
\end{itemize}
&&&&&&&\\
\sphinxhline
\sphinxAtStartPar
15
&
\sphinxAtStartPar
14
&
\sphinxAtStartPar
13
&
\sphinxAtStartPar
12
&
\sphinxAtStartPar
11
&
\sphinxAtStartPar
10
&
\sphinxAtStartPar
9
&
\sphinxAtStartPar
8
\\
\sphinxhline\begin{itemize}
\item {} 
\end{itemize}
&&&&&&&\\
\sphinxhline
\sphinxAtStartPar
7
&
\sphinxAtStartPar
6
&
\sphinxAtStartPar
5
&
\sphinxAtStartPar
4
&
\sphinxAtStartPar
3
&
\sphinxAtStartPar
2
&
\sphinxAtStartPar
1
&
\sphinxAtStartPar
0
\\
\sphinxhline\begin{itemize}
\item {} 
\end{itemize}
&&&
\sphinxAtStartPar
RMC
&&&&\\
\sphinxbottomrule
\end{tabular}
\sphinxtableafterendhook\par
\sphinxattableend\end{savenotes}


\begin{savenotes}\sphinxattablestart
\sphinxthistablewithglobalstyle
\centering
\begin{tabular}[t]{\X{33}{99}\X{33}{99}\X{33}{99}}
\sphinxtoprule
\sphinxtableatstartofbodyhook
\sphinxAtStartPar
位域 |
&
\sphinxAtStartPar
名称     | |
&
\sphinxAtStartPar
描述                                        | |
\\
\sphinxhline
\sphinxAtStartPar
31:5
&\begin{itemize}
\item {} 
\end{itemize}
&\begin{itemize}
\item {} 
\end{itemize}
\\
\sphinxhline
\sphinxAtStartPar
4:0
&
\sphinxAtStartPar
RMC
&
\sphinxAtStartPar
每次接收到报文数目加1,                     |

\sphinxAtStartPar
释放接收缓冲数目减1                         |

\sphinxAtStartPar
复位模式下,寄存器清零                      |
\\
\sphinxbottomrule
\end{tabular}
\sphinxtableafterendhook\par
\sphinxattableend\end{savenotes}


\subsubsection{验收寄存器ACR}
\label{\detokenize{SWM241/_u529f_u80fd_u63cf_u8ff0/_u5c40_u57df_u7f51_u63a7_u5236_u5668:acr}}

\begin{savenotes}\sphinxattablestart
\sphinxthistablewithglobalstyle
\centering
\begin{tabular}[t]{\X{20}{100}\X{20}{100}\X{20}{100}\X{20}{100}\X{20}{100}}
\sphinxtoprule
\sphinxtableatstartofbodyhook
\sphinxAtStartPar
寄存器 |
&
\begin{DUlineblock}{0em}
\item[] 偏移 |
\end{DUlineblock}
&
\begin{DUlineblock}{0em}
\item[] 
\item[] {\color{red}\bfseries{}|}
\end{DUlineblock}
&
\sphinxAtStartPar
复位值 |    描 | |
&
\begin{DUlineblock}{0em}
\item[] |
  |
\end{DUlineblock}
\\
\sphinxhline
\sphinxAtStartPar
ACR0
&
\sphinxAtStartPar
0x300
&&
\sphinxAtStartPar
0 000000
&
\sphinxAtStartPar
验收寄存器0                |
\\
\sphinxbottomrule
\end{tabular}
\sphinxtableafterendhook\par
\sphinxattableend\end{savenotes}


\begin{savenotes}\sphinxattablestart
\sphinxthistablewithglobalstyle
\centering
\begin{tabular}[t]{\X{20}{100}\X{20}{100}\X{20}{100}\X{20}{100}\X{20}{100}}
\sphinxtoprule
\sphinxtableatstartofbodyhook
\sphinxAtStartPar
寄存器 |
&
\begin{DUlineblock}{0em}
\item[] 偏移 |
\end{DUlineblock}
&
\begin{DUlineblock}{0em}
\item[] 
\item[] {\color{red}\bfseries{}|}
\end{DUlineblock}
&
\sphinxAtStartPar
复位值 |    描 | |
&
\begin{DUlineblock}{0em}
\item[] |
  |
\end{DUlineblock}
\\
\sphinxhline
\sphinxAtStartPar
ACR1
&
\sphinxAtStartPar
0x304
&&
\sphinxAtStartPar
0 000000
&
\sphinxAtStartPar
验收寄存器1                |
\\
\sphinxbottomrule
\end{tabular}
\sphinxtableafterendhook\par
\sphinxattableend\end{savenotes}


\begin{savenotes}\sphinxattablestart
\sphinxthistablewithglobalstyle
\centering
\begin{tabular}[t]{\X{20}{100}\X{20}{100}\X{20}{100}\X{20}{100}\X{20}{100}}
\sphinxtoprule
\sphinxtableatstartofbodyhook
\sphinxAtStartPar
寄存器 |
&
\begin{DUlineblock}{0em}
\item[] 偏移 |
\end{DUlineblock}
&
\begin{DUlineblock}{0em}
\item[] 
\item[] {\color{red}\bfseries{}|}
\end{DUlineblock}
&
\sphinxAtStartPar
复位值 |    描 | |
&
\begin{DUlineblock}{0em}
\item[] |
  |
\end{DUlineblock}
\\
\sphinxhline
\sphinxAtStartPar
ACR2
&
\sphinxAtStartPar
0x308
&&
\sphinxAtStartPar
0 000000
&
\sphinxAtStartPar
验收寄存器2                |
\\
\sphinxbottomrule
\end{tabular}
\sphinxtableafterendhook\par
\sphinxattableend\end{savenotes}


\begin{savenotes}\sphinxattablestart
\sphinxthistablewithglobalstyle
\centering
\begin{tabular}[t]{\X{20}{100}\X{20}{100}\X{20}{100}\X{20}{100}\X{20}{100}}
\sphinxtoprule
\sphinxtableatstartofbodyhook
\sphinxAtStartPar
寄存器 |
&
\begin{DUlineblock}{0em}
\item[] 偏移 |
\end{DUlineblock}
&
\begin{DUlineblock}{0em}
\item[] 
\item[] {\color{red}\bfseries{}|}
\end{DUlineblock}
&
\sphinxAtStartPar
复位值 |    描 | |
&
\begin{DUlineblock}{0em}
\item[] |
  |
\end{DUlineblock}
\\
\sphinxhline
\sphinxAtStartPar
ACR3
&
\sphinxAtStartPar
0x30C
&&
\sphinxAtStartPar
0 000000
&
\sphinxAtStartPar
验收寄存器3                |
\\
\sphinxbottomrule
\end{tabular}
\sphinxtableafterendhook\par
\sphinxattableend\end{savenotes}


\begin{savenotes}\sphinxattablestart
\sphinxthistablewithglobalstyle
\centering
\begin{tabular}[t]{\X{20}{100}\X{20}{100}\X{20}{100}\X{20}{100}\X{20}{100}}
\sphinxtoprule
\sphinxtableatstartofbodyhook
\sphinxAtStartPar
寄存器 |
&
\begin{DUlineblock}{0em}
\item[] 偏移 |
\end{DUlineblock}
&
\begin{DUlineblock}{0em}
\item[] 
\item[] {\color{red}\bfseries{}|}
\end{DUlineblock}
&
\sphinxAtStartPar
复位值 |    描 | |
&
\begin{DUlineblock}{0em}
\item[] |
  |
\end{DUlineblock}
\\
\sphinxhline
\sphinxAtStartPar
ACR4
&
\sphinxAtStartPar
0x310
&&
\sphinxAtStartPar
0 000000
&
\sphinxAtStartPar
验收寄存器4                |
\\
\sphinxbottomrule
\end{tabular}
\sphinxtableafterendhook\par
\sphinxattableend\end{savenotes}


\begin{savenotes}\sphinxattablestart
\sphinxthistablewithglobalstyle
\centering
\begin{tabular}[t]{\X{20}{100}\X{20}{100}\X{20}{100}\X{20}{100}\X{20}{100}}
\sphinxtoprule
\sphinxtableatstartofbodyhook
\sphinxAtStartPar
寄存器 |
&
\begin{DUlineblock}{0em}
\item[] 偏移 |
\end{DUlineblock}
&
\begin{DUlineblock}{0em}
\item[] 
\item[] {\color{red}\bfseries{}|}
\end{DUlineblock}
&
\sphinxAtStartPar
复位值 |    描 | |
&
\begin{DUlineblock}{0em}
\item[] |
  |
\end{DUlineblock}
\\
\sphinxhline
\sphinxAtStartPar
ACR5
&
\sphinxAtStartPar
0x314
&&
\sphinxAtStartPar
0 000000
&
\sphinxAtStartPar
验收寄存器5                |
\\
\sphinxbottomrule
\end{tabular}
\sphinxtableafterendhook\par
\sphinxattableend\end{savenotes}


\begin{savenotes}\sphinxattablestart
\sphinxthistablewithglobalstyle
\centering
\begin{tabular}[t]{\X{20}{100}\X{20}{100}\X{20}{100}\X{20}{100}\X{20}{100}}
\sphinxtoprule
\sphinxtableatstartofbodyhook
\sphinxAtStartPar
寄存器 |
&
\begin{DUlineblock}{0em}
\item[] 偏移 |
\end{DUlineblock}
&
\begin{DUlineblock}{0em}
\item[] 
\item[] {\color{red}\bfseries{}|}
\end{DUlineblock}
&
\sphinxAtStartPar
复位值 |    描 | |
&
\begin{DUlineblock}{0em}
\item[] |
  |
\end{DUlineblock}
\\
\sphinxhline
\sphinxAtStartPar
ACR6
&
\sphinxAtStartPar
0x318
&&
\sphinxAtStartPar
0 000000
&
\sphinxAtStartPar
验收寄存器6                |
\\
\sphinxbottomrule
\end{tabular}
\sphinxtableafterendhook\par
\sphinxattableend\end{savenotes}


\begin{savenotes}\sphinxattablestart
\sphinxthistablewithglobalstyle
\centering
\begin{tabular}[t]{\X{20}{100}\X{20}{100}\X{20}{100}\X{20}{100}\X{20}{100}}
\sphinxtoprule
\sphinxtableatstartofbodyhook
\sphinxAtStartPar
寄存器 |
&
\begin{DUlineblock}{0em}
\item[] 偏移 |
\end{DUlineblock}
&
\begin{DUlineblock}{0em}
\item[] 
\item[] {\color{red}\bfseries{}|}
\end{DUlineblock}
&
\sphinxAtStartPar
复位值 |    描 | |
&
\begin{DUlineblock}{0em}
\item[] |
  |
\end{DUlineblock}
\\
\sphinxhline
\sphinxAtStartPar
ACR7
&
\sphinxAtStartPar
0x31C
&&
\sphinxAtStartPar
0 000000
&
\sphinxAtStartPar
验收寄存器7                |
\\
\sphinxbottomrule
\end{tabular}
\sphinxtableafterendhook\par
\sphinxattableend\end{savenotes}


\begin{savenotes}\sphinxattablestart
\sphinxthistablewithglobalstyle
\centering
\begin{tabular}[t]{\X{20}{100}\X{20}{100}\X{20}{100}\X{20}{100}\X{20}{100}}
\sphinxtoprule
\sphinxtableatstartofbodyhook
\sphinxAtStartPar
寄存器 |
&
\begin{DUlineblock}{0em}
\item[] 偏移 |
\end{DUlineblock}
&
\begin{DUlineblock}{0em}
\item[] 
\item[] {\color{red}\bfseries{}|}
\end{DUlineblock}
&
\sphinxAtStartPar
复位值 |    描 | |
&
\begin{DUlineblock}{0em}
\item[] |
  |
\end{DUlineblock}
\\
\sphinxhline
\sphinxAtStartPar
ACR8
&
\sphinxAtStartPar
0x320
&&
\sphinxAtStartPar
0 000000
&
\sphinxAtStartPar
验收寄存器8                |
\\
\sphinxbottomrule
\end{tabular}
\sphinxtableafterendhook\par
\sphinxattableend\end{savenotes}


\begin{savenotes}\sphinxattablestart
\sphinxthistablewithglobalstyle
\centering
\begin{tabular}[t]{\X{20}{100}\X{20}{100}\X{20}{100}\X{20}{100}\X{20}{100}}
\sphinxtoprule
\sphinxtableatstartofbodyhook
\sphinxAtStartPar
寄存器 |
&
\begin{DUlineblock}{0em}
\item[] 偏移 |
\end{DUlineblock}
&
\begin{DUlineblock}{0em}
\item[] 
\item[] {\color{red}\bfseries{}|}
\end{DUlineblock}
&
\sphinxAtStartPar
复位值 |    描 | |
&
\begin{DUlineblock}{0em}
\item[] |
  |
\end{DUlineblock}
\\
\sphinxhline
\sphinxAtStartPar
ACR9
&
\sphinxAtStartPar
0x324
&&
\sphinxAtStartPar
0 000000
&
\sphinxAtStartPar
验收寄存器9                |
\\
\sphinxbottomrule
\end{tabular}
\sphinxtableafterendhook\par
\sphinxattableend\end{savenotes}


\begin{savenotes}\sphinxattablestart
\sphinxthistablewithglobalstyle
\centering
\begin{tabular}[t]{\X{20}{100}\X{20}{100}\X{20}{100}\X{20}{100}\X{20}{100}}
\sphinxtoprule
\sphinxtableatstartofbodyhook
\sphinxAtStartPar
寄存器 |
&
\begin{DUlineblock}{0em}
\item[] 偏移 |
\end{DUlineblock}
&
\begin{DUlineblock}{0em}
\item[] 
\item[] {\color{red}\bfseries{}|}
\end{DUlineblock}
&
\sphinxAtStartPar
复位值 |    描 | |
&
\begin{DUlineblock}{0em}
\item[] |
  |
\end{DUlineblock}
\\
\sphinxhline
\sphinxAtStartPar
ACR10
&
\sphinxAtStartPar
0x328
&&
\sphinxAtStartPar
0 000000
&
\sphinxAtStartPar
验收寄存器10               |
\\
\sphinxbottomrule
\end{tabular}
\sphinxtableafterendhook\par
\sphinxattableend\end{savenotes}


\begin{savenotes}\sphinxattablestart
\sphinxthistablewithglobalstyle
\centering
\begin{tabular}[t]{\X{20}{100}\X{20}{100}\X{20}{100}\X{20}{100}\X{20}{100}}
\sphinxtoprule
\sphinxtableatstartofbodyhook
\sphinxAtStartPar
寄存器 |
&
\begin{DUlineblock}{0em}
\item[] 偏移 |
\end{DUlineblock}
&
\begin{DUlineblock}{0em}
\item[] 
\item[] {\color{red}\bfseries{}|}
\end{DUlineblock}
&
\sphinxAtStartPar
复位值 |    描 | |
&
\begin{DUlineblock}{0em}
\item[] |
  |
\end{DUlineblock}
\\
\sphinxhline
\sphinxAtStartPar
ACR11
&
\sphinxAtStartPar
0x32C
&&
\sphinxAtStartPar
0 000000
&
\sphinxAtStartPar
验收寄存器11               |
\\
\sphinxbottomrule
\end{tabular}
\sphinxtableafterendhook\par
\sphinxattableend\end{savenotes}


\begin{savenotes}\sphinxattablestart
\sphinxthistablewithglobalstyle
\centering
\begin{tabular}[t]{\X{20}{100}\X{20}{100}\X{20}{100}\X{20}{100}\X{20}{100}}
\sphinxtoprule
\sphinxtableatstartofbodyhook
\sphinxAtStartPar
寄存器 |
&
\begin{DUlineblock}{0em}
\item[] 偏移 |
\end{DUlineblock}
&
\begin{DUlineblock}{0em}
\item[] 
\item[] {\color{red}\bfseries{}|}
\end{DUlineblock}
&
\sphinxAtStartPar
复位值 |    描 | |
&
\begin{DUlineblock}{0em}
\item[] |
  |
\end{DUlineblock}
\\
\sphinxhline
\sphinxAtStartPar
ACR12
&
\sphinxAtStartPar
0x330
&&
\sphinxAtStartPar
0 000000
&
\sphinxAtStartPar
验收寄存器12               |
\\
\sphinxbottomrule
\end{tabular}
\sphinxtableafterendhook\par
\sphinxattableend\end{savenotes}


\begin{savenotes}\sphinxattablestart
\sphinxthistablewithglobalstyle
\centering
\begin{tabular}[t]{\X{20}{100}\X{20}{100}\X{20}{100}\X{20}{100}\X{20}{100}}
\sphinxtoprule
\sphinxtableatstartofbodyhook
\sphinxAtStartPar
寄存器 |
&
\begin{DUlineblock}{0em}
\item[] 偏移 |
\end{DUlineblock}
&
\begin{DUlineblock}{0em}
\item[] 
\item[] {\color{red}\bfseries{}|}
\end{DUlineblock}
&
\sphinxAtStartPar
复位值 |    描 | |
&
\begin{DUlineblock}{0em}
\item[] |
  |
\end{DUlineblock}
\\
\sphinxhline
\sphinxAtStartPar
ACR13
&
\sphinxAtStartPar
0x334
&&
\sphinxAtStartPar
0 000000
&
\sphinxAtStartPar
验收寄存器13               |
\\
\sphinxbottomrule
\end{tabular}
\sphinxtableafterendhook\par
\sphinxattableend\end{savenotes}


\begin{savenotes}\sphinxattablestart
\sphinxthistablewithglobalstyle
\centering
\begin{tabular}[t]{\X{20}{100}\X{20}{100}\X{20}{100}\X{20}{100}\X{20}{100}}
\sphinxtoprule
\sphinxtableatstartofbodyhook
\sphinxAtStartPar
寄存器 |
&
\begin{DUlineblock}{0em}
\item[] 偏移 |
\end{DUlineblock}
&
\begin{DUlineblock}{0em}
\item[] 
\item[] {\color{red}\bfseries{}|}
\end{DUlineblock}
&
\sphinxAtStartPar
复位值 |    描 | |
&
\begin{DUlineblock}{0em}
\item[] |
  |
\end{DUlineblock}
\\
\sphinxhline
\sphinxAtStartPar
ACR14
&
\sphinxAtStartPar
0x338
&&
\sphinxAtStartPar
0 000000
&
\sphinxAtStartPar
验收寄存器14               |
\\
\sphinxbottomrule
\end{tabular}
\sphinxtableafterendhook\par
\sphinxattableend\end{savenotes}


\begin{savenotes}\sphinxattablestart
\sphinxthistablewithglobalstyle
\centering
\begin{tabular}[t]{\X{20}{100}\X{20}{100}\X{20}{100}\X{20}{100}\X{20}{100}}
\sphinxtoprule
\sphinxtableatstartofbodyhook
\sphinxAtStartPar
寄存器 |
&
\begin{DUlineblock}{0em}
\item[] 偏移 |
\end{DUlineblock}
&
\begin{DUlineblock}{0em}
\item[] 
\item[] {\color{red}\bfseries{}|}
\end{DUlineblock}
&
\sphinxAtStartPar
复位值 |    描 | |
&
\begin{DUlineblock}{0em}
\item[] |
  |
\end{DUlineblock}
\\
\sphinxhline
\sphinxAtStartPar
ACR15
&
\sphinxAtStartPar
0x33C
&&
\sphinxAtStartPar
0 000000
&
\sphinxAtStartPar
验收寄存器15               |
\\
\sphinxbottomrule
\end{tabular}
\sphinxtableafterendhook\par
\sphinxattableend\end{savenotes}


\begin{savenotes}\sphinxattablestart
\sphinxthistablewithglobalstyle
\centering
\begin{tabular}[t]{\X{12}{96}\X{12}{96}\X{12}{96}\X{12}{96}\X{12}{96}\X{12}{96}\X{12}{96}\X{12}{96}}
\sphinxtoprule
\sphinxtableatstartofbodyhook
\sphinxAtStartPar
31
&
\sphinxAtStartPar
30
&
\sphinxAtStartPar
29
&
\sphinxAtStartPar
28
&
\sphinxAtStartPar
27
&
\sphinxAtStartPar
26
&
\sphinxAtStartPar
25
&
\sphinxAtStartPar
24
\\
\sphinxhline
\sphinxAtStartPar
ACR
&&&&&&&\\
\sphinxhline
\sphinxAtStartPar
23
&
\sphinxAtStartPar
22
&
\sphinxAtStartPar
21
&
\sphinxAtStartPar
20
&
\sphinxAtStartPar
19
&
\sphinxAtStartPar
18
&
\sphinxAtStartPar
17
&
\sphinxAtStartPar
16
\\
\sphinxhline
\sphinxAtStartPar
ACR
&&&&&&&\\
\sphinxhline
\sphinxAtStartPar
15
&
\sphinxAtStartPar
14
&
\sphinxAtStartPar
13
&
\sphinxAtStartPar
12
&
\sphinxAtStartPar
11
&
\sphinxAtStartPar
10
&
\sphinxAtStartPar
9
&
\sphinxAtStartPar
8
\\
\sphinxhline
\sphinxAtStartPar
ACR
&&&&&&&\\
\sphinxhline
\sphinxAtStartPar
7
&
\sphinxAtStartPar
6
&
\sphinxAtStartPar
5
&
\sphinxAtStartPar
4
&
\sphinxAtStartPar
3
&
\sphinxAtStartPar
2
&
\sphinxAtStartPar
1
&
\sphinxAtStartPar
0
\\
\sphinxhline
\sphinxAtStartPar
ACR
&&&&&&&\\
\sphinxbottomrule
\end{tabular}
\sphinxtableafterendhook\par
\sphinxattableend\end{savenotes}


\begin{savenotes}\sphinxattablestart
\sphinxthistablewithglobalstyle
\centering
\begin{tabular}[t]{\X{33}{99}\X{33}{99}\X{33}{99}}
\sphinxtoprule
\sphinxtableatstartofbodyhook
\sphinxAtStartPar
位域 |
&
\sphinxAtStartPar
名称     | |
&
\sphinxAtStartPar
描述                                        | |
\\
\sphinxhline
\sphinxAtStartPar
31:0
&
\sphinxAtStartPar
ACR
&
\sphinxAtStartPar
当                                          | {[}n{]} = 1时,ACR{[}n{]}和AMR{[}n{]}构成一个32位过滤器 |

\sphinxAtStartPar
当                                          | {[}n{]} = 0时,ACR{[}n{]}和AMR{[}n{]}构成两个16位过滤器 |

\sphinxAtStartPar
注 : ACR \& AMR == ID \& AMR的Message通过过滤 |
\\
\sphinxbottomrule
\end{tabular}
\sphinxtableafterendhook\par
\sphinxattableend\end{savenotes}


\subsubsection{验收屏蔽寄存器AMR}
\label{\detokenize{SWM241/_u529f_u80fd_u63cf_u8ff0/_u5c40_u57df_u7f51_u63a7_u5236_u5668:amr}}

\begin{savenotes}\sphinxattablestart
\sphinxthistablewithglobalstyle
\centering
\begin{tabular}[t]{\X{20}{100}\X{20}{100}\X{20}{100}\X{20}{100}\X{20}{100}}
\sphinxtoprule
\sphinxtableatstartofbodyhook
\sphinxAtStartPar
寄存器 |
&
\begin{DUlineblock}{0em}
\item[] 偏移 |
\end{DUlineblock}
&
\begin{DUlineblock}{0em}
\item[] 
\item[] {\color{red}\bfseries{}|}
\end{DUlineblock}
&
\sphinxAtStartPar
复位值 |    描 | |
&
\begin{DUlineblock}{0em}
\item[] |
  |
\end{DUlineblock}
\\
\sphinxhline
\sphinxAtStartPar
AMR0
&
\sphinxAtStartPar
0x380
&&
\sphinxAtStartPar
0 000000
&
\sphinxAtStartPar
验收屏蔽寄存器0            |
\\
\sphinxbottomrule
\end{tabular}
\sphinxtableafterendhook\par
\sphinxattableend\end{savenotes}


\begin{savenotes}\sphinxattablestart
\sphinxthistablewithglobalstyle
\centering
\begin{tabular}[t]{\X{20}{100}\X{20}{100}\X{20}{100}\X{20}{100}\X{20}{100}}
\sphinxtoprule
\sphinxtableatstartofbodyhook
\sphinxAtStartPar
寄存器 |
&
\begin{DUlineblock}{0em}
\item[] 偏移 |
\end{DUlineblock}
&
\begin{DUlineblock}{0em}
\item[] 
\item[] {\color{red}\bfseries{}|}
\end{DUlineblock}
&
\sphinxAtStartPar
复位值 |    描 | |
&
\begin{DUlineblock}{0em}
\item[] |
  |
\end{DUlineblock}
\\
\sphinxhline
\sphinxAtStartPar
AMR1
&
\sphinxAtStartPar
0x384
&&
\sphinxAtStartPar
0 000000
&
\sphinxAtStartPar
验收屏蔽寄存器1            |
\\
\sphinxbottomrule
\end{tabular}
\sphinxtableafterendhook\par
\sphinxattableend\end{savenotes}


\begin{savenotes}\sphinxattablestart
\sphinxthistablewithglobalstyle
\centering
\begin{tabular}[t]{\X{20}{100}\X{20}{100}\X{20}{100}\X{20}{100}\X{20}{100}}
\sphinxtoprule
\sphinxtableatstartofbodyhook
\sphinxAtStartPar
寄存器 |
&
\begin{DUlineblock}{0em}
\item[] 偏移 |
\end{DUlineblock}
&
\begin{DUlineblock}{0em}
\item[] 
\item[] {\color{red}\bfseries{}|}
\end{DUlineblock}
&
\sphinxAtStartPar
复位值 |    描 | |
&
\begin{DUlineblock}{0em}
\item[] |
  |
\end{DUlineblock}
\\
\sphinxhline
\sphinxAtStartPar
AMR2
&
\sphinxAtStartPar
0x388
&&
\sphinxAtStartPar
0 000000
&
\sphinxAtStartPar
验收屏蔽寄存器2            |
\\
\sphinxbottomrule
\end{tabular}
\sphinxtableafterendhook\par
\sphinxattableend\end{savenotes}


\begin{savenotes}\sphinxattablestart
\sphinxthistablewithglobalstyle
\centering
\begin{tabular}[t]{\X{20}{100}\X{20}{100}\X{20}{100}\X{20}{100}\X{20}{100}}
\sphinxtoprule
\sphinxtableatstartofbodyhook
\sphinxAtStartPar
寄存器 |
&
\begin{DUlineblock}{0em}
\item[] 偏移 |
\end{DUlineblock}
&
\begin{DUlineblock}{0em}
\item[] 
\item[] {\color{red}\bfseries{}|}
\end{DUlineblock}
&
\sphinxAtStartPar
复位值 |    描 | |
&
\begin{DUlineblock}{0em}
\item[] |
  |
\end{DUlineblock}
\\
\sphinxhline
\sphinxAtStartPar
AMR3
&
\sphinxAtStartPar
0x38C
&&
\sphinxAtStartPar
0 000000
&
\sphinxAtStartPar
验收屏蔽寄存器3            |
\\
\sphinxbottomrule
\end{tabular}
\sphinxtableafterendhook\par
\sphinxattableend\end{savenotes}


\begin{savenotes}\sphinxattablestart
\sphinxthistablewithglobalstyle
\centering
\begin{tabular}[t]{\X{20}{100}\X{20}{100}\X{20}{100}\X{20}{100}\X{20}{100}}
\sphinxtoprule
\sphinxtableatstartofbodyhook
\sphinxAtStartPar
寄存器 |
&
\begin{DUlineblock}{0em}
\item[] 偏移 |
\end{DUlineblock}
&
\begin{DUlineblock}{0em}
\item[] 
\item[] {\color{red}\bfseries{}|}
\end{DUlineblock}
&
\sphinxAtStartPar
复位值 |    描 | |
&
\begin{DUlineblock}{0em}
\item[] |
  |
\end{DUlineblock}
\\
\sphinxhline
\sphinxAtStartPar
AMR4
&
\sphinxAtStartPar
0x390
&&
\sphinxAtStartPar
0 000000
&
\sphinxAtStartPar
验收屏蔽寄存器4            |
\\
\sphinxbottomrule
\end{tabular}
\sphinxtableafterendhook\par
\sphinxattableend\end{savenotes}


\begin{savenotes}\sphinxattablestart
\sphinxthistablewithglobalstyle
\centering
\begin{tabular}[t]{\X{20}{100}\X{20}{100}\X{20}{100}\X{20}{100}\X{20}{100}}
\sphinxtoprule
\sphinxtableatstartofbodyhook
\sphinxAtStartPar
寄存器 |
&
\begin{DUlineblock}{0em}
\item[] 偏移 |
\end{DUlineblock}
&
\begin{DUlineblock}{0em}
\item[] 
\item[] {\color{red}\bfseries{}|}
\end{DUlineblock}
&
\sphinxAtStartPar
复位值 |    描 | |
&
\begin{DUlineblock}{0em}
\item[] |
  |
\end{DUlineblock}
\\
\sphinxhline
\sphinxAtStartPar
AMR5
&
\sphinxAtStartPar
0x394
&&
\sphinxAtStartPar
0 000000
&
\sphinxAtStartPar
验收屏蔽寄存器5            |
\\
\sphinxbottomrule
\end{tabular}
\sphinxtableafterendhook\par
\sphinxattableend\end{savenotes}


\begin{savenotes}\sphinxattablestart
\sphinxthistablewithglobalstyle
\centering
\begin{tabular}[t]{\X{20}{100}\X{20}{100}\X{20}{100}\X{20}{100}\X{20}{100}}
\sphinxtoprule
\sphinxtableatstartofbodyhook
\sphinxAtStartPar
寄存器 |
&
\begin{DUlineblock}{0em}
\item[] 偏移 |
\end{DUlineblock}
&
\begin{DUlineblock}{0em}
\item[] 
\item[] {\color{red}\bfseries{}|}
\end{DUlineblock}
&
\sphinxAtStartPar
复位值 |    描 | |
&
\begin{DUlineblock}{0em}
\item[] |
  |
\end{DUlineblock}
\\
\sphinxhline
\sphinxAtStartPar
AMR6
&
\sphinxAtStartPar
0x398
&&
\sphinxAtStartPar
0 000000
&
\sphinxAtStartPar
验收屏蔽寄存器6            |
\\
\sphinxbottomrule
\end{tabular}
\sphinxtableafterendhook\par
\sphinxattableend\end{savenotes}


\begin{savenotes}\sphinxattablestart
\sphinxthistablewithglobalstyle
\centering
\begin{tabular}[t]{\X{20}{100}\X{20}{100}\X{20}{100}\X{20}{100}\X{20}{100}}
\sphinxtoprule
\sphinxtableatstartofbodyhook
\sphinxAtStartPar
寄存器 |
&
\begin{DUlineblock}{0em}
\item[] 偏移 |
\end{DUlineblock}
&
\begin{DUlineblock}{0em}
\item[] 
\item[] {\color{red}\bfseries{}|}
\end{DUlineblock}
&
\sphinxAtStartPar
复位值 |    描 | |
&
\begin{DUlineblock}{0em}
\item[] |
  |
\end{DUlineblock}
\\
\sphinxhline
\sphinxAtStartPar
AMR7
&
\sphinxAtStartPar
0x39C
&&
\sphinxAtStartPar
0 000000
&
\sphinxAtStartPar
验收屏蔽寄存器7            |
\\
\sphinxbottomrule
\end{tabular}
\sphinxtableafterendhook\par
\sphinxattableend\end{savenotes}


\begin{savenotes}\sphinxattablestart
\sphinxthistablewithglobalstyle
\centering
\begin{tabular}[t]{\X{20}{100}\X{20}{100}\X{20}{100}\X{20}{100}\X{20}{100}}
\sphinxtoprule
\sphinxtableatstartofbodyhook
\sphinxAtStartPar
寄存器 |
&
\begin{DUlineblock}{0em}
\item[] 偏移 |
\end{DUlineblock}
&
\begin{DUlineblock}{0em}
\item[] 
\item[] {\color{red}\bfseries{}|}
\end{DUlineblock}
&
\sphinxAtStartPar
复位值 |    描 | |
&
\begin{DUlineblock}{0em}
\item[] |
  |
\end{DUlineblock}
\\
\sphinxhline
\sphinxAtStartPar
AMR8
&
\sphinxAtStartPar
0x3A0
&&
\sphinxAtStartPar
0 000000
&
\sphinxAtStartPar
验收屏蔽寄存器8            |
\\
\sphinxbottomrule
\end{tabular}
\sphinxtableafterendhook\par
\sphinxattableend\end{savenotes}


\begin{savenotes}\sphinxattablestart
\sphinxthistablewithglobalstyle
\centering
\begin{tabular}[t]{\X{20}{100}\X{20}{100}\X{20}{100}\X{20}{100}\X{20}{100}}
\sphinxtoprule
\sphinxtableatstartofbodyhook
\sphinxAtStartPar
寄存器 |
&
\begin{DUlineblock}{0em}
\item[] 偏移 |
\end{DUlineblock}
&
\begin{DUlineblock}{0em}
\item[] 
\item[] {\color{red}\bfseries{}|}
\end{DUlineblock}
&
\sphinxAtStartPar
复位值 |    描 | |
&
\begin{DUlineblock}{0em}
\item[] |
  |
\end{DUlineblock}
\\
\sphinxhline
\sphinxAtStartPar
AMR9
&
\sphinxAtStartPar
0x3A4
&&
\sphinxAtStartPar
0 000000
&
\sphinxAtStartPar
验收屏蔽寄存器9            |
\\
\sphinxbottomrule
\end{tabular}
\sphinxtableafterendhook\par
\sphinxattableend\end{savenotes}


\begin{savenotes}\sphinxattablestart
\sphinxthistablewithglobalstyle
\centering
\begin{tabular}[t]{\X{20}{100}\X{20}{100}\X{20}{100}\X{20}{100}\X{20}{100}}
\sphinxtoprule
\sphinxtableatstartofbodyhook
\sphinxAtStartPar
寄存器 |
&
\begin{DUlineblock}{0em}
\item[] 偏移 |
\end{DUlineblock}
&
\begin{DUlineblock}{0em}
\item[] 
\item[] {\color{red}\bfseries{}|}
\end{DUlineblock}
&
\sphinxAtStartPar
复位值 |    描 | |
&
\begin{DUlineblock}{0em}
\item[] |
  |
\end{DUlineblock}
\\
\sphinxhline
\sphinxAtStartPar
AMR10
&
\sphinxAtStartPar
0x3A8
&&
\sphinxAtStartPar
0 000000
&
\sphinxAtStartPar
验收屏蔽寄存器10           |
\\
\sphinxbottomrule
\end{tabular}
\sphinxtableafterendhook\par
\sphinxattableend\end{savenotes}


\begin{savenotes}\sphinxattablestart
\sphinxthistablewithglobalstyle
\centering
\begin{tabular}[t]{\X{20}{100}\X{20}{100}\X{20}{100}\X{20}{100}\X{20}{100}}
\sphinxtoprule
\sphinxtableatstartofbodyhook
\sphinxAtStartPar
寄存器 |
&
\begin{DUlineblock}{0em}
\item[] 偏移 |
\end{DUlineblock}
&
\begin{DUlineblock}{0em}
\item[] 
\item[] {\color{red}\bfseries{}|}
\end{DUlineblock}
&
\sphinxAtStartPar
复位值 |    描 | |
&
\begin{DUlineblock}{0em}
\item[] |
  |
\end{DUlineblock}
\\
\sphinxhline
\sphinxAtStartPar
AMR11
&
\sphinxAtStartPar
0x3AC
&&
\sphinxAtStartPar
0 000000
&
\sphinxAtStartPar
验收屏蔽寄存器11           |
\\
\sphinxbottomrule
\end{tabular}
\sphinxtableafterendhook\par
\sphinxattableend\end{savenotes}


\begin{savenotes}\sphinxattablestart
\sphinxthistablewithglobalstyle
\centering
\begin{tabular}[t]{\X{20}{100}\X{20}{100}\X{20}{100}\X{20}{100}\X{20}{100}}
\sphinxtoprule
\sphinxtableatstartofbodyhook
\sphinxAtStartPar
寄存器 |
&
\begin{DUlineblock}{0em}
\item[] 偏移 |
\end{DUlineblock}
&
\begin{DUlineblock}{0em}
\item[] 
\item[] {\color{red}\bfseries{}|}
\end{DUlineblock}
&
\sphinxAtStartPar
复位值 |    描 | |
&
\begin{DUlineblock}{0em}
\item[] |
  |
\end{DUlineblock}
\\
\sphinxhline
\sphinxAtStartPar
AMR12
&
\sphinxAtStartPar
0x3B0
&&
\sphinxAtStartPar
0 000000
&
\sphinxAtStartPar
验收屏蔽寄存器12           |
\\
\sphinxbottomrule
\end{tabular}
\sphinxtableafterendhook\par
\sphinxattableend\end{savenotes}


\begin{savenotes}\sphinxattablestart
\sphinxthistablewithglobalstyle
\centering
\begin{tabular}[t]{\X{20}{100}\X{20}{100}\X{20}{100}\X{20}{100}\X{20}{100}}
\sphinxtoprule
\sphinxtableatstartofbodyhook
\sphinxAtStartPar
寄存器 |
&
\begin{DUlineblock}{0em}
\item[] 偏移 |
\end{DUlineblock}
&
\begin{DUlineblock}{0em}
\item[] 
\item[] {\color{red}\bfseries{}|}
\end{DUlineblock}
&
\sphinxAtStartPar
复位值 |    描 | |
&
\begin{DUlineblock}{0em}
\item[] |
  |
\end{DUlineblock}
\\
\sphinxhline
\sphinxAtStartPar
AMR13
&
\sphinxAtStartPar
0x3B4
&&
\sphinxAtStartPar
0 000000
&
\sphinxAtStartPar
验收屏蔽寄存器13           |
\\
\sphinxbottomrule
\end{tabular}
\sphinxtableafterendhook\par
\sphinxattableend\end{savenotes}


\begin{savenotes}\sphinxattablestart
\sphinxthistablewithglobalstyle
\centering
\begin{tabular}[t]{\X{20}{100}\X{20}{100}\X{20}{100}\X{20}{100}\X{20}{100}}
\sphinxtoprule
\sphinxtableatstartofbodyhook
\sphinxAtStartPar
寄存器 |
&
\begin{DUlineblock}{0em}
\item[] 偏移 |
\end{DUlineblock}
&
\begin{DUlineblock}{0em}
\item[] 
\item[] {\color{red}\bfseries{}|}
\end{DUlineblock}
&
\sphinxAtStartPar
复位值 |    描 | |
&
\begin{DUlineblock}{0em}
\item[] |
  |
\end{DUlineblock}
\\
\sphinxhline
\sphinxAtStartPar
AMR14
&
\sphinxAtStartPar
0x3B8
&&
\sphinxAtStartPar
0 000000
&
\sphinxAtStartPar
验收屏蔽寄存器14           |
\\
\sphinxbottomrule
\end{tabular}
\sphinxtableafterendhook\par
\sphinxattableend\end{savenotes}


\begin{savenotes}\sphinxattablestart
\sphinxthistablewithglobalstyle
\centering
\begin{tabular}[t]{\X{20}{100}\X{20}{100}\X{20}{100}\X{20}{100}\X{20}{100}}
\sphinxtoprule
\sphinxtableatstartofbodyhook
\sphinxAtStartPar
寄存器 |
&
\begin{DUlineblock}{0em}
\item[] 偏移 |
\end{DUlineblock}
&
\begin{DUlineblock}{0em}
\item[] 
\item[] {\color{red}\bfseries{}|}
\end{DUlineblock}
&
\sphinxAtStartPar
复位值 |    描 | |
&
\begin{DUlineblock}{0em}
\item[] |
  |
\end{DUlineblock}
\\
\sphinxhline
\sphinxAtStartPar
AMR15
&
\sphinxAtStartPar
0x3BC
&&
\sphinxAtStartPar
0 000000
&
\sphinxAtStartPar
验收屏蔽寄存器15           |
\\
\sphinxbottomrule
\end{tabular}
\sphinxtableafterendhook\par
\sphinxattableend\end{savenotes}


\begin{savenotes}\sphinxattablestart
\sphinxthistablewithglobalstyle
\centering
\begin{tabular}[t]{\X{12}{96}\X{12}{96}\X{12}{96}\X{12}{96}\X{12}{96}\X{12}{96}\X{12}{96}\X{12}{96}}
\sphinxtoprule
\sphinxtableatstartofbodyhook
\sphinxAtStartPar
31
&
\sphinxAtStartPar
30
&
\sphinxAtStartPar
29
&
\sphinxAtStartPar
28
&
\sphinxAtStartPar
27
&
\sphinxAtStartPar
26
&
\sphinxAtStartPar
25
&
\sphinxAtStartPar
24
\\
\sphinxhline
\sphinxAtStartPar
AMR
&&&&&&&\\
\sphinxhline
\sphinxAtStartPar
23
&
\sphinxAtStartPar
22
&
\sphinxAtStartPar
21
&
\sphinxAtStartPar
20
&
\sphinxAtStartPar
19
&
\sphinxAtStartPar
18
&
\sphinxAtStartPar
17
&
\sphinxAtStartPar
16
\\
\sphinxhline
\sphinxAtStartPar
AMR
&&&&&&&\\
\sphinxhline
\sphinxAtStartPar
15
&
\sphinxAtStartPar
14
&
\sphinxAtStartPar
13
&
\sphinxAtStartPar
12
&
\sphinxAtStartPar
11
&
\sphinxAtStartPar
10
&
\sphinxAtStartPar
9
&
\sphinxAtStartPar
8
\\
\sphinxhline
\sphinxAtStartPar
AMR
&&&&&&&\\
\sphinxhline
\sphinxAtStartPar
7
&
\sphinxAtStartPar
6
&
\sphinxAtStartPar
5
&
\sphinxAtStartPar
4
&
\sphinxAtStartPar
3
&
\sphinxAtStartPar
2
&
\sphinxAtStartPar
1
&
\sphinxAtStartPar
0
\\
\sphinxhline
\sphinxAtStartPar
AMR
&&&&&&&\\
\sphinxbottomrule
\end{tabular}
\sphinxtableafterendhook\par
\sphinxattableend\end{savenotes}


\begin{savenotes}\sphinxattablestart
\sphinxthistablewithglobalstyle
\centering
\begin{tabular}[t]{\X{33}{99}\X{33}{99}\X{33}{99}}
\sphinxtoprule
\sphinxtableatstartofbodyhook
\sphinxAtStartPar
位域 |
&
\sphinxAtStartPar
名称     | |
&
\sphinxAtStartPar
描述                                        | |
\\
\sphinxhline
\sphinxAtStartPar
31:0
&
\sphinxAtStartPar
AMR
&
\sphinxAtStartPar
当                                          | {[}n{]} = 1时,ACR{[}n{]}和AMR{[}n{]}构成一个32位过滤器 |

\sphinxAtStartPar
当                                          | {[}n{]} = 0时,ACR{[}n{]}和AMR{[}n{]}构成两个16位过滤器 |

\sphinxAtStartPar
注 : ACR \& AMR == ID \& AMR的Message通过过滤 |
\\
\sphinxbottomrule
\end{tabular}
\sphinxtableafterendhook\par
\sphinxattableend\end{savenotes}

\sphinxstepscope


\section{脉冲宽度调制(PWM)发生器}
\label{\detokenize{SWM241/_u529f_u80fd_u63cf_u8ff0/_u8109_u51b2_u5bbd_u5ea6_u8c03_u5236:pwm}}\label{\detokenize{SWM241/_u529f_u80fd_u63cf_u8ff0/_u8109_u51b2_u5bbd_u5ea6_u8c03_u5236::doc}}
\sphinxAtStartPar
概述
\textasciitilde{}\textasciitilde{}

\sphinxAtStartPar
SWM241系列所有型号PWM操作均相同,不同型号PWM通道数可能不同。使用前需使能PWM模块时钟。

\sphinxAtStartPar
PWM模块提供了4组(PWM0、PMW1、PWM2、PWM3)、8路(PWM0A、PWM0B、PWM1A、PWM1B、PWM2A、PWM2B、PWM3A、PWM3B)独立通道,支持边沿模式、中心对称模式。

\sphinxAtStartPar
中心对称模式下,输出是互补输出。如PWM0A驱动PWM0A和PWM0AN两个输出信号,两个信号周期相等、电平值相反,且可设置死区。

\sphinxAtStartPar
特性
\textasciitilde{}\textasciitilde{}
\begin{itemize}
\item {} 
\sphinxAtStartPar
4组16位宽PWM控制,最多可产生8路PWM信号

\item {} 
\sphinxAtStartPar
支持边沿模式、中心对称模式

\item {} 
\sphinxAtStartPar
每个PWM的输出可以独立选择反相输出

\item {} 
\sphinxAtStartPar
提供新周期开始中断,高电平结束中断、刹车中断以及中心对称模式下的半周期中断

\item {} 
\sphinxAtStartPar
支持硬件死区设置

\item {} 
\sphinxAtStartPar
可选择初始输出电平选择

\item {} 
\sphinxAtStartPar
PWM输出的固定值可配

\item {} 
\sphinxAtStartPar
PWM空闲状态下的输出可配

\item {} 
\sphinxAtStartPar
支持刹车功能

\item {} 
\sphinxAtStartPar
支持硬件自动触发ADC采样

\item {} 
\sphinxAtStartPar
每路预分频可独立配置

\end{itemize}


\subsection{模块结构框图}
\label{\detokenize{SWM241/_u529f_u80fd_u63cf_u8ff0/_u8109_u51b2_u5bbd_u5ea6_u8c03_u5236:id1}}
\sphinxAtStartPar
\sphinxincludegraphics{{SWM241/功能描述/media脉冲宽度调制002}.emf}

\sphinxAtStartPar
图 6‑42 PWM模块结构框图


\subsection{功能描述}
\label{\detokenize{SWM241/_u529f_u80fd_u63cf_u8ff0/_u8109_u51b2_u5bbd_u5ea6_u8c03_u5236:id2}}

\subsubsection{时钟分频}
\label{\detokenize{SWM241/_u529f_u80fd_u63cf_u8ff0/_u8109_u51b2_u5bbd_u5ea6_u8c03_u5236:id3}}
\sphinxAtStartPar
通过CLKDIV寄存器,可进行PWM计数时钟周期配置,支持计数周期最多为PWM模块时钟周期的256倍。

\sphinxAtStartPar
注意:分频寄存器需要在初始电平设置完成后进行配置。

\sphinxAtStartPar
PWM模块建议按照如下顺序配置
\begin{itemize}
\item {} 
\sphinxAtStartPar
配置初始电平

\item {} 
\sphinxAtStartPar
配置周期及中断相关寄存器

\item {} 
\sphinxAtStartPar
初始化时钟分频

\item {} 
\sphinxAtStartPar
引脚功能切换

\item {} 
\sphinxAtStartPar
PWM使能

\end{itemize}


\subsubsection{死区保护}
\label{\detokenize{SWM241/_u529f_u80fd_u63cf_u8ff0/_u8109_u51b2_u5bbd_u5ea6_u8c03_u5236:id4}}
\sphinxAtStartPar
PWM输出时Dead Zone(死区)的作用是在电平翻转时插入一个时间间隔,避免关闭前一个设备和打开后一个设备时,因为开关速度的问题,出现同时开启状态而增加负荷的情况(在没有彻底关闭前打开了后一个设备),尤其是电流过大时容易造成短路等损坏设备。

\sphinxAtStartPar
此PWM模块每一路PWM的死区都独立配置,在独立模式和中心对称模式下均可配。效果为将上升沿推后指定周期,但其配置值必须小于高电平持续时长的配置值,且具有死区保护,即当高电平周期设置为全0或者等于周期数时,死区设置失效。

\sphinxAtStartPar
死区示意图如图 6‑43所示:

\sphinxAtStartPar
\sphinxincludegraphics{{SWM241/功能描述/media脉冲宽度调制003}.emf}

\sphinxAtStartPar
图 6‑43 PWM死区示意图


\subsubsection{模式选择}
\label{\detokenize{SWM241/_u529f_u80fd_u63cf_u8ff0/_u8109_u51b2_u5bbd_u5ea6_u8c03_u5236:id5}}
\sphinxAtStartPar
通过配置MODEx寄存器,配置PWM输出模式,包括边沿、中心对称模式。

\sphinxAtStartPar
半周期中断只存在于中心对称模式下。

\sphinxAtStartPar
\sphinxstylestrong{边沿模式}

\sphinxAtStartPar
边沿模式下,每一路PWM输出独立配置,彼此间相互无影响,PWM计数周期、高电平持续时长、死区时长以及起始输出电平可独立配置。

\sphinxAtStartPar
支持的中断类型为:
\begin{itemize}
\item {} 
\sphinxAtStartPar
新周期开始中断

\item {} 
\sphinxAtStartPar
高电平结束中断

\item {} 
\sphinxAtStartPar
刹车中断

\end{itemize}

\sphinxAtStartPar
示意图如图 6‑44所示:

\sphinxAtStartPar
\sphinxincludegraphics{{SWM241/功能描述/media脉冲宽度调制004}.emf}

\sphinxAtStartPar
图 6‑44边沿模式

\sphinxAtStartPar
\sphinxincludegraphics{{SWM241/功能描述/media脉冲宽度调制005}.emf}

\sphinxAtStartPar
图 6‑45 边沿模式带死区

\sphinxAtStartPar
\sphinxstylestrong{中心对称模式}

\sphinxAtStartPar
PWMxA和PWMxB两路输出互相独立,且各自都有其反相输出引脚。与边沿模式相比,中心对称模式精度加倍、频率减半。

\sphinxAtStartPar
此模式下支持的中断类型为:
\begin{itemize}
\item {} 
\sphinxAtStartPar
新周期开始中断

\item {} 
\sphinxAtStartPar
高电平结束中断

\item {} 
\sphinxAtStartPar
半周期中断

\item {} 
\sphinxAtStartPar
刹车中断

\end{itemize}

\sphinxAtStartPar
在此模式下,PWM计数周期、高电平持续时长、死区时长以及起始输出电平可独立配置,具有半周期中断功能(即中心点中断),且该功能仅存在于此模式下,其他模式不可用。每一路均可单独进行使能、屏蔽及查询操作。详情请参考寄存器HCIE、HCIF、HCIM、HCIRS寄存器。

\sphinxAtStartPar
示意图如图 6‑46所示:

\sphinxAtStartPar
\sphinxincludegraphics{{SWM241/功能描述/media脉冲宽度调制006}.emf}

\sphinxAtStartPar
图 6‑46中心对称模式带死区

\sphinxAtStartPar
在此模式下,在半周期和整周期处都会更新高电平周期、计数周期、以及ADC触发阈值。

\sphinxAtStartPar
周期更新示意图如图 6‑47、图 6‑48所示:

\sphinxAtStartPar
\sphinxincludegraphics{{SWM241/功能描述/media脉冲宽度调制007}.emf}

\sphinxAtStartPar
图 6‑47 中心对称模式周期结束更新示意图

\sphinxAtStartPar
\sphinxincludegraphics{{SWM241/功能描述/media脉冲宽度调制008}.emf}

\sphinxAtStartPar
图 6‑48 中心对称模式半周期更新示意图


\subsubsection{触发SAR ADC采样}
\label{\detokenize{SWM241/_u529f_u80fd_u63cf_u8ff0/_u8109_u51b2_u5bbd_u5ea6_u8c03_u5236:sar-adc}}
\sphinxAtStartPar
PWM在任意模式下可以触发ADC,每一路输出独立的ADC触发信号,且每个周期可以设置两个ADC触发点。

\sphinxAtStartPar
将SAR ADC配置寄存器(CTRL)中TRIG方式设置为PWM触发。每路PWM对应2个ADTRG值,当PWM计数到指定值,可触发ADC进行采样。

\sphinxAtStartPar
具体配置方式如下:
\begin{itemize}
\item {} 
\sphinxAtStartPar
配置PWMx路触发ADC控制寄存器,设置触发点是否有效以及PWM触发ADC时间点。在中心对称模式下,前半周期为设置值,后半周期为设置值的对称点,可通过ADTRGDIRx寄存器分别设置计数器递增/递计数时ADTRGXn是否有效。

\item {} 
\sphinxAtStartPar
配置ADC的触发方式为PWM触发

\item {} 
\sphinxAtStartPar
使能PWM模块EN位,当计数值到达MATCH设置值时,触发ADC配置寄存器(CTRL)中选中的通道(CHx)进行采样,采样完成后,将产生EOC标志位,并产生ADC中断

\end{itemize}

\sphinxAtStartPar
示意图如图 6‑49所示:

\sphinxAtStartPar
\sphinxincludegraphics{{SWM241/功能描述/media脉冲宽度调制009}.emf}

\sphinxAtStartPar
图 6‑49 PWM触发ADC采样示意图


\subsubsection{中断配置与清除}
\label{\detokenize{SWM241/_u529f_u80fd_u63cf_u8ff0/_u8109_u51b2_u5bbd_u5ea6_u8c03_u5236:id6}}
\sphinxAtStartPar
PWM模块提供了高电平结束中断、新周期起始中断及半周期中断,每一路均可单独进行使能、屏蔽及查询操作。每组PWM(2路)共享一根中断线。通过IE寄存器、IF寄存器、IM寄存器、IRS寄存器进行操作。IRS寄存器只受EN寄存器影响,当IM寄存器使能后,IF寄存器对应位将被屏蔽。不同模式下工作方式如下:
\begin{itemize}
\item {} 
\sphinxAtStartPar
所有模式下,均通过对IF寄存器写1清除对应中断

\item {} 
\sphinxAtStartPar
对于边沿模式,每路均可单独产生中断
\begin{itemize}
\item {} 
\sphinxAtStartPar
新周期开始中断

\item {} 
\sphinxAtStartPar
高电平结束中断

\item {} 
\sphinxAtStartPar
刹车中断

\end{itemize}

\item {} 
\sphinxAtStartPar
对于中心对称模式,可发生半周期中断
\begin{itemize}
\item {} 
\sphinxAtStartPar
新周期开始中断

\item {} 
\sphinxAtStartPar
高电平结束中断

\item {} 
\sphinxAtStartPar
刹车中断

\item {} 
\sphinxAtStartPar
半周期中断

\end{itemize}

\end{itemize}

\sphinxAtStartPar
可通过配置刹车中断使能寄存器BRKIE、中断使能寄存器IE、半周期中断使能寄存器HCIE等相关寄存器中相应位使能中断。当中断触发后,中断标志寄存器BRKIF、NCIF、HEIF、HCIF对应位置1。如需清除此标志,需在对应标志位中写1清零(R/W1C),否则中断在开启状态下会一直进入。

\sphinxAtStartPar
此模块还具备中断屏蔽功能。当配置了中断屏蔽寄存器IM时,即使数据传输结束,也不会产生中断。


\subsubsection{刹车与暂停功能}
\label{\detokenize{SWM241/_u529f_u80fd_u63cf_u8ff0/_u8109_u51b2_u5bbd_u5ea6_u8c03_u5236:id7}}
\sphinxAtStartPar
PWM模块支持内部软件操作对输出进行暂停(即强制输出高/低电平,但PWM模块计数依然继续进行),外部信号输入对输出进行刹车。

\sphinxAtStartPar
\sphinxstylestrong{暂停功能}

\sphinxAtStartPar
PWM0\textasciitilde{}3都支持内部软件操作对输出进行暂停,通过操作FORCEO寄存器对应位配置。使对应PWM输出为固定电平,并可配置相应位输出为高电平或低电平。此时,PWM模块计数依然继续进行。当配置FORCEO寄存器中对应位,使对应PWM为正常输出时,PWM可继续输出。配置详情请见寄存器。

\sphinxAtStartPar
当PWM输出暂停时,PWM模块计数依然继续进行。

\sphinxAtStartPar
波形如图 6‑50所示:
\begin{quote}

\sphinxAtStartPar
\sphinxincludegraphics{{SWM241/功能描述/media脉冲宽度调制010}.emf}
\end{quote}

\sphinxAtStartPar
图 6‑50输出屏蔽功能示意图

\sphinxAtStartPar
\sphinxstylestrong{刹车功能}

\sphinxAtStartPar
PWM0\textasciitilde{}2支持外部信号输入对输出进行暂停(刹车),输入电平可通过配置BRKCR配置,刹车时输出电平可通过BRKENx寄存器进行配置。

\sphinxAtStartPar
通过配置BRKCR寄存器选择刹车信号生效后是否停止计数。

\sphinxAtStartPar
外部信号可通过PWM\_BRAKE引脚输入指定电平对PWM模块进行刹车操作,使用前配置如下:
\begin{itemize}
\item {} 
\sphinxAtStartPar
配置PORTCON模块中INEN寄存器使能引脚输入功能

\item {} 
\sphinxAtStartPar
通过PORT\_FUNC寄存器将引脚配置为PWM\_BRAKE功能

\item {} 
\sphinxAtStartPar
对BRK相关寄存器进行设置,配置刹车输入有效电平、刹车过程中输出电平、刹车后PWM是否继续计数、该功能影响的通道

\item {} 
\sphinxAtStartPar
配置使能寄存器。使能后,当外部输入指定电平时,对应通道执行刹车功能

\item {} 
\sphinxAtStartPar
刹车电平恢复后,将根据BRK寄存器配置决定PWM波形是否继续输出

\end{itemize}

\sphinxAtStartPar
最多支持2路刹车引脚,每个引脚均可配置使能及对应PWM通道。

\sphinxAtStartPar
刹车结构图如图 6‑51所示:刹车示意图如图 6‑52所示:

\sphinxAtStartPar
\sphinxincludegraphics{{SWM241/功能描述/media脉冲宽度调制011}.emf}

\sphinxAtStartPar
图 6‑51 刹车结构图

\sphinxAtStartPar
\sphinxincludegraphics{{SWM241/功能描述/media脉冲宽度调制012}.emf}

\sphinxAtStartPar
图 6‑52 刹车示意图


\subsubsection{电平翻转}
\label{\detokenize{SWM241/_u529f_u80fd_u63cf_u8ff0/_u8109_u51b2_u5bbd_u5ea6_u8c03_u5236:id8}}
\sphinxAtStartPar
PWM模块支持电平翻转,可通过配置OUTCRx寄存器中INVA和INVB位,分别对应A通道和B通道。

\sphinxAtStartPar
如图 6‑53所示:

\sphinxAtStartPar
\sphinxincludegraphics{{SWM241/功能描述/media脉冲宽度调制013}.emf}

\sphinxAtStartPar
图 6‑53 电平翻转示意图


\subsection{寄存器映射}
\label{\detokenize{SWM241/_u529f_u80fd_u63cf_u8ff0/_u8109_u51b2_u5bbd_u5ea6_u8c03_u5236:id9}}

\begin{savenotes}
\sphinxatlongtablestart
\sphinxthistablewithglobalstyle
\makeatletter
  \LTleft \@totalleftmargin plus1fill
  \LTright\dimexpr\columnwidth-\@totalleftmargin-\linewidth\relax plus1fill
\makeatother
\begin{longtable}{\X{20}{100}\X{20}{100}\X{20}{100}\X{20}{100}\X{20}{100}}
\sphinxtoprule
\endfirsthead

\multicolumn{5}{c}{\sphinxnorowcolor
    \makebox[0pt]{\sphinxtablecontinued{\tablename\ \thetable{} \textendash{} continued from previous page}}%
}\\
\sphinxtoprule
\endhead

\sphinxbottomrule
\multicolumn{5}{r}{\sphinxnorowcolor
    \makebox[0pt][r]{\sphinxtablecontinued{continues on next page}}%
}\\
\endfoot

\endlastfoot
\sphinxtableatstartofbodyhook

\sphinxAtStartPar
名称   |
&
\begin{DUlineblock}{0em}
\item[] 偏移 |
\end{DUlineblock}
&
\begin{DUlineblock}{0em}
\item[] 
\item[] |
|
\end{DUlineblock}
&
\begin{DUlineblock}{0em}
\item[] 
\end{DUlineblock}
\begin{quote}

\begin{DUlineblock}{0em}
\item[] 
\item[] 
\end{DUlineblock}
\end{quote}
&
\sphinxAtStartPar
描述                       | | | |
\\
\sphinxhline
\sphinxAtStartPar
PWM0BASE:0 {\color{red}\bfseries{}|}x40046000PWM1BASE:0 {\color{red}\bfseries{}|}x40046040PWM2BASE:0 {\color{red}\bfseries{}|}x40046080PWM3BASE:0 {\color{red}\bfseries{}|}x400460C0
&
\begin{DUlineblock}{0em}
\item[] |     |     |
\end{DUlineblock}
&&&\\
\sphinxhline
\sphinxAtStartPar
MODEx
&
\sphinxAtStartPar
0x00
&&
\sphinxAtStartPar
0x 00000
&
\sphinxAtStartPar
第x组PWM的工作模式控制     |
\\
\sphinxhline
\sphinxAtStartPar
PERAx
&
\sphinxAtStartPar
0x04
&&
\sphinxAtStartPar
0x 00000
&
\sphinxAtStartPar
第x组A路PWM的计数周期,    | 1,对应一个计数时钟周期 |
\\
\sphinxhline
\sphinxAtStartPar
HIGHAx
&
\sphinxAtStartPar
0x08
&&
\sphinxAtStartPar
0x 00000
&\\
\sphinxhline
\sphinxAtStartPar
DZAx
&
\sphinxAtStartPar
0x0C
&&
\sphinxAtStartPar
0x 00000
&
\sphinxAtStartPar
第x组A路                   | 度控制。必须小于HIGHA0  |
\\
\sphinxhline
\sphinxAtStartPar
PERBx
&
\sphinxAtStartPar
0x10
&&
\sphinxAtStartPar
0x 00000
&
\sphinxAtStartPar
第x组B路PWM的计数周期,    | 1,对应一个计数时钟周期 |
\\
\sphinxhline
\sphinxAtStartPar
HIGHBx
&
\sphinxAtStartPar
0x14
&&
\sphinxAtStartPar
0x 00000
&\\
\sphinxhline
\sphinxAtStartPar
DZBx
&
\sphinxAtStartPar
0x18
&&
\sphinxAtStartPar
0x 00000
&
\sphinxAtStartPar
第x组B路死区长度控制。     |
\\
\sphinxhline
\sphinxAtStartPar
OUTCRx
&
\sphinxAtStartPar
0x1C
&&
\sphinxAtStartPar
0x 00000
&
\sphinxAtStartPar
第x组PWM输出起始值控制     |
\\
\sphinxhline
\sphinxAtStartPar
ADTRGxA0
&
\sphinxAtStartPar
0x20
&&
\sphinxAtStartPar
0x 00000
&
\sphinxAtStartPar
第x组A路ADC触发点0         |
\\
\sphinxhline
\sphinxAtStartPar
ADTRGxA1
&
\sphinxAtStartPar
0x24
&&
\sphinxAtStartPar
0x 00000
&
\sphinxAtStartPar
第x组A路ADC触发点1         |
\\
\sphinxhline
\sphinxAtStartPar
BRKENx
&
\sphinxAtStartPar
0x28
&&
\sphinxAtStartPar
0x 00000
&
\sphinxAtStartPar
第x组刹车使能寄存器        |
\\
\sphinxhline
\sphinxAtStartPar
VALUEAx
&
\sphinxAtStartPar
0x2C
&&
\sphinxAtStartPar
0x 00000
&
\sphinxAtStartPar
第x组A路当前计数值         |
\\
\sphinxhline
\sphinxAtStartPar
VALUEBx
&
\sphinxAtStartPar
0x30
&&
\sphinxAtStartPar
0x 00000
&
\sphinxAtStartPar
第x组B路当前计数值         |
\\
\sphinxhline
\sphinxAtStartPar
ADTRGxB0
&
\sphinxAtStartPar
0x34
&&
\sphinxAtStartPar
0x 00000
&
\sphinxAtStartPar
第x组B路ADC触发点0         |
\\
\sphinxhline
\sphinxAtStartPar
ADTRGxB1
&
\sphinxAtStartPar
0x38
&&
\sphinxAtStartPar
0x 00000
&
\sphinxAtStartPar
第x组B路ADC触发点1         |
\\
\sphinxhline
\sphinxAtStartPar
ADTRGDIRx
&
\sphinxAtStartPar
0x3C
&&
\sphinxAtStartPar
0x 000FF
&
\sphinxAtStartPar
第x组触发ADC条件寄存器     |
\\
\sphinxhline
\sphinxAtStartPar
CONFIG
&
\sphinxAtStartPar
0x200
&&
\sphinxAtStartPar
0x 00000
&
\sphinxAtStartPar
输入脉冲触发沿控制寄存器   |
\\
\sphinxhline
\sphinxAtStartPar
FORCEO
&
\sphinxAtStartPar
0x204
&&
\sphinxAtStartPar
0x 00000
&
\sphinxAtStartPar
强制输出寄存器             |
\\
\sphinxhline
\sphinxAtStartPar
BRKCR
&
\sphinxAtStartPar
0x208
&&
\sphinxAtStartPar
0x 00000
&
\sphinxAtStartPar
刹车控制                   |
\\
\sphinxhline
\sphinxAtStartPar
BRKIE
&
\sphinxAtStartPar
0x20C
&&
\sphinxAtStartPar
0x 00000
&
\sphinxAtStartPar
刹车中断使能               |
\\
\sphinxhline
\sphinxAtStartPar
BRKIF
&
\sphinxAtStartPar
0x210
&&
\sphinxAtStartPar
0x 00000
&
\sphinxAtStartPar
刹车中断状态               |
\\
\sphinxhline
\sphinxAtStartPar
BRKIM
&
\sphinxAtStartPar
0x214
&&
\sphinxAtStartPar
0x 00000
&
\sphinxAtStartPar
刹车中断屏蔽               |
\\
\sphinxhline
\sphinxAtStartPar
BRKIRS
&
\sphinxAtStartPar
0x218
&&
\sphinxAtStartPar
0x 00000
&
\sphinxAtStartPar
刹车中断原始状态           |
\\
\sphinxhline
\sphinxAtStartPar
IE
&
\sphinxAtStartPar
0x21C
&&
\sphinxAtStartPar
0x 00000
&
\sphinxAtStartPar
中断使能                   |
\\
\sphinxhline
\sphinxAtStartPar
CHEN
&
\sphinxAtStartPar
0x220
&&
\sphinxAtStartPar
0x 00000
&
\sphinxAtStartPar
PWM输出使能                |
\\
\sphinxhline
\sphinxAtStartPar
IM
&
\sphinxAtStartPar
0x224
&&
\sphinxAtStartPar
0x 00000
&
\sphinxAtStartPar
中断屏蔽寄存器             |
\\
\sphinxhline
\sphinxAtStartPar
NCIRS
&
\sphinxAtStartPar
0x228
&&
\sphinxAtStartPar
0x 00000
&
\sphinxAtStartPar
新周期中断原始状态         |
\\
\sphinxhline
\sphinxAtStartPar
HEIRS
&
\sphinxAtStartPar
0x22C
&&
\sphinxAtStartPar
0x 00000
&
\sphinxAtStartPar
高电平结束中断原始状态     |
\\
\sphinxhline
\sphinxAtStartPar
NCIF
&
\sphinxAtStartPar
0x230
&&
\sphinxAtStartPar
0x 00000
&
\sphinxAtStartPar
新周期中断状态             |
\\
\sphinxhline
\sphinxAtStartPar
HEIF
&
\sphinxAtStartPar
0x234
&&
\sphinxAtStartPar
0x 00000
&
\sphinxAtStartPar
高电平结束中断状态         |
\\
\sphinxhline
\sphinxAtStartPar
HCIE
&
\sphinxAtStartPar
0x238
&&
\sphinxAtStartPar
0x 00000
&
\sphinxAtStartPar
半周期中断使能             |
\\
\sphinxhline
\sphinxAtStartPar
HCIM
&
\sphinxAtStartPar
0x23C
&&
\sphinxAtStartPar
0x 00000
&
\sphinxAtStartPar
半周期中断屏蔽             |
\\
\sphinxhline
\sphinxAtStartPar
HCIRS
&
\sphinxAtStartPar
0x240
&&
\sphinxAtStartPar
0x 00000
&
\sphinxAtStartPar
半周期中断原始状态         |
\\
\sphinxhline
\sphinxAtStartPar
HCIF
&
\sphinxAtStartPar
0x244
&&
\sphinxAtStartPar
0x 00000
&
\sphinxAtStartPar
半周期中断状态             |
\\
\sphinxhline
\sphinxAtStartPar
FORCEV
&
\sphinxAtStartPar
0x248
&&
\sphinxAtStartPar
0x 00000
&
\sphinxAtStartPar
强制输出电平选择寄存器     |
\\
\sphinxbottomrule
\end{longtable}
\sphinxtableafterendhook
\sphinxatlongtableend
\end{savenotes}


\subsection{寄存器描述}
\label{\detokenize{SWM241/_u529f_u80fd_u63cf_u8ff0/_u8109_u51b2_u5bbd_u5ea6_u8c03_u5236:id18}}

\subsubsection{PWM工作模式寄存器MODEx(x=0,1,2,3)}
\label{\detokenize{SWM241/_u529f_u80fd_u63cf_u8ff0/_u8109_u51b2_u5bbd_u5ea6_u8c03_u5236:pwmmodex-x-0-1-2-3}}

\begin{savenotes}\sphinxattablestart
\sphinxthistablewithglobalstyle
\centering
\begin{tabular}[t]{\X{20}{100}\X{20}{100}\X{20}{100}\X{20}{100}\X{20}{100}}
\sphinxtoprule
\sphinxtableatstartofbodyhook
\sphinxAtStartPar
寄存器 |
&
\begin{DUlineblock}{0em}
\item[] 偏移 |
\end{DUlineblock}
&
\begin{DUlineblock}{0em}
\item[] 
\item[] {\color{red}\bfseries{}|}
\end{DUlineblock}
&
\sphinxAtStartPar
复位值 |    描 | |
&
\begin{DUlineblock}{0em}
\item[] |
  |
\end{DUlineblock}
\\
\sphinxhline
\sphinxAtStartPar
MODEx
&
\sphinxAtStartPar
0x00
&&
\sphinxAtStartPar
0 000000
&
\sphinxAtStartPar
第x组PWM的工作模式控制     |
\\
\sphinxbottomrule
\end{tabular}
\sphinxtableafterendhook\par
\sphinxattableend\end{savenotes}


\begin{savenotes}\sphinxattablestart
\sphinxthistablewithglobalstyle
\centering
\begin{tabular}[t]{\X{12}{96}\X{12}{96}\X{12}{96}\X{12}{96}\X{12}{96}\X{12}{96}\X{12}{96}\X{12}{96}}
\sphinxtoprule
\sphinxtableatstartofbodyhook
\sphinxAtStartPar
31
&
\sphinxAtStartPar
30
&
\sphinxAtStartPar
29
&
\sphinxAtStartPar
28
&
\sphinxAtStartPar
27
&
\sphinxAtStartPar
26
&
\sphinxAtStartPar
25
&
\sphinxAtStartPar
24
\\
\sphinxhline\begin{itemize}
\item {} 
\end{itemize}
&&&&&&&\\
\sphinxhline
\sphinxAtStartPar
23
&
\sphinxAtStartPar
22
&
\sphinxAtStartPar
21
&
\sphinxAtStartPar
20
&
\sphinxAtStartPar
19
&
\sphinxAtStartPar
18
&
\sphinxAtStartPar
17
&
\sphinxAtStartPar
16
\\
\sphinxhline\begin{itemize}
\item {} 
\end{itemize}
&&&&&&&\\
\sphinxhline
\sphinxAtStartPar
15
&
\sphinxAtStartPar
14
&
\sphinxAtStartPar
13
&
\sphinxAtStartPar
12
&
\sphinxAtStartPar
11
&
\sphinxAtStartPar
10
&
\sphinxAtStartPar
9
&
\sphinxAtStartPar
8
\\
\sphinxhline\begin{itemize}
\item {} 
\end{itemize}
&&&&&&&\\
\sphinxhline
\sphinxAtStartPar
7
&
\sphinxAtStartPar
6
&
\sphinxAtStartPar
5
&
\sphinxAtStartPar
4
&
\sphinxAtStartPar
3
&
\sphinxAtStartPar
2
&
\sphinxAtStartPar
1
&
\sphinxAtStartPar
0
\\
\sphinxhline
\sphinxAtStartPar
CLKDIV
&&&
\sphinxAtStartPar
C RCx
&&&&\\
\sphinxbottomrule
\end{tabular}
\sphinxtableafterendhook\par
\sphinxattableend\end{savenotes}


\begin{savenotes}\sphinxattablestart
\sphinxthistablewithglobalstyle
\centering
\begin{tabular}[t]{\X{33}{99}\X{33}{99}\X{33}{99}}
\sphinxtoprule
\sphinxtableatstartofbodyhook
\sphinxAtStartPar
位域 |
&
\sphinxAtStartPar
名称     | |
&
\sphinxAtStartPar
描述                                        | |
\\
\sphinxhline
\sphinxAtStartPar
31:6
&\begin{itemize}
\item {} 
\end{itemize}
&\begin{itemize}
\item {} 
\end{itemize}
\\
\sphinxhline
\sphinxAtStartPar
12:5
&
\sphinxAtStartPar
CLKDIV
&
\sphinxAtStartPar
PWM工作时钟频率相对于系统时钟的分频比选择: |

\sphinxAtStartPar
0:无分频;                                 |

\sphinxAtStartPar
1:2分频                                    |

\sphinxAtStartPar
2:3分频                                    |

\sphinxAtStartPar
…..

\sphinxAtStartPar
255:256分频                                |
\\
\sphinxhline
\sphinxAtStartPar
4:3
&
\sphinxAtStartPar
CLKSRCx
&
\sphinxAtStartPar
第x组PWM的计数使能                          |

\sphinxAtStartPar
00:常1,直接使用PWM的工作时钟计数          |

\sphinxAtStartPar
01:使用PWM\_DIV分频后的时钟计数             |

\sphinxAtStartPar
10:使用Pulse0作为PWM的计数时钟             |

\sphinxAtStartPar
11:使用Pulse1作为PWM的计数时钟             |
\\
\sphinxhline
\sphinxAtStartPar
2:0
&
\sphinxAtStartPar
MODEx
&
\sphinxAtStartPar
第x组PWM的工作模式控制                      |

\sphinxAtStartPar
:边沿模式,每一组PWM中的A、B两路互相独立。 |

\sphinxAtStartPar
101:中心对称模式,综合对称模               | 补模式,由PWMX和PWMXn实现,独立配置死区  |
\\
\sphinxbottomrule
\end{tabular}
\sphinxtableafterendhook\par
\sphinxattableend\end{savenotes}


\subsubsection{PWM\_A路计数周期PERAx(x=0,1,2,3)}
\label{\detokenize{SWM241/_u529f_u80fd_u63cf_u8ff0/_u8109_u51b2_u5bbd_u5ea6_u8c03_u5236:pwm-aperax-x-0-1-2-3}}

\begin{savenotes}\sphinxattablestart
\sphinxthistablewithglobalstyle
\centering
\begin{tabular}[t]{\X{20}{100}\X{20}{100}\X{20}{100}\X{20}{100}\X{20}{100}}
\sphinxtoprule
\sphinxtableatstartofbodyhook
\sphinxAtStartPar
寄存器 |
&
\begin{DUlineblock}{0em}
\item[] 偏移 |
\end{DUlineblock}
&
\begin{DUlineblock}{0em}
\item[] 
\item[] {\color{red}\bfseries{}|}
\end{DUlineblock}
&
\sphinxAtStartPar
复位值 |    描 | |
&
\begin{DUlineblock}{0em}
\item[] |
  |
\end{DUlineblock}
\\
\sphinxhline
\sphinxAtStartPar
PERAx
&
\sphinxAtStartPar
0x04
&&
\sphinxAtStartPar
0 000000
&
\sphinxAtStartPar
第x组A路PWM的计数周期      |
\\
\sphinxbottomrule
\end{tabular}
\sphinxtableafterendhook\par
\sphinxattableend\end{savenotes}


\begin{savenotes}\sphinxattablestart
\sphinxthistablewithglobalstyle
\centering
\begin{tabular}[t]{\X{12}{96}\X{12}{96}\X{12}{96}\X{12}{96}\X{12}{96}\X{12}{96}\X{12}{96}\X{12}{96}}
\sphinxtoprule
\sphinxtableatstartofbodyhook
\sphinxAtStartPar
31
&
\sphinxAtStartPar
30
&
\sphinxAtStartPar
29
&
\sphinxAtStartPar
28
&
\sphinxAtStartPar
27
&
\sphinxAtStartPar
26
&
\sphinxAtStartPar
25
&
\sphinxAtStartPar
24
\\
\sphinxhline\begin{itemize}
\item {} 
\end{itemize}
&&&&&&&\\
\sphinxhline
\sphinxAtStartPar
23
&
\sphinxAtStartPar
22
&
\sphinxAtStartPar
21
&
\sphinxAtStartPar
20
&
\sphinxAtStartPar
19
&
\sphinxAtStartPar
18
&
\sphinxAtStartPar
17
&
\sphinxAtStartPar
16
\\
\sphinxhline\begin{itemize}
\item {} 
\end{itemize}
&&&&&&&\\
\sphinxhline
\sphinxAtStartPar
15
&
\sphinxAtStartPar
14
&
\sphinxAtStartPar
13
&
\sphinxAtStartPar
12
&
\sphinxAtStartPar
11
&
\sphinxAtStartPar
10
&
\sphinxAtStartPar
9
&
\sphinxAtStartPar
8
\\
\sphinxhline
\sphinxAtStartPar
PERAx
&&&&&&&\\
\sphinxhline
\sphinxAtStartPar
7
&
\sphinxAtStartPar
6
&
\sphinxAtStartPar
5
&
\sphinxAtStartPar
4
&
\sphinxAtStartPar
3
&
\sphinxAtStartPar
2
&
\sphinxAtStartPar
1
&
\sphinxAtStartPar
0
\\
\sphinxhline
\sphinxAtStartPar
PERAx
&&&&&&&\\
\sphinxbottomrule
\end{tabular}
\sphinxtableafterendhook\par
\sphinxattableend\end{savenotes}


\begin{savenotes}\sphinxattablestart
\sphinxthistablewithglobalstyle
\centering
\begin{tabular}[t]{\X{33}{99}\X{33}{99}\X{33}{99}}
\sphinxtoprule
\sphinxtableatstartofbodyhook
\sphinxAtStartPar
位域 |
&
\sphinxAtStartPar
名称     | |
&
\sphinxAtStartPar
描述                                        | |
\\
\sphinxhline
\sphinxAtStartPar
31:16
&\begin{itemize}
\item {} 
\end{itemize}
&\begin{itemize}
\item {} 
\end{itemize}
\\
\sphinxhline
\sphinxAtStartPar
15:0
&
\sphinxAtStartPar
PERAx
&
\sphinxAtStartPar
第x组A                                      | M的计数周期,最小为1,对应一个计数时钟周期 |
\\
\sphinxbottomrule
\end{tabular}
\sphinxtableafterendhook\par
\sphinxattableend\end{savenotes}


\subsubsection{PWM\_A路高电平持续时长HIGHAx(x=0,1,2,3)}
\label{\detokenize{SWM241/_u529f_u80fd_u63cf_u8ff0/_u8109_u51b2_u5bbd_u5ea6_u8c03_u5236:pwm-ahighax-x-0-1-2-3}}

\begin{savenotes}\sphinxattablestart
\sphinxthistablewithglobalstyle
\centering
\begin{tabular}[t]{\X{20}{100}\X{20}{100}\X{20}{100}\X{20}{100}\X{20}{100}}
\sphinxtoprule
\sphinxtableatstartofbodyhook
\sphinxAtStartPar
寄存器 |
&
\begin{DUlineblock}{0em}
\item[] 偏移 |
\end{DUlineblock}
&
\begin{DUlineblock}{0em}
\item[] 
\item[] {\color{red}\bfseries{}|}
\end{DUlineblock}
&
\sphinxAtStartPar
复位值 |    描 | |
&
\begin{DUlineblock}{0em}
\item[] |
  |
\end{DUlineblock}
\\
\sphinxhline
\sphinxAtStartPar
HIGHAx
&
\sphinxAtStartPar
0x08
&&
\sphinxAtStartPar
0 000000
&\\
\sphinxbottomrule
\end{tabular}
\sphinxtableafterendhook\par
\sphinxattableend\end{savenotes}


\begin{savenotes}\sphinxattablestart
\sphinxthistablewithglobalstyle
\centering
\begin{tabular}[t]{\X{12}{96}\X{12}{96}\X{12}{96}\X{12}{96}\X{12}{96}\X{12}{96}\X{12}{96}\X{12}{96}}
\sphinxtoprule
\sphinxtableatstartofbodyhook
\sphinxAtStartPar
31
&
\sphinxAtStartPar
30
&
\sphinxAtStartPar
29
&
\sphinxAtStartPar
28
&
\sphinxAtStartPar
27
&
\sphinxAtStartPar
26
&
\sphinxAtStartPar
25
&
\sphinxAtStartPar
24
\\
\sphinxhline\begin{itemize}
\item {} 
\end{itemize}
&&&&&&&\\
\sphinxhline
\sphinxAtStartPar
23
&
\sphinxAtStartPar
22
&
\sphinxAtStartPar
21
&
\sphinxAtStartPar
20
&
\sphinxAtStartPar
19
&
\sphinxAtStartPar
18
&
\sphinxAtStartPar
17
&
\sphinxAtStartPar
16
\\
\sphinxhline\begin{itemize}
\item {} 
\end{itemize}
&&&&&&&\\
\sphinxhline
\sphinxAtStartPar
15
&
\sphinxAtStartPar
14
&
\sphinxAtStartPar
13
&
\sphinxAtStartPar
12
&
\sphinxAtStartPar
11
&
\sphinxAtStartPar
10
&
\sphinxAtStartPar
9
&
\sphinxAtStartPar
8
\\
\sphinxhline
\sphinxAtStartPar
HIGHAx
&&&&&&&\\
\sphinxhline
\sphinxAtStartPar
7
&
\sphinxAtStartPar
6
&
\sphinxAtStartPar
5
&
\sphinxAtStartPar
4
&
\sphinxAtStartPar
3
&
\sphinxAtStartPar
2
&
\sphinxAtStartPar
1
&
\sphinxAtStartPar
0
\\
\sphinxhline
\sphinxAtStartPar
HIGHAx
&&&&&&&\\
\sphinxbottomrule
\end{tabular}
\sphinxtableafterendhook\par
\sphinxattableend\end{savenotes}


\begin{savenotes}\sphinxattablestart
\sphinxthistablewithglobalstyle
\centering
\begin{tabular}[t]{\X{33}{99}\X{33}{99}\X{33}{99}}
\sphinxtoprule
\sphinxtableatstartofbodyhook
\sphinxAtStartPar
位域 |
&
\sphinxAtStartPar
名称     | |
&
\sphinxAtStartPar
描述                                        | |
\\
\sphinxhline
\sphinxAtStartPar
31:17
&\begin{itemize}
\item {} 
\end{itemize}
&\begin{itemize}
\item {} 
\end{itemize}
\\
\sphinxhline
\sphinxAtStartPar
16:1
&
\sphinxAtStartPar
HIGHAx
&
\sphinxAtStartPar
第x组A路PWM的高电平持续周期。               |
\\
\sphinxhline
\sphinxAtStartPar
0
&
\sphinxAtStartPar
DELAY
&
\sphinxAtStartPar
下降沿推迟半个PWM时钟                       | 在中心对称模式下,同样影响pwmxan的输出) |

\sphinxAtStartPar
1:使能                                     |

\sphinxAtStartPar
0:禁能                                     |
\\
\sphinxbottomrule
\end{tabular}
\sphinxtableafterendhook\par
\sphinxattableend\end{savenotes}


\subsubsection{PWM\_A路死区长度DZAx(x=0,1,2,3)}
\label{\detokenize{SWM241/_u529f_u80fd_u63cf_u8ff0/_u8109_u51b2_u5bbd_u5ea6_u8c03_u5236:pwm-adzax-x-0-1-2-3}}

\begin{savenotes}\sphinxattablestart
\sphinxthistablewithglobalstyle
\centering
\begin{tabular}[t]{\X{20}{100}\X{20}{100}\X{20}{100}\X{20}{100}\X{20}{100}}
\sphinxtoprule
\sphinxtableatstartofbodyhook
\sphinxAtStartPar
寄存器 |
&
\begin{DUlineblock}{0em}
\item[] 偏移 |
\end{DUlineblock}
&
\begin{DUlineblock}{0em}
\item[] 
\item[] {\color{red}\bfseries{}|}
\end{DUlineblock}
&
\sphinxAtStartPar
复位值 |    描 | |
&
\begin{DUlineblock}{0em}
\item[] |
  |
\end{DUlineblock}
\\
\sphinxhline
\sphinxAtStartPar
DZAx
&
\sphinxAtStartPar
0x0C
&&
\sphinxAtStartPar
0 000000
&
\sphinxAtStartPar
第x组A路死区长度控制。     |
\\
\sphinxbottomrule
\end{tabular}
\sphinxtableafterendhook\par
\sphinxattableend\end{savenotes}


\begin{savenotes}\sphinxattablestart
\sphinxthistablewithglobalstyle
\centering
\begin{tabular}[t]{\X{12}{96}\X{12}{96}\X{12}{96}\X{12}{96}\X{12}{96}\X{12}{96}\X{12}{96}\X{12}{96}}
\sphinxtoprule
\sphinxtableatstartofbodyhook
\sphinxAtStartPar
31
&
\sphinxAtStartPar
30
&
\sphinxAtStartPar
29
&
\sphinxAtStartPar
28
&
\sphinxAtStartPar
27
&
\sphinxAtStartPar
26
&
\sphinxAtStartPar
25
&
\sphinxAtStartPar
24
\\
\sphinxhline\begin{itemize}
\item {} 
\end{itemize}
&&&&&&&\\
\sphinxhline
\sphinxAtStartPar
23
&
\sphinxAtStartPar
22
&
\sphinxAtStartPar
21
&
\sphinxAtStartPar
20
&
\sphinxAtStartPar
19
&
\sphinxAtStartPar
18
&
\sphinxAtStartPar
17
&
\sphinxAtStartPar
16
\\
\sphinxhline\begin{itemize}
\item {} 
\end{itemize}
&&&&&&&\\
\sphinxhline
\sphinxAtStartPar
15
&
\sphinxAtStartPar
14
&
\sphinxAtStartPar
13
&
\sphinxAtStartPar
12
&
\sphinxAtStartPar
11
&
\sphinxAtStartPar
10
&
\sphinxAtStartPar
9
&
\sphinxAtStartPar
8
\\
\sphinxhline\begin{itemize}
\item {} 
\end{itemize}
&&&&&&&\\
\sphinxhline
\sphinxAtStartPar
7
&
\sphinxAtStartPar
6
&
\sphinxAtStartPar
5
&
\sphinxAtStartPar
4
&
\sphinxAtStartPar
3
&
\sphinxAtStartPar
2
&
\sphinxAtStartPar
1
&
\sphinxAtStartPar
0
\\
\sphinxhline
\sphinxAtStartPar
DZAx
&&&&&&&\\
\sphinxbottomrule
\end{tabular}
\sphinxtableafterendhook\par
\sphinxattableend\end{savenotes}


\begin{savenotes}\sphinxattablestart
\sphinxthistablewithglobalstyle
\centering
\begin{tabular}[t]{\X{33}{99}\X{33}{99}\X{33}{99}}
\sphinxtoprule
\sphinxtableatstartofbodyhook
\sphinxAtStartPar
位域 |
&
\sphinxAtStartPar
名称     | |
&
\sphinxAtStartPar
描述                                        | |
\\
\sphinxhline
\sphinxAtStartPar
31:10
&\begin{itemize}
\item {} 
\end{itemize}
&\begin{itemize}
\item {} 
\end{itemize}
\\
\sphinxhline
\sphinxAtStartPar
9:0
&
\sphinxAtStartPar
DZAx
&
\sphinxAtStartPar
第x组A路死区长度控制。必须小于HIGHAx        |
\\
\sphinxbottomrule
\end{tabular}
\sphinxtableafterendhook\par
\sphinxattableend\end{savenotes}


\subsubsection{PWM\_B路计数周期PERBx(x=0,1,2,3)}
\label{\detokenize{SWM241/_u529f_u80fd_u63cf_u8ff0/_u8109_u51b2_u5bbd_u5ea6_u8c03_u5236:pwm-bperbx-x-0-1-2-3}}

\begin{savenotes}\sphinxattablestart
\sphinxthistablewithglobalstyle
\centering
\begin{tabular}[t]{\X{20}{100}\X{20}{100}\X{20}{100}\X{20}{100}\X{20}{100}}
\sphinxtoprule
\sphinxtableatstartofbodyhook
\sphinxAtStartPar
寄存器 |
&
\begin{DUlineblock}{0em}
\item[] 偏移 |
\end{DUlineblock}
&
\begin{DUlineblock}{0em}
\item[] 
\item[] {\color{red}\bfseries{}|}
\end{DUlineblock}
&
\sphinxAtStartPar
复位值 |    描 | |
&
\begin{DUlineblock}{0em}
\item[] |
  |
\end{DUlineblock}
\\
\sphinxhline
\sphinxAtStartPar
PERBx
&
\sphinxAtStartPar
0x10
&&
\sphinxAtStartPar
0 000000
&
\sphinxAtStartPar
第x组B路PWM的计数周期      |
\\
\sphinxbottomrule
\end{tabular}
\sphinxtableafterendhook\par
\sphinxattableend\end{savenotes}


\begin{savenotes}\sphinxattablestart
\sphinxthistablewithglobalstyle
\centering
\begin{tabular}[t]{\X{12}{96}\X{12}{96}\X{12}{96}\X{12}{96}\X{12}{96}\X{12}{96}\X{12}{96}\X{12}{96}}
\sphinxtoprule
\sphinxtableatstartofbodyhook
\sphinxAtStartPar
31
&
\sphinxAtStartPar
30
&
\sphinxAtStartPar
29
&
\sphinxAtStartPar
28
&
\sphinxAtStartPar
27
&
\sphinxAtStartPar
26
&
\sphinxAtStartPar
25
&
\sphinxAtStartPar
24
\\
\sphinxhline\begin{itemize}
\item {} 
\end{itemize}
&&&&&&&\\
\sphinxhline
\sphinxAtStartPar
23
&
\sphinxAtStartPar
22
&
\sphinxAtStartPar
21
&
\sphinxAtStartPar
20
&
\sphinxAtStartPar
19
&
\sphinxAtStartPar
18
&
\sphinxAtStartPar
17
&
\sphinxAtStartPar
16
\\
\sphinxhline\begin{itemize}
\item {} 
\end{itemize}
&&&&&&&\\
\sphinxhline
\sphinxAtStartPar
15
&
\sphinxAtStartPar
14
&
\sphinxAtStartPar
13
&
\sphinxAtStartPar
12
&
\sphinxAtStartPar
11
&
\sphinxAtStartPar
10
&
\sphinxAtStartPar
9
&
\sphinxAtStartPar
8
\\
\sphinxhline
\sphinxAtStartPar
PERBx
&&&&&&&\\
\sphinxhline
\sphinxAtStartPar
7
&
\sphinxAtStartPar
6
&
\sphinxAtStartPar
5
&
\sphinxAtStartPar
4
&
\sphinxAtStartPar
3
&
\sphinxAtStartPar
2
&
\sphinxAtStartPar
1
&
\sphinxAtStartPar
0
\\
\sphinxhline
\sphinxAtStartPar
PERBx
&&&&&&&\\
\sphinxbottomrule
\end{tabular}
\sphinxtableafterendhook\par
\sphinxattableend\end{savenotes}


\begin{savenotes}\sphinxattablestart
\sphinxthistablewithglobalstyle
\centering
\begin{tabular}[t]{\X{33}{99}\X{33}{99}\X{33}{99}}
\sphinxtoprule
\sphinxtableatstartofbodyhook
\sphinxAtStartPar
位域 |
&
\sphinxAtStartPar
名称     | |
&
\sphinxAtStartPar
描述                                        | |
\\
\sphinxhline
\sphinxAtStartPar
31:16
&\begin{itemize}
\item {} 
\end{itemize}
&\begin{itemize}
\item {} 
\end{itemize}
\\
\sphinxhline
\sphinxAtStartPar
15:0
&
\sphinxAtStartPar
PERBx
&
\sphinxAtStartPar
第x组B                                      | M的计数周期,最小为1,对应一个计数时钟周期 |
\\
\sphinxbottomrule
\end{tabular}
\sphinxtableafterendhook\par
\sphinxattableend\end{savenotes}


\subsubsection{PWM\_B路高电平持续时长HIGHBx(x=0,1,2,3)}
\label{\detokenize{SWM241/_u529f_u80fd_u63cf_u8ff0/_u8109_u51b2_u5bbd_u5ea6_u8c03_u5236:pwm-bhighbx-x-0-1-2-3}}

\begin{savenotes}\sphinxattablestart
\sphinxthistablewithglobalstyle
\centering
\begin{tabular}[t]{\X{20}{100}\X{20}{100}\X{20}{100}\X{20}{100}\X{20}{100}}
\sphinxtoprule
\sphinxtableatstartofbodyhook
\sphinxAtStartPar
寄存器 |
&
\begin{DUlineblock}{0em}
\item[] 偏移 |
\end{DUlineblock}
&
\begin{DUlineblock}{0em}
\item[] 
\item[] {\color{red}\bfseries{}|}
\end{DUlineblock}
&
\sphinxAtStartPar
复位值 |    描 | |
&
\begin{DUlineblock}{0em}
\item[] |
  |
\end{DUlineblock}
\\
\sphinxhline
\sphinxAtStartPar
HIGHBx
&
\sphinxAtStartPar
0x14
&&
\sphinxAtStartPar
0 000000
&\\
\sphinxbottomrule
\end{tabular}
\sphinxtableafterendhook\par
\sphinxattableend\end{savenotes}


\begin{savenotes}\sphinxattablestart
\sphinxthistablewithglobalstyle
\centering
\begin{tabular}[t]{\X{12}{96}\X{12}{96}\X{12}{96}\X{12}{96}\X{12}{96}\X{12}{96}\X{12}{96}\X{12}{96}}
\sphinxtoprule
\sphinxtableatstartofbodyhook
\sphinxAtStartPar
31
&
\sphinxAtStartPar
30
&
\sphinxAtStartPar
29
&
\sphinxAtStartPar
28
&
\sphinxAtStartPar
27
&
\sphinxAtStartPar
26
&
\sphinxAtStartPar
25
&
\sphinxAtStartPar
24
\\
\sphinxhline\begin{itemize}
\item {} 
\end{itemize}
&&&&&&&\\
\sphinxhline
\sphinxAtStartPar
23
&
\sphinxAtStartPar
22
&
\sphinxAtStartPar
21
&
\sphinxAtStartPar
20
&
\sphinxAtStartPar
19
&
\sphinxAtStartPar
18
&
\sphinxAtStartPar
17
&
\sphinxAtStartPar
16
\\
\sphinxhline\begin{itemize}
\item {} 
\end{itemize}
&&&&&&&\\
\sphinxhline
\sphinxAtStartPar
15
&
\sphinxAtStartPar
14
&
\sphinxAtStartPar
13
&
\sphinxAtStartPar
12
&
\sphinxAtStartPar
11
&
\sphinxAtStartPar
10
&
\sphinxAtStartPar
9
&
\sphinxAtStartPar
8
\\
\sphinxhline
\sphinxAtStartPar
HIGHBx
&&&&&&&\\
\sphinxhline
\sphinxAtStartPar
7
&
\sphinxAtStartPar
6
&
\sphinxAtStartPar
5
&
\sphinxAtStartPar
4
&
\sphinxAtStartPar
3
&
\sphinxAtStartPar
2
&
\sphinxAtStartPar
1
&
\sphinxAtStartPar
0
\\
\sphinxhline
\sphinxAtStartPar
HIGHBx
&&&&&&&\\
\sphinxbottomrule
\end{tabular}
\sphinxtableafterendhook\par
\sphinxattableend\end{savenotes}


\begin{savenotes}\sphinxattablestart
\sphinxthistablewithglobalstyle
\centering
\begin{tabular}[t]{\X{33}{99}\X{33}{99}\X{33}{99}}
\sphinxtoprule
\sphinxtableatstartofbodyhook
\sphinxAtStartPar
位域 |
&
\sphinxAtStartPar
名称     | |
&
\sphinxAtStartPar
描述                                        | |
\\
\sphinxhline
\sphinxAtStartPar
31:17
&\begin{itemize}
\item {} 
\end{itemize}
&\begin{itemize}
\item {} 
\end{itemize}
\\
\sphinxhline
\sphinxAtStartPar
16:1
&
\sphinxAtStartPar
HIGHBx
&
\sphinxAtStartPar
第x组B路PWM的高电平持续周期。               |
\\
\sphinxhline
\sphinxAtStartPar
0
&
\sphinxAtStartPar
DELAY
&
\sphinxAtStartPar
下降沿推迟半个PWM时钟周期                   |

\sphinxAtStartPar
1:使能                                     |

\sphinxAtStartPar
0:禁能                                     |
\\
\sphinxbottomrule
\end{tabular}
\sphinxtableafterendhook\par
\sphinxattableend\end{savenotes}


\subsubsection{PWM\_B路死区长度DZBx(x=0,1,2,3)}
\label{\detokenize{SWM241/_u529f_u80fd_u63cf_u8ff0/_u8109_u51b2_u5bbd_u5ea6_u8c03_u5236:pwm-bdzbx-x-0-1-2-3}}

\begin{savenotes}\sphinxattablestart
\sphinxthistablewithglobalstyle
\centering
\begin{tabular}[t]{\X{20}{100}\X{20}{100}\X{20}{100}\X{20}{100}\X{20}{100}}
\sphinxtoprule
\sphinxtableatstartofbodyhook
\sphinxAtStartPar
寄存器 |
&
\begin{DUlineblock}{0em}
\item[] 偏移 |
\end{DUlineblock}
&
\begin{DUlineblock}{0em}
\item[] 
\item[] {\color{red}\bfseries{}|}
\end{DUlineblock}
&
\sphinxAtStartPar
复位值 |    描 | |
&
\begin{DUlineblock}{0em}
\item[] |
  |
\end{DUlineblock}
\\
\sphinxhline
\sphinxAtStartPar
DZBx
&
\sphinxAtStartPar
0x18
&&
\sphinxAtStartPar
0 000000
&
\sphinxAtStartPar
第x组B路死区长度控制。     |
\\
\sphinxbottomrule
\end{tabular}
\sphinxtableafterendhook\par
\sphinxattableend\end{savenotes}


\begin{savenotes}\sphinxattablestart
\sphinxthistablewithglobalstyle
\centering
\begin{tabular}[t]{\X{12}{96}\X{12}{96}\X{12}{96}\X{12}{96}\X{12}{96}\X{12}{96}\X{12}{96}\X{12}{96}}
\sphinxtoprule
\sphinxtableatstartofbodyhook
\sphinxAtStartPar
31
&
\sphinxAtStartPar
30
&
\sphinxAtStartPar
29
&
\sphinxAtStartPar
28
&
\sphinxAtStartPar
27
&
\sphinxAtStartPar
26
&
\sphinxAtStartPar
25
&
\sphinxAtStartPar
24
\\
\sphinxhline\begin{itemize}
\item {} 
\end{itemize}
&&&&&&&\\
\sphinxhline
\sphinxAtStartPar
23
&
\sphinxAtStartPar
22
&
\sphinxAtStartPar
21
&
\sphinxAtStartPar
20
&
\sphinxAtStartPar
19
&
\sphinxAtStartPar
18
&
\sphinxAtStartPar
17
&
\sphinxAtStartPar
16
\\
\sphinxhline\begin{itemize}
\item {} 
\end{itemize}
&&&&&&&\\
\sphinxhline
\sphinxAtStartPar
15
&
\sphinxAtStartPar
14
&
\sphinxAtStartPar
13
&
\sphinxAtStartPar
12
&
\sphinxAtStartPar
11
&
\sphinxAtStartPar
10
&
\sphinxAtStartPar
9
&
\sphinxAtStartPar
8
\\
\sphinxhline\begin{itemize}
\item {} 
\end{itemize}
&&&&&&&\\
\sphinxhline
\sphinxAtStartPar
7
&
\sphinxAtStartPar
6
&
\sphinxAtStartPar
5
&
\sphinxAtStartPar
4
&
\sphinxAtStartPar
3
&
\sphinxAtStartPar
2
&
\sphinxAtStartPar
1
&
\sphinxAtStartPar
0
\\
\sphinxhline
\sphinxAtStartPar
DZBx
&&&&&&&\\
\sphinxbottomrule
\end{tabular}
\sphinxtableafterendhook\par
\sphinxattableend\end{savenotes}


\begin{savenotes}\sphinxattablestart
\sphinxthistablewithglobalstyle
\centering
\begin{tabular}[t]{\X{33}{99}\X{33}{99}\X{33}{99}}
\sphinxtoprule
\sphinxtableatstartofbodyhook
\sphinxAtStartPar
位域 |
&
\sphinxAtStartPar
名称     | |
&
\sphinxAtStartPar
描述                                        | |
\\
\sphinxhline
\sphinxAtStartPar
31:10
&\begin{itemize}
\item {} 
\end{itemize}
&\begin{itemize}
\item {} 
\end{itemize}
\\
\sphinxhline
\sphinxAtStartPar
9:0
&
\sphinxAtStartPar
DZBx
&
\sphinxAtStartPar
第x组B路死区长度控制。必须小于HIGHBx        |
\\
\sphinxbottomrule
\end{tabular}
\sphinxtableafterendhook\par
\sphinxattableend\end{savenotes}


\subsubsection{PWM输出起始值控制OUTCRx(x=0,1,2,3)}
\label{\detokenize{SWM241/_u529f_u80fd_u63cf_u8ff0/_u8109_u51b2_u5bbd_u5ea6_u8c03_u5236:pwmoutcrx-x-0-1-2-3}}

\begin{savenotes}\sphinxattablestart
\sphinxthistablewithglobalstyle
\centering
\begin{tabular}[t]{\X{20}{100}\X{20}{100}\X{20}{100}\X{20}{100}\X{20}{100}}
\sphinxtoprule
\sphinxtableatstartofbodyhook
\sphinxAtStartPar
寄存器 |
&
\begin{DUlineblock}{0em}
\item[] 偏移 |
\end{DUlineblock}
&
\begin{DUlineblock}{0em}
\item[] 
\item[] {\color{red}\bfseries{}|}
\end{DUlineblock}
&
\sphinxAtStartPar
复位值 |    描 | |
&
\begin{DUlineblock}{0em}
\item[] |
  |
\end{DUlineblock}
\\
\sphinxhline
\sphinxAtStartPar
OUTCRx
&
\sphinxAtStartPar
0x1C
&&
\sphinxAtStartPar
0 000000
&
\sphinxAtStartPar
第x组PWM输出起始值控制     |
\\
\sphinxbottomrule
\end{tabular}
\sphinxtableafterendhook\par
\sphinxattableend\end{savenotes}


\begin{savenotes}\sphinxattablestart
\sphinxthistablewithglobalstyle
\centering
\begin{tabular}[t]{\X{12}{96}\X{12}{96}\X{12}{96}\X{12}{96}\X{12}{96}\X{12}{96}\X{12}{96}\X{12}{96}}
\sphinxtoprule
\sphinxtableatstartofbodyhook
\sphinxAtStartPar
31
&
\sphinxAtStartPar
30
&
\sphinxAtStartPar
29
&
\sphinxAtStartPar
28
&
\sphinxAtStartPar
27
&
\sphinxAtStartPar
26
&
\sphinxAtStartPar
25
&
\sphinxAtStartPar
24
\\
\sphinxhline\begin{itemize}
\item {} 
\end{itemize}
&&&&&&&\\
\sphinxhline
\sphinxAtStartPar
23
&
\sphinxAtStartPar
22
&
\sphinxAtStartPar
21
&
\sphinxAtStartPar
20
&
\sphinxAtStartPar
19
&
\sphinxAtStartPar
18
&
\sphinxAtStartPar
17
&
\sphinxAtStartPar
16
\\
\sphinxhline\begin{itemize}
\item {} 
\end{itemize}
&&&&&&&\\
\sphinxhline
\sphinxAtStartPar
15
&
\sphinxAtStartPar
14
&
\sphinxAtStartPar
13
&
\sphinxAtStartPar
12
&
\sphinxAtStartPar
11
&
\sphinxAtStartPar
10
&
\sphinxAtStartPar
9
&
\sphinxAtStartPar
8
\\
\sphinxhline\begin{itemize}
\item {} 
\end{itemize}
&&&&&&&\\
\sphinxhline
\sphinxAtStartPar
7
&
\sphinxAtStartPar
6
&
\sphinxAtStartPar
5
&
\sphinxAtStartPar
4
&
\sphinxAtStartPar
3
&
\sphinxAtStartPar
2
&
\sphinxAtStartPar
1
&
\sphinxAtStartPar
0
\\
\sphinxhline
\sphinxAtStartPar
INVBNx
&
\sphinxAtStartPar
INVANx
&
\sphinxAtStartPar
I Bx
&&&&&\\
\sphinxbottomrule
\end{tabular}
\sphinxtableafterendhook\par
\sphinxattableend\end{savenotes}


\begin{savenotes}\sphinxattablestart
\sphinxthistablewithglobalstyle
\centering
\begin{tabular}[t]{\X{33}{99}\X{33}{99}\X{33}{99}}
\sphinxtoprule
\sphinxtableatstartofbodyhook
\sphinxAtStartPar
位域 |
&
\sphinxAtStartPar
名称     | |
&
\sphinxAtStartPar
描述                                        | |
\\
\sphinxhline
\sphinxAtStartPar
31:8
&\begin{itemize}
\item {} 
\end{itemize}
&\begin{itemize}
\item {} 
\end{itemize}
\\
\sphinxhline
\sphinxAtStartPar
7
&
\sphinxAtStartPar
INVBNx
&
\sphinxAtStartPar
1:将x组B路pwmxbn输出反相后输出             |

\sphinxAtStartPar
0:将x组B路pwmxbn按常规输出                 |
\\
\sphinxhline
\sphinxAtStartPar
6
&
\sphinxAtStartPar
INVANx
&
\sphinxAtStartPar
1:将x组A路pwmxan输出反相后输出             |

\sphinxAtStartPar
0:将x组A路pwmxan按常规输出                 |
\\
\sphinxhline
\sphinxAtStartPar
5
&
\sphinxAtStartPar
IDLEBx
&
\sphinxAtStartPar
1:第x组B路空闲输出为高                     |

\sphinxAtStartPar
0:第x组B路空闲输出为低                     |
\\
\sphinxhline
\sphinxAtStartPar
4
&
\sphinxAtStartPar
IDLEAx
&
\sphinxAtStartPar
1:第x组A路空闲输出为高                     |

\sphinxAtStartPar
0:第x组A路空闲输出为低                     |
\\
\sphinxhline
\sphinxAtStartPar
3
&
\sphinxAtStartPar
INVBx
&
\sphinxAtStartPar
1:第x组B路反向后输出                       |

\sphinxAtStartPar
0:第x组B路正常输出                         |
\\
\sphinxhline
\sphinxAtStartPar
2
&
\sphinxAtStartPar
INVAx
&
\sphinxAtStartPar
1:第x组A路反向后输出                       |

\sphinxAtStartPar
0:第x组A路正常输出                         |
\\
\sphinxhline
\sphinxAtStartPar
1
&
\sphinxAtStartPar
INIBx
&
\sphinxAtStartPar
1:第x组B路输出从高电平开始                 |

\sphinxAtStartPar
0:第x组B路输出从低电平开始                 |
\\
\sphinxhline
\sphinxAtStartPar
0
&
\sphinxAtStartPar
INIAx
&
\sphinxAtStartPar
1:第x组A路输出从高电平开始                 |

\sphinxAtStartPar
0:第x组A路输出从低电平开始                 |
\\
\sphinxbottomrule
\end{tabular}
\sphinxtableafterendhook\par
\sphinxattableend\end{savenotes}


\subsubsection{PWM\_A路触发ADC控制寄存器ADTRGxA0 (x=0,1,2,3)}
\label{\detokenize{SWM241/_u529f_u80fd_u63cf_u8ff0/_u8109_u51b2_u5bbd_u5ea6_u8c03_u5236:pwm-aadcadtrgxa0-x-0-1-2-3}}

\begin{savenotes}\sphinxattablestart
\sphinxthistablewithglobalstyle
\centering
\begin{tabular}[t]{\X{20}{100}\X{20}{100}\X{20}{100}\X{20}{100}\X{20}{100}}
\sphinxtoprule
\sphinxtableatstartofbodyhook
\sphinxAtStartPar
寄存器 |
&
\begin{DUlineblock}{0em}
\item[] 偏移 |
\end{DUlineblock}
&
\begin{DUlineblock}{0em}
\item[] 
\item[] {\color{red}\bfseries{}|}
\end{DUlineblock}
&
\sphinxAtStartPar
复位值 |    描 | |
&
\begin{DUlineblock}{0em}
\item[] |
  |
\end{DUlineblock}
\\
\sphinxhline
\sphinxAtStartPar
ADTRGxA0
&
\sphinxAtStartPar
0x20
&&
\sphinxAtStartPar
0 000000
&
\sphinxAtStartPar
第x组A路ADC触发点0         |
\\
\sphinxbottomrule
\end{tabular}
\sphinxtableafterendhook\par
\sphinxattableend\end{savenotes}


\begin{savenotes}\sphinxattablestart
\sphinxthistablewithglobalstyle
\centering
\begin{tabular}[t]{\X{12}{96}\X{12}{96}\X{12}{96}\X{12}{96}\X{12}{96}\X{12}{96}\X{12}{96}\X{12}{96}}
\sphinxtoprule
\sphinxtableatstartofbodyhook
\sphinxAtStartPar
31
&
\sphinxAtStartPar
30
&
\sphinxAtStartPar
29
&
\sphinxAtStartPar
28
&
\sphinxAtStartPar
27
&
\sphinxAtStartPar
26
&
\sphinxAtStartPar
25
&
\sphinxAtStartPar
24
\\
\sphinxhline\begin{itemize}
\item {} 
\end{itemize}
&&&&&&&\\
\sphinxhline
\sphinxAtStartPar
23
&
\sphinxAtStartPar
22
&
\sphinxAtStartPar
21
&
\sphinxAtStartPar
20
&
\sphinxAtStartPar
19
&
\sphinxAtStartPar
18
&
\sphinxAtStartPar
17
&
\sphinxAtStartPar
16
\\
\sphinxhline\begin{itemize}
\item {} 
\end{itemize}
&&&&&&&
\sphinxAtStartPar
EN
\\
\sphinxhline
\sphinxAtStartPar
15
&
\sphinxAtStartPar
14
&
\sphinxAtStartPar
13
&
\sphinxAtStartPar
12
&
\sphinxAtStartPar
11
&
\sphinxAtStartPar
10
&
\sphinxAtStartPar
9
&
\sphinxAtStartPar
8
\\
\sphinxhline
\sphinxAtStartPar
MATCH
&&&&&&&\\
\sphinxhline
\sphinxAtStartPar
7
&
\sphinxAtStartPar
6
&
\sphinxAtStartPar
5
&
\sphinxAtStartPar
4
&
\sphinxAtStartPar
3
&
\sphinxAtStartPar
2
&
\sphinxAtStartPar
1
&
\sphinxAtStartPar
0
\\
\sphinxhline
\sphinxAtStartPar
MATCH
&&&&&&&\\
\sphinxbottomrule
\end{tabular}
\sphinxtableafterendhook\par
\sphinxattableend\end{savenotes}


\begin{savenotes}\sphinxattablestart
\sphinxthistablewithglobalstyle
\centering
\begin{tabular}[t]{\X{33}{99}\X{33}{99}\X{33}{99}}
\sphinxtoprule
\sphinxtableatstartofbodyhook
\sphinxAtStartPar
位域 |
&
\sphinxAtStartPar
名称     | |
&
\sphinxAtStartPar
描述                                        | |
\\
\sphinxhline
\sphinxAtStartPar
31:17
&\begin{itemize}
\item {} 
\end{itemize}
&\begin{itemize}
\item {} 
\end{itemize}
\\
\sphinxhline
\sphinxAtStartPar
16
&
\sphinxAtStartPar
EN
&
\sphinxAtStartPar
第x组A路trigger0触发点是否有效              |

\sphinxAtStartPar
1:有效                                     |

\sphinxAtStartPar
0:无效                                     |
\\
\sphinxhline
\sphinxAtStartPar
15:0
&
\sphinxAtStartPar
MATCH
&
\sphinxAtStartPar
第x组A路PWM触发ADC时间点0:                 |

\sphinxAtStartPar
当对                                        | M计数器经过MATCH值延时后,输出ADC触发脉冲  |

\sphinxAtStartPar
例如:EN为1,如果第x组a路计数器的值和TRIG   | A1的值相等,则trigger2adc输出一个单周期脉冲 |

\sphinxAtStartPar
注1:当本组进入刹车状态后,不再输出脉冲。   |

\sphinxAtStartPar
注2:当PWM工                                | 心对称模式下时,在偶数周期时,不改变tri  | ra0的值,对应的触发位置为奇数周期的对称点。 |

\sphinxAtStartPar
注3:trigg                                  | 置为0对应第一个周期,而period设置为1表示一 | ,所以trigger的设置值至少要比period小1。 |
\\
\sphinxbottomrule
\end{tabular}
\sphinxtableafterendhook\par
\sphinxattableend\end{savenotes}


\subsubsection{PWM\_A路触发ADC控制寄存器ADTRGxA1(x=0,1,2,3)}
\label{\detokenize{SWM241/_u529f_u80fd_u63cf_u8ff0/_u8109_u51b2_u5bbd_u5ea6_u8c03_u5236:pwm-aadcadtrgxa1-x-0-1-2-3}}

\begin{savenotes}\sphinxattablestart
\sphinxthistablewithglobalstyle
\centering
\begin{tabular}[t]{\X{20}{100}\X{20}{100}\X{20}{100}\X{20}{100}\X{20}{100}}
\sphinxtoprule
\sphinxtableatstartofbodyhook
\sphinxAtStartPar
寄存器 |
&
\begin{DUlineblock}{0em}
\item[] 偏移 |
\end{DUlineblock}
&
\begin{DUlineblock}{0em}
\item[] 
\item[] {\color{red}\bfseries{}|}
\end{DUlineblock}
&
\sphinxAtStartPar
复位值 |    描 | |
&
\begin{DUlineblock}{0em}
\item[] |
  |
\end{DUlineblock}
\\
\sphinxhline
\sphinxAtStartPar
ADTRGxA1
&
\sphinxAtStartPar
0x24
&&
\sphinxAtStartPar
0 000000
&
\sphinxAtStartPar
第x组A路ADC触发点1         |
\\
\sphinxbottomrule
\end{tabular}
\sphinxtableafterendhook\par
\sphinxattableend\end{savenotes}


\begin{savenotes}\sphinxattablestart
\sphinxthistablewithglobalstyle
\centering
\begin{tabular}[t]{\X{12}{96}\X{12}{96}\X{12}{96}\X{12}{96}\X{12}{96}\X{12}{96}\X{12}{96}\X{12}{96}}
\sphinxtoprule
\sphinxtableatstartofbodyhook
\sphinxAtStartPar
31
&
\sphinxAtStartPar
30
&
\sphinxAtStartPar
29
&
\sphinxAtStartPar
28
&
\sphinxAtStartPar
27
&
\sphinxAtStartPar
26
&
\sphinxAtStartPar
25
&
\sphinxAtStartPar
24
\\
\sphinxhline\begin{itemize}
\item {} 
\end{itemize}
&&&&&&&\\
\sphinxhline
\sphinxAtStartPar
23
&
\sphinxAtStartPar
22
&
\sphinxAtStartPar
21
&
\sphinxAtStartPar
20
&
\sphinxAtStartPar
19
&
\sphinxAtStartPar
18
&
\sphinxAtStartPar
17
&
\sphinxAtStartPar
16
\\
\sphinxhline\begin{itemize}
\item {} 
\end{itemize}
&&&&&&&
\sphinxAtStartPar
EN
\\
\sphinxhline
\sphinxAtStartPar
15
&
\sphinxAtStartPar
14
&
\sphinxAtStartPar
13
&
\sphinxAtStartPar
12
&
\sphinxAtStartPar
11
&
\sphinxAtStartPar
10
&
\sphinxAtStartPar
9
&
\sphinxAtStartPar
8
\\
\sphinxhline
\sphinxAtStartPar
MATCH
&&&&&&&\\
\sphinxhline
\sphinxAtStartPar
7
&
\sphinxAtStartPar
6
&
\sphinxAtStartPar
5
&
\sphinxAtStartPar
4
&
\sphinxAtStartPar
3
&
\sphinxAtStartPar
2
&
\sphinxAtStartPar
1
&
\sphinxAtStartPar
0
\\
\sphinxhline
\sphinxAtStartPar
MATCH
&&&&&&&\\
\sphinxbottomrule
\end{tabular}
\sphinxtableafterendhook\par
\sphinxattableend\end{savenotes}


\begin{savenotes}\sphinxattablestart
\sphinxthistablewithglobalstyle
\centering
\begin{tabular}[t]{\X{33}{99}\X{33}{99}\X{33}{99}}
\sphinxtoprule
\sphinxtableatstartofbodyhook
\sphinxAtStartPar
位域 |
&
\sphinxAtStartPar
名称     | |
&
\sphinxAtStartPar
描述                                        | |
\\
\sphinxhline
\sphinxAtStartPar
31:17
&\begin{itemize}
\item {} 
\end{itemize}
&\begin{itemize}
\item {} 
\end{itemize}
\\
\sphinxhline
\sphinxAtStartPar
16
&
\sphinxAtStartPar
EN
&
\sphinxAtStartPar
第x组A路trigger1触发点是否有效              |

\sphinxAtStartPar
1:有效                                     |

\sphinxAtStartPar
0:无效                                     |
\\
\sphinxhline
\sphinxAtStartPar
15:0
&
\sphinxAtStartPar
MATCH
&
\sphinxAtStartPar
第x组A路PWM触发ADC时间点1:                 |

\sphinxAtStartPar
当对                                        | M计数器经过MATCH值延时后,输出ADC触发脉冲  |

\sphinxAtStartPar
例如:EN为1,如果第x组a路计数器的值和TRIG   | A1的值相等,则trigger2adc输出一个单周期脉冲 |

\sphinxAtStartPar
注1:当本组进入刹车状态后,不再输出脉冲。   |

\sphinxAtStartPar
注2:当PWM工                                | 心对称模式下时,在偶数周期时,不改变tri  | ra0的值,对应的触发位置为奇数周期的对称点。 |

\sphinxAtStartPar
注3:trigg                                  | 置为0对应第一个周期,而period设置为1表示一 | ,所以trigger的设置值至少要比period小1。 |
\\
\sphinxbottomrule
\end{tabular}
\sphinxtableafterendhook\par
\sphinxattableend\end{savenotes}


\subsubsection{PWM刹车使能寄存器BRKENx (x=0,1,2,3)}
\label{\detokenize{SWM241/_u529f_u80fd_u63cf_u8ff0/_u8109_u51b2_u5bbd_u5ea6_u8c03_u5236:pwmbrkenx-x-0-1-2-3}}

\begin{savenotes}\sphinxattablestart
\sphinxthistablewithglobalstyle
\centering
\begin{tabular}[t]{\X{20}{100}\X{20}{100}\X{20}{100}\X{20}{100}\X{20}{100}}
\sphinxtoprule
\sphinxtableatstartofbodyhook
\sphinxAtStartPar
寄存器 |
&
\begin{DUlineblock}{0em}
\item[] 偏移 |
\end{DUlineblock}
&
\begin{DUlineblock}{0em}
\item[] 
\item[] {\color{red}\bfseries{}|}
\end{DUlineblock}
&
\sphinxAtStartPar
复位值 |    描 | |
&
\begin{DUlineblock}{0em}
\item[] |
  |
\end{DUlineblock}
\\
\sphinxhline
\sphinxAtStartPar
BRKENx
&
\sphinxAtStartPar
0x28
&&
\sphinxAtStartPar
0 000000
&
\sphinxAtStartPar
第x组刹车使能寄存器        |
\\
\sphinxbottomrule
\end{tabular}
\sphinxtableafterendhook\par
\sphinxattableend\end{savenotes}


\begin{savenotes}\sphinxattablestart
\sphinxthistablewithglobalstyle
\centering
\begin{tabular}[t]{\X{12}{96}\X{12}{96}\X{12}{96}\X{12}{96}\X{12}{96}\X{12}{96}\X{12}{96}\X{12}{96}}
\sphinxtoprule
\sphinxtableatstartofbodyhook
\sphinxAtStartPar
31
&
\sphinxAtStartPar
30
&
\sphinxAtStartPar
29
&
\sphinxAtStartPar
28
&
\sphinxAtStartPar
27
&
\sphinxAtStartPar
26
&
\sphinxAtStartPar
25
&
\sphinxAtStartPar
24
\\
\sphinxhline\begin{itemize}
\item {} 
\end{itemize}
&&&&&&&\\
\sphinxhline
\sphinxAtStartPar
23
&
\sphinxAtStartPar
22
&
\sphinxAtStartPar
21
&
\sphinxAtStartPar
20
&
\sphinxAtStartPar
19
&
\sphinxAtStartPar
18
&
\sphinxAtStartPar
17
&
\sphinxAtStartPar
16
\\
\sphinxhline\begin{itemize}
\item {} 
\end{itemize}
&&&&&&&\\
\sphinxhline
\sphinxAtStartPar
15
&
\sphinxAtStartPar
14
&
\sphinxAtStartPar
13
&
\sphinxAtStartPar
12
&
\sphinxAtStartPar
11
&
\sphinxAtStartPar
10
&
\sphinxAtStartPar
9
&
\sphinxAtStartPar
8
\\
\sphinxhline\begin{itemize}
\item {} 
\end{itemize}
&&&&&&&\\
\sphinxhline
\sphinxAtStartPar
7
&
\sphinxAtStartPar
6
&
\sphinxAtStartPar
5
&
\sphinxAtStartPar
4
&
\sphinxAtStartPar
3
&
\sphinxAtStartPar
2
&
\sphinxAtStartPar
1
&
\sphinxAtStartPar
0
\\
\sphinxhline\begin{itemize}
\item {} 
\end{itemize}
&&&
\sphinxAtStartPar
BRK OUT
&&
\sphinxAtStartPar
BR \_S1
&&
\sphinxAtStartPar
EN
\\
\sphinxbottomrule
\end{tabular}
\sphinxtableafterendhook\par
\sphinxattableend\end{savenotes}


\begin{savenotes}\sphinxattablestart
\sphinxthistablewithglobalstyle
\centering
\begin{tabular}[t]{\X{33}{99}\X{33}{99}\X{33}{99}}
\sphinxtoprule
\sphinxtableatstartofbodyhook
\sphinxAtStartPar
位域 |
&
\sphinxAtStartPar
名称     | |
&
\sphinxAtStartPar
描述                                        | |
\\
\sphinxhline
\sphinxAtStartPar
31:5
&\begin{itemize}
\item {} 
\end{itemize}
&\begin{itemize}
\item {} 
\end{itemize}
\\
\sphinxhline
\sphinxAtStartPar
4
&
\sphinxAtStartPar
BRKEN\_OUT
&
\sphinxAtStartPar
1:刹车过程中输出高电平                     |

\sphinxAtStartPar
0:刹车过程中输出低电平                     |
\\
\sphinxhline
\sphinxAtStartPar
3
&
\sphinxAtStartPar
BRKEN\_S2
&
\sphinxAtStartPar
BRAKE2引脚是否对该组PWM有效                 |

\sphinxAtStartPar
1:有效                                     |

\sphinxAtStartPar
0:无效                                     |
\\
\sphinxhline
\sphinxAtStartPar
2
&
\sphinxAtStartPar
BRKEN\_S1
&
\sphinxAtStartPar
BRAKE1引脚是否对该组PWM有效                 |

\sphinxAtStartPar
1:有效                                     |

\sphinxAtStartPar
0:无效                                     |
\\
\sphinxhline
\sphinxAtStartPar
1
&
\sphinxAtStartPar
BRKEN\_S0
&
\sphinxAtStartPar
BRAKE0引脚是否对该组PWM有效                 |

\sphinxAtStartPar
1:有效                                     |

\sphinxAtStartPar
0:无效                                     |
\\
\sphinxhline
\sphinxAtStartPar
0
&
\sphinxAtStartPar
EN
&
\sphinxAtStartPar
1:刹车功能生效                             |

\sphinxAtStartPar
0:屏蔽刹车功能                             |
\\
\sphinxbottomrule
\end{tabular}
\sphinxtableafterendhook\par
\sphinxattableend\end{savenotes}


\subsubsection{PWMA路计数值寄存器VALUEA x (x=0,1,2,3)}
\label{\detokenize{SWM241/_u529f_u80fd_u63cf_u8ff0/_u8109_u51b2_u5bbd_u5ea6_u8c03_u5236:pwmavaluea-x-x-0-1-2-3}}

\begin{savenotes}\sphinxattablestart
\sphinxthistablewithglobalstyle
\centering
\begin{tabular}[t]{\X{20}{100}\X{20}{100}\X{20}{100}\X{20}{100}\X{20}{100}}
\sphinxtoprule
\sphinxtableatstartofbodyhook
\sphinxAtStartPar
寄存器 |
&
\begin{DUlineblock}{0em}
\item[] 偏移 |
\end{DUlineblock}
&
\begin{DUlineblock}{0em}
\item[] 
\item[] {\color{red}\bfseries{}|}
\end{DUlineblock}
&
\sphinxAtStartPar
复位值 |    描 | |
&
\begin{DUlineblock}{0em}
\item[] |
  |
\end{DUlineblock}
\\
\sphinxhline
\sphinxAtStartPar
VALUEAx
&
\sphinxAtStartPar
0x2C
&&
\sphinxAtStartPar
0 000000
&
\sphinxAtStartPar
第x组A路当前计数值         |
\\
\sphinxbottomrule
\end{tabular}
\sphinxtableafterendhook\par
\sphinxattableend\end{savenotes}


\begin{savenotes}\sphinxattablestart
\sphinxthistablewithglobalstyle
\centering
\begin{tabular}[t]{\X{12}{96}\X{12}{96}\X{12}{96}\X{12}{96}\X{12}{96}\X{12}{96}\X{12}{96}\X{12}{96}}
\sphinxtoprule
\sphinxtableatstartofbodyhook
\sphinxAtStartPar
31
&
\sphinxAtStartPar
30
&
\sphinxAtStartPar
29
&
\sphinxAtStartPar
28
&
\sphinxAtStartPar
27
&
\sphinxAtStartPar
26
&
\sphinxAtStartPar
25
&
\sphinxAtStartPar
24
\\
\sphinxhline\begin{itemize}
\item {} 
\end{itemize}
&&&&&&&\\
\sphinxhline
\sphinxAtStartPar
23
&
\sphinxAtStartPar
22
&
\sphinxAtStartPar
21
&
\sphinxAtStartPar
20
&
\sphinxAtStartPar
19
&
\sphinxAtStartPar
18
&
\sphinxAtStartPar
17
&
\sphinxAtStartPar
16
\\
\sphinxhline\begin{itemize}
\item {} 
\end{itemize}
&&&&&&&\\
\sphinxhline
\sphinxAtStartPar
15
&
\sphinxAtStartPar
14
&
\sphinxAtStartPar
13
&
\sphinxAtStartPar
12
&
\sphinxAtStartPar
11
&
\sphinxAtStartPar
10
&
\sphinxAtStartPar
9
&
\sphinxAtStartPar
8
\\
\sphinxhline
\sphinxAtStartPar
CNT
&&&&&&&\\
\sphinxhline
\sphinxAtStartPar
7
&
\sphinxAtStartPar
6
&
\sphinxAtStartPar
5
&
\sphinxAtStartPar
4
&
\sphinxAtStartPar
3
&
\sphinxAtStartPar
2
&
\sphinxAtStartPar
1
&
\sphinxAtStartPar
0
\\
\sphinxhline
\sphinxAtStartPar
CNT
&&&&&&&\\
\sphinxbottomrule
\end{tabular}
\sphinxtableafterendhook\par
\sphinxattableend\end{savenotes}


\begin{savenotes}\sphinxattablestart
\sphinxthistablewithglobalstyle
\centering
\begin{tabular}[t]{\X{33}{99}\X{33}{99}\X{33}{99}}
\sphinxtoprule
\sphinxtableatstartofbodyhook
\sphinxAtStartPar
位域 |
&
\sphinxAtStartPar
名称     | |
&
\sphinxAtStartPar
描述                                        | |
\\
\sphinxhline
\sphinxAtStartPar
31:16
&\begin{itemize}
\item {} 
\end{itemize}
&\begin{itemize}
\item {} 
\end{itemize}
\\
\sphinxhline
\sphinxAtStartPar
15:0
&
\sphinxAtStartPar
CNT
&
\sphinxAtStartPar
X组A路当前对应计数值                        |
\\
\sphinxbottomrule
\end{tabular}
\sphinxtableafterendhook\par
\sphinxattableend\end{savenotes}


\subsubsection{PWMB路计数值寄存器VALUEB x (x=0,1,2,3)}
\label{\detokenize{SWM241/_u529f_u80fd_u63cf_u8ff0/_u8109_u51b2_u5bbd_u5ea6_u8c03_u5236:pwmbvalueb-x-x-0-1-2-3}}

\begin{savenotes}\sphinxattablestart
\sphinxthistablewithglobalstyle
\centering
\begin{tabular}[t]{\X{20}{100}\X{20}{100}\X{20}{100}\X{20}{100}\X{20}{100}}
\sphinxtoprule
\sphinxtableatstartofbodyhook
\sphinxAtStartPar
寄存器 |
&
\begin{DUlineblock}{0em}
\item[] 偏移 |
\end{DUlineblock}
&
\begin{DUlineblock}{0em}
\item[] 
\item[] {\color{red}\bfseries{}|}
\end{DUlineblock}
&
\sphinxAtStartPar
复位值 |    描 | |
&
\begin{DUlineblock}{0em}
\item[] |
  |
\end{DUlineblock}
\\
\sphinxhline
\sphinxAtStartPar
VALUEBx
&
\sphinxAtStartPar
0x30
&&
\sphinxAtStartPar
0 000000
&
\sphinxAtStartPar
第x组B路当前计数值         |
\\
\sphinxbottomrule
\end{tabular}
\sphinxtableafterendhook\par
\sphinxattableend\end{savenotes}


\begin{savenotes}\sphinxattablestart
\sphinxthistablewithglobalstyle
\centering
\begin{tabular}[t]{\X{12}{96}\X{12}{96}\X{12}{96}\X{12}{96}\X{12}{96}\X{12}{96}\X{12}{96}\X{12}{96}}
\sphinxtoprule
\sphinxtableatstartofbodyhook
\sphinxAtStartPar
31
&
\sphinxAtStartPar
30
&
\sphinxAtStartPar
29
&
\sphinxAtStartPar
28
&
\sphinxAtStartPar
27
&
\sphinxAtStartPar
26
&
\sphinxAtStartPar
25
&
\sphinxAtStartPar
24
\\
\sphinxhline\begin{itemize}
\item {} 
\end{itemize}
&&&&&&&\\
\sphinxhline
\sphinxAtStartPar
23
&
\sphinxAtStartPar
22
&
\sphinxAtStartPar
21
&
\sphinxAtStartPar
20
&
\sphinxAtStartPar
19
&
\sphinxAtStartPar
18
&
\sphinxAtStartPar
17
&
\sphinxAtStartPar
16
\\
\sphinxhline\begin{itemize}
\item {} 
\end{itemize}
&&&&&&&\\
\sphinxhline
\sphinxAtStartPar
15
&
\sphinxAtStartPar
14
&
\sphinxAtStartPar
13
&
\sphinxAtStartPar
12
&
\sphinxAtStartPar
11
&
\sphinxAtStartPar
10
&
\sphinxAtStartPar
9
&
\sphinxAtStartPar
8
\\
\sphinxhline
\sphinxAtStartPar
CNT
&&&&&&&\\
\sphinxhline
\sphinxAtStartPar
7
&
\sphinxAtStartPar
6
&
\sphinxAtStartPar
5
&
\sphinxAtStartPar
4
&
\sphinxAtStartPar
3
&
\sphinxAtStartPar
2
&
\sphinxAtStartPar
1
&
\sphinxAtStartPar
0
\\
\sphinxhline
\sphinxAtStartPar
CNT
&&&&&&&\\
\sphinxbottomrule
\end{tabular}
\sphinxtableafterendhook\par
\sphinxattableend\end{savenotes}


\begin{savenotes}\sphinxattablestart
\sphinxthistablewithglobalstyle
\centering
\begin{tabular}[t]{\X{33}{99}\X{33}{99}\X{33}{99}}
\sphinxtoprule
\sphinxtableatstartofbodyhook
\sphinxAtStartPar
位域 |
&
\sphinxAtStartPar
名称     | |
&
\sphinxAtStartPar
描述                                        | |
\\
\sphinxhline
\sphinxAtStartPar
31:16
&\begin{itemize}
\item {} 
\end{itemize}
&\begin{itemize}
\item {} 
\end{itemize}
\\
\sphinxhline
\sphinxAtStartPar
15:0
&
\sphinxAtStartPar
CNT
&
\sphinxAtStartPar
X组B路当前对应计数值                        |
\\
\sphinxbottomrule
\end{tabular}
\sphinxtableafterendhook\par
\sphinxattableend\end{savenotes}


\subsubsection{PWMB路触发ADC控制寄存器ADTRGxB0 (x=0,1,2,3)}
\label{\detokenize{SWM241/_u529f_u80fd_u63cf_u8ff0/_u8109_u51b2_u5bbd_u5ea6_u8c03_u5236:pwmbadcadtrgxb0-x-0-1-2-3}}

\begin{savenotes}\sphinxattablestart
\sphinxthistablewithglobalstyle
\centering
\begin{tabular}[t]{\X{20}{100}\X{20}{100}\X{20}{100}\X{20}{100}\X{20}{100}}
\sphinxtoprule
\sphinxtableatstartofbodyhook
\sphinxAtStartPar
寄存器 |
&
\begin{DUlineblock}{0em}
\item[] 偏移 |
\end{DUlineblock}
&
\begin{DUlineblock}{0em}
\item[] 
\item[] {\color{red}\bfseries{}|}
\end{DUlineblock}
&
\sphinxAtStartPar
复位值 |    描 | |
&
\begin{DUlineblock}{0em}
\item[] |
  |
\end{DUlineblock}
\\
\sphinxhline
\sphinxAtStartPar
ADTRGxB0
&
\sphinxAtStartPar
0x34
&&
\sphinxAtStartPar
0 000000
&
\sphinxAtStartPar
第x组B路ADC触发点0         |
\\
\sphinxbottomrule
\end{tabular}
\sphinxtableafterendhook\par
\sphinxattableend\end{savenotes}


\begin{savenotes}\sphinxattablestart
\sphinxthistablewithglobalstyle
\centering
\begin{tabular}[t]{\X{12}{96}\X{12}{96}\X{12}{96}\X{12}{96}\X{12}{96}\X{12}{96}\X{12}{96}\X{12}{96}}
\sphinxtoprule
\sphinxtableatstartofbodyhook
\sphinxAtStartPar
31
&
\sphinxAtStartPar
30
&
\sphinxAtStartPar
29
&
\sphinxAtStartPar
28
&
\sphinxAtStartPar
27
&
\sphinxAtStartPar
26
&
\sphinxAtStartPar
25
&
\sphinxAtStartPar
24
\\
\sphinxhline\begin{itemize}
\item {} 
\end{itemize}
&&&&&&&\\
\sphinxhline
\sphinxAtStartPar
23
&
\sphinxAtStartPar
22
&
\sphinxAtStartPar
21
&
\sphinxAtStartPar
20
&
\sphinxAtStartPar
19
&
\sphinxAtStartPar
18
&
\sphinxAtStartPar
17
&
\sphinxAtStartPar
16
\\
\sphinxhline\begin{itemize}
\item {} 
\end{itemize}
&&&&&&&
\sphinxAtStartPar
EN
\\
\sphinxhline
\sphinxAtStartPar
15
&
\sphinxAtStartPar
14
&
\sphinxAtStartPar
13
&
\sphinxAtStartPar
12
&
\sphinxAtStartPar
11
&
\sphinxAtStartPar
10
&
\sphinxAtStartPar
9
&
\sphinxAtStartPar
8
\\
\sphinxhline
\sphinxAtStartPar
MATCH
&&&&&&&\\
\sphinxhline
\sphinxAtStartPar
7
&
\sphinxAtStartPar
6
&
\sphinxAtStartPar
5
&
\sphinxAtStartPar
4
&
\sphinxAtStartPar
3
&
\sphinxAtStartPar
2
&
\sphinxAtStartPar
1
&
\sphinxAtStartPar
0
\\
\sphinxhline
\sphinxAtStartPar
MATCH
&&&&&&&\\
\sphinxbottomrule
\end{tabular}
\sphinxtableafterendhook\par
\sphinxattableend\end{savenotes}


\begin{savenotes}\sphinxattablestart
\sphinxthistablewithglobalstyle
\centering
\begin{tabular}[t]{\X{33}{99}\X{33}{99}\X{33}{99}}
\sphinxtoprule
\sphinxtableatstartofbodyhook
\sphinxAtStartPar
位域 |
&
\sphinxAtStartPar
名称     | |
&
\sphinxAtStartPar
描述                                        | |
\\
\sphinxhline
\sphinxAtStartPar
31:17
&\begin{itemize}
\item {} 
\end{itemize}
&\begin{itemize}
\item {} 
\end{itemize}
\\
\sphinxhline
\sphinxAtStartPar
16
&
\sphinxAtStartPar
EN
&
\sphinxAtStartPar
第x组B路trigger0触发点是否有效              |

\sphinxAtStartPar
1:有效                                     |

\sphinxAtStartPar
0:无效                                     |
\\
\sphinxhline
\sphinxAtStartPar
15:0
&
\sphinxAtStartPar
MATCH
&
\sphinxAtStartPar
第x组B路PWM触发ADC时间点0:                 |

\sphinxAtStartPar
当对                                        | M计数器经过MATCH值延时后,输出ADC触发脉冲  |

\sphinxAtStartPar
例如:EN为1,如果第x组a路计数器的值和TRIG   | A1的值相等,则trigger2adc输出一个单周期脉冲 |

\sphinxAtStartPar
注1:当本组进入刹车状态后,不再输出脉冲。   |

\sphinxAtStartPar
注2:当PWM工                                | 心对称模式下时,在偶数周期时,不改变tri  | ra0的值,对应的触发位置为奇数周期的对称点。 |

\sphinxAtStartPar
注3:trigg                                  | 置为0对应第一个周期,而period设置为1表示一 | ,所以trigger的设置值至少要比period小1。 |
\\
\sphinxbottomrule
\end{tabular}
\sphinxtableafterendhook\par
\sphinxattableend\end{savenotes}


\subsubsection{PWMB路触发ADC控制寄存器ADTRGxB1(x=0,1,2,3)}
\label{\detokenize{SWM241/_u529f_u80fd_u63cf_u8ff0/_u8109_u51b2_u5bbd_u5ea6_u8c03_u5236:pwmbadcadtrgxb1-x-0-1-2-3}}

\begin{savenotes}\sphinxattablestart
\sphinxthistablewithglobalstyle
\centering
\begin{tabular}[t]{\X{20}{100}\X{20}{100}\X{20}{100}\X{20}{100}\X{20}{100}}
\sphinxtoprule
\sphinxtableatstartofbodyhook
\sphinxAtStartPar
寄存器 |
&
\begin{DUlineblock}{0em}
\item[] 偏移 |
\end{DUlineblock}
&
\begin{DUlineblock}{0em}
\item[] 
\item[] {\color{red}\bfseries{}|}
\end{DUlineblock}
&
\sphinxAtStartPar
复位值 |    描 | |
&
\begin{DUlineblock}{0em}
\item[] |
  |
\end{DUlineblock}
\\
\sphinxhline
\sphinxAtStartPar
ADTRGxB1
&
\sphinxAtStartPar
0x38
&&
\sphinxAtStartPar
0 000000
&
\sphinxAtStartPar
第x组B路ADC触发点1         |
\\
\sphinxbottomrule
\end{tabular}
\sphinxtableafterendhook\par
\sphinxattableend\end{savenotes}


\begin{savenotes}\sphinxattablestart
\sphinxthistablewithglobalstyle
\centering
\begin{tabular}[t]{\X{12}{96}\X{12}{96}\X{12}{96}\X{12}{96}\X{12}{96}\X{12}{96}\X{12}{96}\X{12}{96}}
\sphinxtoprule
\sphinxtableatstartofbodyhook
\sphinxAtStartPar
31
&
\sphinxAtStartPar
30
&
\sphinxAtStartPar
29
&
\sphinxAtStartPar
28
&
\sphinxAtStartPar
27
&
\sphinxAtStartPar
26
&
\sphinxAtStartPar
25
&
\sphinxAtStartPar
24
\\
\sphinxhline\begin{itemize}
\item {} 
\end{itemize}
&&&&&&&\\
\sphinxhline
\sphinxAtStartPar
23
&
\sphinxAtStartPar
22
&
\sphinxAtStartPar
21
&
\sphinxAtStartPar
20
&
\sphinxAtStartPar
19
&
\sphinxAtStartPar
18
&
\sphinxAtStartPar
17
&
\sphinxAtStartPar
16
\\
\sphinxhline\begin{itemize}
\item {} 
\end{itemize}
&&&&&&&
\sphinxAtStartPar
EN
\\
\sphinxhline
\sphinxAtStartPar
15
&
\sphinxAtStartPar
14
&
\sphinxAtStartPar
13
&
\sphinxAtStartPar
12
&
\sphinxAtStartPar
11
&
\sphinxAtStartPar
10
&
\sphinxAtStartPar
9
&
\sphinxAtStartPar
8
\\
\sphinxhline
\sphinxAtStartPar
MATCH
&&&&&&&\\
\sphinxhline
\sphinxAtStartPar
7
&
\sphinxAtStartPar
6
&
\sphinxAtStartPar
5
&
\sphinxAtStartPar
4
&
\sphinxAtStartPar
3
&
\sphinxAtStartPar
2
&
\sphinxAtStartPar
1
&
\sphinxAtStartPar
0
\\
\sphinxhline
\sphinxAtStartPar
MATCH
&&&&&&&\\
\sphinxbottomrule
\end{tabular}
\sphinxtableafterendhook\par
\sphinxattableend\end{savenotes}


\begin{savenotes}\sphinxattablestart
\sphinxthistablewithglobalstyle
\centering
\begin{tabular}[t]{\X{33}{99}\X{33}{99}\X{33}{99}}
\sphinxtoprule
\sphinxtableatstartofbodyhook
\sphinxAtStartPar
位域 |
&
\sphinxAtStartPar
名称     | |
&
\sphinxAtStartPar
描述                                        | |
\\
\sphinxhline
\sphinxAtStartPar
31:17
&\begin{itemize}
\item {} 
\end{itemize}
&\begin{itemize}
\item {} 
\end{itemize}
\\
\sphinxhline
\sphinxAtStartPar
16
&
\sphinxAtStartPar
EN
&
\sphinxAtStartPar
第x组B路trigger1触发点是否有效              |

\sphinxAtStartPar
1:有效                                     |

\sphinxAtStartPar
0:无效                                     |
\\
\sphinxhline
\sphinxAtStartPar
15:0
&
\sphinxAtStartPar
MATCH
&
\sphinxAtStartPar
第x组B路PWM触发ADC时间点1:                 |

\sphinxAtStartPar
当对                                        | M计数器经过MATCH值延时后,输出ADC触发脉冲  |

\sphinxAtStartPar
例如:EN为1,如果第x组a路计数器的值和TRIG   | A1的值相等,则trigger2adc输出一个单周期脉冲 |

\sphinxAtStartPar
注1:当本组进入刹车状态后,不再输出脉冲。   |

\sphinxAtStartPar
注2:当PWM工                                | 心对称模式下时,在偶数周期时,不改变tri  | ra0的值,对应的触发位置为奇数周期的对称点。 |

\sphinxAtStartPar
注3:trigg                                  | 置为0对应第一个周期,而period设置为1表示一 | ,所以trigger的设置值至少要比period小1。 |
\\
\sphinxbottomrule
\end{tabular}
\sphinxtableafterendhook\par
\sphinxattableend\end{savenotes}


\subsubsection{第x组触发ADC条件寄存器ADTRGDIRx(x=0,1,2,3)}
\label{\detokenize{SWM241/_u529f_u80fd_u63cf_u8ff0/_u8109_u51b2_u5bbd_u5ea6_u8c03_u5236:xadcadtrgdirx-x-0-1-2-3}}

\begin{savenotes}\sphinxattablestart
\sphinxthistablewithglobalstyle
\centering
\begin{tabular}[t]{\X{20}{100}\X{20}{100}\X{20}{100}\X{20}{100}\X{20}{100}}
\sphinxtoprule
\sphinxtableatstartofbodyhook
\sphinxAtStartPar
寄存器 |
&
\begin{DUlineblock}{0em}
\item[] 偏移 |
\end{DUlineblock}
&
\begin{DUlineblock}{0em}
\item[] 
\item[] {\color{red}\bfseries{}|}
\end{DUlineblock}
&
\sphinxAtStartPar
复位值 |    描 | |
&
\begin{DUlineblock}{0em}
\item[] |
  |
\end{DUlineblock}
\\
\sphinxhline
\sphinxAtStartPar
ADTRGDIRx
&
\sphinxAtStartPar
0x3C
&&
\sphinxAtStartPar
0 0000FF
&
\sphinxAtStartPar
第x组触发ADC条件寄存器     |
\\
\sphinxbottomrule
\end{tabular}
\sphinxtableafterendhook\par
\sphinxattableend\end{savenotes}


\begin{savenotes}\sphinxattablestart
\sphinxthistablewithglobalstyle
\centering
\begin{tabular}[t]{\X{12}{96}\X{12}{96}\X{12}{96}\X{12}{96}\X{12}{96}\X{12}{96}\X{12}{96}\X{12}{96}}
\sphinxtoprule
\sphinxtableatstartofbodyhook
\sphinxAtStartPar
31
&
\sphinxAtStartPar
30
&
\sphinxAtStartPar
29
&
\sphinxAtStartPar
28
&
\sphinxAtStartPar
27
&
\sphinxAtStartPar
26
&
\sphinxAtStartPar
25
&
\sphinxAtStartPar
24
\\
\sphinxhline\begin{itemize}
\item {} 
\end{itemize}
&&&&&&&\\
\sphinxhline
\sphinxAtStartPar
23
&
\sphinxAtStartPar
22
&
\sphinxAtStartPar
21
&
\sphinxAtStartPar
20
&
\sphinxAtStartPar
19
&
\sphinxAtStartPar
18
&
\sphinxAtStartPar
17
&
\sphinxAtStartPar
16
\\
\sphinxhline\begin{itemize}
\item {} 
\end{itemize}
&&&&&&&\\
\sphinxhline
\sphinxAtStartPar
15
&
\sphinxAtStartPar
14
&
\sphinxAtStartPar
13
&
\sphinxAtStartPar
12
&
\sphinxAtStartPar
11
&
\sphinxAtStartPar
10
&
\sphinxAtStartPar
9
&
\sphinxAtStartPar
8
\\
\sphinxhline\begin{itemize}
\item {} 
\end{itemize}
&&&&&&&\\
\sphinxhline
\sphinxAtStartPar
7
&
\sphinxAtStartPar
6
&
\sphinxAtStartPar
5
&
\sphinxAtStartPar
4
&
\sphinxAtStartPar
3
&
\sphinxAtStartPar
2
&
\sphinxAtStartPar
1
&
\sphinxAtStartPar
0
\\
\sphinxhline
\sphinxAtStartPar
B1INC
&
\sphinxAtStartPar
B1DEC
&&&&&&\\
\sphinxbottomrule
\end{tabular}
\sphinxtableafterendhook\par
\sphinxattableend\end{savenotes}


\begin{savenotes}\sphinxattablestart
\sphinxthistablewithglobalstyle
\centering
\begin{tabular}[t]{\X{33}{99}\X{33}{99}\X{33}{99}}
\sphinxtoprule
\sphinxtableatstartofbodyhook
\sphinxAtStartPar
位域 |
&
\sphinxAtStartPar
名称     | |
&
\sphinxAtStartPar
描述                                        | |
\\
\sphinxhline
\sphinxAtStartPar
31:8
&\begin{itemize}
\item {} 
\end{itemize}
&\begin{itemize}
\item {} 
\end{itemize}
\\
\sphinxhline
\sphinxAtStartPar
7
&
\sphinxAtStartPar
B1INC
&
\sphinxAtStartPar
ADTRGB1在奇数周期有效使能                   |

\sphinxAtStartPar
0:禁能                                     |

\sphinxAtStartPar
1:使能                                     |
\\
\sphinxhline
\sphinxAtStartPar
6
&
\sphinxAtStartPar
B1DEC
&
\sphinxAtStartPar
ADTRGB1在偶数周期有效使能                   |

\sphinxAtStartPar
0:禁能                                     |

\sphinxAtStartPar
1:使能                                     |
\\
\sphinxhline
\sphinxAtStartPar
5
&
\sphinxAtStartPar
B0INC
&
\sphinxAtStartPar
ADTRGB1在奇数周期有效使能                   |

\sphinxAtStartPar
0:禁能                                     |

\sphinxAtStartPar
1:使能                                     |
\\
\sphinxhline
\sphinxAtStartPar
4
&
\sphinxAtStartPar
B0DEC
&
\sphinxAtStartPar
ADTRGB0在偶数周期有效使能                   |

\sphinxAtStartPar
0:禁能                                     |

\sphinxAtStartPar
1:使能                                     |
\\
\sphinxhline
\sphinxAtStartPar
3
&
\sphinxAtStartPar
A1INC
&
\sphinxAtStartPar
ADTRGA1在奇数周期有效使能                   |

\sphinxAtStartPar
0:禁能                                     |

\sphinxAtStartPar
1:使能                                     |
\\
\sphinxhline
\sphinxAtStartPar
2
&
\sphinxAtStartPar
A1DEC
&
\sphinxAtStartPar
ADTRGA1在偶数周期有效使能                   |

\sphinxAtStartPar
0:禁能                                     |

\sphinxAtStartPar
1:使能                                     |
\\
\sphinxhline
\sphinxAtStartPar
1
&
\sphinxAtStartPar
A0INC
&
\sphinxAtStartPar
ADTRGA0在奇数周期有效使能                   |

\sphinxAtStartPar
0:禁能                                     |

\sphinxAtStartPar
1:使能                                     |
\\
\sphinxhline
\sphinxAtStartPar
0
&
\sphinxAtStartPar
A0DEC
&
\sphinxAtStartPar
ADTRGA0在偶数周期有效使能                   |

\sphinxAtStartPar
0:禁能                                     |

\sphinxAtStartPar
1:使能                                     |
\\
\sphinxbottomrule
\end{tabular}
\sphinxtableafterendhook\par
\sphinxattableend\end{savenotes}


\subsubsection{输入脉冲触发沿控制寄存器CONFIG}
\label{\detokenize{SWM241/_u529f_u80fd_u63cf_u8ff0/_u8109_u51b2_u5bbd_u5ea6_u8c03_u5236:config}}

\begin{savenotes}\sphinxattablestart
\sphinxthistablewithglobalstyle
\centering
\begin{tabular}[t]{\X{20}{100}\X{20}{100}\X{20}{100}\X{20}{100}\X{20}{100}}
\sphinxtoprule
\sphinxtableatstartofbodyhook
\sphinxAtStartPar
寄存器 |
&
\begin{DUlineblock}{0em}
\item[] 偏移 |
\end{DUlineblock}
&
\begin{DUlineblock}{0em}
\item[] 
\item[] {\color{red}\bfseries{}|}
\end{DUlineblock}
&
\sphinxAtStartPar
复位值 |    描 | |
&
\begin{DUlineblock}{0em}
\item[] |
  |
\end{DUlineblock}
\\
\sphinxhline
\sphinxAtStartPar
CONFIG
&
\sphinxAtStartPar
0x200
&&
\sphinxAtStartPar
0 000000
&
\sphinxAtStartPar
输入脉冲触发沿控制寄存器   |
\\
\sphinxbottomrule
\end{tabular}
\sphinxtableafterendhook\par
\sphinxattableend\end{savenotes}


\begin{savenotes}\sphinxattablestart
\sphinxthistablewithglobalstyle
\centering
\begin{tabular}[t]{\X{12}{96}\X{12}{96}\X{12}{96}\X{12}{96}\X{12}{96}\X{12}{96}\X{12}{96}\X{12}{96}}
\sphinxtoprule
\sphinxtableatstartofbodyhook
\sphinxAtStartPar
31
&
\sphinxAtStartPar
30
&
\sphinxAtStartPar
29
&
\sphinxAtStartPar
28
&
\sphinxAtStartPar
27
&
\sphinxAtStartPar
26
&
\sphinxAtStartPar
25
&
\sphinxAtStartPar
24
\\
\sphinxhline\begin{itemize}
\item {} 
\end{itemize}
&&&&&&&\\
\sphinxhline
\sphinxAtStartPar
23
&
\sphinxAtStartPar
22
&
\sphinxAtStartPar
21
&
\sphinxAtStartPar
20
&
\sphinxAtStartPar
19
&
\sphinxAtStartPar
18
&
\sphinxAtStartPar
17
&
\sphinxAtStartPar
16
\\
\sphinxhline\begin{itemize}
\item {} 
\end{itemize}
&&&&&&&\\
\sphinxhline
\sphinxAtStartPar
15
&
\sphinxAtStartPar
14
&
\sphinxAtStartPar
13
&
\sphinxAtStartPar
12
&
\sphinxAtStartPar
11
&
\sphinxAtStartPar
10
&
\sphinxAtStartPar
9
&
\sphinxAtStartPar
8
\\
\sphinxhline\begin{itemize}
\item {} 
\end{itemize}
&&&&&&&\\
\sphinxhline
\sphinxAtStartPar
7
&
\sphinxAtStartPar
6
&
\sphinxAtStartPar
5
&
\sphinxAtStartPar
4
&
\sphinxAtStartPar
3
&
\sphinxAtStartPar
2
&
\sphinxAtStartPar
1
&
\sphinxAtStartPar
0
\\
\sphinxhline\begin{itemize}
\item {} 
\end{itemize}
&&&&&\begin{itemize}
\item {} 
\end{itemize}
&&\\
\sphinxbottomrule
\end{tabular}
\sphinxtableafterendhook\par
\sphinxattableend\end{savenotes}


\begin{savenotes}\sphinxattablestart
\sphinxthistablewithglobalstyle
\centering
\begin{tabular}[t]{\X{33}{99}\X{33}{99}\X{33}{99}}
\sphinxtoprule
\sphinxtableatstartofbodyhook
\sphinxAtStartPar
位域 |
&
\sphinxAtStartPar
名称     | |
&
\sphinxAtStartPar
描述                                        | |
\\
\sphinxhline
\sphinxAtStartPar
31:5
&\begin{itemize}
\item {} 
\end{itemize}
&\begin{itemize}
\item {} 
\end{itemize}
\\
\sphinxhline
\sphinxAtStartPar
4
&
\sphinxAtStartPar
P1RISE
&
\sphinxAtStartPar
选择输入Pulse1的触发沿                      |

\sphinxAtStartPar
1:上升沿                                   |

\sphinxAtStartPar
0:下降沿                                   |
\\
\sphinxhline
\sphinxAtStartPar
3
&
\sphinxAtStartPar
P0RISE
&
\sphinxAtStartPar
选择输入Pulse0的触发沿                      |

\sphinxAtStartPar
1:上升沿                                   |

\sphinxAtStartPar
0:下降沿                                   |
\\
\sphinxhline
\sphinxAtStartPar
2:0
&\begin{itemize}
\item {} 
\end{itemize}
&\begin{itemize}
\item {} 
\end{itemize}
\\
\sphinxbottomrule
\end{tabular}
\sphinxtableafterendhook\par
\sphinxattableend\end{savenotes}


\subsubsection{强制输出寄存器FORCEO}
\label{\detokenize{SWM241/_u529f_u80fd_u63cf_u8ff0/_u8109_u51b2_u5bbd_u5ea6_u8c03_u5236:forceo}}

\begin{savenotes}\sphinxattablestart
\sphinxthistablewithglobalstyle
\centering
\begin{tabular}[t]{\X{20}{100}\X{20}{100}\X{20}{100}\X{20}{100}\X{20}{100}}
\sphinxtoprule
\sphinxtableatstartofbodyhook
\sphinxAtStartPar
寄存器 |
&
\begin{DUlineblock}{0em}
\item[] 偏移 |
\end{DUlineblock}
&
\begin{DUlineblock}{0em}
\item[] 
\item[] {\color{red}\bfseries{}|}
\end{DUlineblock}
&
\sphinxAtStartPar
复位值 |    描 | |
&
\begin{DUlineblock}{0em}
\item[] |
  |
\end{DUlineblock}
\\
\sphinxhline
\sphinxAtStartPar
FORCEO
&
\sphinxAtStartPar
0x204
&&
\sphinxAtStartPar
0 000000
&
\sphinxAtStartPar
强制输出寄存器             |
\\
\sphinxbottomrule
\end{tabular}
\sphinxtableafterendhook\par
\sphinxattableend\end{savenotes}


\begin{savenotes}\sphinxattablestart
\sphinxthistablewithglobalstyle
\centering
\begin{tabular}[t]{\X{12}{96}\X{12}{96}\X{12}{96}\X{12}{96}\X{12}{96}\X{12}{96}\X{12}{96}\X{12}{96}}
\sphinxtoprule
\sphinxtableatstartofbodyhook
\sphinxAtStartPar
31
&
\sphinxAtStartPar
30
&
\sphinxAtStartPar
29
&
\sphinxAtStartPar
28
&
\sphinxAtStartPar
27
&
\sphinxAtStartPar
26
&
\sphinxAtStartPar
25
&
\sphinxAtStartPar
24
\\
\sphinxhline\begin{itemize}
\item {} 
\end{itemize}
&&&&
\sphinxAtStartPar
P Bn
&&
\sphinxAtStartPar
P Bn
&\\
\sphinxhline
\sphinxAtStartPar
23
&
\sphinxAtStartPar
22
&
\sphinxAtStartPar
21
&
\sphinxAtStartPar
20
&
\sphinxAtStartPar
19
&
\sphinxAtStartPar
18
&
\sphinxAtStartPar
17
&
\sphinxAtStartPar
16
\\
\sphinxhline\begin{itemize}
\item {} 
\end{itemize}
&&&&&&&\\
\sphinxhline
\sphinxAtStartPar
15
&
\sphinxAtStartPar
14
&
\sphinxAtStartPar
13
&
\sphinxAtStartPar
12
&
\sphinxAtStartPar
11
&
\sphinxAtStartPar
10
&
\sphinxAtStartPar
9
&
\sphinxAtStartPar
8
\\
\sphinxhline\begin{itemize}
\item {} 
\end{itemize}
&&&&
\sphinxAtStartPar
P An
&&
\sphinxAtStartPar
P An
&\\
\sphinxhline
\sphinxAtStartPar
7
&
\sphinxAtStartPar
6
&
\sphinxAtStartPar
5
&
\sphinxAtStartPar
4
&
\sphinxAtStartPar
3
&
\sphinxAtStartPar
2
&
\sphinxAtStartPar
1
&
\sphinxAtStartPar
0
\\
\sphinxhline\begin{itemize}
\item {} 
\end{itemize}
&&&&&&&\\
\sphinxbottomrule
\end{tabular}
\sphinxtableafterendhook\par
\sphinxattableend\end{savenotes}


\begin{savenotes}\sphinxattablestart
\sphinxthistablewithglobalstyle
\centering
\begin{tabular}[t]{\X{33}{99}\X{33}{99}\X{33}{99}}
\sphinxtoprule
\sphinxtableatstartofbodyhook
\sphinxAtStartPar
位域 |
&
\sphinxAtStartPar
名称     | |
&
\sphinxAtStartPar
描述                                        | |
\\
\sphinxhline
\sphinxAtStartPar
31:28
&\begin{itemize}
\item {} 
\end{itemize}
&\begin{itemize}
\item {} 
\end{itemize}
\\
\sphinxhline
\sphinxAtStartPar
27
&
\sphinxAtStartPar
PWM3Bn
&
\sphinxAtStartPar
PWM3Bn强制输出固定电平使能                  |

\sphinxAtStartPar
1:使能                                     |

\sphinxAtStartPar
0:输出正常                                 |
\\
\sphinxhline
\sphinxAtStartPar
26
&
\sphinxAtStartPar
PWM2Bn
&
\sphinxAtStartPar
PWM2Bn强制输出固定电平使能                  |

\sphinxAtStartPar
1:使能                                     |

\sphinxAtStartPar
0:输出正常                                 |
\\
\sphinxhline
\sphinxAtStartPar
25
&
\sphinxAtStartPar
PWM1Bn
&
\sphinxAtStartPar
PWM1Bn强制输出固定电平使能                  |

\sphinxAtStartPar
1:使能                                     |

\sphinxAtStartPar
0:输出正常                                 |
\\
\sphinxhline
\sphinxAtStartPar
24
&
\sphinxAtStartPar
PWM0Bn
&
\sphinxAtStartPar
PWM0Bn强制输出固定电平使能                  |

\sphinxAtStartPar
1:使能                                     |

\sphinxAtStartPar
0:输出正常                                 |
\\
\sphinxhline
\sphinxAtStartPar
23:20
&\begin{itemize}
\item {} 
\end{itemize}
&\begin{itemize}
\item {} 
\end{itemize}
\\
\sphinxhline
\sphinxAtStartPar
19
&
\sphinxAtStartPar
PWM3B
&
\sphinxAtStartPar
PWM3B强制输出固定电平使能                   |

\sphinxAtStartPar
1:使能                                     |

\sphinxAtStartPar
0:输出正常                                 |
\\
\sphinxhline
\sphinxAtStartPar
18
&
\sphinxAtStartPar
PWM2B
&
\sphinxAtStartPar
PWM2B强制输出固定电平使能                   |

\sphinxAtStartPar
1:使能                                     |

\sphinxAtStartPar
0:输出正常                                 |
\\
\sphinxhline
\sphinxAtStartPar
17
&
\sphinxAtStartPar
PWM1B
&
\sphinxAtStartPar
PWM1B强制输出固定电平使能                   |

\sphinxAtStartPar
1:使能                                     |

\sphinxAtStartPar
0:输出正常                                 |
\\
\sphinxhline
\sphinxAtStartPar
16
&
\sphinxAtStartPar
PWM0B
&
\sphinxAtStartPar
PWM0B强制输出固定电平使能                   |

\sphinxAtStartPar
1:使能                                     |

\sphinxAtStartPar
0:输出正常                                 |
\\
\sphinxhline
\sphinxAtStartPar
15:12
&\begin{itemize}
\item {} 
\end{itemize}
&\begin{itemize}
\item {} 
\end{itemize}
\\
\sphinxhline
\sphinxAtStartPar
11
&
\sphinxAtStartPar
PWM3An
&
\sphinxAtStartPar
PWM3An强制输出固定电平使能                  |

\sphinxAtStartPar
1:使能                                     |

\sphinxAtStartPar
0:输出正常                                 |
\\
\sphinxhline
\sphinxAtStartPar
10
&
\sphinxAtStartPar
PWM2An
&
\sphinxAtStartPar
PWM2An强制输出固定电平使能                  |

\sphinxAtStartPar
1:使能                                     |

\sphinxAtStartPar
0:输出正常                                 |
\\
\sphinxhline
\sphinxAtStartPar
9
&
\sphinxAtStartPar
PWM1An
&
\sphinxAtStartPar
PWM1An强制输出固定电平使能                  |

\sphinxAtStartPar
1:使能                                     |

\sphinxAtStartPar
0:输出正常                                 |
\\
\sphinxhline
\sphinxAtStartPar
8
&
\sphinxAtStartPar
PWM0An
&
\sphinxAtStartPar
PWM0An强制输出固定电平使能                  |

\sphinxAtStartPar
1:使能                                     |

\sphinxAtStartPar
0:输出正常                                 |
\\
\sphinxhline
\sphinxAtStartPar
7:4
&\begin{itemize}
\item {} 
\end{itemize}
&\begin{itemize}
\item {} 
\end{itemize}
\\
\sphinxhline
\sphinxAtStartPar
3
&
\sphinxAtStartPar
PWM3A
&
\sphinxAtStartPar
PWM3A强制输出固定电平使能                   |

\sphinxAtStartPar
1:使能                                     |

\sphinxAtStartPar
0:输出正常                                 |
\\
\sphinxhline
\sphinxAtStartPar
2
&
\sphinxAtStartPar
PWM2A
&
\sphinxAtStartPar
PWM2A强制输出固定电平使能                   |

\sphinxAtStartPar
1:使能                                     |

\sphinxAtStartPar
0:输出正常                                 |
\\
\sphinxhline
\sphinxAtStartPar
1
&
\sphinxAtStartPar
PWM1A
&
\sphinxAtStartPar
PWM1A强制输出固定电平使能                   |

\sphinxAtStartPar
1:使能                                     |

\sphinxAtStartPar
0:输出正常                                 |
\\
\sphinxhline
\sphinxAtStartPar
0
&
\sphinxAtStartPar
PWM0A
&
\sphinxAtStartPar
PWM0A强制输出固定电平使能                   |

\sphinxAtStartPar
1:使能                                     |

\sphinxAtStartPar
0:输出正常                                 |
\\
\sphinxbottomrule
\end{tabular}
\sphinxtableafterendhook\par
\sphinxattableend\end{savenotes}


\subsubsection{PWM刹车控制寄存器BRKCR}
\label{\detokenize{SWM241/_u529f_u80fd_u63cf_u8ff0/_u8109_u51b2_u5bbd_u5ea6_u8c03_u5236:pwmbrkcr}}

\begin{savenotes}\sphinxattablestart
\sphinxthistablewithglobalstyle
\centering
\begin{tabular}[t]{\X{20}{100}\X{20}{100}\X{20}{100}\X{20}{100}\X{20}{100}}
\sphinxtoprule
\sphinxtableatstartofbodyhook
\sphinxAtStartPar
寄存器 |
&
\begin{DUlineblock}{0em}
\item[] 偏移 |
\end{DUlineblock}
&
\begin{DUlineblock}{0em}
\item[] 
\item[] {\color{red}\bfseries{}|}
\end{DUlineblock}
&
\sphinxAtStartPar
复位值 |    描 | |
&
\begin{DUlineblock}{0em}
\item[] |
  |
\end{DUlineblock}
\\
\sphinxhline
\sphinxAtStartPar
BRKCR
&
\sphinxAtStartPar
0x208
&&
\sphinxAtStartPar
0 000000
&
\sphinxAtStartPar
刹车控制                   |
\\
\sphinxbottomrule
\end{tabular}
\sphinxtableafterendhook\par
\sphinxattableend\end{savenotes}


\begin{savenotes}\sphinxattablestart
\sphinxthistablewithglobalstyle
\centering
\begin{tabular}[t]{\X{12}{96}\X{12}{96}\X{12}{96}\X{12}{96}\X{12}{96}\X{12}{96}\X{12}{96}\X{12}{96}}
\sphinxtoprule
\sphinxtableatstartofbodyhook
\sphinxAtStartPar
31
&
\sphinxAtStartPar
30
&
\sphinxAtStartPar
29
&
\sphinxAtStartPar
28
&
\sphinxAtStartPar
27
&
\sphinxAtStartPar
26
&
\sphinxAtStartPar
25
&
\sphinxAtStartPar
24
\\
\sphinxhline\begin{itemize}
\item {} 
\end{itemize}
&&&&&&&\\
\sphinxhline
\sphinxAtStartPar
23
&
\sphinxAtStartPar
22
&
\sphinxAtStartPar
21
&
\sphinxAtStartPar
20
&
\sphinxAtStartPar
19
&
\sphinxAtStartPar
18
&
\sphinxAtStartPar
17
&
\sphinxAtStartPar
16
\\
\sphinxhline\begin{itemize}
\item {} 
\end{itemize}
&&&&&&&\\
\sphinxhline
\sphinxAtStartPar
15
&
\sphinxAtStartPar
14
&
\sphinxAtStartPar
13
&
\sphinxAtStartPar
12
&
\sphinxAtStartPar
11
&
\sphinxAtStartPar
10
&
\sphinxAtStartPar
9
&
\sphinxAtStartPar
8
\\
\sphinxhline\begin{itemize}
\item {} 
\end{itemize}
&&&&
\sphinxAtStartPar
S2 OL
&
\sphinxAtStartPar
S POL
&
\sphinxAtStartPar
S0 OL
&
\sphinxAtStartPar
S CLR
\\
\sphinxhline
\sphinxAtStartPar
7
&
\sphinxAtStartPar
6
&
\sphinxAtStartPar
5
&
\sphinxAtStartPar
4
&
\sphinxAtStartPar
3
&
\sphinxAtStartPar
2
&
\sphinxAtStartPar
1
&
\sphinxAtStartPar
0
\\
\sphinxhline
\sphinxAtStartPar
S1STCLR
&
\sphinxAtStartPar
S 0STCLR
&
\sphinxAtStartPar
S2 TE
&
\sphinxAtStartPar
S ATE
&
\sphinxAtStartPar
S0 TE
&\begin{itemize}
\item {} 
\end{itemize}
&&
\sphinxAtStartPar
S CNT
\\
\sphinxbottomrule
\end{tabular}
\sphinxtableafterendhook\par
\sphinxattableend\end{savenotes}


\begin{savenotes}\sphinxattablestart
\sphinxthistablewithglobalstyle
\centering
\begin{tabular}[t]{\X{33}{99}\X{33}{99}\X{33}{99}}
\sphinxtoprule
\sphinxtableatstartofbodyhook
\sphinxAtStartPar
位域 |
&
\sphinxAtStartPar
名称     | |
&
\sphinxAtStartPar
描述                                        | |
\\
\sphinxhline
\sphinxAtStartPar
31:12
&\begin{itemize}
\item {} 
\end{itemize}
&\begin{itemize}
\item {} 
\end{itemize}
\\
\sphinxhline
\sphinxAtStartPar
11
&
\sphinxAtStartPar
S2INPOL
&
\sphinxAtStartPar
BRAKE2输入有效电平                          |

\sphinxAtStartPar
1:高电平有效                               |

\sphinxAtStartPar
0:低电平有效                               |
\\
\sphinxhline
\sphinxAtStartPar
10
&
\sphinxAtStartPar
S1INPOL
&
\sphinxAtStartPar
BRAKE1输入有效电平                          |

\sphinxAtStartPar
1:高电平有效                               |

\sphinxAtStartPar
0:低电平有效                               |
\\
\sphinxhline
\sphinxAtStartPar
9
&
\sphinxAtStartPar
S0INPOL
&
\sphinxAtStartPar
BRAKE0输入有效电平                          |

\sphinxAtStartPar
1:高电平有效                               |

\sphinxAtStartPar
0:低电平有效                               |
\\
\sphinxhline
\sphinxAtStartPar
8
&
\sphinxAtStartPar
S2STCLR
&
\sphinxAtStartPar
BRAKE2状态清除,RO                          |

\sphinxAtStartPar
1:清除状态                                 |

\sphinxAtStartPar
0:保持状态                                 |
\\
\sphinxhline
\sphinxAtStartPar
7
&
\sphinxAtStartPar
S1STCLR
&
\sphinxAtStartPar
BRAKE1状态清除,RO                          |

\sphinxAtStartPar
1:清除状态                                 |

\sphinxAtStartPar
0:保持状态                                 |
\\
\sphinxhline
\sphinxAtStartPar
6
&
\sphinxAtStartPar
S0STCLR
&
\sphinxAtStartPar
BRAKE0状态清除,RO                          |

\sphinxAtStartPar
1:清除状态                                 |

\sphinxAtStartPar
0:保持状态                                 |
\\
\sphinxhline
\sphinxAtStartPar
5
&
\sphinxAtStartPar
S2STATE
&
\sphinxAtStartPar
BRAKE2状态(软件清除),RO                  |

\sphinxAtStartPar
1:正在刹车                                 |

\sphinxAtStartPar
0:未刹车                                   |
\\
\sphinxhline
\sphinxAtStartPar
4
&
\sphinxAtStartPar
S1STATE
&
\sphinxAtStartPar
BRAKE1状态(软件清除),RO                  |

\sphinxAtStartPar
1:正在刹车                                 |

\sphinxAtStartPar
0:未刹车                                   |
\\
\sphinxhline
\sphinxAtStartPar
3
&
\sphinxAtStartPar
S0STATE
&
\sphinxAtStartPar
BRAKE0状态(软件清除),RO                  |

\sphinxAtStartPar
1:正在刹车                                 |

\sphinxAtStartPar
0:未刹车                                   |
\\
\sphinxhline
\sphinxAtStartPar
2:1
&\begin{itemize}
\item {} 
\end{itemize}
&\begin{itemize}
\item {} 
\end{itemize}
\\
\sphinxhline
\sphinxAtStartPar
0
&
\sphinxAtStartPar
STOPCNT
&
\sphinxAtStartPar
1:                                         | 效后PWM清零,停止计数,使能位会自动清除  |

\sphinxAtStartPar
0:刹车生效后PWM继续计数,输出停止          |
\\
\sphinxbottomrule
\end{tabular}
\sphinxtableafterendhook\par
\sphinxattableend\end{savenotes}


\subsubsection{PWM刹车中断使能寄存器BRKIE}
\label{\detokenize{SWM241/_u529f_u80fd_u63cf_u8ff0/_u8109_u51b2_u5bbd_u5ea6_u8c03_u5236:pwmbrkie}}

\begin{savenotes}\sphinxattablestart
\sphinxthistablewithglobalstyle
\centering
\begin{tabular}[t]{\X{20}{100}\X{20}{100}\X{20}{100}\X{20}{100}\X{20}{100}}
\sphinxtoprule
\sphinxtableatstartofbodyhook
\sphinxAtStartPar
寄存器 |
&
\begin{DUlineblock}{0em}
\item[] 偏移 |
\end{DUlineblock}
&
\begin{DUlineblock}{0em}
\item[] 
\item[] {\color{red}\bfseries{}|}
\end{DUlineblock}
&
\sphinxAtStartPar
复位值 |    描 | |
&
\begin{DUlineblock}{0em}
\item[] |
  |
\end{DUlineblock}
\\
\sphinxhline
\sphinxAtStartPar
BRKIE
&
\sphinxAtStartPar
0x20C
&&
\sphinxAtStartPar
0 000000
&
\sphinxAtStartPar
刹车中断使能               |
\\
\sphinxbottomrule
\end{tabular}
\sphinxtableafterendhook\par
\sphinxattableend\end{savenotes}


\begin{savenotes}\sphinxattablestart
\sphinxthistablewithglobalstyle
\centering
\begin{tabular}[t]{\X{12}{96}\X{12}{96}\X{12}{96}\X{12}{96}\X{12}{96}\X{12}{96}\X{12}{96}\X{12}{96}}
\sphinxtoprule
\sphinxtableatstartofbodyhook
\sphinxAtStartPar
31
&
\sphinxAtStartPar
30
&
\sphinxAtStartPar
29
&
\sphinxAtStartPar
28
&
\sphinxAtStartPar
27
&
\sphinxAtStartPar
26
&
\sphinxAtStartPar
25
&
\sphinxAtStartPar
24
\\
\sphinxhline\begin{itemize}
\item {} 
\end{itemize}
&&&&&&&\\
\sphinxhline
\sphinxAtStartPar
23
&
\sphinxAtStartPar
22
&
\sphinxAtStartPar
21
&
\sphinxAtStartPar
20
&
\sphinxAtStartPar
19
&
\sphinxAtStartPar
18
&
\sphinxAtStartPar
17
&
\sphinxAtStartPar
16
\\
\sphinxhline\begin{itemize}
\item {} 
\end{itemize}
&&&&&&&\\
\sphinxhline
\sphinxAtStartPar
15
&
\sphinxAtStartPar
14
&
\sphinxAtStartPar
13
&
\sphinxAtStartPar
12
&
\sphinxAtStartPar
11
&
\sphinxAtStartPar
10
&
\sphinxAtStartPar
9
&
\sphinxAtStartPar
8
\\
\sphinxhline\begin{itemize}
\item {} 
\end{itemize}
&&&&&&&\\
\sphinxhline
\sphinxAtStartPar
7
&
\sphinxAtStartPar
6
&
\sphinxAtStartPar
5
&
\sphinxAtStartPar
4
&
\sphinxAtStartPar
3
&
\sphinxAtStartPar
2
&
\sphinxAtStartPar
1
&
\sphinxAtStartPar
0
\\
\sphinxhline\begin{itemize}
\item {} 
\end{itemize}
&&&&&
\sphinxAtStartPar
S2
&
\sphinxAtStartPar
S1
&
\sphinxAtStartPar
S0
\\
\sphinxbottomrule
\end{tabular}
\sphinxtableafterendhook\par
\sphinxattableend\end{savenotes}


\begin{savenotes}\sphinxattablestart
\sphinxthistablewithglobalstyle
\centering
\begin{tabular}[t]{\X{33}{99}\X{33}{99}\X{33}{99}}
\sphinxtoprule
\sphinxtableatstartofbodyhook
\sphinxAtStartPar
位域 |
&
\sphinxAtStartPar
名称     | |
&
\sphinxAtStartPar
描述                                        | |
\\
\sphinxhline
\sphinxAtStartPar
31:3
&\begin{itemize}
\item {} 
\end{itemize}
&\begin{itemize}
\item {} 
\end{itemize}
\\
\sphinxhline
\sphinxAtStartPar
2
&
\sphinxAtStartPar
S2
&
\sphinxAtStartPar
BRAKE2中断使能                              |

\sphinxAtStartPar
1:使能                                     |

\sphinxAtStartPar
0:禁能                                     |
\\
\sphinxhline
\sphinxAtStartPar
1
&
\sphinxAtStartPar
S1
&
\sphinxAtStartPar
BRAKE1中断使能                              |

\sphinxAtStartPar
1:使能                                     |

\sphinxAtStartPar
0:禁能                                     |
\\
\sphinxhline
\sphinxAtStartPar
0
&
\sphinxAtStartPar
S0
&
\sphinxAtStartPar
BRAKE0中断使能                              |

\sphinxAtStartPar
1:使能                                     |

\sphinxAtStartPar
0:禁能                                     |
\\
\sphinxbottomrule
\end{tabular}
\sphinxtableafterendhook\par
\sphinxattableend\end{savenotes}


\subsubsection{PWM刹车中断状态寄存器BRKIF}
\label{\detokenize{SWM241/_u529f_u80fd_u63cf_u8ff0/_u8109_u51b2_u5bbd_u5ea6_u8c03_u5236:pwmbrkif}}

\begin{savenotes}\sphinxattablestart
\sphinxthistablewithglobalstyle
\centering
\begin{tabular}[t]{\X{20}{100}\X{20}{100}\X{20}{100}\X{20}{100}\X{20}{100}}
\sphinxtoprule
\sphinxtableatstartofbodyhook
\sphinxAtStartPar
寄存器 |
&
\begin{DUlineblock}{0em}
\item[] 偏移 |
\end{DUlineblock}
&
\begin{DUlineblock}{0em}
\item[] 
\item[] {\color{red}\bfseries{}|}
\end{DUlineblock}
&
\sphinxAtStartPar
复位值 |    描 | |
&
\begin{DUlineblock}{0em}
\item[] |
  |
\end{DUlineblock}
\\
\sphinxhline
\sphinxAtStartPar
BRKIF
&
\sphinxAtStartPar
0x210
&&
\sphinxAtStartPar
0 000000
&
\sphinxAtStartPar
刹车中断状态               |
\\
\sphinxbottomrule
\end{tabular}
\sphinxtableafterendhook\par
\sphinxattableend\end{savenotes}


\begin{savenotes}\sphinxattablestart
\sphinxthistablewithglobalstyle
\centering
\begin{tabular}[t]{\X{12}{96}\X{12}{96}\X{12}{96}\X{12}{96}\X{12}{96}\X{12}{96}\X{12}{96}\X{12}{96}}
\sphinxtoprule
\sphinxtableatstartofbodyhook
\sphinxAtStartPar
31
&
\sphinxAtStartPar
30
&
\sphinxAtStartPar
29
&
\sphinxAtStartPar
28
&
\sphinxAtStartPar
27
&
\sphinxAtStartPar
26
&
\sphinxAtStartPar
25
&
\sphinxAtStartPar
24
\\
\sphinxhline\begin{itemize}
\item {} 
\end{itemize}
&&&&&&&\\
\sphinxhline
\sphinxAtStartPar
23
&
\sphinxAtStartPar
22
&
\sphinxAtStartPar
21
&
\sphinxAtStartPar
20
&
\sphinxAtStartPar
19
&
\sphinxAtStartPar
18
&
\sphinxAtStartPar
17
&
\sphinxAtStartPar
16
\\
\sphinxhline\begin{itemize}
\item {} 
\end{itemize}
&&&&&&&\\
\sphinxhline
\sphinxAtStartPar
15
&
\sphinxAtStartPar
14
&
\sphinxAtStartPar
13
&
\sphinxAtStartPar
12
&
\sphinxAtStartPar
11
&
\sphinxAtStartPar
10
&
\sphinxAtStartPar
9
&
\sphinxAtStartPar
8
\\
\sphinxhline\begin{itemize}
\item {} 
\end{itemize}
&&&&&&&\\
\sphinxhline
\sphinxAtStartPar
7
&
\sphinxAtStartPar
6
&
\sphinxAtStartPar
5
&
\sphinxAtStartPar
4
&
\sphinxAtStartPar
3
&
\sphinxAtStartPar
2
&
\sphinxAtStartPar
1
&
\sphinxAtStartPar
0
\\
\sphinxhline\begin{itemize}
\item {} 
\end{itemize}
&&&&&
\sphinxAtStartPar
S2
&
\sphinxAtStartPar
S1
&
\sphinxAtStartPar
S0
\\
\sphinxbottomrule
\end{tabular}
\sphinxtableafterendhook\par
\sphinxattableend\end{savenotes}


\begin{savenotes}\sphinxattablestart
\sphinxthistablewithglobalstyle
\centering
\begin{tabular}[t]{\X{33}{99}\X{33}{99}\X{33}{99}}
\sphinxtoprule
\sphinxtableatstartofbodyhook
\sphinxAtStartPar
位域 |
&
\sphinxAtStartPar
名称     | |
&
\sphinxAtStartPar
描述                                        | |
\\
\sphinxhline
\sphinxAtStartPar
31:3
&\begin{itemize}
\item {} 
\end{itemize}
&\begin{itemize}
\item {} 
\end{itemize}
\\
\sphinxhline
\sphinxAtStartPar
2
&
\sphinxAtStartPar
S2
&
\sphinxAtStartPar
BRAKE2中断状态,R/W1C                       |

\sphinxAtStartPar
1:已发生                                   |

\sphinxAtStartPar
0:未发生                                   |
\\
\sphinxhline
\sphinxAtStartPar
1
&
\sphinxAtStartPar
S1
&
\sphinxAtStartPar
BRAKE1中断状态,R/W1C                       |

\sphinxAtStartPar
1:已发生                                   |

\sphinxAtStartPar
0:未发生                                   |
\\
\sphinxhline
\sphinxAtStartPar
0
&
\sphinxAtStartPar
S0
&
\sphinxAtStartPar
BRAKE0中断状态,R/W1C                       |

\sphinxAtStartPar
1:已发生                                   |

\sphinxAtStartPar
0:未发生                                   |
\\
\sphinxbottomrule
\end{tabular}
\sphinxtableafterendhook\par
\sphinxattableend\end{savenotes}


\subsubsection{PWM刹车中断屏蔽寄存器BRKIM}
\label{\detokenize{SWM241/_u529f_u80fd_u63cf_u8ff0/_u8109_u51b2_u5bbd_u5ea6_u8c03_u5236:pwmbrkim}}

\begin{savenotes}\sphinxattablestart
\sphinxthistablewithglobalstyle
\centering
\begin{tabular}[t]{\X{20}{100}\X{20}{100}\X{20}{100}\X{20}{100}\X{20}{100}}
\sphinxtoprule
\sphinxtableatstartofbodyhook
\sphinxAtStartPar
寄存器 |
&
\begin{DUlineblock}{0em}
\item[] 偏移 |
\end{DUlineblock}
&
\begin{DUlineblock}{0em}
\item[] 
\item[] {\color{red}\bfseries{}|}
\end{DUlineblock}
&
\sphinxAtStartPar
复位值 |    描 | |
&
\begin{DUlineblock}{0em}
\item[] |
  |
\end{DUlineblock}
\\
\sphinxhline
\sphinxAtStartPar
BRKIM
&
\sphinxAtStartPar
0x214
&&
\sphinxAtStartPar
0 000000
&
\sphinxAtStartPar
刹车中断屏蔽               |
\\
\sphinxbottomrule
\end{tabular}
\sphinxtableafterendhook\par
\sphinxattableend\end{savenotes}


\begin{savenotes}\sphinxattablestart
\sphinxthistablewithglobalstyle
\centering
\begin{tabular}[t]{\X{12}{96}\X{12}{96}\X{12}{96}\X{12}{96}\X{12}{96}\X{12}{96}\X{12}{96}\X{12}{96}}
\sphinxtoprule
\sphinxtableatstartofbodyhook
\sphinxAtStartPar
31
&
\sphinxAtStartPar
30
&
\sphinxAtStartPar
29
&
\sphinxAtStartPar
28
&
\sphinxAtStartPar
27
&
\sphinxAtStartPar
26
&
\sphinxAtStartPar
25
&
\sphinxAtStartPar
24
\\
\sphinxhline\begin{itemize}
\item {} 
\end{itemize}
&&&&&&&\\
\sphinxhline
\sphinxAtStartPar
23
&
\sphinxAtStartPar
22
&
\sphinxAtStartPar
21
&
\sphinxAtStartPar
20
&
\sphinxAtStartPar
19
&
\sphinxAtStartPar
18
&
\sphinxAtStartPar
17
&
\sphinxAtStartPar
16
\\
\sphinxhline\begin{itemize}
\item {} 
\end{itemize}
&&&&&&&\\
\sphinxhline
\sphinxAtStartPar
15
&
\sphinxAtStartPar
14
&
\sphinxAtStartPar
13
&
\sphinxAtStartPar
12
&
\sphinxAtStartPar
11
&
\sphinxAtStartPar
10
&
\sphinxAtStartPar
9
&
\sphinxAtStartPar
8
\\
\sphinxhline\begin{itemize}
\item {} 
\end{itemize}
&&&&&&&\\
\sphinxhline
\sphinxAtStartPar
7
&
\sphinxAtStartPar
6
&
\sphinxAtStartPar
5
&
\sphinxAtStartPar
4
&
\sphinxAtStartPar
3
&
\sphinxAtStartPar
2
&
\sphinxAtStartPar
1
&
\sphinxAtStartPar
0
\\
\sphinxhline\begin{itemize}
\item {} 
\end{itemize}
&&&&&
\sphinxAtStartPar
S2
&
\sphinxAtStartPar
S1
&
\sphinxAtStartPar
S0
\\
\sphinxbottomrule
\end{tabular}
\sphinxtableafterendhook\par
\sphinxattableend\end{savenotes}


\begin{savenotes}\sphinxattablestart
\sphinxthistablewithglobalstyle
\centering
\begin{tabular}[t]{\X{33}{99}\X{33}{99}\X{33}{99}}
\sphinxtoprule
\sphinxtableatstartofbodyhook
\sphinxAtStartPar
位域 |
&
\sphinxAtStartPar
名称     | |
&
\sphinxAtStartPar
描述                                        | |
\\
\sphinxhline
\sphinxAtStartPar
31:3
&\begin{itemize}
\item {} 
\end{itemize}
&\begin{itemize}
\item {} 
\end{itemize}
\\
\sphinxhline
\sphinxAtStartPar
2
&
\sphinxAtStartPar
S2
&
\sphinxAtStartPar
BRAKE2中断屏蔽                              |

\sphinxAtStartPar
1:屏蔽                                     |

\sphinxAtStartPar
0:非屏蔽                                   |
\\
\sphinxhline
\sphinxAtStartPar
1
&
\sphinxAtStartPar
S1
&
\sphinxAtStartPar
BRAKE1中断屏蔽                              |

\sphinxAtStartPar
1:屏蔽                                     |

\sphinxAtStartPar
0:非屏蔽                                   |
\\
\sphinxhline
\sphinxAtStartPar
0
&
\sphinxAtStartPar
S0
&
\sphinxAtStartPar
BRAKE0中断屏蔽                              |

\sphinxAtStartPar
1:屏蔽                                     |

\sphinxAtStartPar
0:非屏蔽                                   |
\\
\sphinxbottomrule
\end{tabular}
\sphinxtableafterendhook\par
\sphinxattableend\end{savenotes}


\subsubsection{PWM刹车中断原始状态寄存器BRKIRS}
\label{\detokenize{SWM241/_u529f_u80fd_u63cf_u8ff0/_u8109_u51b2_u5bbd_u5ea6_u8c03_u5236:pwmbrkirs}}

\begin{savenotes}\sphinxattablestart
\sphinxthistablewithglobalstyle
\centering
\begin{tabular}[t]{\X{20}{100}\X{20}{100}\X{20}{100}\X{20}{100}\X{20}{100}}
\sphinxtoprule
\sphinxtableatstartofbodyhook
\sphinxAtStartPar
寄存器 |
&
\begin{DUlineblock}{0em}
\item[] 偏移 |
\end{DUlineblock}
&
\begin{DUlineblock}{0em}
\item[] 
\item[] {\color{red}\bfseries{}|}
\end{DUlineblock}
&
\sphinxAtStartPar
复位值 |    描 | |
&
\begin{DUlineblock}{0em}
\item[] |
  |
\end{DUlineblock}
\\
\sphinxhline
\sphinxAtStartPar
BRKIRS
&
\sphinxAtStartPar
0x218
&&
\sphinxAtStartPar
0 000000
&
\sphinxAtStartPar
刹车中断原始状态           |
\\
\sphinxbottomrule
\end{tabular}
\sphinxtableafterendhook\par
\sphinxattableend\end{savenotes}


\begin{savenotes}\sphinxattablestart
\sphinxthistablewithglobalstyle
\centering
\begin{tabular}[t]{\X{12}{96}\X{12}{96}\X{12}{96}\X{12}{96}\X{12}{96}\X{12}{96}\X{12}{96}\X{12}{96}}
\sphinxtoprule
\sphinxtableatstartofbodyhook
\sphinxAtStartPar
31
&
\sphinxAtStartPar
30
&
\sphinxAtStartPar
29
&
\sphinxAtStartPar
28
&
\sphinxAtStartPar
27
&
\sphinxAtStartPar
26
&
\sphinxAtStartPar
25
&
\sphinxAtStartPar
24
\\
\sphinxhline\begin{itemize}
\item {} 
\end{itemize}
&&&&&&&\\
\sphinxhline
\sphinxAtStartPar
23
&
\sphinxAtStartPar
22
&
\sphinxAtStartPar
21
&
\sphinxAtStartPar
20
&
\sphinxAtStartPar
19
&
\sphinxAtStartPar
18
&
\sphinxAtStartPar
17
&
\sphinxAtStartPar
16
\\
\sphinxhline\begin{itemize}
\item {} 
\end{itemize}
&&&&&&&\\
\sphinxhline
\sphinxAtStartPar
15
&
\sphinxAtStartPar
14
&
\sphinxAtStartPar
13
&
\sphinxAtStartPar
12
&
\sphinxAtStartPar
11
&
\sphinxAtStartPar
10
&
\sphinxAtStartPar
9
&
\sphinxAtStartPar
8
\\
\sphinxhline\begin{itemize}
\item {} 
\end{itemize}
&&&&&&&\\
\sphinxhline
\sphinxAtStartPar
7
&
\sphinxAtStartPar
6
&
\sphinxAtStartPar
5
&
\sphinxAtStartPar
4
&
\sphinxAtStartPar
3
&
\sphinxAtStartPar
2
&
\sphinxAtStartPar
1
&
\sphinxAtStartPar
0
\\
\sphinxhline\begin{itemize}
\item {} 
\end{itemize}
&&&&&
\sphinxAtStartPar
S2
&
\sphinxAtStartPar
S1
&
\sphinxAtStartPar
S0
\\
\sphinxbottomrule
\end{tabular}
\sphinxtableafterendhook\par
\sphinxattableend\end{savenotes}


\begin{savenotes}\sphinxattablestart
\sphinxthistablewithglobalstyle
\centering
\begin{tabular}[t]{\X{33}{99}\X{33}{99}\X{33}{99}}
\sphinxtoprule
\sphinxtableatstartofbodyhook
\sphinxAtStartPar
位域 |
&
\sphinxAtStartPar
名称     | |
&
\sphinxAtStartPar
描述                                        | |
\\
\sphinxhline
\sphinxAtStartPar
31:3
&\begin{itemize}
\item {} 
\end{itemize}
&\begin{itemize}
\item {} 
\end{itemize}
\\
\sphinxhline
\sphinxAtStartPar
2
&
\sphinxAtStartPar
S2
&
\sphinxAtStartPar
BRAKE2原始中断状态,R/W1C                   |

\sphinxAtStartPar
1:已发生                                   |

\sphinxAtStartPar
0:未发生                                   |
\\
\sphinxhline
\sphinxAtStartPar
1
&
\sphinxAtStartPar
S1
&
\sphinxAtStartPar
BRAKE1原始中断状态,R/W1C                   |

\sphinxAtStartPar
1:已发生                                   |

\sphinxAtStartPar
0:未发生                                   |
\\
\sphinxhline
\sphinxAtStartPar
0
&
\sphinxAtStartPar
S0
&
\sphinxAtStartPar
BRAKE0原始中断状态,R/W1C                   |

\sphinxAtStartPar
1:已发生                                   |

\sphinxAtStartPar
0:未发生                                   |
\\
\sphinxbottomrule
\end{tabular}
\sphinxtableafterendhook\par
\sphinxattableend\end{savenotes}


\subsubsection{PWM中断使能寄存器IE}
\label{\detokenize{SWM241/_u529f_u80fd_u63cf_u8ff0/_u8109_u51b2_u5bbd_u5ea6_u8c03_u5236:pwmie}}

\begin{savenotes}\sphinxattablestart
\sphinxthistablewithglobalstyle
\centering
\begin{tabular}[t]{\X{20}{100}\X{20}{100}\X{20}{100}\X{20}{100}\X{20}{100}}
\sphinxtoprule
\sphinxtableatstartofbodyhook
\sphinxAtStartPar
寄存器 |
&
\begin{DUlineblock}{0em}
\item[] 偏移 |
\end{DUlineblock}
&
\begin{DUlineblock}{0em}
\item[] 
\item[] {\color{red}\bfseries{}|}
\end{DUlineblock}
&
\sphinxAtStartPar
复位值 |    描 | |
&
\begin{DUlineblock}{0em}
\item[] |
  |
\end{DUlineblock}
\\
\sphinxhline
\sphinxAtStartPar
IE
&
\sphinxAtStartPar
0x21C
&&
\sphinxAtStartPar
0 000000
&
\sphinxAtStartPar
中断使能                   |
\\
\sphinxbottomrule
\end{tabular}
\sphinxtableafterendhook\par
\sphinxattableend\end{savenotes}


\begin{savenotes}\sphinxattablestart
\sphinxthistablewithglobalstyle
\centering
\begin{tabular}[t]{\X{12}{96}\X{12}{96}\X{12}{96}\X{12}{96}\X{12}{96}\X{12}{96}\X{12}{96}\X{12}{96}}
\sphinxtoprule
\sphinxtableatstartofbodyhook
\sphinxAtStartPar
31
&
\sphinxAtStartPar
30
&
\sphinxAtStartPar
29
&
\sphinxAtStartPar
28
&
\sphinxAtStartPar
27
&
\sphinxAtStartPar
26
&
\sphinxAtStartPar
25
&
\sphinxAtStartPar
24
\\
\sphinxhline\begin{itemize}
\item {} 
\end{itemize}
&&&&&&&\\
\sphinxhline
\sphinxAtStartPar
23
&
\sphinxAtStartPar
22
&
\sphinxAtStartPar
21
&
\sphinxAtStartPar
20
&
\sphinxAtStartPar
19
&
\sphinxAtStartPar
18
&
\sphinxAtStartPar
17
&
\sphinxAtStartPar
16
\\
\sphinxhline
\sphinxAtStartPar
HE3B
&
\sphinxAtStartPar
HE3A
&&&&&&\\
\sphinxhline
\sphinxAtStartPar
15
&
\sphinxAtStartPar
14
&
\sphinxAtStartPar
13
&
\sphinxAtStartPar
12
&
\sphinxAtStartPar
11
&
\sphinxAtStartPar
10
&
\sphinxAtStartPar
9
&
\sphinxAtStartPar
8
\\
\sphinxhline\begin{itemize}
\item {} 
\end{itemize}
&&&&&&&\\
\sphinxhline
\sphinxAtStartPar
7
&
\sphinxAtStartPar
6
&
\sphinxAtStartPar
5
&
\sphinxAtStartPar
4
&
\sphinxAtStartPar
3
&
\sphinxAtStartPar
2
&
\sphinxAtStartPar
1
&
\sphinxAtStartPar
0
\\
\sphinxhline
\sphinxAtStartPar
NC3B
&
\sphinxAtStartPar
NC3A
&&&&&&\\
\sphinxbottomrule
\end{tabular}
\sphinxtableafterendhook\par
\sphinxattableend\end{savenotes}


\begin{savenotes}\sphinxattablestart
\sphinxthistablewithglobalstyle
\centering
\begin{tabular}[t]{\X{33}{99}\X{33}{99}\X{33}{99}}
\sphinxtoprule
\sphinxtableatstartofbodyhook
\sphinxAtStartPar
位域 |
&
\sphinxAtStartPar
名称     | |
&
\sphinxAtStartPar
描述                                        | |
\\
\sphinxhline
\sphinxAtStartPar
31:24
&\begin{itemize}
\item {} 
\end{itemize}
&\begin{itemize}
\item {} 
\end{itemize}
\\
\sphinxhline
\sphinxAtStartPar
23
&
\sphinxAtStartPar
HE3B
&
\sphinxAtStartPar
第3组B路高电平结束中断使能                  |

\sphinxAtStartPar
1:使能                                     |

\sphinxAtStartPar
0:禁能                                     |
\\
\sphinxhline
\sphinxAtStartPar
22
&
\sphinxAtStartPar
HE3A
&
\sphinxAtStartPar
第3组A路高电平结束中断使能                  |

\sphinxAtStartPar
1:使能                                     |

\sphinxAtStartPar
0:禁能                                     |
\\
\sphinxhline
\sphinxAtStartPar
21
&
\sphinxAtStartPar
HE2B
&
\sphinxAtStartPar
第2组B路高电平结束中断使能                  |

\sphinxAtStartPar
1:使能                                     |

\sphinxAtStartPar
0:禁能                                     |
\\
\sphinxhline
\sphinxAtStartPar
20
&
\sphinxAtStartPar
HE2A
&
\sphinxAtStartPar
第2组A路高电平结束中断使能                  |

\sphinxAtStartPar
1:使能                                     |

\sphinxAtStartPar
0:禁能                                     |
\\
\sphinxhline
\sphinxAtStartPar
19
&
\sphinxAtStartPar
HE1B
&
\sphinxAtStartPar
第1组B路高电平结束中断使能                  |

\sphinxAtStartPar
1:使能                                     |

\sphinxAtStartPar
0:禁能                                     |
\\
\sphinxhline
\sphinxAtStartPar
18
&
\sphinxAtStartPar
HE1A
&
\sphinxAtStartPar
第1组A路高电平结束中断使能                  |

\sphinxAtStartPar
1:使能                                     |

\sphinxAtStartPar
0:禁能                                     |
\\
\sphinxhline
\sphinxAtStartPar
17
&
\sphinxAtStartPar
HE0B
&
\sphinxAtStartPar
第0组B路高电平结束中断使能                  |

\sphinxAtStartPar
1:使能                                     |

\sphinxAtStartPar
0:禁能                                     |
\\
\sphinxhline
\sphinxAtStartPar
16
&
\sphinxAtStartPar
HE0A
&
\sphinxAtStartPar
第0组A路高电平结束中断使能                  |

\sphinxAtStartPar
1:使能                                     |

\sphinxAtStartPar
0:禁能                                     |
\\
\sphinxhline
\sphinxAtStartPar
15:8
&\begin{itemize}
\item {} 
\end{itemize}
&\begin{itemize}
\item {} 
\end{itemize}
\\
\sphinxhline
\sphinxAtStartPar
7
&
\sphinxAtStartPar
NC3B
&
\sphinxAtStartPar
第3组B路新周期中断使能                      |

\sphinxAtStartPar
1:使能                                     |

\sphinxAtStartPar
0:禁能                                     |
\\
\sphinxhline
\sphinxAtStartPar
6
&
\sphinxAtStartPar
NC3A
&
\sphinxAtStartPar
第3组A路新周期中断使能                      |

\sphinxAtStartPar
1:使能                                     |

\sphinxAtStartPar
0:禁能                                     |
\\
\sphinxhline
\sphinxAtStartPar
5
&
\sphinxAtStartPar
NC2B
&
\sphinxAtStartPar
第2组B路新周期中断使能                      |

\sphinxAtStartPar
1:使能                                     |

\sphinxAtStartPar
0:禁能                                     |
\\
\sphinxhline
\sphinxAtStartPar
4
&
\sphinxAtStartPar
NC2A
&
\sphinxAtStartPar
第2组A路新周期中断使能                      |

\sphinxAtStartPar
1:使能                                     |

\sphinxAtStartPar
0:禁能                                     |
\\
\sphinxhline
\sphinxAtStartPar
3
&
\sphinxAtStartPar
NC1B
&
\sphinxAtStartPar
第1组B路新周期中断使能                      |

\sphinxAtStartPar
1:使能                                     |

\sphinxAtStartPar
0:禁能                                     |
\\
\sphinxhline
\sphinxAtStartPar
2
&
\sphinxAtStartPar
NC1A
&
\sphinxAtStartPar
第1组A路新周期中断使能                      |

\sphinxAtStartPar
1:使能                                     |

\sphinxAtStartPar
0:禁能                                     |
\\
\sphinxhline
\sphinxAtStartPar
1
&
\sphinxAtStartPar
NC0B
&
\sphinxAtStartPar
第0组B路新周期中断使能                      |

\sphinxAtStartPar
1:使能                                     |

\sphinxAtStartPar
0:禁能                                     |
\\
\sphinxhline
\sphinxAtStartPar
0
&
\sphinxAtStartPar
NC0A
&
\sphinxAtStartPar
第0组A路新周期中断使能                      |

\sphinxAtStartPar
1:使能                                     |

\sphinxAtStartPar
0:禁能                                     |
\\
\sphinxbottomrule
\end{tabular}
\sphinxtableafterendhook\par
\sphinxattableend\end{savenotes}


\subsubsection{PWM使能控制寄存器CHEN}
\label{\detokenize{SWM241/_u529f_u80fd_u63cf_u8ff0/_u8109_u51b2_u5bbd_u5ea6_u8c03_u5236:pwmchen}}

\begin{savenotes}\sphinxattablestart
\sphinxthistablewithglobalstyle
\centering
\begin{tabular}[t]{\X{20}{100}\X{20}{100}\X{20}{100}\X{20}{100}\X{20}{100}}
\sphinxtoprule
\sphinxtableatstartofbodyhook
\sphinxAtStartPar
寄存器 |
&
\begin{DUlineblock}{0em}
\item[] 偏移 |
\end{DUlineblock}
&
\begin{DUlineblock}{0em}
\item[] 
\item[] {\color{red}\bfseries{}|}
\end{DUlineblock}
&
\sphinxAtStartPar
复位值 |    描 | |
&
\begin{DUlineblock}{0em}
\item[] |
  |
\end{DUlineblock}
\\
\sphinxhline
\sphinxAtStartPar
CHEN
&
\sphinxAtStartPar
0x220
&&
\sphinxAtStartPar
0 000000
&
\sphinxAtStartPar
PWM输出使能                |
\\
\sphinxbottomrule
\end{tabular}
\sphinxtableafterendhook\par
\sphinxattableend\end{savenotes}


\begin{savenotes}\sphinxattablestart
\sphinxthistablewithglobalstyle
\centering
\begin{tabular}[t]{\X{12}{96}\X{12}{96}\X{12}{96}\X{12}{96}\X{12}{96}\X{12}{96}\X{12}{96}\X{12}{96}}
\sphinxtoprule
\sphinxtableatstartofbodyhook
\sphinxAtStartPar
31
&
\sphinxAtStartPar
30
&
\sphinxAtStartPar
29
&
\sphinxAtStartPar
28
&
\sphinxAtStartPar
27
&
\sphinxAtStartPar
26
&
\sphinxAtStartPar
25
&
\sphinxAtStartPar
24
\\
\sphinxhline\begin{itemize}
\item {} 
\end{itemize}
&&&&&&&\\
\sphinxhline
\sphinxAtStartPar
23
&
\sphinxAtStartPar
22
&
\sphinxAtStartPar
21
&
\sphinxAtStartPar
20
&
\sphinxAtStartPar
19
&
\sphinxAtStartPar
18
&
\sphinxAtStartPar
17
&
\sphinxAtStartPar
16
\\
\sphinxhline\begin{itemize}
\item {} 
\end{itemize}
&&&&&&&\\
\sphinxhline
\sphinxAtStartPar
15
&
\sphinxAtStartPar
14
&
\sphinxAtStartPar
13
&
\sphinxAtStartPar
12
&
\sphinxAtStartPar
11
&
\sphinxAtStartPar
10
&
\sphinxAtStartPar
9
&
\sphinxAtStartPar
8
\\
\sphinxhline\begin{itemize}
\item {} 
\end{itemize}
&&&&&&&\\
\sphinxhline
\sphinxAtStartPar
7
&
\sphinxAtStartPar
6
&
\sphinxAtStartPar
5
&
\sphinxAtStartPar
4
&
\sphinxAtStartPar
3
&
\sphinxAtStartPar
2
&
\sphinxAtStartPar
1
&
\sphinxAtStartPar
0
\\
\sphinxhline\begin{itemize}
\item {} 
\end{itemize}
&&&&&&&\\
\sphinxbottomrule
\end{tabular}
\sphinxtableafterendhook\par
\sphinxattableend\end{savenotes}


\begin{savenotes}\sphinxattablestart
\sphinxthistablewithglobalstyle
\centering
\begin{tabular}[t]{\X{33}{99}\X{33}{99}\X{33}{99}}
\sphinxtoprule
\sphinxtableatstartofbodyhook
\sphinxAtStartPar
位域 |
&
\sphinxAtStartPar
名称     | |
&
\sphinxAtStartPar
描述                                        | |
\\
\sphinxhline
\sphinxAtStartPar
31:12
&\begin{itemize}
\item {} 
\end{itemize}
&\begin{itemize}
\item {} 
\end{itemize}
\\
\sphinxhline
\sphinxAtStartPar
11
&
\sphinxAtStartPar
PWM3B
&
\sphinxAtStartPar
第3组B路PWM使能                             |

\sphinxAtStartPar
1:使能                                     |

\sphinxAtStartPar
0:禁能                                     |
\\
\sphinxhline
\sphinxAtStartPar
10
&
\sphinxAtStartPar
PWM2B
&
\sphinxAtStartPar
第2组B路PWM使能                             |

\sphinxAtStartPar
1:使能                                     |

\sphinxAtStartPar
0:禁能                                     |
\\
\sphinxhline
\sphinxAtStartPar
9
&
\sphinxAtStartPar
PWM1B
&
\sphinxAtStartPar
第1组B路PWM使能                             |

\sphinxAtStartPar
1:使能                                     |

\sphinxAtStartPar
0:禁能                                     |
\\
\sphinxhline
\sphinxAtStartPar
8
&
\sphinxAtStartPar
PWM0B
&
\sphinxAtStartPar
第0组B路PWM使能                             |

\sphinxAtStartPar
1:使能                                     |

\sphinxAtStartPar
0:禁能                                     |
\\
\sphinxhline
\sphinxAtStartPar
7:4
&\begin{itemize}
\item {} 
\end{itemize}
&\begin{itemize}
\item {} 
\end{itemize}
\\
\sphinxhline
\sphinxAtStartPar
3
&
\sphinxAtStartPar
PWM3A
&
\sphinxAtStartPar
第3组A路PWM使能                             |

\sphinxAtStartPar
1:使能                                     |

\sphinxAtStartPar
0:禁能                                     |
\\
\sphinxhline
\sphinxAtStartPar
2
&
\sphinxAtStartPar
PWM2A
&
\sphinxAtStartPar
第2组A路PWM使能                             |

\sphinxAtStartPar
1:使能                                     |

\sphinxAtStartPar
0:禁能                                     |
\\
\sphinxhline
\sphinxAtStartPar
1
&
\sphinxAtStartPar
PWM1A
&
\sphinxAtStartPar
第1组A路PWM使能                             |

\sphinxAtStartPar
1:使能                                     |

\sphinxAtStartPar
0:禁能                                     |
\\
\sphinxhline
\sphinxAtStartPar
0
&
\sphinxAtStartPar
PWM0A
&
\sphinxAtStartPar
第0组A路PWM使能                             |

\sphinxAtStartPar
1:使能                                     |

\sphinxAtStartPar
0:禁能                                     |
\\
\sphinxbottomrule
\end{tabular}
\sphinxtableafterendhook\par
\sphinxattableend\end{savenotes}


\subsubsection{PWM中断屏蔽寄存器IM}
\label{\detokenize{SWM241/_u529f_u80fd_u63cf_u8ff0/_u8109_u51b2_u5bbd_u5ea6_u8c03_u5236:pwmim}}

\begin{savenotes}\sphinxattablestart
\sphinxthistablewithglobalstyle
\centering
\begin{tabular}[t]{\X{20}{100}\X{20}{100}\X{20}{100}\X{20}{100}\X{20}{100}}
\sphinxtoprule
\sphinxtableatstartofbodyhook
\sphinxAtStartPar
寄存器 |
&
\begin{DUlineblock}{0em}
\item[] 偏移 |
\end{DUlineblock}
&
\begin{DUlineblock}{0em}
\item[] 
\item[] {\color{red}\bfseries{}|}
\end{DUlineblock}
&
\sphinxAtStartPar
复位值 |    描 | |
&
\begin{DUlineblock}{0em}
\item[] |
  |
\end{DUlineblock}
\\
\sphinxhline
\sphinxAtStartPar
IM
&
\sphinxAtStartPar
0x224
&&
\sphinxAtStartPar
0 000000
&
\sphinxAtStartPar
中断屏蔽寄存器             |
\\
\sphinxbottomrule
\end{tabular}
\sphinxtableafterendhook\par
\sphinxattableend\end{savenotes}


\begin{savenotes}\sphinxattablestart
\sphinxthistablewithglobalstyle
\centering
\begin{tabular}[t]{\X{12}{96}\X{12}{96}\X{12}{96}\X{12}{96}\X{12}{96}\X{12}{96}\X{12}{96}\X{12}{96}}
\sphinxtoprule
\sphinxtableatstartofbodyhook
\sphinxAtStartPar
31
&
\sphinxAtStartPar
30
&
\sphinxAtStartPar
29
&
\sphinxAtStartPar
28
&
\sphinxAtStartPar
27
&
\sphinxAtStartPar
26
&
\sphinxAtStartPar
25
&
\sphinxAtStartPar
24
\\
\sphinxhline\begin{itemize}
\item {} 
\end{itemize}
&&&&&&&\\
\sphinxhline
\sphinxAtStartPar
23
&
\sphinxAtStartPar
22
&
\sphinxAtStartPar
21
&
\sphinxAtStartPar
20
&
\sphinxAtStartPar
19
&
\sphinxAtStartPar
18
&
\sphinxAtStartPar
17
&
\sphinxAtStartPar
16
\\
\sphinxhline
\sphinxAtStartPar
HE3B
&
\sphinxAtStartPar
HE3A
&&&&&&\\
\sphinxhline
\sphinxAtStartPar
15
&
\sphinxAtStartPar
14
&
\sphinxAtStartPar
13
&
\sphinxAtStartPar
12
&
\sphinxAtStartPar
11
&
\sphinxAtStartPar
10
&
\sphinxAtStartPar
9
&
\sphinxAtStartPar
8
\\
\sphinxhline\begin{itemize}
\item {} 
\end{itemize}
&&&&&&&\\
\sphinxhline
\sphinxAtStartPar
7
&
\sphinxAtStartPar
6
&
\sphinxAtStartPar
5
&
\sphinxAtStartPar
4
&
\sphinxAtStartPar
3
&
\sphinxAtStartPar
2
&
\sphinxAtStartPar
1
&
\sphinxAtStartPar
0
\\
\sphinxhline
\sphinxAtStartPar
NC3B
&
\sphinxAtStartPar
NC3A
&&&&&&\\
\sphinxbottomrule
\end{tabular}
\sphinxtableafterendhook\par
\sphinxattableend\end{savenotes}


\begin{savenotes}\sphinxattablestart
\sphinxthistablewithglobalstyle
\centering
\begin{tabular}[t]{\X{33}{99}\X{33}{99}\X{33}{99}}
\sphinxtoprule
\sphinxtableatstartofbodyhook
\sphinxAtStartPar
位域 |
&
\sphinxAtStartPar
名称     | |
&
\sphinxAtStartPar
描述                                        | |
\\
\sphinxhline
\sphinxAtStartPar
31:24
&\begin{itemize}
\item {} 
\end{itemize}
&\begin{itemize}
\item {} 
\end{itemize}
\\
\sphinxhline
\sphinxAtStartPar
23
&
\sphinxAtStartPar
HE3B
&
\sphinxAtStartPar
第3组B路高电平结束中断屏蔽                  |

\sphinxAtStartPar
1:屏蔽                                     |

\sphinxAtStartPar
0:未屏蔽                                   |
\\
\sphinxhline
\sphinxAtStartPar
22
&
\sphinxAtStartPar
HE3A
&
\sphinxAtStartPar
第3组A路高电平结束中断屏蔽                  |

\sphinxAtStartPar
1:屏蔽                                     |

\sphinxAtStartPar
0:未屏蔽                                   |
\\
\sphinxhline
\sphinxAtStartPar
21
&
\sphinxAtStartPar
HE2B
&
\sphinxAtStartPar
第2组B路高电平结束中断屏蔽                  |

\sphinxAtStartPar
1:屏蔽                                     |

\sphinxAtStartPar
0:未屏蔽                                   |
\\
\sphinxhline
\sphinxAtStartPar
20
&
\sphinxAtStartPar
HE2A
&
\sphinxAtStartPar
第2组A路高电平结束中断屏蔽                  |

\sphinxAtStartPar
1:屏蔽                                     |

\sphinxAtStartPar
0:未屏蔽                                   |
\\
\sphinxhline
\sphinxAtStartPar
19
&
\sphinxAtStartPar
HE1B
&
\sphinxAtStartPar
第1组B路高电平结束中断屏蔽                  |

\sphinxAtStartPar
1:屏蔽                                     |

\sphinxAtStartPar
0:未屏蔽                                   |
\\
\sphinxhline
\sphinxAtStartPar
18
&
\sphinxAtStartPar
HE1A
&
\sphinxAtStartPar
第1组A路高电平结束中断屏蔽                  |

\sphinxAtStartPar
1:屏蔽                                     |

\sphinxAtStartPar
0:未屏蔽                                   |
\\
\sphinxhline
\sphinxAtStartPar
17
&
\sphinxAtStartPar
HE0B
&
\sphinxAtStartPar
第0组B路高电平结束中断屏蔽                  |

\sphinxAtStartPar
1:屏蔽                                     |

\sphinxAtStartPar
0:未屏蔽                                   |
\\
\sphinxhline
\sphinxAtStartPar
16
&
\sphinxAtStartPar
HE0A
&
\sphinxAtStartPar
第0组A路高电平结束中断屏蔽                  |

\sphinxAtStartPar
1:屏蔽                                     |

\sphinxAtStartPar
0:未屏蔽                                   |
\\
\sphinxhline
\sphinxAtStartPar
15:8
&\begin{itemize}
\item {} 
\end{itemize}
&\begin{itemize}
\item {} 
\end{itemize}
\\
\sphinxhline
\sphinxAtStartPar
7
&
\sphinxAtStartPar
NC3B
&
\sphinxAtStartPar
第3组B路新周期中断屏蔽                      |

\sphinxAtStartPar
1:屏蔽                                     |

\sphinxAtStartPar
0:未屏蔽                                   |
\\
\sphinxhline
\sphinxAtStartPar
6
&
\sphinxAtStartPar
NC3A
&
\sphinxAtStartPar
第3组A路新周期中断屏蔽                      |

\sphinxAtStartPar
1:屏蔽                                     |

\sphinxAtStartPar
0:未屏蔽                                   |
\\
\sphinxhline
\sphinxAtStartPar
5
&
\sphinxAtStartPar
NC2B
&
\sphinxAtStartPar
第2组B路新周期中断屏蔽                      |

\sphinxAtStartPar
1:屏蔽                                     |

\sphinxAtStartPar
0:未屏蔽                                   |
\\
\sphinxhline
\sphinxAtStartPar
4
&
\sphinxAtStartPar
NC2A
&
\sphinxAtStartPar
第2组A路新周期中断屏蔽                      |

\sphinxAtStartPar
1:屏蔽                                     |

\sphinxAtStartPar
0:未屏蔽                                   |
\\
\sphinxhline
\sphinxAtStartPar
3
&
\sphinxAtStartPar
NC1B
&
\sphinxAtStartPar
第1组B路新周期中断屏蔽                      |

\sphinxAtStartPar
1:屏蔽                                     |

\sphinxAtStartPar
0:未屏蔽                                   |
\\
\sphinxhline
\sphinxAtStartPar
2
&
\sphinxAtStartPar
NC1A
&
\sphinxAtStartPar
第1组A路新周期中断屏蔽                      |

\sphinxAtStartPar
1:屏蔽                                     |

\sphinxAtStartPar
0:未屏蔽                                   |
\\
\sphinxhline
\sphinxAtStartPar
1
&
\sphinxAtStartPar
NC0B
&
\sphinxAtStartPar
第0组B路新周期中断屏蔽                      |

\sphinxAtStartPar
1:屏蔽                                     |

\sphinxAtStartPar
0:未屏蔽                                   |
\\
\sphinxhline
\sphinxAtStartPar
0
&
\sphinxAtStartPar
NC0A
&
\sphinxAtStartPar
第0组A路新周期中断屏蔽                      |

\sphinxAtStartPar
1:屏蔽                                     |

\sphinxAtStartPar
0:未屏蔽                                   |
\\
\sphinxbottomrule
\end{tabular}
\sphinxtableafterendhook\par
\sphinxattableend\end{savenotes}


\subsubsection{PWM新周期中断原始状态寄存器NCIRS}
\label{\detokenize{SWM241/_u529f_u80fd_u63cf_u8ff0/_u8109_u51b2_u5bbd_u5ea6_u8c03_u5236:pwmncirs}}

\begin{savenotes}\sphinxattablestart
\sphinxthistablewithglobalstyle
\centering
\begin{tabular}[t]{\X{20}{100}\X{20}{100}\X{20}{100}\X{20}{100}\X{20}{100}}
\sphinxtoprule
\sphinxtableatstartofbodyhook
\sphinxAtStartPar
寄存器 |
&
\begin{DUlineblock}{0em}
\item[] 偏移 |
\end{DUlineblock}
&
\begin{DUlineblock}{0em}
\item[] 
\item[] {\color{red}\bfseries{}|}
\end{DUlineblock}
&
\sphinxAtStartPar
复位值 |    描 | |
&
\begin{DUlineblock}{0em}
\item[] |
  |
\end{DUlineblock}
\\
\sphinxhline
\sphinxAtStartPar
NCIRS
&
\sphinxAtStartPar
0x228
&&
\sphinxAtStartPar
0 000000
&
\sphinxAtStartPar
新周期中断原始状态         |
\\
\sphinxbottomrule
\end{tabular}
\sphinxtableafterendhook\par
\sphinxattableend\end{savenotes}


\begin{savenotes}\sphinxattablestart
\sphinxthistablewithglobalstyle
\centering
\begin{tabular}[t]{\X{12}{96}\X{12}{96}\X{12}{96}\X{12}{96}\X{12}{96}\X{12}{96}\X{12}{96}\X{12}{96}}
\sphinxtoprule
\sphinxtableatstartofbodyhook
\sphinxAtStartPar
31
&
\sphinxAtStartPar
30
&
\sphinxAtStartPar
29
&
\sphinxAtStartPar
28
&
\sphinxAtStartPar
27
&
\sphinxAtStartPar
26
&
\sphinxAtStartPar
25
&
\sphinxAtStartPar
24
\\
\sphinxhline\begin{itemize}
\item {} 
\end{itemize}
&&&&&&&\\
\sphinxhline
\sphinxAtStartPar
23
&
\sphinxAtStartPar
22
&
\sphinxAtStartPar
21
&
\sphinxAtStartPar
20
&
\sphinxAtStartPar
19
&
\sphinxAtStartPar
18
&
\sphinxAtStartPar
17
&
\sphinxAtStartPar
16
\\
\sphinxhline\begin{itemize}
\item {} 
\end{itemize}
&&&&&&&\\
\sphinxhline
\sphinxAtStartPar
15
&
\sphinxAtStartPar
14
&
\sphinxAtStartPar
13
&
\sphinxAtStartPar
12
&
\sphinxAtStartPar
11
&
\sphinxAtStartPar
10
&
\sphinxAtStartPar
9
&
\sphinxAtStartPar
8
\\
\sphinxhline\begin{itemize}
\item {} 
\end{itemize}
&&&&&&&\\
\sphinxhline
\sphinxAtStartPar
7
&
\sphinxAtStartPar
6
&
\sphinxAtStartPar
5
&
\sphinxAtStartPar
4
&
\sphinxAtStartPar
3
&
\sphinxAtStartPar
2
&
\sphinxAtStartPar
1
&
\sphinxAtStartPar
0
\\
\sphinxhline
\sphinxAtStartPar
PWM3B
&
\sphinxAtStartPar
PWM3A
&&&&&&\\
\sphinxbottomrule
\end{tabular}
\sphinxtableafterendhook\par
\sphinxattableend\end{savenotes}


\begin{savenotes}\sphinxattablestart
\sphinxthistablewithglobalstyle
\centering
\begin{tabular}[t]{\X{33}{99}\X{33}{99}\X{33}{99}}
\sphinxtoprule
\sphinxtableatstartofbodyhook
\sphinxAtStartPar
位域 |
&
\sphinxAtStartPar
名称     | |
&
\sphinxAtStartPar
描述                                        | |
\\
\sphinxhline
\sphinxAtStartPar
31:8
&\begin{itemize}
\item {} 
\end{itemize}
&\begin{itemize}
\item {} 
\end{itemize}
\\
\sphinxhline
\sphinxAtStartPar
7
&
\sphinxAtStartPar
PWM3B
&
\sphinxAtStartPar
第3组B路新周期开始原始中断状态,R/W1C       |

\sphinxAtStartPar
1:中断已发生                               |

\sphinxAtStartPar
0:中断未发生                               |
\\
\sphinxhline
\sphinxAtStartPar
6
&
\sphinxAtStartPar
PWM3A
&
\sphinxAtStartPar
第3组A路新周期开始原始中断状态,R/W1C       |

\sphinxAtStartPar
1:中断已发生                               |

\sphinxAtStartPar
0:中断未发生                               |
\\
\sphinxhline
\sphinxAtStartPar
5
&
\sphinxAtStartPar
PWM2B
&
\sphinxAtStartPar
第2组B路新周期开始原始中断状态,R/W1C       |

\sphinxAtStartPar
1:中断已发生                               |

\sphinxAtStartPar
0:中断未发生                               |
\\
\sphinxhline
\sphinxAtStartPar
4
&
\sphinxAtStartPar
PWM2A
&
\sphinxAtStartPar
第2组A路新周期开始原始中断状态,R/W1C       |

\sphinxAtStartPar
1:中断已发生                               |

\sphinxAtStartPar
0:中断未发生                               |
\\
\sphinxhline
\sphinxAtStartPar
3
&
\sphinxAtStartPar
PWM1B
&
\sphinxAtStartPar
第1组B路新周期开始原始中断状态,R/W1C       |

\sphinxAtStartPar
1:中断已发生                               |

\sphinxAtStartPar
0:中断未发生                               |
\\
\sphinxhline
\sphinxAtStartPar
2
&
\sphinxAtStartPar
PWM1A
&
\sphinxAtStartPar
第1组A路新周期开始原始中断状态,R/W1C       |

\sphinxAtStartPar
1:中断已发生                               |

\sphinxAtStartPar
0:中断未发生                               |
\\
\sphinxhline
\sphinxAtStartPar
1
&
\sphinxAtStartPar
PWM0B
&
\sphinxAtStartPar
第0组B路新周期开始原始中断状态,R/W1C       |

\sphinxAtStartPar
1:中断已发生                               |

\sphinxAtStartPar
0:中断未发生                               |
\\
\sphinxhline
\sphinxAtStartPar
0
&
\sphinxAtStartPar
PWM0A
&
\sphinxAtStartPar
第0组A路新周期开始原始中断状态,R/W1C       |

\sphinxAtStartPar
1:中断已发生                               |

\sphinxAtStartPar
0:中断未发生                               |
\\
\sphinxbottomrule
\end{tabular}
\sphinxtableafterendhook\par
\sphinxattableend\end{savenotes}


\subsubsection{PWM高电平结束中断原始状态寄存器HEIRS}
\label{\detokenize{SWM241/_u529f_u80fd_u63cf_u8ff0/_u8109_u51b2_u5bbd_u5ea6_u8c03_u5236:pwmheirs}}

\begin{savenotes}\sphinxattablestart
\sphinxthistablewithglobalstyle
\centering
\begin{tabular}[t]{\X{20}{100}\X{20}{100}\X{20}{100}\X{20}{100}\X{20}{100}}
\sphinxtoprule
\sphinxtableatstartofbodyhook
\sphinxAtStartPar
寄存器 |
&
\begin{DUlineblock}{0em}
\item[] 偏移 |
\end{DUlineblock}
&
\begin{DUlineblock}{0em}
\item[] 
\item[] {\color{red}\bfseries{}|}
\end{DUlineblock}
&
\sphinxAtStartPar
复位值 |    描 | |
&
\begin{DUlineblock}{0em}
\item[] |
  |
\end{DUlineblock}
\\
\sphinxhline
\sphinxAtStartPar
HEIRS
&
\sphinxAtStartPar
0x22C
&&
\sphinxAtStartPar
0 000000
&
\sphinxAtStartPar
高电平结束中断原始状态     |
\\
\sphinxbottomrule
\end{tabular}
\sphinxtableafterendhook\par
\sphinxattableend\end{savenotes}


\begin{savenotes}\sphinxattablestart
\sphinxthistablewithglobalstyle
\centering
\begin{tabular}[t]{\X{12}{96}\X{12}{96}\X{12}{96}\X{12}{96}\X{12}{96}\X{12}{96}\X{12}{96}\X{12}{96}}
\sphinxtoprule
\sphinxtableatstartofbodyhook
\sphinxAtStartPar
31
&
\sphinxAtStartPar
30
&
\sphinxAtStartPar
29
&
\sphinxAtStartPar
28
&
\sphinxAtStartPar
27
&
\sphinxAtStartPar
26
&
\sphinxAtStartPar
25
&
\sphinxAtStartPar
24
\\
\sphinxhline\begin{itemize}
\item {} 
\end{itemize}
&&&&&&&\\
\sphinxhline
\sphinxAtStartPar
23
&
\sphinxAtStartPar
22
&
\sphinxAtStartPar
21
&
\sphinxAtStartPar
20
&
\sphinxAtStartPar
19
&
\sphinxAtStartPar
18
&
\sphinxAtStartPar
17
&
\sphinxAtStartPar
16
\\
\sphinxhline\begin{itemize}
\item {} 
\end{itemize}
&&&&&&&\\
\sphinxhline
\sphinxAtStartPar
15
&
\sphinxAtStartPar
14
&
\sphinxAtStartPar
13
&
\sphinxAtStartPar
12
&
\sphinxAtStartPar
11
&
\sphinxAtStartPar
10
&
\sphinxAtStartPar
9
&
\sphinxAtStartPar
8
\\
\sphinxhline\begin{itemize}
\item {} 
\end{itemize}
&&&&&&&\\
\sphinxhline
\sphinxAtStartPar
7
&
\sphinxAtStartPar
6
&
\sphinxAtStartPar
5
&
\sphinxAtStartPar
4
&
\sphinxAtStartPar
3
&
\sphinxAtStartPar
2
&
\sphinxAtStartPar
1
&
\sphinxAtStartPar
0
\\
\sphinxhline
\sphinxAtStartPar
PWM3B
&
\sphinxAtStartPar
PWM3A
&&&&&&\\
\sphinxbottomrule
\end{tabular}
\sphinxtableafterendhook\par
\sphinxattableend\end{savenotes}


\begin{savenotes}\sphinxattablestart
\sphinxthistablewithglobalstyle
\centering
\begin{tabular}[t]{\X{33}{99}\X{33}{99}\X{33}{99}}
\sphinxtoprule
\sphinxtableatstartofbodyhook
\sphinxAtStartPar
位域 |
&
\sphinxAtStartPar
名称     | |
&
\sphinxAtStartPar
描述                                        | |
\\
\sphinxhline
\sphinxAtStartPar
31:8
&\begin{itemize}
\item {} 
\end{itemize}
&\begin{itemize}
\item {} 
\end{itemize}
\\
\sphinxhline
\sphinxAtStartPar
7
&
\sphinxAtStartPar
PWM3B
&
\sphinxAtStartPar
第3组B路高电平结束原始中断状态,写1清除     |

\sphinxAtStartPar
1:中断已发生                               |

\sphinxAtStartPar
0:中断未发生                               |
\\
\sphinxhline
\sphinxAtStartPar
6
&
\sphinxAtStartPar
PWM3A
&
\sphinxAtStartPar
第3组A路高电平结束原始中断状态,写1清除     |

\sphinxAtStartPar
1:中断已发生                               |

\sphinxAtStartPar
0:中断未发生                               |
\\
\sphinxhline
\sphinxAtStartPar
5
&
\sphinxAtStartPar
PWM2B
&
\sphinxAtStartPar
第2组B路高电平结束原始中断状态,写1清除     |

\sphinxAtStartPar
1:中断已发生                               |

\sphinxAtStartPar
0:中断未发生                               |
\\
\sphinxhline
\sphinxAtStartPar
4
&
\sphinxAtStartPar
PWM2A
&
\sphinxAtStartPar
第2组A路高电平结束原始中断状态,写1清除     |

\sphinxAtStartPar
1:中断已发生                               |

\sphinxAtStartPar
0:中断未发生                               |
\\
\sphinxhline
\sphinxAtStartPar
3
&
\sphinxAtStartPar
PWM1B
&
\sphinxAtStartPar
第1组B路高电平结束原始中断状态,写1清除     |

\sphinxAtStartPar
1:中断已发生                               |

\sphinxAtStartPar
0:中断未发生                               |
\\
\sphinxhline
\sphinxAtStartPar
2
&
\sphinxAtStartPar
PWM1A
&
\sphinxAtStartPar
第1组A路高电平结束原始中断状态,写1清除     |

\sphinxAtStartPar
1:中断已发生                               |

\sphinxAtStartPar
0:中断未发生                               |
\\
\sphinxhline
\sphinxAtStartPar
1
&
\sphinxAtStartPar
PWM0B
&
\sphinxAtStartPar
第0组B路高电平结束原始中断状态,写1清除     |

\sphinxAtStartPar
1:中断已发生                               |

\sphinxAtStartPar
0:中断未发生                               |
\\
\sphinxhline
\sphinxAtStartPar
0
&
\sphinxAtStartPar
PWM0A
&
\sphinxAtStartPar
第0组A路高电平结束原始中断状态,写1清除     |

\sphinxAtStartPar
1:中断已发生                               |

\sphinxAtStartPar
0:中断未发生                               |
\\
\sphinxbottomrule
\end{tabular}
\sphinxtableafterendhook\par
\sphinxattableend\end{savenotes}


\subsubsection{PWM新周期中断状态寄存器NCIF}
\label{\detokenize{SWM241/_u529f_u80fd_u63cf_u8ff0/_u8109_u51b2_u5bbd_u5ea6_u8c03_u5236:pwmncif}}

\begin{savenotes}\sphinxattablestart
\sphinxthistablewithglobalstyle
\centering
\begin{tabular}[t]{\X{20}{100}\X{20}{100}\X{20}{100}\X{20}{100}\X{20}{100}}
\sphinxtoprule
\sphinxtableatstartofbodyhook
\sphinxAtStartPar
寄存器 |
&
\begin{DUlineblock}{0em}
\item[] 偏移 |
\end{DUlineblock}
&
\begin{DUlineblock}{0em}
\item[] 
\item[] {\color{red}\bfseries{}|}
\end{DUlineblock}
&
\sphinxAtStartPar
复位值 |    描 | |
&
\begin{DUlineblock}{0em}
\item[] |
  |
\end{DUlineblock}
\\
\sphinxhline
\sphinxAtStartPar
NCIF
&
\sphinxAtStartPar
0x230
&&
\sphinxAtStartPar
0 000000
&
\sphinxAtStartPar
新周期中断状态             |
\\
\sphinxbottomrule
\end{tabular}
\sphinxtableafterendhook\par
\sphinxattableend\end{savenotes}


\begin{savenotes}\sphinxattablestart
\sphinxthistablewithglobalstyle
\centering
\begin{tabular}[t]{\X{12}{96}\X{12}{96}\X{12}{96}\X{12}{96}\X{12}{96}\X{12}{96}\X{12}{96}\X{12}{96}}
\sphinxtoprule
\sphinxtableatstartofbodyhook
\sphinxAtStartPar
31
&
\sphinxAtStartPar
30
&
\sphinxAtStartPar
29
&
\sphinxAtStartPar
28
&
\sphinxAtStartPar
27
&
\sphinxAtStartPar
26
&
\sphinxAtStartPar
25
&
\sphinxAtStartPar
24
\\
\sphinxhline\begin{itemize}
\item {} 
\end{itemize}
&&&&&&&\\
\sphinxhline
\sphinxAtStartPar
23
&
\sphinxAtStartPar
22
&
\sphinxAtStartPar
21
&
\sphinxAtStartPar
20
&
\sphinxAtStartPar
19
&
\sphinxAtStartPar
18
&
\sphinxAtStartPar
17
&
\sphinxAtStartPar
16
\\
\sphinxhline\begin{itemize}
\item {} 
\end{itemize}
&&&&&&&\\
\sphinxhline
\sphinxAtStartPar
15
&
\sphinxAtStartPar
14
&
\sphinxAtStartPar
13
&
\sphinxAtStartPar
12
&
\sphinxAtStartPar
11
&
\sphinxAtStartPar
10
&
\sphinxAtStartPar
9
&
\sphinxAtStartPar
8
\\
\sphinxhline\begin{itemize}
\item {} 
\end{itemize}
&&&&&&&\\
\sphinxhline
\sphinxAtStartPar
7
&
\sphinxAtStartPar
6
&
\sphinxAtStartPar
5
&
\sphinxAtStartPar
4
&
\sphinxAtStartPar
3
&
\sphinxAtStartPar
2
&
\sphinxAtStartPar
1
&
\sphinxAtStartPar
0
\\
\sphinxhline
\sphinxAtStartPar
PWM3B
&
\sphinxAtStartPar
PWM3A
&&&&&&\\
\sphinxbottomrule
\end{tabular}
\sphinxtableafterendhook\par
\sphinxattableend\end{savenotes}


\begin{savenotes}\sphinxattablestart
\sphinxthistablewithglobalstyle
\centering
\begin{tabular}[t]{\X{33}{99}\X{33}{99}\X{33}{99}}
\sphinxtoprule
\sphinxtableatstartofbodyhook
\sphinxAtStartPar
位域 |
&
\sphinxAtStartPar
名称     | |
&
\sphinxAtStartPar
描述                                        | |
\\
\sphinxhline
\sphinxAtStartPar
31:8
&\begin{itemize}
\item {} 
\end{itemize}
&\begin{itemize}
\item {} 
\end{itemize}
\\
\sphinxhline
\sphinxAtStartPar
7
&
\sphinxAtStartPar
PWM3B
&
\sphinxAtStartPar
第3组B路新周期开始中断状态,R/W1C           |

\sphinxAtStartPar
1:中断已发生                               |

\sphinxAtStartPar
0:中断未发生                               |
\\
\sphinxhline
\sphinxAtStartPar
6
&
\sphinxAtStartPar
PWM3A
&
\sphinxAtStartPar
第3组A路新周期开始中断状态,R/W1C           |

\sphinxAtStartPar
1:中断已发生                               |

\sphinxAtStartPar
0:中断未发生                               |
\\
\sphinxhline
\sphinxAtStartPar
5
&
\sphinxAtStartPar
PWM2B
&
\sphinxAtStartPar
第2组B路新周期开始中断状态,R/W1C           |

\sphinxAtStartPar
1:中断已发生                               |

\sphinxAtStartPar
0:中断未发生                               |
\\
\sphinxhline
\sphinxAtStartPar
4
&
\sphinxAtStartPar
PWM2A
&
\sphinxAtStartPar
第2组A路新周期开始中断状态,R/W1C           |

\sphinxAtStartPar
1:中断已发生                               |

\sphinxAtStartPar
0:中断未发生                               |
\\
\sphinxhline
\sphinxAtStartPar
3
&
\sphinxAtStartPar
PWM1B
&
\sphinxAtStartPar
第1组B路新周期开始中断状态,R/W1C           |

\sphinxAtStartPar
1:中断已发生                               |

\sphinxAtStartPar
0:中断未发生                               |
\\
\sphinxhline
\sphinxAtStartPar
2
&
\sphinxAtStartPar
PWM1A
&
\sphinxAtStartPar
第1组A路新周期开始中断状态,R/W1C           |

\sphinxAtStartPar
1:中断已发生                               |

\sphinxAtStartPar
0:中断未发生                               |
\\
\sphinxhline
\sphinxAtStartPar
1
&
\sphinxAtStartPar
PWM0B
&
\sphinxAtStartPar
第0组B路新周期开始中断状态,R/W1C           |

\sphinxAtStartPar
1:中断已发生                               |

\sphinxAtStartPar
0:中断未发生                               |
\\
\sphinxhline
\sphinxAtStartPar
0
&
\sphinxAtStartPar
PWM0A
&
\sphinxAtStartPar
第0组A路新周期开始中断状态,R/W1C           |

\sphinxAtStartPar
1:中断已发生                               |

\sphinxAtStartPar
0:中断未发生                               |
\\
\sphinxbottomrule
\end{tabular}
\sphinxtableafterendhook\par
\sphinxattableend\end{savenotes}


\subsubsection{PWM高电平结束中断状态寄存器HEIF}
\label{\detokenize{SWM241/_u529f_u80fd_u63cf_u8ff0/_u8109_u51b2_u5bbd_u5ea6_u8c03_u5236:pwmheif}}

\begin{savenotes}\sphinxattablestart
\sphinxthistablewithglobalstyle
\centering
\begin{tabular}[t]{\X{20}{100}\X{20}{100}\X{20}{100}\X{20}{100}\X{20}{100}}
\sphinxtoprule
\sphinxtableatstartofbodyhook
\sphinxAtStartPar
寄存器 |
&
\begin{DUlineblock}{0em}
\item[] 偏移 |
\end{DUlineblock}
&
\begin{DUlineblock}{0em}
\item[] 
\item[] {\color{red}\bfseries{}|}
\end{DUlineblock}
&
\sphinxAtStartPar
复位值 |    描 | |
&
\begin{DUlineblock}{0em}
\item[] |
  |
\end{DUlineblock}
\\
\sphinxhline
\sphinxAtStartPar
HEIF
&
\sphinxAtStartPar
0x234
&&
\sphinxAtStartPar
0 000000
&
\sphinxAtStartPar
高电平结束中断状态         |
\\
\sphinxbottomrule
\end{tabular}
\sphinxtableafterendhook\par
\sphinxattableend\end{savenotes}


\begin{savenotes}\sphinxattablestart
\sphinxthistablewithglobalstyle
\centering
\begin{tabular}[t]{\X{12}{96}\X{12}{96}\X{12}{96}\X{12}{96}\X{12}{96}\X{12}{96}\X{12}{96}\X{12}{96}}
\sphinxtoprule
\sphinxtableatstartofbodyhook
\sphinxAtStartPar
31
&
\sphinxAtStartPar
30
&
\sphinxAtStartPar
29
&
\sphinxAtStartPar
28
&
\sphinxAtStartPar
27
&
\sphinxAtStartPar
26
&
\sphinxAtStartPar
25
&
\sphinxAtStartPar
24
\\
\sphinxhline\begin{itemize}
\item {} 
\end{itemize}
&&&&&&&\\
\sphinxhline
\sphinxAtStartPar
23
&
\sphinxAtStartPar
22
&
\sphinxAtStartPar
21
&
\sphinxAtStartPar
20
&
\sphinxAtStartPar
19
&
\sphinxAtStartPar
18
&
\sphinxAtStartPar
17
&
\sphinxAtStartPar
16
\\
\sphinxhline\begin{itemize}
\item {} 
\end{itemize}
&&&&&&&\\
\sphinxhline
\sphinxAtStartPar
15
&
\sphinxAtStartPar
14
&
\sphinxAtStartPar
13
&
\sphinxAtStartPar
12
&
\sphinxAtStartPar
11
&
\sphinxAtStartPar
10
&
\sphinxAtStartPar
9
&
\sphinxAtStartPar
8
\\
\sphinxhline\begin{itemize}
\item {} 
\end{itemize}
&&&&&&&\\
\sphinxhline
\sphinxAtStartPar
7
&
\sphinxAtStartPar
6
&
\sphinxAtStartPar
5
&
\sphinxAtStartPar
4
&
\sphinxAtStartPar
3
&
\sphinxAtStartPar
2
&
\sphinxAtStartPar
1
&
\sphinxAtStartPar
0
\\
\sphinxhline
\sphinxAtStartPar
PWM3B
&
\sphinxAtStartPar
PWM3A
&&&&&&\\
\sphinxbottomrule
\end{tabular}
\sphinxtableafterendhook\par
\sphinxattableend\end{savenotes}


\begin{savenotes}\sphinxattablestart
\sphinxthistablewithglobalstyle
\centering
\begin{tabular}[t]{\X{33}{99}\X{33}{99}\X{33}{99}}
\sphinxtoprule
\sphinxtableatstartofbodyhook
\sphinxAtStartPar
位域 |
&
\sphinxAtStartPar
名称     | |
&
\sphinxAtStartPar
描述                                        | |
\\
\sphinxhline
\sphinxAtStartPar
31:8
&\begin{itemize}
\item {} 
\end{itemize}
&\begin{itemize}
\item {} 
\end{itemize}
\\
\sphinxhline
\sphinxAtStartPar
7
&
\sphinxAtStartPar
PWM3B
&
\sphinxAtStartPar
第3组B路高电平结束中断状态,R/W1C           |

\sphinxAtStartPar
1:中断已发生                               |

\sphinxAtStartPar
0:中断未发生                               |
\\
\sphinxhline
\sphinxAtStartPar
6
&
\sphinxAtStartPar
PWM3A
&
\sphinxAtStartPar
第3组A路高电平结束中断状态,R/W1C           |

\sphinxAtStartPar
1:中断已发生                               |

\sphinxAtStartPar
0:中断未发生                               |
\\
\sphinxhline
\sphinxAtStartPar
5
&
\sphinxAtStartPar
PWM2B
&
\sphinxAtStartPar
第2组B路高电平结束中断状态,R/W1C           |

\sphinxAtStartPar
1:中断已发生                               |

\sphinxAtStartPar
0:中断未发生                               |
\\
\sphinxhline
\sphinxAtStartPar
4
&
\sphinxAtStartPar
PWM2A
&
\sphinxAtStartPar
第2组A路高电平结束中断状态,R/W1C           |

\sphinxAtStartPar
1:中断已发生                               |

\sphinxAtStartPar
0:中断未发生                               |
\\
\sphinxhline
\sphinxAtStartPar
3
&
\sphinxAtStartPar
PWM1B
&
\sphinxAtStartPar
第1组B路高电平结束中断状态,R/W1C           |

\sphinxAtStartPar
1:中断已发生                               |

\sphinxAtStartPar
0:中断未发生                               |
\\
\sphinxhline
\sphinxAtStartPar
2
&
\sphinxAtStartPar
PWM1A
&
\sphinxAtStartPar
第1组A路高电平结束中断状态,R/W1C           |

\sphinxAtStartPar
1:中断已发生                               |

\sphinxAtStartPar
0:中断未发生                               |
\\
\sphinxhline
\sphinxAtStartPar
1
&
\sphinxAtStartPar
PWM0B
&
\sphinxAtStartPar
第0组B路高电平结束中断状态,R/W1C           |

\sphinxAtStartPar
1:中断已发生                               |

\sphinxAtStartPar
0:中断未发生                               |
\\
\sphinxhline
\sphinxAtStartPar
0
&
\sphinxAtStartPar
PWM0A
&
\sphinxAtStartPar
第0组A路高电平结束中断状态,R/W1C           |

\sphinxAtStartPar
1:中断已发生                               |

\sphinxAtStartPar
0:中断未发生                               |
\\
\sphinxbottomrule
\end{tabular}
\sphinxtableafterendhook\par
\sphinxattableend\end{savenotes}


\subsubsection{PWM半周期中断使能寄存器HCIE}
\label{\detokenize{SWM241/_u529f_u80fd_u63cf_u8ff0/_u8109_u51b2_u5bbd_u5ea6_u8c03_u5236:pwmhcie}}

\begin{savenotes}\sphinxattablestart
\sphinxthistablewithglobalstyle
\centering
\begin{tabular}[t]{\X{20}{100}\X{20}{100}\X{20}{100}\X{20}{100}\X{20}{100}}
\sphinxtoprule
\sphinxtableatstartofbodyhook
\sphinxAtStartPar
寄存器 |
&
\begin{DUlineblock}{0em}
\item[] 偏移 |
\end{DUlineblock}
&
\begin{DUlineblock}{0em}
\item[] 
\item[] {\color{red}\bfseries{}|}
\end{DUlineblock}
&
\sphinxAtStartPar
复位值 |    描 | |
&
\begin{DUlineblock}{0em}
\item[] |
  |
\end{DUlineblock}
\\
\sphinxhline
\sphinxAtStartPar
HCIE
&
\sphinxAtStartPar
0x238
&&
\sphinxAtStartPar
0 000000
&
\sphinxAtStartPar
半周期中断使能             |
\\
\sphinxbottomrule
\end{tabular}
\sphinxtableafterendhook\par
\sphinxattableend\end{savenotes}


\begin{savenotes}\sphinxattablestart
\sphinxthistablewithglobalstyle
\centering
\begin{tabular}[t]{\X{12}{96}\X{12}{96}\X{12}{96}\X{12}{96}\X{12}{96}\X{12}{96}\X{12}{96}\X{12}{96}}
\sphinxtoprule
\sphinxtableatstartofbodyhook
\sphinxAtStartPar
31
&
\sphinxAtStartPar
30
&
\sphinxAtStartPar
29
&
\sphinxAtStartPar
28
&
\sphinxAtStartPar
27
&
\sphinxAtStartPar
26
&
\sphinxAtStartPar
25
&
\sphinxAtStartPar
24
\\
\sphinxhline\begin{itemize}
\item {} 
\end{itemize}
&&&&&&&\\
\sphinxhline
\sphinxAtStartPar
23
&
\sphinxAtStartPar
22
&
\sphinxAtStartPar
21
&
\sphinxAtStartPar
20
&
\sphinxAtStartPar
19
&
\sphinxAtStartPar
18
&
\sphinxAtStartPar
17
&
\sphinxAtStartPar
16
\\
\sphinxhline\begin{itemize}
\item {} 
\end{itemize}
&&&&&&&\\
\sphinxhline
\sphinxAtStartPar
15
&
\sphinxAtStartPar
14
&
\sphinxAtStartPar
13
&
\sphinxAtStartPar
12
&
\sphinxAtStartPar
11
&
\sphinxAtStartPar
10
&
\sphinxAtStartPar
9
&
\sphinxAtStartPar
8
\\
\sphinxhline\begin{itemize}
\item {} 
\end{itemize}
&&&&&&&\\
\sphinxhline
\sphinxAtStartPar
7
&
\sphinxAtStartPar
6
&
\sphinxAtStartPar
5
&
\sphinxAtStartPar
4
&
\sphinxAtStartPar
3
&
\sphinxAtStartPar
2
&
\sphinxAtStartPar
1
&
\sphinxAtStartPar
0
\\
\sphinxhline
\sphinxAtStartPar
PWM3B
&
\sphinxAtStartPar
PWM3A
&&&&&&\\
\sphinxbottomrule
\end{tabular}
\sphinxtableafterendhook\par
\sphinxattableend\end{savenotes}


\begin{savenotes}\sphinxattablestart
\sphinxthistablewithglobalstyle
\centering
\begin{tabular}[t]{\X{33}{99}\X{33}{99}\X{33}{99}}
\sphinxtoprule
\sphinxtableatstartofbodyhook
\sphinxAtStartPar
位域 |
&
\sphinxAtStartPar
名称     | |
&
\sphinxAtStartPar
描述                                        | |
\\
\sphinxhline
\sphinxAtStartPar
31:8
&\begin{itemize}
\item {} 
\end{itemize}
&\begin{itemize}
\item {} 
\end{itemize}
\\
\sphinxhline
\sphinxAtStartPar
7
&
\sphinxAtStartPar
PWM3B
&
\sphinxAtStartPar
第3组B路半周期中断使能                      |

\sphinxAtStartPar
1:使能                                     |

\sphinxAtStartPar
0:禁能                                     |
\\
\sphinxhline
\sphinxAtStartPar
6
&
\sphinxAtStartPar
PWM3A
&
\sphinxAtStartPar
第3组A路半周期中断使能                      |

\sphinxAtStartPar
1:使能                                     |

\sphinxAtStartPar
0:禁能                                     |
\\
\sphinxhline
\sphinxAtStartPar
5
&
\sphinxAtStartPar
PWM2B
&
\sphinxAtStartPar
第2组B路半周期中断使能                      |

\sphinxAtStartPar
1:使能                                     |

\sphinxAtStartPar
0:禁能                                     |
\\
\sphinxhline
\sphinxAtStartPar
4
&
\sphinxAtStartPar
PWM2A
&
\sphinxAtStartPar
第2组A路半周期中断使能                      |

\sphinxAtStartPar
1:使能                                     |

\sphinxAtStartPar
0:禁能                                     |
\\
\sphinxhline
\sphinxAtStartPar
3
&
\sphinxAtStartPar
PWM1B
&
\sphinxAtStartPar
第1组B路半周期中断使能                      |

\sphinxAtStartPar
1:使能                                     |

\sphinxAtStartPar
0:禁能                                     |
\\
\sphinxhline
\sphinxAtStartPar
2
&
\sphinxAtStartPar
PWM1A
&
\sphinxAtStartPar
第1组A路半周期中断使能                      |

\sphinxAtStartPar
1:使能                                     |

\sphinxAtStartPar
0:禁能                                     |
\\
\sphinxhline
\sphinxAtStartPar
1
&
\sphinxAtStartPar
PWM0B
&
\sphinxAtStartPar
第0组B路半周期中断使能                      |

\sphinxAtStartPar
1:使能                                     |

\sphinxAtStartPar
0:禁能                                     |
\\
\sphinxhline
\sphinxAtStartPar
0
&
\sphinxAtStartPar
PWM0A
&
\sphinxAtStartPar
第0组A路半周期中断使能                      |

\sphinxAtStartPar
1:使能                                     |

\sphinxAtStartPar
0:禁能                                     |
\\
\sphinxbottomrule
\end{tabular}
\sphinxtableafterendhook\par
\sphinxattableend\end{savenotes}


\subsubsection{PWM半周期中断屏蔽寄存器HCIM}
\label{\detokenize{SWM241/_u529f_u80fd_u63cf_u8ff0/_u8109_u51b2_u5bbd_u5ea6_u8c03_u5236:pwmhcim}}

\begin{savenotes}\sphinxattablestart
\sphinxthistablewithglobalstyle
\centering
\begin{tabular}[t]{\X{20}{100}\X{20}{100}\X{20}{100}\X{20}{100}\X{20}{100}}
\sphinxtoprule
\sphinxtableatstartofbodyhook
\sphinxAtStartPar
寄存器 |
&
\begin{DUlineblock}{0em}
\item[] 偏移 |
\end{DUlineblock}
&
\begin{DUlineblock}{0em}
\item[] 
\item[] {\color{red}\bfseries{}|}
\end{DUlineblock}
&
\sphinxAtStartPar
复位值 |    描 | |
&
\begin{DUlineblock}{0em}
\item[] |
  |
\end{DUlineblock}
\\
\sphinxhline
\sphinxAtStartPar
HCIM
&
\sphinxAtStartPar
0x23C
&&
\sphinxAtStartPar
0 000000
&
\sphinxAtStartPar
半周期中断屏蔽             |
\\
\sphinxbottomrule
\end{tabular}
\sphinxtableafterendhook\par
\sphinxattableend\end{savenotes}


\begin{savenotes}\sphinxattablestart
\sphinxthistablewithglobalstyle
\centering
\begin{tabular}[t]{\X{12}{96}\X{12}{96}\X{12}{96}\X{12}{96}\X{12}{96}\X{12}{96}\X{12}{96}\X{12}{96}}
\sphinxtoprule
\sphinxtableatstartofbodyhook
\sphinxAtStartPar
31
&
\sphinxAtStartPar
30
&
\sphinxAtStartPar
29
&
\sphinxAtStartPar
28
&
\sphinxAtStartPar
27
&
\sphinxAtStartPar
26
&
\sphinxAtStartPar
25
&
\sphinxAtStartPar
24
\\
\sphinxhline\begin{itemize}
\item {} 
\end{itemize}
&&&&&&&\\
\sphinxhline
\sphinxAtStartPar
23
&
\sphinxAtStartPar
22
&
\sphinxAtStartPar
21
&
\sphinxAtStartPar
20
&
\sphinxAtStartPar
19
&
\sphinxAtStartPar
18
&
\sphinxAtStartPar
17
&
\sphinxAtStartPar
16
\\
\sphinxhline\begin{itemize}
\item {} 
\end{itemize}
&&&&&&&\\
\sphinxhline
\sphinxAtStartPar
15
&
\sphinxAtStartPar
14
&
\sphinxAtStartPar
13
&
\sphinxAtStartPar
12
&
\sphinxAtStartPar
11
&
\sphinxAtStartPar
10
&
\sphinxAtStartPar
9
&
\sphinxAtStartPar
8
\\
\sphinxhline\begin{itemize}
\item {} 
\end{itemize}
&&&&&&&\\
\sphinxhline
\sphinxAtStartPar
7
&
\sphinxAtStartPar
6
&
\sphinxAtStartPar
5
&
\sphinxAtStartPar
4
&
\sphinxAtStartPar
3
&
\sphinxAtStartPar
2
&
\sphinxAtStartPar
1
&
\sphinxAtStartPar
0
\\
\sphinxhline
\sphinxAtStartPar
PWM3B
&
\sphinxAtStartPar
PWM3A
&&&&&&\\
\sphinxbottomrule
\end{tabular}
\sphinxtableafterendhook\par
\sphinxattableend\end{savenotes}


\begin{savenotes}\sphinxattablestart
\sphinxthistablewithglobalstyle
\centering
\begin{tabular}[t]{\X{33}{99}\X{33}{99}\X{33}{99}}
\sphinxtoprule
\sphinxtableatstartofbodyhook
\sphinxAtStartPar
位域 |
&
\sphinxAtStartPar
名称     | |
&
\sphinxAtStartPar
描述                                        | |
\\
\sphinxhline
\sphinxAtStartPar
31:8
&\begin{itemize}
\item {} 
\end{itemize}
&\begin{itemize}
\item {} 
\end{itemize}
\\
\sphinxhline
\sphinxAtStartPar
7
&
\sphinxAtStartPar
PWM3B
&\\
\sphinxhline
\sphinxAtStartPar
6
&
\sphinxAtStartPar
PWM3A
&\\
\sphinxhline
\sphinxAtStartPar
5
&
\sphinxAtStartPar
PWM2B
&\\
\sphinxhline
\sphinxAtStartPar
4
&
\sphinxAtStartPar
PWM2A
&\\
\sphinxhline
\sphinxAtStartPar
3
&
\sphinxAtStartPar
PWM1B
&\\
\sphinxhline
\sphinxAtStartPar
2
&
\sphinxAtStartPar
PWM1A
&\\
\sphinxhline
\sphinxAtStartPar
1
&
\sphinxAtStartPar
PWM0B
&\\
\sphinxhline
\sphinxAtStartPar
0
&
\sphinxAtStartPar
PWM0A
&\\
\sphinxbottomrule
\end{tabular}
\sphinxtableafterendhook\par
\sphinxattableend\end{savenotes}


\subsubsection{PWM半周期原始中断状态寄存器HCIRS}
\label{\detokenize{SWM241/_u529f_u80fd_u63cf_u8ff0/_u8109_u51b2_u5bbd_u5ea6_u8c03_u5236:pwmhcirs}}

\begin{savenotes}\sphinxattablestart
\sphinxthistablewithglobalstyle
\centering
\begin{tabular}[t]{\X{20}{100}\X{20}{100}\X{20}{100}\X{20}{100}\X{20}{100}}
\sphinxtoprule
\sphinxtableatstartofbodyhook
\sphinxAtStartPar
寄存器 |
&
\begin{DUlineblock}{0em}
\item[] 偏移 |
\end{DUlineblock}
&
\begin{DUlineblock}{0em}
\item[] 
\item[] {\color{red}\bfseries{}|}
\end{DUlineblock}
&
\sphinxAtStartPar
复位值 |    描 | |
&
\begin{DUlineblock}{0em}
\item[] |
  |
\end{DUlineblock}
\\
\sphinxhline
\sphinxAtStartPar
HCIRS
&
\sphinxAtStartPar
0x240
&&
\sphinxAtStartPar
0 000000
&
\sphinxAtStartPar
半周期原始中断状态         |
\\
\sphinxbottomrule
\end{tabular}
\sphinxtableafterendhook\par
\sphinxattableend\end{savenotes}


\begin{savenotes}\sphinxattablestart
\sphinxthistablewithglobalstyle
\centering
\begin{tabular}[t]{\X{12}{96}\X{12}{96}\X{12}{96}\X{12}{96}\X{12}{96}\X{12}{96}\X{12}{96}\X{12}{96}}
\sphinxtoprule
\sphinxtableatstartofbodyhook
\sphinxAtStartPar
31
&
\sphinxAtStartPar
30
&
\sphinxAtStartPar
29
&
\sphinxAtStartPar
28
&
\sphinxAtStartPar
27
&
\sphinxAtStartPar
26
&
\sphinxAtStartPar
25
&
\sphinxAtStartPar
24
\\
\sphinxhline\begin{itemize}
\item {} 
\end{itemize}
&&&&&&&\\
\sphinxhline
\sphinxAtStartPar
23
&
\sphinxAtStartPar
22
&
\sphinxAtStartPar
21
&
\sphinxAtStartPar
20
&
\sphinxAtStartPar
19
&
\sphinxAtStartPar
18
&
\sphinxAtStartPar
17
&
\sphinxAtStartPar
16
\\
\sphinxhline\begin{itemize}
\item {} 
\end{itemize}
&&&&&&&\\
\sphinxhline
\sphinxAtStartPar
15
&
\sphinxAtStartPar
14
&
\sphinxAtStartPar
13
&
\sphinxAtStartPar
12
&
\sphinxAtStartPar
11
&
\sphinxAtStartPar
10
&
\sphinxAtStartPar
9
&
\sphinxAtStartPar
8
\\
\sphinxhline\begin{itemize}
\item {} 
\end{itemize}
&&&&&&&\\
\sphinxhline
\sphinxAtStartPar
7
&
\sphinxAtStartPar
6
&
\sphinxAtStartPar
5
&
\sphinxAtStartPar
4
&
\sphinxAtStartPar
3
&
\sphinxAtStartPar
2
&
\sphinxAtStartPar
1
&
\sphinxAtStartPar
0
\\
\sphinxhline
\sphinxAtStartPar
PWM3B
&
\sphinxAtStartPar
PWM3A
&&&&&&\\
\sphinxbottomrule
\end{tabular}
\sphinxtableafterendhook\par
\sphinxattableend\end{savenotes}


\begin{savenotes}\sphinxattablestart
\sphinxthistablewithglobalstyle
\centering
\begin{tabular}[t]{\X{33}{99}\X{33}{99}\X{33}{99}}
\sphinxtoprule
\sphinxtableatstartofbodyhook
\sphinxAtStartPar
位域 |
&
\sphinxAtStartPar
名称     | |
&
\sphinxAtStartPar
描述                                        | |
\\
\sphinxhline
\sphinxAtStartPar
31:8
&\begin{itemize}
\item {} 
\end{itemize}
&\begin{itemize}
\item {} 
\end{itemize}
\\
\sphinxhline
\sphinxAtStartPar
7
&
\sphinxAtStartPar
PWM3B
&
\sphinxAtStartPar
第3组B路半周期原始中断状态,R/W1C           |

\sphinxAtStartPar
1:中断已发生                               |

\sphinxAtStartPar
0:中断未发生                               |
\\
\sphinxhline
\sphinxAtStartPar
6
&
\sphinxAtStartPar
PWM3A
&
\sphinxAtStartPar
第3组A路半周期原始中断状态,R/W1C           |

\sphinxAtStartPar
1:中断已发生                               |

\sphinxAtStartPar
0:中断未发生                               |
\\
\sphinxhline
\sphinxAtStartPar
5
&
\sphinxAtStartPar
PWM2B
&
\sphinxAtStartPar
第2组B路半周期原始中断状态,R/W1C           |

\sphinxAtStartPar
1:中断已发生                               |

\sphinxAtStartPar
0:中断未发生                               |
\\
\sphinxhline
\sphinxAtStartPar
4
&
\sphinxAtStartPar
PWM2A
&
\sphinxAtStartPar
第2组A路半周期原始中断状态,R/W1C           |

\sphinxAtStartPar
1:中断已发生                               |

\sphinxAtStartPar
0:中断未发生                               |
\\
\sphinxhline
\sphinxAtStartPar
3
&
\sphinxAtStartPar
PWM1B
&
\sphinxAtStartPar
第1组B路半周期原始中断状态,R/W1C           |

\sphinxAtStartPar
1:中断已发生                               |

\sphinxAtStartPar
0:中断未发生                               |
\\
\sphinxhline
\sphinxAtStartPar
2
&
\sphinxAtStartPar
PWM1A
&
\sphinxAtStartPar
第1组A路半周期原始中断状态,R/W1C           |

\sphinxAtStartPar
1:中断已发生                               |

\sphinxAtStartPar
0:中断未发生                               |
\\
\sphinxhline
\sphinxAtStartPar
1
&
\sphinxAtStartPar
PWM0B
&
\sphinxAtStartPar
第0组B路半周期原始中断状态,R/W1C           |

\sphinxAtStartPar
1:中断已发生                               |

\sphinxAtStartPar
0:中断未发生                               |
\\
\sphinxhline
\sphinxAtStartPar
0
&
\sphinxAtStartPar
PWM0A
&
\sphinxAtStartPar
第0组A路半周期原始中断状态,R/W1C           |

\sphinxAtStartPar
1:中断已发生                               |

\sphinxAtStartPar
0:中断未发生                               |
\\
\sphinxbottomrule
\end{tabular}
\sphinxtableafterendhook\par
\sphinxattableend\end{savenotes}


\subsubsection{PWM半周期中断状态寄存器HCIF}
\label{\detokenize{SWM241/_u529f_u80fd_u63cf_u8ff0/_u8109_u51b2_u5bbd_u5ea6_u8c03_u5236:pwmhcif}}

\begin{savenotes}\sphinxattablestart
\sphinxthistablewithglobalstyle
\centering
\begin{tabular}[t]{\X{20}{100}\X{20}{100}\X{20}{100}\X{20}{100}\X{20}{100}}
\sphinxtoprule
\sphinxtableatstartofbodyhook
\sphinxAtStartPar
寄存器 |
&
\begin{DUlineblock}{0em}
\item[] 偏移 |
\end{DUlineblock}
&
\begin{DUlineblock}{0em}
\item[] 
\item[] {\color{red}\bfseries{}|}
\end{DUlineblock}
&
\sphinxAtStartPar
复位值 |    描 | |
&
\begin{DUlineblock}{0em}
\item[] |
  |
\end{DUlineblock}
\\
\sphinxhline
\sphinxAtStartPar
HCIF
&
\sphinxAtStartPar
0x244
&&
\sphinxAtStartPar
0 000000
&
\sphinxAtStartPar
半周期中断状态             |
\\
\sphinxbottomrule
\end{tabular}
\sphinxtableafterendhook\par
\sphinxattableend\end{savenotes}


\begin{savenotes}\sphinxattablestart
\sphinxthistablewithglobalstyle
\centering
\begin{tabular}[t]{\X{12}{96}\X{12}{96}\X{12}{96}\X{12}{96}\X{12}{96}\X{12}{96}\X{12}{96}\X{12}{96}}
\sphinxtoprule
\sphinxtableatstartofbodyhook
\sphinxAtStartPar
31
&
\sphinxAtStartPar
30
&
\sphinxAtStartPar
29
&
\sphinxAtStartPar
28
&
\sphinxAtStartPar
27
&
\sphinxAtStartPar
26
&
\sphinxAtStartPar
25
&
\sphinxAtStartPar
24
\\
\sphinxhline\begin{itemize}
\item {} 
\end{itemize}
&&&&&&&\\
\sphinxhline
\sphinxAtStartPar
23
&
\sphinxAtStartPar
22
&
\sphinxAtStartPar
21
&
\sphinxAtStartPar
20
&
\sphinxAtStartPar
19
&
\sphinxAtStartPar
18
&
\sphinxAtStartPar
17
&
\sphinxAtStartPar
16
\\
\sphinxhline\begin{itemize}
\item {} 
\end{itemize}
&&&&&&&\\
\sphinxhline
\sphinxAtStartPar
15
&
\sphinxAtStartPar
14
&
\sphinxAtStartPar
13
&
\sphinxAtStartPar
12
&
\sphinxAtStartPar
11
&
\sphinxAtStartPar
10
&
\sphinxAtStartPar
9
&
\sphinxAtStartPar
8
\\
\sphinxhline\begin{itemize}
\item {} 
\end{itemize}
&&&&&&&\\
\sphinxhline
\sphinxAtStartPar
7
&
\sphinxAtStartPar
6
&
\sphinxAtStartPar
5
&
\sphinxAtStartPar
4
&
\sphinxAtStartPar
3
&
\sphinxAtStartPar
2
&
\sphinxAtStartPar
1
&
\sphinxAtStartPar
0
\\
\sphinxhline
\sphinxAtStartPar
PWM3B
&
\sphinxAtStartPar
PWM3A
&&&&&&\\
\sphinxbottomrule
\end{tabular}
\sphinxtableafterendhook\par
\sphinxattableend\end{savenotes}


\begin{savenotes}\sphinxattablestart
\sphinxthistablewithglobalstyle
\centering
\begin{tabular}[t]{\X{33}{99}\X{33}{99}\X{33}{99}}
\sphinxtoprule
\sphinxtableatstartofbodyhook
\sphinxAtStartPar
位域 |
&
\sphinxAtStartPar
名称     | |
&
\sphinxAtStartPar
描述                                        | |
\\
\sphinxhline
\sphinxAtStartPar
31:8
&\begin{itemize}
\item {} 
\end{itemize}
&\begin{itemize}
\item {} 
\end{itemize}
\\
\sphinxhline
\sphinxAtStartPar
7
&
\sphinxAtStartPar
PWM3B
&
\sphinxAtStartPar
第3组B路半周期中断状态,R/W1C               |

\sphinxAtStartPar
1:中断已发生                               |

\sphinxAtStartPar
0:中断未发生                               |
\\
\sphinxhline
\sphinxAtStartPar
6
&
\sphinxAtStartPar
PWM3A
&
\sphinxAtStartPar
第3组A路半周期中断状态,R/W1C               |

\sphinxAtStartPar
1:中断已发生                               |

\sphinxAtStartPar
0:中断未发生                               |
\\
\sphinxhline
\sphinxAtStartPar
5
&
\sphinxAtStartPar
PWM2B
&
\sphinxAtStartPar
第2组B路半周期中断状态,R/W1C               |

\sphinxAtStartPar
1:中断已发生                               |

\sphinxAtStartPar
0:中断未发生                               |
\\
\sphinxhline
\sphinxAtStartPar
4
&
\sphinxAtStartPar
PWM2A
&
\sphinxAtStartPar
第2组A路半周期中断状态,R/W1C               |

\sphinxAtStartPar
1:中断已发生                               |

\sphinxAtStartPar
0:中断未发生                               |
\\
\sphinxhline
\sphinxAtStartPar
3
&
\sphinxAtStartPar
PWM1B
&
\sphinxAtStartPar
第1组B路半周期中断状态,R/W1C               |

\sphinxAtStartPar
1:中断已发生                               |

\sphinxAtStartPar
0:中断未发生                               |
\\
\sphinxhline
\sphinxAtStartPar
2
&
\sphinxAtStartPar
PWM1A
&
\sphinxAtStartPar
第1组A路半周期中断状态,R/W1C               |

\sphinxAtStartPar
1:中断已发生                               |

\sphinxAtStartPar
0:中断未发生                               |
\\
\sphinxhline
\sphinxAtStartPar
1
&
\sphinxAtStartPar
PWM0B
&
\sphinxAtStartPar
第0组B路半周期中断状态,R/W1C               |

\sphinxAtStartPar
1:中断已发生                               |

\sphinxAtStartPar
0:中断未发生                               |
\\
\sphinxhline
\sphinxAtStartPar
0
&
\sphinxAtStartPar
PWM0A
&
\sphinxAtStartPar
第0组A路半周期中断状态,R/W1C               |

\sphinxAtStartPar
1:中断已发生                               |

\sphinxAtStartPar
0:中断未发生                               |
\\
\sphinxbottomrule
\end{tabular}
\sphinxtableafterendhook\par
\sphinxattableend\end{savenotes}


\subsubsection{强制输出电平选择寄存器FORCEV}
\label{\detokenize{SWM241/_u529f_u80fd_u63cf_u8ff0/_u8109_u51b2_u5bbd_u5ea6_u8c03_u5236:forcev}}

\begin{savenotes}\sphinxattablestart
\sphinxthistablewithglobalstyle
\centering
\begin{tabular}[t]{\X{20}{100}\X{20}{100}\X{20}{100}\X{20}{100}\X{20}{100}}
\sphinxtoprule
\sphinxtableatstartofbodyhook
\sphinxAtStartPar
寄存器 |
&
\begin{DUlineblock}{0em}
\item[] 偏移 |
\end{DUlineblock}
&
\begin{DUlineblock}{0em}
\item[] 
\item[] {\color{red}\bfseries{}|}
\end{DUlineblock}
&
\sphinxAtStartPar
复位值 |    描 | |
&
\begin{DUlineblock}{0em}
\item[] |
  |
\end{DUlineblock}
\\
\sphinxhline
\sphinxAtStartPar
FORCEV
&
\sphinxAtStartPar
0x248
&&
\sphinxAtStartPar
0 000000
&
\sphinxAtStartPar
强制输出电平选择寄存器     |
\\
\sphinxbottomrule
\end{tabular}
\sphinxtableafterendhook\par
\sphinxattableend\end{savenotes}


\begin{savenotes}\sphinxattablestart
\sphinxthistablewithglobalstyle
\centering
\begin{tabular}[t]{\X{12}{96}\X{12}{96}\X{12}{96}\X{12}{96}\X{12}{96}\X{12}{96}\X{12}{96}\X{12}{96}}
\sphinxtoprule
\sphinxtableatstartofbodyhook
\sphinxAtStartPar
31
&
\sphinxAtStartPar
30
&
\sphinxAtStartPar
29
&
\sphinxAtStartPar
28
&
\sphinxAtStartPar
27
&
\sphinxAtStartPar
26
&
\sphinxAtStartPar
25
&
\sphinxAtStartPar
24
\\
\sphinxhline\begin{itemize}
\item {} 
\end{itemize}
&&&&
\sphinxAtStartPar
P Bn
&&
\sphinxAtStartPar
P Bn
&\\
\sphinxhline
\sphinxAtStartPar
23
&
\sphinxAtStartPar
22
&
\sphinxAtStartPar
21
&
\sphinxAtStartPar
20
&
\sphinxAtStartPar
19
&
\sphinxAtStartPar
18
&
\sphinxAtStartPar
17
&
\sphinxAtStartPar
16
\\
\sphinxhline\begin{itemize}
\item {} 
\end{itemize}
&&&&&&&\\
\sphinxhline
\sphinxAtStartPar
15
&
\sphinxAtStartPar
14
&
\sphinxAtStartPar
13
&
\sphinxAtStartPar
12
&
\sphinxAtStartPar
11
&
\sphinxAtStartPar
10
&
\sphinxAtStartPar
9
&
\sphinxAtStartPar
8
\\
\sphinxhline\begin{itemize}
\item {} 
\end{itemize}
&&&&
\sphinxAtStartPar
P An
&&
\sphinxAtStartPar
P An
&\\
\sphinxhline
\sphinxAtStartPar
7
&
\sphinxAtStartPar
6
&
\sphinxAtStartPar
5
&
\sphinxAtStartPar
4
&
\sphinxAtStartPar
3
&
\sphinxAtStartPar
2
&
\sphinxAtStartPar
1
&
\sphinxAtStartPar
0
\\
\sphinxhline\begin{itemize}
\item {} 
\end{itemize}
&&&&&&&\\
\sphinxbottomrule
\end{tabular}
\sphinxtableafterendhook\par
\sphinxattableend\end{savenotes}


\begin{savenotes}\sphinxattablestart
\sphinxthistablewithglobalstyle
\centering
\begin{tabular}[t]{\X{33}{99}\X{33}{99}\X{33}{99}}
\sphinxtoprule
\sphinxtableatstartofbodyhook
\sphinxAtStartPar
位域 |
&
\sphinxAtStartPar
名称     | |
&
\sphinxAtStartPar
描述                                        | |
\\
\sphinxhline
\sphinxAtStartPar
31:28
&\begin{itemize}
\item {} 
\end{itemize}
&\begin{itemize}
\item {} 
\end{itemize}
\\
\sphinxhline
\sphinxAtStartPar
27
&
\sphinxAtStartPar
PWM3Bn
&\\
\sphinxhline
\sphinxAtStartPar
26
&
\sphinxAtStartPar
PWM2Bn
&\\
\sphinxhline
\sphinxAtStartPar
25
&
\sphinxAtStartPar
PWM1Bn
&\\
\sphinxhline
\sphinxAtStartPar
24
&
\sphinxAtStartPar
PWM0Bn
&\\
\sphinxhline
\sphinxAtStartPar
23:18
&\begin{itemize}
\item {} 
\end{itemize}
&\begin{itemize}
\item {} 
\end{itemize}
\\
\sphinxhline
\sphinxAtStartPar
19
&
\sphinxAtStartPar
PWM3B
&\\
\sphinxhline
\sphinxAtStartPar
18
&
\sphinxAtStartPar
PWM2B
&\\
\sphinxhline
\sphinxAtStartPar
17
&
\sphinxAtStartPar
PWM1B
&\\
\sphinxhline
\sphinxAtStartPar
16
&
\sphinxAtStartPar
PWM0B
&\\
\sphinxhline
\sphinxAtStartPar
15:12
&\begin{itemize}
\item {} 
\end{itemize}
&\begin{itemize}
\item {} 
\end{itemize}
\\
\sphinxhline
\sphinxAtStartPar
11
&
\sphinxAtStartPar
PWM3An
&\\
\sphinxhline
\sphinxAtStartPar
10
&
\sphinxAtStartPar
PWM2An
&\\
\sphinxhline
\sphinxAtStartPar
9
&
\sphinxAtStartPar
PWM1An
&\\
\sphinxhline
\sphinxAtStartPar
8
&
\sphinxAtStartPar
PWM0An
&\\
\sphinxhline
\sphinxAtStartPar
7:4
&\begin{itemize}
\item {} 
\end{itemize}
&\begin{itemize}
\item {} 
\end{itemize}
\\
\sphinxhline
\sphinxAtStartPar
3
&
\sphinxAtStartPar
PWM3A
&\\
\sphinxhline
\sphinxAtStartPar
2
&
\sphinxAtStartPar
PWM2A
&\\
\sphinxhline
\sphinxAtStartPar
1
&
\sphinxAtStartPar
PWM1A
&\\
\sphinxhline
\sphinxAtStartPar
0
&
\sphinxAtStartPar
PWM0A
&\\
\sphinxbottomrule
\end{tabular}
\sphinxtableafterendhook\par
\sphinxattableend\end{savenotes}

\sphinxstepscope


\section{模拟数字转换器(SAR ADC)}
\label{\detokenize{SWM241/_u529f_u80fd_u63cf_u8ff0/_u6a21_u62df_u6570_u5b57_u8f6c_u6362_u5668:sar-adc}}\label{\detokenize{SWM241/_u529f_u80fd_u63cf_u8ff0/_u6a21_u62df_u6570_u5b57_u8f6c_u6362_u5668::doc}}
\sphinxAtStartPar
概述
\textasciitilde{}\textasciitilde{}

\sphinxAtStartPar
SWM241系列所有型号SAR ADC操作均相同,不同型号ADC通道数量可能不同,最多支持1组12通道。使用前需使能SAR ADC模块时钟。

\sphinxAtStartPar
特性
\textasciitilde{}\textasciitilde{}
\begin{itemize}
\item {} 
\sphinxAtStartPar
12\sphinxhyphen{}bits 分辨率

\item {} 
\sphinxAtStartPar
最高1MSPS 转换速率

\item {} 
\sphinxAtStartPar
支持单次模式和连续模式

\item {} 
\sphinxAtStartPar
具备深度为8的FIFO

\item {} 
\sphinxAtStartPar
灵活的转换启动方式,支持软件、PWM、TIMER启动

\item {} 
\sphinxAtStartPar
每个通道都有自己独立的转换结果数据寄存器和转换完成、数据溢出状态寄存器

\item {} 
\sphinxAtStartPar
支持DMA传输

\end{itemize}


\subsection{模块结构框图}
\label{\detokenize{SWM241/_u529f_u80fd_u63cf_u8ff0/_u6a21_u62df_u6570_u5b57_u8f6c_u6362_u5668:id1}}
\sphinxAtStartPar
\sphinxincludegraphics{{SWM241/功能描述/media模拟数字转换002}.emf}

\sphinxAtStartPar
图 6‑54 ADC模块结构框图


\subsection{功能描述}
\label{\detokenize{SWM241/_u529f_u80fd_u63cf_u8ff0/_u6a21_u62df_u6570_u5b57_u8f6c_u6362_u5668:id2}}

\subsubsection{操作说明}
\label{\detokenize{SWM241/_u529f_u80fd_u63cf_u8ff0/_u6a21_u62df_u6570_u5b57_u8f6c_u6362_u5668:id3}}
\sphinxAtStartPar
使用SAR ADC前,需针对对应引脚及模块进行如下操作:
\begin{itemize}
\item {} 
\sphinxAtStartPar
通过PORT\_FUNC寄存器将引脚切换GPIO功能,注意使用ADC功能时,不可将数字输入使能

\item {} 
\sphinxAtStartPar
通过CTRL寄存器中TRIG位配置触发方式

\item {} 
\sphinxAtStartPar
通过CTRL寄存器中CONT位配置采样方式(单次采样、连续采样)

\item {} 
\sphinxAtStartPar
通过CTRL寄存器中DMAEN位配置是否需要DMA读取

\item {} 
\sphinxAtStartPar
通过CTRL寄存器中AVG位配置是否需要硬件计算平均值

\item {} 
\sphinxAtStartPar
如需使用中断,通过IE寄存器使能对应中断

\item {} 
\sphinxAtStartPar
配置CTRL寄存器中对应通道(CHx)选通

\item {} 
\sphinxAtStartPar
使能CTRL寄存器中EN位

\item {} 
\sphinxAtStartPar
使用软件使能START寄存器GO位触发采样或使用TIMER、PWM模块触发采样

\item {} 
\sphinxAtStartPar
单次采样模式下,START寄存器自动清零;连续采样模式下,软件向 START 寄存器写0停止转换。

\end{itemize}


\subsubsection{触发源选择}
\label{\detokenize{SWM241/_u529f_u80fd_u63cf_u8ff0/_u6a21_u62df_u6570_u5b57_u8f6c_u6362_u5668:id4}}
\sphinxAtStartPar
SAR ADC支持CPU触发、PWM触发、TIMER触发。通过将SAR ADC CTRL寄存器中TRIG进行设置,该设置对所有选中通道均有效,当不同通道需要不同触发方式时,需要在采样间隔配置TRIG位进行切换。

\sphinxAtStartPar
各模式触发操作方式如下:

\sphinxAtStartPar
使用PWM触发

\sphinxAtStartPar
PWM配置所需模式,将SARADC的CTRL寄存器中TRIG方式设置为PWM触发。每路PWM对应一个ADTRG寄存器值,当PWM计数到指定值,可触发ADC进行采样。当8路PWM工作在中心对称互补模式下时,最多可触发16次ADC采样。

\sphinxAtStartPar
具体配置方式如下(以ADTRG0A0为例):
\begin{itemize}
\item {} 
\sphinxAtStartPar
PWM配置所需模式,

\item {} 
\sphinxAtStartPar
配置PWM模块ADTRG0A0数值,该数值为触发延时时长,在中心对称模式下,前半周期从周期起始记,后半周期采样点与前半周期中心对称

\item {} 
\sphinxAtStartPar
使能ADTRG0A0寄存器EN位

\item {} 
\sphinxAtStartPar
配置ADC寄存器中TRIG寄存器A0对应位,确认该通道未被屏蔽

\item {} 
\sphinxAtStartPar
使能PWM模块EN位,当计数值到达ADTRG0A0设置值时,触发ADC CTRL寄存器中选中的通道(CHx)进行采样,采样完成后,将产生EOC标志位,并产生ADC中断

\end{itemize}

\sphinxAtStartPar
示意图如图 6‑55所示。

\sphinxAtStartPar
\sphinxincludegraphics{{SWM241/功能描述/media模拟数字转换003}.emf}

\sphinxAtStartPar
图 6‑55 中心对称模式下PWM触发ADC采样示意图

\sphinxAtStartPar
使用TIMER触发

\sphinxAtStartPar
将SAR ADC CTRL寄存器中TRIG设置为TIMERx触发。TIMER可作为定时器或计数器使用。当对应TIMER计数值减至0时,将触发ADC CTRL寄存器中选中的通道(CHx)进行采样。可以通过ADC采样完成中断进行结果获取。TIMER触发支持单次模式和连续模式,且支持多次采样求平均值。

\sphinxAtStartPar
使用软件触发

\sphinxAtStartPar
将CTRL寄存器中TRIG设置为CPU触发。ADC配置完成后,通过程序将START寄存器GO位置1触发采样。采样完成后,该位自动清0。可以通过ADC采样完成中断或标志位查询进行结果获取。软件触发支持单次模式和连续模式。


\subsubsection{数据处理}
\label{\detokenize{SWM241/_u529f_u80fd_u63cf_u8ff0/_u6a21_u62df_u6570_u5b57_u8f6c_u6362_u5668:id5}}
\sphinxAtStartPar
SAR ADC支持针对采样数据硬件自动完成平均值计算。通过配置CTRL寄存器中AVG位设置为结果取平均。支持2到16次取平均。设置N次平均,则采集完成N次后EOC标志有效,同时取平均值的结果被送至对应通道数据寄存器。


\subsubsection{参考源选择}
\label{\detokenize{SWM241/_u529f_u80fd_u63cf_u8ff0/_u6a21_u62df_u6570_u5b57_u8f6c_u6362_u5668:id6}}
\sphinxAtStartPar
SAR
ADC支持使用REFP和REFN作为输入电压参考。部分ADCx可具有独立的参考电压输入(不同封装可能有所变化,具体见封装引脚图),当封装图上有REFP/REFN引脚时,需接外部参考电压,此时参考电压为接入电压;当封装图上没有REFP/REFN引脚时,参考电压为ADC电源电压AVDD/AVSS。


\subsubsection{模式说明}
\label{\detokenize{SWM241/_u529f_u80fd_u63cf_u8ff0/_u6a21_u62df_u6570_u5b57_u8f6c_u6362_u5668:id7}}
\sphinxAtStartPar
单次模式

\sphinxAtStartPar
单次模式在所有选通的通道上执行一次转换,然后自动停止,其运作流程如下:
\begin{itemize}
\item {} 
\sphinxAtStartPar
启动ADC采样前,CTRL寄存器CONT位配置为单次模式

\item {} 
\sphinxAtStartPar
START寄存器写1启动转换,也可以用PWM和TIMER触发启动

\item {} 
\sphinxAtStartPar
所有CTRL寄存器中选通通道从小到大依次完成一次转换,并将转换结果和转换完成EOC标志存入通道对应的数据和状态寄存器

\item {} 
\sphinxAtStartPar
每个通道转换完成时对应通道状态寄存器的EOC标志会置位,如果该通道的EOC中断使能,则该通道转换完成时会触发中断处理程序

\item {} 
\sphinxAtStartPar
所有通道转换完成后,START寄存器自动清零,停止转换,ADC进入Idle模式。

\end{itemize}

\sphinxAtStartPar
连续模式

\sphinxAtStartPar
连续模式下ADC会不断的重复在所有选通的通道上执行转换,直到软件向START寄存器写0,示意图如图 6‑57所示。

\sphinxAtStartPar
具体操作步骤如下:
\begin{itemize}
\item {} 
\sphinxAtStartPar
启动ADC采样前,CTRL寄存器CONT位配置为连续模式

\item {} 
\sphinxAtStartPar
START寄存器写1启动转换,也可以用PWM和TIMER触发启动

\item {} 
\sphinxAtStartPar
所有CTRL寄存器中选通通道从小到大依次完成一次转换,转换完成后EOC标志将存入通道对应的状态寄存器

\item {} 
\sphinxAtStartPar
使用FIFO时,采样结果及对应通道将存至FIFO,未使用FIFO时,转换结果存入通道对应的数据寄存器

\item {} 
\sphinxAtStartPar
每个通道转换完成时对应通道状态寄存器的EOC标志会置位,如果该通道的EOC中断使能,则该通道转换完成时会触发中断处理程序

\item {} 
\sphinxAtStartPar
重复采样及结果存储,直到START寄存器写0,A/D转换停止,A/D转换器进入空闲状态。

\end{itemize}

\sphinxAtStartPar
\sphinxincludegraphics[width=5.68611in,height=1.47708in]{{SWM241/功能描述/media模拟数字转换004}.jpeg}

\sphinxAtStartPar
图 6‑56 SAR ADC连续采样示意图

\sphinxAtStartPar
\sphinxincludegraphics{{SWM241/功能描述/media模拟数字转换005}.emf}

\sphinxAtStartPar
图 6‑57 SAR ADC多通道连续采样示意图


\subsubsection{供电电压}
\label{\detokenize{SWM241/_u529f_u80fd_u63cf_u8ff0/_u6a21_u62df_u6570_u5b57_u8f6c_u6362_u5668:id8}}
\sphinxAtStartPar
ADC正常供电电压范围为2.5V\textasciitilde{}5.5V,其特性详情请参考表格 8‑7所示。

\sphinxAtStartPar
ADC工作电压在2.5V以下时会影响ADC精度,建议2.5V电压点以下不使用ADC值。


\subsubsection{中断配置与清除}
\label{\detokenize{SWM241/_u529f_u80fd_u63cf_u8ff0/_u6a21_u62df_u6570_u5b57_u8f6c_u6362_u5668:id9}}
\sphinxAtStartPar
可通过配置中断使能寄存器IE中相应位使能中断。当中断触发后,中断标志寄存器IF中对应位置1。如需清除此标志,需在对应标志位中写1清零(R/W1C),否则中断在开启状态下会一直进入。


\subsection{寄存器映射}
\label{\detokenize{SWM241/_u529f_u80fd_u63cf_u8ff0/_u6a21_u62df_u6570_u5b57_u8f6c_u6362_u5668:id10}}

\begin{savenotes}
\sphinxatlongtablestart
\sphinxthistablewithglobalstyle
\makeatletter
  \LTleft \@totalleftmargin plus1fill
  \LTright\dimexpr\columnwidth-\@totalleftmargin-\linewidth\relax plus1fill
\makeatother
\begin{longtable}{\X{20}{100}\X{20}{100}\X{20}{100}\X{20}{100}\X{20}{100}}
\sphinxtoprule
\endfirsthead

\multicolumn{5}{c}{\sphinxnorowcolor
    \makebox[0pt]{\sphinxtablecontinued{\tablename\ \thetable{} \textendash{} continued from previous page}}%
}\\
\sphinxtoprule
\endhead

\sphinxbottomrule
\multicolumn{5}{r}{\sphinxnorowcolor
    \makebox[0pt][r]{\sphinxtablecontinued{continues on next page}}%
}\\
\endfoot

\endlastfoot
\sphinxtableatstartofbodyhook

\sphinxAtStartPar
名称   |
&
\begin{DUlineblock}{0em}
\item[] 偏移 |
\end{DUlineblock}
&
\begin{DUlineblock}{0em}
\item[] 
\item[] |
|
\end{DUlineblock}
&
\begin{DUlineblock}{0em}
\item[] 
\end{DUlineblock}
\begin{quote}

\begin{DUlineblock}{0em}
\item[] 
\item[] 
\end{DUlineblock}
\end{quote}
&
\sphinxAtStartPar
描述                       | | | |
\\
\sphinxhline
\sphinxAtStartPar
SAR\sphinxhyphen{}ADC0BASE:0 {\color{red}\bfseries{}|}x40049000
&
\begin{DUlineblock}{0em}
\item[] 
\end{DUlineblock}
&&&\\
\sphinxhline
\sphinxAtStartPar
CTRL
&
\sphinxAtStartPar
0x00
&&
\sphinxAtStartPar
0x 00000
&
\sphinxAtStartPar
ADC配置寄存器              |
\\
\sphinxhline
\sphinxAtStartPar
START
&
\sphinxAtStartPar
0x04
&&
\sphinxAtStartPar
0x 00000
&
\sphinxAtStartPar
ADC启动寄存器              |
\\
\sphinxhline
\sphinxAtStartPar
IE
&
\sphinxAtStartPar
0x08
&&
\sphinxAtStartPar
0x 00000
&
\sphinxAtStartPar
ADC中断使能寄存器          |
\\
\sphinxhline
\sphinxAtStartPar
IF
&
\sphinxAtStartPar
0x0C
&&
\sphinxAtStartPar
0x 00000
&
\sphinxAtStartPar
ADC中断状态寄存器          |
\\
\sphinxhline
\sphinxAtStartPar
STAT0
&
\sphinxAtStartPar
0x10
&&
\sphinxAtStartPar
0x 00000
&
\sphinxAtStartPar
ADC通道0状态寄存器         |
\\
\sphinxhline
\sphinxAtStartPar
DATA0
&
\sphinxAtStartPar
0x14
&&
\sphinxAtStartPar
0x 00000
&
\sphinxAtStartPar
ADC通道0数据寄存器         |
\\
\sphinxhline
\sphinxAtStartPar
STAT1
&
\sphinxAtStartPar
0x20
&&
\sphinxAtStartPar
0x 00000
&
\sphinxAtStartPar
ADC通道1状态寄存器         |
\\
\sphinxhline
\sphinxAtStartPar
DATA1
&
\sphinxAtStartPar
0x24
&&
\sphinxAtStartPar
0x 00000
&
\sphinxAtStartPar
ADC通道1数据寄存器         |
\\
\sphinxhline
\sphinxAtStartPar
STAT2
&
\sphinxAtStartPar
0x30
&&
\sphinxAtStartPar
0x 00000
&
\sphinxAtStartPar
ADC通道2状态寄存器         |
\\
\sphinxhline
\sphinxAtStartPar
DATA2
&
\sphinxAtStartPar
0x34
&&
\sphinxAtStartPar
0x 00000
&
\sphinxAtStartPar
ADC通道2数据寄存器         |
\\
\sphinxhline
\sphinxAtStartPar
STAT3
&
\sphinxAtStartPar
0x40
&&
\sphinxAtStartPar
0x 00000
&
\sphinxAtStartPar
ADC通道3状态寄存器         |
\\
\sphinxhline
\sphinxAtStartPar
DATA3
&
\sphinxAtStartPar
0x44
&&
\sphinxAtStartPar
0x 00000
&
\sphinxAtStartPar
ADC通道3数据寄存器         |
\\
\sphinxhline
\sphinxAtStartPar
STAT4
&
\sphinxAtStartPar
0x50
&&
\sphinxAtStartPar
0x 00000
&
\sphinxAtStartPar
ADC通道4状态寄存器         |
\\
\sphinxhline
\sphinxAtStartPar
DATA4
&
\sphinxAtStartPar
0x54
&&
\sphinxAtStartPar
0x 00000
&
\sphinxAtStartPar
ADC通道4数据寄存器         |
\\
\sphinxhline
\sphinxAtStartPar
STAT5
&
\sphinxAtStartPar
0x60
&&
\sphinxAtStartPar
0x 00000
&
\sphinxAtStartPar
ADC通道5状态寄存器         |
\\
\sphinxhline
\sphinxAtStartPar
DATA5
&
\sphinxAtStartPar
0x64
&&
\sphinxAtStartPar
0x 00000
&
\sphinxAtStartPar
ADC通道5数据寄存器         |
\\
\sphinxhline
\sphinxAtStartPar
STAT6
&
\sphinxAtStartPar
0x70
&&
\sphinxAtStartPar
0x 00000
&
\sphinxAtStartPar
ADC通道6状态寄存器         |
\\
\sphinxhline
\sphinxAtStartPar
DATA6
&
\sphinxAtStartPar
0x74
&&
\sphinxAtStartPar
0x 00000
&
\sphinxAtStartPar
ADC通道6数据寄存器         |
\\
\sphinxhline
\sphinxAtStartPar
STAT7
&
\sphinxAtStartPar
0x80
&&
\sphinxAtStartPar
0x 00000
&
\sphinxAtStartPar
ADC通道7状态寄存器         |
\\
\sphinxhline
\sphinxAtStartPar
DATA7
&
\sphinxAtStartPar
0x84
&&
\sphinxAtStartPar
0x 00000
&
\sphinxAtStartPar
ADC通道7数据寄存器         |
\\
\sphinxhline
\sphinxAtStartPar
STAT8
&
\sphinxAtStartPar
0x90
&&
\sphinxAtStartPar
0x 00000
&
\sphinxAtStartPar
ADC通道8状态寄存器         |
\\
\sphinxhline
\sphinxAtStartPar
DATA8
&
\sphinxAtStartPar
0x94
&&
\sphinxAtStartPar
0x 00000
&
\sphinxAtStartPar
ADC通道8数据寄存器         |
\\
\sphinxhline
\sphinxAtStartPar
STAT9
&
\sphinxAtStartPar
0xa0
&&
\sphinxAtStartPar
0x 00000
&
\sphinxAtStartPar
ADC通道9状态寄存器         |
\\
\sphinxhline
\sphinxAtStartPar
DATA9
&
\sphinxAtStartPar
0xa4
&&
\sphinxAtStartPar
0x 00000
&
\sphinxAtStartPar
ADC通道9数据寄存器         |
\\
\sphinxhline
\sphinxAtStartPar
STAT10
&
\sphinxAtStartPar
0xb0
&&
\sphinxAtStartPar
0x 00000
&
\sphinxAtStartPar
ADC通道10状态寄存器        |
\\
\sphinxhline
\sphinxAtStartPar
DATA10
&
\sphinxAtStartPar
0xb4
&&
\sphinxAtStartPar
0x 00000
&
\sphinxAtStartPar
ADC通道10数据寄存器        |
\\
\sphinxhline
\sphinxAtStartPar
STAT11
&
\sphinxAtStartPar
0xc0
&&
\sphinxAtStartPar
0x 00000
&
\sphinxAtStartPar
ADC通道11状态寄存器        |
\\
\sphinxhline
\sphinxAtStartPar
DATA11
&
\sphinxAtStartPar
0xc4
&&
\sphinxAtStartPar
0x 00000
&
\sphinxAtStartPar
ADC通道11数据寄存器        |
\\
\sphinxhline
\sphinxAtStartPar
CHSEL
&
\sphinxAtStartPar
0xd0
&&
\sphinxAtStartPar
0x 00000
&
\sphinxAtStartPar
ADC通道配置寄存器          |
\\
\sphinxhline
\sphinxAtStartPar
FFSTAT
&
\sphinxAtStartPar
0x190
&&
\sphinxAtStartPar
0x 00008
&
\sphinxAtStartPar
ADC FIFO状态寄存器         |
\\
\sphinxhline
\sphinxAtStartPar
FFDATA
&
\sphinxAtStartPar
0x194
&&
\sphinxAtStartPar
0x 00000
&
\sphinxAtStartPar
ADC所有通道数据寄存器      |
\\
\sphinxhline
\sphinxAtStartPar
CTRL1
&
\sphinxAtStartPar
0x1a0
&&
\sphinxAtStartPar
0x 00000
&
\sphinxAtStartPar
ADC配置寄存器1             |
\\
\sphinxhline
\sphinxAtStartPar
CTRL2
&
\sphinxAtStartPar
0x1a4
&&
\sphinxAtStartPar
0x 00F10
&
\sphinxAtStartPar
ADC配置寄存器2             |
\\
\sphinxhline
\sphinxAtStartPar
CTRL3
&
\sphinxAtStartPar
0x1a8
&&
\sphinxAtStartPar
0x 00000
&
\sphinxAtStartPar
ADC配置寄存器3             |
\\
\sphinxhline
\sphinxAtStartPar
TRGMSK
&
\sphinxAtStartPar
0x1b0
&&
\sphinxAtStartPar
0x 00000
&
\sphinxAtStartPar
PWM通道触发ADC屏蔽寄存器   |
\\
\sphinxhline
\sphinxAtStartPar
CALIBSET
&
\sphinxAtStartPar
0x1f4
&&
\sphinxAtStartPar
0x 00000
&
\sphinxAtStartPar
ADC数据调整寄存器          |
\\
\sphinxhline
\sphinxAtStartPar
CALIBEN
&
\sphinxAtStartPar
0x1f8
&&
\sphinxAtStartPar
0x 00000
&
\sphinxAtStartPar
ADC数据调整使能寄存器      |
\\
\sphinxbottomrule
\end{longtable}
\sphinxtableafterendhook
\sphinxatlongtableend
\end{savenotes}


\subsection{寄存器描述}
\label{\detokenize{SWM241/_u529f_u80fd_u63cf_u8ff0/_u6a21_u62df_u6570_u5b57_u8f6c_u6362_u5668:id13}}

\subsubsection{配置寄存器CTRL}
\label{\detokenize{SWM241/_u529f_u80fd_u63cf_u8ff0/_u6a21_u62df_u6570_u5b57_u8f6c_u6362_u5668:ctrl}}

\begin{savenotes}\sphinxattablestart
\sphinxthistablewithglobalstyle
\centering
\begin{tabular}[t]{\X{20}{100}\X{20}{100}\X{20}{100}\X{20}{100}\X{20}{100}}
\sphinxtoprule
\sphinxtableatstartofbodyhook
\sphinxAtStartPar
寄存器 |
&
\begin{DUlineblock}{0em}
\item[] 偏移 |
\end{DUlineblock}
&
\begin{DUlineblock}{0em}
\item[] 
\item[] {\color{red}\bfseries{}|}
\end{DUlineblock}
&
\sphinxAtStartPar
复位值 |    描 | |
&
\begin{DUlineblock}{0em}
\item[] |
  |
\end{DUlineblock}
\\
\sphinxhline
\sphinxAtStartPar
CTRL
&
\sphinxAtStartPar
0x00
&&
\sphinxAtStartPar
0 000000
&
\sphinxAtStartPar
ADC配置寄存器              |
\\
\sphinxbottomrule
\end{tabular}
\sphinxtableafterendhook\par
\sphinxattableend\end{savenotes}


\begin{savenotes}\sphinxattablestart
\sphinxthistablewithglobalstyle
\centering
\begin{tabular}[t]{\X{12}{96}\X{12}{96}\X{12}{96}\X{12}{96}\X{12}{96}\X{12}{96}\X{12}{96}\X{12}{96}}
\sphinxtoprule
\sphinxtableatstartofbodyhook
\sphinxAtStartPar
31
&
\sphinxAtStartPar
30
&
\sphinxAtStartPar
29
&
\sphinxAtStartPar
28
&
\sphinxAtStartPar
27
&
\sphinxAtStartPar
26
&
\sphinxAtStartPar
25
&
\sphinxAtStartPar
24
\\
\sphinxhline\begin{itemize}
\item {} 
\end{itemize}
&&&&&&&
\sphinxAtStartPar
AVG
\\
\sphinxhline
\sphinxAtStartPar
23
&
\sphinxAtStartPar
22
&
\sphinxAtStartPar
21
&
\sphinxAtStartPar
20
&
\sphinxAtStartPar
19
&
\sphinxAtStartPar
18
&
\sphinxAtStartPar
17
&
\sphinxAtStartPar
16
\\
\sphinxhline
\sphinxAtStartPar
AVG
&&&
\sphinxAtStartPar
RST
&&&&\\
\sphinxhline
\sphinxAtStartPar
15
&
\sphinxAtStartPar
14
&
\sphinxAtStartPar
13
&
\sphinxAtStartPar
12
&
\sphinxAtStartPar
11
&
\sphinxAtStartPar
10
&
\sphinxAtStartPar
9
&
\sphinxAtStartPar
8
\\
\sphinxhline
\sphinxAtStartPar
TRIG
&&&
\sphinxAtStartPar
EN
&&&
\sphinxAtStartPar
CH9
&
\sphinxAtStartPar
CH8
\\
\sphinxhline
\sphinxAtStartPar
7
&
\sphinxAtStartPar
6
&
\sphinxAtStartPar
5
&
\sphinxAtStartPar
4
&
\sphinxAtStartPar
3
&
\sphinxAtStartPar
2
&
\sphinxAtStartPar
1
&
\sphinxAtStartPar
0
\\
\sphinxhline
\sphinxAtStartPar
CH7
&
\sphinxAtStartPar
CH6
&
\sphinxAtStartPar
CH5
&
\sphinxAtStartPar
CH4
&
\sphinxAtStartPar
CH3
&
\sphinxAtStartPar
CH2
&
\sphinxAtStartPar
CH1
&
\sphinxAtStartPar
CH0
\\
\sphinxbottomrule
\end{tabular}
\sphinxtableafterendhook\par
\sphinxattableend\end{savenotes}


\begin{savenotes}\sphinxattablestart
\sphinxthistablewithglobalstyle
\centering
\begin{tabular}[t]{\X{33}{99}\X{33}{99}\X{33}{99}}
\sphinxtoprule
\sphinxtableatstartofbodyhook
\sphinxAtStartPar
位域 |
&
\sphinxAtStartPar
名称     | |
&
\sphinxAtStartPar
描述                                        | |
\\
\sphinxhline
\sphinxAtStartPar
31:25
&\begin{itemize}
\item {} 
\end{itemize}
&\begin{itemize}
\item {} 
\end{itemize}
\\
\sphinxhline
\sphinxAtStartPar
24:21
&
\sphinxAtStartPar
AVG
&
\sphinxAtStartPar
一次启动ADC采样次数配置寄存器               |

\sphinxAtStartPar
0000:1次采样                               |

\sphinxAtStartPar
0001:2次采样并取平均                       |

\sphinxAtStartPar
0010:保留(不可配置)                      |

\sphinxAtStartPar
0011:4次采样并取平均                       |

\sphinxAtStartPar
0100、0101、0110:保留(不可配置)          |

\sphinxAtStartPar
0111:8次采样并取平均                       |

\sphinxAtStartPar
1000、1001、                                | 0、1011、1100、1101、1110:保留(不可配置) |

\sphinxAtStartPar
1111:16次采样并取平均                      |
\\
\sphinxhline
\sphinxAtStartPar
20
&
\sphinxAtStartPar
RST
&
\sphinxAtStartPar
ADC复位                                     |

\sphinxAtStartPar
0:正常                                     |

\sphinxAtStartPar
1:复位                                     |
\\
\sphinxhline
\sphinxAtStartPar
19
&
\sphinxAtStartPar
FFCLR
&
\sphinxAtStartPar
FIFO清除使能                                |

\sphinxAtStartPar
0:FIFO正常工作;                           |

\sphinxAtStartPar
1:FIFO复位;                               |
\\
\sphinxhline
\sphinxAtStartPar
18
&
\sphinxAtStartPar
RES2FF
&
\sphinxAtStartPar
0:ADC数据存储为通道模式;                  |

\sphinxAtStartPar
1:ADC数据存储为FIFO模式;                  |

\sphinxAtStartPar
DMA模式必须使用FIFO模式;                   |
\\
\sphinxhline
\sphinxAtStartPar
17
&
\sphinxAtStartPar
DMAEN
&
\sphinxAtStartPar
DMA使能,高电平有效                         |

\sphinxAtStartPar
仅支持FIFO模式                              |

\sphinxAtStartPar
0:只能通过CPU读取FFDATA;                  |

\sphinxAtStartPar
1:只能通过DMA读取FFDATA;                  |
\\
\sphinxhline
\sphinxAtStartPar
16:14
&
\sphinxAtStartPar
TRIG
&
\sphinxAtStartPar
ADC triger方式选择                          |

\sphinxAtStartPar
000:CPU触发                                |

\sphinxAtStartPar
001:PWM触发                                |

\sphinxAtStartPar
010:TIMER2触发                             |

\sphinxAtStartPar
011:TIMER3触发                             |
\\
\sphinxhline
\sphinxAtStartPar
13
&
\sphinxAtStartPar
CONT
&
\sphinxAtStartPar
ADC工作模式(只在CPU触发方式下有效)        |

\sphinxAtStartPar
0:单次模式                                 |

\sphinxAtStartPar
1:连续模式                                 |
\\
\sphinxhline
\sphinxAtStartPar
12
&
\sphinxAtStartPar
EN
&
\sphinxAtStartPar
ADC使能                                     |

\sphinxAtStartPar
1:使能                                     |

\sphinxAtStartPar
0:禁能                                     |
\\
\sphinxhline
\sphinxAtStartPar
11
&
\sphinxAtStartPar
CH11
&
\sphinxAtStartPar
ADC通道11选择控制                           |

\sphinxAtStartPar
0:通道未选中 1:通道选中                   |
\\
\sphinxhline
\sphinxAtStartPar
10
&
\sphinxAtStartPar
CH10
&
\sphinxAtStartPar
ADC通道10选择控制                           |

\sphinxAtStartPar
0:通道未选中 1:通道选中                   |
\\
\sphinxhline
\sphinxAtStartPar
9
&
\sphinxAtStartPar
CH9
&
\sphinxAtStartPar
ADC通道9选择控制                            |

\sphinxAtStartPar
0:通道未选中 1:通道选中                   |
\\
\sphinxhline
\sphinxAtStartPar
8
&
\sphinxAtStartPar
CH8
&
\sphinxAtStartPar
ADC通道8选择控制                            |

\sphinxAtStartPar
0:通道未选中 1:通道选中                   |
\\
\sphinxhline
\sphinxAtStartPar
7
&
\sphinxAtStartPar
CH7
&
\sphinxAtStartPar
ADC通道7选择控制                            |

\sphinxAtStartPar
0:通道未选中 1:通道选中                   |
\\
\sphinxhline
\sphinxAtStartPar
6
&
\sphinxAtStartPar
CH6
&
\sphinxAtStartPar
ADC通道6选择控制                            |

\sphinxAtStartPar
0:通道未选中 1:通道选中                   |
\\
\sphinxhline
\sphinxAtStartPar
5
&
\sphinxAtStartPar
CH5
&
\sphinxAtStartPar
ADC通道5选择控制                            |

\sphinxAtStartPar
0:通道未选中 1:通道选中                   |
\\
\sphinxhline
\sphinxAtStartPar
4
&
\sphinxAtStartPar
CH4
&
\sphinxAtStartPar
ADC通道4选择控制                            |

\sphinxAtStartPar
0:通道未选中 1:通道选中                   |
\\
\sphinxhline
\sphinxAtStartPar
3
&
\sphinxAtStartPar
CH3
&
\sphinxAtStartPar
ADC通道3选择控制                            |

\sphinxAtStartPar
0:通道未选中 1:通道选中                   |
\\
\sphinxhline
\sphinxAtStartPar
2
&
\sphinxAtStartPar
CH2
&
\sphinxAtStartPar
ADC通道2选择控制                            |

\sphinxAtStartPar
0:通道未选中 1:通道选中                   |
\\
\sphinxhline
\sphinxAtStartPar
1
&
\sphinxAtStartPar
CH1
&
\sphinxAtStartPar
ADC通道1选择控制                            |

\sphinxAtStartPar
0:通道未选中 1:通道选中                   |
\\
\sphinxhline
\sphinxAtStartPar
0
&
\sphinxAtStartPar
CH0
&
\sphinxAtStartPar
ADC通道0选择控制                            |

\sphinxAtStartPar
0:通道未选中 1:通道选中                   |
\\
\sphinxbottomrule
\end{tabular}
\sphinxtableafterendhook\par
\sphinxattableend\end{savenotes}


\subsubsection{启动寄存器START}
\label{\detokenize{SWM241/_u529f_u80fd_u63cf_u8ff0/_u6a21_u62df_u6570_u5b57_u8f6c_u6362_u5668:start}}

\begin{savenotes}\sphinxattablestart
\sphinxthistablewithglobalstyle
\centering
\begin{tabular}[t]{\X{20}{100}\X{20}{100}\X{20}{100}\X{20}{100}\X{20}{100}}
\sphinxtoprule
\sphinxtableatstartofbodyhook
\sphinxAtStartPar
寄存器 |
&
\begin{DUlineblock}{0em}
\item[] 偏移 |
\end{DUlineblock}
&
\begin{DUlineblock}{0em}
\item[] 
\item[] {\color{red}\bfseries{}|}
\end{DUlineblock}
&
\sphinxAtStartPar
复位值 |    描 | |
&
\begin{DUlineblock}{0em}
\item[] |
  |
\end{DUlineblock}
\\
\sphinxhline
\sphinxAtStartPar
START
&
\sphinxAtStartPar
0x04
&&
\sphinxAtStartPar
0 000000
&
\sphinxAtStartPar
ADC启动寄存器              |
\\
\sphinxbottomrule
\end{tabular}
\sphinxtableafterendhook\par
\sphinxattableend\end{savenotes}


\begin{savenotes}\sphinxattablestart
\sphinxthistablewithglobalstyle
\centering
\begin{tabular}[t]{\X{12}{96}\X{12}{96}\X{12}{96}\X{12}{96}\X{12}{96}\X{12}{96}\X{12}{96}\X{12}{96}}
\sphinxtoprule
\sphinxtableatstartofbodyhook
\sphinxAtStartPar
31
&
\sphinxAtStartPar
30
&
\sphinxAtStartPar
29
&
\sphinxAtStartPar
28
&
\sphinxAtStartPar
27
&
\sphinxAtStartPar
26
&
\sphinxAtStartPar
25
&
\sphinxAtStartPar
24
\\
\sphinxhline\begin{itemize}
\item {} 
\end{itemize}
&&&&&&&\\
\sphinxhline
\sphinxAtStartPar
23
&
\sphinxAtStartPar
22
&
\sphinxAtStartPar
21
&
\sphinxAtStartPar
20
&
\sphinxAtStartPar
19
&
\sphinxAtStartPar
18
&
\sphinxAtStartPar
17
&
\sphinxAtStartPar
16
\\
\sphinxhline\begin{itemize}
\item {} 
\end{itemize}
&&&&&&&\\
\sphinxhline
\sphinxAtStartPar
15
&
\sphinxAtStartPar
14
&
\sphinxAtStartPar
13
&
\sphinxAtStartPar
12
&
\sphinxAtStartPar
11
&
\sphinxAtStartPar
10
&
\sphinxAtStartPar
9
&
\sphinxAtStartPar
8
\\
\sphinxhline\begin{itemize}
\item {} 
\end{itemize}
&&&&&&&\\
\sphinxhline
\sphinxAtStartPar
7
&
\sphinxAtStartPar
6
&
\sphinxAtStartPar
5
&
\sphinxAtStartPar
4
&
\sphinxAtStartPar
3
&
\sphinxAtStartPar
2
&
\sphinxAtStartPar
1
&
\sphinxAtStartPar
0
\\
\sphinxhline\begin{itemize}
\item {} 
\end{itemize}
&&&&\begin{itemize}
\item {} 
\end{itemize}
&&&
\sphinxAtStartPar
GO
\\
\sphinxbottomrule
\end{tabular}
\sphinxtableafterendhook\par
\sphinxattableend\end{savenotes}


\begin{savenotes}\sphinxattablestart
\sphinxthistablewithglobalstyle
\centering
\begin{tabular}[t]{\X{33}{99}\X{33}{99}\X{33}{99}}
\sphinxtoprule
\sphinxtableatstartofbodyhook
\sphinxAtStartPar
位域 |
&
\sphinxAtStartPar
名称     | |
&
\sphinxAtStartPar
描述                                        | |
\\
\sphinxhline
\sphinxAtStartPar
31:5
&\begin{itemize}
\item {} 
\end{itemize}
&\begin{itemize}
\item {} 
\end{itemize}
\\
\sphinxhline
\sphinxAtStartPar
4
&
\sphinxAtStartPar
BUSY
&
\sphinxAtStartPar
ADC工作状态标识                             |
\\
\sphinxhline
\sphinxAtStartPar
3:1
&\begin{itemize}
\item {} 
\end{itemize}
&\begin{itemize}
\item {} 
\end{itemize}
\\
\sphinxhline
\sphinxAtStartPar
0
&
\sphinxAtStartPar
GO
&
\sphinxAtStartPar
ADC启动信号(只在CPU触发方式下有效)        |

\sphinxAtStartPar
该位写1,则启动一次转换。                   |

\sphinxAtStartPar
若CON                                       | 单次采样模式,则该位置1后,将对有效通道依 | 进行采样转换,并将转换的数据保存在相应通 | IFO或寄存器中。转换完成后硬件会自动清零。 |

\sphinxAtStartPar
若CO                                        | 于多次采样模式,则该位置1表示启动ADC转换, | 表示停止ADC转换。启动ADC转换后,将对有效 | 次轮询进行采样转换,并将转换的数据保存在 | 道的FIFO或寄存器中。每次转换完成后判断该 |
为1,若为1则继续转换,若为0则停止转换。  |
\\
\sphinxbottomrule
\end{tabular}
\sphinxtableafterendhook\par
\sphinxattableend\end{savenotes}


\subsubsection{中断寄存器IE}
\label{\detokenize{SWM241/_u529f_u80fd_u63cf_u8ff0/_u6a21_u62df_u6570_u5b57_u8f6c_u6362_u5668:ie}}

\begin{savenotes}\sphinxattablestart
\sphinxthistablewithglobalstyle
\centering
\begin{tabular}[t]{\X{20}{100}\X{20}{100}\X{20}{100}\X{20}{100}\X{20}{100}}
\sphinxtoprule
\sphinxtableatstartofbodyhook
\sphinxAtStartPar
寄存器 |
&
\begin{DUlineblock}{0em}
\item[] 偏移 |
\end{DUlineblock}
&
\begin{DUlineblock}{0em}
\item[] 
\item[] {\color{red}\bfseries{}|}
\end{DUlineblock}
&
\sphinxAtStartPar
复位值 |    描 | |
&
\begin{DUlineblock}{0em}
\item[] |
  |
\end{DUlineblock}
\\
\sphinxhline
\sphinxAtStartPar
IE
&
\sphinxAtStartPar
0x08
&&
\sphinxAtStartPar
0 000000
&
\sphinxAtStartPar
ADC中断使能寄存器          |
\\
\sphinxbottomrule
\end{tabular}
\sphinxtableafterendhook\par
\sphinxattableend\end{savenotes}


\begin{savenotes}\sphinxattablestart
\sphinxthistablewithglobalstyle
\centering
\begin{tabular}[t]{\X{12}{96}\X{12}{96}\X{12}{96}\X{12}{96}\X{12}{96}\X{12}{96}\X{12}{96}\X{12}{96}}
\sphinxtoprule
\sphinxtableatstartofbodyhook
\sphinxAtStartPar
31
&
\sphinxAtStartPar
30
&
\sphinxAtStartPar
29
&
\sphinxAtStartPar
28
&
\sphinxAtStartPar
27
&
\sphinxAtStartPar
26
&
\sphinxAtStartPar
25
&
\sphinxAtStartPar
24
\\
\sphinxhline\begin{itemize}
\item {} 
\end{itemize}
&&&&&&&\\
\sphinxhline
\sphinxAtStartPar
23
&
\sphinxAtStartPar
22
&
\sphinxAtStartPar
21
&
\sphinxAtStartPar
20
&
\sphinxAtStartPar
19
&
\sphinxAtStartPar
18
&
\sphinxAtStartPar
17
&
\sphinxAtStartPar
16
\\
\sphinxhline
\sphinxAtStartPar
CH11OVF
&
\sphinxAtStartPar
C H11EOC
&
\sphinxAtStartPar
C OVF
&
\sphinxAtStartPar
C EOC
&&&&\\
\sphinxhline
\sphinxAtStartPar
15
&
\sphinxAtStartPar
14
&
\sphinxAtStartPar
13
&
\sphinxAtStartPar
12
&
\sphinxAtStartPar
11
&
\sphinxAtStartPar
10
&
\sphinxAtStartPar
9
&
\sphinxAtStartPar
8
\\
\sphinxhline
\sphinxAtStartPar
CH7OVF
&
\sphinxAtStartPar
CH7EOC
&&&&&&\\
\sphinxhline
\sphinxAtStartPar
7
&
\sphinxAtStartPar
6
&
\sphinxAtStartPar
5
&
\sphinxAtStartPar
4
&
\sphinxAtStartPar
3
&
\sphinxAtStartPar
2
&
\sphinxAtStartPar
1
&
\sphinxAtStartPar
0
\\
\sphinxhline
\sphinxAtStartPar
CH3OVF
&
\sphinxAtStartPar
CH3EOC
&&&&&&\\
\sphinxbottomrule
\end{tabular}
\sphinxtableafterendhook\par
\sphinxattableend\end{savenotes}


\begin{savenotes}\sphinxattablestart
\sphinxthistablewithglobalstyle
\centering
\begin{tabular}[t]{\X{33}{99}\X{33}{99}\X{33}{99}}
\sphinxtoprule
\sphinxtableatstartofbodyhook
\sphinxAtStartPar
位域 |
&
\sphinxAtStartPar
名称     | |
&
\sphinxAtStartPar
描述                                        | |
\\
\sphinxhline
\sphinxAtStartPar
31:27
&\begin{itemize}
\item {} 
\end{itemize}
&\begin{itemize}
\item {} 
\end{itemize}
\\
\sphinxhline
\sphinxAtStartPar
26
&
\sphinxAtStartPar
FIFOF
&
\sphinxAtStartPar
ADC数据FIFO满中断使能                       |

\sphinxAtStartPar
1:使能                                     |

\sphinxAtStartPar
0:禁能                                     |
\\
\sphinxhline
\sphinxAtStartPar
25
&
\sphinxAtStartPar
FIFOHF
&
\sphinxAtStartPar
ADC数据FIFO半满中断使能                     |

\sphinxAtStartPar
1:使能                                     |

\sphinxAtStartPar
0:禁能                                     |
\\
\sphinxhline
\sphinxAtStartPar
24
&
\sphinxAtStartPar
FIFOOV
&
\sphinxAtStartPar
ADC数据FIFO溢出中断使能                     |

\sphinxAtStartPar
1:使能                                     |

\sphinxAtStartPar
0:禁能                                     |
\\
\sphinxhline
\sphinxAtStartPar
23
&
\sphinxAtStartPar
CH11OVF
&
\sphinxAtStartPar
ADC通道11数据寄存器溢出中断使能             |

\sphinxAtStartPar
1:使能                                     |

\sphinxAtStartPar
0:禁能                                     |
\\
\sphinxhline
\sphinxAtStartPar
22
&
\sphinxAtStartPar
CH11EOC
&
\sphinxAtStartPar
ADC通道11数据转换完成中断使能               |

\sphinxAtStartPar
1:使能                                     |

\sphinxAtStartPar
0:禁能                                     |
\\
\sphinxhline
\sphinxAtStartPar
21
&
\sphinxAtStartPar
CH10OVF
&
\sphinxAtStartPar
ADC通道10数据寄存器溢出中断使能             |

\sphinxAtStartPar
1:使能                                     |

\sphinxAtStartPar
0:禁能                                     |
\\
\sphinxhline
\sphinxAtStartPar
20
&
\sphinxAtStartPar
CH10EOC
&
\sphinxAtStartPar
ADC通道10数据转换完成中断使能               |

\sphinxAtStartPar
1:使能                                     |

\sphinxAtStartPar
0:禁能                                     |
\\
\sphinxhline
\sphinxAtStartPar
19
&
\sphinxAtStartPar
CH9OVF
&
\sphinxAtStartPar
ADC通道9数据寄存器溢出中断使能              |

\sphinxAtStartPar
1:使能                                     |

\sphinxAtStartPar
0:禁能                                     |
\\
\sphinxhline
\sphinxAtStartPar
18
&
\sphinxAtStartPar
CH9EOC
&
\sphinxAtStartPar
ADC通道9数据转换完成中断使能                |

\sphinxAtStartPar
1:使能                                     |

\sphinxAtStartPar
0:禁能                                     |
\\
\sphinxhline
\sphinxAtStartPar
17
&
\sphinxAtStartPar
CH8OVF
&
\sphinxAtStartPar
ADC通道8数据寄存器溢出中断使能              |

\sphinxAtStartPar
1:使能                                     |

\sphinxAtStartPar
0:禁能                                     |
\\
\sphinxhline
\sphinxAtStartPar
16
&
\sphinxAtStartPar
CH8EOC
&
\sphinxAtStartPar
ADC通道8数据转换完成中断使能                |

\sphinxAtStartPar
1:使能                                     |

\sphinxAtStartPar
0:禁能                                     |
\\
\sphinxhline
\sphinxAtStartPar
15
&
\sphinxAtStartPar
CH7OVF
&
\sphinxAtStartPar
ADC通道7数据寄存器溢出中断使能              |

\sphinxAtStartPar
1:使能                                     |

\sphinxAtStartPar
0:禁能                                     |
\\
\sphinxhline
\sphinxAtStartPar
14
&
\sphinxAtStartPar
CH7EOC
&
\sphinxAtStartPar
ADC通道7数据转换完成中断使能                |

\sphinxAtStartPar
1:使能                                     |

\sphinxAtStartPar
0:禁能                                     |
\\
\sphinxhline
\sphinxAtStartPar
13
&
\sphinxAtStartPar
CH6OVF
&
\sphinxAtStartPar
ADC通道6数据寄存器溢出中断使能              |

\sphinxAtStartPar
1:使能                                     |

\sphinxAtStartPar
0:禁能                                     |
\\
\sphinxhline
\sphinxAtStartPar
12
&
\sphinxAtStartPar
CH6EOC
&
\sphinxAtStartPar
ADC通道6数据转换完成中断使能                |

\sphinxAtStartPar
1:使能                                     |

\sphinxAtStartPar
0:禁能                                     |
\\
\sphinxhline
\sphinxAtStartPar
11
&
\sphinxAtStartPar
CH5OVF
&
\sphinxAtStartPar
ADC通道5数据寄存器溢出中断使能              |

\sphinxAtStartPar
1:使能                                     |

\sphinxAtStartPar
0:禁能                                     |
\\
\sphinxhline
\sphinxAtStartPar
10
&
\sphinxAtStartPar
CH5EOC
&
\sphinxAtStartPar
ADC通道5数据转换完成中断使能                |

\sphinxAtStartPar
1:使能                                     |

\sphinxAtStartPar
0:禁能                                     |
\\
\sphinxhline
\sphinxAtStartPar
9
&
\sphinxAtStartPar
CH4OVF
&
\sphinxAtStartPar
ADC通道4数据寄存器溢出中断使能              |

\sphinxAtStartPar
1:使能                                     |

\sphinxAtStartPar
0:禁能                                     |
\\
\sphinxhline
\sphinxAtStartPar
8
&
\sphinxAtStartPar
CH4EOC
&
\sphinxAtStartPar
ADC通道4数据转换完成中断使能                |

\sphinxAtStartPar
1:使能                                     |

\sphinxAtStartPar
0:禁能                                     |
\\
\sphinxhline
\sphinxAtStartPar
7
&
\sphinxAtStartPar
CH3OVF
&
\sphinxAtStartPar
ADC通道3数据寄存器溢出中断使能              |

\sphinxAtStartPar
1:使能                                     |

\sphinxAtStartPar
0:禁能                                     |
\\
\sphinxhline
\sphinxAtStartPar
6
&
\sphinxAtStartPar
CH3EOC
&
\sphinxAtStartPar
ADC通道3数据转换完成中断使能                |

\sphinxAtStartPar
1:使能                                     |

\sphinxAtStartPar
0:禁能                                     |
\\
\sphinxhline
\sphinxAtStartPar
5
&
\sphinxAtStartPar
CH2OVF
&
\sphinxAtStartPar
ADC通道2数据寄存器溢出中断使能              |

\sphinxAtStartPar
1:使能                                     |

\sphinxAtStartPar
0:禁能                                     |
\\
\sphinxhline
\sphinxAtStartPar
4
&
\sphinxAtStartPar
CH2EOC
&
\sphinxAtStartPar
ADC通道2数据转换完成中断使能                |

\sphinxAtStartPar
1:使能                                     |

\sphinxAtStartPar
0:禁能                                     |
\\
\sphinxhline
\sphinxAtStartPar
3
&
\sphinxAtStartPar
CH1OVF
&
\sphinxAtStartPar
ADC通道1数据寄存器溢出中断使能              |

\sphinxAtStartPar
1:使能                                     |

\sphinxAtStartPar
0:禁能                                     |
\\
\sphinxhline
\sphinxAtStartPar
2
&
\sphinxAtStartPar
CH1EOC
&
\sphinxAtStartPar
ADC通道1数据转换完成中断使能                |

\sphinxAtStartPar
1:使能                                     |

\sphinxAtStartPar
0:禁能                                     |
\\
\sphinxhline
\sphinxAtStartPar
1
&
\sphinxAtStartPar
CH0OVF
&
\sphinxAtStartPar
ADC通道0数据寄存器溢出中断使能              |

\sphinxAtStartPar
1:使能                                     |

\sphinxAtStartPar
0:禁能                                     |
\\
\sphinxhline
\sphinxAtStartPar
0
&
\sphinxAtStartPar
CH0EOC
&
\sphinxAtStartPar
ADC通道0数据转换完成中断使能                |

\sphinxAtStartPar
1:使能                                     |

\sphinxAtStartPar
0:禁能                                     |
\\
\sphinxbottomrule
\end{tabular}
\sphinxtableafterendhook\par
\sphinxattableend\end{savenotes}


\subsubsection{中断状态寄存器IF}
\label{\detokenize{SWM241/_u529f_u80fd_u63cf_u8ff0/_u6a21_u62df_u6570_u5b57_u8f6c_u6362_u5668:if}}

\begin{savenotes}\sphinxattablestart
\sphinxthistablewithglobalstyle
\centering
\begin{tabular}[t]{\X{20}{100}\X{20}{100}\X{20}{100}\X{20}{100}\X{20}{100}}
\sphinxtoprule
\sphinxtableatstartofbodyhook
\sphinxAtStartPar
寄存器 |
&
\begin{DUlineblock}{0em}
\item[] 偏移 |
\end{DUlineblock}
&
\begin{DUlineblock}{0em}
\item[] 
\item[] {\color{red}\bfseries{}|}
\end{DUlineblock}
&
\sphinxAtStartPar
复位值 |    描 | |
&
\begin{DUlineblock}{0em}
\item[] |
  |
\end{DUlineblock}
\\
\sphinxhline
\sphinxAtStartPar
IF
&
\sphinxAtStartPar
0x0C
&&
\sphinxAtStartPar
0 000000
&
\sphinxAtStartPar
ADC中断状态寄存器          |
\\
\sphinxbottomrule
\end{tabular}
\sphinxtableafterendhook\par
\sphinxattableend\end{savenotes}


\begin{savenotes}\sphinxattablestart
\sphinxthistablewithglobalstyle
\centering
\begin{tabular}[t]{\X{12}{96}\X{12}{96}\X{12}{96}\X{12}{96}\X{12}{96}\X{12}{96}\X{12}{96}\X{12}{96}}
\sphinxtoprule
\sphinxtableatstartofbodyhook
\sphinxAtStartPar
31
&
\sphinxAtStartPar
30
&
\sphinxAtStartPar
29
&
\sphinxAtStartPar
28
&
\sphinxAtStartPar
27
&
\sphinxAtStartPar
26
&
\sphinxAtStartPar
25
&
\sphinxAtStartPar
24
\\
\sphinxhline\begin{itemize}
\item {} 
\end{itemize}
&&&&&&&\\
\sphinxhline
\sphinxAtStartPar
23
&
\sphinxAtStartPar
22
&
\sphinxAtStartPar
21
&
\sphinxAtStartPar
20
&
\sphinxAtStartPar
19
&
\sphinxAtStartPar
18
&
\sphinxAtStartPar
17
&
\sphinxAtStartPar
16
\\
\sphinxhline
\sphinxAtStartPar
CH11OVF
&
\sphinxAtStartPar
C H11EOC
&
\sphinxAtStartPar
C OVF
&
\sphinxAtStartPar
C EOC
&&&&\\
\sphinxhline
\sphinxAtStartPar
15
&
\sphinxAtStartPar
14
&
\sphinxAtStartPar
13
&
\sphinxAtStartPar
12
&
\sphinxAtStartPar
11
&
\sphinxAtStartPar
10
&
\sphinxAtStartPar
9
&
\sphinxAtStartPar
8
\\
\sphinxhline
\sphinxAtStartPar
CH7OVF
&
\sphinxAtStartPar
CH7EOC
&&&&&&\\
\sphinxhline
\sphinxAtStartPar
7
&
\sphinxAtStartPar
6
&
\sphinxAtStartPar
5
&
\sphinxAtStartPar
4
&
\sphinxAtStartPar
3
&
\sphinxAtStartPar
2
&
\sphinxAtStartPar
1
&
\sphinxAtStartPar
0
\\
\sphinxhline
\sphinxAtStartPar
CH3OVF
&
\sphinxAtStartPar
CH3EOC
&&&&&&\\
\sphinxbottomrule
\end{tabular}
\sphinxtableafterendhook\par
\sphinxattableend\end{savenotes}


\begin{savenotes}\sphinxattablestart
\sphinxthistablewithglobalstyle
\centering
\begin{tabular}[t]{\X{33}{99}\X{33}{99}\X{33}{99}}
\sphinxtoprule
\sphinxtableatstartofbodyhook
\sphinxAtStartPar
位域 |
&
\sphinxAtStartPar
名称     | |
&
\sphinxAtStartPar
描述                                        | |
\\
\sphinxhline
\sphinxAtStartPar
31:27
&\begin{itemize}
\item {} 
\end{itemize}
&\begin{itemize}
\item {} 
\end{itemize}
\\
\sphinxhline
\sphinxAtStartPar
26
&
\sphinxAtStartPar
FIFOF
&
\sphinxAtStartPar
ADC数据FIFO满中断状态,写1清除              |

\sphinxAtStartPar
0:未产生                                   |

\sphinxAtStartPar
1:产生中断                                 |
\\
\sphinxhline
\sphinxAtStartPar
25
&
\sphinxAtStartPar
FIFOHF
&
\sphinxAtStartPar
ADC数据FIFO半满中断状态,写1清除            |

\sphinxAtStartPar
0:未产生                                   |

\sphinxAtStartPar
1:产生中断                                 |
\\
\sphinxhline
\sphinxAtStartPar
24
&
\sphinxAtStartPar
FIFOOV
&
\sphinxAtStartPar
ADC数据FIFO溢出中断状态,写1清除            |

\sphinxAtStartPar
0:未产生                                   |

\sphinxAtStartPar
1:产生中断                                 |
\\
\sphinxhline
\sphinxAtStartPar
23
&
\sphinxAtStartPar
CH11OVF
&
\sphinxAtStartPar
ADC通道11数据寄存器溢出中断状态,写1清除    |

\sphinxAtStartPar
0:未产生                                   |

\sphinxAtStartPar
1:产生中断                                 |
\\
\sphinxhline
\sphinxAtStartPar
22
&
\sphinxAtStartPar
CH11EOC
&
\sphinxAtStartPar
ADC通道11数据转换完成中断状态,写1清除      |

\sphinxAtStartPar
0:未产生                                   |

\sphinxAtStartPar
1:产生中断                                 |
\\
\sphinxhline
\sphinxAtStartPar
21
&
\sphinxAtStartPar
CH10OVF
&
\sphinxAtStartPar
ADC通道10数据寄存器溢出中断状态,写1清除    |

\sphinxAtStartPar
0:未产生                                   |

\sphinxAtStartPar
1:产生中断                                 |
\\
\sphinxhline
\sphinxAtStartPar
20
&
\sphinxAtStartPar
CH10EOC
&
\sphinxAtStartPar
ADC通道10数据转换完成中断状态,写1清除      |

\sphinxAtStartPar
0:未产生                                   |

\sphinxAtStartPar
1:产生中断                                 |
\\
\sphinxhline
\sphinxAtStartPar
19
&
\sphinxAtStartPar
CH9OVF
&
\sphinxAtStartPar
ADC通道9数据寄存器溢出中断状态,写1清除     |

\sphinxAtStartPar
0:未产生                                   |

\sphinxAtStartPar
1:产生中断                                 |
\\
\sphinxhline
\sphinxAtStartPar
18
&
\sphinxAtStartPar
CH9EOC
&
\sphinxAtStartPar
ADC通道9数据转换完成中断状态,写1清除       |

\sphinxAtStartPar
0:未产生                                   |

\sphinxAtStartPar
1:产生中断                                 |
\\
\sphinxhline
\sphinxAtStartPar
17
&
\sphinxAtStartPar
CH8OVF
&
\sphinxAtStartPar
ADC通道8数据寄存器溢出中断状态,写1清除     |

\sphinxAtStartPar
0:未产生                                   |

\sphinxAtStartPar
1:产生中断                                 |
\\
\sphinxhline
\sphinxAtStartPar
16
&
\sphinxAtStartPar
CH8EOC
&
\sphinxAtStartPar
ADC通道8数据转换完成中断状态,写1清除       |

\sphinxAtStartPar
0:未产生                                   |

\sphinxAtStartPar
1:产生中断                                 |
\\
\sphinxhline
\sphinxAtStartPar
15
&
\sphinxAtStartPar
CH7OVF
&
\sphinxAtStartPar
ADC通道7数据寄存器溢出中断状态,写1清除     |

\sphinxAtStartPar
0:未产生                                   |

\sphinxAtStartPar
1:产生中断                                 |
\\
\sphinxhline
\sphinxAtStartPar
14
&
\sphinxAtStartPar
CH7EOC
&
\sphinxAtStartPar
ADC通道7数据转换完成中断状态,写1清除       |

\sphinxAtStartPar
0:未产生                                   |

\sphinxAtStartPar
1:产生中断                                 |
\\
\sphinxhline
\sphinxAtStartPar
13
&
\sphinxAtStartPar
CH6OVF
&
\sphinxAtStartPar
ADC通道6数据寄存器溢出中断状态,写1清除     |

\sphinxAtStartPar
0:未产生                                   |

\sphinxAtStartPar
1:产生中断                                 |
\\
\sphinxhline
\sphinxAtStartPar
12
&
\sphinxAtStartPar
CH6EOC
&
\sphinxAtStartPar
ADC通道6数据转换完成中断状态,写1清除       |

\sphinxAtStartPar
0:未产生                                   |

\sphinxAtStartPar
1:产生中断                                 |
\\
\sphinxhline
\sphinxAtStartPar
11
&
\sphinxAtStartPar
CH5OVF
&
\sphinxAtStartPar
ADC通道5数据寄存器溢出中断状态,写1清除     |

\sphinxAtStartPar
0:未产生                                   |

\sphinxAtStartPar
1:产生中断                                 |
\\
\sphinxhline
\sphinxAtStartPar
10
&
\sphinxAtStartPar
CH5EOC
&
\sphinxAtStartPar
ADC通道5数据转换完成中断状态,写1清除       |

\sphinxAtStartPar
0:未产生                                   |

\sphinxAtStartPar
1:产生中断                                 |
\\
\sphinxhline
\sphinxAtStartPar
9
&
\sphinxAtStartPar
CH4OVF
&
\sphinxAtStartPar
ADC通道4数据寄存器溢出中断状态,写1清除     |

\sphinxAtStartPar
0:未产生                                   |

\sphinxAtStartPar
1:产生中断                                 |
\\
\sphinxhline
\sphinxAtStartPar
8
&
\sphinxAtStartPar
CH4EOC
&
\sphinxAtStartPar
ADC通道4数据转换完成中断状态,写1清除       |

\sphinxAtStartPar
0:未产生                                   |

\sphinxAtStartPar
1:产生中断                                 |
\\
\sphinxhline
\sphinxAtStartPar
7
&
\sphinxAtStartPar
CH3OVF
&
\sphinxAtStartPar
ADC通道3数据寄存器溢出中断状态,写1清除     |

\sphinxAtStartPar
0:未产生                                   |

\sphinxAtStartPar
1:产生中断                                 |
\\
\sphinxhline
\sphinxAtStartPar
6
&
\sphinxAtStartPar
CH3EOC
&
\sphinxAtStartPar
ADC通道3数据转换完成中断状态,写1清除       |

\sphinxAtStartPar
0:未产生                                   |

\sphinxAtStartPar
1:产生中断                                 |
\\
\sphinxhline
\sphinxAtStartPar
5
&
\sphinxAtStartPar
CH2OVF
&
\sphinxAtStartPar
ADC通道2数据寄存器溢出中断状态,写1清除     |

\sphinxAtStartPar
0:未产生                                   |

\sphinxAtStartPar
1:产生中断                                 |
\\
\sphinxhline
\sphinxAtStartPar
4
&
\sphinxAtStartPar
CH2EOC
&
\sphinxAtStartPar
ADC通道2数据转换完成中断状态,写1清除       |

\sphinxAtStartPar
0:未产生                                   |

\sphinxAtStartPar
1:产生中断                                 |
\\
\sphinxhline
\sphinxAtStartPar
3
&
\sphinxAtStartPar
CH1OVF
&
\sphinxAtStartPar
ADC通道1数据寄存器溢出中断状态,写1清除     |

\sphinxAtStartPar
0:未产生                                   |

\sphinxAtStartPar
1:产生中断                                 |
\\
\sphinxhline
\sphinxAtStartPar
2
&
\sphinxAtStartPar
CH1EOC
&
\sphinxAtStartPar
ADC通道1数据转换完成中断状态,写1清除       |

\sphinxAtStartPar
0:未产生                                   |

\sphinxAtStartPar
1:产生中断                                 |
\\
\sphinxhline
\sphinxAtStartPar
1
&
\sphinxAtStartPar
CH0OVF
&
\sphinxAtStartPar
ADC通道0数据寄存器溢出中断状态,写1清除     |

\sphinxAtStartPar
0:未产生                                   |

\sphinxAtStartPar
1:产生中断                                 |
\\
\sphinxhline
\sphinxAtStartPar
0
&
\sphinxAtStartPar
CH0EOC
&
\sphinxAtStartPar
ADC通道0数据转换完成中断状态,写1清除       |

\sphinxAtStartPar
0:未产生                                   |

\sphinxAtStartPar
1:产生中断                                 |
\\
\sphinxbottomrule
\end{tabular}
\sphinxtableafterendhook\par
\sphinxattableend\end{savenotes}


\subsubsection{通道状态寄存器STATx(0\textasciitilde{}11)}
\label{\detokenize{SWM241/_u529f_u80fd_u63cf_u8ff0/_u6a21_u62df_u6570_u5b57_u8f6c_u6362_u5668:statx-0-11}}

\begin{savenotes}\sphinxattablestart
\sphinxthistablewithglobalstyle
\centering
\begin{tabular}[t]{\X{20}{100}\X{20}{100}\X{20}{100}\X{20}{100}\X{20}{100}}
\sphinxtoprule
\sphinxtableatstartofbodyhook
\sphinxAtStartPar
寄存器 |
&
\begin{DUlineblock}{0em}
\item[] 偏移 |
\end{DUlineblock}
&
\begin{DUlineblock}{0em}
\item[] 
\item[] {\color{red}\bfseries{}|}
\end{DUlineblock}
&
\sphinxAtStartPar
复位值 |    描 | |
&
\begin{DUlineblock}{0em}
\item[] |
  |
\end{DUlineblock}
\\
\sphinxhline
\sphinxAtStartPar
STAT0
&
\sphinxAtStartPar
0x10
&&
\sphinxAtStartPar
0 000000
&
\sphinxAtStartPar
ADC通道0状态寄存器         |
\\
\sphinxbottomrule
\end{tabular}
\sphinxtableafterendhook\par
\sphinxattableend\end{savenotes}


\begin{savenotes}\sphinxattablestart
\sphinxthistablewithglobalstyle
\centering
\begin{tabular}[t]{\X{20}{100}\X{20}{100}\X{20}{100}\X{20}{100}\X{20}{100}}
\sphinxtoprule
\sphinxtableatstartofbodyhook
\sphinxAtStartPar
寄存器 |
&
\begin{DUlineblock}{0em}
\item[] 偏移 |
\end{DUlineblock}
&
\begin{DUlineblock}{0em}
\item[] 
\item[] {\color{red}\bfseries{}|}
\end{DUlineblock}
&
\sphinxAtStartPar
复位值 |    描 | |
&
\begin{DUlineblock}{0em}
\item[] |
  |
\end{DUlineblock}
\\
\sphinxhline
\sphinxAtStartPar
STAT1
&
\sphinxAtStartPar
0x20
&&
\sphinxAtStartPar
0 000000
&
\sphinxAtStartPar
ADC通道1状态寄存器         |
\\
\sphinxbottomrule
\end{tabular}
\sphinxtableafterendhook\par
\sphinxattableend\end{savenotes}


\begin{savenotes}\sphinxattablestart
\sphinxthistablewithglobalstyle
\centering
\begin{tabular}[t]{\X{20}{100}\X{20}{100}\X{20}{100}\X{20}{100}\X{20}{100}}
\sphinxtoprule
\sphinxtableatstartofbodyhook
\sphinxAtStartPar
寄存器 |
&
\begin{DUlineblock}{0em}
\item[] 偏移 |
\end{DUlineblock}
&
\begin{DUlineblock}{0em}
\item[] 
\item[] {\color{red}\bfseries{}|}
\end{DUlineblock}
&
\sphinxAtStartPar
复位值 |    描 | |
&
\begin{DUlineblock}{0em}
\item[] |
  |
\end{DUlineblock}
\\
\sphinxhline
\sphinxAtStartPar
STAT2
&
\sphinxAtStartPar
0x30
&&
\sphinxAtStartPar
0 000000
&
\sphinxAtStartPar
ADC通道2状态寄存器         |
\\
\sphinxbottomrule
\end{tabular}
\sphinxtableafterendhook\par
\sphinxattableend\end{savenotes}


\begin{savenotes}\sphinxattablestart
\sphinxthistablewithglobalstyle
\centering
\begin{tabular}[t]{\X{20}{100}\X{20}{100}\X{20}{100}\X{20}{100}\X{20}{100}}
\sphinxtoprule
\sphinxtableatstartofbodyhook
\sphinxAtStartPar
寄存器 |
&
\begin{DUlineblock}{0em}
\item[] 偏移 |
\end{DUlineblock}
&
\begin{DUlineblock}{0em}
\item[] 
\item[] {\color{red}\bfseries{}|}
\end{DUlineblock}
&
\sphinxAtStartPar
复位值 |    描 | |
&
\begin{DUlineblock}{0em}
\item[] |
  |
\end{DUlineblock}
\\
\sphinxhline
\sphinxAtStartPar
STAT3
&
\sphinxAtStartPar
0x40
&&
\sphinxAtStartPar
0 000000
&
\sphinxAtStartPar
ADC通道3状态寄存器         |
\\
\sphinxbottomrule
\end{tabular}
\sphinxtableafterendhook\par
\sphinxattableend\end{savenotes}


\begin{savenotes}\sphinxattablestart
\sphinxthistablewithglobalstyle
\centering
\begin{tabular}[t]{\X{20}{100}\X{20}{100}\X{20}{100}\X{20}{100}\X{20}{100}}
\sphinxtoprule
\sphinxtableatstartofbodyhook
\sphinxAtStartPar
寄存器 |
&
\begin{DUlineblock}{0em}
\item[] 偏移 |
\end{DUlineblock}
&
\begin{DUlineblock}{0em}
\item[] 
\item[] {\color{red}\bfseries{}|}
\end{DUlineblock}
&
\sphinxAtStartPar
复位值 |    描 | |
&
\begin{DUlineblock}{0em}
\item[] |
  |
\end{DUlineblock}
\\
\sphinxhline
\sphinxAtStartPar
STAT4
&
\sphinxAtStartPar
0x50
&&
\sphinxAtStartPar
0 000000
&
\sphinxAtStartPar
ADC通道4状态寄存器         |
\\
\sphinxbottomrule
\end{tabular}
\sphinxtableafterendhook\par
\sphinxattableend\end{savenotes}


\begin{savenotes}\sphinxattablestart
\sphinxthistablewithglobalstyle
\centering
\begin{tabular}[t]{\X{20}{100}\X{20}{100}\X{20}{100}\X{20}{100}\X{20}{100}}
\sphinxtoprule
\sphinxtableatstartofbodyhook
\sphinxAtStartPar
寄存器 |
&
\begin{DUlineblock}{0em}
\item[] 偏移 |
\end{DUlineblock}
&
\begin{DUlineblock}{0em}
\item[] 
\item[] {\color{red}\bfseries{}|}
\end{DUlineblock}
&
\sphinxAtStartPar
复位值 |    描 | |
&
\begin{DUlineblock}{0em}
\item[] |
  |
\end{DUlineblock}
\\
\sphinxhline
\sphinxAtStartPar
STAT5
&
\sphinxAtStartPar
0x60
&&
\sphinxAtStartPar
0 000000
&
\sphinxAtStartPar
ADC通道5状态寄存器         |
\\
\sphinxbottomrule
\end{tabular}
\sphinxtableafterendhook\par
\sphinxattableend\end{savenotes}


\begin{savenotes}\sphinxattablestart
\sphinxthistablewithglobalstyle
\centering
\begin{tabular}[t]{\X{20}{100}\X{20}{100}\X{20}{100}\X{20}{100}\X{20}{100}}
\sphinxtoprule
\sphinxtableatstartofbodyhook
\sphinxAtStartPar
寄存器 |
&
\begin{DUlineblock}{0em}
\item[] 偏移 |
\end{DUlineblock}
&
\begin{DUlineblock}{0em}
\item[] 
\item[] {\color{red}\bfseries{}|}
\end{DUlineblock}
&
\sphinxAtStartPar
复位值 |    描 | |
&
\begin{DUlineblock}{0em}
\item[] |
  |
\end{DUlineblock}
\\
\sphinxhline
\sphinxAtStartPar
STAT6
&
\sphinxAtStartPar
0x70
&&
\sphinxAtStartPar
0 000000
&
\sphinxAtStartPar
ADC通道6状态寄存器         |
\\
\sphinxbottomrule
\end{tabular}
\sphinxtableafterendhook\par
\sphinxattableend\end{savenotes}


\begin{savenotes}\sphinxattablestart
\sphinxthistablewithglobalstyle
\centering
\begin{tabular}[t]{\X{20}{100}\X{20}{100}\X{20}{100}\X{20}{100}\X{20}{100}}
\sphinxtoprule
\sphinxtableatstartofbodyhook
\sphinxAtStartPar
寄存器 |
&
\begin{DUlineblock}{0em}
\item[] 偏移 |
\end{DUlineblock}
&
\begin{DUlineblock}{0em}
\item[] 
\item[] {\color{red}\bfseries{}|}
\end{DUlineblock}
&
\sphinxAtStartPar
复位值 |    描 | |
&
\begin{DUlineblock}{0em}
\item[] |
  |
\end{DUlineblock}
\\
\sphinxhline
\sphinxAtStartPar
STAT7
&
\sphinxAtStartPar
0x80
&&
\sphinxAtStartPar
0 000000
&
\sphinxAtStartPar
ADC通道7状态寄存器         |
\\
\sphinxbottomrule
\end{tabular}
\sphinxtableafterendhook\par
\sphinxattableend\end{savenotes}


\begin{savenotes}\sphinxattablestart
\sphinxthistablewithglobalstyle
\centering
\begin{tabular}[t]{\X{20}{100}\X{20}{100}\X{20}{100}\X{20}{100}\X{20}{100}}
\sphinxtoprule
\sphinxtableatstartofbodyhook
\sphinxAtStartPar
寄存器 |
&
\begin{DUlineblock}{0em}
\item[] 偏移 |
\end{DUlineblock}
&
\begin{DUlineblock}{0em}
\item[] 
\item[] {\color{red}\bfseries{}|}
\end{DUlineblock}
&
\sphinxAtStartPar
复位值 |    描 | |
&
\begin{DUlineblock}{0em}
\item[] |
  |
\end{DUlineblock}
\\
\sphinxhline
\sphinxAtStartPar
STAT8
&
\sphinxAtStartPar
0x90
&&
\sphinxAtStartPar
0 000000
&
\sphinxAtStartPar
ADC通道8状态寄存器         |
\\
\sphinxbottomrule
\end{tabular}
\sphinxtableafterendhook\par
\sphinxattableend\end{savenotes}


\begin{savenotes}\sphinxattablestart
\sphinxthistablewithglobalstyle
\centering
\begin{tabular}[t]{\X{20}{100}\X{20}{100}\X{20}{100}\X{20}{100}\X{20}{100}}
\sphinxtoprule
\sphinxtableatstartofbodyhook
\sphinxAtStartPar
寄存器 |
&
\begin{DUlineblock}{0em}
\item[] 偏移 |
\end{DUlineblock}
&
\begin{DUlineblock}{0em}
\item[] 
\item[] {\color{red}\bfseries{}|}
\end{DUlineblock}
&
\sphinxAtStartPar
复位值 |    描 | |
&
\begin{DUlineblock}{0em}
\item[] |
  |
\end{DUlineblock}
\\
\sphinxhline
\sphinxAtStartPar
STAT9
&
\sphinxAtStartPar
0xa0
&&
\sphinxAtStartPar
0 000000
&
\sphinxAtStartPar
ADC通道9状态寄存器         |
\\
\sphinxbottomrule
\end{tabular}
\sphinxtableafterendhook\par
\sphinxattableend\end{savenotes}


\begin{savenotes}\sphinxattablestart
\sphinxthistablewithglobalstyle
\centering
\begin{tabular}[t]{\X{20}{100}\X{20}{100}\X{20}{100}\X{20}{100}\X{20}{100}}
\sphinxtoprule
\sphinxtableatstartofbodyhook
\sphinxAtStartPar
寄存器 |
&
\begin{DUlineblock}{0em}
\item[] 偏移 |
\end{DUlineblock}
&
\begin{DUlineblock}{0em}
\item[] 
\item[] {\color{red}\bfseries{}|}
\end{DUlineblock}
&
\sphinxAtStartPar
复位值 |    描 | |
&
\begin{DUlineblock}{0em}
\item[] |
  |
\end{DUlineblock}
\\
\sphinxhline
\sphinxAtStartPar
STAT10
&
\sphinxAtStartPar
0xb0
&&
\sphinxAtStartPar
0 000000
&
\sphinxAtStartPar
ADC通道10状态寄存器        |
\\
\sphinxbottomrule
\end{tabular}
\sphinxtableafterendhook\par
\sphinxattableend\end{savenotes}


\begin{savenotes}\sphinxattablestart
\sphinxthistablewithglobalstyle
\centering
\begin{tabular}[t]{\X{20}{100}\X{20}{100}\X{20}{100}\X{20}{100}\X{20}{100}}
\sphinxtoprule
\sphinxtableatstartofbodyhook
\sphinxAtStartPar
寄存器 |
&
\begin{DUlineblock}{0em}
\item[] 偏移 |
\end{DUlineblock}
&
\begin{DUlineblock}{0em}
\item[] 
\item[] {\color{red}\bfseries{}|}
\end{DUlineblock}
&
\sphinxAtStartPar
复位值 |    描 | |
&
\begin{DUlineblock}{0em}
\item[] |
  |
\end{DUlineblock}
\\
\sphinxhline
\sphinxAtStartPar
STAT11
&
\sphinxAtStartPar
0xc0
&&
\sphinxAtStartPar
0 000000
&
\sphinxAtStartPar
ADC通道11状态寄存器        |
\\
\sphinxbottomrule
\end{tabular}
\sphinxtableafterendhook\par
\sphinxattableend\end{savenotes}


\begin{savenotes}\sphinxattablestart
\sphinxthistablewithglobalstyle
\centering
\begin{tabular}[t]{\X{12}{96}\X{12}{96}\X{12}{96}\X{12}{96}\X{12}{96}\X{12}{96}\X{12}{96}\X{12}{96}}
\sphinxtoprule
\sphinxtableatstartofbodyhook
\sphinxAtStartPar
31
&
\sphinxAtStartPar
30
&
\sphinxAtStartPar
29
&
\sphinxAtStartPar
28
&
\sphinxAtStartPar
27
&
\sphinxAtStartPar
26
&
\sphinxAtStartPar
25
&
\sphinxAtStartPar
24
\\
\sphinxhline\begin{itemize}
\item {} 
\end{itemize}
&&&&&&&\\
\sphinxhline
\sphinxAtStartPar
23
&
\sphinxAtStartPar
22
&
\sphinxAtStartPar
21
&
\sphinxAtStartPar
20
&
\sphinxAtStartPar
19
&
\sphinxAtStartPar
18
&
\sphinxAtStartPar
17
&
\sphinxAtStartPar
16
\\
\sphinxhline\begin{itemize}
\item {} 
\end{itemize}
&&&&&&&\\
\sphinxhline
\sphinxAtStartPar
15
&
\sphinxAtStartPar
14
&
\sphinxAtStartPar
13
&
\sphinxAtStartPar
12
&
\sphinxAtStartPar
11
&
\sphinxAtStartPar
10
&
\sphinxAtStartPar
9
&
\sphinxAtStartPar
8
\\
\sphinxhline\begin{itemize}
\item {} 
\end{itemize}
&&&&&&&\\
\sphinxhline
\sphinxAtStartPar
7
&
\sphinxAtStartPar
6
&
\sphinxAtStartPar
5
&
\sphinxAtStartPar
4
&
\sphinxAtStartPar
3
&
\sphinxAtStartPar
2
&
\sphinxAtStartPar
1
&
\sphinxAtStartPar
0
\\
\sphinxhline\begin{itemize}
\item {} 
\end{itemize}
&&&&&&&
\sphinxAtStartPar
EOC
\\
\sphinxbottomrule
\end{tabular}
\sphinxtableafterendhook\par
\sphinxattableend\end{savenotes}


\begin{savenotes}\sphinxattablestart
\sphinxthistablewithglobalstyle
\centering
\begin{tabular}[t]{\X{33}{99}\X{33}{99}\X{33}{99}}
\sphinxtoprule
\sphinxtableatstartofbodyhook
\sphinxAtStartPar
位域 |
&
\sphinxAtStartPar
名称     | |
&
\sphinxAtStartPar
描述                                        | |
\\
\sphinxhline
\sphinxAtStartPar
31:2
&\begin{itemize}
\item {} 
\end{itemize}
&\begin{itemize}
\item {} 
\end{itemize}
\\
\sphinxhline
\sphinxAtStartPar
1
&
\sphinxAtStartPar
OVF
&
\sphinxAtStartPar
ADC通道x数据寄存器溢出标志                  |

\sphinxAtStartPar
1:溢出                                     |

\sphinxAtStartPar
0:未溢出                                   |

\sphinxAtStartPar
读数据寄存器清除                            |
\\
\sphinxhline
\sphinxAtStartPar
0
&
\sphinxAtStartPar
EOC
&
\sphinxAtStartPar
ADC通道x数据转换完成标志,写1清除           |

\sphinxAtStartPar
1:ADC对通道x一次采样转换完成               |

\sphinxAtStartPar
0:转换未完成                               |
\\
\sphinxbottomrule
\end{tabular}
\sphinxtableafterendhook\par
\sphinxattableend\end{savenotes}


\subsubsection{通道数据寄存器DATAx(0\textasciitilde{}11)}
\label{\detokenize{SWM241/_u529f_u80fd_u63cf_u8ff0/_u6a21_u62df_u6570_u5b57_u8f6c_u6362_u5668:datax-0-11}}

\begin{savenotes}\sphinxattablestart
\sphinxthistablewithglobalstyle
\centering
\begin{tabular}[t]{\X{20}{100}\X{20}{100}\X{20}{100}\X{20}{100}\X{20}{100}}
\sphinxtoprule
\sphinxtableatstartofbodyhook
\sphinxAtStartPar
寄存器 |
&
\begin{DUlineblock}{0em}
\item[] 偏移 |
\end{DUlineblock}
&
\begin{DUlineblock}{0em}
\item[] 
\item[] {\color{red}\bfseries{}|}
\end{DUlineblock}
&
\sphinxAtStartPar
复位值 |    描 | |
&
\begin{DUlineblock}{0em}
\item[] |
  |
\end{DUlineblock}
\\
\sphinxhline
\sphinxAtStartPar
DATA0
&
\sphinxAtStartPar
0x14
&&
\sphinxAtStartPar
0 000000
&
\sphinxAtStartPar
ADC通道0数据寄存器         |
\\
\sphinxbottomrule
\end{tabular}
\sphinxtableafterendhook\par
\sphinxattableend\end{savenotes}


\begin{savenotes}\sphinxattablestart
\sphinxthistablewithglobalstyle
\centering
\begin{tabular}[t]{\X{20}{100}\X{20}{100}\X{20}{100}\X{20}{100}\X{20}{100}}
\sphinxtoprule
\sphinxtableatstartofbodyhook
\sphinxAtStartPar
寄存器 |
&
\begin{DUlineblock}{0em}
\item[] 偏移 |
\end{DUlineblock}
&
\begin{DUlineblock}{0em}
\item[] 
\item[] {\color{red}\bfseries{}|}
\end{DUlineblock}
&
\sphinxAtStartPar
复位值 |    描 | |
&
\begin{DUlineblock}{0em}
\item[] |
  |
\end{DUlineblock}
\\
\sphinxhline
\sphinxAtStartPar
DATA1
&
\sphinxAtStartPar
0x24
&&
\sphinxAtStartPar
0 000000
&
\sphinxAtStartPar
ADC通道1数据寄存器         |
\\
\sphinxbottomrule
\end{tabular}
\sphinxtableafterendhook\par
\sphinxattableend\end{savenotes}


\begin{savenotes}\sphinxattablestart
\sphinxthistablewithglobalstyle
\centering
\begin{tabular}[t]{\X{20}{100}\X{20}{100}\X{20}{100}\X{20}{100}\X{20}{100}}
\sphinxtoprule
\sphinxtableatstartofbodyhook
\sphinxAtStartPar
寄存器 |
&
\begin{DUlineblock}{0em}
\item[] 偏移 |
\end{DUlineblock}
&
\begin{DUlineblock}{0em}
\item[] 
\item[] {\color{red}\bfseries{}|}
\end{DUlineblock}
&
\sphinxAtStartPar
复位值 |    描 | |
&
\begin{DUlineblock}{0em}
\item[] |
  |
\end{DUlineblock}
\\
\sphinxhline
\sphinxAtStartPar
DATA2
&
\sphinxAtStartPar
0x34
&&
\sphinxAtStartPar
0 000000
&
\sphinxAtStartPar
ADC通道2数据寄存器         |
\\
\sphinxbottomrule
\end{tabular}
\sphinxtableafterendhook\par
\sphinxattableend\end{savenotes}


\begin{savenotes}\sphinxattablestart
\sphinxthistablewithglobalstyle
\centering
\begin{tabular}[t]{\X{20}{100}\X{20}{100}\X{20}{100}\X{20}{100}\X{20}{100}}
\sphinxtoprule
\sphinxtableatstartofbodyhook
\sphinxAtStartPar
寄存器 |
&
\begin{DUlineblock}{0em}
\item[] 偏移 |
\end{DUlineblock}
&
\begin{DUlineblock}{0em}
\item[] 
\item[] {\color{red}\bfseries{}|}
\end{DUlineblock}
&
\sphinxAtStartPar
复位值 |    描 | |
&
\begin{DUlineblock}{0em}
\item[] |
  |
\end{DUlineblock}
\\
\sphinxhline
\sphinxAtStartPar
DATA3
&
\sphinxAtStartPar
0x44
&&
\sphinxAtStartPar
0 000000
&
\sphinxAtStartPar
ADC通道3数据寄存器         |
\\
\sphinxbottomrule
\end{tabular}
\sphinxtableafterendhook\par
\sphinxattableend\end{savenotes}


\begin{savenotes}\sphinxattablestart
\sphinxthistablewithglobalstyle
\centering
\begin{tabular}[t]{\X{20}{100}\X{20}{100}\X{20}{100}\X{20}{100}\X{20}{100}}
\sphinxtoprule
\sphinxtableatstartofbodyhook
\sphinxAtStartPar
寄存器 |
&
\begin{DUlineblock}{0em}
\item[] 偏移 |
\end{DUlineblock}
&
\begin{DUlineblock}{0em}
\item[] 
\item[] {\color{red}\bfseries{}|}
\end{DUlineblock}
&
\sphinxAtStartPar
复位值 |    描 | |
&
\begin{DUlineblock}{0em}
\item[] |
  |
\end{DUlineblock}
\\
\sphinxhline
\sphinxAtStartPar
DATA4
&
\sphinxAtStartPar
0x54
&&
\sphinxAtStartPar
0 000000
&
\sphinxAtStartPar
ADC通道4数据寄存器         |
\\
\sphinxbottomrule
\end{tabular}
\sphinxtableafterendhook\par
\sphinxattableend\end{savenotes}


\begin{savenotes}\sphinxattablestart
\sphinxthistablewithglobalstyle
\centering
\begin{tabular}[t]{\X{20}{100}\X{20}{100}\X{20}{100}\X{20}{100}\X{20}{100}}
\sphinxtoprule
\sphinxtableatstartofbodyhook
\sphinxAtStartPar
寄存器 |
&
\begin{DUlineblock}{0em}
\item[] 偏移 |
\end{DUlineblock}
&
\begin{DUlineblock}{0em}
\item[] 
\item[] {\color{red}\bfseries{}|}
\end{DUlineblock}
&
\sphinxAtStartPar
复位值 |    描 | |
&
\begin{DUlineblock}{0em}
\item[] |
  |
\end{DUlineblock}
\\
\sphinxhline
\sphinxAtStartPar
DATA5
&
\sphinxAtStartPar
0x64
&&
\sphinxAtStartPar
0 000000
&
\sphinxAtStartPar
ADC通道5数据寄存器         |
\\
\sphinxbottomrule
\end{tabular}
\sphinxtableafterendhook\par
\sphinxattableend\end{savenotes}


\begin{savenotes}\sphinxattablestart
\sphinxthistablewithglobalstyle
\centering
\begin{tabular}[t]{\X{20}{100}\X{20}{100}\X{20}{100}\X{20}{100}\X{20}{100}}
\sphinxtoprule
\sphinxtableatstartofbodyhook
\sphinxAtStartPar
寄存器 |
&
\begin{DUlineblock}{0em}
\item[] 偏移 |
\end{DUlineblock}
&
\begin{DUlineblock}{0em}
\item[] 
\item[] {\color{red}\bfseries{}|}
\end{DUlineblock}
&
\sphinxAtStartPar
复位值 |    描 | |
&
\begin{DUlineblock}{0em}
\item[] |
  |
\end{DUlineblock}
\\
\sphinxhline
\sphinxAtStartPar
DATA6
&
\sphinxAtStartPar
0x74
&&
\sphinxAtStartPar
0 000000
&
\sphinxAtStartPar
ADC通道6数据寄存器         |
\\
\sphinxbottomrule
\end{tabular}
\sphinxtableafterendhook\par
\sphinxattableend\end{savenotes}


\begin{savenotes}\sphinxattablestart
\sphinxthistablewithglobalstyle
\centering
\begin{tabular}[t]{\X{20}{100}\X{20}{100}\X{20}{100}\X{20}{100}\X{20}{100}}
\sphinxtoprule
\sphinxtableatstartofbodyhook
\sphinxAtStartPar
寄存器 |
&
\begin{DUlineblock}{0em}
\item[] 偏移 |
\end{DUlineblock}
&
\begin{DUlineblock}{0em}
\item[] 
\item[] {\color{red}\bfseries{}|}
\end{DUlineblock}
&
\sphinxAtStartPar
复位值 |    描 | |
&
\begin{DUlineblock}{0em}
\item[] |
  |
\end{DUlineblock}
\\
\sphinxhline
\sphinxAtStartPar
DATA7
&
\sphinxAtStartPar
0x84
&&
\sphinxAtStartPar
0 000000
&
\sphinxAtStartPar
ADC通道7数据寄存器         |
\\
\sphinxbottomrule
\end{tabular}
\sphinxtableafterendhook\par
\sphinxattableend\end{savenotes}


\begin{savenotes}\sphinxattablestart
\sphinxthistablewithglobalstyle
\centering
\begin{tabular}[t]{\X{20}{100}\X{20}{100}\X{20}{100}\X{20}{100}\X{20}{100}}
\sphinxtoprule
\sphinxtableatstartofbodyhook
\sphinxAtStartPar
寄存器 |
&
\begin{DUlineblock}{0em}
\item[] 偏移 |
\end{DUlineblock}
&
\begin{DUlineblock}{0em}
\item[] 
\item[] {\color{red}\bfseries{}|}
\end{DUlineblock}
&
\sphinxAtStartPar
复位值 |    描 | |
&
\begin{DUlineblock}{0em}
\item[] |
  |
\end{DUlineblock}
\\
\sphinxhline
\sphinxAtStartPar
DATA8
&
\sphinxAtStartPar
0x94
&&
\sphinxAtStartPar
0 000000
&
\sphinxAtStartPar
ADC通道8数据寄存器         |
\\
\sphinxbottomrule
\end{tabular}
\sphinxtableafterendhook\par
\sphinxattableend\end{savenotes}


\begin{savenotes}\sphinxattablestart
\sphinxthistablewithglobalstyle
\centering
\begin{tabular}[t]{\X{20}{100}\X{20}{100}\X{20}{100}\X{20}{100}\X{20}{100}}
\sphinxtoprule
\sphinxtableatstartofbodyhook
\sphinxAtStartPar
寄存器 |
&
\begin{DUlineblock}{0em}
\item[] 偏移 |
\end{DUlineblock}
&
\begin{DUlineblock}{0em}
\item[] 
\item[] {\color{red}\bfseries{}|}
\end{DUlineblock}
&
\sphinxAtStartPar
复位值 |    描 | |
&
\begin{DUlineblock}{0em}
\item[] |
  |
\end{DUlineblock}
\\
\sphinxhline
\sphinxAtStartPar
DATA9
&
\sphinxAtStartPar
0xa4
&&
\sphinxAtStartPar
0 000000
&
\sphinxAtStartPar
ADC通道9数据寄存器         |
\\
\sphinxbottomrule
\end{tabular}
\sphinxtableafterendhook\par
\sphinxattableend\end{savenotes}


\begin{savenotes}\sphinxattablestart
\sphinxthistablewithglobalstyle
\centering
\begin{tabular}[t]{\X{20}{100}\X{20}{100}\X{20}{100}\X{20}{100}\X{20}{100}}
\sphinxtoprule
\sphinxtableatstartofbodyhook
\sphinxAtStartPar
寄存器 |
&
\begin{DUlineblock}{0em}
\item[] 偏移 |
\end{DUlineblock}
&
\begin{DUlineblock}{0em}
\item[] 
\item[] {\color{red}\bfseries{}|}
\end{DUlineblock}
&
\sphinxAtStartPar
复位值 |    描 | |
&
\begin{DUlineblock}{0em}
\item[] |
  |
\end{DUlineblock}
\\
\sphinxhline
\sphinxAtStartPar
DATA10
&
\sphinxAtStartPar
0b4
&&
\sphinxAtStartPar
0 000000
&
\sphinxAtStartPar
ADC通道10数据寄存器        |
\\
\sphinxbottomrule
\end{tabular}
\sphinxtableafterendhook\par
\sphinxattableend\end{savenotes}


\begin{savenotes}\sphinxattablestart
\sphinxthistablewithglobalstyle
\centering
\begin{tabular}[t]{\X{20}{100}\X{20}{100}\X{20}{100}\X{20}{100}\X{20}{100}}
\sphinxtoprule
\sphinxtableatstartofbodyhook
\sphinxAtStartPar
寄存器 |
&
\begin{DUlineblock}{0em}
\item[] 偏移 |
\end{DUlineblock}
&
\begin{DUlineblock}{0em}
\item[] 
\item[] {\color{red}\bfseries{}|}
\end{DUlineblock}
&
\sphinxAtStartPar
复位值 |    描 | |
&
\begin{DUlineblock}{0em}
\item[] |
  |
\end{DUlineblock}
\\
\sphinxhline
\sphinxAtStartPar
DATA11
&
\sphinxAtStartPar
0xc4
&&
\sphinxAtStartPar
0 000000
&
\sphinxAtStartPar
ADC通道11数据寄存器        |
\\
\sphinxbottomrule
\end{tabular}
\sphinxtableafterendhook\par
\sphinxattableend\end{savenotes}


\begin{savenotes}\sphinxattablestart
\sphinxthistablewithglobalstyle
\centering
\begin{tabular}[t]{\X{12}{96}\X{12}{96}\X{12}{96}\X{12}{96}\X{12}{96}\X{12}{96}\X{12}{96}\X{12}{96}}
\sphinxtoprule
\sphinxtableatstartofbodyhook
\sphinxAtStartPar
31
&
\sphinxAtStartPar
30
&
\sphinxAtStartPar
29
&
\sphinxAtStartPar
28
&
\sphinxAtStartPar
27
&
\sphinxAtStartPar
26
&
\sphinxAtStartPar
25
&
\sphinxAtStartPar
24
\\
\sphinxhline\begin{itemize}
\item {} 
\end{itemize}
&&&&&&&\\
\sphinxhline
\sphinxAtStartPar
23
&
\sphinxAtStartPar
22
&
\sphinxAtStartPar
21
&
\sphinxAtStartPar
20
&
\sphinxAtStartPar
19
&
\sphinxAtStartPar
18
&
\sphinxAtStartPar
17
&
\sphinxAtStartPar
16
\\
\sphinxhline\begin{itemize}
\item {} 
\end{itemize}
&&&&&&&\\
\sphinxhline
\sphinxAtStartPar
15
&
\sphinxAtStartPar
14
&
\sphinxAtStartPar
13
&
\sphinxAtStartPar
12
&
\sphinxAtStartPar
11
&
\sphinxAtStartPar
10
&
\sphinxAtStartPar
9
&
\sphinxAtStartPar
8
\\
\sphinxhline
\sphinxAtStartPar
CHNUM
&&&&&&&\\
\sphinxhline
\sphinxAtStartPar
7
&
\sphinxAtStartPar
6
&
\sphinxAtStartPar
5
&
\sphinxAtStartPar
4
&
\sphinxAtStartPar
3
&
\sphinxAtStartPar
2
&
\sphinxAtStartPar
1
&
\sphinxAtStartPar
0
\\
\sphinxhline
\sphinxAtStartPar
VALUE
&&&&&&&\\
\sphinxbottomrule
\end{tabular}
\sphinxtableafterendhook\par
\sphinxattableend\end{savenotes}


\begin{savenotes}\sphinxattablestart
\sphinxthistablewithglobalstyle
\centering
\begin{tabular}[t]{\X{33}{99}\X{33}{99}\X{33}{99}}
\sphinxtoprule
\sphinxtableatstartofbodyhook
\sphinxAtStartPar
位域 |
&
\sphinxAtStartPar
名称     | |
&
\sphinxAtStartPar
描述                                        | |
\\
\sphinxhline
\sphinxAtStartPar
31:16
&\begin{itemize}
\item {} 
\end{itemize}
&\begin{itemize}
\item {} 
\end{itemize}
\\
\sphinxhline
\sphinxAtStartPar
15:12
&
\sphinxAtStartPar
CHNUM
&
\sphinxAtStartPar
ADC数据对应的通道编号                       |

\sphinxAtStartPar
0000:通道0                                 |

\sphinxAtStartPar
0001:通道1                                 |

\sphinxAtStartPar
0010:通道2                                 |

\sphinxAtStartPar
0011:通道3                                 |

\sphinxAtStartPar
0100:通道4                                 |

\sphinxAtStartPar
0101:通道5                                 |

\sphinxAtStartPar
0110:通道6                                 |

\sphinxAtStartPar
0111:通道7                                 |

\sphinxAtStartPar
1000:通道8                                 |

\sphinxAtStartPar
1001:通道9                                 |

\sphinxAtStartPar
1010:通道10                                |

\sphinxAtStartPar
1011:通道11                                |
\\
\sphinxhline
\sphinxAtStartPar
11:0
&
\sphinxAtStartPar
VALUE
&
\sphinxAtStartPar
ADC通道x数据寄存器                          |

\sphinxAtStartPar
注:溢出后,再次转换的数据会覆盖旧数据      |
\\
\sphinxbottomrule
\end{tabular}
\sphinxtableafterendhook\par
\sphinxattableend\end{savenotes}


\subsubsection{ADC通道配置寄存器器CHSEL}
\label{\detokenize{SWM241/_u529f_u80fd_u63cf_u8ff0/_u6a21_u62df_u6570_u5b57_u8f6c_u6362_u5668:adcchsel}}

\begin{savenotes}\sphinxattablestart
\sphinxthistablewithglobalstyle
\centering
\begin{tabular}[t]{\X{20}{100}\X{20}{100}\X{20}{100}\X{20}{100}\X{20}{100}}
\sphinxtoprule
\sphinxtableatstartofbodyhook
\sphinxAtStartPar
寄存器 |
&
\begin{DUlineblock}{0em}
\item[] 偏移 |
\end{DUlineblock}
&
\begin{DUlineblock}{0em}
\item[] 
\item[] {\color{red}\bfseries{}|}
\end{DUlineblock}
&
\sphinxAtStartPar
复位值 |    描 | |
&
\begin{DUlineblock}{0em}
\item[] |
  |
\end{DUlineblock}
\\
\sphinxhline
\sphinxAtStartPar
CHSEL
&
\sphinxAtStartPar
0xd0
&&
\sphinxAtStartPar
0 000000
&
\sphinxAtStartPar
ADC通道配置寄存器          |
\\
\sphinxbottomrule
\end{tabular}
\sphinxtableafterendhook\par
\sphinxattableend\end{savenotes}


\begin{savenotes}\sphinxattablestart
\sphinxthistablewithglobalstyle
\centering
\begin{tabular}[t]{\X{12}{96}\X{12}{96}\X{12}{96}\X{12}{96}\X{12}{96}\X{12}{96}\X{12}{96}\X{12}{96}}
\sphinxtoprule
\sphinxtableatstartofbodyhook
\sphinxAtStartPar
31
&
\sphinxAtStartPar
30
&
\sphinxAtStartPar
29
&
\sphinxAtStartPar
28
&
\sphinxAtStartPar
27
&
\sphinxAtStartPar
26
&
\sphinxAtStartPar
25
&
\sphinxAtStartPar
24
\\
\sphinxhline\begin{itemize}
\item {} 
\end{itemize}
&&&&&&&\\
\sphinxhline
\sphinxAtStartPar
23
&
\sphinxAtStartPar
22
&
\sphinxAtStartPar
21
&
\sphinxAtStartPar
20
&
\sphinxAtStartPar
19
&
\sphinxAtStartPar
18
&
\sphinxAtStartPar
17
&
\sphinxAtStartPar
16
\\
\sphinxhline
\sphinxAtStartPar
PWM
&&&&&&&\\
\sphinxhline
\sphinxAtStartPar
15
&
\sphinxAtStartPar
14
&
\sphinxAtStartPar
13
&
\sphinxAtStartPar
12
&
\sphinxAtStartPar
11
&
\sphinxAtStartPar
10
&
\sphinxAtStartPar
9
&
\sphinxAtStartPar
8
\\
\sphinxhline\begin{itemize}
\item {} 
\end{itemize}
&&&&
\sphinxAtStartPar
SW
&&&\\
\sphinxhline
\sphinxAtStartPar
7
&
\sphinxAtStartPar
6
&
\sphinxAtStartPar
5
&
\sphinxAtStartPar
4
&
\sphinxAtStartPar
3
&
\sphinxAtStartPar
2
&
\sphinxAtStartPar
1
&
\sphinxAtStartPar
0
\\
\sphinxhline
\sphinxAtStartPar
SW
&&&&&&&\\
\sphinxbottomrule
\end{tabular}
\sphinxtableafterendhook\par
\sphinxattableend\end{savenotes}


\begin{savenotes}\sphinxattablestart
\sphinxthistablewithglobalstyle
\centering
\begin{tabular}[t]{\X{33}{99}\X{33}{99}\X{33}{99}}
\sphinxtoprule
\sphinxtableatstartofbodyhook
\sphinxAtStartPar
位域 |
&
\sphinxAtStartPar
名称     | |
&
\sphinxAtStartPar
描述                                        | |
\\
\sphinxhline
\sphinxAtStartPar
31:28
&\begin{itemize}
\item {} 
\end{itemize}
&\begin{itemize}
\item {} 
\end{itemize}
\\
\sphinxhline
\sphinxAtStartPar
27:16
&
\sphinxAtStartPar
PWM
&
\sphinxAtStartPar
PWM启动ADC采样时的通道号                    |

\sphinxAtStartPar
Bit16=1:CH0启动                            |

\sphinxAtStartPar
Bit17=1:CH1启动                            |

\sphinxAtStartPar
Bit18=1:CH2启动                            |

\sphinxAtStartPar
Bit19=1:CH3启动                            |

\sphinxAtStartPar
Bit20=1:CH4启动                            |

\sphinxAtStartPar
Bit21=1:CH5启动                            |

\sphinxAtStartPar
Bit22=1:CH6启动                            |

\sphinxAtStartPar
Bit23=1:CH7启动                            |

\sphinxAtStartPar
Bit24=1:CH8启动                            |

\sphinxAtStartPar
Bit25=1:CH9启动                            |

\sphinxAtStartPar
Bit26=1:CH10启动                           |

\sphinxAtStartPar
Bit27=1:CH11启动                           |

\sphinxAtStartPar
注1:当配                                   | m触发ADC采样使能且PWM触发信号有效时,实际  | ADC采样通道会自动切换为PWM2ADC\_CH\_SEL值  |
\\
\sphinxhline
\sphinxAtStartPar
15:12
&\begin{itemize}
\item {} 
\end{itemize}
&\begin{itemize}
\item {} 
\end{itemize}
\\
\sphinxhline
\sphinxAtStartPar
11:0
&
\sphinxAtStartPar
SW
&
\sphinxAtStartPar
CPU启动ADC采样的通道号                      |

\sphinxAtStartPar
Bit0=1:CH0启动                             |

\sphinxAtStartPar
Bit1=1:CH1启动                             |

\sphinxAtStartPar
Bit2=1:CH2启动                             |

\sphinxAtStartPar
Bit3=1:CH3启动                             |

\sphinxAtStartPar
Bit4=1:CH4启动                             |

\sphinxAtStartPar
Bit5=1:CH5启动                             |

\sphinxAtStartPar
Bit6=1:CH6启动                             |

\sphinxAtStartPar
Bit7=1:CH7启动                             |

\sphinxAtStartPar
Bit8=1:CH8启动                             |

\sphinxAtStartPar
Bit9=1:CH9启动                             |

\sphinxAtStartPar
Bit10=1:CH10启动                           |

\sphinxAtStartPar
Bit11=1:CH11启动                           |

\sphinxAtStartPar
注1:当CPU启动了ADC采样过程                 | 遇到PWM触发ADC的启动信号,PWM信号被忽略  |
\\
\sphinxbottomrule
\end{tabular}
\sphinxtableafterendhook\par
\sphinxattableend\end{savenotes}


\subsubsection{FIFO状态寄存器FFSTAT}
\label{\detokenize{SWM241/_u529f_u80fd_u63cf_u8ff0/_u6a21_u62df_u6570_u5b57_u8f6c_u6362_u5668:fifoffstat}}

\begin{savenotes}\sphinxattablestart
\sphinxthistablewithglobalstyle
\centering
\begin{tabular}[t]{\X{20}{100}\X{20}{100}\X{20}{100}\X{20}{100}\X{20}{100}}
\sphinxtoprule
\sphinxtableatstartofbodyhook
\sphinxAtStartPar
寄存器 |
&
\begin{DUlineblock}{0em}
\item[] 偏移 |
\end{DUlineblock}
&
\begin{DUlineblock}{0em}
\item[] 
\item[] {\color{red}\bfseries{}|}
\end{DUlineblock}
&
\sphinxAtStartPar
复位值 |    描 | |
&
\begin{DUlineblock}{0em}
\item[] |
  |
\end{DUlineblock}
\\
\sphinxhline
\sphinxAtStartPar
FFSTAT
&
\sphinxAtStartPar
0x190
&&
\sphinxAtStartPar
0 000008
&
\sphinxAtStartPar
ADC FIFO状态寄存器         |
\\
\sphinxbottomrule
\end{tabular}
\sphinxtableafterendhook\par
\sphinxattableend\end{savenotes}


\begin{savenotes}\sphinxattablestart
\sphinxthistablewithglobalstyle
\centering
\begin{tabular}[t]{\X{12}{96}\X{12}{96}\X{12}{96}\X{12}{96}\X{12}{96}\X{12}{96}\X{12}{96}\X{12}{96}}
\sphinxtoprule
\sphinxtableatstartofbodyhook
\sphinxAtStartPar
31
&
\sphinxAtStartPar
30
&
\sphinxAtStartPar
29
&
\sphinxAtStartPar
28
&
\sphinxAtStartPar
27
&
\sphinxAtStartPar
26
&
\sphinxAtStartPar
25
&
\sphinxAtStartPar
24
\\
\sphinxhline\begin{itemize}
\item {} 
\end{itemize}
&&&&&&&\\
\sphinxhline
\sphinxAtStartPar
23
&
\sphinxAtStartPar
22
&
\sphinxAtStartPar
21
&
\sphinxAtStartPar
20
&
\sphinxAtStartPar
19
&
\sphinxAtStartPar
18
&
\sphinxAtStartPar
17
&
\sphinxAtStartPar
16
\\
\sphinxhline\begin{itemize}
\item {} 
\end{itemize}
&&&&&&&\\
\sphinxhline
\sphinxAtStartPar
15
&
\sphinxAtStartPar
14
&
\sphinxAtStartPar
13
&
\sphinxAtStartPar
12
&
\sphinxAtStartPar
11
&
\sphinxAtStartPar
10
&
\sphinxAtStartPar
9
&
\sphinxAtStartPar
8
\\
\sphinxhline\begin{itemize}
\item {} 
\end{itemize}
&&&&&&&\\
\sphinxhline
\sphinxAtStartPar
7
&
\sphinxAtStartPar
6
&
\sphinxAtStartPar
5
&
\sphinxAtStartPar
4
&
\sphinxAtStartPar
3
&
\sphinxAtStartPar
2
&
\sphinxAtStartPar
1
&
\sphinxAtStartPar
0
\\
\sphinxhline\begin{itemize}
\item {} 
\end{itemize}
&
\sphinxAtStartPar
LEVEL
&&&&&&
\sphinxAtStartPar
OVF
\\
\sphinxbottomrule
\end{tabular}
\sphinxtableafterendhook\par
\sphinxattableend\end{savenotes}


\begin{savenotes}\sphinxattablestart
\sphinxthistablewithglobalstyle
\centering
\begin{tabular}[t]{\X{33}{99}\X{33}{99}\X{33}{99}}
\sphinxtoprule
\sphinxtableatstartofbodyhook
\sphinxAtStartPar
位域 |
&
\sphinxAtStartPar
名称     | |
&
\sphinxAtStartPar
描述                                        | |
\\
\sphinxhline
\sphinxAtStartPar
31:7
&\begin{itemize}
\item {} 
\end{itemize}
&\begin{itemize}
\item {} 
\end{itemize}
\\
\sphinxhline
\sphinxAtStartPar
6:4
&
\sphinxAtStartPar
LEVEL
&
\sphinxAtStartPar
ADC数据FIFO LEVEL标志                       |

\sphinxAtStartPar
000:FIFO有0个数据                          |

\sphinxAtStartPar
001:FIFO有1个数据                          |

\sphinxAtStartPar
010:FIFO有2个数据                          |

\sphinxAtStartPar
011:FIFO有3个数据                          |

\sphinxAtStartPar
100:FIFO有4个数据                          |

\sphinxAtStartPar
101:FIFO有5个数据                          |

\sphinxAtStartPar
110:FIFO有6个数据                          |

\sphinxAtStartPar
111:FIFO有7个数据                          |
\\
\sphinxhline
\sphinxAtStartPar
3
&
\sphinxAtStartPar
EMPTY
&
\sphinxAtStartPar
ADC数据FIFO空标志                           |

\sphinxAtStartPar
1:FIFO空                                   |

\sphinxAtStartPar
0:FIFO非空                                 |
\\
\sphinxhline
\sphinxAtStartPar
2
&
\sphinxAtStartPar
FULL
&
\sphinxAtStartPar
ADC数据FIFO满标志                           |

\sphinxAtStartPar
1:FIFO满                                   |

\sphinxAtStartPar
0:FIFO非满                                 |
\\
\sphinxhline
\sphinxAtStartPar
1
&
\sphinxAtStartPar
HFULL
&
\sphinxAtStartPar
ADC数据FIFO半满标志                         |

\sphinxAtStartPar
1:FIFO半满                                 |

\sphinxAtStartPar
0:FIFO满或未达到半满                       |
\\
\sphinxhline
\sphinxAtStartPar
0
&
\sphinxAtStartPar
OVF
&
\sphinxAtStartPar
ADC数据FIFO溢出标志                         |

\sphinxAtStartPar
1:FIFO出现溢出                             |

\sphinxAtStartPar
0:FIFO未溢出                               |
\\
\sphinxbottomrule
\end{tabular}
\sphinxtableafterendhook\par
\sphinxattableend\end{savenotes}


\subsubsection{所有通道FIFO数据寄存器FFDATA}
\label{\detokenize{SWM241/_u529f_u80fd_u63cf_u8ff0/_u6a21_u62df_u6570_u5b57_u8f6c_u6362_u5668:fifoffdata}}

\begin{savenotes}\sphinxattablestart
\sphinxthistablewithglobalstyle
\centering
\begin{tabular}[t]{\X{20}{100}\X{20}{100}\X{20}{100}\X{20}{100}\X{20}{100}}
\sphinxtoprule
\sphinxtableatstartofbodyhook
\sphinxAtStartPar
寄存器 |
&
\begin{DUlineblock}{0em}
\item[] 偏移 |
\end{DUlineblock}
&
\begin{DUlineblock}{0em}
\item[] 
\item[] {\color{red}\bfseries{}|}
\end{DUlineblock}
&
\sphinxAtStartPar
复位值 |    描 | |
&
\begin{DUlineblock}{0em}
\item[] |
  |
\end{DUlineblock}
\\
\sphinxhline
\sphinxAtStartPar
FFDATA
&
\sphinxAtStartPar
0x194
&&
\sphinxAtStartPar
0 000000
&
\sphinxAtStartPar
ADC所有通道数据寄存器      |
\\
\sphinxbottomrule
\end{tabular}
\sphinxtableafterendhook\par
\sphinxattableend\end{savenotes}


\begin{savenotes}\sphinxattablestart
\sphinxthistablewithglobalstyle
\centering
\begin{tabular}[t]{\X{12}{96}\X{12}{96}\X{12}{96}\X{12}{96}\X{12}{96}\X{12}{96}\X{12}{96}\X{12}{96}}
\sphinxtoprule
\sphinxtableatstartofbodyhook
\sphinxAtStartPar
31
&
\sphinxAtStartPar
30
&
\sphinxAtStartPar
29
&
\sphinxAtStartPar
28
&
\sphinxAtStartPar
27
&
\sphinxAtStartPar
26
&
\sphinxAtStartPar
25
&
\sphinxAtStartPar
24
\\
\sphinxhline\begin{itemize}
\item {} 
\end{itemize}
&&&&&&&\\
\sphinxhline
\sphinxAtStartPar
23
&
\sphinxAtStartPar
22
&
\sphinxAtStartPar
21
&
\sphinxAtStartPar
20
&
\sphinxAtStartPar
19
&
\sphinxAtStartPar
18
&
\sphinxAtStartPar
17
&
\sphinxAtStartPar
16
\\
\sphinxhline\begin{itemize}
\item {} 
\end{itemize}
&&&&&&&\\
\sphinxhline
\sphinxAtStartPar
15
&
\sphinxAtStartPar
14
&
\sphinxAtStartPar
13
&
\sphinxAtStartPar
12
&
\sphinxAtStartPar
11
&
\sphinxAtStartPar
10
&
\sphinxAtStartPar
9
&
\sphinxAtStartPar
8
\\
\sphinxhline
\sphinxAtStartPar
CHNUM
&&&&&&&\\
\sphinxhline
\sphinxAtStartPar
7
&
\sphinxAtStartPar
6
&
\sphinxAtStartPar
5
&
\sphinxAtStartPar
4
&
\sphinxAtStartPar
3
&
\sphinxAtStartPar
2
&
\sphinxAtStartPar
1
&
\sphinxAtStartPar
0
\\
\sphinxhline
\sphinxAtStartPar
VALUE
&&&&&&&\\
\sphinxbottomrule
\end{tabular}
\sphinxtableafterendhook\par
\sphinxattableend\end{savenotes}


\begin{savenotes}\sphinxattablestart
\sphinxthistablewithglobalstyle
\centering
\begin{tabular}[t]{\X{33}{99}\X{33}{99}\X{33}{99}}
\sphinxtoprule
\sphinxtableatstartofbodyhook
\sphinxAtStartPar
位域 |
&
\sphinxAtStartPar
名称     | |
&
\sphinxAtStartPar
描述                                        | |
\\
\sphinxhline
\sphinxAtStartPar
31:16
&\begin{itemize}
\item {} 
\end{itemize}
&\begin{itemize}
\item {} 
\end{itemize}
\\
\sphinxhline
\sphinxAtStartPar
15:12
&
\sphinxAtStartPar
CHNUM
&
\sphinxAtStartPar
ADC数据对应的通道编号                       |

\sphinxAtStartPar
0000:通道0                                 |

\sphinxAtStartPar
0001:通道1                                 |

\sphinxAtStartPar
0010:通道2                                 |

\sphinxAtStartPar
0011:通道3                                 |

\sphinxAtStartPar
0100:通道4                                 |

\sphinxAtStartPar
0101:通道5                                 |

\sphinxAtStartPar
0110:通道6                                 |

\sphinxAtStartPar
0111:通道7                                 |

\sphinxAtStartPar
1000:通道8                                 |

\sphinxAtStartPar
1001:通道9                                 |

\sphinxAtStartPar
1010:通道10                                |

\sphinxAtStartPar
1011:通道11                                |
\\
\sphinxhline
\sphinxAtStartPar
11:0
&
\sphinxAtStartPar
VALUE
&
\sphinxAtStartPar
ADC通道x数据FIFO寄存器                      |

\sphinxAtStartPar
注:溢出后,再次转换的数据会被丢掉          |
\\
\sphinxbottomrule
\end{tabular}
\sphinxtableafterendhook\par
\sphinxattableend\end{savenotes}


\subsubsection{配置寄存器CTRL1}
\label{\detokenize{SWM241/_u529f_u80fd_u63cf_u8ff0/_u6a21_u62df_u6570_u5b57_u8f6c_u6362_u5668:ctrl1}}

\begin{savenotes}\sphinxattablestart
\sphinxthistablewithglobalstyle
\centering
\begin{tabular}[t]{\X{20}{100}\X{20}{100}\X{20}{100}\X{20}{100}\X{20}{100}}
\sphinxtoprule
\sphinxtableatstartofbodyhook
\sphinxAtStartPar
寄存器 |
&
\begin{DUlineblock}{0em}
\item[] 偏移 |
\end{DUlineblock}
&
\begin{DUlineblock}{0em}
\item[] 
\item[] {\color{red}\bfseries{}|}
\end{DUlineblock}
&
\sphinxAtStartPar
复位值 |    描 | |
&
\begin{DUlineblock}{0em}
\item[] |
  |
\end{DUlineblock}
\\
\sphinxhline
\sphinxAtStartPar
CTRL1
&
\sphinxAtStartPar
0x1a0
&&
\sphinxAtStartPar
0 000000
&
\sphinxAtStartPar
ADC配置寄存器1             |
\\
\sphinxbottomrule
\end{tabular}
\sphinxtableafterendhook\par
\sphinxattableend\end{savenotes}


\begin{savenotes}\sphinxattablestart
\sphinxthistablewithglobalstyle
\centering
\begin{tabular}[t]{\X{12}{96}\X{12}{96}\X{12}{96}\X{12}{96}\X{12}{96}\X{12}{96}\X{12}{96}\X{12}{96}}
\sphinxtoprule
\sphinxtableatstartofbodyhook
\sphinxAtStartPar
31
&
\sphinxAtStartPar
30
&
\sphinxAtStartPar
29
&
\sphinxAtStartPar
28
&
\sphinxAtStartPar
27
&
\sphinxAtStartPar
26
&
\sphinxAtStartPar
25
&
\sphinxAtStartPar
24
\\
\sphinxhline\begin{itemize}
\item {} 
\end{itemize}
&&&&&&&\\
\sphinxhline
\sphinxAtStartPar
23
&
\sphinxAtStartPar
22
&
\sphinxAtStartPar
21
&
\sphinxAtStartPar
20
&
\sphinxAtStartPar
19
&
\sphinxAtStartPar
18
&
\sphinxAtStartPar
17
&
\sphinxAtStartPar
16
\\
\sphinxhline\begin{itemize}
\item {} 
\end{itemize}
&&&&&&&\\
\sphinxhline
\sphinxAtStartPar
15
&
\sphinxAtStartPar
14
&
\sphinxAtStartPar
13
&
\sphinxAtStartPar
12
&
\sphinxAtStartPar
11
&
\sphinxAtStartPar
10
&
\sphinxAtStartPar
9
&
\sphinxAtStartPar
8
\\
\sphinxhline\begin{itemize}
\item {} 
\end{itemize}
&&&&&&&\\
\sphinxhline
\sphinxAtStartPar
7
&
\sphinxAtStartPar
6
&
\sphinxAtStartPar
5
&
\sphinxAtStartPar
4
&
\sphinxAtStartPar
3
&
\sphinxAtStartPar
2
&
\sphinxAtStartPar
1
&
\sphinxAtStartPar
0
\\
\sphinxhline\begin{itemize}
\item {} 
\end{itemize}
&&&&&&&\\
\sphinxbottomrule
\end{tabular}
\sphinxtableafterendhook\par
\sphinxattableend\end{savenotes}


\begin{savenotes}\sphinxattablestart
\sphinxthistablewithglobalstyle
\centering
\begin{tabular}[t]{\X{33}{99}\X{33}{99}\X{33}{99}}
\sphinxtoprule
\sphinxtableatstartofbodyhook
\sphinxAtStartPar
位域 |
&
\sphinxAtStartPar
名称     | |
&
\sphinxAtStartPar
描述                                        | |
\\
\sphinxhline
\sphinxAtStartPar
31:1
&\begin{itemize}
\item {} 
\end{itemize}
&\begin{itemize}
\item {} 
\end{itemize}
\\
\sphinxhline
\sphinxAtStartPar
0
&
\sphinxAtStartPar
CLKSRC
&
\sphinxAtStartPar
ADC采样时钟选择                             |

\sphinxAtStartPar
0:系统时钟                                 |

\sphinxAtStartPar
1:外置晶振                                 |
\\
\sphinxbottomrule
\end{tabular}
\sphinxtableafterendhook\par
\sphinxattableend\end{savenotes}


\subsubsection{配置寄存器CTRL2}
\label{\detokenize{SWM241/_u529f_u80fd_u63cf_u8ff0/_u6a21_u62df_u6570_u5b57_u8f6c_u6362_u5668:ctrl2}}

\begin{savenotes}\sphinxattablestart
\sphinxthistablewithglobalstyle
\centering
\begin{tabular}[t]{\X{20}{100}\X{20}{100}\X{20}{100}\X{20}{100}\X{20}{100}}
\sphinxtoprule
\sphinxtableatstartofbodyhook
\sphinxAtStartPar
寄存器 |
&
\begin{DUlineblock}{0em}
\item[] 偏移 |
\end{DUlineblock}
&
\begin{DUlineblock}{0em}
\item[] 
\item[] {\color{red}\bfseries{}|}
\end{DUlineblock}
&
\sphinxAtStartPar
复位值 |    描 | |
&
\begin{DUlineblock}{0em}
\item[] |
  |
\end{DUlineblock}
\\
\sphinxhline
\sphinxAtStartPar
CTRL2
&
\sphinxAtStartPar
0x1a4
&&
\sphinxAtStartPar
0 000F10
&
\sphinxAtStartPar
ADC配置寄存器2             |
\\
\sphinxbottomrule
\end{tabular}
\sphinxtableafterendhook\par
\sphinxattableend\end{savenotes}


\begin{savenotes}\sphinxattablestart
\sphinxthistablewithglobalstyle
\centering
\begin{tabular}[t]{\X{12}{96}\X{12}{96}\X{12}{96}\X{12}{96}\X{12}{96}\X{12}{96}\X{12}{96}\X{12}{96}}
\sphinxtoprule
\sphinxtableatstartofbodyhook
\sphinxAtStartPar
31
&
\sphinxAtStartPar
30
&
\sphinxAtStartPar
29
&
\sphinxAtStartPar
28
&
\sphinxAtStartPar
27
&
\sphinxAtStartPar
26
&
\sphinxAtStartPar
25
&
\sphinxAtStartPar
24
\\
\sphinxhline\begin{itemize}
\item {} 
\end{itemize}
&
\sphinxAtStartPar
CL KDIV1
&&
\sphinxAtStartPar
C IV2
&&&&\\
\sphinxhline
\sphinxAtStartPar
23
&
\sphinxAtStartPar
22
&
\sphinxAtStartPar
21
&
\sphinxAtStartPar
20
&
\sphinxAtStartPar
19
&
\sphinxAtStartPar
18
&
\sphinxAtStartPar
17
&
\sphinxAtStartPar
16
\\
\sphinxhline\begin{itemize}
\item {} 
\end{itemize}
&&&&&&&\\
\sphinxhline
\sphinxAtStartPar
15
&
\sphinxAtStartPar
14
&
\sphinxAtStartPar
13
&
\sphinxAtStartPar
12
&
\sphinxAtStartPar
11
&
\sphinxAtStartPar
10
&
\sphinxAtStartPar
9
&
\sphinxAtStartPar
8
\\
\sphinxhline\begin{itemize}
\item {} 
\end{itemize}
&&&&&&&\\
\sphinxhline
\sphinxAtStartPar
7
&
\sphinxAtStartPar
6
&
\sphinxAtStartPar
5
&
\sphinxAtStartPar
4
&
\sphinxAtStartPar
3
&
\sphinxAtStartPar
2
&
\sphinxAtStartPar
1
&
\sphinxAtStartPar
0
\\
\sphinxhline
\sphinxAtStartPar
PGAGAIN
&&&&&
\sphinxAtStartPar
P VCM
&
\sphinxAtStartPar
A VCM
&\begin{itemize}
\item {} 
\end{itemize}
\\
\sphinxbottomrule
\end{tabular}
\sphinxtableafterendhook\par
\sphinxattableend\end{savenotes}


\begin{savenotes}\sphinxattablestart
\sphinxthistablewithglobalstyle
\centering
\begin{tabular}[t]{\X{33}{99}\X{33}{99}\X{33}{99}}
\sphinxtoprule
\sphinxtableatstartofbodyhook
\sphinxAtStartPar
位域 |
&
\sphinxAtStartPar
名称     | |
&
\sphinxAtStartPar
描述                                        | |
\\
\sphinxhline
\sphinxAtStartPar
31
&\begin{itemize}
\item {} 
\end{itemize}
&\begin{itemize}
\item {} 
\end{itemize}
\\
\sphinxhline
\sphinxAtStartPar
30:29
&
\sphinxAtStartPar
CLKDIV1
&
\sphinxAtStartPar
RC Clock Post Divi EXT\_REF\_CLKIN或RC48MHZ\_CLKIN时钟的预分频) |

\sphinxAtStartPar
00:4分频                                   |

\sphinxAtStartPar
01:2分频                                   |

\sphinxAtStartPar
10:1分频                                   |

\sphinxAtStartPar
11:不可配                                  |
\\
\sphinxhline
\sphinxAtStartPar
28:24
&
\sphinxAtStartPar
CLKDIV2
&
\sphinxAtStartPar
RC clock input divider ratio(对分频时钟的再分频)                 |

\sphinxAtStartPar
1对应1分频,以此类推                        |

\sphinxAtStartPar
00001:1分频                                |

\sphinxAtStartPar
00010:2分频                                |

\sphinxAtStartPar
00011:3分频                                |

\sphinxAtStartPar
00100:4分频                                |

\sphinxAtStartPar
00101:5分频                                |

\sphinxAtStartPar
00110:6分频                                |

\sphinxAtStartPar
00111:7分频                                |

\sphinxAtStartPar
01000:8分频                                |

\sphinxAtStartPar
01001:9分频                                |

\sphinxAtStartPar
01010:10分频                               |

\sphinxAtStartPar
01011:11分频                               |

\sphinxAtStartPar
01100:12分频                               |

\sphinxAtStartPar
01101:13分频                               |
\\
\sphinxhline
\sphinxAtStartPar
23:12
&\begin{itemize}
\item {} 
\end{itemize}
&\begin{itemize}
\item {} 
\end{itemize}
\\
\sphinxhline
\sphinxAtStartPar
11:8
&
\sphinxAtStartPar
VCMSEL
&
\sphinxAtStartPar
PGA Common Mode Voltage Select
\\
\sphinxhline
\sphinxAtStartPar
7:3
&
\sphinxAtStartPar
PGAGAIN
&
\sphinxAtStartPar
PGA GAIN program
\\
\sphinxhline
\sphinxAtStartPar
2
&
\sphinxAtStartPar
PGAEVCM
&
\sphinxAtStartPar
使能外部reference, PGA输入共模电平选择      |

\sphinxAtStartPar
切换内部/外部基准                           |

\sphinxAtStartPar
0:内部基准(5V/3.6V)                      |

\sphinxAtStartPar
1:外部基准(AVDD5/REFP)                   |
\\
\sphinxhline
\sphinxAtStartPar
1
&
\sphinxAtStartPar
ADCEVCM
&
\sphinxAtStartPar
使能ADC外部reference ,ADC External          | VCM,ADC与PGA输出共模电平选择               |
\\
\sphinxhline
\sphinxAtStartPar
0
&\begin{itemize}
\item {} 
\end{itemize}
&\begin{itemize}
\item {} 
\end{itemize}
\\
\sphinxbottomrule
\end{tabular}
\sphinxtableafterendhook\par
\sphinxattableend\end{savenotes}


\subsubsection{配置寄存器CTRL3}
\label{\detokenize{SWM241/_u529f_u80fd_u63cf_u8ff0/_u6a21_u62df_u6570_u5b57_u8f6c_u6362_u5668:ctrl3}}

\begin{savenotes}\sphinxattablestart
\sphinxthistablewithglobalstyle
\centering
\begin{tabular}[t]{\X{20}{100}\X{20}{100}\X{20}{100}\X{20}{100}\X{20}{100}}
\sphinxtoprule
\sphinxtableatstartofbodyhook
\sphinxAtStartPar
寄存器 |
&
\begin{DUlineblock}{0em}
\item[] 偏移 |
\end{DUlineblock}
&
\begin{DUlineblock}{0em}
\item[] 
\item[] {\color{red}\bfseries{}|}
\end{DUlineblock}
&
\sphinxAtStartPar
复位值 |    描 | |
&
\begin{DUlineblock}{0em}
\item[] |
  |
\end{DUlineblock}
\\
\sphinxhline
\sphinxAtStartPar
CTRL3
&
\sphinxAtStartPar
0x1a8
&&
\sphinxAtStartPar
0 000000
&
\sphinxAtStartPar
ADC配置寄存器8             |
\\
\sphinxbottomrule
\end{tabular}
\sphinxtableafterendhook\par
\sphinxattableend\end{savenotes}


\begin{savenotes}\sphinxattablestart
\sphinxthistablewithglobalstyle
\centering
\begin{tabular}[t]{\X{12}{96}\X{12}{96}\X{12}{96}\X{12}{96}\X{12}{96}\X{12}{96}\X{12}{96}\X{12}{96}}
\sphinxtoprule
\sphinxtableatstartofbodyhook
\sphinxAtStartPar
31
&
\sphinxAtStartPar
30
&
\sphinxAtStartPar
29
&
\sphinxAtStartPar
28
&
\sphinxAtStartPar
27
&
\sphinxAtStartPar
26
&
\sphinxAtStartPar
25
&
\sphinxAtStartPar
24
\\
\sphinxhline\begin{itemize}
\item {} 
\end{itemize}
&
\sphinxAtStartPar
RCDIV
&&&&&&\\
\sphinxhline
\sphinxAtStartPar
23
&
\sphinxAtStartPar
22
&
\sphinxAtStartPar
21
&
\sphinxAtStartPar
20
&
\sphinxAtStartPar
19
&
\sphinxAtStartPar
18
&
\sphinxAtStartPar
17
&
\sphinxAtStartPar
16
\\
\sphinxhline\begin{itemize}
\item {} 
\end{itemize}
&&&&&&&\\
\sphinxhline
\sphinxAtStartPar
15
&
\sphinxAtStartPar
14
&
\sphinxAtStartPar
13
&
\sphinxAtStartPar
12
&
\sphinxAtStartPar
11
&
\sphinxAtStartPar
10
&
\sphinxAtStartPar
9
&
\sphinxAtStartPar
8
\\
\sphinxhline\begin{itemize}
\item {} 
\end{itemize}
&&&&&&&\\
\sphinxhline
\sphinxAtStartPar
7
&
\sphinxAtStartPar
6
&
\sphinxAtStartPar
5
&
\sphinxAtStartPar
4
&
\sphinxAtStartPar
3
&
\sphinxAtStartPar
2
&
\sphinxAtStartPar
1
&
\sphinxAtStartPar
0
\\
\sphinxhline\begin{itemize}
\item {} 
\end{itemize}
&&&
\sphinxAtStartPar
C IV0
&&
\sphinxAtStartPar
R SEL
&\begin{itemize}
\item {} 
\end{itemize}
&\\
\sphinxbottomrule
\end{tabular}
\sphinxtableafterendhook\par
\sphinxattableend\end{savenotes}


\begin{savenotes}\sphinxattablestart
\sphinxthistablewithglobalstyle
\centering
\begin{tabular}[t]{\X{33}{99}\X{33}{99}\X{33}{99}}
\sphinxtoprule
\sphinxtableatstartofbodyhook
\sphinxAtStartPar
位域 |
&
\sphinxAtStartPar
名称     | |
&
\sphinxAtStartPar
描述                                        | |
\\
\sphinxhline
\sphinxAtStartPar
31:3
&\begin{itemize}
\item {} 
\end{itemize}
&\begin{itemize}
\item {} 
\end{itemize}
\\
\sphinxhline
\sphinxAtStartPar
4:3
&
\sphinxAtStartPar
CLKDIV0
&
\sphinxAtStartPar
CLK分频                                     |

\sphinxAtStartPar
00:4分频                                   |

\sphinxAtStartPar
01:2分频                                   |

\sphinxAtStartPar
10:1分频                                   |

\sphinxAtStartPar
11:保留                                    |
\\
\sphinxhline
\sphinxAtStartPar
2
&
\sphinxAtStartPar
REFPSEL
&
\sphinxAtStartPar
直接连接至ADC模块的ADC\_REF\_SEL\_VDD5端口     |

\sphinxAtStartPar
基准切换AVDD5/外部refp                      |

\sphinxAtStartPar
0:外部refp                                 |

\sphinxAtStartPar
1:AVDD5                                    |

\sphinxAtStartPar
直接连接至ADC模块的ADC\_REF\_TEST端口         |
\\
\sphinxhline
\sphinxAtStartPar
1:0
&\begin{itemize}
\item {} 
\end{itemize}
&\begin{itemize}
\item {} 
\end{itemize}
\\
\sphinxbottomrule
\end{tabular}
\sphinxtableafterendhook\par
\sphinxattableend\end{savenotes}


\subsubsection{PWM通道触发ADC屏蔽寄存器TRGMSK}
\label{\detokenize{SWM241/_u529f_u80fd_u63cf_u8ff0/_u6a21_u62df_u6570_u5b57_u8f6c_u6362_u5668:pwmadctrgmsk}}

\begin{savenotes}\sphinxattablestart
\sphinxthistablewithglobalstyle
\centering
\begin{tabular}[t]{\X{20}{100}\X{20}{100}\X{20}{100}\X{20}{100}\X{20}{100}}
\sphinxtoprule
\sphinxtableatstartofbodyhook
\sphinxAtStartPar
寄存器 |
&
\begin{DUlineblock}{0em}
\item[] 偏移 |
\end{DUlineblock}
&
\begin{DUlineblock}{0em}
\item[] 
\item[] {\color{red}\bfseries{}|}
\end{DUlineblock}
&
\sphinxAtStartPar
复位值 |    描 | |
&
\begin{DUlineblock}{0em}
\item[] |
  |
\end{DUlineblock}
\\
\sphinxhline
\sphinxAtStartPar
TRGMSK
&
\sphinxAtStartPar
0x1b0
&&
\sphinxAtStartPar
0 000000
&
\sphinxAtStartPar
PWM通道                    | DC屏蔽寄存器,可通过此寄 | 分不同ADC的PWM触发通道  |
\\
\sphinxbottomrule
\end{tabular}
\sphinxtableafterendhook\par
\sphinxattableend\end{savenotes}


\begin{savenotes}\sphinxattablestart
\sphinxthistablewithglobalstyle
\centering
\begin{tabular}[t]{\X{12}{96}\X{12}{96}\X{12}{96}\X{12}{96}\X{12}{96}\X{12}{96}\X{12}{96}\X{12}{96}}
\sphinxtoprule
\sphinxtableatstartofbodyhook
\sphinxAtStartPar
31
&
\sphinxAtStartPar
30
&
\sphinxAtStartPar
29
&
\sphinxAtStartPar
28
&
\sphinxAtStartPar
27
&
\sphinxAtStartPar
26
&
\sphinxAtStartPar
25
&
\sphinxAtStartPar
24
\\
\sphinxhline\begin{itemize}
\item {} 
\end{itemize}
&&&&&&&\\
\sphinxhline
\sphinxAtStartPar
23
&
\sphinxAtStartPar
22
&
\sphinxAtStartPar
21
&
\sphinxAtStartPar
20
&
\sphinxAtStartPar
19
&
\sphinxAtStartPar
18
&
\sphinxAtStartPar
17
&
\sphinxAtStartPar
16
\\
\sphinxhline\begin{itemize}
\item {} 
\end{itemize}
&&&&&&&\\
\sphinxhline
\sphinxAtStartPar
15
&
\sphinxAtStartPar
14
&
\sphinxAtStartPar
13
&
\sphinxAtStartPar
12
&
\sphinxAtStartPar
11
&
\sphinxAtStartPar
10
&
\sphinxAtStartPar
9
&
\sphinxAtStartPar
8
\\
\sphinxhline\begin{itemize}
\item {} 
\end{itemize}
&&&&&&&\\
\sphinxhline
\sphinxAtStartPar
7
&
\sphinxAtStartPar
6
&
\sphinxAtStartPar
5
&
\sphinxAtStartPar
4
&
\sphinxAtStartPar
3
&
\sphinxAtStartPar
2
&
\sphinxAtStartPar
1
&
\sphinxAtStartPar
0
\\
\sphinxhline
\sphinxAtStartPar
PWM3B
&
\sphinxAtStartPar
PWM3A
&&&&&&\\
\sphinxbottomrule
\end{tabular}
\sphinxtableafterendhook\par
\sphinxattableend\end{savenotes}


\begin{savenotes}\sphinxattablestart
\sphinxthistablewithglobalstyle
\centering
\begin{tabular}[t]{\X{33}{99}\X{33}{99}\X{33}{99}}
\sphinxtoprule
\sphinxtableatstartofbodyhook
\sphinxAtStartPar
位域 |
&
\sphinxAtStartPar
名称     | |
&
\sphinxAtStartPar
描述                                        | |
\\
\sphinxhline
\sphinxAtStartPar
31:8
&\begin{itemize}
\item {} 
\end{itemize}
&\begin{itemize}
\item {} 
\end{itemize}
\\
\sphinxhline
\sphinxAtStartPar
7
&
\sphinxAtStartPar
PWM3B
&
\sphinxAtStartPar
PWM3B触发ADC屏蔽寄存器                      |

\sphinxAtStartPar
0:不屏蔽                                   |

\sphinxAtStartPar
1:屏蔽                                     |
\\
\sphinxhline
\sphinxAtStartPar
6
&
\sphinxAtStartPar
PWM3A
&
\sphinxAtStartPar
PWM3A触发ADC屏蔽寄存器                      |

\sphinxAtStartPar
0:不屏蔽                                   |

\sphinxAtStartPar
1:屏蔽                                     |
\\
\sphinxhline
\sphinxAtStartPar
5
&
\sphinxAtStartPar
PWM2B
&
\sphinxAtStartPar
PWM2B触发ADC屏蔽寄存器                      |

\sphinxAtStartPar
0:不屏蔽                                   |

\sphinxAtStartPar
1:屏蔽                                     |
\\
\sphinxhline
\sphinxAtStartPar
4
&
\sphinxAtStartPar
PWM2A
&
\sphinxAtStartPar
PWM2A触发ADC屏蔽寄存器                      |

\sphinxAtStartPar
0:不屏蔽                                   |

\sphinxAtStartPar
1:屏蔽                                     |
\\
\sphinxhline
\sphinxAtStartPar
3
&
\sphinxAtStartPar
PWM1B
&
\sphinxAtStartPar
PWM1B触发ADC屏蔽寄存器                      |

\sphinxAtStartPar
0:不屏蔽                                   |

\sphinxAtStartPar
1:屏蔽                                     |
\\
\sphinxhline
\sphinxAtStartPar
2
&
\sphinxAtStartPar
PWM1A
&
\sphinxAtStartPar
PWM1A触发ADC屏蔽寄存器                      |

\sphinxAtStartPar
0:不屏蔽                                   |

\sphinxAtStartPar
1:屏蔽                                     |
\\
\sphinxhline
\sphinxAtStartPar
1
&
\sphinxAtStartPar
PWM0B
&
\sphinxAtStartPar
PWM0B触发ADC屏蔽寄存器                      |

\sphinxAtStartPar
0:不屏蔽                                   |

\sphinxAtStartPar
1:屏蔽                                     |
\\
\sphinxhline
\sphinxAtStartPar
0
&
\sphinxAtStartPar
PWM0A
&
\sphinxAtStartPar
PWM0A触发ADC屏蔽寄存器                      |

\sphinxAtStartPar
0:不屏蔽                                   |

\sphinxAtStartPar
1:屏蔽                                     |
\\
\sphinxbottomrule
\end{tabular}
\sphinxtableafterendhook\par
\sphinxattableend\end{savenotes}


\subsubsection{ADC数据调整寄存器CALIBSET}
\label{\detokenize{SWM241/_u529f_u80fd_u63cf_u8ff0/_u6a21_u62df_u6570_u5b57_u8f6c_u6362_u5668:adccalibset}}

\begin{savenotes}\sphinxattablestart
\sphinxthistablewithglobalstyle
\centering
\begin{tabular}[t]{\X{20}{100}\X{20}{100}\X{20}{100}\X{20}{100}\X{20}{100}}
\sphinxtoprule
\sphinxtableatstartofbodyhook
\sphinxAtStartPar
寄存器 |
&
\begin{DUlineblock}{0em}
\item[] 偏移 |
\end{DUlineblock}
&
\begin{DUlineblock}{0em}
\item[] 
\item[] {\color{red}\bfseries{}|}
\end{DUlineblock}
&
\sphinxAtStartPar
复位值 |    描 | |
&
\begin{DUlineblock}{0em}
\item[] |
  |
\end{DUlineblock}
\\
\sphinxhline
\sphinxAtStartPar
CALIBSET
&
\sphinxAtStartPar
0x1f4
&&
\sphinxAtStartPar
0 000000
&
\sphinxAtStartPar
ADC数据调整寄存器          |
\\
\sphinxbottomrule
\end{tabular}
\sphinxtableafterendhook\par
\sphinxattableend\end{savenotes}


\begin{savenotes}\sphinxattablestart
\sphinxthistablewithglobalstyle
\centering
\begin{tabular}[t]{\X{12}{96}\X{12}{96}\X{12}{96}\X{12}{96}\X{12}{96}\X{12}{96}\X{12}{96}\X{12}{96}}
\sphinxtoprule
\sphinxtableatstartofbodyhook
\sphinxAtStartPar
31
&
\sphinxAtStartPar
30
&
\sphinxAtStartPar
29
&
\sphinxAtStartPar
28
&
\sphinxAtStartPar
27
&
\sphinxAtStartPar
26
&
\sphinxAtStartPar
25
&
\sphinxAtStartPar
24
\\
\sphinxhline\begin{itemize}
\item {} 
\end{itemize}
&&&&&&&
\sphinxAtStartPar
K
\\
\sphinxhline
\sphinxAtStartPar
23
&
\sphinxAtStartPar
22
&
\sphinxAtStartPar
21
&
\sphinxAtStartPar
20
&
\sphinxAtStartPar
19
&
\sphinxAtStartPar
18
&
\sphinxAtStartPar
17
&
\sphinxAtStartPar
16
\\
\sphinxhline
\sphinxAtStartPar
K
&&&&&&&\\
\sphinxhline
\sphinxAtStartPar
15
&
\sphinxAtStartPar
14
&
\sphinxAtStartPar
13
&
\sphinxAtStartPar
12
&
\sphinxAtStartPar
11
&
\sphinxAtStartPar
10
&
\sphinxAtStartPar
9
&
\sphinxAtStartPar
8
\\
\sphinxhline\begin{itemize}
\item {} 
\end{itemize}
&&&&&&&\\
\sphinxhline
\sphinxAtStartPar
7
&
\sphinxAtStartPar
6
&
\sphinxAtStartPar
5
&
\sphinxAtStartPar
4
&
\sphinxAtStartPar
3
&
\sphinxAtStartPar
2
&
\sphinxAtStartPar
1
&
\sphinxAtStartPar
0
\\
\sphinxhline
\sphinxAtStartPar
OFFSET
&&&&&&&\\
\sphinxbottomrule
\end{tabular}
\sphinxtableafterendhook\par
\sphinxattableend\end{savenotes}


\begin{savenotes}\sphinxattablestart
\sphinxthistablewithglobalstyle
\centering
\begin{tabular}[t]{\X{33}{99}\X{33}{99}\X{33}{99}}
\sphinxtoprule
\sphinxtableatstartofbodyhook
\sphinxAtStartPar
位域 |
&
\sphinxAtStartPar
名称     | |
&
\sphinxAtStartPar
描述                                        | |
\\
\sphinxhline
\sphinxAtStartPar
31:25
&\begin{itemize}
\item {} 
\end{itemize}
&\begin{itemize}
\item {} 
\end{itemize}
\\
\sphinxhline
\sphinxAtStartPar
24:16
&
\sphinxAtStartPar
K
&
\sphinxAtStartPar
ADC 整的K值(K始终大于1小于1.511)的小数部分 |

\sphinxAtStartPar
例如:要                                    | K值为1.230,则该寄存器直接写入230即可。  |
\\
\sphinxhline
\sphinxAtStartPar
15:9
&\begin{itemize}
\item {} 
\end{itemize}
&\begin{itemize}
\item {} 
\end{itemize}
\\
\sphinxhline
\sphinxAtStartPar
8:0
&
\sphinxAtStartPar
OFFSET
&
\sphinxAtStartPar
ADC数据调整的OFFSET值                       |
\\
\sphinxbottomrule
\end{tabular}
\sphinxtableafterendhook\par
\sphinxattableend\end{savenotes}


\subsubsection{ADC数据调整使能寄存器CALIBEN}
\label{\detokenize{SWM241/_u529f_u80fd_u63cf_u8ff0/_u6a21_u62df_u6570_u5b57_u8f6c_u6362_u5668:adccaliben}}

\begin{savenotes}\sphinxattablestart
\sphinxthistablewithglobalstyle
\centering
\begin{tabular}[t]{\X{20}{100}\X{20}{100}\X{20}{100}\X{20}{100}\X{20}{100}}
\sphinxtoprule
\sphinxtableatstartofbodyhook
\sphinxAtStartPar
寄存器 |
&
\begin{DUlineblock}{0em}
\item[] 偏移 |
\end{DUlineblock}
&
\begin{DUlineblock}{0em}
\item[] 
\item[] {\color{red}\bfseries{}|}
\end{DUlineblock}
&
\sphinxAtStartPar
复位值 |    描 | |
&
\begin{DUlineblock}{0em}
\item[] |
  |
\end{DUlineblock}
\\
\sphinxhline
\sphinxAtStartPar
CALIBEN
&
\sphinxAtStartPar
0x1f8
&&
\sphinxAtStartPar
0 000000
&
\sphinxAtStartPar
ADC数据调整使能寄存器      |
\\
\sphinxbottomrule
\end{tabular}
\sphinxtableafterendhook\par
\sphinxattableend\end{savenotes}


\begin{savenotes}\sphinxattablestart
\sphinxthistablewithglobalstyle
\centering
\begin{tabular}[t]{\X{12}{96}\X{12}{96}\X{12}{96}\X{12}{96}\X{12}{96}\X{12}{96}\X{12}{96}\X{12}{96}}
\sphinxtoprule
\sphinxtableatstartofbodyhook
\sphinxAtStartPar
31
&
\sphinxAtStartPar
30
&
\sphinxAtStartPar
29
&
\sphinxAtStartPar
28
&
\sphinxAtStartPar
27
&
\sphinxAtStartPar
26
&
\sphinxAtStartPar
25
&
\sphinxAtStartPar
24
\\
\sphinxhline\begin{itemize}
\item {} 
\end{itemize}
&&&&&&&\\
\sphinxhline
\sphinxAtStartPar
23
&
\sphinxAtStartPar
22
&
\sphinxAtStartPar
21
&
\sphinxAtStartPar
20
&
\sphinxAtStartPar
19
&
\sphinxAtStartPar
18
&
\sphinxAtStartPar
17
&
\sphinxAtStartPar
16
\\
\sphinxhline\begin{itemize}
\item {} 
\end{itemize}
&&&&&&&\\
\sphinxhline
\sphinxAtStartPar
15
&
\sphinxAtStartPar
14
&
\sphinxAtStartPar
13
&
\sphinxAtStartPar
12
&
\sphinxAtStartPar
11
&
\sphinxAtStartPar
10
&
\sphinxAtStartPar
9
&
\sphinxAtStartPar
8
\\
\sphinxhline\begin{itemize}
\item {} 
\end{itemize}
&&&&&&&\\
\sphinxhline
\sphinxAtStartPar
7
&
\sphinxAtStartPar
6
&
\sphinxAtStartPar
5
&
\sphinxAtStartPar
4
&
\sphinxAtStartPar
3
&
\sphinxAtStartPar
2
&
\sphinxAtStartPar
1
&
\sphinxAtStartPar
0
\\
\sphinxhline\begin{itemize}
\item {} 
\end{itemize}
&&&&&&
\sphinxAtStartPar
K
&\\
\sphinxbottomrule
\end{tabular}
\sphinxtableafterendhook\par
\sphinxattableend\end{savenotes}


\begin{savenotes}\sphinxattablestart
\sphinxthistablewithglobalstyle
\centering
\begin{tabular}[t]{\X{33}{99}\X{33}{99}\X{33}{99}}
\sphinxtoprule
\sphinxtableatstartofbodyhook
\sphinxAtStartPar
位域 |
&
\sphinxAtStartPar
名称     | |
&
\sphinxAtStartPar
描述                                        | |
\\
\sphinxhline
\sphinxAtStartPar
31:2
&\begin{itemize}
\item {} 
\end{itemize}
&\begin{itemize}
\item {} 
\end{itemize}
\\
\sphinxhline
\sphinxAtStartPar
1
&
\sphinxAtStartPar
K
&
\sphinxAtStartPar
ADC\_CALIB\_SET寄存器K配置数据是否有效        |

\sphinxAtStartPar
0:数据无效                                 |

\sphinxAtStartPar
1:数据有效                                 |
\\
\sphinxhline
\sphinxAtStartPar
0
&
\sphinxAtStartPar
OFFSET
&
\sphinxAtStartPar
ADC\_CALIB\_SET寄存器OFFSET配置数据是否有效   |

\sphinxAtStartPar
0:数据无效                                 |

\sphinxAtStartPar
1:数据有效                                 |
\\
\sphinxbottomrule
\end{tabular}
\sphinxtableafterendhook\par
\sphinxattableend\end{savenotes}

\sphinxstepscope


\section{直接内存存取(DMA)控制器}
\label{\detokenize{SWM241/_u529f_u80fd_u63cf_u8ff0/_u76f4_u63a5_u5185_u5b58_u5b58_u53d6:dma}}\label{\detokenize{SWM241/_u529f_u80fd_u63cf_u8ff0/_u76f4_u63a5_u5185_u5b58_u5b58_u53d6::doc}}
\sphinxAtStartPar
概述
\textasciitilde{}\textasciitilde{}

\sphinxAtStartPar
SWM241系列所有型号DMA模块操作均相同,用来提供特定外设(UART、SPI、ADC)和存储器(SRAM)之间或总线地址和存储器(SRAM)之间的高速数据传输,无需CPU干涉,数据可以快速的通过DMA传输,从而节省了CPU的资源来做其他操作。

\sphinxAtStartPar
DMA传输规则为按字传输,单次可传输字数多达4096Word。数据交换过程中,无需软件参与。

\sphinxAtStartPar
本文中RX指MIU0到MIU1的数据搬移,TX指MIU1到MIU0的数据搬移。

\sphinxAtStartPar
特性
\textasciitilde{}\textasciitilde{}
\begin{itemize}
\item {} 
\sphinxAtStartPar
支持UART/SPI/ADC与SRAM间数据交互

\item {} 
\sphinxAtStartPar
支持总线地址至SRAM间数据交互

\item {} 
\sphinxAtStartPar
支持多种传输模式及数据单位

\item {} 
\sphinxAtStartPar
支持TIMER触发使能

\item {} 
\sphinxAtStartPar
支持三种地址变化方式:递增,固定,scatter gathe

\item {} 
\sphinxAtStartPar
Master接口支持BYTE、HALFWORD和WORD操作

\end{itemize}


\subsection{模块结构框图}
\label{\detokenize{SWM241/_u529f_u80fd_u63cf_u8ff0/_u76f4_u63a5_u5185_u5b58_u5b58_u53d6:id1}}
\sphinxAtStartPar
DMA模块结构如图 6‑58所示:

\sphinxAtStartPar
\sphinxincludegraphics{{SWM241/功能描述/media直接内存存取002}.emf}

\sphinxAtStartPar
图 6‑58 DMA模块结构图

\sphinxAtStartPar
SIU是AHB slave接口,MCU通过这个接口配置相关的控制寄存器,同时也完成和外设之间的握手。

\sphinxAtStartPar
ARB0和ARB1用于仲裁各个通道的数据传输请求。

\sphinxAtStartPar
HALFPLEXCH是单向传输通道,在任意时刻只能配置为发送或接收方向。


\subsection{功能描述}
\label{\detokenize{SWM241/_u529f_u80fd_u63cf_u8ff0/_u76f4_u63a5_u5185_u5b58_u5b58_u53d6:id2}}

\subsubsection{通道选择}
\label{\detokenize{SWM241/_u529f_u80fd_u63cf_u8ff0/_u76f4_u63a5_u5185_u5b58_u5b58_u53d6:id3}}
\sphinxAtStartPar
DMA共有2组2个通道,可同时传输2组不同方向数据。通道与模块对应关系如表格 6‑3所示:

\sphinxAtStartPar
表格 6‑3 DMA各通道操作明细


\begin{savenotes}\sphinxattablestart
\sphinxthistablewithglobalstyle
\centering
\begin{tabular}[t]{\X{25}{100}\X{25}{100}\X{25}{100}\X{25}{100}}
\sphinxtoprule
\sphinxtableatstartofbodyhook
\sphinxAtStartPar
M0通道      |
&
\sphinxAtStartPar
对应外设     |    M
&
\sphinxAtStartPar
通道      |    对应外
&
\begin{DUlineblock}{0em}
\item[] 
\end{DUlineblock}
\\
\sphinxhline
\sphinxAtStartPar
CH0配置00   |
&
\sphinxAtStartPar
UART0 TX     |
&
\sphinxAtStartPar
CH0配置00   |
&
\sphinxAtStartPar
UART1 RX     |
\\
\sphinxhline
\sphinxAtStartPar
CH0配置01   |
&
\sphinxAtStartPar
SPI0 TX      |
&
\sphinxAtStartPar
CH0配置01   |
&
\sphinxAtStartPar
SPI1 RX      |
\\
\sphinxhline
\sphinxAtStartPar
CH0配置02   |
&
\sphinxAtStartPar
UART3 TX     |
&
\sphinxAtStartPar
CH0配置02   |
&
\sphinxAtStartPar
SARADC0      |
\\
\sphinxhline
\sphinxAtStartPar
CH0配置03   |
&\begin{itemize}
\item {} 
\begin{DUlineblock}{0em}
\item[] 
\end{DUlineblock}

\end{itemize}
&
\sphinxAtStartPar
CH0配置03   |
&
\sphinxAtStartPar
UART2 RX     |
\\
\sphinxhline
\sphinxAtStartPar
CH1配置00   |
&
\sphinxAtStartPar
UART1 TX     |
&
\sphinxAtStartPar
CH1配置00   |
&
\sphinxAtStartPar
UART0 RX     |
\\
\sphinxhline
\sphinxAtStartPar
CH1配置01   |
&
\sphinxAtStartPar
SPI1 TX      |
&
\sphinxAtStartPar
CH1配置01   |
&
\sphinxAtStartPar
SPI0 RX      |
\\
\sphinxhline
\sphinxAtStartPar
CH1配置02   |
&
\sphinxAtStartPar
UART2 TX     |
&
\sphinxAtStartPar
CH1配置02   |
&
\sphinxAtStartPar
SARADC0      |
\\
\sphinxhline
\sphinxAtStartPar
CH1配置03   |
&\begin{itemize}
\item {} 
\begin{DUlineblock}{0em}
\item[] 
\end{DUlineblock}

\end{itemize}
&
\sphinxAtStartPar
CH1配置03   |
&
\sphinxAtStartPar
UART3 RX     |
\\
\sphinxbottomrule
\end{tabular}
\sphinxtableafterendhook\par
\sphinxattableend\end{savenotes}

\sphinxAtStartPar
\sphinxstyleemphasis{注:在一个时间段内,同时使用的外设必须占用在不同的通道上,否则不能通过中断状态来区分哪个外设发生的事件}。


\subsubsection{模式选择}
\label{\detokenize{SWM241/_u529f_u80fd_u63cf_u8ff0/_u76f4_u63a5_u5185_u5b58_u5b58_u53d6:id4}}
\sphinxAtStartPar
支持三种地址变化方式:递增,固定,scatter gather。可通过配置AMn寄存器,分别配置源地址模式和目的地址模式,并可分别配置源和目的地址的位宽和传输模式。

\sphinxAtStartPar
\sphinxstylestrong{递增}

\sphinxAtStartPar
传输单位为字节时,从SRC指定地址+n处取数据(向DST指定地址+n处存数据),n表示第n个数据

\sphinxAtStartPar
传输单位为半字时,从SRC指定地址+2n处取数据(向DST指定地址+2n处存数据),n表示第n个数据

\sphinxAtStartPar
传输单位为 字 时,从SRC指定地址+4n处取数据(向DST指定地址+4n处存数据),n表示第n个数据

\sphinxAtStartPar
\sphinxstylestrong{固定}

\sphinxAtStartPar
固定从SRC指定地址处取数据、固定向DST指定地址处存数据。

\sphinxAtStartPar
\sphinxstylestrong{scatter} \sphinxstylestrong{gather}

\sphinxAtStartPar
源地址模式:

\sphinxAtStartPar
从SRCn开始,传输总长度1/4的数据;跳转到SRCSGADDRn1地址开始,再传输总长度1/4的数据;跳转到SRCSGADDRn2地址开始,再传输总长度1/4的数据;跳转到 SRCSGADDRn3地址开始,直到全部数据传输结束。

\sphinxAtStartPar
以源地址模式为scatter gather为例,传输40个字过程如下:
\begin{quote}

\sphinxAtStartPar
第一步、从SRCn指定地址处取10个字传输,

\sphinxAtStartPar
第二步、从SRCSGADDRn1指定地址处取10个字传输

\sphinxAtStartPar
第三步、从SRCSGADDRn2指定地址处取10个字传输

\sphinxAtStartPar
第四步、从SRCSGADDRn3指定地址处取10个字传输
\end{quote}

\sphinxAtStartPar
目的地址模式:
\begin{quote}

\sphinxAtStartPar
从DSTn0开始,传输总长度1/4的数据;跳转到DSTSGADDRn1地址开始,再传输总长度1/4的数据;跳转到DSTSGADDRn2地址开始,再传输总长度1/4的数据;跳转到DSTSGADDRn3地址开始,直到全部数据传输结束。

\sphinxAtStartPar
以目的地址模式为scatter gather为例,传输40个字过程如下:

\sphinxAtStartPar
第一步、向DSTn指定地址处存10个字传输,

\sphinxAtStartPar
第二步、向DSTSGADDRn1指定地址处存10个字传输

\sphinxAtStartPar
第三步、向DSTSGADDRn2指定地址处存10个字传输

\sphinxAtStartPar
第四步、向DSTSGADDRn3指定地址处存10个字传输
\end{quote}

\sphinxAtStartPar
三种模式下DMA搬运40个字流程如图 6‑59所示:

\sphinxAtStartPar
\sphinxincludegraphics{{SWM241/功能描述/media直接内存存取003}.emf}

\sphinxAtStartPar
图 6‑59 DMA搬运40个字流程图


\subsubsection{握手信号选择}
\label{\detokenize{SWM241/_u529f_u80fd_u63cf_u8ff0/_u76f4_u63a5_u5185_u5b58_u5b58_u53d6:id5}}
\sphinxAtStartPar
DMA通道可选择M1/M0总线上是否采用握手信号,可通过MUXn寄存器对应位来选择。

\sphinxAtStartPar
\sphinxstylestrong{握手}

\sphinxAtStartPar
需要通过握手信号进行信息交换

\sphinxAtStartPar
DMA通道可通过握手信号进行信息交换,选择由哪个外设的硬件信号来控制源或目标外设之间的传输,具体外设可以通过MUXn寄存器对应位来选择。

\sphinxAtStartPar
外设握手信号详情参考表格 6‑3。

\sphinxAtStartPar
具体外设有SPI、UART、ADC,在一个时间段内,可同时使用多个外设,但同时使用的外设必须占用在不同的通道上,否则不能通过中断状态来区分哪个外设发生的事件。

\sphinxAtStartPar
\sphinxstylestrong{非握手}

\sphinxAtStartPar
非握手状态下所有的地址都可以搬运,可以任何地址到任何地址。

\sphinxAtStartPar
握手、非握手传输示意图如图 6‑60所示:

\sphinxAtStartPar
\sphinxincludegraphics{{SWM241/功能描述/media直接内存存取004}.emf}

\sphinxAtStartPar
图 6‑60 握手、非握手信号传输图


\subsubsection{启动方式}
\label{\detokenize{SWM241/_u529f_u80fd_u63cf_u8ff0/_u76f4_u63a5_u5185_u5b58_u5b58_u53d6:id6}}
\sphinxAtStartPar
DMA通道启动传输的方式有两种,一种为通过软件操作TXEN/RXEN位启动、一种为通过外部trigger信号启动,可通过配置MUXn寄存器来选择。

\sphinxAtStartPar
\sphinxstylestrong{软件操作启动}

\sphinxAtStartPar
软件操作可通过配置CRn寄存器中的TXEN活RXEN启动DMA传输。

\sphinxAtStartPar
\sphinxstylestrong{外部trigger信号启动}

\sphinxAtStartPar
外部trigger信号触发有TMER0\textasciitilde{}8,可通过配置MUXn寄存器选择使用哪个TIMER溢出信号触发、,且通道会等到外部trigger信号为高后才启动传输。

\sphinxAtStartPar
配置如下:
\begin{itemize}
\item {} 
\sphinxAtStartPar
配置DMA模式,CRn寄存器相应位
\begin{itemize}
\item {} 
\sphinxAtStartPar
单次模式,传输完成后停止

\item {} 
\sphinxAtStartPar
环形模式,传输完成后从头执行下一轮传输

\end{itemize}

\item {} 
\sphinxAtStartPar
配置源地址传输位宽,AMn寄存器相应位
\begin{itemize}
\item {} 
\sphinxAtStartPar
字节

\item {} 
\sphinxAtStartPar
半字

\item {} 
\sphinxAtStartPar
字

\end{itemize}

\item {} 
\sphinxAtStartPar
配置传输长度,CRn寄存器相应位

\item {} 
\sphinxAtStartPar
配置源地址、目的地址,SRCn、DSTn寄存器

\item {} 
\sphinxAtStartPar
配置源和目标地址模式,AMn寄存器相应位
\begin{itemize}
\item {} 
\sphinxAtStartPar
地址固定

\item {} 
\sphinxAtStartPar
地址递增

\item {} 
\sphinxAtStartPar
scatter gather

\end{itemize}

\item {} 
\sphinxAtStartPar
配置通道由trigger信号启动,MUXn相应位

\item {} 
\sphinxAtStartPar
配置信号由哪个TIMER溢出信号触发,MUXn相应位

\item {} 
\sphinxAtStartPar
初始化定时器

\item {} 
\sphinxAtStartPar
启动定时器

\end{itemize}

\sphinxAtStartPar
DMA启动方式如图 6‑61所示:

\sphinxAtStartPar
\sphinxincludegraphics{{SWM241/功能描述/media直接内存存取005}.emf}

\sphinxAtStartPar
图 6‑61 DMA启动方式


\subsubsection{中断处理}
\label{\detokenize{SWM241/_u529f_u80fd_u63cf_u8ff0/_u76f4_u63a5_u5185_u5b58_u5b58_u53d6:id7}}
\sphinxAtStartPar
DMA控制器模块2个通道均支持传输结束中断。DMA初始化时如果配置了中断使能寄存器IE,当所配置通道传输完指定数据长度时会产生中断,此时中断状态寄存器IF对应位自动置1,对该位写1则清除中断,用户可通过读此寄存器来判断是否产生了中断。

\sphinxAtStartPar
DMA各个通道还具备中断屏蔽功能。当配置了中断屏蔽寄存器IM时,即使数据传输结束,也不会产生中断。


\subsubsection{优先级配置}
\label{\detokenize{SWM241/_u529f_u80fd_u63cf_u8ff0/_u76f4_u63a5_u5185_u5b58_u5b58_u53d6:id8}}
\sphinxAtStartPar
可通过PRI寄存器来配置DMA各个通道的优先级。当多个通道同时请求传输时,先执行优先级高的。但当低优先级正在传输时,高优先级不会打断低优先级的传输。


\subsection{寄存器映射}
\label{\detokenize{SWM241/_u529f_u80fd_u63cf_u8ff0/_u76f4_u63a5_u5185_u5b58_u5b58_u53d6:id9}}

\begin{savenotes}\sphinxattablestart
\sphinxthistablewithglobalstyle
\centering
\begin{tabular}[t]{\X{20}{100}\X{20}{100}\X{20}{100}\X{20}{100}\X{20}{100}}
\sphinxtoprule
\sphinxtableatstartofbodyhook
\sphinxAtStartPar
名称   |
&
\begin{DUlineblock}{0em}
\item[] 偏移 |
\end{DUlineblock}
&
\begin{DUlineblock}{0em}
\item[] 
\item[] |
|
\end{DUlineblock}
&
\begin{DUlineblock}{0em}
\item[] 
\end{DUlineblock}
\begin{quote}

\begin{DUlineblock}{0em}
\item[] 
\item[] 
\end{DUlineblock}
\end{quote}
&
\sphinxAtStartPar
描述                       | | | |
\\
\sphinxhline
\sphinxAtStartPar
DMABASE:0 {\color{red}\bfseries{}|}x40000800
&
\begin{DUlineblock}{0em}
\item[] 
\end{DUlineblock}
&&&\\
\sphinxhline
\sphinxAtStartPar
EN
&
\sphinxAtStartPar
0x00
&&
\sphinxAtStartPar
0x 00000
&
\sphinxAtStartPar
使能寄存器                 |
\\
\sphinxhline
\sphinxAtStartPar
IE
&
\sphinxAtStartPar
0x04
&&
\sphinxAtStartPar
0x 00000
&
\sphinxAtStartPar
通道中断使能               |
\\
\sphinxhline
\sphinxAtStartPar
IM
&
\sphinxAtStartPar
0x08
&&
\sphinxAtStartPar
0x 00000
&
\sphinxAtStartPar
通道中断屏蔽               |
\\
\sphinxhline
\sphinxAtStartPar
IF
&
\sphinxAtStartPar
0x0C
&&
\sphinxAtStartPar
0x 00000
&
\sphinxAtStartPar
通道中断状态               |
\\
\sphinxhline
\sphinxAtStartPar
DSTSGIE
&
\sphinxAtStartPar
0x10
&&
\sphinxAtStartPar
0x 00000
&
\sphinxAtStartPar
scatter gathe 0总线一侧传输完成中断使能 |
\\
\sphinxhline
\sphinxAtStartPar
DSTSGIM
&
\sphinxAtStartPar
0x14
&&
\sphinxAtStartPar
0x 00000
&
\sphinxAtStartPar
scatter gather M0总线一侧传输完成中断屏蔽 |
\\
\sphinxhline
\sphinxAtStartPar
DSTSGIF
&
\sphinxAtStartPar
0x18
&&
\sphinxAtStartPar
0x 00000
&
\sphinxAtStartPar
scatter gathe 0总线一侧传输完成中断状态 |
\\
\sphinxhline
\sphinxAtStartPar
SRCSGIE
&
\sphinxAtStartPar
0x1C
&&
\sphinxAtStartPar
0x 00000
&
\sphinxAtStartPar
scatter gather M1总线一侧传输完成中断使能 |
\\
\sphinxhline
\sphinxAtStartPar
SRCSGIM
&
\sphinxAtStartPar
0x20
&&
\sphinxAtStartPar
0x 00000
&
\sphinxAtStartPar
scatter gather M1总线一侧传输完成中断屏蔽 |
\\
\sphinxhline
\sphinxAtStartPar
SRCSGIF
&
\sphinxAtStartPar
0x24
&&
\sphinxAtStartPar
0x 00000
&
\sphinxAtStartPar
scatter gathe 1总线一侧传输完成中断状态 |
\\
\sphinxhline
\sphinxAtStartPar
PRI
&
\sphinxAtStartPar
0x3C
&&
\sphinxAtStartPar
0x 00000
&
\sphinxAtStartPar
通道优先级设定             |
\\
\sphinxhline
\sphinxAtStartPar
CRn
&
\sphinxAtStartPar
0 x40*n +  0x00
&&
\sphinxAtStartPar
0x 00000
&
\sphinxAtStartPar
通道控制                   |
\\
\sphinxhline
\sphinxAtStartPar
AMn
&
\sphinxAtStartPar
0 x40*n +  0x04
&&
\sphinxAtStartPar
0x 00000
&
\sphinxAtStartPar
通道地址模式               |
\\
\sphinxhline
\sphinxAtStartPar
DSTn
&
\sphinxAtStartPar
0 x40*n +  0x08
&&
\sphinxAtStartPar
0x 00000
&
\sphinxAtStartPar
通道n目的地址寄存器        |
\\
\sphinxhline
\sphinxAtStartPar
DSTSGADDRn1
&
\sphinxAtStartPar
0 x40*n +  0xC
&&
\sphinxAtStartPar
0x 00000
&
\sphinxAtStartPar
Destination Scatter Gather

\sphinxAtStartPar
ress通道n目的分散收集地址1 |
\\
\sphinxhline
\sphinxAtStartPar
DSTSGADDRn2
&
\sphinxAtStartPar
0 x40*n +  0x10
&&
\sphinxAtStartPar
0x 00000
&
\sphinxAtStartPar
Destination Scatter Gather

\sphinxAtStartPar
ress通道n目的分散收集地址2 |
\\
\sphinxhline
\sphinxAtStartPar
DSTSGADDRn3
&
\sphinxAtStartPar
0 x40*n +  0x24
&&
\sphinxAtStartPar
0x 00000
&
\sphinxAtStartPar
Destination Scatter Gather

\sphinxAtStartPar
ress通道n目的分散收集地址3 |
\\
\sphinxhline
\sphinxAtStartPar
MUXn
&
\sphinxAtStartPar
0 x40*n +  0x18
&&
\sphinxAtStartPar
0x 00000
&
\sphinxAtStartPar
通道n握手信号选择寄存器    |
\\
\sphinxhline
\sphinxAtStartPar
SRCn
&
\sphinxAtStartPar
0 x40*n +  0x1C
&&
\sphinxAtStartPar
0x 00000
&
\sphinxAtStartPar
通道n源地址寄存器          |
\\
\sphinxhline
\sphinxAtStartPar
SRCSGADDRn1
&
\sphinxAtStartPar
0 x40*n +  0x20
&&
\sphinxAtStartPar
0x 00000
&
\sphinxAtStartPar
Source Scatter Gather

\sphinxAtStartPar
ddress通道n源分散收集地址1 |
\\
\sphinxhline
\sphinxAtStartPar
SRCSGADDRn2
&
\sphinxAtStartPar
0 x40*n +  0x24
&&
\sphinxAtStartPar
0x 00000
&
\sphinxAtStartPar
Source Scatter Gather

\sphinxAtStartPar
ddress通道n源分散收集地址2 |
\\
\sphinxhline
\sphinxAtStartPar
SRCSGADDRn3
&
\sphinxAtStartPar
0 x40*n +  0x28
&&
\sphinxAtStartPar
0x 00000
&
\sphinxAtStartPar
Source Scatter Gather

\sphinxAtStartPar
ddress通道n源分散收集地址3 |
\\
\sphinxhline
\sphinxAtStartPar
DSTSR
&
\sphinxAtStartPar
0 x40*n +  0x2C
&&
\sphinxAtStartPar
0x 00000
&
\sphinxAtStartPar
M0通道n状态寄存器          |
\\
\sphinxhline
\sphinxAtStartPar
SRCSR
&
\sphinxAtStartPar
0 x40*n +  0x30
&&
\sphinxAtStartPar
0x 00000
&
\sphinxAtStartPar
M1通道n状态寄存器          |
\\
\sphinxbottomrule
\end{tabular}
\sphinxtableafterendhook\par
\sphinxattableend\end{savenotes}


\subsection{寄存器描述}
\label{\detokenize{SWM241/_u529f_u80fd_u63cf_u8ff0/_u76f4_u63a5_u5185_u5b58_u5b58_u53d6:id12}}

\subsubsection{DMA使能寄存器EN}
\label{\detokenize{SWM241/_u529f_u80fd_u63cf_u8ff0/_u76f4_u63a5_u5185_u5b58_u5b58_u53d6:dmaen}}

\begin{savenotes}\sphinxattablestart
\sphinxthistablewithglobalstyle
\centering
\begin{tabular}[t]{\X{20}{100}\X{20}{100}\X{20}{100}\X{20}{100}\X{20}{100}}
\sphinxtoprule
\sphinxtableatstartofbodyhook
\sphinxAtStartPar
寄存器 |
&
\begin{DUlineblock}{0em}
\item[] 偏移 |
\end{DUlineblock}
&
\begin{DUlineblock}{0em}
\item[] 
\item[] {\color{red}\bfseries{}|}
\end{DUlineblock}
&
\sphinxAtStartPar
复位值 |    描 | |
&
\begin{DUlineblock}{0em}
\item[] |
  |
\end{DUlineblock}
\\
\sphinxhline
\sphinxAtStartPar
EN
&
\sphinxAtStartPar
0x00
&&
\sphinxAtStartPar
0 000000
&
\sphinxAtStartPar
使能寄存器                 |
\\
\sphinxbottomrule
\end{tabular}
\sphinxtableafterendhook\par
\sphinxattableend\end{savenotes}


\begin{savenotes}\sphinxattablestart
\sphinxthistablewithglobalstyle
\centering
\begin{tabular}[t]{\X{12}{96}\X{12}{96}\X{12}{96}\X{12}{96}\X{12}{96}\X{12}{96}\X{12}{96}\X{12}{96}}
\sphinxtoprule
\sphinxtableatstartofbodyhook
\sphinxAtStartPar
31
&
\sphinxAtStartPar
30
&
\sphinxAtStartPar
29
&
\sphinxAtStartPar
28
&
\sphinxAtStartPar
27
&
\sphinxAtStartPar
26
&
\sphinxAtStartPar
25
&
\sphinxAtStartPar
24
\\
\sphinxhline\begin{itemize}
\item {} 
\end{itemize}
&&&&&&&\\
\sphinxhline
\sphinxAtStartPar
23
&
\sphinxAtStartPar
22
&
\sphinxAtStartPar
21
&
\sphinxAtStartPar
20
&
\sphinxAtStartPar
19
&
\sphinxAtStartPar
18
&
\sphinxAtStartPar
17
&
\sphinxAtStartPar
16
\\
\sphinxhline\begin{itemize}
\item {} 
\end{itemize}
&&&&&&&\\
\sphinxhline
\sphinxAtStartPar
15
&
\sphinxAtStartPar
14
&
\sphinxAtStartPar
13
&
\sphinxAtStartPar
12
&
\sphinxAtStartPar
11
&
\sphinxAtStartPar
10
&
\sphinxAtStartPar
9
&
\sphinxAtStartPar
8
\\
\sphinxhline\begin{itemize}
\item {} 
\end{itemize}
&&&&&&&\\
\sphinxhline
\sphinxAtStartPar
7
&
\sphinxAtStartPar
6
&
\sphinxAtStartPar
5
&
\sphinxAtStartPar
4
&
\sphinxAtStartPar
3
&
\sphinxAtStartPar
2
&
\sphinxAtStartPar
1
&
\sphinxAtStartPar
0
\\
\sphinxhline\begin{itemize}
\item {} 
\end{itemize}
&&&&&&&
\sphinxAtStartPar
EN
\\
\sphinxbottomrule
\end{tabular}
\sphinxtableafterendhook\par
\sphinxattableend\end{savenotes}


\begin{savenotes}\sphinxattablestart
\sphinxthistablewithglobalstyle
\centering
\begin{tabular}[t]{\X{33}{99}\X{33}{99}\X{33}{99}}
\sphinxtoprule
\sphinxtableatstartofbodyhook
\sphinxAtStartPar
位域 |
&
\sphinxAtStartPar
名称     | |
&
\sphinxAtStartPar
描述                                        | |
\\
\sphinxhline
\sphinxAtStartPar
31:1
&\begin{itemize}
\item {} 
\end{itemize}
&\begin{itemize}
\item {} 
\end{itemize}
\\
\sphinxhline
\sphinxAtStartPar
0
&
\sphinxAtStartPar
EN
&
\sphinxAtStartPar
DMA使能                                     |

\sphinxAtStartPar
1:使能                                     |

\sphinxAtStartPar
0:禁能                                     |
\\
\sphinxbottomrule
\end{tabular}
\sphinxtableafterendhook\par
\sphinxattableend\end{savenotes}


\subsubsection{DMA中断使能寄存器IE}
\label{\detokenize{SWM241/_u529f_u80fd_u63cf_u8ff0/_u76f4_u63a5_u5185_u5b58_u5b58_u53d6:dmaie}}

\begin{savenotes}\sphinxattablestart
\sphinxthistablewithglobalstyle
\centering
\begin{tabular}[t]{\X{20}{100}\X{20}{100}\X{20}{100}\X{20}{100}\X{20}{100}}
\sphinxtoprule
\sphinxtableatstartofbodyhook
\sphinxAtStartPar
寄存器 |
&
\begin{DUlineblock}{0em}
\item[] 偏移 |
\end{DUlineblock}
&
\begin{DUlineblock}{0em}
\item[] 
\item[] {\color{red}\bfseries{}|}
\end{DUlineblock}
&
\sphinxAtStartPar
复位值 |    描 | |
&
\begin{DUlineblock}{0em}
\item[] |
  |
\end{DUlineblock}
\\
\sphinxhline
\sphinxAtStartPar
IE
&
\sphinxAtStartPar
0x04
&&
\sphinxAtStartPar
0 000000
&
\sphinxAtStartPar
通道中断使能               |
\\
\sphinxbottomrule
\end{tabular}
\sphinxtableafterendhook\par
\sphinxattableend\end{savenotes}


\begin{savenotes}\sphinxattablestart
\sphinxthistablewithglobalstyle
\centering
\begin{tabular}[t]{\X{12}{96}\X{12}{96}\X{12}{96}\X{12}{96}\X{12}{96}\X{12}{96}\X{12}{96}\X{12}{96}}
\sphinxtoprule
\sphinxtableatstartofbodyhook
\sphinxAtStartPar
31
&
\sphinxAtStartPar
30
&
\sphinxAtStartPar
29
&
\sphinxAtStartPar
28
&
\sphinxAtStartPar
27
&
\sphinxAtStartPar
26
&
\sphinxAtStartPar
25
&
\sphinxAtStartPar
24
\\
\sphinxhline\begin{itemize}
\item {} 
\end{itemize}
&&&&&&&\\
\sphinxhline
\sphinxAtStartPar
23
&
\sphinxAtStartPar
22
&
\sphinxAtStartPar
21
&
\sphinxAtStartPar
20
&
\sphinxAtStartPar
19
&
\sphinxAtStartPar
18
&
\sphinxAtStartPar
17
&
\sphinxAtStartPar
16
\\
\sphinxhline\begin{itemize}
\item {} 
\end{itemize}
&&&&&&&\\
\sphinxhline
\sphinxAtStartPar
15
&
\sphinxAtStartPar
14
&
\sphinxAtStartPar
13
&
\sphinxAtStartPar
12
&
\sphinxAtStartPar
11
&
\sphinxAtStartPar
10
&
\sphinxAtStartPar
9
&
\sphinxAtStartPar
8
\\
\sphinxhline\begin{itemize}
\item {} 
\end{itemize}
&&&&&&&\\
\sphinxhline
\sphinxAtStartPar
7
&
\sphinxAtStartPar
6
&
\sphinxAtStartPar
5
&
\sphinxAtStartPar
4
&
\sphinxAtStartPar
3
&
\sphinxAtStartPar
2
&
\sphinxAtStartPar
1
&
\sphinxAtStartPar
0
\\
\sphinxhline\begin{itemize}
\item {} 
\end{itemize}
&&&&&&&
\sphinxAtStartPar
CH0
\\
\sphinxbottomrule
\end{tabular}
\sphinxtableafterendhook\par
\sphinxattableend\end{savenotes}


\begin{savenotes}\sphinxattablestart
\sphinxthistablewithglobalstyle
\centering
\begin{tabular}[t]{\X{33}{99}\X{33}{99}\X{33}{99}}
\sphinxtoprule
\sphinxtableatstartofbodyhook
\sphinxAtStartPar
位域 |
&
\sphinxAtStartPar
名称     | |
&
\sphinxAtStartPar
描述                                        | |
\\
\sphinxhline
\sphinxAtStartPar
31:2
&\begin{itemize}
\item {} 
\end{itemize}
&\begin{itemize}
\item {} 
\end{itemize}
\\
\sphinxhline
\sphinxAtStartPar
1
&
\sphinxAtStartPar
CH1
&
\sphinxAtStartPar
通道1传输完成中断使能                       |

\sphinxAtStartPar
1:使能                                     |

\sphinxAtStartPar
0:禁能                                     |
\\
\sphinxhline
\sphinxAtStartPar
0
&
\sphinxAtStartPar
CH0
&
\sphinxAtStartPar
通道0传输完成中断使能                       |

\sphinxAtStartPar
1:使能                                     |

\sphinxAtStartPar
0:禁能                                     |
\\
\sphinxbottomrule
\end{tabular}
\sphinxtableafterendhook\par
\sphinxattableend\end{savenotes}


\subsubsection{DMA中断屏蔽寄存器IM}
\label{\detokenize{SWM241/_u529f_u80fd_u63cf_u8ff0/_u76f4_u63a5_u5185_u5b58_u5b58_u53d6:dmaim}}

\begin{savenotes}\sphinxattablestart
\sphinxthistablewithglobalstyle
\centering
\begin{tabular}[t]{\X{20}{100}\X{20}{100}\X{20}{100}\X{20}{100}\X{20}{100}}
\sphinxtoprule
\sphinxtableatstartofbodyhook
\sphinxAtStartPar
寄存器 |
&
\begin{DUlineblock}{0em}
\item[] 偏移 |
\end{DUlineblock}
&
\begin{DUlineblock}{0em}
\item[] 
\item[] {\color{red}\bfseries{}|}
\end{DUlineblock}
&
\sphinxAtStartPar
复位值 |    描 | |
&
\begin{DUlineblock}{0em}
\item[] |
  |
\end{DUlineblock}
\\
\sphinxhline
\sphinxAtStartPar
IM
&
\sphinxAtStartPar
0x08
&&
\sphinxAtStartPar
0 000000
&
\sphinxAtStartPar
通道中断屏蔽               |
\\
\sphinxbottomrule
\end{tabular}
\sphinxtableafterendhook\par
\sphinxattableend\end{savenotes}


\begin{savenotes}\sphinxattablestart
\sphinxthistablewithglobalstyle
\centering
\begin{tabular}[t]{\X{12}{96}\X{12}{96}\X{12}{96}\X{12}{96}\X{12}{96}\X{12}{96}\X{12}{96}\X{12}{96}}
\sphinxtoprule
\sphinxtableatstartofbodyhook
\sphinxAtStartPar
31
&
\sphinxAtStartPar
30
&
\sphinxAtStartPar
29
&
\sphinxAtStartPar
28
&
\sphinxAtStartPar
27
&
\sphinxAtStartPar
26
&
\sphinxAtStartPar
25
&
\sphinxAtStartPar
24
\\
\sphinxhline\begin{itemize}
\item {} 
\end{itemize}
&&&&&&&\\
\sphinxhline
\sphinxAtStartPar
23
&
\sphinxAtStartPar
22
&
\sphinxAtStartPar
21
&
\sphinxAtStartPar
20
&
\sphinxAtStartPar
19
&
\sphinxAtStartPar
18
&
\sphinxAtStartPar
17
&
\sphinxAtStartPar
16
\\
\sphinxhline\begin{itemize}
\item {} 
\end{itemize}
&&&&&&&\\
\sphinxhline
\sphinxAtStartPar
15
&
\sphinxAtStartPar
14
&
\sphinxAtStartPar
13
&
\sphinxAtStartPar
12
&
\sphinxAtStartPar
11
&
\sphinxAtStartPar
10
&
\sphinxAtStartPar
9
&
\sphinxAtStartPar
8
\\
\sphinxhline\begin{itemize}
\item {} 
\end{itemize}
&&&&&&&\\
\sphinxhline
\sphinxAtStartPar
7
&
\sphinxAtStartPar
6
&
\sphinxAtStartPar
5
&
\sphinxAtStartPar
4
&
\sphinxAtStartPar
3
&
\sphinxAtStartPar
2
&
\sphinxAtStartPar
1
&
\sphinxAtStartPar
0
\\
\sphinxhline\begin{itemize}
\item {} 
\end{itemize}
&&&&&&&
\sphinxAtStartPar
CH0
\\
\sphinxbottomrule
\end{tabular}
\sphinxtableafterendhook\par
\sphinxattableend\end{savenotes}


\begin{savenotes}\sphinxattablestart
\sphinxthistablewithglobalstyle
\centering
\begin{tabular}[t]{\X{33}{99}\X{33}{99}\X{33}{99}}
\sphinxtoprule
\sphinxtableatstartofbodyhook
\sphinxAtStartPar
位域 |
&
\sphinxAtStartPar
名称     | |
&
\sphinxAtStartPar
描述                                        | |
\\
\sphinxhline
\sphinxAtStartPar
31:2
&\begin{itemize}
\item {} 
\end{itemize}
&\begin{itemize}
\item {} 
\end{itemize}
\\
\sphinxhline
\sphinxAtStartPar
1
&
\sphinxAtStartPar
CH1
&
\sphinxAtStartPar
通道1传输完成中断屏蔽                       |

\sphinxAtStartPar
1:屏蔽                                     |

\sphinxAtStartPar
0:非屏蔽                                   |
\\
\sphinxhline
\sphinxAtStartPar
0
&
\sphinxAtStartPar
CH0
&
\sphinxAtStartPar
通道0传输完成中断屏蔽                       |

\sphinxAtStartPar
1:屏蔽                                     |

\sphinxAtStartPar
0:非屏蔽                                   |
\\
\sphinxbottomrule
\end{tabular}
\sphinxtableafterendhook\par
\sphinxattableend\end{savenotes}


\subsubsection{DMA中断状态寄存器IF}
\label{\detokenize{SWM241/_u529f_u80fd_u63cf_u8ff0/_u76f4_u63a5_u5185_u5b58_u5b58_u53d6:dmaif}}

\begin{savenotes}\sphinxattablestart
\sphinxthistablewithglobalstyle
\centering
\begin{tabular}[t]{\X{20}{100}\X{20}{100}\X{20}{100}\X{20}{100}\X{20}{100}}
\sphinxtoprule
\sphinxtableatstartofbodyhook
\sphinxAtStartPar
寄存器 |
&
\begin{DUlineblock}{0em}
\item[] 偏移 |
\end{DUlineblock}
&
\begin{DUlineblock}{0em}
\item[] 
\item[] {\color{red}\bfseries{}|}
\end{DUlineblock}
&
\sphinxAtStartPar
复位值 |    描 | |
&
\begin{DUlineblock}{0em}
\item[] |
  |
\end{DUlineblock}
\\
\sphinxhline
\sphinxAtStartPar
IF
&
\sphinxAtStartPar
0x0C
&&
\sphinxAtStartPar
0 000000
&
\sphinxAtStartPar
通道中断状态               |
\\
\sphinxbottomrule
\end{tabular}
\sphinxtableafterendhook\par
\sphinxattableend\end{savenotes}


\begin{savenotes}\sphinxattablestart
\sphinxthistablewithglobalstyle
\centering
\begin{tabular}[t]{\X{12}{96}\X{12}{96}\X{12}{96}\X{12}{96}\X{12}{96}\X{12}{96}\X{12}{96}\X{12}{96}}
\sphinxtoprule
\sphinxtableatstartofbodyhook
\sphinxAtStartPar
31
&
\sphinxAtStartPar
30
&
\sphinxAtStartPar
29
&
\sphinxAtStartPar
28
&
\sphinxAtStartPar
27
&
\sphinxAtStartPar
26
&
\sphinxAtStartPar
25
&
\sphinxAtStartPar
24
\\
\sphinxhline\begin{itemize}
\item {} 
\end{itemize}
&&&&&&&\\
\sphinxhline
\sphinxAtStartPar
23
&
\sphinxAtStartPar
22
&
\sphinxAtStartPar
21
&
\sphinxAtStartPar
20
&
\sphinxAtStartPar
19
&
\sphinxAtStartPar
18
&
\sphinxAtStartPar
17
&
\sphinxAtStartPar
16
\\
\sphinxhline\begin{itemize}
\item {} 
\end{itemize}
&&&&&&&\\
\sphinxhline
\sphinxAtStartPar
15
&
\sphinxAtStartPar
14
&
\sphinxAtStartPar
13
&
\sphinxAtStartPar
12
&
\sphinxAtStartPar
11
&
\sphinxAtStartPar
10
&
\sphinxAtStartPar
9
&
\sphinxAtStartPar
8
\\
\sphinxhline\begin{itemize}
\item {} 
\end{itemize}
&&&&&&&\\
\sphinxhline
\sphinxAtStartPar
7
&
\sphinxAtStartPar
6
&
\sphinxAtStartPar
5
&
\sphinxAtStartPar
4
&
\sphinxAtStartPar
3
&
\sphinxAtStartPar
2
&
\sphinxAtStartPar
1
&
\sphinxAtStartPar
0
\\
\sphinxhline\begin{itemize}
\item {} 
\end{itemize}
&&&&&&&
\sphinxAtStartPar
CH0
\\
\sphinxbottomrule
\end{tabular}
\sphinxtableafterendhook\par
\sphinxattableend\end{savenotes}


\begin{savenotes}\sphinxattablestart
\sphinxthistablewithglobalstyle
\centering
\begin{tabular}[t]{\X{33}{99}\X{33}{99}\X{33}{99}}
\sphinxtoprule
\sphinxtableatstartofbodyhook
\sphinxAtStartPar
位域 |
&
\sphinxAtStartPar
名称     | |
&
\sphinxAtStartPar
描述                                        | |
\\
\sphinxhline
\sphinxAtStartPar
31:2
&\begin{itemize}
\item {} 
\end{itemize}
&\begin{itemize}
\item {} 
\end{itemize}
\\
\sphinxhline
\sphinxAtStartPar
1
&
\sphinxAtStartPar
CH1
&
\sphinxAtStartPar
通道1传输完成中断,写1清零                  |

\sphinxAtStartPar
1:中断发生                                 |

\sphinxAtStartPar
0:中断未发生                               |
\\
\sphinxhline
\sphinxAtStartPar
0
&
\sphinxAtStartPar
CH0
&
\sphinxAtStartPar
通道0传输完成中断,写1清零                  |

\sphinxAtStartPar
1:中断发生                                 |

\sphinxAtStartPar
0:中断未发生                               |
\\
\sphinxbottomrule
\end{tabular}
\sphinxtableafterendhook\par
\sphinxattableend\end{savenotes}


\subsubsection{M0总线一侧传输完成中断使能寄存器DSTSGIE}
\label{\detokenize{SWM241/_u529f_u80fd_u63cf_u8ff0/_u76f4_u63a5_u5185_u5b58_u5b58_u53d6:m0dstsgie}}

\begin{savenotes}\sphinxattablestart
\sphinxthistablewithglobalstyle
\centering
\begin{tabular}[t]{\X{20}{100}\X{20}{100}\X{20}{100}\X{20}{100}\X{20}{100}}
\sphinxtoprule
\sphinxtableatstartofbodyhook
\sphinxAtStartPar
寄存器 |
&
\begin{DUlineblock}{0em}
\item[] 偏移 |
\end{DUlineblock}
&
\begin{DUlineblock}{0em}
\item[] 
\item[] {\color{red}\bfseries{}|}
\end{DUlineblock}
&
\sphinxAtStartPar
复位值 |    描 | |
&
\begin{DUlineblock}{0em}
\item[] |
  |
\end{DUlineblock}
\\
\sphinxhline
\sphinxAtStartPar
DSTSGIE
&
\sphinxAtStartPar
0x10
&&
\sphinxAtStartPar
0 000000
&
\sphinxAtStartPar
scatter gathe 0总线一侧传输完成中断使能 |
\\
\sphinxbottomrule
\end{tabular}
\sphinxtableafterendhook\par
\sphinxattableend\end{savenotes}


\begin{savenotes}\sphinxattablestart
\sphinxthistablewithglobalstyle
\centering
\begin{tabular}[t]{\X{12}{96}\X{12}{96}\X{12}{96}\X{12}{96}\X{12}{96}\X{12}{96}\X{12}{96}\X{12}{96}}
\sphinxtoprule
\sphinxtableatstartofbodyhook
\sphinxAtStartPar
31
&
\sphinxAtStartPar
30
&
\sphinxAtStartPar
29
&
\sphinxAtStartPar
28
&
\sphinxAtStartPar
27
&
\sphinxAtStartPar
26
&
\sphinxAtStartPar
25
&
\sphinxAtStartPar
24
\\
\sphinxhline\begin{itemize}
\item {} 
\end{itemize}
&&&&&&&\\
\sphinxhline
\sphinxAtStartPar
23
&
\sphinxAtStartPar
22
&
\sphinxAtStartPar
21
&
\sphinxAtStartPar
20
&
\sphinxAtStartPar
19
&
\sphinxAtStartPar
18
&
\sphinxAtStartPar
17
&
\sphinxAtStartPar
16
\\
\sphinxhline\begin{itemize}
\item {} 
\end{itemize}
&&&&&&&\\
\sphinxhline
\sphinxAtStartPar
15
&
\sphinxAtStartPar
14
&
\sphinxAtStartPar
13
&
\sphinxAtStartPar
12
&
\sphinxAtStartPar
11
&
\sphinxAtStartPar
10
&
\sphinxAtStartPar
9
&
\sphinxAtStartPar
8
\\
\sphinxhline\begin{itemize}
\item {} 
\end{itemize}
&&&&&&&\\
\sphinxhline
\sphinxAtStartPar
7
&
\sphinxAtStartPar
6
&
\sphinxAtStartPar
5
&
\sphinxAtStartPar
4
&
\sphinxAtStartPar
3
&
\sphinxAtStartPar
2
&
\sphinxAtStartPar
1
&
\sphinxAtStartPar
0
\\
\sphinxhline\begin{itemize}
\item {} 
\end{itemize}
&&&&&&&\\
\sphinxbottomrule
\end{tabular}
\sphinxtableafterendhook\par
\sphinxattableend\end{savenotes}


\begin{savenotes}\sphinxattablestart
\sphinxthistablewithglobalstyle
\centering
\begin{tabular}[t]{\X{33}{99}\X{33}{99}\X{33}{99}}
\sphinxtoprule
\sphinxtableatstartofbodyhook
\sphinxAtStartPar
位域 |
&
\sphinxAtStartPar
名称     | |
&
\sphinxAtStartPar
描述                                        | |
\\
\sphinxhline
\sphinxAtStartPar
31:4
&\begin{itemize}
\item {} 
\end{itemize}
&\begin{itemize}
\item {} 
\end{itemize}
\\
\sphinxhline
\sphinxAtStartPar
3
&
\sphinxAtStartPar
CH1
&
\sphinxAtStartPar
CH1 M0总线一侧传输完成中断使能              |

\sphinxAtStartPar
1:使能                                     |

\sphinxAtStartPar
0:禁能                                     |
\\
\sphinxhline
\sphinxAtStartPar
2
&
\sphinxAtStartPar
CH1HF
&
\sphinxAtStartPar
CH1 M0总线一侧传输完成1/2中断使能           |

\sphinxAtStartPar
1:使能                                     |

\sphinxAtStartPar
0:禁能                                     |
\\
\sphinxhline
\sphinxAtStartPar
1
&
\sphinxAtStartPar
CH0
&
\sphinxAtStartPar
CH0 M0总线一侧传输完成中断使能              |

\sphinxAtStartPar
1:使能                                     |

\sphinxAtStartPar
0:禁能                                     |
\\
\sphinxhline
\sphinxAtStartPar
0
&
\sphinxAtStartPar
CH0HF
&
\sphinxAtStartPar
CH0 M0总线一侧传输完成1/2中断使能           |

\sphinxAtStartPar
1:使能                                     |

\sphinxAtStartPar
0:禁能                                     |
\\
\sphinxbottomrule
\end{tabular}
\sphinxtableafterendhook\par
\sphinxattableend\end{savenotes}


\subsubsection{M0总线一侧传输完成中断屏蔽寄存器DSTSGIM}
\label{\detokenize{SWM241/_u529f_u80fd_u63cf_u8ff0/_u76f4_u63a5_u5185_u5b58_u5b58_u53d6:m0dstsgim}}

\begin{savenotes}\sphinxattablestart
\sphinxthistablewithglobalstyle
\centering
\begin{tabular}[t]{\X{20}{100}\X{20}{100}\X{20}{100}\X{20}{100}\X{20}{100}}
\sphinxtoprule
\sphinxtableatstartofbodyhook
\sphinxAtStartPar
寄存器 |
&
\begin{DUlineblock}{0em}
\item[] 偏移 |
\end{DUlineblock}
&
\begin{DUlineblock}{0em}
\item[] 
\item[] {\color{red}\bfseries{}|}
\end{DUlineblock}
&
\sphinxAtStartPar
复位值 |    描 | |
&
\begin{DUlineblock}{0em}
\item[] |
  |
\end{DUlineblock}
\\
\sphinxhline
\sphinxAtStartPar
DSTSGIM
&
\sphinxAtStartPar
0x14
&&
\sphinxAtStartPar
0 000000
&
\sphinxAtStartPar
scatter gather M0总线一侧传输完成中断屏蔽 |
\\
\sphinxbottomrule
\end{tabular}
\sphinxtableafterendhook\par
\sphinxattableend\end{savenotes}


\begin{savenotes}\sphinxattablestart
\sphinxthistablewithglobalstyle
\centering
\begin{tabular}[t]{\X{12}{96}\X{12}{96}\X{12}{96}\X{12}{96}\X{12}{96}\X{12}{96}\X{12}{96}\X{12}{96}}
\sphinxtoprule
\sphinxtableatstartofbodyhook
\sphinxAtStartPar
31
&
\sphinxAtStartPar
30
&
\sphinxAtStartPar
29
&
\sphinxAtStartPar
28
&
\sphinxAtStartPar
27
&
\sphinxAtStartPar
26
&
\sphinxAtStartPar
25
&
\sphinxAtStartPar
24
\\
\sphinxhline\begin{itemize}
\item {} 
\end{itemize}
&&&&&&&\\
\sphinxhline
\sphinxAtStartPar
23
&
\sphinxAtStartPar
22
&
\sphinxAtStartPar
21
&
\sphinxAtStartPar
20
&
\sphinxAtStartPar
19
&
\sphinxAtStartPar
18
&
\sphinxAtStartPar
17
&
\sphinxAtStartPar
16
\\
\sphinxhline\begin{itemize}
\item {} 
\end{itemize}
&&&&&&&\\
\sphinxhline
\sphinxAtStartPar
15
&
\sphinxAtStartPar
14
&
\sphinxAtStartPar
13
&
\sphinxAtStartPar
12
&
\sphinxAtStartPar
11
&
\sphinxAtStartPar
10
&
\sphinxAtStartPar
9
&
\sphinxAtStartPar
8
\\
\sphinxhline\begin{itemize}
\item {} 
\end{itemize}
&&&&&&&\\
\sphinxhline
\sphinxAtStartPar
7
&
\sphinxAtStartPar
6
&
\sphinxAtStartPar
5
&
\sphinxAtStartPar
4
&
\sphinxAtStartPar
3
&
\sphinxAtStartPar
2
&
\sphinxAtStartPar
1
&
\sphinxAtStartPar
0
\\
\sphinxhline\begin{itemize}
\item {} 
\end{itemize}
&&&&&&&\\
\sphinxbottomrule
\end{tabular}
\sphinxtableafterendhook\par
\sphinxattableend\end{savenotes}


\begin{savenotes}\sphinxattablestart
\sphinxthistablewithglobalstyle
\centering
\begin{tabular}[t]{\X{33}{99}\X{33}{99}\X{33}{99}}
\sphinxtoprule
\sphinxtableatstartofbodyhook
\sphinxAtStartPar
位域 |
&
\sphinxAtStartPar
名称     | |
&
\sphinxAtStartPar
描述                                        | |
\\
\sphinxhline
\sphinxAtStartPar
31:4
&\begin{itemize}
\item {} 
\end{itemize}
&\begin{itemize}
\item {} 
\end{itemize}
\\
\sphinxhline
\sphinxAtStartPar
3
&
\sphinxAtStartPar
CH1
&
\sphinxAtStartPar
CH1 M0总线一侧传输完成中断屏蔽              |

\sphinxAtStartPar
1:屏蔽                                     |

\sphinxAtStartPar
0:非屏蔽                                   |
\\
\sphinxhline
\sphinxAtStartPar
2
&
\sphinxAtStartPar
CH1HF
&
\sphinxAtStartPar
CH1 M0总线一侧传输完成1/2中断屏蔽           |

\sphinxAtStartPar
1:屏蔽                                     |

\sphinxAtStartPar
0:非屏蔽                                   |
\\
\sphinxhline
\sphinxAtStartPar
1
&
\sphinxAtStartPar
CH0
&
\sphinxAtStartPar
CH0 M0总线一侧传输完成中断屏蔽              |

\sphinxAtStartPar
1:屏蔽                                     |

\sphinxAtStartPar
0:非屏蔽                                   |
\\
\sphinxhline
\sphinxAtStartPar
0
&
\sphinxAtStartPar
CH0HF
&
\sphinxAtStartPar
CH0 M0总线一侧传输完成1/2中断屏蔽           |

\sphinxAtStartPar
1:屏蔽                                     |

\sphinxAtStartPar
0:非屏蔽                                   |
\\
\sphinxbottomrule
\end{tabular}
\sphinxtableafterendhook\par
\sphinxattableend\end{savenotes}


\subsubsection{M0总线一侧传输完成中断状态寄存器DSTSGIF}
\label{\detokenize{SWM241/_u529f_u80fd_u63cf_u8ff0/_u76f4_u63a5_u5185_u5b58_u5b58_u53d6:m0dstsgif}}

\begin{savenotes}\sphinxattablestart
\sphinxthistablewithglobalstyle
\centering
\begin{tabular}[t]{\X{20}{100}\X{20}{100}\X{20}{100}\X{20}{100}\X{20}{100}}
\sphinxtoprule
\sphinxtableatstartofbodyhook
\sphinxAtStartPar
寄存器 |
&
\begin{DUlineblock}{0em}
\item[] 偏移 |
\end{DUlineblock}
&
\begin{DUlineblock}{0em}
\item[] 
\item[] {\color{red}\bfseries{}|}
\end{DUlineblock}
&
\sphinxAtStartPar
复位值 |    描 | |
&
\begin{DUlineblock}{0em}
\item[] |
  |
\end{DUlineblock}
\\
\sphinxhline
\sphinxAtStartPar
DSTSGIF
&
\sphinxAtStartPar
0x18
&&
\sphinxAtStartPar
0 000000
&
\sphinxAtStartPar
scatter gathe 0总线一侧传输完成中断状态 |
\\
\sphinxbottomrule
\end{tabular}
\sphinxtableafterendhook\par
\sphinxattableend\end{savenotes}


\begin{savenotes}\sphinxattablestart
\sphinxthistablewithglobalstyle
\centering
\begin{tabular}[t]{\X{12}{96}\X{12}{96}\X{12}{96}\X{12}{96}\X{12}{96}\X{12}{96}\X{12}{96}\X{12}{96}}
\sphinxtoprule
\sphinxtableatstartofbodyhook
\sphinxAtStartPar
31
&
\sphinxAtStartPar
30
&
\sphinxAtStartPar
29
&
\sphinxAtStartPar
28
&
\sphinxAtStartPar
27
&
\sphinxAtStartPar
26
&
\sphinxAtStartPar
25
&
\sphinxAtStartPar
24
\\
\sphinxhline\begin{itemize}
\item {} 
\end{itemize}
&&&&&&&\\
\sphinxhline
\sphinxAtStartPar
23
&
\sphinxAtStartPar
22
&
\sphinxAtStartPar
21
&
\sphinxAtStartPar
20
&
\sphinxAtStartPar
19
&
\sphinxAtStartPar
18
&
\sphinxAtStartPar
17
&
\sphinxAtStartPar
16
\\
\sphinxhline\begin{itemize}
\item {} 
\end{itemize}
&&&&&&&\\
\sphinxhline
\sphinxAtStartPar
15
&
\sphinxAtStartPar
14
&
\sphinxAtStartPar
13
&
\sphinxAtStartPar
12
&
\sphinxAtStartPar
11
&
\sphinxAtStartPar
10
&
\sphinxAtStartPar
9
&
\sphinxAtStartPar
8
\\
\sphinxhline\begin{itemize}
\item {} 
\end{itemize}
&&&&&&&\\
\sphinxhline
\sphinxAtStartPar
7
&
\sphinxAtStartPar
6
&
\sphinxAtStartPar
5
&
\sphinxAtStartPar
4
&
\sphinxAtStartPar
3
&
\sphinxAtStartPar
2
&
\sphinxAtStartPar
1
&
\sphinxAtStartPar
0
\\
\sphinxhline\begin{itemize}
\item {} 
\end{itemize}
&&&&&&&\\
\sphinxbottomrule
\end{tabular}
\sphinxtableafterendhook\par
\sphinxattableend\end{savenotes}


\begin{savenotes}\sphinxattablestart
\sphinxthistablewithglobalstyle
\centering
\begin{tabular}[t]{\X{33}{99}\X{33}{99}\X{33}{99}}
\sphinxtoprule
\sphinxtableatstartofbodyhook
\sphinxAtStartPar
位域 |
&
\sphinxAtStartPar
名称     | |
&
\sphinxAtStartPar
描述                                        | |
\\
\sphinxhline
\sphinxAtStartPar
31:4
&\begin{itemize}
\item {} 
\end{itemize}
&\begin{itemize}
\item {} 
\end{itemize}
\\
\sphinxhline
\sphinxAtStartPar
3
&
\sphinxAtStartPar
CH1
&
\sphinxAtStartPar
CH1 M0总线一侧传输完成中断状态,R/W1C       |

\sphinxAtStartPar
1:中断已发生                               |

\sphinxAtStartPar
0:中断未发生                               |
\\
\sphinxhline
\sphinxAtStartPar
2
&
\sphinxAtStartPar
CH1HF
&
\sphinxAtStartPar
CH1 M0总线一侧传输完成1/2中断状态,R/W1C    |

\sphinxAtStartPar
1:中断已发生                               |

\sphinxAtStartPar
0:中断未发生                               |
\\
\sphinxhline
\sphinxAtStartPar
1
&
\sphinxAtStartPar
CH0
&
\sphinxAtStartPar
CH0 M1总线一侧传输完成中断状态,R/W1C       |

\sphinxAtStartPar
1:中断已发生                               |

\sphinxAtStartPar
0:中断未发生                               |
\\
\sphinxhline
\sphinxAtStartPar
0
&
\sphinxAtStartPar
CH0HF
&
\sphinxAtStartPar
CH0 M1总线一侧传输完成1/2中断状态,R/W1C    |

\sphinxAtStartPar
1:中断已发生                               |

\sphinxAtStartPar
0:中断未发生                               |
\\
\sphinxbottomrule
\end{tabular}
\sphinxtableafterendhook\par
\sphinxattableend\end{savenotes}


\subsubsection{M1总线一侧传输完成中断使能寄存器SRCSGIE}
\label{\detokenize{SWM241/_u529f_u80fd_u63cf_u8ff0/_u76f4_u63a5_u5185_u5b58_u5b58_u53d6:m1srcsgie}}

\begin{savenotes}\sphinxattablestart
\sphinxthistablewithglobalstyle
\centering
\begin{tabular}[t]{\X{20}{100}\X{20}{100}\X{20}{100}\X{20}{100}\X{20}{100}}
\sphinxtoprule
\sphinxtableatstartofbodyhook
\sphinxAtStartPar
寄存器 |
&
\begin{DUlineblock}{0em}
\item[] 偏移 |
\end{DUlineblock}
&
\begin{DUlineblock}{0em}
\item[] 
\item[] {\color{red}\bfseries{}|}
\end{DUlineblock}
&
\sphinxAtStartPar
复位值 |    描 | |
&
\begin{DUlineblock}{0em}
\item[] |
  |
\end{DUlineblock}
\\
\sphinxhline
\sphinxAtStartPar
SRCSGIE
&
\sphinxAtStartPar
0x1C
&&
\sphinxAtStartPar
0 000000
&
\sphinxAtStartPar
scatter gather M1总线一侧传输完成中断使能 |
\\
\sphinxbottomrule
\end{tabular}
\sphinxtableafterendhook\par
\sphinxattableend\end{savenotes}


\begin{savenotes}\sphinxattablestart
\sphinxthistablewithglobalstyle
\centering
\begin{tabular}[t]{\X{12}{96}\X{12}{96}\X{12}{96}\X{12}{96}\X{12}{96}\X{12}{96}\X{12}{96}\X{12}{96}}
\sphinxtoprule
\sphinxtableatstartofbodyhook
\sphinxAtStartPar
31
&
\sphinxAtStartPar
30
&
\sphinxAtStartPar
29
&
\sphinxAtStartPar
28
&
\sphinxAtStartPar
27
&
\sphinxAtStartPar
26
&
\sphinxAtStartPar
25
&
\sphinxAtStartPar
24
\\
\sphinxhline\begin{itemize}
\item {} 
\end{itemize}
&&&&&&&\\
\sphinxhline
\sphinxAtStartPar
23
&
\sphinxAtStartPar
22
&
\sphinxAtStartPar
21
&
\sphinxAtStartPar
20
&
\sphinxAtStartPar
19
&
\sphinxAtStartPar
18
&
\sphinxAtStartPar
17
&
\sphinxAtStartPar
16
\\
\sphinxhline\begin{itemize}
\item {} 
\end{itemize}
&&&&&&&\\
\sphinxhline
\sphinxAtStartPar
15
&
\sphinxAtStartPar
14
&
\sphinxAtStartPar
13
&
\sphinxAtStartPar
12
&
\sphinxAtStartPar
11
&
\sphinxAtStartPar
10
&
\sphinxAtStartPar
9
&
\sphinxAtStartPar
8
\\
\sphinxhline\begin{itemize}
\item {} 
\end{itemize}
&&&&&&&\\
\sphinxhline
\sphinxAtStartPar
7
&
\sphinxAtStartPar
6
&
\sphinxAtStartPar
5
&
\sphinxAtStartPar
4
&
\sphinxAtStartPar
3
&
\sphinxAtStartPar
2
&
\sphinxAtStartPar
1
&
\sphinxAtStartPar
0
\\
\sphinxhline\begin{itemize}
\item {} 
\end{itemize}
&&&&&&&\\
\sphinxbottomrule
\end{tabular}
\sphinxtableafterendhook\par
\sphinxattableend\end{savenotes}


\begin{savenotes}\sphinxattablestart
\sphinxthistablewithglobalstyle
\centering
\begin{tabular}[t]{\X{33}{99}\X{33}{99}\X{33}{99}}
\sphinxtoprule
\sphinxtableatstartofbodyhook
\sphinxAtStartPar
位域 |
&
\sphinxAtStartPar
名称     | |
&
\sphinxAtStartPar
描述                                        | |
\\
\sphinxhline
\sphinxAtStartPar
31:4
&\begin{itemize}
\item {} 
\end{itemize}
&\begin{itemize}
\item {} 
\end{itemize}
\\
\sphinxhline
\sphinxAtStartPar
3
&
\sphinxAtStartPar
CH1
&
\sphinxAtStartPar
CH1 M1总线一侧传输完成中断使能              |

\sphinxAtStartPar
1:使能                                     |

\sphinxAtStartPar
0:禁能                                     |
\\
\sphinxhline
\sphinxAtStartPar
2
&
\sphinxAtStartPar
CH1HF
&
\sphinxAtStartPar
CH1 M1总线一侧传输完成1/2中断使能           |

\sphinxAtStartPar
1:使能                                     |

\sphinxAtStartPar
0:禁能                                     |
\\
\sphinxhline
\sphinxAtStartPar
1
&
\sphinxAtStartPar
CH0
&
\sphinxAtStartPar
CH0 M1总线一侧传输完成中断使能              |

\sphinxAtStartPar
1:使能                                     |

\sphinxAtStartPar
0:禁能                                     |
\\
\sphinxhline
\sphinxAtStartPar
0
&
\sphinxAtStartPar
CH0HF
&
\sphinxAtStartPar
CH0 M1总线一侧传输完成1/2中断使能           |

\sphinxAtStartPar
1:使能                                     |

\sphinxAtStartPar
0:禁能                                     |
\\
\sphinxbottomrule
\end{tabular}
\sphinxtableafterendhook\par
\sphinxattableend\end{savenotes}


\subsubsection{M1总线一侧传输完成中断屏蔽寄存器SRCSGIM}
\label{\detokenize{SWM241/_u529f_u80fd_u63cf_u8ff0/_u76f4_u63a5_u5185_u5b58_u5b58_u53d6:m1srcsgim}}

\begin{savenotes}\sphinxattablestart
\sphinxthistablewithglobalstyle
\centering
\begin{tabular}[t]{\X{20}{100}\X{20}{100}\X{20}{100}\X{20}{100}\X{20}{100}}
\sphinxtoprule
\sphinxtableatstartofbodyhook
\sphinxAtStartPar
寄存器 |
&
\begin{DUlineblock}{0em}
\item[] 偏移 |
\end{DUlineblock}
&
\begin{DUlineblock}{0em}
\item[] 
\item[] {\color{red}\bfseries{}|}
\end{DUlineblock}
&
\sphinxAtStartPar
复位值 |    描 | |
&
\begin{DUlineblock}{0em}
\item[] |
  |
\end{DUlineblock}
\\
\sphinxhline
\sphinxAtStartPar
SRCSGIM
&
\sphinxAtStartPar
0x20
&&
\sphinxAtStartPar
0 000000
&
\sphinxAtStartPar
scatter gather M1总线一侧传输完成中断屏蔽 |
\\
\sphinxbottomrule
\end{tabular}
\sphinxtableafterendhook\par
\sphinxattableend\end{savenotes}


\begin{savenotes}\sphinxattablestart
\sphinxthistablewithglobalstyle
\centering
\begin{tabular}[t]{\X{12}{96}\X{12}{96}\X{12}{96}\X{12}{96}\X{12}{96}\X{12}{96}\X{12}{96}\X{12}{96}}
\sphinxtoprule
\sphinxtableatstartofbodyhook
\sphinxAtStartPar
31
&
\sphinxAtStartPar
30
&
\sphinxAtStartPar
29
&
\sphinxAtStartPar
28
&
\sphinxAtStartPar
27
&
\sphinxAtStartPar
26
&
\sphinxAtStartPar
25
&
\sphinxAtStartPar
24
\\
\sphinxhline\begin{itemize}
\item {} 
\end{itemize}
&&&&&&&\\
\sphinxhline
\sphinxAtStartPar
23
&
\sphinxAtStartPar
22
&
\sphinxAtStartPar
21
&
\sphinxAtStartPar
20
&
\sphinxAtStartPar
19
&
\sphinxAtStartPar
18
&
\sphinxAtStartPar
17
&
\sphinxAtStartPar
16
\\
\sphinxhline\begin{itemize}
\item {} 
\end{itemize}
&&&&&&&\\
\sphinxhline
\sphinxAtStartPar
15
&
\sphinxAtStartPar
14
&
\sphinxAtStartPar
13
&
\sphinxAtStartPar
12
&
\sphinxAtStartPar
11
&
\sphinxAtStartPar
10
&
\sphinxAtStartPar
9
&
\sphinxAtStartPar
8
\\
\sphinxhline\begin{itemize}
\item {} 
\end{itemize}
&&&&&&&\\
\sphinxhline
\sphinxAtStartPar
7
&
\sphinxAtStartPar
6
&
\sphinxAtStartPar
5
&
\sphinxAtStartPar
4
&
\sphinxAtStartPar
3
&
\sphinxAtStartPar
2
&
\sphinxAtStartPar
1
&
\sphinxAtStartPar
0
\\
\sphinxhline\begin{itemize}
\item {} 
\end{itemize}
&&&&&&&\\
\sphinxbottomrule
\end{tabular}
\sphinxtableafterendhook\par
\sphinxattableend\end{savenotes}


\begin{savenotes}\sphinxattablestart
\sphinxthistablewithglobalstyle
\centering
\begin{tabular}[t]{\X{33}{99}\X{33}{99}\X{33}{99}}
\sphinxtoprule
\sphinxtableatstartofbodyhook
\sphinxAtStartPar
位域 |
&
\sphinxAtStartPar
名称     | |
&
\sphinxAtStartPar
描述                                        | |
\\
\sphinxhline
\sphinxAtStartPar
31:4
&\begin{itemize}
\item {} 
\end{itemize}
&\begin{itemize}
\item {} 
\end{itemize}
\\
\sphinxhline
\sphinxAtStartPar
3
&
\sphinxAtStartPar
CH1
&
\sphinxAtStartPar
CH1 M1总线一侧传输完成中断屏蔽              |

\sphinxAtStartPar
1:屏蔽                                     |

\sphinxAtStartPar
0:非屏蔽                                   |
\\
\sphinxhline
\sphinxAtStartPar
2
&
\sphinxAtStartPar
CH1HF
&
\sphinxAtStartPar
CH1 M1总线一侧传输完成1/2中断屏蔽           |

\sphinxAtStartPar
1:屏蔽                                     |

\sphinxAtStartPar
0:非屏蔽                                   |
\\
\sphinxhline
\sphinxAtStartPar
1
&
\sphinxAtStartPar
CH0
&
\sphinxAtStartPar
CH0 M1总线一侧传输完成中断屏蔽              |

\sphinxAtStartPar
1:屏蔽                                     |

\sphinxAtStartPar
0:非屏蔽                                   |
\\
\sphinxhline
\sphinxAtStartPar
0
&
\sphinxAtStartPar
CH0HF
&
\sphinxAtStartPar
CH0 M1总线一侧传输完成1/2中断屏蔽           |

\sphinxAtStartPar
1:屏蔽                                     |

\sphinxAtStartPar
0:非屏蔽                                   |
\\
\sphinxbottomrule
\end{tabular}
\sphinxtableafterendhook\par
\sphinxattableend\end{savenotes}


\subsubsection{M1总线一侧传输完成中断状态寄存器SRCSGIF}
\label{\detokenize{SWM241/_u529f_u80fd_u63cf_u8ff0/_u76f4_u63a5_u5185_u5b58_u5b58_u53d6:m1srcsgif}}

\begin{savenotes}\sphinxattablestart
\sphinxthistablewithglobalstyle
\centering
\begin{tabular}[t]{\X{20}{100}\X{20}{100}\X{20}{100}\X{20}{100}\X{20}{100}}
\sphinxtoprule
\sphinxtableatstartofbodyhook
\sphinxAtStartPar
寄存器 |
&
\begin{DUlineblock}{0em}
\item[] 偏移 |
\end{DUlineblock}
&
\begin{DUlineblock}{0em}
\item[] 
\item[] {\color{red}\bfseries{}|}
\end{DUlineblock}
&
\sphinxAtStartPar
复位值 |    描 | |
&
\begin{DUlineblock}{0em}
\item[] |
  |
\end{DUlineblock}
\\
\sphinxhline
\sphinxAtStartPar
SRCSGIF
&
\sphinxAtStartPar
0x24
&&
\sphinxAtStartPar
0 000000
&
\sphinxAtStartPar
scatter gathe 1总线一侧传输完成中断状态 |
\\
\sphinxbottomrule
\end{tabular}
\sphinxtableafterendhook\par
\sphinxattableend\end{savenotes}


\begin{savenotes}\sphinxattablestart
\sphinxthistablewithglobalstyle
\centering
\begin{tabular}[t]{\X{12}{96}\X{12}{96}\X{12}{96}\X{12}{96}\X{12}{96}\X{12}{96}\X{12}{96}\X{12}{96}}
\sphinxtoprule
\sphinxtableatstartofbodyhook
\sphinxAtStartPar
31
&
\sphinxAtStartPar
30
&
\sphinxAtStartPar
29
&
\sphinxAtStartPar
28
&
\sphinxAtStartPar
27
&
\sphinxAtStartPar
26
&
\sphinxAtStartPar
25
&
\sphinxAtStartPar
24
\\
\sphinxhline\begin{itemize}
\item {} 
\end{itemize}
&&&&&&&\\
\sphinxhline
\sphinxAtStartPar
23
&
\sphinxAtStartPar
22
&
\sphinxAtStartPar
21
&
\sphinxAtStartPar
20
&
\sphinxAtStartPar
19
&
\sphinxAtStartPar
18
&
\sphinxAtStartPar
17
&
\sphinxAtStartPar
16
\\
\sphinxhline\begin{itemize}
\item {} 
\end{itemize}
&&&&&&&\\
\sphinxhline
\sphinxAtStartPar
15
&
\sphinxAtStartPar
14
&
\sphinxAtStartPar
13
&
\sphinxAtStartPar
12
&
\sphinxAtStartPar
11
&
\sphinxAtStartPar
10
&
\sphinxAtStartPar
9
&
\sphinxAtStartPar
8
\\
\sphinxhline\begin{itemize}
\item {} 
\end{itemize}
&&&&&&&\\
\sphinxhline
\sphinxAtStartPar
7
&
\sphinxAtStartPar
6
&
\sphinxAtStartPar
5
&
\sphinxAtStartPar
4
&
\sphinxAtStartPar
3
&
\sphinxAtStartPar
2
&
\sphinxAtStartPar
1
&
\sphinxAtStartPar
0
\\
\sphinxhline\begin{itemize}
\item {} 
\end{itemize}
&&&&&&&\\
\sphinxbottomrule
\end{tabular}
\sphinxtableafterendhook\par
\sphinxattableend\end{savenotes}


\begin{savenotes}\sphinxattablestart
\sphinxthistablewithglobalstyle
\centering
\begin{tabular}[t]{\X{33}{99}\X{33}{99}\X{33}{99}}
\sphinxtoprule
\sphinxtableatstartofbodyhook
\sphinxAtStartPar
位域 |
&
\sphinxAtStartPar
名称     | |
&
\sphinxAtStartPar
描述                                        | |
\\
\sphinxhline
\sphinxAtStartPar
31:4
&\begin{itemize}
\item {} 
\end{itemize}
&\begin{itemize}
\item {} 
\end{itemize}
\\
\sphinxhline
\sphinxAtStartPar
3
&
\sphinxAtStartPar
CH1
&
\sphinxAtStartPar
CH1 M1总线一侧传输完成中断状态,R/W1C       |

\sphinxAtStartPar
1:中断已发生                               |

\sphinxAtStartPar
0:中断未发生                               |
\\
\sphinxhline
\sphinxAtStartPar
2
&
\sphinxAtStartPar
CH1HF
&
\sphinxAtStartPar
CH1 M1总线一侧传输完成1/2中断状态,R/W1C    |

\sphinxAtStartPar
1:中断已发生                               |

\sphinxAtStartPar
0:中断未发生                               |
\\
\sphinxhline
\sphinxAtStartPar
1
&
\sphinxAtStartPar
CH0
&
\sphinxAtStartPar
CH0 M1总线一侧传输完成中断状态,R/W1C       |

\sphinxAtStartPar
1:中断已发生                               |

\sphinxAtStartPar
0:中断未发生                               |
\\
\sphinxhline
\sphinxAtStartPar
0
&
\sphinxAtStartPar
CH0HF
&
\sphinxAtStartPar
CH0 M1总线一侧传输完成1/2中断状态,R/W1C    |

\sphinxAtStartPar
1:中断已发生                               |

\sphinxAtStartPar
0:中断未发生                               |
\\
\sphinxbottomrule
\end{tabular}
\sphinxtableafterendhook\par
\sphinxattableend\end{savenotes}


\subsubsection{通道优先设定寄存器PRI}
\label{\detokenize{SWM241/_u529f_u80fd_u63cf_u8ff0/_u76f4_u63a5_u5185_u5b58_u5b58_u53d6:pri}}

\begin{savenotes}\sphinxattablestart
\sphinxthistablewithglobalstyle
\centering
\begin{tabular}[t]{\X{20}{100}\X{20}{100}\X{20}{100}\X{20}{100}\X{20}{100}}
\sphinxtoprule
\sphinxtableatstartofbodyhook
\sphinxAtStartPar
寄存器 |
&
\begin{DUlineblock}{0em}
\item[] 偏移 |
\end{DUlineblock}
&
\begin{DUlineblock}{0em}
\item[] 
\item[] {\color{red}\bfseries{}|}
\end{DUlineblock}
&
\sphinxAtStartPar
复位值 |    描 | |
&
\begin{DUlineblock}{0em}
\item[] |
  |
\end{DUlineblock}
\\
\sphinxhline
\sphinxAtStartPar
PRI
&
\sphinxAtStartPar
0x3C
&&
\sphinxAtStartPar
0 000000
&
\sphinxAtStartPar
通道优先级设定             |
\\
\sphinxbottomrule
\end{tabular}
\sphinxtableafterendhook\par
\sphinxattableend\end{savenotes}


\begin{savenotes}\sphinxattablestart
\sphinxthistablewithglobalstyle
\centering
\begin{tabular}[t]{\X{12}{96}\X{12}{96}\X{12}{96}\X{12}{96}\X{12}{96}\X{12}{96}\X{12}{96}\X{12}{96}}
\sphinxtoprule
\sphinxtableatstartofbodyhook
\sphinxAtStartPar
31
&
\sphinxAtStartPar
30
&
\sphinxAtStartPar
29
&
\sphinxAtStartPar
28
&
\sphinxAtStartPar
27
&
\sphinxAtStartPar
26
&
\sphinxAtStartPar
25
&
\sphinxAtStartPar
24
\\
\sphinxhline\begin{itemize}
\item {} 
\end{itemize}
&&&&&&&\\
\sphinxhline
\sphinxAtStartPar
23
&
\sphinxAtStartPar
22
&
\sphinxAtStartPar
21
&
\sphinxAtStartPar
20
&
\sphinxAtStartPar
19
&
\sphinxAtStartPar
18
&
\sphinxAtStartPar
17
&
\sphinxAtStartPar
16
\\
\sphinxhline\begin{itemize}
\item {} 
\end{itemize}
&&&&&&&\\
\sphinxhline
\sphinxAtStartPar
15
&
\sphinxAtStartPar
14
&
\sphinxAtStartPar
13
&
\sphinxAtStartPar
12
&
\sphinxAtStartPar
11
&
\sphinxAtStartPar
10
&
\sphinxAtStartPar
9
&
\sphinxAtStartPar
8
\\
\sphinxhline\begin{itemize}
\item {} 
\end{itemize}
&&&&&&&\\
\sphinxhline
\sphinxAtStartPar
7
&
\sphinxAtStartPar
6
&
\sphinxAtStartPar
5
&
\sphinxAtStartPar
4
&
\sphinxAtStartPar
3
&
\sphinxAtStartPar
2
&
\sphinxAtStartPar
1
&
\sphinxAtStartPar
0
\\
\sphinxhline\begin{itemize}
\item {} 
\end{itemize}
&&&&&&&
\sphinxAtStartPar
CH0
\\
\sphinxbottomrule
\end{tabular}
\sphinxtableafterendhook\par
\sphinxattableend\end{savenotes}


\begin{savenotes}\sphinxattablestart
\sphinxthistablewithglobalstyle
\centering
\begin{tabular}[t]{\X{33}{99}\X{33}{99}\X{33}{99}}
\sphinxtoprule
\sphinxtableatstartofbodyhook
\sphinxAtStartPar
位域 |
&
\sphinxAtStartPar
名称     | |
&
\sphinxAtStartPar
描述                                        | |
\\
\sphinxhline
\sphinxAtStartPar
31:2
&\begin{itemize}
\item {} 
\end{itemize}
&\begin{itemize}
\item {} 
\end{itemize}
\\
\sphinxhline
\sphinxAtStartPar
1
&
\sphinxAtStartPar
CH1
&
\sphinxAtStartPar
DMA CH1优先级                               |

\sphinxAtStartPar
1:该通道为高优先级                         |

\sphinxAtStartPar
0:该通道为低优先级。                       |
\\
\sphinxhline
\sphinxAtStartPar
0
&
\sphinxAtStartPar
CH0
&
\sphinxAtStartPar
DMA CH0优先级                               |

\sphinxAtStartPar
1:该通道为高优先级                         |

\sphinxAtStartPar
0:该通道为低优先级。                       |
\\
\sphinxbottomrule
\end{tabular}
\sphinxtableafterendhook\par
\sphinxattableend\end{savenotes}


\subsubsection{通道n控制寄存器CRn}
\label{\detokenize{SWM241/_u529f_u80fd_u63cf_u8ff0/_u76f4_u63a5_u5185_u5b58_u5b58_u53d6:ncrn}}

\begin{savenotes}\sphinxattablestart
\sphinxthistablewithglobalstyle
\centering
\begin{tabular}[t]{\X{20}{100}\X{20}{100}\X{20}{100}\X{20}{100}\X{20}{100}}
\sphinxtoprule
\sphinxtableatstartofbodyhook
\sphinxAtStartPar
寄存器 |
&
\begin{DUlineblock}{0em}
\item[] 偏移 |
\end{DUlineblock}
&
\begin{DUlineblock}{0em}
\item[] 
\item[] {\color{red}\bfseries{}|}
\end{DUlineblock}
&
\sphinxAtStartPar
复位值 |    描 | |
&
\begin{DUlineblock}{0em}
\item[] |
  |
\end{DUlineblock}
\\
\sphinxhline
\sphinxAtStartPar
CRn
&
\sphinxAtStartPar
0 x40*n +  0x00
&&
\sphinxAtStartPar
0 000000
&
\sphinxAtStartPar
通道控制                   |
\\
\sphinxbottomrule
\end{tabular}
\sphinxtableafterendhook\par
\sphinxattableend\end{savenotes}


\begin{savenotes}\sphinxattablestart
\sphinxthistablewithglobalstyle
\centering
\begin{tabular}[t]{\X{12}{96}\X{12}{96}\X{12}{96}\X{12}{96}\X{12}{96}\X{12}{96}\X{12}{96}\X{12}{96}}
\sphinxtoprule
\sphinxtableatstartofbodyhook
\sphinxAtStartPar
31
&
\sphinxAtStartPar
30
&
\sphinxAtStartPar
29
&
\sphinxAtStartPar
28
&
\sphinxAtStartPar
27
&
\sphinxAtStartPar
26
&
\sphinxAtStartPar
25
&
\sphinxAtStartPar
24
\\
\sphinxhline\begin{itemize}
\item {} 
\end{itemize}
&&&&&&&\\
\sphinxhline
\sphinxAtStartPar
23
&
\sphinxAtStartPar
22
&
\sphinxAtStartPar
21
&
\sphinxAtStartPar
20
&
\sphinxAtStartPar
19
&
\sphinxAtStartPar
18
&
\sphinxAtStartPar
17
&
\sphinxAtStartPar
16
\\
\sphinxhline\begin{itemize}
\item {} 
\end{itemize}
&&&&
\sphinxAtStartPar
LEN
&&&\\
\sphinxhline
\sphinxAtStartPar
15
&
\sphinxAtStartPar
14
&
\sphinxAtStartPar
13
&
\sphinxAtStartPar
12
&
\sphinxAtStartPar
11
&
\sphinxAtStartPar
10
&
\sphinxAtStartPar
9
&
\sphinxAtStartPar
8
\\
\sphinxhline
\sphinxAtStartPar
LEN
&&&&&&&\\
\sphinxhline
\sphinxAtStartPar
7
&
\sphinxAtStartPar
6
&
\sphinxAtStartPar
5
&
\sphinxAtStartPar
4
&
\sphinxAtStartPar
3
&
\sphinxAtStartPar
2
&
\sphinxAtStartPar
1
&
\sphinxAtStartPar
0
\\
\sphinxhline
\sphinxAtStartPar
\sphinxstylestrong{LEN}
&&&&&&&\\
\sphinxbottomrule
\end{tabular}
\sphinxtableafterendhook\par
\sphinxattableend\end{savenotes}


\begin{savenotes}\sphinxattablestart
\sphinxthistablewithglobalstyle
\centering
\begin{tabular}[t]{\X{33}{99}\X{33}{99}\X{33}{99}}
\sphinxtoprule
\sphinxtableatstartofbodyhook
\sphinxAtStartPar
位域 |
&
\sphinxAtStartPar
名称     | |
&
\sphinxAtStartPar
描述                                        | |
\\
\sphinxhline
\sphinxAtStartPar
31:28
&\begin{itemize}
\item {} 
\end{itemize}
&\begin{itemize}
\item {} 
\end{itemize}
\\
\sphinxhline
\sphinxAtStartPar
27
&
\sphinxAtStartPar
SETPOP
&
\sphinxAtStartPar
步进传输。在C                               | RIEN为1时,每触发一次,传送一个单位的数据。 |

\sphinxAtStartPar
注意:                                      |

\sphinxAtStartPar
步进模式将屏蔽OPBURST配置,即在             | URST配置为INCR4的情况下,仍然按SINGLE传输。 |

\sphinxAtStartPar
步进模式不支持M0                            | 配置不同的OPWIDTH,因为OPWIDTH不同时,两组 | 要的TRIG次数不一样,会造成数据传输出错。 |

\sphinxAtStartPar
模式下,当传输目标地址为固定地址模式,且O | DTH为BYTE或HALFWORD时,每次传输的有效数据在 | 置于低位。(因为无法通过地址识别有效字节) |
\\
\sphinxhline
\sphinxAtStartPar
26
&
\sphinxAtStartPar
AUTORE
&
\sphinxAtStartPar
Auto Restart, 通道在传输完成后,是否自动重新启动          |

\sphinxAtStartPar
0:传输完成后停止                           |

\sphinxAtStartPar
1 完成后自动按照上一次的配置重新启动传输。 |
\\
\sphinxhline
\sphinxAtStartPar
25
&
\sphinxAtStartPar
TXEN
&
\sphinxAtStartPar
TX软件启动传输,传输方向为SRCDST            |
\\
\sphinxhline
\sphinxAtStartPar
24
&
\sphinxAtStartPar
RXEN
&
\sphinxAtStartPar
RX软件启动传输,传输方向为DSTSRC            |
\\
\sphinxhline
\sphinxAtStartPar
23:20
&\begin{itemize}
\item {} 
\end{itemize}
&\begin{itemize}
\item {} 
\end{itemize}
\\
\sphinxhline
\sphinxAtStartPar
19:0
&
\sphinxAtStartPar
LEN
&
\sphinxAtStartPar
DMA传输单元数量                             |

\sphinxAtStartPar
0对应1个单位长度                            |
\\
\sphinxbottomrule
\end{tabular}
\sphinxtableafterendhook\par
\sphinxattableend\end{savenotes}


\subsubsection{通道n地址模式寄存器AMn}
\label{\detokenize{SWM241/_u529f_u80fd_u63cf_u8ff0/_u76f4_u63a5_u5185_u5b58_u5b58_u53d6:namn}}

\begin{savenotes}\sphinxattablestart
\sphinxthistablewithglobalstyle
\centering
\begin{tabular}[t]{\X{20}{100}\X{20}{100}\X{20}{100}\X{20}{100}\X{20}{100}}
\sphinxtoprule
\sphinxtableatstartofbodyhook
\sphinxAtStartPar
寄存器 |
&
\begin{DUlineblock}{0em}
\item[] 偏移 |
\end{DUlineblock}
&
\begin{DUlineblock}{0em}
\item[] 
\item[] {\color{red}\bfseries{}|}
\end{DUlineblock}
&
\sphinxAtStartPar
复位值 |    描 | |
&
\begin{DUlineblock}{0em}
\item[] |
  |
\end{DUlineblock}
\\
\sphinxhline
\sphinxAtStartPar
AMn
&
\sphinxAtStartPar
0 x40*n +  0x04
&&
\sphinxAtStartPar
0 000000
&
\sphinxAtStartPar
通道地址模式               |
\\
\sphinxbottomrule
\end{tabular}
\sphinxtableafterendhook\par
\sphinxattableend\end{savenotes}


\begin{savenotes}\sphinxattablestart
\sphinxthistablewithglobalstyle
\centering
\begin{tabular}[t]{\X{12}{96}\X{12}{96}\X{12}{96}\X{12}{96}\X{12}{96}\X{12}{96}\X{12}{96}\X{12}{96}}
\sphinxtoprule
\sphinxtableatstartofbodyhook
\sphinxAtStartPar
31
&
\sphinxAtStartPar
30
&
\sphinxAtStartPar
29
&
\sphinxAtStartPar
28
&
\sphinxAtStartPar
27
&
\sphinxAtStartPar
26
&
\sphinxAtStartPar
25
&
\sphinxAtStartPar
24
\\
\sphinxhline\begin{itemize}
\item {} 
\end{itemize}
&&&&&&&\\
\sphinxhline
\sphinxAtStartPar
23
&
\sphinxAtStartPar
22
&
\sphinxAtStartPar
21
&
\sphinxAtStartPar
20
&
\sphinxAtStartPar
19
&
\sphinxAtStartPar
18
&
\sphinxAtStartPar
17
&
\sphinxAtStartPar
16
\\
\sphinxhline\begin{itemize}
\item {} 
\end{itemize}
&&&&&&&\\
\sphinxhline
\sphinxAtStartPar
15
&
\sphinxAtStartPar
14
&
\sphinxAtStartPar
13
&
\sphinxAtStartPar
12
&
\sphinxAtStartPar
11
&
\sphinxAtStartPar
10
&
\sphinxAtStartPar
9
&
\sphinxAtStartPar
8
\\
\sphinxhline\begin{itemize}
\item {} 
\end{itemize}
&&&
\sphinxAtStartPar
SR RST
&
\sphinxAtStartPar
S IT
&&&\\
\sphinxhline
\sphinxAtStartPar
7
&
\sphinxAtStartPar
6
&
\sphinxAtStartPar
5
&
\sphinxAtStartPar
4
&
\sphinxAtStartPar
3
&
\sphinxAtStartPar
2
&
\sphinxAtStartPar
1
&
\sphinxAtStartPar
0
\\
\sphinxhline\begin{itemize}
\item {} 
\end{itemize}
&&&
\sphinxAtStartPar
DS RST
&
\sphinxAtStartPar
D IT
&&&\\
\sphinxbottomrule
\end{tabular}
\sphinxtableafterendhook\par
\sphinxattableend\end{savenotes}


\begin{savenotes}\sphinxattablestart
\sphinxthistablewithglobalstyle
\centering
\begin{tabular}[t]{\X{33}{99}\X{33}{99}\X{33}{99}}
\sphinxtoprule
\sphinxtableatstartofbodyhook
\sphinxAtStartPar
位域 |
&
\sphinxAtStartPar
名称     | |
&
\sphinxAtStartPar
描述                                        | |
\\
\sphinxhline
\sphinxAtStartPar
31:13
&\begin{itemize}
\item {} 
\end{itemize}
&\begin{itemize}
\item {} 
\end{itemize}
\\
\sphinxhline
\sphinxAtStartPar
12
&
\sphinxAtStartPar
SRCBURST
&
\sphinxAtStartPar
源地址传输模式                              |

\sphinxAtStartPar
0: Single                                  |

\sphinxAtStartPar
1: Burst(Inc4)                           |
\\
\sphinxhline
\sphinxAtStartPar
11:10
&
\sphinxAtStartPar
SRCBIT
&
\sphinxAtStartPar
源地址传输宽度                              |

\sphinxAtStartPar
00:字节                                    |

\sphinxAtStartPar
01:半字                                    |

\sphinxAtStartPar
10:字                                      |

\sphinxAtStartPar
11:保留                                    |
\\
\sphinxhline
\sphinxAtStartPar
9:8
&
\sphinxAtStartPar
SRCAM
&
\sphinxAtStartPar
源地址模式                                  |

\sphinxAtStartPar
00:地址固定                                |

\sphinxAtStartPar
01:地址递增                                |

\sphinxAtStartPar
10:scatter gather                          |

\sphinxAtStartPar
11:保留                                    |
\\
\sphinxhline
\sphinxAtStartPar
7:5
&\begin{itemize}
\item {} 
\end{itemize}
&\begin{itemize}
\item {} 
\end{itemize}
\\
\sphinxhline
\sphinxAtStartPar
4
&
\sphinxAtStartPar
DSTBURST
&
\sphinxAtStartPar
目的地址传输类型                            |

\sphinxAtStartPar
0:Single                                   |

\sphinxAtStartPar
1:Burst(Inc4)                            |
\\
\sphinxhline
\sphinxAtStartPar
3:2
&
\sphinxAtStartPar
DSTBIT
&
\sphinxAtStartPar
目的地址传输宽度                            |

\sphinxAtStartPar
00:字节                                    |

\sphinxAtStartPar
01:半字                                    |

\sphinxAtStartPar
10:字                                      |

\sphinxAtStartPar
11:保留                                    |
\\
\sphinxhline
\sphinxAtStartPar
1:0
&
\sphinxAtStartPar
DSTAM
&
\sphinxAtStartPar
目的地址模式                                |

\sphinxAtStartPar
00:地址固定                                |

\sphinxAtStartPar
01:地址递增                                |

\sphinxAtStartPar
10:scatter gather                          |

\sphinxAtStartPar
11:保留                                    |
\\
\sphinxbottomrule
\end{tabular}
\sphinxtableafterendhook\par
\sphinxattableend\end{savenotes}


\subsubsection{通道n目的地址寄存器DSTn}
\label{\detokenize{SWM241/_u529f_u80fd_u63cf_u8ff0/_u76f4_u63a5_u5185_u5b58_u5b58_u53d6:ndstn}}

\begin{savenotes}\sphinxattablestart
\sphinxthistablewithglobalstyle
\centering
\begin{tabular}[t]{\X{20}{100}\X{20}{100}\X{20}{100}\X{20}{100}\X{20}{100}}
\sphinxtoprule
\sphinxtableatstartofbodyhook
\sphinxAtStartPar
寄存器 |
&
\begin{DUlineblock}{0em}
\item[] 偏移 |
\end{DUlineblock}
&
\begin{DUlineblock}{0em}
\item[] 
\item[] {\color{red}\bfseries{}|}
\end{DUlineblock}
&
\sphinxAtStartPar
复位值 |    描 | |
&
\begin{DUlineblock}{0em}
\item[] |
  |
\end{DUlineblock}
\\
\sphinxhline
\sphinxAtStartPar
DSTn
&
\sphinxAtStartPar
0 x40*n +  0x08
&&
\sphinxAtStartPar
0 000000
&
\sphinxAtStartPar
通道n目的地址寄存器        |
\\
\sphinxbottomrule
\end{tabular}
\sphinxtableafterendhook\par
\sphinxattableend\end{savenotes}


\begin{savenotes}\sphinxattablestart
\sphinxthistablewithglobalstyle
\centering
\begin{tabular}[t]{\X{12}{96}\X{12}{96}\X{12}{96}\X{12}{96}\X{12}{96}\X{12}{96}\X{12}{96}\X{12}{96}}
\sphinxtoprule
\sphinxtableatstartofbodyhook
\sphinxAtStartPar
31
&
\sphinxAtStartPar
30
&
\sphinxAtStartPar
29
&
\sphinxAtStartPar
28
&
\sphinxAtStartPar
27
&
\sphinxAtStartPar
26
&
\sphinxAtStartPar
25
&
\sphinxAtStartPar
24
\\
\sphinxhline
\sphinxAtStartPar
DST
&&&&&&&\\
\sphinxhline
\sphinxAtStartPar
23
&
\sphinxAtStartPar
22
&
\sphinxAtStartPar
21
&
\sphinxAtStartPar
20
&
\sphinxAtStartPar
19
&
\sphinxAtStartPar
18
&
\sphinxAtStartPar
17
&
\sphinxAtStartPar
16
\\
\sphinxhline
\sphinxAtStartPar
DST
&&&&&&&\\
\sphinxhline
\sphinxAtStartPar
15
&
\sphinxAtStartPar
14
&
\sphinxAtStartPar
13
&
\sphinxAtStartPar
12
&
\sphinxAtStartPar
11
&
\sphinxAtStartPar
10
&
\sphinxAtStartPar
9
&
\sphinxAtStartPar
8
\\
\sphinxhline
\sphinxAtStartPar
DST
&&&&&&&\\
\sphinxhline
\sphinxAtStartPar
7
&
\sphinxAtStartPar
6
&
\sphinxAtStartPar
5
&
\sphinxAtStartPar
4
&
\sphinxAtStartPar
3
&
\sphinxAtStartPar
2
&
\sphinxAtStartPar
1
&
\sphinxAtStartPar
0
\\
\sphinxhline
\sphinxAtStartPar
DST
&&&&&&&\\
\sphinxbottomrule
\end{tabular}
\sphinxtableafterendhook\par
\sphinxattableend\end{savenotes}


\begin{savenotes}\sphinxattablestart
\sphinxthistablewithglobalstyle
\centering
\begin{tabular}[t]{\X{33}{99}\X{33}{99}\X{33}{99}}
\sphinxtoprule
\sphinxtableatstartofbodyhook
\sphinxAtStartPar
位域 |
&
\sphinxAtStartPar
名称     | |
&
\sphinxAtStartPar
描述                                        | |
\\
\sphinxhline
\sphinxAtStartPar
31:0
&
\sphinxAtStartPar
DST
&
\sphinxAtStartPar
目的地址                                    |
\\
\sphinxbottomrule
\end{tabular}
\sphinxtableafterendhook\par
\sphinxattableend\end{savenotes}


\subsubsection{通道n目的分散收集地址1寄存器DSTSGADDRn1}
\label{\detokenize{SWM241/_u529f_u80fd_u63cf_u8ff0/_u76f4_u63a5_u5185_u5b58_u5b58_u53d6:n1dstsgaddrn1}}

\begin{savenotes}\sphinxattablestart
\sphinxthistablewithglobalstyle
\centering
\begin{tabular}[t]{\X{20}{100}\X{20}{100}\X{20}{100}\X{20}{100}\X{20}{100}}
\sphinxtoprule
\sphinxtableatstartofbodyhook
\sphinxAtStartPar
寄存器 |
&
\begin{DUlineblock}{0em}
\item[] 偏移 |
\end{DUlineblock}
&
\begin{DUlineblock}{0em}
\item[] 
\item[] {\color{red}\bfseries{}|}
\end{DUlineblock}
&
\sphinxAtStartPar
复位值 |    描 | |
&
\begin{DUlineblock}{0em}
\item[] |
  |
\end{DUlineblock}
\\
\sphinxhline
\sphinxAtStartPar
DSTSGADDRn1
&
\sphinxAtStartPar
0 x40*n +  0xC
&&
\sphinxAtStartPar
0 000000
&
\sphinxAtStartPar
Destination Scatter Gather

\sphinxAtStartPar
ress通道n目的分散收集地址1 |
\\
\sphinxbottomrule
\end{tabular}
\sphinxtableafterendhook\par
\sphinxattableend\end{savenotes}


\begin{savenotes}\sphinxattablestart
\sphinxthistablewithglobalstyle
\centering
\begin{tabular}[t]{\X{12}{96}\X{12}{96}\X{12}{96}\X{12}{96}\X{12}{96}\X{12}{96}\X{12}{96}\X{12}{96}}
\sphinxtoprule
\sphinxtableatstartofbodyhook
\sphinxAtStartPar
31
&
\sphinxAtStartPar
30
&
\sphinxAtStartPar
29
&
\sphinxAtStartPar
28
&
\sphinxAtStartPar
27
&
\sphinxAtStartPar
26
&
\sphinxAtStartPar
25
&
\sphinxAtStartPar
24
\\
\sphinxhline
\sphinxAtStartPar
DSTSGADDRn1
&&&&&&&\\
\sphinxhline
\sphinxAtStartPar
23
&
\sphinxAtStartPar
22
&
\sphinxAtStartPar
21
&
\sphinxAtStartPar
20
&
\sphinxAtStartPar
19
&
\sphinxAtStartPar
18
&
\sphinxAtStartPar
17
&
\sphinxAtStartPar
16
\\
\sphinxhline
\sphinxAtStartPar
DSTSGADDRn1
&&&&&&&\\
\sphinxhline
\sphinxAtStartPar
15
&
\sphinxAtStartPar
14
&
\sphinxAtStartPar
13
&
\sphinxAtStartPar
12
&
\sphinxAtStartPar
11
&
\sphinxAtStartPar
10
&
\sphinxAtStartPar
9
&
\sphinxAtStartPar
8
\\
\sphinxhline
\sphinxAtStartPar
DSTSGADDRn1
&&&&&&&\\
\sphinxhline
\sphinxAtStartPar
7
&
\sphinxAtStartPar
6
&
\sphinxAtStartPar
5
&
\sphinxAtStartPar
4
&
\sphinxAtStartPar
3
&
\sphinxAtStartPar
2
&
\sphinxAtStartPar
1
&
\sphinxAtStartPar
0
\\
\sphinxhline
\sphinxAtStartPar
DSTSGADDRn1
&&&&&&&\\
\sphinxbottomrule
\end{tabular}
\sphinxtableafterendhook\par
\sphinxattableend\end{savenotes}


\begin{savenotes}\sphinxattablestart
\sphinxthistablewithglobalstyle
\centering
\begin{tabular}[t]{\X{33}{99}\X{33}{99}\X{33}{99}}
\sphinxtoprule
\sphinxtableatstartofbodyhook
\sphinxAtStartPar
位域 |
&
\sphinxAtStartPar
名称     | |
&
\sphinxAtStartPar
描述                                        | |
\\
\sphinxhline
\sphinxAtStartPar
31:0
&
\sphinxAtStartPar
DSTSGADDRn1
&
\sphinxAtStartPar
Destination Scatter Gather Address通道n目的分散收集地址1               |
\\
\sphinxbottomrule
\end{tabular}
\sphinxtableafterendhook\par
\sphinxattableend\end{savenotes}


\subsubsection{通道n目的分散收集地址2寄存器DSTSGADDRn2}
\label{\detokenize{SWM241/_u529f_u80fd_u63cf_u8ff0/_u76f4_u63a5_u5185_u5b58_u5b58_u53d6:n2dstsgaddrn2}}

\begin{savenotes}\sphinxattablestart
\sphinxthistablewithglobalstyle
\centering
\begin{tabular}[t]{\X{20}{100}\X{20}{100}\X{20}{100}\X{20}{100}\X{20}{100}}
\sphinxtoprule
\sphinxtableatstartofbodyhook
\sphinxAtStartPar
寄存器 |
&
\begin{DUlineblock}{0em}
\item[] 偏移 |
\end{DUlineblock}
&
\begin{DUlineblock}{0em}
\item[] 
\item[] {\color{red}\bfseries{}|}
\end{DUlineblock}
&
\sphinxAtStartPar
复位值 |    描 | |
&
\begin{DUlineblock}{0em}
\item[] |
  |
\end{DUlineblock}
\\
\sphinxhline
\sphinxAtStartPar
DSTSGADDRn2
&
\sphinxAtStartPar
0 x40*n +  0x10
&&
\sphinxAtStartPar
0 000000
&
\sphinxAtStartPar
Destination Scatter Gather

\sphinxAtStartPar
ress通道n目的分散收集地址2 |
\\
\sphinxbottomrule
\end{tabular}
\sphinxtableafterendhook\par
\sphinxattableend\end{savenotes}


\begin{savenotes}\sphinxattablestart
\sphinxthistablewithglobalstyle
\centering
\begin{tabular}[t]{\X{12}{96}\X{12}{96}\X{12}{96}\X{12}{96}\X{12}{96}\X{12}{96}\X{12}{96}\X{12}{96}}
\sphinxtoprule
\sphinxtableatstartofbodyhook
\sphinxAtStartPar
31
&
\sphinxAtStartPar
30
&
\sphinxAtStartPar
29
&
\sphinxAtStartPar
28
&
\sphinxAtStartPar
27
&
\sphinxAtStartPar
26
&
\sphinxAtStartPar
25
&
\sphinxAtStartPar
24
\\
\sphinxhline
\sphinxAtStartPar
DSTSGADDRn2
&&&&&&&\\
\sphinxhline
\sphinxAtStartPar
23
&
\sphinxAtStartPar
22
&
\sphinxAtStartPar
21
&
\sphinxAtStartPar
20
&
\sphinxAtStartPar
19
&
\sphinxAtStartPar
18
&
\sphinxAtStartPar
17
&
\sphinxAtStartPar
16
\\
\sphinxhline
\sphinxAtStartPar
DSTSGADDRn2
&&&&&&&\\
\sphinxhline
\sphinxAtStartPar
15
&
\sphinxAtStartPar
14
&
\sphinxAtStartPar
13
&
\sphinxAtStartPar
12
&
\sphinxAtStartPar
11
&
\sphinxAtStartPar
10
&
\sphinxAtStartPar
9
&
\sphinxAtStartPar
8
\\
\sphinxhline
\sphinxAtStartPar
DSTSGADDRn2
&&&&&&&\\
\sphinxhline
\sphinxAtStartPar
7
&
\sphinxAtStartPar
6
&
\sphinxAtStartPar
5
&
\sphinxAtStartPar
4
&
\sphinxAtStartPar
3
&
\sphinxAtStartPar
2
&
\sphinxAtStartPar
1
&
\sphinxAtStartPar
0
\\
\sphinxhline
\sphinxAtStartPar
DSTSGADDRn2
&&&&&&&\\
\sphinxbottomrule
\end{tabular}
\sphinxtableafterendhook\par
\sphinxattableend\end{savenotes}


\begin{savenotes}\sphinxattablestart
\sphinxthistablewithglobalstyle
\centering
\begin{tabular}[t]{\X{33}{99}\X{33}{99}\X{33}{99}}
\sphinxtoprule
\sphinxtableatstartofbodyhook
\sphinxAtStartPar
位域 |
&
\sphinxAtStartPar
名称     | |
&
\sphinxAtStartPar
描述                                        | |
\\
\sphinxhline
\sphinxAtStartPar
31:0
&
\sphinxAtStartPar
DSTSGADDRn2
&
\sphinxAtStartPar
Destination Scatter Gather Address通道n目的分散收集地址2               |
\\
\sphinxbottomrule
\end{tabular}
\sphinxtableafterendhook\par
\sphinxattableend\end{savenotes}


\subsubsection{通道n目的分散收集地址寄存器DSTSGADDRn3}
\label{\detokenize{SWM241/_u529f_u80fd_u63cf_u8ff0/_u76f4_u63a5_u5185_u5b58_u5b58_u53d6:ndstsgaddrn3}}

\begin{savenotes}\sphinxattablestart
\sphinxthistablewithglobalstyle
\centering
\begin{tabular}[t]{\X{20}{100}\X{20}{100}\X{20}{100}\X{20}{100}\X{20}{100}}
\sphinxtoprule
\sphinxtableatstartofbodyhook
\sphinxAtStartPar
寄存器 |
&
\begin{DUlineblock}{0em}
\item[] 偏移 |
\end{DUlineblock}
&
\begin{DUlineblock}{0em}
\item[] 
\item[] {\color{red}\bfseries{}|}
\end{DUlineblock}
&
\sphinxAtStartPar
复位值 |    描 | |
&
\begin{DUlineblock}{0em}
\item[] |
  |
\end{DUlineblock}
\\
\sphinxhline
\sphinxAtStartPar
DSTSGADDRn3
&
\sphinxAtStartPar
0 x40*n +  0x24
&&
\sphinxAtStartPar
0 000000
&
\sphinxAtStartPar
Destination Scatter Gather

\sphinxAtStartPar
ress通道n目的分散收集地址3 |
\\
\sphinxbottomrule
\end{tabular}
\sphinxtableafterendhook\par
\sphinxattableend\end{savenotes}


\begin{savenotes}\sphinxattablestart
\sphinxthistablewithglobalstyle
\centering
\begin{tabular}[t]{\X{12}{96}\X{12}{96}\X{12}{96}\X{12}{96}\X{12}{96}\X{12}{96}\X{12}{96}\X{12}{96}}
\sphinxtoprule
\sphinxtableatstartofbodyhook
\sphinxAtStartPar
31
&
\sphinxAtStartPar
30
&
\sphinxAtStartPar
29
&
\sphinxAtStartPar
28
&
\sphinxAtStartPar
27
&
\sphinxAtStartPar
26
&
\sphinxAtStartPar
25
&
\sphinxAtStartPar
24
\\
\sphinxhline
\sphinxAtStartPar
DSTSGADDRn3
&&&&&&&\\
\sphinxhline
\sphinxAtStartPar
23
&
\sphinxAtStartPar
22
&
\sphinxAtStartPar
21
&
\sphinxAtStartPar
20
&
\sphinxAtStartPar
19
&
\sphinxAtStartPar
18
&
\sphinxAtStartPar
17
&
\sphinxAtStartPar
16
\\
\sphinxhline
\sphinxAtStartPar
DSTSGADDRn3
&&&&&&&\\
\sphinxhline
\sphinxAtStartPar
15
&
\sphinxAtStartPar
14
&
\sphinxAtStartPar
13
&
\sphinxAtStartPar
12
&
\sphinxAtStartPar
11
&
\sphinxAtStartPar
10
&
\sphinxAtStartPar
9
&
\sphinxAtStartPar
8
\\
\sphinxhline
\sphinxAtStartPar
DSTSGADDRn3
&&&&&&&\\
\sphinxhline
\sphinxAtStartPar
7
&
\sphinxAtStartPar
6
&
\sphinxAtStartPar
5
&
\sphinxAtStartPar
4
&
\sphinxAtStartPar
3
&
\sphinxAtStartPar
2
&
\sphinxAtStartPar
1
&
\sphinxAtStartPar
0
\\
\sphinxhline
\sphinxAtStartPar
DSTSGADDRn3
&&&&&&&\\
\sphinxbottomrule
\end{tabular}
\sphinxtableafterendhook\par
\sphinxattableend\end{savenotes}


\begin{savenotes}\sphinxattablestart
\sphinxthistablewithglobalstyle
\centering
\begin{tabular}[t]{\X{33}{99}\X{33}{99}\X{33}{99}}
\sphinxtoprule
\sphinxtableatstartofbodyhook
\sphinxAtStartPar
位域 |
&
\sphinxAtStartPar
名称     | |
&
\sphinxAtStartPar
描述                                        | |
\\
\sphinxhline
\sphinxAtStartPar
31:0
&
\sphinxAtStartPar
DSTSGADDRn3
&
\sphinxAtStartPar
Destination Scatter Gather Address通道n目的分散收集地址3               |
\\
\sphinxbottomrule
\end{tabular}
\sphinxtableafterendhook\par
\sphinxattableend\end{savenotes}


\subsubsection{通道n握手信号选择寄存器MUXn}
\label{\detokenize{SWM241/_u529f_u80fd_u63cf_u8ff0/_u76f4_u63a5_u5185_u5b58_u5b58_u53d6:nmuxn}}

\begin{savenotes}\sphinxattablestart
\sphinxthistablewithglobalstyle
\centering
\begin{tabular}[t]{\X{20}{100}\X{20}{100}\X{20}{100}\X{20}{100}\X{20}{100}}
\sphinxtoprule
\sphinxtableatstartofbodyhook
\sphinxAtStartPar
寄存器 |
&
\begin{DUlineblock}{0em}
\item[] 偏移 |
\end{DUlineblock}
&
\begin{DUlineblock}{0em}
\item[] 
\item[] {\color{red}\bfseries{}|}
\end{DUlineblock}
&
\sphinxAtStartPar
复位值 |    描 | |
&
\begin{DUlineblock}{0em}
\item[] |
  |
\end{DUlineblock}
\\
\sphinxhline
\sphinxAtStartPar
MUXn
&
\sphinxAtStartPar
0 x40*n +  0x18
&&
\sphinxAtStartPar
0 000000
&
\sphinxAtStartPar
通道n握手信号选择寄存器    |
\\
\sphinxbottomrule
\end{tabular}
\sphinxtableafterendhook\par
\sphinxattableend\end{savenotes}


\begin{savenotes}\sphinxattablestart
\sphinxthistablewithglobalstyle
\centering
\begin{tabular}[t]{\X{12}{96}\X{12}{96}\X{12}{96}\X{12}{96}\X{12}{96}\X{12}{96}\X{12}{96}\X{12}{96}}
\sphinxtoprule
\sphinxtableatstartofbodyhook
\sphinxAtStartPar
31
&
\sphinxAtStartPar
30
&
\sphinxAtStartPar
29
&
\sphinxAtStartPar
28
&
\sphinxAtStartPar
27
&
\sphinxAtStartPar
26
&
\sphinxAtStartPar
25
&
\sphinxAtStartPar
24
\\
\sphinxhline\begin{itemize}
\item {} 
\end{itemize}
&&&&&&&\\
\sphinxhline
\sphinxAtStartPar
23
&
\sphinxAtStartPar
22
&
\sphinxAtStartPar
21
&
\sphinxAtStartPar
20
&
\sphinxAtStartPar
19
&
\sphinxAtStartPar
18
&
\sphinxAtStartPar
17
&
\sphinxAtStartPar
16
\\
\sphinxhline\begin{itemize}
\item {} 
\end{itemize}
&&&&
\sphinxAtStartPar
EX EN
&&&\\
\sphinxhline
\sphinxAtStartPar
15
&
\sphinxAtStartPar
14
&
\sphinxAtStartPar
13
&
\sphinxAtStartPar
12
&
\sphinxAtStartPar
11
&
\sphinxAtStartPar
10
&
\sphinxAtStartPar
9
&
\sphinxAtStartPar
8
\\
\sphinxhline\begin{itemize}
\item {} 
\end{itemize}
&&&&&
\sphinxAtStartPar
S SEN
&&\\
\sphinxhline
\sphinxAtStartPar
7
&
\sphinxAtStartPar
6
&
\sphinxAtStartPar
5
&
\sphinxAtStartPar
4
&
\sphinxAtStartPar
3
&
\sphinxAtStartPar
2
&
\sphinxAtStartPar
1
&
\sphinxAtStartPar
0
\\
\sphinxhline\begin{itemize}
\item {} 
\end{itemize}
&&&&&
\sphinxAtStartPar
D SEN
&&\\
\sphinxbottomrule
\end{tabular}
\sphinxtableafterendhook\par
\sphinxattableend\end{savenotes}


\begin{savenotes}\sphinxattablestart
\sphinxthistablewithglobalstyle
\centering
\begin{tabular}[t]{\X{33}{99}\X{33}{99}\X{33}{99}}
\sphinxtoprule
\sphinxtableatstartofbodyhook
\sphinxAtStartPar
位域 |
&
\sphinxAtStartPar
名称     | |
&
\sphinxAtStartPar
描述                                        | |
\\
\sphinxhline
\sphinxAtStartPar
31:20
&\begin{itemize}
\item {} 
\end{itemize}
&\begin{itemize}
\item {} 
\end{itemize}
\\
\sphinxhline
\sphinxAtStartPar
19
&
\sphinxAtStartPar
EXTHSEN
&
\sphinxAtStartPar
触发功能使能                                |

\sphinxAtStartPar
1:使能,TRIGGER触发                        |

\sphinxAtStartPar
0:关闭,由启动信号TXEN/RXEN进行启动        |

\sphinxAtStartPar
注1:需要使用这个寄                         | ,必须在其它相关寄存器配置完成之后再使能 |

\sphinxAtStartPar
注2:EXT                                    | N=1时,也需置TXEN或RXEN才能启动触发传输功能 |
\\
\sphinxhline
\sphinxAtStartPar
18:16
&
\sphinxAtStartPar
EXTHSSIG
&
\sphinxAtStartPar
TRIGGER信号触发配置                         |

\sphinxAtStartPar
000:TIMER0                                 |

\sphinxAtStartPar
001:TIMER1                                 |

\sphinxAtStartPar
010:TIMER2                                 |

\sphinxAtStartPar
011:TIMER3                                 |

\sphinxAtStartPar
100:TIMER4                                 |

\sphinxAtStartPar
101:DMA\_TRIG0                              |

\sphinxAtStartPar
110:DMA\_TRIG1                              |

\sphinxAtStartPar
111:保留                                   |
\\
\sphinxhline
\sphinxAtStartPar
15:11
&\begin{itemize}
\item {} 
\end{itemize}
&\begin{itemize}
\item {} 
\end{itemize}
\\
\sphinxhline
\sphinxAtStartPar
10
&
\sphinxAtStartPar
SRCHSEN
&
\sphinxAtStartPar
M1上硬件触发源使能                          |

\sphinxAtStartPar
1:硬件触发                                 |

\sphinxAtStartPar
0:RXEN软件启动                             |
\\
\sphinxhline
\sphinxAtStartPar
9:8
&
\sphinxAtStartPar
SRCHSSIG
&
\sphinxAtStartPar
M1上硬件触发源                              |

\sphinxAtStartPar
11:选择编号为4*x+3的握手信号               |

\sphinxAtStartPar
10:选择编号为4*x+2的握手信号               |

\sphinxAtStartPar
01:选择编号为4*x+1的握手信号               |

\sphinxAtStartPar
00:选择编号为4*x的握手信号                 |

\sphinxAtStartPar
详见表格 6‑3                                |
\\
\sphinxhline
\sphinxAtStartPar
7:3
&\begin{itemize}
\item {} 
\end{itemize}
&\begin{itemize}
\item {} 
\end{itemize}
\\
\sphinxhline
\sphinxAtStartPar
2
&
\sphinxAtStartPar
DSTHSEN
&
\sphinxAtStartPar
M0上硬件触发源使能                          |

\sphinxAtStartPar
1:硬件触发                                 |

\sphinxAtStartPar
0:TXEN软件启动                             |
\\
\sphinxhline
\sphinxAtStartPar
1:0
&
\sphinxAtStartPar
DSTHSSIG
&
\sphinxAtStartPar
M0上硬件触发源                              |

\sphinxAtStartPar
11:选择编号为4*x+3的握手信号               |

\sphinxAtStartPar
10:选择编号为4*x+2的握手信号               |

\sphinxAtStartPar
01:选择编号为4*x+1的握手信号               |

\sphinxAtStartPar
00:选择编号为4*x的握手信号                 |

\sphinxAtStartPar
详见表格 6‑3                                |
\\
\sphinxbottomrule
\end{tabular}
\sphinxtableafterendhook\par
\sphinxattableend\end{savenotes}


\subsubsection{通道n源地址寄存器SRCn}
\label{\detokenize{SWM241/_u529f_u80fd_u63cf_u8ff0/_u76f4_u63a5_u5185_u5b58_u5b58_u53d6:nsrcn}}

\begin{savenotes}\sphinxattablestart
\sphinxthistablewithglobalstyle
\centering
\begin{tabular}[t]{\X{20}{100}\X{20}{100}\X{20}{100}\X{20}{100}\X{20}{100}}
\sphinxtoprule
\sphinxtableatstartofbodyhook
\sphinxAtStartPar
寄存器 |
&
\begin{DUlineblock}{0em}
\item[] 偏移 |
\end{DUlineblock}
&
\begin{DUlineblock}{0em}
\item[] 
\item[] {\color{red}\bfseries{}|}
\end{DUlineblock}
&
\sphinxAtStartPar
复位值 |    描 | |
&
\begin{DUlineblock}{0em}
\item[] |
  |
\end{DUlineblock}
\\
\sphinxhline
\sphinxAtStartPar
SRCn
&
\sphinxAtStartPar
0 x40*n +  0x1C
&&
\sphinxAtStartPar
0 000000
&
\sphinxAtStartPar
通道n源地址寄存器          |
\\
\sphinxbottomrule
\end{tabular}
\sphinxtableafterendhook\par
\sphinxattableend\end{savenotes}


\begin{savenotes}\sphinxattablestart
\sphinxthistablewithglobalstyle
\centering
\begin{tabular}[t]{\X{12}{96}\X{12}{96}\X{12}{96}\X{12}{96}\X{12}{96}\X{12}{96}\X{12}{96}\X{12}{96}}
\sphinxtoprule
\sphinxtableatstartofbodyhook
\sphinxAtStartPar
31
&
\sphinxAtStartPar
30
&
\sphinxAtStartPar
29
&
\sphinxAtStartPar
28
&
\sphinxAtStartPar
27
&
\sphinxAtStartPar
26
&
\sphinxAtStartPar
25
&
\sphinxAtStartPar
24
\\
\sphinxhline
\sphinxAtStartPar
SRCn
&&&&&&&\\
\sphinxhline
\sphinxAtStartPar
23
&
\sphinxAtStartPar
22
&
\sphinxAtStartPar
21
&
\sphinxAtStartPar
20
&
\sphinxAtStartPar
19
&
\sphinxAtStartPar
18
&
\sphinxAtStartPar
17
&
\sphinxAtStartPar
16
\\
\sphinxhline
\sphinxAtStartPar
SRCn
&&&&&&&\\
\sphinxhline
\sphinxAtStartPar
15
&
\sphinxAtStartPar
14
&
\sphinxAtStartPar
13
&
\sphinxAtStartPar
12
&
\sphinxAtStartPar
11
&
\sphinxAtStartPar
10
&
\sphinxAtStartPar
9
&
\sphinxAtStartPar
8
\\
\sphinxhline
\sphinxAtStartPar
SRCn
&&&&&&&\\
\sphinxhline
\sphinxAtStartPar
7
&
\sphinxAtStartPar
6
&
\sphinxAtStartPar
5
&
\sphinxAtStartPar
4
&
\sphinxAtStartPar
3
&
\sphinxAtStartPar
2
&
\sphinxAtStartPar
1
&
\sphinxAtStartPar
0
\\
\sphinxhline
\sphinxAtStartPar
SRCn
&&&&&&&\\
\sphinxbottomrule
\end{tabular}
\sphinxtableafterendhook\par
\sphinxattableend\end{savenotes}


\begin{savenotes}\sphinxattablestart
\sphinxthistablewithglobalstyle
\centering
\begin{tabular}[t]{\X{33}{99}\X{33}{99}\X{33}{99}}
\sphinxtoprule
\sphinxtableatstartofbodyhook
\sphinxAtStartPar
位域 |
&
\sphinxAtStartPar
名称     | |
&
\sphinxAtStartPar
描述                                        | |
\\
\sphinxhline
\sphinxAtStartPar
31:0
&
\sphinxAtStartPar
SRCn
&
\sphinxAtStartPar
源地址                                      |
\\
\sphinxbottomrule
\end{tabular}
\sphinxtableafterendhook\par
\sphinxattableend\end{savenotes}


\subsubsection{通道n源分散收集地址1寄存器SRCSGADDRn1}
\label{\detokenize{SWM241/_u529f_u80fd_u63cf_u8ff0/_u76f4_u63a5_u5185_u5b58_u5b58_u53d6:n1srcsgaddrn1}}

\begin{savenotes}\sphinxattablestart
\sphinxthistablewithglobalstyle
\centering
\begin{tabular}[t]{\X{20}{100}\X{20}{100}\X{20}{100}\X{20}{100}\X{20}{100}}
\sphinxtoprule
\sphinxtableatstartofbodyhook
\sphinxAtStartPar
寄存器 |
&
\begin{DUlineblock}{0em}
\item[] 偏移 |
\end{DUlineblock}
&
\begin{DUlineblock}{0em}
\item[] 
\item[] {\color{red}\bfseries{}|}
\end{DUlineblock}
&
\sphinxAtStartPar
复位值 |    描 | |
&
\begin{DUlineblock}{0em}
\item[] |
  |
\end{DUlineblock}
\\
\sphinxhline
\sphinxAtStartPar
SRCSGADDRn1
&
\sphinxAtStartPar
0 x40*n +  0x20
&&
\sphinxAtStartPar
0 000000
&
\sphinxAtStartPar
Source Scatter Gather

\sphinxAtStartPar
ddress通道n源分散收集地址1 |
\\
\sphinxbottomrule
\end{tabular}
\sphinxtableafterendhook\par
\sphinxattableend\end{savenotes}


\begin{savenotes}\sphinxattablestart
\sphinxthistablewithglobalstyle
\centering
\begin{tabular}[t]{\X{12}{96}\X{12}{96}\X{12}{96}\X{12}{96}\X{12}{96}\X{12}{96}\X{12}{96}\X{12}{96}}
\sphinxtoprule
\sphinxtableatstartofbodyhook
\sphinxAtStartPar
31
&
\sphinxAtStartPar
30
&
\sphinxAtStartPar
29
&
\sphinxAtStartPar
28
&
\sphinxAtStartPar
27
&
\sphinxAtStartPar
26
&
\sphinxAtStartPar
25
&
\sphinxAtStartPar
24
\\
\sphinxhline
\sphinxAtStartPar
SRCSGADDRn1
&&&&&&&\\
\sphinxhline
\sphinxAtStartPar
23
&
\sphinxAtStartPar
22
&
\sphinxAtStartPar
21
&
\sphinxAtStartPar
20
&
\sphinxAtStartPar
19
&
\sphinxAtStartPar
18
&
\sphinxAtStartPar
17
&
\sphinxAtStartPar
16
\\
\sphinxhline
\sphinxAtStartPar
SRCSGADDRn1
&&&&&&&\\
\sphinxhline
\sphinxAtStartPar
15
&
\sphinxAtStartPar
14
&
\sphinxAtStartPar
13
&
\sphinxAtStartPar
12
&
\sphinxAtStartPar
11
&
\sphinxAtStartPar
10
&
\sphinxAtStartPar
9
&
\sphinxAtStartPar
8
\\
\sphinxhline
\sphinxAtStartPar
SRCSGADDRn1
&&&&&&&\\
\sphinxhline
\sphinxAtStartPar
7
&
\sphinxAtStartPar
6
&
\sphinxAtStartPar
5
&
\sphinxAtStartPar
4
&
\sphinxAtStartPar
3
&
\sphinxAtStartPar
2
&
\sphinxAtStartPar
1
&
\sphinxAtStartPar
0
\\
\sphinxhline
\sphinxAtStartPar
SRCSGADDRn1
&&&&&&&\\
\sphinxbottomrule
\end{tabular}
\sphinxtableafterendhook\par
\sphinxattableend\end{savenotes}


\begin{savenotes}\sphinxattablestart
\sphinxthistablewithglobalstyle
\centering
\begin{tabular}[t]{\X{33}{99}\X{33}{99}\X{33}{99}}
\sphinxtoprule
\sphinxtableatstartofbodyhook
\sphinxAtStartPar
位域 |
&
\sphinxAtStartPar
名称     | |
&
\sphinxAtStartPar
描述                                        | |
\\
\sphinxhline
\sphinxAtStartPar
31:0
&
\sphinxAtStartPar
SRCSGADDRn1
&
\sphinxAtStartPar
Source Scatter Gather Address通道n源分散收集地址1                 |
\\
\sphinxbottomrule
\end{tabular}
\sphinxtableafterendhook\par
\sphinxattableend\end{savenotes}


\subsubsection{通道n源分散收集地址2寄存器SRCSGADDRn2}
\label{\detokenize{SWM241/_u529f_u80fd_u63cf_u8ff0/_u76f4_u63a5_u5185_u5b58_u5b58_u53d6:n2srcsgaddrn2}}

\begin{savenotes}\sphinxattablestart
\sphinxthistablewithglobalstyle
\centering
\begin{tabular}[t]{\X{20}{100}\X{20}{100}\X{20}{100}\X{20}{100}\X{20}{100}}
\sphinxtoprule
\sphinxtableatstartofbodyhook
\sphinxAtStartPar
寄存器 |
&
\begin{DUlineblock}{0em}
\item[] 偏移 |
\end{DUlineblock}
&
\begin{DUlineblock}{0em}
\item[] 
\item[] {\color{red}\bfseries{}|}
\end{DUlineblock}
&
\sphinxAtStartPar
复位值 |    描 | |
&
\begin{DUlineblock}{0em}
\item[] |
  |
\end{DUlineblock}
\\
\sphinxhline
\sphinxAtStartPar
SRCSGADDRn2
&
\sphinxAtStartPar
0 x40*n +  0x24
&&
\sphinxAtStartPar
0 000000
&
\sphinxAtStartPar
Source Scatter Gather

\sphinxAtStartPar
ddress通道n源分散收集地址2 |
\\
\sphinxbottomrule
\end{tabular}
\sphinxtableafterendhook\par
\sphinxattableend\end{savenotes}


\begin{savenotes}\sphinxattablestart
\sphinxthistablewithglobalstyle
\centering
\begin{tabular}[t]{\X{12}{96}\X{12}{96}\X{12}{96}\X{12}{96}\X{12}{96}\X{12}{96}\X{12}{96}\X{12}{96}}
\sphinxtoprule
\sphinxtableatstartofbodyhook
\sphinxAtStartPar
31
&
\sphinxAtStartPar
30
&
\sphinxAtStartPar
29
&
\sphinxAtStartPar
28
&
\sphinxAtStartPar
27
&
\sphinxAtStartPar
26
&
\sphinxAtStartPar
25
&
\sphinxAtStartPar
24
\\
\sphinxhline
\sphinxAtStartPar
SRCSGADDRn2
&&&&&&&\\
\sphinxhline
\sphinxAtStartPar
23
&
\sphinxAtStartPar
22
&
\sphinxAtStartPar
21
&
\sphinxAtStartPar
20
&
\sphinxAtStartPar
19
&
\sphinxAtStartPar
18
&
\sphinxAtStartPar
17
&
\sphinxAtStartPar
16
\\
\sphinxhline
\sphinxAtStartPar
SRCSGADDRn2
&&&&&&&\\
\sphinxhline
\sphinxAtStartPar
15
&
\sphinxAtStartPar
14
&
\sphinxAtStartPar
13
&
\sphinxAtStartPar
12
&
\sphinxAtStartPar
11
&
\sphinxAtStartPar
10
&
\sphinxAtStartPar
9
&
\sphinxAtStartPar
8
\\
\sphinxhline
\sphinxAtStartPar
SRCSGADDRn2
&&&&&&&\\
\sphinxhline
\sphinxAtStartPar
7
&
\sphinxAtStartPar
6
&
\sphinxAtStartPar
5
&
\sphinxAtStartPar
4
&
\sphinxAtStartPar
3
&
\sphinxAtStartPar
2
&
\sphinxAtStartPar
1
&
\sphinxAtStartPar
0
\\
\sphinxhline
\sphinxAtStartPar
SRCSGADDRn2
&&&&&&&\\
\sphinxbottomrule
\end{tabular}
\sphinxtableafterendhook\par
\sphinxattableend\end{savenotes}


\begin{savenotes}\sphinxattablestart
\sphinxthistablewithglobalstyle
\centering
\begin{tabular}[t]{\X{33}{99}\X{33}{99}\X{33}{99}}
\sphinxtoprule
\sphinxtableatstartofbodyhook
\sphinxAtStartPar
位域 |
&
\sphinxAtStartPar
名称     | |
&
\sphinxAtStartPar
描述                                        | |
\\
\sphinxhline
\sphinxAtStartPar
31:0
&
\sphinxAtStartPar
SRCSGADDRn2
&
\sphinxAtStartPar
Source Scatter Gather Address通道n源分散收集地址2                 |
\\
\sphinxbottomrule
\end{tabular}
\sphinxtableafterendhook\par
\sphinxattableend\end{savenotes}


\subsubsection{通道n源分散收集地址3寄存器SRCSGADDRn3}
\label{\detokenize{SWM241/_u529f_u80fd_u63cf_u8ff0/_u76f4_u63a5_u5185_u5b58_u5b58_u53d6:n3srcsgaddrn3}}

\begin{savenotes}\sphinxattablestart
\sphinxthistablewithglobalstyle
\centering
\begin{tabular}[t]{\X{20}{100}\X{20}{100}\X{20}{100}\X{20}{100}\X{20}{100}}
\sphinxtoprule
\sphinxtableatstartofbodyhook
\sphinxAtStartPar
寄存器 |
&
\begin{DUlineblock}{0em}
\item[] 偏移 |
\end{DUlineblock}
&
\begin{DUlineblock}{0em}
\item[] 
\item[] {\color{red}\bfseries{}|}
\end{DUlineblock}
&
\sphinxAtStartPar
复位值 |    描 | |
&
\begin{DUlineblock}{0em}
\item[] |
  |
\end{DUlineblock}
\\
\sphinxhline
\sphinxAtStartPar
SRCSGADDRn3
&
\sphinxAtStartPar
0 x40*n +  0x28
&&
\sphinxAtStartPar
0 000000
&
\sphinxAtStartPar
Source Scatter Gather

\sphinxAtStartPar
ddress通道n源分散收集地址3 |
\\
\sphinxbottomrule
\end{tabular}
\sphinxtableafterendhook\par
\sphinxattableend\end{savenotes}


\begin{savenotes}\sphinxattablestart
\sphinxthistablewithglobalstyle
\centering
\begin{tabular}[t]{\X{12}{96}\X{12}{96}\X{12}{96}\X{12}{96}\X{12}{96}\X{12}{96}\X{12}{96}\X{12}{96}}
\sphinxtoprule
\sphinxtableatstartofbodyhook
\sphinxAtStartPar
31
&
\sphinxAtStartPar
30
&
\sphinxAtStartPar
29
&
\sphinxAtStartPar
28
&
\sphinxAtStartPar
27
&
\sphinxAtStartPar
26
&
\sphinxAtStartPar
25
&
\sphinxAtStartPar
24
\\
\sphinxhline
\sphinxAtStartPar
SRCSGADDRn3
&&&&&&&\\
\sphinxhline
\sphinxAtStartPar
23
&
\sphinxAtStartPar
22
&
\sphinxAtStartPar
21
&
\sphinxAtStartPar
20
&
\sphinxAtStartPar
19
&
\sphinxAtStartPar
18
&
\sphinxAtStartPar
17
&
\sphinxAtStartPar
16
\\
\sphinxhline
\sphinxAtStartPar
SRCSGADDRn3
&&&&&&&\\
\sphinxhline
\sphinxAtStartPar
15
&
\sphinxAtStartPar
14
&
\sphinxAtStartPar
13
&
\sphinxAtStartPar
12
&
\sphinxAtStartPar
11
&
\sphinxAtStartPar
10
&
\sphinxAtStartPar
9
&
\sphinxAtStartPar
8
\\
\sphinxhline
\sphinxAtStartPar
SRCSGADDRn3
&&&&&&&\\
\sphinxhline
\sphinxAtStartPar
7
&
\sphinxAtStartPar
6
&
\sphinxAtStartPar
5
&
\sphinxAtStartPar
4
&
\sphinxAtStartPar
3
&
\sphinxAtStartPar
2
&
\sphinxAtStartPar
1
&
\sphinxAtStartPar
0
\\
\sphinxhline
\sphinxAtStartPar
SRCSGADDRn3
&&&&&&&\\
\sphinxbottomrule
\end{tabular}
\sphinxtableafterendhook\par
\sphinxattableend\end{savenotes}


\begin{savenotes}\sphinxattablestart
\sphinxthistablewithglobalstyle
\centering
\begin{tabular}[t]{\X{33}{99}\X{33}{99}\X{33}{99}}
\sphinxtoprule
\sphinxtableatstartofbodyhook
\sphinxAtStartPar
位域 |
&
\sphinxAtStartPar
名称     | |
&
\sphinxAtStartPar
描述                                        | |
\\
\sphinxhline
\sphinxAtStartPar
31:0
&
\sphinxAtStartPar
SRCSGADDRn3
&
\sphinxAtStartPar
Source Scatter Gather Address通道n源分散收集地址3                 |
\\
\sphinxbottomrule
\end{tabular}
\sphinxtableafterendhook\par
\sphinxattableend\end{savenotes}


\subsubsection{M0通道n状态寄存器DSTSR}
\label{\detokenize{SWM241/_u529f_u80fd_u63cf_u8ff0/_u76f4_u63a5_u5185_u5b58_u5b58_u53d6:m0ndstsr}}

\begin{savenotes}\sphinxattablestart
\sphinxthistablewithglobalstyle
\centering
\begin{tabular}[t]{\X{20}{100}\X{20}{100}\X{20}{100}\X{20}{100}\X{20}{100}}
\sphinxtoprule
\sphinxtableatstartofbodyhook
\sphinxAtStartPar
寄存器 |
&
\begin{DUlineblock}{0em}
\item[] 偏移 |
\end{DUlineblock}
&
\begin{DUlineblock}{0em}
\item[] 
\item[] {\color{red}\bfseries{}|}
\end{DUlineblock}
&
\sphinxAtStartPar
复位值 |    描 | |
&
\begin{DUlineblock}{0em}
\item[] |
  |
\end{DUlineblock}
\\
\sphinxhline
\sphinxAtStartPar
DSTSR
&
\sphinxAtStartPar
0 x40*n +  0x2C
&&
\sphinxAtStartPar
0 000000
&
\sphinxAtStartPar
M0通道n状态寄存器          |
\\
\sphinxbottomrule
\end{tabular}
\sphinxtableafterendhook\par
\sphinxattableend\end{savenotes}


\begin{savenotes}\sphinxattablestart
\sphinxthistablewithglobalstyle
\centering
\begin{tabular}[t]{\X{12}{96}\X{12}{96}\X{12}{96}\X{12}{96}\X{12}{96}\X{12}{96}\X{12}{96}\X{12}{96}}
\sphinxtoprule
\sphinxtableatstartofbodyhook
\sphinxAtStartPar
31
&
\sphinxAtStartPar
30
&
\sphinxAtStartPar
29
&
\sphinxAtStartPar
28
&
\sphinxAtStartPar
27
&
\sphinxAtStartPar
26
&
\sphinxAtStartPar
25
&
\sphinxAtStartPar
24
\\
\sphinxhline
\sphinxAtStartPar
DSTERR
&\begin{itemize}
\item {} 
\end{itemize}
&&&&&&\\
\sphinxhline
\sphinxAtStartPar
23
&
\sphinxAtStartPar
22
&
\sphinxAtStartPar
21
&
\sphinxAtStartPar
20
&
\sphinxAtStartPar
19
&
\sphinxAtStartPar
18
&
\sphinxAtStartPar
17
&
\sphinxAtStartPar
16
\\
\sphinxhline\begin{itemize}
\item {} 
\end{itemize}
&&&&&&&\\
\sphinxhline
\sphinxAtStartPar
15
&
\sphinxAtStartPar
14
&
\sphinxAtStartPar
13
&
\sphinxAtStartPar
12
&
\sphinxAtStartPar
11
&
\sphinxAtStartPar
10
&
\sphinxAtStartPar
9
&
\sphinxAtStartPar
8
\\
\sphinxhline\begin{itemize}
\item {} 
\end{itemize}
&&&&
\sphinxAtStartPar
D EN
&&&\\
\sphinxhline
\sphinxAtStartPar
7
&
\sphinxAtStartPar
6
&
\sphinxAtStartPar
5
&
\sphinxAtStartPar
4
&
\sphinxAtStartPar
3
&
\sphinxAtStartPar
2
&
\sphinxAtStartPar
1
&
\sphinxAtStartPar
0
\\
\sphinxhline
\sphinxAtStartPar
DSTLEN
&&&&&&&\\
\sphinxbottomrule
\end{tabular}
\sphinxtableafterendhook\par
\sphinxattableend\end{savenotes}


\begin{savenotes}\sphinxattablestart
\sphinxthistablewithglobalstyle
\centering
\begin{tabular}[t]{\X{33}{99}\X{33}{99}\X{33}{99}}
\sphinxtoprule
\sphinxtableatstartofbodyhook
\sphinxAtStartPar
位域 |
&
\sphinxAtStartPar
名称     | |
&
\sphinxAtStartPar
描述                                        | |
\\
\sphinxhline
\sphinxAtStartPar
31
&
\sphinxAtStartPar
DSTERR
&
\sphinxAtStartPar
M0长度配置错误                              |
\\
\sphinxhline
\sphinxAtStartPar
30:12
&\begin{itemize}
\item {} 
\end{itemize}
&\begin{itemize}
\item {} 
\end{itemize}
\\
\sphinxhline
\sphinxAtStartPar
11:0
&
\sphinxAtStartPar
DSTLEN
&
\sphinxAtStartPar
M0剩余传输量                                |
\\
\sphinxbottomrule
\end{tabular}
\sphinxtableafterendhook\par
\sphinxattableend\end{savenotes}


\subsubsection{M1通道n状态寄存器SRCSR}
\label{\detokenize{SWM241/_u529f_u80fd_u63cf_u8ff0/_u76f4_u63a5_u5185_u5b58_u5b58_u53d6:m1nsrcsr}}

\begin{savenotes}\sphinxattablestart
\sphinxthistablewithglobalstyle
\centering
\begin{tabular}[t]{\X{20}{100}\X{20}{100}\X{20}{100}\X{20}{100}\X{20}{100}}
\sphinxtoprule
\sphinxtableatstartofbodyhook
\sphinxAtStartPar
寄存器 |
&
\begin{DUlineblock}{0em}
\item[] 偏移 |
\end{DUlineblock}
&
\begin{DUlineblock}{0em}
\item[] 
\item[] {\color{red}\bfseries{}|}
\end{DUlineblock}
&
\sphinxAtStartPar
复位值 |    描 | |
&
\begin{DUlineblock}{0em}
\item[] |
  |
\end{DUlineblock}
\\
\sphinxhline
\sphinxAtStartPar
SRCSR
&
\sphinxAtStartPar
0 x40*n +  0x30
&&
\sphinxAtStartPar
0 000000
&
\sphinxAtStartPar
M1通道n状态寄存器          |
\\
\sphinxbottomrule
\end{tabular}
\sphinxtableafterendhook\par
\sphinxattableend\end{savenotes}


\begin{savenotes}\sphinxattablestart
\sphinxthistablewithglobalstyle
\centering
\begin{tabular}[t]{\X{12}{96}\X{12}{96}\X{12}{96}\X{12}{96}\X{12}{96}\X{12}{96}\X{12}{96}\X{12}{96}}
\sphinxtoprule
\sphinxtableatstartofbodyhook
\sphinxAtStartPar
31
&
\sphinxAtStartPar
30
&
\sphinxAtStartPar
29
&
\sphinxAtStartPar
28
&
\sphinxAtStartPar
27
&
\sphinxAtStartPar
26
&
\sphinxAtStartPar
25
&
\sphinxAtStartPar
24
\\
\sphinxhline
\sphinxAtStartPar
SRCERR
&\begin{itemize}
\item {} 
\end{itemize}
&&&&&&\\
\sphinxhline
\sphinxAtStartPar
23
&
\sphinxAtStartPar
22
&
\sphinxAtStartPar
21
&
\sphinxAtStartPar
20
&
\sphinxAtStartPar
19
&
\sphinxAtStartPar
18
&
\sphinxAtStartPar
17
&
\sphinxAtStartPar
16
\\
\sphinxhline\begin{itemize}
\item {} 
\end{itemize}
&&&&&&&\\
\sphinxhline
\sphinxAtStartPar
15
&
\sphinxAtStartPar
14
&
\sphinxAtStartPar
13
&
\sphinxAtStartPar
12
&
\sphinxAtStartPar
11
&
\sphinxAtStartPar
10
&
\sphinxAtStartPar
9
&
\sphinxAtStartPar
8
\\
\sphinxhline\begin{itemize}
\item {} 
\end{itemize}
&&&&
\sphinxAtStartPar
S EN
&&&\\
\sphinxhline
\sphinxAtStartPar
7
&
\sphinxAtStartPar
6
&
\sphinxAtStartPar
5
&
\sphinxAtStartPar
4
&
\sphinxAtStartPar
3
&
\sphinxAtStartPar
2
&
\sphinxAtStartPar
1
&
\sphinxAtStartPar
0
\\
\sphinxhline
\sphinxAtStartPar
SRCLEN
&&&&&&&\\
\sphinxbottomrule
\end{tabular}
\sphinxtableafterendhook\par
\sphinxattableend\end{savenotes}


\begin{savenotes}\sphinxattablestart
\sphinxthistablewithglobalstyle
\centering
\begin{tabular}[t]{\X{33}{99}\X{33}{99}\X{33}{99}}
\sphinxtoprule
\sphinxtableatstartofbodyhook
\sphinxAtStartPar
位域 |
&
\sphinxAtStartPar
名称     | |
&
\sphinxAtStartPar
描述                                        | |
\\
\sphinxhline
\sphinxAtStartPar
31
&
\sphinxAtStartPar
SRCERR
&
\sphinxAtStartPar
M1长度配置错误                              |
\\
\sphinxhline
\sphinxAtStartPar
30:12
&\begin{itemize}
\item {} 
\end{itemize}
&\begin{itemize}
\item {} 
\end{itemize}
\\
\sphinxhline
\sphinxAtStartPar
11:0
&
\sphinxAtStartPar
SRCLEN
&
\sphinxAtStartPar
M1剩余传输量                                |
\\
\sphinxbottomrule
\end{tabular}
\sphinxtableafterendhook\par
\sphinxattableend\end{savenotes}

\sphinxstepscope


\section{CRC计算单元(CRC)}
\label{\detokenize{SWM241/_u529f_u80fd_u63cf_u8ff0/CRC_u8ba1_u7b97_u5355_u5143:crc-crc}}\label{\detokenize{SWM241/_u529f_u80fd_u63cf_u8ff0/CRC_u8ba1_u7b97_u5355_u5143::doc}}
\sphinxAtStartPar
概述
\textasciitilde{}\textasciitilde{}

\sphinxAtStartPar
SWM241系列所有型号CRC模块操作均相同,主要应用于核实数据传输或者数据存储的正确性和完整性,使用前需使能CRC模块时钟。

\sphinxAtStartPar
CRC模块分为CRC\sphinxhyphen{}32、CRC\sphinxhyphen{}16、CRC\sphinxhyphen{}8。使用CRC\sphinxhyphen{}32多项式进行计算时,输入数据有效位宽可选择为32Bit、16Bit、8Bit,使用CRC\sphinxhyphen{}16多项式进行计算时,输入数据有效位宽可选择16Bit、8Bit。

\sphinxAtStartPar
特性
\textasciitilde{}\textasciitilde{}
\begin{itemize}
\item {} 
\sphinxAtStartPar
支持四种多项式
\begin{itemize}
\item {} 
\sphinxAtStartPar
x\textasciicircum{}32+x\textasciicircum{}26+x\textasciicircum{}23+x\textasciicircum{}22+x\textasciicircum{}16+x\textasciicircum{}12+x\textasciicircum{}11+x\textasciicircum{}10+x\textasciicircum{}8+x\textasciicircum{}7+x\textasciicircum{}5+x\textasciicircum{}4+x\textasciicircum{}2+x+1

\item {} 
\sphinxAtStartPar
x\textasciicircum{}16+x\textasciicircum{}12+x\textasciicircum{}5+1

\item {} 
\sphinxAtStartPar
x\textasciicircum{}16+x\textasciicircum{}15+x\textasciicircum{}2+1

\item {} 
\sphinxAtStartPar
x\textasciicircum{}8+x\textasciicircum{}2+x+1

\end{itemize}

\item {} 
\sphinxAtStartPar
支持输出结果设置,包括翻转、取反

\item {} 
\sphinxAtStartPar
支持初始值自定义

\item {} 
\sphinxAtStartPar
支持输入可选择取反

\end{itemize}


\subsection{模块结构框图}
\label{\detokenize{SWM241/_u529f_u80fd_u63cf_u8ff0/CRC_u8ba1_u7b97_u5355_u5143:id1}}
\sphinxAtStartPar
CRC循环冗余检验结构框图如图 6‑62所示:

\sphinxAtStartPar
\sphinxincludegraphics{{SWM241/功能描述/mediaCRC计算单002}.emf}

\sphinxAtStartPar
图 6‑62 CRC结构框图


\subsection{功能描述}
\label{\detokenize{SWM241/_u529f_u80fd_u63cf_u8ff0/CRC_u8ba1_u7b97_u5355_u5143:id2}}

\subsubsection{计算步骤}
\label{\detokenize{SWM241/_u529f_u80fd_u63cf_u8ff0/CRC_u8ba1_u7b97_u5355_u5143:id3}}\begin{itemize}
\item {} 
\sphinxAtStartPar
根据需求,通过CR寄存器选择CRC算法、输入数据有效位宽、输出结果

\item {} 
\sphinxAtStartPar
根据需求,通过INIVAL寄存器设置CRC初始值

\item {} 
\sphinxAtStartPar
通过CR寄存器使能CRC计算

\item {} 
\sphinxAtStartPar
通过DATAIN寄存器向CRC计算单元输入要计算的数据

\item {} 
\sphinxAtStartPar
通过RESULT寄存器读取计算结果

\end{itemize}


\subsection{寄存器映射}
\label{\detokenize{SWM241/_u529f_u80fd_u63cf_u8ff0/CRC_u8ba1_u7b97_u5355_u5143:id4}}

\begin{savenotes}\sphinxattablestart
\sphinxthistablewithglobalstyle
\centering
\begin{tabular}[t]{\X{20}{100}\X{20}{100}\X{20}{100}\X{20}{100}\X{20}{100}}
\sphinxtoprule
\sphinxtableatstartofbodyhook
\sphinxAtStartPar
名称   |
&
\begin{DUlineblock}{0em}
\item[] 偏移 |
\end{DUlineblock}
&
\begin{DUlineblock}{0em}
\item[] 
\item[] |
|
\end{DUlineblock}
&
\begin{DUlineblock}{0em}
\item[] 
\end{DUlineblock}
\begin{quote}

\begin{DUlineblock}{0em}
\item[] 
\item[] 
\end{DUlineblock}
\end{quote}
&
\sphinxAtStartPar
描述                       | | | |
\\
\sphinxhline
\sphinxAtStartPar
CRCBASE:0 {\color{red}\bfseries{}|}x40002800
&
\begin{DUlineblock}{0em}
\item[] 
\end{DUlineblock}
&&&\\
\sphinxhline
\sphinxAtStartPar
CR
&
\sphinxAtStartPar
0x00
&&
\sphinxAtStartPar
0x 00000
&
\sphinxAtStartPar
CRC状态控制寄存器          |
\\
\sphinxhline
\sphinxAtStartPar
DATAIN
&
\sphinxAtStartPar
0x04
&&
\sphinxAtStartPar
0x 00000
&
\sphinxAtStartPar
CRC数据输入寄存器          |
\\
\sphinxhline
\sphinxAtStartPar
INIVAL
&
\sphinxAtStartPar
0x08
&&
\sphinxAtStartPar
0x 00000
&
\sphinxAtStartPar
CRC初始值设置寄存器        |
\\
\sphinxhline
\sphinxAtStartPar
RESULT
&
\sphinxAtStartPar
0x0C
&&
\sphinxAtStartPar
0x 00000
&
\sphinxAtStartPar
CRC结果输出寄存器          |
\\
\sphinxbottomrule
\end{tabular}
\sphinxtableafterendhook\par
\sphinxattableend\end{savenotes}


\subsection{寄存器描述}
\label{\detokenize{SWM241/_u529f_u80fd_u63cf_u8ff0/CRC_u8ba1_u7b97_u5355_u5143:id7}}

\subsubsection{控制寄存器CR}
\label{\detokenize{SWM241/_u529f_u80fd_u63cf_u8ff0/CRC_u8ba1_u7b97_u5355_u5143:cr}}

\begin{savenotes}\sphinxattablestart
\sphinxthistablewithglobalstyle
\centering
\begin{tabular}[t]{\X{20}{100}\X{20}{100}\X{20}{100}\X{20}{100}\X{20}{100}}
\sphinxtoprule
\sphinxtableatstartofbodyhook
\sphinxAtStartPar
寄存器 |
&
\begin{DUlineblock}{0em}
\item[] 偏移 |
\end{DUlineblock}
&
\begin{DUlineblock}{0em}
\item[] 
\item[] {\color{red}\bfseries{}|}
\end{DUlineblock}
&
\sphinxAtStartPar
复位值 |    描 | |
&
\begin{DUlineblock}{0em}
\item[] |
  |
\end{DUlineblock}
\\
\sphinxhline
\sphinxAtStartPar
CR
&
\sphinxAtStartPar
0x00
&&
\sphinxAtStartPar
0 000000
&
\sphinxAtStartPar
CRC状态控制寄存器          |
\\
\sphinxbottomrule
\end{tabular}
\sphinxtableafterendhook\par
\sphinxattableend\end{savenotes}


\begin{savenotes}\sphinxattablestart
\sphinxthistablewithglobalstyle
\centering
\begin{tabular}[t]{\X{12}{96}\X{12}{96}\X{12}{96}\X{12}{96}\X{12}{96}\X{12}{96}\X{12}{96}\X{12}{96}}
\sphinxtoprule
\sphinxtableatstartofbodyhook
\sphinxAtStartPar
31
&
\sphinxAtStartPar
30
&
\sphinxAtStartPar
29
&
\sphinxAtStartPar
28
&
\sphinxAtStartPar
27
&
\sphinxAtStartPar
26
&
\sphinxAtStartPar
25
&
\sphinxAtStartPar
24
\\
\sphinxhline\begin{itemize}
\item {} 
\end{itemize}
&&&&&&&\\
\sphinxhline
\sphinxAtStartPar
23
&
\sphinxAtStartPar
22
&
\sphinxAtStartPar
21
&
\sphinxAtStartPar
20
&
\sphinxAtStartPar
19
&
\sphinxAtStartPar
18
&
\sphinxAtStartPar
17
&
\sphinxAtStartPar
16
\\
\sphinxhline\begin{itemize}
\item {} 
\end{itemize}
&&&&&&&\\
\sphinxhline
\sphinxAtStartPar
15
&
\sphinxAtStartPar
14
&
\sphinxAtStartPar
13
&
\sphinxAtStartPar
12
&
\sphinxAtStartPar
11
&
\sphinxAtStartPar
10
&
\sphinxAtStartPar
9
&
\sphinxAtStartPar
8
\\
\sphinxhline\begin{itemize}
\item {} 
\end{itemize}
&&&&&
\sphinxAtStartPar
IBIT
&&
\sphinxAtStartPar
POLY
\\
\sphinxhline
\sphinxAtStartPar
7
&
\sphinxAtStartPar
6
&
\sphinxAtStartPar
5
&
\sphinxAtStartPar
4
&
\sphinxAtStartPar
3
&
\sphinxAtStartPar
2
&
\sphinxAtStartPar
1
&
\sphinxAtStartPar
0
\\
\sphinxhline
\sphinxAtStartPar
POLY
&
\sphinxAtStartPar
ONOT
&
\sphinxAtStartPar
OREV
&&
\sphinxAtStartPar
INOT
&
\sphinxAtStartPar
IREV
&&
\sphinxAtStartPar
EN
\\
\sphinxbottomrule
\end{tabular}
\sphinxtableafterendhook\par
\sphinxattableend\end{savenotes}


\begin{savenotes}\sphinxattablestart
\sphinxthistablewithglobalstyle
\centering
\begin{tabular}[t]{\X{33}{99}\X{33}{99}\X{33}{99}}
\sphinxtoprule
\sphinxtableatstartofbodyhook
\sphinxAtStartPar
位域 |
&
\sphinxAtStartPar
名称     | |
&
\sphinxAtStartPar
描述                                        | |
\\
\sphinxhline
\sphinxAtStartPar
31:11
&\begin{itemize}
\item {} 
\end{itemize}
&\begin{itemize}
\item {} 
\end{itemize}
\\
\sphinxhline
\sphinxAtStartPar
10:9
&
\sphinxAtStartPar
IBIT
&
\sphinxAtStartPar
CRC输入数据有效位数寄存器                   |

\sphinxAtStartPar
00:32位输入数据有效                        |

\sphinxAtStartPar
01:低16位输入数据有效                      |

\sphinxAtStartPar
10:低8位输入数据有效                       |

\sphinxAtStartPar
11:保留                                    |
\\
\sphinxhline
\sphinxAtStartPar
8:7
&
\sphinxAtStartPar
POLY
&
\sphinxAtStartPar
CRC算法选择寄存器                           |

\sphinxAtStartPar
00:x\textasciicircum{}16+x\textasciicircum{}12+x\textasciicircum{}5+1                         |

\sphinxAtStartPar
01:x\textasciicircum{}8+x\textasciicircum{}2+x+1                             |

\sphinxAtStartPar
10:x\textasciicircum{}16+x\textasciicircum{}15+x\textasciicircum{}2+1                         |

\sphinxAtStartPar
11:x\textasciicircum{}32+x\textasciicircum{}26+x\textasciicircum{}23+x\textasciicircum{}                       | x\textasciicircum{}16+x\textasciicircum{}12+x\textasciicircum{}11+x\textasciicircum{}10+x\textasciicircum{}8+x\textasciicircum{}7+x\textasciicircum{}5+x\textasciicircum{}4+x\textasciicircum{}2+x+1
\\
\sphinxhline
\sphinxAtStartPar
6
&
\sphinxAtStartPar
ONOT
&
\sphinxAtStartPar
输出结果是否取反寄存器                      |

\sphinxAtStartPar
1:输出结果取反                             |

\sphinxAtStartPar
0:输出结果不需要取反                       |
\\
\sphinxhline
\sphinxAtStartPar
5:4
&
\sphinxAtStartPar
OREV
&
\sphinxAtStartPar
输出结果是否翻转寄存器                      |

\sphinxAtStartPar
00:bit顺序不变                             |

\sphinxAtStartPar
01:bit顺序完全翻转(32位数据宽度31:0 \sphinxhyphen{}\textgreater{}    | 0:31;16位数据宽度15:0 \sphinxhyphen{}\textgreater{}                   | 0:15;8位数据宽度7:0 \sphinxhyphen{}\textgreater{} 0:7)               |

\sphinxAtStartPar
10:bit顺序在字节范围内翻转(32位数据宽度   | 31:0 \sphinxhyphen{}\textgreater{} 24:31, 16:23, 8:15, 0:7;16位数据宽度15:0 \sphinxhyphen{}\textgreater{} 8:15,              | 0:7;8位数据宽度同01                        |

\sphinxAtStartPar
11:仅字节顺序翻转(32位数据宽度 31:0 \sphinxhyphen{}\textgreater{}    | 7:0,15:8,23:16,31:24;16位数据宽度15:0 \sphinxhyphen{}\textgreater{}   | 7:0,15:8;8位数据宽度同00)                 |
\\
\sphinxhline
\sphinxAtStartPar
3
&
\sphinxAtStartPar
INOT
&
\sphinxAtStartPar
输入数据是否取反                            |

\sphinxAtStartPar
1:输入数据取反                             |

\sphinxAtStartPar
0:输入数据不取反                           |
\\
\sphinxhline
\sphinxAtStartPar
2:1
&
\sphinxAtStartPar
IREV
&
\sphinxAtStartPar
输入数据是否翻转。                          |

\sphinxAtStartPar
00:bit顺序不变                             |

\sphinxAtStartPar
01:bit顺序完全翻转(32位数据宽度31:0 \sphinxhyphen{}\textgreater{}    | 0:31;16位数据宽度15:0 \sphinxhyphen{}\textgreater{}                   | 0:15;8位数据宽度7:0 \sphinxhyphen{}\textgreater{} 0:7)               |

\sphinxAtStartPar
10:bit顺序在字节范围内翻转(32位数据宽度   | 31:0 \sphinxhyphen{}\textgreater{} 24:31, 16:23, 8:15, 0:7;16位数据宽度15:0 \sphinxhyphen{}\textgreater{} 8:15,              | 0:7;8位数据宽度同01                        |

\sphinxAtStartPar
11:仅字节顺序翻转(32位数据宽度 31:0 \sphinxhyphen{}\textgreater{}    | 7:0,15:8,23:16,31:24;16位数据宽度15:0 \sphinxhyphen{}\textgreater{}   | 7:0,15:8;8位数据宽度同00)                 |
\\
\sphinxhline
\sphinxAtStartPar
0
&
\sphinxAtStartPar
EN
&
\sphinxAtStartPar
CRC使能控制位                               |

\sphinxAtStartPar
1:CRC使能                                  |

\sphinxAtStartPar
0:CRC禁能                                  |
\\
\sphinxbottomrule
\end{tabular}
\sphinxtableafterendhook\par
\sphinxattableend\end{savenotes}


\subsubsection{数据输入寄存器DATAIN}
\label{\detokenize{SWM241/_u529f_u80fd_u63cf_u8ff0/CRC_u8ba1_u7b97_u5355_u5143:datain}}

\begin{savenotes}\sphinxattablestart
\sphinxthistablewithglobalstyle
\centering
\begin{tabular}[t]{\X{20}{100}\X{20}{100}\X{20}{100}\X{20}{100}\X{20}{100}}
\sphinxtoprule
\sphinxtableatstartofbodyhook
\sphinxAtStartPar
寄存器 |
&
\begin{DUlineblock}{0em}
\item[] 偏移 |
\end{DUlineblock}
&
\begin{DUlineblock}{0em}
\item[] 
\item[] {\color{red}\bfseries{}|}
\end{DUlineblock}
&
\sphinxAtStartPar
复位值 |    描 | |
&
\begin{DUlineblock}{0em}
\item[] |
  |
\end{DUlineblock}
\\
\sphinxhline
\sphinxAtStartPar
DATAIN
&
\sphinxAtStartPar
0x04
&&
\sphinxAtStartPar
0 000000
&
\sphinxAtStartPar
CRC数据输入寄存器          |
\\
\sphinxbottomrule
\end{tabular}
\sphinxtableafterendhook\par
\sphinxattableend\end{savenotes}


\begin{savenotes}\sphinxattablestart
\sphinxthistablewithglobalstyle
\centering
\begin{tabular}[t]{\X{12}{96}\X{12}{96}\X{12}{96}\X{12}{96}\X{12}{96}\X{12}{96}\X{12}{96}\X{12}{96}}
\sphinxtoprule
\sphinxtableatstartofbodyhook
\sphinxAtStartPar
31
&
\sphinxAtStartPar
30
&
\sphinxAtStartPar
29
&
\sphinxAtStartPar
28
&
\sphinxAtStartPar
27
&
\sphinxAtStartPar
26
&
\sphinxAtStartPar
25
&
\sphinxAtStartPar
24
\\
\sphinxhline
\sphinxAtStartPar
DATAIN
&&&&&&&\\
\sphinxhline
\sphinxAtStartPar
23
&
\sphinxAtStartPar
22
&
\sphinxAtStartPar
21
&
\sphinxAtStartPar
20
&
\sphinxAtStartPar
19
&
\sphinxAtStartPar
18
&
\sphinxAtStartPar
17
&
\sphinxAtStartPar
16
\\
\sphinxhline
\sphinxAtStartPar
DATAIN
&&&&&&&\\
\sphinxhline
\sphinxAtStartPar
15
&
\sphinxAtStartPar
14
&
\sphinxAtStartPar
13
&
\sphinxAtStartPar
12
&
\sphinxAtStartPar
11
&
\sphinxAtStartPar
10
&
\sphinxAtStartPar
9
&
\sphinxAtStartPar
8
\\
\sphinxhline
\sphinxAtStartPar
DATAIN
&&&&&&&\\
\sphinxhline
\sphinxAtStartPar
7
&
\sphinxAtStartPar
6
&
\sphinxAtStartPar
5
&
\sphinxAtStartPar
4
&
\sphinxAtStartPar
3
&
\sphinxAtStartPar
2
&
\sphinxAtStartPar
1
&
\sphinxAtStartPar
0
\\
\sphinxhline
\sphinxAtStartPar
DATAIN
&&&&&&&\\
\sphinxbottomrule
\end{tabular}
\sphinxtableafterendhook\par
\sphinxattableend\end{savenotes}


\begin{savenotes}\sphinxattablestart
\sphinxthistablewithglobalstyle
\centering
\begin{tabular}[t]{\X{33}{99}\X{33}{99}\X{33}{99}}
\sphinxtoprule
\sphinxtableatstartofbodyhook
\sphinxAtStartPar
位域 |
&
\sphinxAtStartPar
名称     | |
&
\sphinxAtStartPar
描述                                        | |
\\
\sphinxhline
\sphinxAtStartPar
31:0
&
\sphinxAtStartPar
DATAIN
&
\sphinxAtStartPar
C 据输入寄存器,有效位根据CR寄存器IBIT位选择 |
\\
\sphinxbottomrule
\end{tabular}
\sphinxtableafterendhook\par
\sphinxattableend\end{savenotes}


\subsubsection{初始值设置寄存器INIVAL}
\label{\detokenize{SWM241/_u529f_u80fd_u63cf_u8ff0/CRC_u8ba1_u7b97_u5355_u5143:inival}}

\begin{savenotes}\sphinxattablestart
\sphinxthistablewithglobalstyle
\centering
\begin{tabular}[t]{\X{20}{100}\X{20}{100}\X{20}{100}\X{20}{100}\X{20}{100}}
\sphinxtoprule
\sphinxtableatstartofbodyhook
\sphinxAtStartPar
寄存器 |
&
\begin{DUlineblock}{0em}
\item[] 偏移 |
\end{DUlineblock}
&
\begin{DUlineblock}{0em}
\item[] 
\item[] {\color{red}\bfseries{}|}
\end{DUlineblock}
&
\sphinxAtStartPar
复位值 |    描 | |
&
\begin{DUlineblock}{0em}
\item[] |
  |
\end{DUlineblock}
\\
\sphinxhline
\sphinxAtStartPar
INIVAL
&
\sphinxAtStartPar
0x08
&&
\sphinxAtStartPar
0 000000
&
\sphinxAtStartPar
CRC初始值设置寄存器        |
\\
\sphinxbottomrule
\end{tabular}
\sphinxtableafterendhook\par
\sphinxattableend\end{savenotes}


\begin{savenotes}\sphinxattablestart
\sphinxthistablewithglobalstyle
\centering
\begin{tabular}[t]{\X{12}{96}\X{12}{96}\X{12}{96}\X{12}{96}\X{12}{96}\X{12}{96}\X{12}{96}\X{12}{96}}
\sphinxtoprule
\sphinxtableatstartofbodyhook
\sphinxAtStartPar
31
&
\sphinxAtStartPar
30
&
\sphinxAtStartPar
29
&
\sphinxAtStartPar
28
&
\sphinxAtStartPar
27
&
\sphinxAtStartPar
26
&
\sphinxAtStartPar
25
&
\sphinxAtStartPar
24
\\
\sphinxhline
\sphinxAtStartPar
INIVAL
&&&&&&&\\
\sphinxhline
\sphinxAtStartPar
23
&
\sphinxAtStartPar
22
&
\sphinxAtStartPar
21
&
\sphinxAtStartPar
20
&
\sphinxAtStartPar
19
&
\sphinxAtStartPar
18
&
\sphinxAtStartPar
17
&
\sphinxAtStartPar
16
\\
\sphinxhline
\sphinxAtStartPar
INIVAL
&&&&&&&\\
\sphinxhline
\sphinxAtStartPar
15
&
\sphinxAtStartPar
14
&
\sphinxAtStartPar
13
&
\sphinxAtStartPar
12
&
\sphinxAtStartPar
11
&
\sphinxAtStartPar
10
&
\sphinxAtStartPar
9
&
\sphinxAtStartPar
8
\\
\sphinxhline
\sphinxAtStartPar
INIVAL
&&&&&&&\\
\sphinxhline
\sphinxAtStartPar
7
&
\sphinxAtStartPar
6
&
\sphinxAtStartPar
5
&
\sphinxAtStartPar
4
&
\sphinxAtStartPar
3
&
\sphinxAtStartPar
2
&
\sphinxAtStartPar
1
&
\sphinxAtStartPar
0
\\
\sphinxhline
\sphinxAtStartPar
INIVAL
&&&&&&&\\
\sphinxbottomrule
\end{tabular}
\sphinxtableafterendhook\par
\sphinxattableend\end{savenotes}


\begin{savenotes}\sphinxattablestart
\sphinxthistablewithglobalstyle
\centering
\begin{tabular}[t]{\X{33}{99}\X{33}{99}\X{33}{99}}
\sphinxtoprule
\sphinxtableatstartofbodyhook
\sphinxAtStartPar
位域 |
&
\sphinxAtStartPar
名称     | |
&
\sphinxAtStartPar
描述                                        | |
\\
\sphinxhline
\sphinxAtStartPar
31:0
&
\sphinxAtStartPar
INIVAL
&
\sphinxAtStartPar
CRC初始值寄存器                             |
\\
\sphinxbottomrule
\end{tabular}
\sphinxtableafterendhook\par
\sphinxattableend\end{savenotes}


\subsubsection{结果输出寄存器RESULT}
\label{\detokenize{SWM241/_u529f_u80fd_u63cf_u8ff0/CRC_u8ba1_u7b97_u5355_u5143:result}}

\begin{savenotes}\sphinxattablestart
\sphinxthistablewithglobalstyle
\centering
\begin{tabular}[t]{\X{20}{100}\X{20}{100}\X{20}{100}\X{20}{100}\X{20}{100}}
\sphinxtoprule
\sphinxtableatstartofbodyhook
\sphinxAtStartPar
寄存器 |
&
\begin{DUlineblock}{0em}
\item[] 偏移 |
\end{DUlineblock}
&
\begin{DUlineblock}{0em}
\item[] 
\item[] {\color{red}\bfseries{}|}
\end{DUlineblock}
&
\sphinxAtStartPar
复位值 |    描 | |
&
\begin{DUlineblock}{0em}
\item[] |
  |
\end{DUlineblock}
\\
\sphinxhline
\sphinxAtStartPar
RESULT
&
\sphinxAtStartPar
0x0C
&&
\sphinxAtStartPar
0 000000
&
\sphinxAtStartPar
CRC结果输出寄存器          |
\\
\sphinxbottomrule
\end{tabular}
\sphinxtableafterendhook\par
\sphinxattableend\end{savenotes}


\begin{savenotes}\sphinxattablestart
\sphinxthistablewithglobalstyle
\centering
\begin{tabular}[t]{\X{12}{96}\X{12}{96}\X{12}{96}\X{12}{96}\X{12}{96}\X{12}{96}\X{12}{96}\X{12}{96}}
\sphinxtoprule
\sphinxtableatstartofbodyhook
\sphinxAtStartPar
31
&
\sphinxAtStartPar
30
&
\sphinxAtStartPar
29
&
\sphinxAtStartPar
28
&
\sphinxAtStartPar
27
&
\sphinxAtStartPar
26
&
\sphinxAtStartPar
25
&
\sphinxAtStartPar
24
\\
\sphinxhline
\sphinxAtStartPar
RESULT
&&&&&&&\\
\sphinxhline
\sphinxAtStartPar
23
&
\sphinxAtStartPar
22
&
\sphinxAtStartPar
21
&
\sphinxAtStartPar
20
&
\sphinxAtStartPar
19
&
\sphinxAtStartPar
18
&
\sphinxAtStartPar
17
&
\sphinxAtStartPar
16
\\
\sphinxhline
\sphinxAtStartPar
RESULT
&&&&&&&\\
\sphinxhline
\sphinxAtStartPar
15
&
\sphinxAtStartPar
14
&
\sphinxAtStartPar
13
&
\sphinxAtStartPar
12
&
\sphinxAtStartPar
11
&
\sphinxAtStartPar
10
&
\sphinxAtStartPar
9
&
\sphinxAtStartPar
8
\\
\sphinxhline
\sphinxAtStartPar
RESULT
&&&&&&&\\
\sphinxhline
\sphinxAtStartPar
7
&
\sphinxAtStartPar
6
&
\sphinxAtStartPar
5
&
\sphinxAtStartPar
4
&
\sphinxAtStartPar
3
&
\sphinxAtStartPar
2
&
\sphinxAtStartPar
1
&
\sphinxAtStartPar
0
\\
\sphinxhline\begin{itemize}
\item {} 
\end{itemize}
&&&&&&&\\
\sphinxbottomrule
\end{tabular}
\sphinxtableafterendhook\par
\sphinxattableend\end{savenotes}


\begin{savenotes}\sphinxattablestart
\sphinxthistablewithglobalstyle
\centering
\begin{tabular}[t]{\X{33}{99}\X{33}{99}\X{33}{99}}
\sphinxtoprule
\sphinxtableatstartofbodyhook
\sphinxAtStartPar
位域 |
&
\sphinxAtStartPar
名称     | |
&
\sphinxAtStartPar
描述                                        | |
\\
\sphinxhline
\sphinxAtStartPar
31:0
&
\sphinxAtStartPar
RESULT
&
\sphinxAtStartPar
C 果输出寄存器,有效位根据CR寄存器POLY位选择 |

\sphinxAtStartPar
当POLY为:                                  |

\sphinxAtStartPar
00:低16位有效                              |

\sphinxAtStartPar
01:低8位有效                               |

\sphinxAtStartPar
10:低16位有效                              |

\sphinxAtStartPar
11:32位有效                                |
\\
\sphinxbottomrule
\end{tabular}
\sphinxtableafterendhook\par
\sphinxattableend\end{savenotes}

\sphinxstepscope


\section{除法器(DIV)}
\label{\detokenize{SWM241/_u529f_u80fd_u63cf_u8ff0/_u9664_u6cd5_u5668:div}}\label{\detokenize{SWM241/_u529f_u80fd_u63cf_u8ff0/_u9664_u6cd5_u5668::doc}}
\sphinxAtStartPar
概述
\textasciitilde{}\textasciitilde{}

\sphinxAtStartPar
SWM241系列所有型号除法器模块操作均相同。使用前需使能除法器模块时钟。

\sphinxAtStartPar
特性
\textasciitilde{}\textasciitilde{}
\begin{itemize}
\item {} 
\sphinxAtStartPar
支持32位整数除法运算及求余运算

\item {} 
\sphinxAtStartPar
支持32位开方运算,支持小数位

\item {} 
\sphinxAtStartPar
除法单次运算耗时32个时钟,不包括读写寄存器时间

\item {} 
\sphinxAtStartPar
开方单次运算耗时16/32个时钟,不包括读写寄存器时间

\item {} 
\sphinxAtStartPar
开方可选择两种模式
\begin{itemize}
\item {} 
\sphinxAtStartPar
只取整数(16位)

\item {} 
\sphinxAtStartPar
包含小数(16+16位)

\end{itemize}

\item {} 
\sphinxAtStartPar
运算启动自动清除运算使能查询

\item {} 
\sphinxAtStartPar
提供运算进行标志和完成标志

\item {} 
\sphinxAtStartPar
支持有符号数和无符号数运算

\end{itemize}


\subsection{功能描述}
\label{\detokenize{SWM241/_u529f_u80fd_u63cf_u8ff0/_u9664_u6cd5_u5668:id1}}
\sphinxAtStartPar
使用除法器模块计算商/余数流程如下:
\begin{itemize}
\item {} 
\sphinxAtStartPar
配置DIVIDEND寄存器和DIVISOR寄存器

\item {} 
\sphinxAtStartPar
配置CR寄存器。选择有符号数或无符号数,DIVGO启动运算

\item {} 
\sphinxAtStartPar
读取SR寄存器,查看运算进程
\begin{itemize}
\item {} 
\sphinxAtStartPar
DIVBUSY:运算标志

\item {} 
\sphinxAtStartPar
DIVEND:运算完成标志

\end{itemize}

\item {} 
\sphinxAtStartPar
读取QUO寄存器及余数寄存器REMAIN

\end{itemize}

\sphinxAtStartPar
注1:当除数为0时,商数为全1,余数为被除数;当被除数为0,商数为全0,余数为0

\sphinxAtStartPar
注2:计算过程中,不可更改除数及被除数

\sphinxAtStartPar
注3:如果除数为0,商数为全‘1’,余数为被除数

\sphinxAtStartPar
注4:如果被除数为0,商数为全‘0’,余数为0

\sphinxAtStartPar
使用除法器模块计算开方流程如下:
\begin{itemize}
\item {} 
\sphinxAtStartPar
配置RADICAND寄存器;

\item {} 
\sphinxAtStartPar
配置CR寄存器;ROOTMOD:运算模式;ROOTGO:启动运算;

\item {} 
\sphinxAtStartPar
读取SR寄存器;ROOTBUSY:开方运算标志;ROOTENDI:开方整数运算完成标志;ROOTENDF:开方小数运算完成标志;

\item {} 
\sphinxAtStartPar
读取ROOT寄存器;

\end{itemize}

\sphinxAtStartPar
注1:当选择只计算整数时, ROOT寄存器低16位仍保存最后一次的小数计算结果。

\sphinxAtStartPar
注2:计算过程中,不可更改被开方数。


\subsection{寄存器映射}
\label{\detokenize{SWM241/_u529f_u80fd_u63cf_u8ff0/_u9664_u6cd5_u5668:id2}}

\begin{savenotes}\sphinxattablestart
\sphinxthistablewithglobalstyle
\centering
\begin{tabular}[t]{\X{20}{100}\X{20}{100}\X{20}{100}\X{20}{100}\X{20}{100}}
\sphinxtoprule
\sphinxtableatstartofbodyhook
\sphinxAtStartPar
名称   |
&
\begin{DUlineblock}{0em}
\item[] 偏移 |
\end{DUlineblock}
&
\begin{DUlineblock}{0em}
\item[] 
\end{DUlineblock}
&
\begin{DUlineblock}{0em}
\item[] 
\end{DUlineblock}
&
\sphinxAtStartPar
描述                       | | | |
\\
\sphinxhline
\sphinxAtStartPar
DIVBASE:0x40003800
&&&&\\
\sphinxhline
\sphinxAtStartPar
CR
&
\sphinxAtStartPar
0x00
&&
\sphinxAtStartPar
0x 00000
&
\sphinxAtStartPar
控制寄存器                 |
\\
\sphinxhline
\sphinxAtStartPar
SR
&
\sphinxAtStartPar
0x04
&&
\sphinxAtStartPar
0x 00000
&
\sphinxAtStartPar
状态寄存器                 |
\\
\sphinxhline
\sphinxAtStartPar
DIVIDEND
&
\sphinxAtStartPar
0x10
&&
\sphinxAtStartPar
0x 00000
&
\sphinxAtStartPar
被除数寄存器               |
\\
\sphinxhline
\sphinxAtStartPar
DIVISOR
&
\sphinxAtStartPar
0x14
&&
\sphinxAtStartPar
0x 00000
&
\sphinxAtStartPar
除数寄存器                 |
\\
\sphinxhline
\sphinxAtStartPar
QUO
&
\sphinxAtStartPar
0x18
&&
\sphinxAtStartPar
0x 00000
&
\sphinxAtStartPar
商寄存器                   |
\\
\sphinxhline
\sphinxAtStartPar
REMAIN
&
\sphinxAtStartPar
0x1C
&&
\sphinxAtStartPar
0x 00000
&
\sphinxAtStartPar
余数寄存器                 |
\\
\sphinxhline
\sphinxAtStartPar
RADICAND
&
\sphinxAtStartPar
0x20
&&
\sphinxAtStartPar
0x 00000
&
\sphinxAtStartPar
开方数据寄存器             |
\\
\sphinxhline
\sphinxAtStartPar
ROOT
&
\sphinxAtStartPar
0x24
&&
\sphinxAtStartPar
0x 00000
&
\sphinxAtStartPar
平方根数据寄存器           |
\\
\sphinxbottomrule
\end{tabular}
\sphinxtableafterendhook\par
\sphinxattableend\end{savenotes}


\subsection{寄存器描述}
\label{\detokenize{SWM241/_u529f_u80fd_u63cf_u8ff0/_u9664_u6cd5_u5668:id3}}
\sphinxAtStartPar
控制寄存器CR


\begin{savenotes}\sphinxattablestart
\sphinxthistablewithglobalstyle
\centering
\begin{tabular}[t]{\X{20}{100}\X{20}{100}\X{20}{100}\X{20}{100}\X{20}{100}}
\sphinxtoprule
\sphinxtableatstartofbodyhook
\sphinxAtStartPar
寄存器 |
&
\begin{DUlineblock}{0em}
\item[] 偏移 |
\end{DUlineblock}
&
\begin{DUlineblock}{0em}
\item[] 
\end{DUlineblock}
&
\sphinxAtStartPar
复位值 |    描 | |
&
\begin{DUlineblock}{0em}
\item[] 
\end{DUlineblock}
\\
\sphinxhline
\sphinxAtStartPar
CR
&
\sphinxAtStartPar
0x00
&&
\sphinxAtStartPar
0 000000
&
\sphinxAtStartPar
控制寄存器                 |
\\
\sphinxbottomrule
\end{tabular}
\sphinxtableafterendhook\par
\sphinxattableend\end{savenotes}


\begin{savenotes}\sphinxattablestart
\sphinxthistablewithglobalstyle
\centering
\begin{tabular}[t]{\X{12}{96}\X{12}{96}\X{12}{96}\X{12}{96}\X{12}{96}\X{12}{96}\X{12}{96}\X{12}{96}}
\sphinxtoprule
\sphinxtableatstartofbodyhook
\sphinxAtStartPar
31
&
\sphinxAtStartPar
30
&
\sphinxAtStartPar
29
&
\sphinxAtStartPar
28
&
\sphinxAtStartPar
27
&
\sphinxAtStartPar
26
&
\sphinxAtStartPar
25
&
\sphinxAtStartPar
24
\\
\sphinxhline\begin{itemize}
\item {} 
\end{itemize}
&&&&&&&\\
\sphinxhline
\sphinxAtStartPar
23
&
\sphinxAtStartPar
22
&
\sphinxAtStartPar
21
&
\sphinxAtStartPar
20
&
\sphinxAtStartPar
19
&
\sphinxAtStartPar
18
&
\sphinxAtStartPar
17
&
\sphinxAtStartPar
16
\\
\sphinxhline\begin{itemize}
\item {} 
\end{itemize}
&&&&&&&\\
\sphinxhline
\sphinxAtStartPar
15
&
\sphinxAtStartPar
14
&
\sphinxAtStartPar
13
&
\sphinxAtStartPar
12
&
\sphinxAtStartPar
11
&
\sphinxAtStartPar
10
&
\sphinxAtStartPar
9
&
\sphinxAtStartPar
8
\\
\sphinxhline\begin{itemize}
\item {} 
\end{itemize}
&&&&&&
\sphinxAtStartPar
R OOTMOD
&
\sphinxAtStartPar
ROOTGO
\\
\sphinxhline
\sphinxAtStartPar
7
&
\sphinxAtStartPar
6
&
\sphinxAtStartPar
5
&
\sphinxAtStartPar
4
&
\sphinxAtStartPar
3
&
\sphinxAtStartPar
2
&
\sphinxAtStartPar
1
&
\sphinxAtStartPar
0
\\
\sphinxhline\begin{itemize}
\item {} 
\end{itemize}
&&&&&&
\sphinxAtStartPar
D IVSIGN
&
\sphinxAtStartPar
DIVGO
\\
\sphinxbottomrule
\end{tabular}
\sphinxtableafterendhook\par
\sphinxattableend\end{savenotes}


\begin{savenotes}\sphinxattablestart
\sphinxthistablewithglobalstyle
\centering
\begin{tabular}[t]{\X{33}{99}\X{33}{99}\X{33}{99}}
\sphinxtoprule
\sphinxtableatstartofbodyhook
\sphinxAtStartPar
位域 |
&
\sphinxAtStartPar
名称     | |
&
\sphinxAtStartPar
描述                                        | |
\\
\sphinxhline
\sphinxAtStartPar
31:10
&\begin{itemize}
\item {} 
\end{itemize}
&\begin{itemize}
\item {} 
\end{itemize}
\\
\sphinxhline
\sphinxAtStartPar
9
&
\sphinxAtStartPar
ROOTMOD
&
\sphinxAtStartPar
开方运算模式                                |

\sphinxAtStartPar
0:开方运算结果只保留16位整数;             |

\sphinxAtStartPar
1:开方运算结果保留16位整数+16位小数;      |
\\
\sphinxhline
\sphinxAtStartPar
8
&
\sphinxAtStartPar
ROOTGO
&
\sphinxAtStartPar
开方运算启动信号                            |

\sphinxAtStartPar
1:启动                                     |

\sphinxAtStartPar
0:停止                                     |

\sphinxAtStartPar
运算完成后硬件会自动清零。                  |
\\
\sphinxhline
\sphinxAtStartPar
7:2
&\begin{itemize}
\item {} 
\end{itemize}
&\begin{itemize}
\item {} 
\end{itemize}
\\
\sphinxhline
\sphinxAtStartPar
1
&
\sphinxAtStartPar
DIVSIGN
&
\sphinxAtStartPar
0:表示有符号数                             |

\sphinxAtStartPar
1:表示无符号数                             |

\sphinxAtStartPar
注:当为有符号数时,                        | 最高位(31bit)表示符号,有效数据为31位  |

\sphinxAtStartPar
当为无符号数时,32bit数据都是有效数据       |
\\
\sphinxhline
\sphinxAtStartPar
0
&
\sphinxAtStartPar
DIVGO
&
\sphinxAtStartPar
除法运算启动信号                            |

\sphinxAtStartPar
1:启动                                     |

\sphinxAtStartPar
0:停止                                     |

\sphinxAtStartPar
运算完成后硬件会自动清零。                  |
\\
\sphinxbottomrule
\end{tabular}
\sphinxtableafterendhook\par
\sphinxattableend\end{savenotes}


\subsubsection{状态寄存器SR}
\label{\detokenize{SWM241/_u529f_u80fd_u63cf_u8ff0/_u9664_u6cd5_u5668:sr}}

\begin{savenotes}\sphinxattablestart
\sphinxthistablewithglobalstyle
\centering
\begin{tabular}[t]{\X{20}{100}\X{20}{100}\X{20}{100}\X{20}{100}\X{20}{100}}
\sphinxtoprule
\sphinxtableatstartofbodyhook
\sphinxAtStartPar
寄存器 |
&
\begin{DUlineblock}{0em}
\item[] 偏移 |
\end{DUlineblock}
&
\begin{DUlineblock}{0em}
\item[] 
\end{DUlineblock}
&
\sphinxAtStartPar
复位值 |    描 | |
&
\begin{DUlineblock}{0em}
\item[] 
\end{DUlineblock}
\\
\sphinxhline
\sphinxAtStartPar
SR
&
\sphinxAtStartPar
0x04
&&
\sphinxAtStartPar
0 000000
&
\sphinxAtStartPar
状态寄存器                 |
\\
\sphinxbottomrule
\end{tabular}
\sphinxtableafterendhook\par
\sphinxattableend\end{savenotes}


\begin{savenotes}\sphinxattablestart
\sphinxthistablewithglobalstyle
\centering
\begin{tabular}[t]{\X{12}{96}\X{12}{96}\X{12}{96}\X{12}{96}\X{12}{96}\X{12}{96}\X{12}{96}\X{12}{96}}
\sphinxtoprule
\sphinxtableatstartofbodyhook
\sphinxAtStartPar
31
&
\sphinxAtStartPar
30
&
\sphinxAtStartPar
29
&
\sphinxAtStartPar
28
&
\sphinxAtStartPar
27
&
\sphinxAtStartPar
26
&
\sphinxAtStartPar
25
&
\sphinxAtStartPar
24
\\
\sphinxhline\begin{itemize}
\item {} 
\end{itemize}
&&&&&&&\\
\sphinxhline
\sphinxAtStartPar
23
&
\sphinxAtStartPar
22
&
\sphinxAtStartPar
21
&
\sphinxAtStartPar
20
&
\sphinxAtStartPar
19
&
\sphinxAtStartPar
18
&
\sphinxAtStartPar
17
&
\sphinxAtStartPar
16
\\
\sphinxhline\begin{itemize}
\item {} 
\end{itemize}
&&&&&&&\\
\sphinxhline
\sphinxAtStartPar
15
&
\sphinxAtStartPar
14
&
\sphinxAtStartPar
13
&
\sphinxAtStartPar
12
&
\sphinxAtStartPar
11
&
\sphinxAtStartPar
10
&
\sphinxAtStartPar
9
&
\sphinxAtStartPar
8
\\
\sphinxhline\begin{itemize}
\item {} 
\end{itemize}
&&&&&
\sphinxAtStartPar
RO OTBUSY
&
\sphinxAtStartPar
ROO TENDF
&
\sphinxAtStartPar
RO OTENDI
\\
\sphinxhline
\sphinxAtStartPar
7
&
\sphinxAtStartPar
6
&
\sphinxAtStartPar
5
&
\sphinxAtStartPar
4
&
\sphinxAtStartPar
3
&
\sphinxAtStartPar
2
&
\sphinxAtStartPar
1
&
\sphinxAtStartPar
0
\\
\sphinxhline\begin{itemize}
\item {} 
\end{itemize}
&&&&&&
\sphinxAtStartPar
DI VBUSY
&
\sphinxAtStartPar
DIVEND
\\
\sphinxbottomrule
\end{tabular}
\sphinxtableafterendhook\par
\sphinxattableend\end{savenotes}


\begin{savenotes}\sphinxattablestart
\sphinxthistablewithglobalstyle
\centering
\begin{tabular}[t]{\X{33}{99}\X{33}{99}\X{33}{99}}
\sphinxtoprule
\sphinxtableatstartofbodyhook
\sphinxAtStartPar
位域 |
&
\sphinxAtStartPar
名称     | |
&
\sphinxAtStartPar
描述                                        | |
\\
\sphinxhline
\sphinxAtStartPar
31:11
&\begin{itemize}
\item {} 
\end{itemize}
&\begin{itemize}
\item {} 
\end{itemize}
\\
\sphinxhline
\sphinxAtStartPar
10
&
\sphinxAtStartPar
ROOTBUSY
&
\sphinxAtStartPar
开方运算过程标志。                          |

\sphinxAtStartPar
1:运算中                                   |

\sphinxAtStartPar
0:运算完成                                 |

\sphinxAtStartPar
RO

\sphinxAtStartPar
运算完成后硬件自动清零。                    |
\\
\sphinxhline
\sphinxAtStartPar
9
&
\sphinxAtStartPar
ROOTENDF
&
\sphinxAtStartPar
开方小数运算完成标志。                      |

\sphinxAtStartPar
1:运算完成                                 |

\sphinxAtStartPar
0:运算未完成                               |

\sphinxAtStartPar
写1清除。                                   |
\\
\sphinxhline
\sphinxAtStartPar
8
&
\sphinxAtStartPar
ROOTENDI
&
\sphinxAtStartPar
开方整数运算完成标志。                      |

\sphinxAtStartPar
1:运算完成                                 |

\sphinxAtStartPar
0:运算未完成                               |

\sphinxAtStartPar
写1清除。                                   |
\\
\sphinxhline
\sphinxAtStartPar
7:2
&\begin{itemize}
\item {} 
\end{itemize}
&\begin{itemize}
\item {} 
\end{itemize}
\\
\sphinxhline
\sphinxAtStartPar
1
&
\sphinxAtStartPar
DIVBUSY
&
\sphinxAtStartPar
除法运算过程标志。                          |

\sphinxAtStartPar
1:运算中                                   |

\sphinxAtStartPar
0:运算完成                                 |

\sphinxAtStartPar
RO

\sphinxAtStartPar
运算完成后硬件自动清零                      |
\\
\sphinxhline
\sphinxAtStartPar
0
&
\sphinxAtStartPar
DIVEND
&
\sphinxAtStartPar
除法运算完成标志。                          |

\sphinxAtStartPar
1:运算完成                                 |

\sphinxAtStartPar
0:运算未完成                               |

\sphinxAtStartPar
写1清除。                                   |
\\
\sphinxbottomrule
\end{tabular}
\sphinxtableafterendhook\par
\sphinxattableend\end{savenotes}


\subsubsection{被除数寄存器DIVIDEND}
\label{\detokenize{SWM241/_u529f_u80fd_u63cf_u8ff0/_u9664_u6cd5_u5668:dividend}}

\begin{savenotes}\sphinxattablestart
\sphinxthistablewithglobalstyle
\centering
\begin{tabular}[t]{\X{20}{100}\X{20}{100}\X{20}{100}\X{20}{100}\X{20}{100}}
\sphinxtoprule
\sphinxtableatstartofbodyhook
\sphinxAtStartPar
寄存器 |
&
\begin{DUlineblock}{0em}
\item[] 偏移 |
\end{DUlineblock}
&
\begin{DUlineblock}{0em}
\item[] 
\end{DUlineblock}
&
\sphinxAtStartPar
复位值 |    描 | |
&
\begin{DUlineblock}{0em}
\item[] 
\end{DUlineblock}
\\
\sphinxhline
\sphinxAtStartPar
DIVIDEND
&
\sphinxAtStartPar
0x10
&&
\sphinxAtStartPar
0 000000
&
\sphinxAtStartPar
被除数寄存器               |
\\
\sphinxbottomrule
\end{tabular}
\sphinxtableafterendhook\par
\sphinxattableend\end{savenotes}


\begin{savenotes}\sphinxattablestart
\sphinxthistablewithglobalstyle
\centering
\begin{tabular}[t]{\X{12}{96}\X{12}{96}\X{12}{96}\X{12}{96}\X{12}{96}\X{12}{96}\X{12}{96}\X{12}{96}}
\sphinxtoprule
\sphinxtableatstartofbodyhook
\sphinxAtStartPar
31
&
\sphinxAtStartPar
30
&
\sphinxAtStartPar
29
&
\sphinxAtStartPar
28
&
\sphinxAtStartPar
27
&
\sphinxAtStartPar
26
&
\sphinxAtStartPar
25
&
\sphinxAtStartPar
24
\\
\sphinxhline
\sphinxAtStartPar
DIVIDEND
&&&&&&&\\
\sphinxhline
\sphinxAtStartPar
23
&
\sphinxAtStartPar
22
&
\sphinxAtStartPar
21
&
\sphinxAtStartPar
20
&
\sphinxAtStartPar
19
&
\sphinxAtStartPar
18
&
\sphinxAtStartPar
17
&
\sphinxAtStartPar
16
\\
\sphinxhline
\sphinxAtStartPar
DIVIDEND
&&&&&&&\\
\sphinxhline
\sphinxAtStartPar
15
&
\sphinxAtStartPar
14
&
\sphinxAtStartPar
13
&
\sphinxAtStartPar
12
&
\sphinxAtStartPar
11
&
\sphinxAtStartPar
10
&
\sphinxAtStartPar
9
&
\sphinxAtStartPar
8
\\
\sphinxhline
\sphinxAtStartPar
DIVIDEND
&&&&&&&\\
\sphinxhline
\sphinxAtStartPar
7
&
\sphinxAtStartPar
6
&
\sphinxAtStartPar
5
&
\sphinxAtStartPar
4
&
\sphinxAtStartPar
3
&
\sphinxAtStartPar
2
&
\sphinxAtStartPar
1
&
\sphinxAtStartPar
0
\\
\sphinxhline
\sphinxAtStartPar
DIVIDEND
&&&&&&&\\
\sphinxbottomrule
\end{tabular}
\sphinxtableafterendhook\par
\sphinxattableend\end{savenotes}


\begin{savenotes}\sphinxattablestart
\sphinxthistablewithglobalstyle
\centering
\begin{tabular}[t]{\X{33}{99}\X{33}{99}\X{33}{99}}
\sphinxtoprule
\sphinxtableatstartofbodyhook
\sphinxAtStartPar
位域 |
&
\sphinxAtStartPar
名称     | |
&
\sphinxAtStartPar
描述                                        | |
\\
\sphinxhline
\sphinxAtStartPar
31:0
&
\sphinxAtStartPar
DIVIDEND
&
\sphinxAtStartPar
被除数                                      |
\\
\sphinxbottomrule
\end{tabular}
\sphinxtableafterendhook\par
\sphinxattableend\end{savenotes}


\subsubsection{除数寄存器DIVISOR}
\label{\detokenize{SWM241/_u529f_u80fd_u63cf_u8ff0/_u9664_u6cd5_u5668:divisor}}

\begin{savenotes}\sphinxattablestart
\sphinxthistablewithglobalstyle
\centering
\begin{tabular}[t]{\X{20}{100}\X{20}{100}\X{20}{100}\X{20}{100}\X{20}{100}}
\sphinxtoprule
\sphinxtableatstartofbodyhook
\sphinxAtStartPar
寄存器 |
&
\begin{DUlineblock}{0em}
\item[] 偏移 |
\end{DUlineblock}
&
\begin{DUlineblock}{0em}
\item[] 
\end{DUlineblock}
&
\sphinxAtStartPar
复位值 |    描 | |
&
\begin{DUlineblock}{0em}
\item[] 
\end{DUlineblock}
\\
\sphinxhline
\sphinxAtStartPar
DIVISOR
&
\sphinxAtStartPar
0x14
&&
\sphinxAtStartPar
0 000000
&
\sphinxAtStartPar
除数寄存器                 |
\\
\sphinxbottomrule
\end{tabular}
\sphinxtableafterendhook\par
\sphinxattableend\end{savenotes}


\begin{savenotes}\sphinxattablestart
\sphinxthistablewithglobalstyle
\centering
\begin{tabular}[t]{\X{12}{96}\X{12}{96}\X{12}{96}\X{12}{96}\X{12}{96}\X{12}{96}\X{12}{96}\X{12}{96}}
\sphinxtoprule
\sphinxtableatstartofbodyhook
\sphinxAtStartPar
31
&
\sphinxAtStartPar
30
&
\sphinxAtStartPar
29
&
\sphinxAtStartPar
28
&
\sphinxAtStartPar
27
&
\sphinxAtStartPar
26
&
\sphinxAtStartPar
25
&
\sphinxAtStartPar
24
\\
\sphinxhline
\sphinxAtStartPar
DIVISOR
&&&&&&&\\
\sphinxhline
\sphinxAtStartPar
23
&
\sphinxAtStartPar
22
&
\sphinxAtStartPar
21
&
\sphinxAtStartPar
20
&
\sphinxAtStartPar
19
&
\sphinxAtStartPar
18
&
\sphinxAtStartPar
17
&
\sphinxAtStartPar
16
\\
\sphinxhline
\sphinxAtStartPar
DIVISOR
&&&&&&&\\
\sphinxhline
\sphinxAtStartPar
15
&
\sphinxAtStartPar
14
&
\sphinxAtStartPar
13
&
\sphinxAtStartPar
12
&
\sphinxAtStartPar
11
&
\sphinxAtStartPar
10
&
\sphinxAtStartPar
9
&
\sphinxAtStartPar
8
\\
\sphinxhline
\sphinxAtStartPar
DIVISOR
&&&&&&&\\
\sphinxhline
\sphinxAtStartPar
7
&
\sphinxAtStartPar
6
&
\sphinxAtStartPar
5
&
\sphinxAtStartPar
4
&
\sphinxAtStartPar
3
&
\sphinxAtStartPar
2
&
\sphinxAtStartPar
1
&
\sphinxAtStartPar
0
\\
\sphinxhline
\sphinxAtStartPar
DIVISOR
&&&&&&&\\
\sphinxbottomrule
\end{tabular}
\sphinxtableafterendhook\par
\sphinxattableend\end{savenotes}


\begin{savenotes}\sphinxattablestart
\sphinxthistablewithglobalstyle
\centering
\begin{tabular}[t]{\X{33}{99}\X{33}{99}\X{33}{99}}
\sphinxtoprule
\sphinxtableatstartofbodyhook
\sphinxAtStartPar
位域 |
&
\sphinxAtStartPar
名称     | |
&
\sphinxAtStartPar
描述                                        | |
\\
\sphinxhline
\sphinxAtStartPar
31:0
&
\sphinxAtStartPar
DIVISOR
&
\sphinxAtStartPar
除数                                        |
\\
\sphinxbottomrule
\end{tabular}
\sphinxtableafterendhook\par
\sphinxattableend\end{savenotes}


\subsubsection{商寄存器QUO}
\label{\detokenize{SWM241/_u529f_u80fd_u63cf_u8ff0/_u9664_u6cd5_u5668:quo}}

\begin{savenotes}\sphinxattablestart
\sphinxthistablewithglobalstyle
\centering
\begin{tabular}[t]{\X{20}{100}\X{20}{100}\X{20}{100}\X{20}{100}\X{20}{100}}
\sphinxtoprule
\sphinxtableatstartofbodyhook
\sphinxAtStartPar
寄存器 |
&
\begin{DUlineblock}{0em}
\item[] 偏移 |
\end{DUlineblock}
&
\begin{DUlineblock}{0em}
\item[] 
\end{DUlineblock}
&
\sphinxAtStartPar
复位值 |    描 | |
&
\begin{DUlineblock}{0em}
\item[] 
\end{DUlineblock}
\\
\sphinxhline
\sphinxAtStartPar
QUO
&
\sphinxAtStartPar
0x18
&&
\sphinxAtStartPar
0 000000
&
\sphinxAtStartPar
商寄存器                   |
\\
\sphinxbottomrule
\end{tabular}
\sphinxtableafterendhook\par
\sphinxattableend\end{savenotes}


\begin{savenotes}\sphinxattablestart
\sphinxthistablewithglobalstyle
\centering
\begin{tabular}[t]{\X{12}{96}\X{12}{96}\X{12}{96}\X{12}{96}\X{12}{96}\X{12}{96}\X{12}{96}\X{12}{96}}
\sphinxtoprule
\sphinxtableatstartofbodyhook
\sphinxAtStartPar
31
&
\sphinxAtStartPar
30
&
\sphinxAtStartPar
29
&
\sphinxAtStartPar
28
&
\sphinxAtStartPar
27
&
\sphinxAtStartPar
26
&
\sphinxAtStartPar
25
&
\sphinxAtStartPar
24
\\
\sphinxhline
\sphinxAtStartPar
QUO
&&&&&&&\\
\sphinxhline
\sphinxAtStartPar
23
&
\sphinxAtStartPar
22
&
\sphinxAtStartPar
21
&
\sphinxAtStartPar
20
&
\sphinxAtStartPar
19
&
\sphinxAtStartPar
18
&
\sphinxAtStartPar
17
&
\sphinxAtStartPar
16
\\
\sphinxhline
\sphinxAtStartPar
QUO
&&&&&&&\\
\sphinxhline
\sphinxAtStartPar
15
&
\sphinxAtStartPar
14
&
\sphinxAtStartPar
13
&
\sphinxAtStartPar
12
&
\sphinxAtStartPar
11
&
\sphinxAtStartPar
10
&
\sphinxAtStartPar
9
&
\sphinxAtStartPar
8
\\
\sphinxhline
\sphinxAtStartPar
QUO
&&&&&&&\\
\sphinxhline
\sphinxAtStartPar
7
&
\sphinxAtStartPar
6
&
\sphinxAtStartPar
5
&
\sphinxAtStartPar
4
&
\sphinxAtStartPar
3
&
\sphinxAtStartPar
2
&
\sphinxAtStartPar
1
&
\sphinxAtStartPar
0
\\
\sphinxhline
\sphinxAtStartPar
QUO
&&&&&&&\\
\sphinxbottomrule
\end{tabular}
\sphinxtableafterendhook\par
\sphinxattableend\end{savenotes}


\begin{savenotes}\sphinxattablestart
\sphinxthistablewithglobalstyle
\centering
\begin{tabular}[t]{\X{33}{99}\X{33}{99}\X{33}{99}}
\sphinxtoprule
\sphinxtableatstartofbodyhook
\sphinxAtStartPar
位域 |
&
\sphinxAtStartPar
名称     | |
&
\sphinxAtStartPar
描述                                        | |
\\
\sphinxhline
\sphinxAtStartPar
31:0
&
\sphinxAtStartPar
QUO
&
\sphinxAtStartPar
商数                                        |
\\
\sphinxbottomrule
\end{tabular}
\sphinxtableafterendhook\par
\sphinxattableend\end{savenotes}


\subsubsection{余数寄存器REMAIN}
\label{\detokenize{SWM241/_u529f_u80fd_u63cf_u8ff0/_u9664_u6cd5_u5668:remain}}

\begin{savenotes}\sphinxattablestart
\sphinxthistablewithglobalstyle
\centering
\begin{tabular}[t]{\X{20}{100}\X{20}{100}\X{20}{100}\X{20}{100}\X{20}{100}}
\sphinxtoprule
\sphinxtableatstartofbodyhook
\sphinxAtStartPar
寄存器 |
&
\begin{DUlineblock}{0em}
\item[] 偏移 |
\end{DUlineblock}
&
\begin{DUlineblock}{0em}
\item[] 
\end{DUlineblock}
&
\sphinxAtStartPar
复位值 |    描 | |
&
\begin{DUlineblock}{0em}
\item[] 
\end{DUlineblock}
\\
\sphinxhline
\sphinxAtStartPar
REMAIN
&
\sphinxAtStartPar
0x1C
&&
\sphinxAtStartPar
0 000000
&
\sphinxAtStartPar
余数寄存器                 |
\\
\sphinxbottomrule
\end{tabular}
\sphinxtableafterendhook\par
\sphinxattableend\end{savenotes}


\begin{savenotes}\sphinxattablestart
\sphinxthistablewithglobalstyle
\centering
\begin{tabular}[t]{\X{12}{96}\X{12}{96}\X{12}{96}\X{12}{96}\X{12}{96}\X{12}{96}\X{12}{96}\X{12}{96}}
\sphinxtoprule
\sphinxtableatstartofbodyhook
\sphinxAtStartPar
31
&
\sphinxAtStartPar
30
&
\sphinxAtStartPar
29
&
\sphinxAtStartPar
28
&
\sphinxAtStartPar
27
&
\sphinxAtStartPar
26
&
\sphinxAtStartPar
25
&
\sphinxAtStartPar
24
\\
\sphinxhline
\sphinxAtStartPar
REMAIN
&&&&&&&\\
\sphinxhline
\sphinxAtStartPar
23
&
\sphinxAtStartPar
22
&
\sphinxAtStartPar
21
&
\sphinxAtStartPar
20
&
\sphinxAtStartPar
19
&
\sphinxAtStartPar
18
&
\sphinxAtStartPar
17
&
\sphinxAtStartPar
16
\\
\sphinxhline
\sphinxAtStartPar
REMAIN
&&&&&&&\\
\sphinxhline
\sphinxAtStartPar
15
&
\sphinxAtStartPar
14
&
\sphinxAtStartPar
13
&
\sphinxAtStartPar
12
&
\sphinxAtStartPar
11
&
\sphinxAtStartPar
10
&
\sphinxAtStartPar
9
&
\sphinxAtStartPar
8
\\
\sphinxhline
\sphinxAtStartPar
REMAIN
&&&&&&&\\
\sphinxhline
\sphinxAtStartPar
7
&
\sphinxAtStartPar
6
&
\sphinxAtStartPar
5
&
\sphinxAtStartPar
4
&
\sphinxAtStartPar
3
&
\sphinxAtStartPar
2
&
\sphinxAtStartPar
1
&
\sphinxAtStartPar
0
\\
\sphinxhline
\sphinxAtStartPar
REMAIN
&&&&&&&\\
\sphinxbottomrule
\end{tabular}
\sphinxtableafterendhook\par
\sphinxattableend\end{savenotes}


\begin{savenotes}\sphinxattablestart
\sphinxthistablewithglobalstyle
\centering
\begin{tabular}[t]{\X{33}{99}\X{33}{99}\X{33}{99}}
\sphinxtoprule
\sphinxtableatstartofbodyhook
\sphinxAtStartPar
位域 |
&
\sphinxAtStartPar
名称     | |
&
\sphinxAtStartPar
描述                                        | |
\\
\sphinxhline
\sphinxAtStartPar
31:0
&
\sphinxAtStartPar
REMAIN
&
\sphinxAtStartPar
余数                                        |
\\
\sphinxbottomrule
\end{tabular}
\sphinxtableafterendhook\par
\sphinxattableend\end{savenotes}


\subsubsection{平方数据寄存器RADICAND}
\label{\detokenize{SWM241/_u529f_u80fd_u63cf_u8ff0/_u9664_u6cd5_u5668:radicand}}

\begin{savenotes}\sphinxattablestart
\sphinxthistablewithglobalstyle
\centering
\begin{tabular}[t]{\X{20}{100}\X{20}{100}\X{20}{100}\X{20}{100}\X{20}{100}}
\sphinxtoprule
\sphinxtableatstartofbodyhook
\sphinxAtStartPar
寄存器 |
&
\begin{DUlineblock}{0em}
\item[] 偏移 |
\end{DUlineblock}
&
\begin{DUlineblock}{0em}
\item[] 
\end{DUlineblock}
&
\sphinxAtStartPar
复位值 |    描 | |
&
\begin{DUlineblock}{0em}
\item[] 
\end{DUlineblock}
\\
\sphinxhline
\sphinxAtStartPar
RADICAND
&
\sphinxAtStartPar
0x20
&&
\sphinxAtStartPar
0 000000
&
\sphinxAtStartPar
平方数据寄存器             |
\\
\sphinxbottomrule
\end{tabular}
\sphinxtableafterendhook\par
\sphinxattableend\end{savenotes}


\begin{savenotes}\sphinxattablestart
\sphinxthistablewithglobalstyle
\centering
\begin{tabular}[t]{\X{12}{96}\X{12}{96}\X{12}{96}\X{12}{96}\X{12}{96}\X{12}{96}\X{12}{96}\X{12}{96}}
\sphinxtoprule
\sphinxtableatstartofbodyhook
\sphinxAtStartPar
31
&
\sphinxAtStartPar
30
&
\sphinxAtStartPar
29
&
\sphinxAtStartPar
28
&
\sphinxAtStartPar
27
&
\sphinxAtStartPar
26
&
\sphinxAtStartPar
25
&
\sphinxAtStartPar
24
\\
\sphinxhline
\sphinxAtStartPar
RADICAND
&&&&&&&\\
\sphinxhline
\sphinxAtStartPar
23
&
\sphinxAtStartPar
22
&
\sphinxAtStartPar
21
&
\sphinxAtStartPar
20
&
\sphinxAtStartPar
19
&
\sphinxAtStartPar
18
&
\sphinxAtStartPar
17
&
\sphinxAtStartPar
16
\\
\sphinxhline
\sphinxAtStartPar
RADICAND
&&&&&&&\\
\sphinxhline
\sphinxAtStartPar
15
&
\sphinxAtStartPar
14
&
\sphinxAtStartPar
13
&
\sphinxAtStartPar
12
&
\sphinxAtStartPar
11
&
\sphinxAtStartPar
10
&
\sphinxAtStartPar
9
&
\sphinxAtStartPar
8
\\
\sphinxhline
\sphinxAtStartPar
RADICAND
&&&&&&&\\
\sphinxhline
\sphinxAtStartPar
7
&
\sphinxAtStartPar
6
&
\sphinxAtStartPar
5
&
\sphinxAtStartPar
4
&
\sphinxAtStartPar
3
&
\sphinxAtStartPar
2
&
\sphinxAtStartPar
1
&
\sphinxAtStartPar
0
\\
\sphinxhline
\sphinxAtStartPar
RADICAND
&&&&&&&\\
\sphinxbottomrule
\end{tabular}
\sphinxtableafterendhook\par
\sphinxattableend\end{savenotes}


\begin{savenotes}\sphinxattablestart
\sphinxthistablewithglobalstyle
\centering
\begin{tabular}[t]{\X{33}{99}\X{33}{99}\X{33}{99}}
\sphinxtoprule
\sphinxtableatstartofbodyhook
\sphinxAtStartPar
位域 |
&
\sphinxAtStartPar
名称     | |
&
\sphinxAtStartPar
描述                                        | |
\\
\sphinxhline
\sphinxAtStartPar
31:0
&
\sphinxAtStartPar
RADICAND
&
\sphinxAtStartPar
平方数据                                    |
\\
\sphinxbottomrule
\end{tabular}
\sphinxtableafterendhook\par
\sphinxattableend\end{savenotes}


\subsubsection{平方根寄存器ROOT}
\label{\detokenize{SWM241/_u529f_u80fd_u63cf_u8ff0/_u9664_u6cd5_u5668:root}}

\begin{savenotes}\sphinxattablestart
\sphinxthistablewithglobalstyle
\centering
\begin{tabular}[t]{\X{20}{100}\X{20}{100}\X{20}{100}\X{20}{100}\X{20}{100}}
\sphinxtoprule
\sphinxtableatstartofbodyhook
\sphinxAtStartPar
寄存器 |
&
\begin{DUlineblock}{0em}
\item[] 偏移 |
\end{DUlineblock}
&
\begin{DUlineblock}{0em}
\item[] 
\end{DUlineblock}
&
\sphinxAtStartPar
复位值 |    描 | |
&
\begin{DUlineblock}{0em}
\item[] 
\end{DUlineblock}
\\
\sphinxhline
\sphinxAtStartPar
ROOT
&
\sphinxAtStartPar
0x24
&&
\sphinxAtStartPar
0 000000
&
\sphinxAtStartPar
平方根数据寄存器           |
\\
\sphinxbottomrule
\end{tabular}
\sphinxtableafterendhook\par
\sphinxattableend\end{savenotes}


\begin{savenotes}\sphinxattablestart
\sphinxthistablewithglobalstyle
\centering
\begin{tabular}[t]{\X{12}{96}\X{12}{96}\X{12}{96}\X{12}{96}\X{12}{96}\X{12}{96}\X{12}{96}\X{12}{96}}
\sphinxtoprule
\sphinxtableatstartofbodyhook
\sphinxAtStartPar
31
&
\sphinxAtStartPar
30
&
\sphinxAtStartPar
29
&
\sphinxAtStartPar
28
&
\sphinxAtStartPar
27
&
\sphinxAtStartPar
26
&
\sphinxAtStartPar
25
&
\sphinxAtStartPar
24
\\
\sphinxhline
\sphinxAtStartPar
ROOTI
&&&&&&&\\
\sphinxhline
\sphinxAtStartPar
23
&
\sphinxAtStartPar
22
&
\sphinxAtStartPar
21
&
\sphinxAtStartPar
20
&
\sphinxAtStartPar
19
&
\sphinxAtStartPar
18
&
\sphinxAtStartPar
17
&
\sphinxAtStartPar
16
\\
\sphinxhline
\sphinxAtStartPar
ROOTI
&&&&&&&\\
\sphinxhline
\sphinxAtStartPar
15
&
\sphinxAtStartPar
14
&
\sphinxAtStartPar
13
&
\sphinxAtStartPar
12
&
\sphinxAtStartPar
11
&
\sphinxAtStartPar
10
&
\sphinxAtStartPar
9
&
\sphinxAtStartPar
8
\\
\sphinxhline
\sphinxAtStartPar
ROOTF
&&&&&&&\\
\sphinxhline
\sphinxAtStartPar
7
&
\sphinxAtStartPar
6
&
\sphinxAtStartPar
5
&
\sphinxAtStartPar
4
&
\sphinxAtStartPar
3
&
\sphinxAtStartPar
2
&
\sphinxAtStartPar
1
&
\sphinxAtStartPar
0
\\
\sphinxhline
\sphinxAtStartPar
ROOTF
&&&&&&&\\
\sphinxbottomrule
\end{tabular}
\sphinxtableafterendhook\par
\sphinxattableend\end{savenotes}


\begin{savenotes}\sphinxattablestart
\sphinxthistablewithglobalstyle
\centering
\begin{tabular}[t]{\X{33}{99}\X{33}{99}\X{33}{99}}
\sphinxtoprule
\sphinxtableatstartofbodyhook
\sphinxAtStartPar
位域 |
&
\sphinxAtStartPar
名称     | |
&
\sphinxAtStartPar
描述                                        | |
\\
\sphinxhline
\sphinxAtStartPar
31:16
&
\sphinxAtStartPar
ROOTI
&
\sphinxAtStartPar
平方根整数数据                              |
\\
\sphinxhline
\sphinxAtStartPar
15:0
&
\sphinxAtStartPar
ROOTF
&
\sphinxAtStartPar
平方根小数数据                              |
\\
\sphinxbottomrule
\end{tabular}
\sphinxtableafterendhook\par
\sphinxattableend\end{savenotes}

\sphinxstepscope


\section{SLED控制模块(SLED)}
\label{\detokenize{SWM241/_u529f_u80fd_u63cf_u8ff0/SLED_u63a7_u5236_u6a21_u5757:sled-sled}}\label{\detokenize{SWM241/_u529f_u80fd_u63cf_u8ff0/SLED_u63a7_u5236_u6a21_u5757::doc}}
\sphinxAtStartPar
概述
\textasciitilde{}\textasciitilde{}

\sphinxAtStartPar
SWM241系列所有型号SLED控制模块操作均相同,不同型号SEG个数可能不同。使用前需使能SLED控制模块时钟。

\sphinxAtStartPar
特性
\textasciitilde{}\textasciitilde{}
\begin{itemize}
\item {} 
\sphinxAtStartPar
支持1/4占空比或1/8占空比

\item {} 
\sphinxAtStartPar
最多8个COM输出引脚和20个SEG输出引脚

\item {} 
\sphinxAtStartPar
SLED刷新频率可灵活配置

\item {} 
\sphinxAtStartPar
间隔时间可灵活配置

\item {} 
\sphinxAtStartPar
灰度时间可灵活配置

\end{itemize}


\subsection{模块结构框图}
\label{\detokenize{SWM241/_u529f_u80fd_u63cf_u8ff0/SLED_u63a7_u5236_u6a21_u5757:id1}}
\sphinxAtStartPar
\sphinxincludegraphics{{SWM241/功能描述/mediaSLED控制002}.emf}

\sphinxAtStartPar
图 6‑63 SLED控制模块结构框图

\sphinxAtStartPar
功能
\textasciitilde{}\textasciitilde{}

\sphinxAtStartPar
本模块为SLED控制器模块,支持1/4占空比或1/8占空比驱动方式。具有最多8个COM输出引脚和20个SEG输出引脚。SLED刷新频率可灵活配置。

\sphinxAtStartPar
SLED每个刷新周期内相关时间配置关系如图 6‑64所示:

\sphinxAtStartPar
\sphinxincludegraphics{{SWM241/功能描述/mediaSLED控制003}.emf}

\sphinxAtStartPar
图 6‑64 SLED刷新周期内相关时间配置关系图


\subsection{寄存器映射}
\label{\detokenize{SWM241/_u529f_u80fd_u63cf_u8ff0/SLED_u63a7_u5236_u6a21_u5757:id2}}

\begin{savenotes}\sphinxattablestart
\sphinxthistablewithglobalstyle
\centering
\begin{tabular}[t]{\X{20}{100}\X{20}{100}\X{20}{100}\X{20}{100}\X{20}{100}}
\sphinxtoprule
\sphinxtableatstartofbodyhook
\sphinxAtStartPar
名称   |
&
\begin{DUlineblock}{0em}
\item[] 偏移 |
\end{DUlineblock}
&
\begin{DUlineblock}{0em}
\item[] 
\item[] |
|
\end{DUlineblock}
&
\begin{DUlineblock}{0em}
\item[] 
\end{DUlineblock}
\begin{quote}

\begin{DUlineblock}{0em}
\item[] 
\item[] 
\end{DUlineblock}
\end{quote}
&
\sphinxAtStartPar
描述                       | | | |
\\
\sphinxhline
\sphinxAtStartPar
SLEDBASE:0 {\color{red}\bfseries{}|}x400AA800
&
\begin{DUlineblock}{0em}
\item[] 
\end{DUlineblock}
&&&\\
\sphinxhline
\sphinxAtStartPar
CR
&
\sphinxAtStartPar
0x00
&&
\sphinxAtStartPar
0x 00000
&
\sphinxAtStartPar
SLED控制寄存器             |
\\
\sphinxhline
\sphinxAtStartPar
CLKDIV
&
\sphinxAtStartPar
0x04
&&
\sphinxAtStartPar
0x 00008
&
\sphinxAtStartPar
SLED时钟分频寄存器         |
\\
\sphinxhline
\sphinxAtStartPar
TIM
&
\sphinxAtStartPar
0x08
&&
\sphinxAtStartPar
0x 000FF
&
\sphinxAtStartPar
SLED刷新时间配置寄存器     |
\\
\sphinxhline
\sphinxAtStartPar
DATA0
&
\sphinxAtStartPar
0x10
&&
\sphinxAtStartPar
0x 00000
&
\sphinxAtStartPar
SLED数据0寄存器            |
\\
\sphinxhline
\sphinxAtStartPar
DATA1
&
\sphinxAtStartPar
0x14
&&
\sphinxAtStartPar
0x 00000
&
\sphinxAtStartPar
SLED数据1寄存器            |
\\
\sphinxhline
\sphinxAtStartPar
DATA2
&
\sphinxAtStartPar
0x18
&&
\sphinxAtStartPar
0x 00000
&
\sphinxAtStartPar
SLED数据2寄存器            |
\\
\sphinxhline
\sphinxAtStartPar
DATA3
&
\sphinxAtStartPar
0x1c
&&
\sphinxAtStartPar
0x 00000
&
\sphinxAtStartPar
SLED数据3寄存器            |
\\
\sphinxhline
\sphinxAtStartPar
DATA4
&
\sphinxAtStartPar
0x20
&&
\sphinxAtStartPar
0x 00000
&
\sphinxAtStartPar
SLED数据4寄存器            |
\\
\sphinxhline
\sphinxAtStartPar
DATA5
&
\sphinxAtStartPar
0x24
&&
\sphinxAtStartPar
0x 00000
&
\sphinxAtStartPar
SLED数据5寄存器            |
\\
\sphinxhline
\sphinxAtStartPar
DATA6
&
\sphinxAtStartPar
0x28
&&
\sphinxAtStartPar
0x 00000
&
\sphinxAtStartPar
SLED数据6寄存器            |
\\
\sphinxhline
\sphinxAtStartPar
DATA7
&
\sphinxAtStartPar
0x2c
&&
\sphinxAtStartPar
0x 00000
&
\sphinxAtStartPar
SLED数据7寄存器            |
\\
\sphinxbottomrule
\end{tabular}
\sphinxtableafterendhook\par
\sphinxattableend\end{savenotes}


\subsection{寄存器描述}
\label{\detokenize{SWM241/_u529f_u80fd_u63cf_u8ff0/SLED_u63a7_u5236_u6a21_u5757:id5}}
\sphinxAtStartPar
SLED控制寄存器CR


\begin{savenotes}\sphinxattablestart
\sphinxthistablewithglobalstyle
\centering
\begin{tabular}[t]{\X{20}{100}\X{20}{100}\X{20}{100}\X{20}{100}\X{20}{100}}
\sphinxtoprule
\sphinxtableatstartofbodyhook
\sphinxAtStartPar
寄存器 |
&
\begin{DUlineblock}{0em}
\item[] 偏移 |
\end{DUlineblock}
&
\begin{DUlineblock}{0em}
\item[] 
\item[] {\color{red}\bfseries{}|}
\end{DUlineblock}
&
\sphinxAtStartPar
复位值 |    描 | |
&
\begin{DUlineblock}{0em}
\item[] |
  |
\end{DUlineblock}
\\
\sphinxhline
\sphinxAtStartPar
CR
&
\sphinxAtStartPar
0x00
&&
\sphinxAtStartPar
0 000000
&
\sphinxAtStartPar
SLED控制寄存器             |
\\
\sphinxbottomrule
\end{tabular}
\sphinxtableafterendhook\par
\sphinxattableend\end{savenotes}


\begin{savenotes}\sphinxattablestart
\sphinxthistablewithglobalstyle
\centering
\begin{tabular}[t]{\X{12}{96}\X{12}{96}\X{12}{96}\X{12}{96}\X{12}{96}\X{12}{96}\X{12}{96}\X{12}{96}}
\sphinxtoprule
\sphinxtableatstartofbodyhook
\sphinxAtStartPar
31
&
\sphinxAtStartPar
30
&
\sphinxAtStartPar
29
&
\sphinxAtStartPar
28
&
\sphinxAtStartPar
27
&
\sphinxAtStartPar
26
&
\sphinxAtStartPar
25
&
\sphinxAtStartPar
24
\\
\sphinxhline
\sphinxAtStartPar
COMINV
&\begin{itemize}
\item {} 
\end{itemize}
&&&&&&\\
\sphinxhline
\sphinxAtStartPar
23
&
\sphinxAtStartPar
22
&
\sphinxAtStartPar
21
&
\sphinxAtStartPar
20
&
\sphinxAtStartPar
19
&
\sphinxAtStartPar
18
&
\sphinxAtStartPar
17
&
\sphinxAtStartPar
16
\\
\sphinxhline\begin{itemize}
\item {} 
\end{itemize}
&&&&&&&\\
\sphinxhline
\sphinxAtStartPar
15
&
\sphinxAtStartPar
14
&
\sphinxAtStartPar
13
&
\sphinxAtStartPar
12
&
\sphinxAtStartPar
11
&
\sphinxAtStartPar
10
&
\sphinxAtStartPar
9
&
\sphinxAtStartPar
8
\\
\sphinxhline\begin{itemize}
\item {} 
\end{itemize}
&&&&&&&\\
\sphinxhline
\sphinxAtStartPar
7
&
\sphinxAtStartPar
6
&
\sphinxAtStartPar
5
&
\sphinxAtStartPar
4
&
\sphinxAtStartPar
3
&
\sphinxAtStartPar
2
&
\sphinxAtStartPar
1
&
\sphinxAtStartPar
0
\\
\sphinxhline\begin{itemize}
\item {} 
\end{itemize}
&&&&&&
\sphinxAtStartPar
DUTY
&
\sphinxAtStartPar
EN
\\
\sphinxbottomrule
\end{tabular}
\sphinxtableafterendhook\par
\sphinxattableend\end{savenotes}


\begin{savenotes}\sphinxattablestart
\sphinxthistablewithglobalstyle
\centering
\begin{tabular}[t]{\X{33}{99}\X{33}{99}\X{33}{99}}
\sphinxtoprule
\sphinxtableatstartofbodyhook
\sphinxAtStartPar
位域 |
&
\sphinxAtStartPar
名称     | |
&
\sphinxAtStartPar
描述                                        | |
\\
\sphinxhline
\sphinxAtStartPar
31
&
\sphinxAtStartPar
COMINV
&
\sphinxAtStartPar
COM反向配置寄存器                           |

\sphinxAtStartPar
0:COM波形正常                              |

\sphinxAtStartPar
1:COM波形反向                              |
\\
\sphinxhline
\sphinxAtStartPar
30:2
&\begin{itemize}
\item {} 
\end{itemize}
&\begin{itemize}
\item {} 
\end{itemize}
\\
\sphinxhline
\sphinxAtStartPar
1
&
\sphinxAtStartPar
DUTY
&
\sphinxAtStartPar
SLED占空比选择                              |

\sphinxAtStartPar
0:1/4占空比                                |

\sphinxAtStartPar
1:1/8占空比                                |
\\
\sphinxhline
\sphinxAtStartPar
0
&
\sphinxAtStartPar
EN
&
\sphinxAtStartPar
SLED使能控制位                              |

\sphinxAtStartPar
0:禁止SLED驱动器                           |

\sphinxAtStartPar
1:使能SLED驱动器                           |
\\
\sphinxbottomrule
\end{tabular}
\sphinxtableafterendhook\par
\sphinxattableend\end{savenotes}


\subsubsection{SLED时钟分频寄存器CLKDIV}
\label{\detokenize{SWM241/_u529f_u80fd_u63cf_u8ff0/SLED_u63a7_u5236_u6a21_u5757:sledclkdiv}}

\begin{savenotes}\sphinxattablestart
\sphinxthistablewithglobalstyle
\centering
\begin{tabular}[t]{\X{20}{100}\X{20}{100}\X{20}{100}\X{20}{100}\X{20}{100}}
\sphinxtoprule
\sphinxtableatstartofbodyhook
\sphinxAtStartPar
寄存器 |
&
\begin{DUlineblock}{0em}
\item[] 偏移 |
\end{DUlineblock}
&
\begin{DUlineblock}{0em}
\item[] 
\item[] {\color{red}\bfseries{}|}
\end{DUlineblock}
&
\sphinxAtStartPar
复位值 |    描 | |
&
\begin{DUlineblock}{0em}
\item[] |
  |
\end{DUlineblock}
\\
\sphinxhline
\sphinxAtStartPar
CLKDIV
&
\sphinxAtStartPar
0x04
&&
\sphinxAtStartPar
0 000008
&
\sphinxAtStartPar
SLED时钟分频寄存器         |
\\
\sphinxbottomrule
\end{tabular}
\sphinxtableafterendhook\par
\sphinxattableend\end{savenotes}


\begin{savenotes}\sphinxattablestart
\sphinxthistablewithglobalstyle
\centering
\begin{tabular}[t]{\X{12}{96}\X{12}{96}\X{12}{96}\X{12}{96}\X{12}{96}\X{12}{96}\X{12}{96}\X{12}{96}}
\sphinxtoprule
\sphinxtableatstartofbodyhook
\sphinxAtStartPar
31
&
\sphinxAtStartPar
30
&
\sphinxAtStartPar
29
&
\sphinxAtStartPar
28
&
\sphinxAtStartPar
27
&
\sphinxAtStartPar
26
&
\sphinxAtStartPar
25
&
\sphinxAtStartPar
24
\\
\sphinxhline\begin{itemize}
\item {} 
\end{itemize}
&&&&&&&\\
\sphinxhline
\sphinxAtStartPar
23
&
\sphinxAtStartPar
22
&
\sphinxAtStartPar
21
&
\sphinxAtStartPar
20
&
\sphinxAtStartPar
19
&
\sphinxAtStartPar
18
&
\sphinxAtStartPar
17
&
\sphinxAtStartPar
16
\\
\sphinxhline\begin{itemize}
\item {} 
\end{itemize}
&&&&&&&\\
\sphinxhline
\sphinxAtStartPar
15
&
\sphinxAtStartPar
14
&
\sphinxAtStartPar
13
&
\sphinxAtStartPar
12
&
\sphinxAtStartPar
11
&
\sphinxAtStartPar
10
&
\sphinxAtStartPar
9
&
\sphinxAtStartPar
8
\\
\sphinxhline\begin{itemize}
\item {} 
\end{itemize}
&&&&&&&\\
\sphinxhline
\sphinxAtStartPar
7
&
\sphinxAtStartPar
6
&
\sphinxAtStartPar
5
&
\sphinxAtStartPar
4
&
\sphinxAtStartPar
3
&
\sphinxAtStartPar
2
&
\sphinxAtStartPar
1
&
\sphinxAtStartPar
0
\\
\sphinxhline\begin{itemize}
\item {} 
\end{itemize}
&&&&
\sphinxAtStartPar
DIV
&&&\\
\sphinxbottomrule
\end{tabular}
\sphinxtableafterendhook\par
\sphinxattableend\end{savenotes}


\begin{savenotes}\sphinxattablestart
\sphinxthistablewithglobalstyle
\centering
\begin{tabular}[t]{\X{33}{99}\X{33}{99}\X{33}{99}}
\sphinxtoprule
\sphinxtableatstartofbodyhook
\sphinxAtStartPar
位域 |
&
\sphinxAtStartPar
名称     | |
&
\sphinxAtStartPar
描述                                        | |
\\
\sphinxhline
\sphinxAtStartPar
31:4
&\begin{itemize}
\item {} 
\end{itemize}
&\begin{itemize}
\item {} 
\end{itemize}
\\
\sphinxhline
\sphinxAtStartPar
3:0
&
\sphinxAtStartPar
DIV
&
\sphinxAtStartPar
SLED时钟(led\_clk)分频寄存器               |

\sphinxAtStartPar
0000:无效                                  |

\sphinxAtStartPar
0001:pclk2的2分频                          |

\sphinxAtStartPar
0010:pclk2的4分频                          |

\sphinxAtStartPar
0011:pclk2的8分频                          |

\sphinxAtStartPar
0100:pclk2的16分频                         |

\sphinxAtStartPar
0101:pclk2的32分频                         |

\sphinxAtStartPar
0110:pclk2的64分频                         |

\sphinxAtStartPar
0111:pclk2的128分频                        |

\sphinxAtStartPar
1000:pclk2的256分频                        |

\sphinxAtStartPar
1001:pclk2的512分频                        |

\sphinxAtStartPar
1010:pclk2的1024分频                       |

\sphinxAtStartPar
1011:pclk2的2048分频                       |

\sphinxAtStartPar
1100:pclk2的4096分频                       |

\sphinxAtStartPar
1101:pclk2的8192分频                       |

\sphinxAtStartPar
1110:pclk2的16384分频                      |

\sphinxAtStartPar
1111:pclk2的32768分频                      |
\\
\sphinxbottomrule
\end{tabular}
\sphinxtableafterendhook\par
\sphinxattableend\end{savenotes}


\subsubsection{SLED刷新时间配置寄存器TIM}
\label{\detokenize{SWM241/_u529f_u80fd_u63cf_u8ff0/SLED_u63a7_u5236_u6a21_u5757:sledtim}}

\begin{savenotes}\sphinxattablestart
\sphinxthistablewithglobalstyle
\centering
\begin{tabular}[t]{\X{20}{100}\X{20}{100}\X{20}{100}\X{20}{100}\X{20}{100}}
\sphinxtoprule
\sphinxtableatstartofbodyhook
\sphinxAtStartPar
寄存器 |
&
\begin{DUlineblock}{0em}
\item[] 偏移 |
\end{DUlineblock}
&
\begin{DUlineblock}{0em}
\item[] 
\item[] {\color{red}\bfseries{}|}
\end{DUlineblock}
&
\sphinxAtStartPar
复位值 |    描 | |
&
\begin{DUlineblock}{0em}
\item[] |
  |
\end{DUlineblock}
\\
\sphinxhline
\sphinxAtStartPar
TIM
&
\sphinxAtStartPar
0x08
&&
\sphinxAtStartPar
0 C000FF
&
\sphinxAtStartPar
SLED刷新时间配置寄存器     |
\\
\sphinxbottomrule
\end{tabular}
\sphinxtableafterendhook\par
\sphinxattableend\end{savenotes}


\begin{savenotes}\sphinxattablestart
\sphinxthistablewithglobalstyle
\centering
\begin{tabular}[t]{\X{12}{96}\X{12}{96}\X{12}{96}\X{12}{96}\X{12}{96}\X{12}{96}\X{12}{96}\X{12}{96}}
\sphinxtoprule
\sphinxtableatstartofbodyhook
\sphinxAtStartPar
31
&
\sphinxAtStartPar
30
&
\sphinxAtStartPar
29
&
\sphinxAtStartPar
28
&
\sphinxAtStartPar
27
&
\sphinxAtStartPar
26
&
\sphinxAtStartPar
25
&
\sphinxAtStartPar
24
\\
\sphinxhline\begin{itemize}
\item {} 
\end{itemize}
&&&&&&
\sphinxAtStartPar
HIGH
&\\
\sphinxhline
\sphinxAtStartPar
23
&
\sphinxAtStartPar
22
&
\sphinxAtStartPar
21
&
\sphinxAtStartPar
20
&
\sphinxAtStartPar
19
&
\sphinxAtStartPar
18
&
\sphinxAtStartPar
17
&
\sphinxAtStartPar
16
\\
\sphinxhline
\sphinxAtStartPar
HIGH
&&&&&&&\\
\sphinxhline
\sphinxAtStartPar
15
&
\sphinxAtStartPar
14
&
\sphinxAtStartPar
13
&
\sphinxAtStartPar
12
&
\sphinxAtStartPar
11
&
\sphinxAtStartPar
10
&
\sphinxAtStartPar
9
&
\sphinxAtStartPar
8
\\
\sphinxhline\begin{itemize}
\item {} 
\end{itemize}
&&&&&&
\sphinxAtStartPar
P ERIOD
&\\
\sphinxhline
\sphinxAtStartPar
7
&
\sphinxAtStartPar
6
&
\sphinxAtStartPar
5
&
\sphinxAtStartPar
4
&
\sphinxAtStartPar
3
&
\sphinxAtStartPar
2
&
\sphinxAtStartPar
1
&
\sphinxAtStartPar
0
\\
\sphinxhline
\sphinxAtStartPar
PERIOD
&&&&&&&\\
\sphinxbottomrule
\end{tabular}
\sphinxtableafterendhook\par
\sphinxattableend\end{savenotes}


\begin{savenotes}\sphinxattablestart
\sphinxthistablewithglobalstyle
\centering
\begin{tabular}[t]{\X{33}{99}\X{33}{99}\X{33}{99}}
\sphinxtoprule
\sphinxtableatstartofbodyhook
\sphinxAtStartPar
位域 |
&
\sphinxAtStartPar
名称     | |
&
\sphinxAtStartPar
描述                                        | |
\\
\sphinxhline
\sphinxAtStartPar
31:26
&\begin{itemize}
\item {} 
\end{itemize}
&\begin{itemize}
\item {} 
\end{itemize}
\\
\sphinxhline
\sphinxAtStartPar
25:16
&
\sphinxAtStartPar
HIGH
&
\sphinxAtStartPar
SLED刷新周期内,COM高电平配置寄存器         |

\sphinxAtStartPar
high\_tim = Tled\_clk * (HIGH+1)            |

\sphinxAtStartPar
注:LED\_HIGH配置值必                        | PERIOD配置值,可用于实现灵活调节显示亮度 |
\\
\sphinxhline
\sphinxAtStartPar
15:10
&\begin{itemize}
\item {} 
\end{itemize}
&\begin{itemize}
\item {} 
\end{itemize}
\\
\sphinxhline
\sphinxAtStartPar
9:0
&
\sphinxAtStartPar
PERIOD
&
\sphinxAtStartPar
SLED刷新周期内,COM/SEG电平周期配置寄存器   |

\sphinxAtStartPar
period\_tim = Tled\_clk *(PERIOD +1)       |
\\
\sphinxbottomrule
\end{tabular}
\sphinxtableafterendhook\par
\sphinxattableend\end{savenotes}


\subsubsection{SLED数据x寄存器DATAx(0\textasciitilde{}7)}
\label{\detokenize{SWM241/_u529f_u80fd_u63cf_u8ff0/SLED_u63a7_u5236_u6a21_u5757:sledxdatax-0-7}}

\begin{savenotes}\sphinxattablestart
\sphinxthistablewithglobalstyle
\centering
\begin{tabular}[t]{\X{20}{100}\X{20}{100}\X{20}{100}\X{20}{100}\X{20}{100}}
\sphinxtoprule
\sphinxtableatstartofbodyhook
\sphinxAtStartPar
寄存器 |
&
\begin{DUlineblock}{0em}
\item[] 偏移 |
\end{DUlineblock}
&
\begin{DUlineblock}{0em}
\item[] 
\item[] {\color{red}\bfseries{}|}
\end{DUlineblock}
&
\sphinxAtStartPar
复位值 |    描 | |
&
\begin{DUlineblock}{0em}
\item[] |
  |
\end{DUlineblock}
\\
\sphinxhline
\sphinxAtStartPar
DATA0
&
\sphinxAtStartPar
0x10
&&
\sphinxAtStartPar
0 000000
&
\sphinxAtStartPar
SLED数据0寄存器            |
\\
\sphinxbottomrule
\end{tabular}
\sphinxtableafterendhook\par
\sphinxattableend\end{savenotes}


\begin{savenotes}\sphinxattablestart
\sphinxthistablewithglobalstyle
\centering
\begin{tabular}[t]{\X{20}{100}\X{20}{100}\X{20}{100}\X{20}{100}\X{20}{100}}
\sphinxtoprule
\sphinxtableatstartofbodyhook
\sphinxAtStartPar
寄存器 |
&
\begin{DUlineblock}{0em}
\item[] 偏移 |
\end{DUlineblock}
&
\begin{DUlineblock}{0em}
\item[] 
\item[] {\color{red}\bfseries{}|}
\end{DUlineblock}
&
\sphinxAtStartPar
复位值 |    描 | |
&
\begin{DUlineblock}{0em}
\item[] |
  |
\end{DUlineblock}
\\
\sphinxhline
\sphinxAtStartPar
DATA1
&
\sphinxAtStartPar
0x14
&&
\sphinxAtStartPar
0 000000
&
\sphinxAtStartPar
SLED数据1寄存器            |
\\
\sphinxbottomrule
\end{tabular}
\sphinxtableafterendhook\par
\sphinxattableend\end{savenotes}


\begin{savenotes}\sphinxattablestart
\sphinxthistablewithglobalstyle
\centering
\begin{tabular}[t]{\X{20}{100}\X{20}{100}\X{20}{100}\X{20}{100}\X{20}{100}}
\sphinxtoprule
\sphinxtableatstartofbodyhook
\sphinxAtStartPar
寄存器 |
&
\begin{DUlineblock}{0em}
\item[] 偏移 |
\end{DUlineblock}
&
\begin{DUlineblock}{0em}
\item[] 
\item[] {\color{red}\bfseries{}|}
\end{DUlineblock}
&
\sphinxAtStartPar
复位值 |    描 | |
&
\begin{DUlineblock}{0em}
\item[] |
  |
\end{DUlineblock}
\\
\sphinxhline
\sphinxAtStartPar
DATA2
&
\sphinxAtStartPar
0x18
&&
\sphinxAtStartPar
0 000000
&
\sphinxAtStartPar
SLED数据2寄存器            |
\\
\sphinxbottomrule
\end{tabular}
\sphinxtableafterendhook\par
\sphinxattableend\end{savenotes}


\begin{savenotes}\sphinxattablestart
\sphinxthistablewithglobalstyle
\centering
\begin{tabular}[t]{\X{20}{100}\X{20}{100}\X{20}{100}\X{20}{100}\X{20}{100}}
\sphinxtoprule
\sphinxtableatstartofbodyhook
\sphinxAtStartPar
寄存器 |
&
\begin{DUlineblock}{0em}
\item[] 偏移 |
\end{DUlineblock}
&
\begin{DUlineblock}{0em}
\item[] 
\item[] {\color{red}\bfseries{}|}
\end{DUlineblock}
&
\sphinxAtStartPar
复位值 |    描 | |
&
\begin{DUlineblock}{0em}
\item[] |
  |
\end{DUlineblock}
\\
\sphinxhline
\sphinxAtStartPar
DATA3
&
\sphinxAtStartPar
0x1c
&&
\sphinxAtStartPar
0 000000
&
\sphinxAtStartPar
SLED数据3寄存器            |
\\
\sphinxbottomrule
\end{tabular}
\sphinxtableafterendhook\par
\sphinxattableend\end{savenotes}


\begin{savenotes}\sphinxattablestart
\sphinxthistablewithglobalstyle
\centering
\begin{tabular}[t]{\X{20}{100}\X{20}{100}\X{20}{100}\X{20}{100}\X{20}{100}}
\sphinxtoprule
\sphinxtableatstartofbodyhook
\sphinxAtStartPar
寄存器 |
&
\begin{DUlineblock}{0em}
\item[] 偏移 |
\end{DUlineblock}
&
\begin{DUlineblock}{0em}
\item[] 
\item[] {\color{red}\bfseries{}|}
\end{DUlineblock}
&
\sphinxAtStartPar
复位值 |    描 | |
&
\begin{DUlineblock}{0em}
\item[] |
  |
\end{DUlineblock}
\\
\sphinxhline
\sphinxAtStartPar
DATA4
&
\sphinxAtStartPar
0x20
&&
\sphinxAtStartPar
0 000000
&
\sphinxAtStartPar
SLED数据4寄存器            |
\\
\sphinxbottomrule
\end{tabular}
\sphinxtableafterendhook\par
\sphinxattableend\end{savenotes}


\begin{savenotes}\sphinxattablestart
\sphinxthistablewithglobalstyle
\centering
\begin{tabular}[t]{\X{20}{100}\X{20}{100}\X{20}{100}\X{20}{100}\X{20}{100}}
\sphinxtoprule
\sphinxtableatstartofbodyhook
\sphinxAtStartPar
寄存器 |
&
\begin{DUlineblock}{0em}
\item[] 偏移 |
\end{DUlineblock}
&
\begin{DUlineblock}{0em}
\item[] 
\item[] {\color{red}\bfseries{}|}
\end{DUlineblock}
&
\sphinxAtStartPar
复位值 |    描 | |
&
\begin{DUlineblock}{0em}
\item[] |
  |
\end{DUlineblock}
\\
\sphinxhline
\sphinxAtStartPar
DATA5
&
\sphinxAtStartPar
0x24
&&
\sphinxAtStartPar
0 000000
&
\sphinxAtStartPar
SLED数据5寄存器            |
\\
\sphinxbottomrule
\end{tabular}
\sphinxtableafterendhook\par
\sphinxattableend\end{savenotes}


\begin{savenotes}\sphinxattablestart
\sphinxthistablewithglobalstyle
\centering
\begin{tabular}[t]{\X{20}{100}\X{20}{100}\X{20}{100}\X{20}{100}\X{20}{100}}
\sphinxtoprule
\sphinxtableatstartofbodyhook
\sphinxAtStartPar
寄存器 |
&
\begin{DUlineblock}{0em}
\item[] 偏移 |
\end{DUlineblock}
&
\begin{DUlineblock}{0em}
\item[] 
\item[] {\color{red}\bfseries{}|}
\end{DUlineblock}
&
\sphinxAtStartPar
复位值 |    描 | |
&
\begin{DUlineblock}{0em}
\item[] |
  |
\end{DUlineblock}
\\
\sphinxhline
\sphinxAtStartPar
DATA6
&
\sphinxAtStartPar
0x28
&&
\sphinxAtStartPar
0 000000
&
\sphinxAtStartPar
SLED数据6寄存器            |
\\
\sphinxbottomrule
\end{tabular}
\sphinxtableafterendhook\par
\sphinxattableend\end{savenotes}


\begin{savenotes}\sphinxattablestart
\sphinxthistablewithglobalstyle
\centering
\begin{tabular}[t]{\X{20}{100}\X{20}{100}\X{20}{100}\X{20}{100}\X{20}{100}}
\sphinxtoprule
\sphinxtableatstartofbodyhook
\sphinxAtStartPar
寄存器 |
&
\begin{DUlineblock}{0em}
\item[] 偏移 |
\end{DUlineblock}
&
\begin{DUlineblock}{0em}
\item[] 
\item[] {\color{red}\bfseries{}|}
\end{DUlineblock}
&
\sphinxAtStartPar
复位值 |    描 | |
&
\begin{DUlineblock}{0em}
\item[] |
  |
\end{DUlineblock}
\\
\sphinxhline
\sphinxAtStartPar
DATA7
&
\sphinxAtStartPar
0x2c
&&
\sphinxAtStartPar
0 000000
&
\sphinxAtStartPar
SLED数据7寄存器            |
\\
\sphinxbottomrule
\end{tabular}
\sphinxtableafterendhook\par
\sphinxattableend\end{savenotes}


\begin{savenotes}\sphinxattablestart
\sphinxthistablewithglobalstyle
\centering
\begin{tabular}[t]{\X{12}{96}\X{12}{96}\X{12}{96}\X{12}{96}\X{12}{96}\X{12}{96}\X{12}{96}\X{12}{96}}
\sphinxtoprule
\sphinxtableatstartofbodyhook
\sphinxAtStartPar
31
&
\sphinxAtStartPar
30
&
\sphinxAtStartPar
29
&
\sphinxAtStartPar
28
&
\sphinxAtStartPar
27
&
\sphinxAtStartPar
26
&
\sphinxAtStartPar
25
&
\sphinxAtStartPar
24
\\
\sphinxhline\begin{itemize}
\item {} 
\end{itemize}
&&&&&&&\\
\sphinxhline
\sphinxAtStartPar
23
&
\sphinxAtStartPar
22
&
\sphinxAtStartPar
21
&
\sphinxAtStartPar
20
&
\sphinxAtStartPar
19
&
\sphinxAtStartPar
18
&
\sphinxAtStartPar
17
&
\sphinxAtStartPar
16
\\
\sphinxhline\begin{itemize}
\item {} 
\end{itemize}
&&&&
\sphinxAtStartPar
SEG19
&
\sphinxAtStartPar
SEG18
&
\sphinxAtStartPar
SEG17
&
\sphinxAtStartPar
SEG16
\\
\sphinxhline
\sphinxAtStartPar
15
&
\sphinxAtStartPar
14
&
\sphinxAtStartPar
13
&
\sphinxAtStartPar
12
&
\sphinxAtStartPar
11
&
\sphinxAtStartPar
10
&
\sphinxAtStartPar
9
&
\sphinxAtStartPar
8
\\
\sphinxhline
\sphinxAtStartPar
SEG15
&
\sphinxAtStartPar
SEG14
&
\sphinxAtStartPar
SEG13
&
\sphinxAtStartPar
SEG12
&
\sphinxAtStartPar
SEG11
&
\sphinxAtStartPar
SEG10
&
\sphinxAtStartPar
SEG9
&
\sphinxAtStartPar
SEG8
\\
\sphinxhline
\sphinxAtStartPar
7
&
\sphinxAtStartPar
6
&
\sphinxAtStartPar
5
&
\sphinxAtStartPar
4
&
\sphinxAtStartPar
3
&
\sphinxAtStartPar
2
&
\sphinxAtStartPar
1
&
\sphinxAtStartPar
0
\\
\sphinxhline
\sphinxAtStartPar
SEG7
&
\sphinxAtStartPar
SEG6
&
\sphinxAtStartPar
SEG5
&
\sphinxAtStartPar
SEG4
&
\sphinxAtStartPar
SEG3
&
\sphinxAtStartPar
SEG2
&
\sphinxAtStartPar
SEG1
&
\sphinxAtStartPar
SEG0
\\
\sphinxbottomrule
\end{tabular}
\sphinxtableafterendhook\par
\sphinxattableend\end{savenotes}


\begin{savenotes}\sphinxattablestart
\sphinxthistablewithglobalstyle
\centering
\begin{tabular}[t]{\X{33}{99}\X{33}{99}\X{33}{99}}
\sphinxtoprule
\sphinxtableatstartofbodyhook
\sphinxAtStartPar
位域 |
&
\sphinxAtStartPar
名称     | |
&
\sphinxAtStartPar
描述                                        | |
\\
\sphinxhline
\sphinxAtStartPar
31:20
&\begin{itemize}
\item {} 
\end{itemize}
&\begin{itemize}
\item {} 
\end{itemize}
\\
\sphinxhline
\sphinxAtStartPar
19
&
\sphinxAtStartPar
SEG19
&
\sphinxAtStartPar
SEG19数据输出                               |
\\
\sphinxhline
\sphinxAtStartPar
18
&
\sphinxAtStartPar
SEG18
&
\sphinxAtStartPar
SEG18数据输出                               |
\\
\sphinxhline
\sphinxAtStartPar
17
&
\sphinxAtStartPar
SEG17
&
\sphinxAtStartPar
SEG17数据输出                               |
\\
\sphinxhline
\sphinxAtStartPar
16
&
\sphinxAtStartPar
SEG16
&
\sphinxAtStartPar
SEG16数据输出                               |
\\
\sphinxhline
\sphinxAtStartPar
15
&
\sphinxAtStartPar
SEG15
&
\sphinxAtStartPar
SEG15数据输出                               |
\\
\sphinxhline
\sphinxAtStartPar
14
&
\sphinxAtStartPar
SEG14
&
\sphinxAtStartPar
SEG14数据输出                               |
\\
\sphinxhline
\sphinxAtStartPar
13
&
\sphinxAtStartPar
SEG13
&
\sphinxAtStartPar
SEG13数据输出                               |
\\
\sphinxhline
\sphinxAtStartPar
12
&
\sphinxAtStartPar
SEG12
&
\sphinxAtStartPar
SEG12数据输出                               |
\\
\sphinxhline
\sphinxAtStartPar
11
&
\sphinxAtStartPar
SEG11
&
\sphinxAtStartPar
SEG11数据输出                               |
\\
\sphinxhline
\sphinxAtStartPar
10
&
\sphinxAtStartPar
SEG10
&
\sphinxAtStartPar
SEG10数据输出                               |
\\
\sphinxhline
\sphinxAtStartPar
9
&
\sphinxAtStartPar
SEG9
&
\sphinxAtStartPar
SEG9数据输出                                |
\\
\sphinxhline
\sphinxAtStartPar
8
&
\sphinxAtStartPar
SEG8
&
\sphinxAtStartPar
SEG8数据输出                                |
\\
\sphinxhline
\sphinxAtStartPar
7
&
\sphinxAtStartPar
SEG7
&
\sphinxAtStartPar
SEG7数据输出                                |
\\
\sphinxhline
\sphinxAtStartPar
6
&
\sphinxAtStartPar
SEG6
&
\sphinxAtStartPar
SEG6数据输出                                |
\\
\sphinxhline
\sphinxAtStartPar
5
&
\sphinxAtStartPar
SEG5
&
\sphinxAtStartPar
SEG5数据输出                                |
\\
\sphinxhline
\sphinxAtStartPar
4
&
\sphinxAtStartPar
SEG4
&
\sphinxAtStartPar
SEG4数据输出                                |
\\
\sphinxhline
\sphinxAtStartPar
3
&
\sphinxAtStartPar
SEG3
&
\sphinxAtStartPar
SEG3数据输出                                |
\\
\sphinxhline
\sphinxAtStartPar
2
&
\sphinxAtStartPar
SEG2
&
\sphinxAtStartPar
SEG2数据输出                                |
\\
\sphinxhline
\sphinxAtStartPar
1
&
\sphinxAtStartPar
SEG1
&
\sphinxAtStartPar
SEG1数据输出                                |
\\
\sphinxhline
\sphinxAtStartPar
0
&
\sphinxAtStartPar
SEG0
&
\sphinxAtStartPar
SEG0数据输出                                |
\\
\sphinxbottomrule
\end{tabular}
\sphinxtableafterendhook\par
\sphinxattableend\end{savenotes}

\sphinxstepscope


\section{SLCD控制模块(SLCD)}
\label{\detokenize{SWM241/_u529f_u80fd_u63cf_u8ff0/SLCD_u63a7_u5236_u6a21_u5757:slcd-slcd}}\label{\detokenize{SWM241/_u529f_u80fd_u63cf_u8ff0/SLCD_u63a7_u5236_u6a21_u5757::doc}}
\sphinxAtStartPar
概述
\textasciitilde{}\textasciitilde{}

\sphinxAtStartPar
SWM241系列所有型号SLCD控制模块操作均相同,使用前需使能SLCD控制模块时钟, 并通过设置PA15输出高电平开启SLCD电源域。

\sphinxAtStartPar
特性
\textasciitilde{}\textasciitilde{}
\begin{itemize}
\item {} 
\sphinxAtStartPar
LCD帧频率设置

\item {} 
\sphinxAtStartPar
4 COM*32 SEG

\item {} 
\sphinxAtStartPar
BIAS和DUTY设置

\item {} 
\sphinxAtStartPar
根据图形显示数据,产生驱动电平控制时序

\item {} 
\sphinxAtStartPar
本模块仅支持TYPEB驱动波形

\item {} 
\sphinxAtStartPar
支持按键扫描功能

\end{itemize}


\subsection{模块结构框图}
\label{\detokenize{SWM241/_u529f_u80fd_u63cf_u8ff0/SLCD_u63a7_u5236_u6a21_u5757:id1}}
\sphinxAtStartPar
\sphinxincludegraphics{{SWM241/功能描述/mediaSLCD控制002}.emf}

\sphinxAtStartPar
图 6‑65 SLCD控制模块结构框图


\subsection{功能描述}
\label{\detokenize{SWM241/_u529f_u80fd_u63cf_u8ff0/SLCD_u63a7_u5236_u6a21_u5757:id2}}
\sphinxAtStartPar
本模块为SLCD控制器模块。模块支持标准APB总线操作,只支持WORD读写。


\subsubsection{LCD显示}
\label{\detokenize{SWM241/_u529f_u80fd_u63cf_u8ff0/SLCD_u63a7_u5236_u6a21_u5757:lcd}}\begin{itemize}
\item {} 
\sphinxAtStartPar
配置LCDC\_CR寄存器
\begin{itemize}
\item {} 
\sphinxAtStartPar
CLKDIV:指示系统高频频率分频

\item {} 
\sphinxAtStartPar
KEYSCAN:LCD按键扫描

\item {} 
\sphinxAtStartPar
DRIVSEL:LCD驱动电流

\item {} 
\sphinxAtStartPar
SCANFRQ:LCD帧频率

\item {} 
\sphinxAtStartPar
DUTY:LCD扫描模式

\item {} 
\sphinxAtStartPar
BIAS:LCD偏压

\item {} 
\sphinxAtStartPar
DISP:LCD显示模式

\item {} 
\sphinxAtStartPar
DRIVEEN:LCD驱动使能

\end{itemize}

\item {} 
\sphinxAtStartPar
配置LCD显示数据寄存器

\item {} 
\sphinxAtStartPar
配置LCDC\_CR寄存器
\begin{itemize}
\item {} 
\sphinxAtStartPar
SCANEN:LCD扫描使能

\end{itemize}

\end{itemize}


\subsubsection{按键扫描}
\label{\detokenize{SWM241/_u529f_u80fd_u63cf_u8ff0/SLCD_u63a7_u5236_u6a21_u5757:id3}}
\sphinxAtStartPar
LCD驱动端口如果和IO端口复用,可以通过按键扫描功能来配合GPIO端口实现组合按键,同时不影响LCD驱动显示。当按键扫描功能控制位使能后,所有复用端口的波形都会在每个COM开始的时候出现一个约30us宽度的低脉冲;此脉冲可以用作IO中断唤醒信号。

\sphinxAtStartPar
使用软件实现按键扫描功能的步骤如下:
\begin{itemize}
\item {} 
\sphinxAtStartPar
把GPIO设置为输入端口,并使能上拉;

\item {} 
\sphinxAtStartPar
启动GPIO中断功能;

\item {} 
\sphinxAtStartPar
启动按键扫描功能;

\item {} 
\sphinxAtStartPar
等待按键;

\item {} 
\sphinxAtStartPar
当有按键按下时,系统进入GPIO中断处理子程序:
\begin{itemize}
\item {} 
\sphinxAtStartPar
设置LCD为显示空白状态,并取消按键扫描功能;

\item {} 
\sphinxAtStartPar
设置IO为IO功能,并输出按键探测信号;

\item {} 
\sphinxAtStartPar
读取GPIO端口,判断按键;

\item {} 
\sphinxAtStartPar
设置IO为LCD驱动,并启动LCD显示;

\end{itemize}

\end{itemize}


\subsubsection{驱动波形}
\label{\detokenize{SWM241/_u529f_u80fd_u63cf_u8ff0/SLCD_u63a7_u5236_u6a21_u5757:id4}}
\sphinxAtStartPar
表格 6‑4 1/3 BIAS Driver Output Levels


\begin{savenotes}\sphinxattablestart
\sphinxthistablewithglobalstyle
\centering
\begin{tabular}[t]{\X{25}{100}\X{25}{100}\X{25}{100}\X{25}{100}}
\sphinxtoprule
\sphinxtableatstartofbodyhook
\sphinxAtStartPar
Driver
&
\sphinxAtStartPar
Mode
&
\sphinxAtStartPar
Alternation
&
\sphinxAtStartPar
Output Level
\\
\sphinxhline
\sphinxAtStartPar
COMMON
&
\sphinxAtStartPar
Selected
&
\sphinxAtStartPar
H
&
\sphinxAtStartPar
VDD
\\
\sphinxhline&&
\sphinxAtStartPar
L
&
\sphinxAtStartPar
GND
\\
\sphinxhline&
\sphinxAtStartPar
Non\sphinxhyphen{}Selected
&
\sphinxAtStartPar
H
&
\sphinxAtStartPar
2*VDD/3
\\
\sphinxhline&&
\sphinxAtStartPar
L
&
\sphinxAtStartPar
VDD/3
\\
\sphinxhline
\sphinxAtStartPar
SEGMENT
&
\sphinxAtStartPar
Selected
&
\sphinxAtStartPar
H
&
\sphinxAtStartPar
VDD
\\
\sphinxhline&&
\sphinxAtStartPar
L
&
\sphinxAtStartPar
GND
\\
\sphinxhline&
\sphinxAtStartPar
Non\sphinxhyphen{}Selected
&
\sphinxAtStartPar
H
&
\sphinxAtStartPar
2*VDD/3
\\
\sphinxhline&&
\sphinxAtStartPar
L
&
\sphinxAtStartPar
VDD/3
\\
\sphinxbottomrule
\end{tabular}
\sphinxtableafterendhook\par
\sphinxattableend\end{savenotes}

\sphinxAtStartPar
\sphinxincludegraphics{{SWM241/功能描述/mediaSLCD控制003}.emf}

\sphinxAtStartPar
图 6‑66 1/4DUTY和1/3BIAS驱动模式波形

\sphinxAtStartPar
\sphinxincludegraphics{{SWM241/功能描述/mediaSLCD控制004}.emf}

\sphinxAtStartPar
图 6‑67 1/3DUTY和1/3BIAS驱动模式波形

\sphinxAtStartPar
表格 6‑5 1/2 BIAS Driver Output Levels


\begin{savenotes}\sphinxattablestart
\sphinxthistablewithglobalstyle
\centering
\begin{tabular}[t]{\X{25}{100}\X{25}{100}\X{25}{100}\X{25}{100}}
\sphinxtoprule
\sphinxtableatstartofbodyhook
\sphinxAtStartPar
Driver
&
\sphinxAtStartPar
Mode
&
\sphinxAtStartPar
Alternation
&
\sphinxAtStartPar
Output Level
\\
\sphinxhline
\sphinxAtStartPar
COMMON
&
\sphinxAtStartPar
Selected
&
\sphinxAtStartPar
H
&
\sphinxAtStartPar
VDD
\\
\sphinxhline&&
\sphinxAtStartPar
L
&
\sphinxAtStartPar
GND
\\
\sphinxhline&
\sphinxAtStartPar
Non\sphinxhyphen{}Selected
&
\sphinxAtStartPar
H
&
\sphinxAtStartPar
VDD/2
\\
\sphinxhline&&
\sphinxAtStartPar
L
&
\sphinxAtStartPar
VDD/2
\\
\sphinxhline
\sphinxAtStartPar
SEGMENT
&
\sphinxAtStartPar
Selected
&
\sphinxAtStartPar
H
&
\sphinxAtStartPar
VDD
\\
\sphinxhline&&
\sphinxAtStartPar
L
&
\sphinxAtStartPar
GND
\\
\sphinxhline&
\sphinxAtStartPar
Non\sphinxhyphen{}Selected
&
\sphinxAtStartPar
H
&
\sphinxAtStartPar
VDD
\\
\sphinxhline&&
\sphinxAtStartPar
L
&
\sphinxAtStartPar
GND
\\
\sphinxbottomrule
\end{tabular}
\sphinxtableafterendhook\par
\sphinxattableend\end{savenotes}

\sphinxAtStartPar
\sphinxincludegraphics{{SWM241/功能描述/mediaSLCD控制005}.emf}

\sphinxAtStartPar
图 6‑68 1/3DUTY和1/2BIAS驱动模式波形


\subsubsection{数据显示映射表}
\label{\detokenize{SWM241/_u529f_u80fd_u63cf_u8ff0/SLCD_u63a7_u5236_u6a21_u5757:id5}}
\sphinxAtStartPar
表格 6‑6 显示数据映射表


\begin{savenotes}\sphinxattablestart
\sphinxthistablewithglobalstyle
\centering
\begin{tabular}[t]{\X{16}{96}\X{16}{96}\X{16}{96}\X{16}{96}\X{16}{96}\X{16}{96}}
\sphinxtoprule
\sphinxtableatstartofbodyhook&
\sphinxAtStartPar
SEG0
&
\sphinxAtStartPar
SEG1
&
\sphinxAtStartPar
……
&&\\
\sphinxhline
\sphinxAtStartPar
COM0
&
\sphinxAtStartPar
LCDC \_DATA0/B0
&
\sphinxAtStartPar
LC DATA0/B1
&
\sphinxAtStartPar
……
&
\sphinxAtStartPar
LC DAT B30
&
\sphinxAtStartPar
LC DAT B31
\\
\sphinxhline
\sphinxAtStartPar
COM1
&
\sphinxAtStartPar
LCDC \_DATA1/B0
&
\sphinxAtStartPar
LC DATA1/B1
&
\sphinxAtStartPar
……
&
\sphinxAtStartPar
LC DAT B30
&
\sphinxAtStartPar
LC DAT B31
\\
\sphinxhline
\sphinxAtStartPar
COM2
&
\sphinxAtStartPar
LCDC \_DATA2/B0
&
\sphinxAtStartPar
LC DATA2/B1
&
\sphinxAtStartPar
……
&
\sphinxAtStartPar
LC DAT B30
&
\sphinxAtStartPar
LC DAT B31
\\
\sphinxhline
\sphinxAtStartPar
COM3
&
\sphinxAtStartPar
LCDC \_DATA3/B0
&
\sphinxAtStartPar
LC DATA3/B1
&
\sphinxAtStartPar
……
&
\sphinxAtStartPar
LC DAT B30
&
\sphinxAtStartPar
LC DAT B31
\\
\sphinxbottomrule
\end{tabular}
\sphinxtableafterendhook\par
\sphinxattableend\end{savenotes}


\subsection{寄存器映射}
\label{\detokenize{SWM241/_u529f_u80fd_u63cf_u8ff0/SLCD_u63a7_u5236_u6a21_u5757:id6}}

\begin{savenotes}\sphinxattablestart
\sphinxthistablewithglobalstyle
\centering
\begin{tabular}[t]{\X{20}{100}\X{20}{100}\X{20}{100}\X{20}{100}\X{20}{100}}
\sphinxtoprule
\sphinxtableatstartofbodyhook
\sphinxAtStartPar
名称   |
&
\begin{DUlineblock}{0em}
\item[] 偏移 |
\end{DUlineblock}
&
\begin{DUlineblock}{0em}
\item[] 
\item[] |
|
\end{DUlineblock}
&
\begin{DUlineblock}{0em}
\item[] 
\end{DUlineblock}
\begin{quote}

\begin{DUlineblock}{0em}
\item[] 
\item[] 
\end{DUlineblock}
\end{quote}
&
\sphinxAtStartPar
描述                       | | | |
\\
\sphinxhline
\sphinxAtStartPar
SLCDBASE:0 {\color{red}\bfseries{}|}x400A9800
&
\begin{DUlineblock}{0em}
\item[] 
\end{DUlineblock}
&&&\\
\sphinxhline
\sphinxAtStartPar
CR
&
\sphinxAtStartPar
0x00
&&
\sphinxAtStartPar
0x F0000
&
\sphinxAtStartPar
SLCD配置寄存器             |
\\
\sphinxhline
\sphinxAtStartPar
DATA0
&
\sphinxAtStartPar
0x10
&&
\sphinxAtStartPar
0x 00000
&
\sphinxAtStartPar
SLCD数据寄存器0            |
\\
\sphinxhline
\sphinxAtStartPar
DATA1
&
\sphinxAtStartPar
0x14
&&
\sphinxAtStartPar
0x 00000
&
\sphinxAtStartPar
SLCD数据寄存器1            |
\\
\sphinxhline
\sphinxAtStartPar
DATA2
&
\sphinxAtStartPar
0x18
&&
\sphinxAtStartPar
0x 00000
&
\sphinxAtStartPar
SLCD数据寄存器2            |
\\
\sphinxhline
\sphinxAtStartPar
DATA3
&
\sphinxAtStartPar
0x1c
&&
\sphinxAtStartPar
0x 00000
&
\sphinxAtStartPar
SLCD数据寄存器3            |
\\
\sphinxbottomrule
\end{tabular}
\sphinxtableafterendhook\par
\sphinxattableend\end{savenotes}


\subsection{寄存器描述}
\label{\detokenize{SWM241/_u529f_u80fd_u63cf_u8ff0/SLCD_u63a7_u5236_u6a21_u5757:id9}}

\subsubsection{SLCD控制寄存器CR}
\label{\detokenize{SWM241/_u529f_u80fd_u63cf_u8ff0/SLCD_u63a7_u5236_u6a21_u5757:slcdcr}}

\begin{savenotes}\sphinxattablestart
\sphinxthistablewithglobalstyle
\centering
\begin{tabular}[t]{\X{20}{100}\X{20}{100}\X{20}{100}\X{20}{100}\X{20}{100}}
\sphinxtoprule
\sphinxtableatstartofbodyhook
\sphinxAtStartPar
寄存器 |
&
\begin{DUlineblock}{0em}
\item[] 偏移 |
\end{DUlineblock}
&
\begin{DUlineblock}{0em}
\item[] 
\item[] {\color{red}\bfseries{}|}
\end{DUlineblock}
&
\sphinxAtStartPar
复位值 |    描 | |
&
\begin{DUlineblock}{0em}
\item[] |
  |
\end{DUlineblock}
\\
\sphinxhline
\sphinxAtStartPar
CR
&
\sphinxAtStartPar
0x00
&&
\sphinxAtStartPar
0 3F0000
&
\sphinxAtStartPar
SLCD配置寄存器             |
\\
\sphinxbottomrule
\end{tabular}
\sphinxtableafterendhook\par
\sphinxattableend\end{savenotes}


\begin{savenotes}\sphinxattablestart
\sphinxthistablewithglobalstyle
\centering
\begin{tabular}[t]{\X{12}{96}\X{12}{96}\X{12}{96}\X{12}{96}\X{12}{96}\X{12}{96}\X{12}{96}\X{12}{96}}
\sphinxtoprule
\sphinxtableatstartofbodyhook
\sphinxAtStartPar
31
&
\sphinxAtStartPar
30
&
\sphinxAtStartPar
29
&
\sphinxAtStartPar
28
&
\sphinxAtStartPar
27
&
\sphinxAtStartPar
26
&
\sphinxAtStartPar
25
&
\sphinxAtStartPar
24
\\
\sphinxhline\begin{itemize}
\item {} 
\end{itemize}
&&&&&&&\\
\sphinxhline
\sphinxAtStartPar
23
&
\sphinxAtStartPar
22
&
\sphinxAtStartPar
21
&
\sphinxAtStartPar
20
&
\sphinxAtStartPar
19
&
\sphinxAtStartPar
18
&
\sphinxAtStartPar
17
&
\sphinxAtStartPar
16
\\
\sphinxhline\begin{itemize}
\item {} 
\end{itemize}
&&
\sphinxAtStartPar
CL K\_DIV
&&&&&\\
\sphinxhline
\sphinxAtStartPar
15
&
\sphinxAtStartPar
14
&
\sphinxAtStartPar
13
&
\sphinxAtStartPar
12
&
\sphinxAtStartPar
11
&
\sphinxAtStartPar
10
&
\sphinxAtStartPar
9
&
\sphinxAtStartPar
8
\\
\sphinxhline\begin{itemize}
\item {} 
\end{itemize}
&&&&&
\sphinxAtStartPar
K EYSCAN
&\begin{itemize}
\item {} 
\end{itemize}
&\\
\sphinxhline
\sphinxAtStartPar
7
&
\sphinxAtStartPar
6
&
\sphinxAtStartPar
5
&
\sphinxAtStartPar
4
&
\sphinxAtStartPar
3
&
\sphinxAtStartPar
2
&
\sphinxAtStartPar
1
&
\sphinxAtStartPar
0
\\
\sphinxhline
\sphinxAtStartPar
SCANFRQ
&&
\sphinxAtStartPar
DUTY
&
\sphinxAtStartPar
BIAS
&
\sphinxAtStartPar
DISP
&&
\sphinxAtStartPar
S CANEN
&
\sphinxAtStartPar
DRIVEN
\\
\sphinxbottomrule
\end{tabular}
\sphinxtableafterendhook\par
\sphinxattableend\end{savenotes}


\begin{savenotes}\sphinxattablestart
\sphinxthistablewithglobalstyle
\centering
\begin{tabular}[t]{\X{33}{99}\X{33}{99}\X{33}{99}}
\sphinxtoprule
\sphinxtableatstartofbodyhook
\sphinxAtStartPar
位域 |
&
\sphinxAtStartPar
名称     | |
&
\sphinxAtStartPar
描述                                        | |
\\
\sphinxhline
\sphinxAtStartPar
31:22
&\begin{itemize}
\item {} 
\end{itemize}
&\begin{itemize}
\item {} 
\end{itemize}
\\
\sphinxhline
\sphinxAtStartPar
21:16
&
\sphinxAtStartPar
CLKDIV
&
\sphinxAtStartPar
系统分频控制                                |

\sphinxAtStartPar
Flcd = PCLK2 / (CLKDIV*32)

\sphinxAtStartPar
注:由                                      | CD模块挂载在APB2总线上,APB2总线时钟(PCLK  | 系统时钟(SYSCLK)的二分频,故PCLK2=SYSCLK/2 |
\\
\sphinxhline
\sphinxAtStartPar
15:11
&\begin{itemize}
\item {} 
\end{itemize}
&\begin{itemize}
\item {} 
\end{itemize}
\\
\sphinxhline
\sphinxAtStartPar
10
&
\sphinxAtStartPar
KEYSCAN
&
\sphinxAtStartPar
按键扫描功能                                |

\sphinxAtStartPar
1使能                                       |
\\
\sphinxhline
\sphinxAtStartPar
9:8
&\begin{itemize}
\item {} 
\end{itemize}
&\begin{itemize}
\item {} 
\end{itemize}
\\
\sphinxhline
\sphinxAtStartPar
7:6
&
\sphinxAtStartPar
SCANFRQ
&
\sphinxAtStartPar
LCD帧频率控制                               |

\sphinxAtStartPar
00:Flcd/256 Hz                             |

\sphinxAtStartPar
01:Flcd/128 Hz                             |

\sphinxAtStartPar
10:Flcd/64 Hz                              |

\sphinxAtStartPar
11:Flcd/32 Hz                              |
\\
\sphinxhline
\sphinxAtStartPar
5
&
\sphinxAtStartPar
DUTY
&
\sphinxAtStartPar
LCD扫描模式选择                             |

\sphinxAtStartPar
0:1/4 Duty                                 |

\sphinxAtStartPar
1:1/3 Duty                                 |
\\
\sphinxhline
\sphinxAtStartPar
4
&
\sphinxAtStartPar
BIAS
&
\sphinxAtStartPar
LCD偏压模式选择                             |

\sphinxAtStartPar
0:1/3 Bias                                 |

\sphinxAtStartPar
1:1/2 Bias                                 |
\\
\sphinxhline
\sphinxAtStartPar
3:2
&
\sphinxAtStartPar
DISP
&
\sphinxAtStartPar
LCD显示模式                                 |

\sphinxAtStartPar
00:正常显示                                |

\sphinxAtStartPar
01:显示空白                                |

\sphinxAtStartPar
10:显示全部                                |

\sphinxAtStartPar
11:保留                                    |
\\
\sphinxhline
\sphinxAtStartPar
1
&
\sphinxAtStartPar
SCANEN
&
\sphinxAtStartPar
LCD扫描电路使能                             |

\sphinxAtStartPar
0:不使能                                   |

\sphinxAtStartPar
1:使能                                     |
\\
\sphinxhline
\sphinxAtStartPar
0
&
\sphinxAtStartPar
DRIVEN
&
\sphinxAtStartPar
LCD驱动电路使能                             |

\sphinxAtStartPar
0:不使能                                   |

\sphinxAtStartPar
1:使能                                     |
\\
\sphinxbottomrule
\end{tabular}
\sphinxtableafterendhook\par
\sphinxattableend\end{savenotes}


\subsubsection{SLCD数据寄存器0 DATA0}
\label{\detokenize{SWM241/_u529f_u80fd_u63cf_u8ff0/SLCD_u63a7_u5236_u6a21_u5757:slcd0-data0}}

\begin{savenotes}\sphinxattablestart
\sphinxthistablewithglobalstyle
\centering
\begin{tabular}[t]{\X{20}{100}\X{20}{100}\X{20}{100}\X{20}{100}\X{20}{100}}
\sphinxtoprule
\sphinxtableatstartofbodyhook
\sphinxAtStartPar
寄存器 |
&
\begin{DUlineblock}{0em}
\item[] 偏移 |
\end{DUlineblock}
&
\begin{DUlineblock}{0em}
\item[] 
\item[] {\color{red}\bfseries{}|}
\end{DUlineblock}
&
\sphinxAtStartPar
复位值 |    描 | |
&
\begin{DUlineblock}{0em}
\item[] |
  |
\end{DUlineblock}
\\
\sphinxhline
\sphinxAtStartPar
DATA0
&
\sphinxAtStartPar
0x10
&&
\sphinxAtStartPar
0 000000
&
\sphinxAtStartPar
SLCD数据寄存器0            |
\\
\sphinxbottomrule
\end{tabular}
\sphinxtableafterendhook\par
\sphinxattableend\end{savenotes}


\begin{savenotes}\sphinxattablestart
\sphinxthistablewithglobalstyle
\centering
\begin{tabular}[t]{\X{12}{96}\X{12}{96}\X{12}{96}\X{12}{96}\X{12}{96}\X{12}{96}\X{12}{96}\X{12}{96}}
\sphinxtoprule
\sphinxtableatstartofbodyhook
\sphinxAtStartPar
31
&
\sphinxAtStartPar
30
&
\sphinxAtStartPar
29
&
\sphinxAtStartPar
28
&
\sphinxAtStartPar
27
&
\sphinxAtStartPar
26
&
\sphinxAtStartPar
25
&
\sphinxAtStartPar
24
\\
\sphinxhline
\sphinxAtStartPar
DATA
&&&&&&&\\
\sphinxhline
\sphinxAtStartPar
23
&
\sphinxAtStartPar
22
&
\sphinxAtStartPar
21
&
\sphinxAtStartPar
20
&
\sphinxAtStartPar
19
&
\sphinxAtStartPar
18
&
\sphinxAtStartPar
17
&
\sphinxAtStartPar
16
\\
\sphinxhline
\sphinxAtStartPar
DATA
&&&&&&&\\
\sphinxhline
\sphinxAtStartPar
15
&
\sphinxAtStartPar
14
&
\sphinxAtStartPar
13
&
\sphinxAtStartPar
12
&
\sphinxAtStartPar
11
&
\sphinxAtStartPar
10
&
\sphinxAtStartPar
9
&
\sphinxAtStartPar
8
\\
\sphinxhline
\sphinxAtStartPar
DATA
&&&&&&&\\
\sphinxhline
\sphinxAtStartPar
7
&
\sphinxAtStartPar
6
&
\sphinxAtStartPar
5
&
\sphinxAtStartPar
4
&
\sphinxAtStartPar
3
&
\sphinxAtStartPar
2
&
\sphinxAtStartPar
1
&
\sphinxAtStartPar
0
\\
\sphinxhline
\sphinxAtStartPar
DATA
&&&&&&&\\
\sphinxbottomrule
\end{tabular}
\sphinxtableafterendhook\par
\sphinxattableend\end{savenotes}


\begin{savenotes}\sphinxattablestart
\sphinxthistablewithglobalstyle
\centering
\begin{tabular}[t]{\X{33}{99}\X{33}{99}\X{33}{99}}
\sphinxtoprule
\sphinxtableatstartofbodyhook
\sphinxAtStartPar
位域 |
&
\sphinxAtStartPar
名称     | |
&
\sphinxAtStartPar
描述                                        | |
\\
\sphinxhline
\sphinxAtStartPar
31:0
&
\sphinxAtStartPar
DATA0
&
\sphinxAtStartPar
LCD数据寄存器0                              |
\\
\sphinxbottomrule
\end{tabular}
\sphinxtableafterendhook\par
\sphinxattableend\end{savenotes}


\subsubsection{SLCD数据寄存器1 DATA1}
\label{\detokenize{SWM241/_u529f_u80fd_u63cf_u8ff0/SLCD_u63a7_u5236_u6a21_u5757:slcd1-data1}}

\begin{savenotes}\sphinxattablestart
\sphinxthistablewithglobalstyle
\centering
\begin{tabular}[t]{\X{20}{100}\X{20}{100}\X{20}{100}\X{20}{100}\X{20}{100}}
\sphinxtoprule
\sphinxtableatstartofbodyhook
\sphinxAtStartPar
寄存器 |
&
\begin{DUlineblock}{0em}
\item[] 偏移 |
\end{DUlineblock}
&
\begin{DUlineblock}{0em}
\item[] 
\item[] {\color{red}\bfseries{}|}
\end{DUlineblock}
&
\sphinxAtStartPar
复位值 |    描 | |
&
\begin{DUlineblock}{0em}
\item[] |
  |
\end{DUlineblock}
\\
\sphinxhline
\sphinxAtStartPar
DATA1
&
\sphinxAtStartPar
0x14
&&
\sphinxAtStartPar
0 000000
&
\sphinxAtStartPar
SLCD数据寄存器1            |
\\
\sphinxbottomrule
\end{tabular}
\sphinxtableafterendhook\par
\sphinxattableend\end{savenotes}


\begin{savenotes}\sphinxattablestart
\sphinxthistablewithglobalstyle
\centering
\begin{tabular}[t]{\X{12}{96}\X{12}{96}\X{12}{96}\X{12}{96}\X{12}{96}\X{12}{96}\X{12}{96}\X{12}{96}}
\sphinxtoprule
\sphinxtableatstartofbodyhook
\sphinxAtStartPar
31
&
\sphinxAtStartPar
30
&
\sphinxAtStartPar
29
&
\sphinxAtStartPar
28
&
\sphinxAtStartPar
27
&
\sphinxAtStartPar
26
&
\sphinxAtStartPar
25
&
\sphinxAtStartPar
24
\\
\sphinxhline
\sphinxAtStartPar
DATA
&&&&&&&\\
\sphinxhline
\sphinxAtStartPar
23
&
\sphinxAtStartPar
22
&
\sphinxAtStartPar
21
&
\sphinxAtStartPar
20
&
\sphinxAtStartPar
19
&
\sphinxAtStartPar
18
&
\sphinxAtStartPar
17
&
\sphinxAtStartPar
16
\\
\sphinxhline
\sphinxAtStartPar
DATA
&&&&&&&\\
\sphinxhline
\sphinxAtStartPar
15
&
\sphinxAtStartPar
14
&
\sphinxAtStartPar
13
&
\sphinxAtStartPar
12
&
\sphinxAtStartPar
11
&
\sphinxAtStartPar
10
&
\sphinxAtStartPar
9
&
\sphinxAtStartPar
8
\\
\sphinxhline
\sphinxAtStartPar
DATA
&&&&&&&\\
\sphinxhline
\sphinxAtStartPar
7
&
\sphinxAtStartPar
6
&
\sphinxAtStartPar
5
&
\sphinxAtStartPar
4
&
\sphinxAtStartPar
3
&
\sphinxAtStartPar
2
&
\sphinxAtStartPar
1
&
\sphinxAtStartPar
0
\\
\sphinxhline
\sphinxAtStartPar
DATA
&&&&&&&\\
\sphinxbottomrule
\end{tabular}
\sphinxtableafterendhook\par
\sphinxattableend\end{savenotes}


\begin{savenotes}\sphinxattablestart
\sphinxthistablewithglobalstyle
\centering
\begin{tabular}[t]{\X{33}{99}\X{33}{99}\X{33}{99}}
\sphinxtoprule
\sphinxtableatstartofbodyhook
\sphinxAtStartPar
位域 |
&
\sphinxAtStartPar
名称     | |
&
\sphinxAtStartPar
描述                                        | |
\\
\sphinxhline
\sphinxAtStartPar
31:0
&
\sphinxAtStartPar
DATA1
&
\sphinxAtStartPar
LCD数据寄存器1                              |
\\
\sphinxbottomrule
\end{tabular}
\sphinxtableafterendhook\par
\sphinxattableend\end{savenotes}


\subsubsection{SLCD数据寄存器2 DATA2}
\label{\detokenize{SWM241/_u529f_u80fd_u63cf_u8ff0/SLCD_u63a7_u5236_u6a21_u5757:slcd2-data2}}

\begin{savenotes}\sphinxattablestart
\sphinxthistablewithglobalstyle
\centering
\begin{tabular}[t]{\X{20}{100}\X{20}{100}\X{20}{100}\X{20}{100}\X{20}{100}}
\sphinxtoprule
\sphinxtableatstartofbodyhook
\sphinxAtStartPar
寄存器 |
&
\begin{DUlineblock}{0em}
\item[] 偏移 |
\end{DUlineblock}
&
\begin{DUlineblock}{0em}
\item[] 
\item[] {\color{red}\bfseries{}|}
\end{DUlineblock}
&
\sphinxAtStartPar
复位值 |    描 | |
&
\begin{DUlineblock}{0em}
\item[] |
  |
\end{DUlineblock}
\\
\sphinxhline
\sphinxAtStartPar
DATA2
&
\sphinxAtStartPar
0x18
&&
\sphinxAtStartPar
0 000000
&
\sphinxAtStartPar
SLCD数据寄存器2            |
\\
\sphinxbottomrule
\end{tabular}
\sphinxtableafterendhook\par
\sphinxattableend\end{savenotes}


\begin{savenotes}\sphinxattablestart
\sphinxthistablewithglobalstyle
\centering
\begin{tabular}[t]{\X{12}{96}\X{12}{96}\X{12}{96}\X{12}{96}\X{12}{96}\X{12}{96}\X{12}{96}\X{12}{96}}
\sphinxtoprule
\sphinxtableatstartofbodyhook
\sphinxAtStartPar
31
&
\sphinxAtStartPar
30
&
\sphinxAtStartPar
29
&
\sphinxAtStartPar
28
&
\sphinxAtStartPar
27
&
\sphinxAtStartPar
26
&
\sphinxAtStartPar
25
&
\sphinxAtStartPar
24
\\
\sphinxhline
\sphinxAtStartPar
DATA
&&&&&&&\\
\sphinxhline
\sphinxAtStartPar
23
&
\sphinxAtStartPar
22
&
\sphinxAtStartPar
21
&
\sphinxAtStartPar
20
&
\sphinxAtStartPar
19
&
\sphinxAtStartPar
18
&
\sphinxAtStartPar
17
&
\sphinxAtStartPar
16
\\
\sphinxhline
\sphinxAtStartPar
DATA
&&&&&&&\\
\sphinxhline
\sphinxAtStartPar
15
&
\sphinxAtStartPar
14
&
\sphinxAtStartPar
13
&
\sphinxAtStartPar
12
&
\sphinxAtStartPar
11
&
\sphinxAtStartPar
10
&
\sphinxAtStartPar
9
&
\sphinxAtStartPar
8
\\
\sphinxhline
\sphinxAtStartPar
DATA
&&&&&&&\\
\sphinxhline
\sphinxAtStartPar
7
&
\sphinxAtStartPar
6
&
\sphinxAtStartPar
5
&
\sphinxAtStartPar
4
&
\sphinxAtStartPar
3
&
\sphinxAtStartPar
2
&
\sphinxAtStartPar
1
&
\sphinxAtStartPar
0
\\
\sphinxhline
\sphinxAtStartPar
DATA
&&&&&&&\\
\sphinxbottomrule
\end{tabular}
\sphinxtableafterendhook\par
\sphinxattableend\end{savenotes}


\begin{savenotes}\sphinxattablestart
\sphinxthistablewithglobalstyle
\centering
\begin{tabular}[t]{\X{33}{99}\X{33}{99}\X{33}{99}}
\sphinxtoprule
\sphinxtableatstartofbodyhook
\sphinxAtStartPar
位域 |
&
\sphinxAtStartPar
名称     | |
&
\sphinxAtStartPar
描述                                        | |
\\
\sphinxhline
\sphinxAtStartPar
31:0
&
\sphinxAtStartPar
DATA2
&
\sphinxAtStartPar
LCD数据寄存器2                              |
\\
\sphinxbottomrule
\end{tabular}
\sphinxtableafterendhook\par
\sphinxattableend\end{savenotes}


\subsubsection{SLCD数据寄存器3 DATA3}
\label{\detokenize{SWM241/_u529f_u80fd_u63cf_u8ff0/SLCD_u63a7_u5236_u6a21_u5757:slcd3-data3}}

\begin{savenotes}\sphinxattablestart
\sphinxthistablewithglobalstyle
\centering
\begin{tabular}[t]{\X{20}{100}\X{20}{100}\X{20}{100}\X{20}{100}\X{20}{100}}
\sphinxtoprule
\sphinxtableatstartofbodyhook
\sphinxAtStartPar
寄存器 |
&
\begin{DUlineblock}{0em}
\item[] 偏移 |
\end{DUlineblock}
&
\begin{DUlineblock}{0em}
\item[] 
\item[] {\color{red}\bfseries{}|}
\end{DUlineblock}
&
\sphinxAtStartPar
复位值 |    描 | |
&
\begin{DUlineblock}{0em}
\item[] |
  |
\end{DUlineblock}
\\
\sphinxhline
\sphinxAtStartPar
DATA3
&
\sphinxAtStartPar
0x1c
&&
\sphinxAtStartPar
0 000000
&
\sphinxAtStartPar
SLCD数据寄存器3            |
\\
\sphinxbottomrule
\end{tabular}
\sphinxtableafterendhook\par
\sphinxattableend\end{savenotes}


\begin{savenotes}\sphinxattablestart
\sphinxthistablewithglobalstyle
\centering
\begin{tabular}[t]{\X{12}{96}\X{12}{96}\X{12}{96}\X{12}{96}\X{12}{96}\X{12}{96}\X{12}{96}\X{12}{96}}
\sphinxtoprule
\sphinxtableatstartofbodyhook
\sphinxAtStartPar
31
&
\sphinxAtStartPar
30
&
\sphinxAtStartPar
29
&
\sphinxAtStartPar
28
&
\sphinxAtStartPar
27
&
\sphinxAtStartPar
26
&
\sphinxAtStartPar
25
&
\sphinxAtStartPar
24
\\
\sphinxhline
\sphinxAtStartPar
DATA
&&&&&&&\\
\sphinxhline
\sphinxAtStartPar
23
&
\sphinxAtStartPar
22
&
\sphinxAtStartPar
21
&
\sphinxAtStartPar
20
&
\sphinxAtStartPar
19
&
\sphinxAtStartPar
18
&
\sphinxAtStartPar
17
&
\sphinxAtStartPar
16
\\
\sphinxhline
\sphinxAtStartPar
DATA
&&&&&&&\\
\sphinxhline
\sphinxAtStartPar
15
&
\sphinxAtStartPar
14
&
\sphinxAtStartPar
13
&
\sphinxAtStartPar
12
&
\sphinxAtStartPar
11
&
\sphinxAtStartPar
10
&
\sphinxAtStartPar
9
&
\sphinxAtStartPar
8
\\
\sphinxhline
\sphinxAtStartPar
DATA
&&&&&&&\\
\sphinxhline
\sphinxAtStartPar
7
&
\sphinxAtStartPar
6
&
\sphinxAtStartPar
5
&
\sphinxAtStartPar
4
&
\sphinxAtStartPar
3
&
\sphinxAtStartPar
2
&
\sphinxAtStartPar
1
&
\sphinxAtStartPar
0
\\
\sphinxhline
\sphinxAtStartPar
DATA
&&&&&&&\\
\sphinxbottomrule
\end{tabular}
\sphinxtableafterendhook\par
\sphinxattableend\end{savenotes}


\begin{savenotes}\sphinxattablestart
\sphinxthistablewithglobalstyle
\centering
\begin{tabular}[t]{\X{33}{99}\X{33}{99}\X{33}{99}}
\sphinxtoprule
\sphinxtableatstartofbodyhook
\sphinxAtStartPar
位域 |
&
\sphinxAtStartPar
名称     | |
&
\sphinxAtStartPar
描述                                        | |
\\
\sphinxhline
\sphinxAtStartPar
31:0
&
\sphinxAtStartPar
DATA3
&
\sphinxAtStartPar
LCD数据寄存器3                              |
\\
\sphinxbottomrule
\end{tabular}
\sphinxtableafterendhook\par
\sphinxattableend\end{savenotes}

\sphinxstepscope


\section{SAFETY模块(SAFETY)}
\label{\detokenize{SWM241/_u529f_u80fd_u63cf_u8ff0/SAFETY_u6a21_u5757:safety-safety}}\label{\detokenize{SWM241/_u529f_u80fd_u63cf_u8ff0/SAFETY_u6a21_u5757::doc}}
\sphinxAtStartPar
概述
\textasciitilde{}\textasciitilde{}

\sphinxAtStartPar
SWM241系列所有型号SAFETY模块操作均相同。使用前需使能SAFETY模块时钟。

\sphinxAtStartPar
特性
\textasciitilde{}\textasciitilde{}
\begin{itemize}
\item {} 
\sphinxAtStartPar
支持通过APB总线进行配置

\item {} 
\sphinxAtStartPar
安全使能控制的模块包括:
\begin{itemize}
\item {} 
\sphinxAtStartPar
模拟电路配置写保护

\item {} 
\sphinxAtStartPar
时钟配置写保护

\item {} 
\sphinxAtStartPar
IO配置写保护

\item {} 
\sphinxAtStartPar
各模块中断使能写保护

\item {} 
\sphinxAtStartPar
非法地址访问检测

\end{itemize}

\item {} 
\sphinxAtStartPar
非法访问检测
\begin{itemize}
\item {} 
\sphinxAtStartPar
支持4个Region

\item {} 
\sphinxAtStartPar
每个Region非法访问触发时可选择产生复位或中断

\end{itemize}

\end{itemize}


\subsection{模块结构框图}
\label{\detokenize{SWM241/_u529f_u80fd_u63cf_u8ff0/SAFETY_u6a21_u5757:id1}}
\sphinxAtStartPar
\sphinxincludegraphics{{SWM241/功能描述/mediaSAFETY002}.emf}

\sphinxAtStartPar
图 6‑69 SAFETY模块结构框图


\subsection{功能描述}
\label{\detokenize{SWM241/_u529f_u80fd_u63cf_u8ff0/SAFETY_u6a21_u5757:id2}}

\subsubsection{RAM写保护}
\label{\detokenize{SWM241/_u529f_u80fd_u63cf_u8ff0/SAFETY_u6a21_u5757:ram}}
\sphinxAtStartPar
RAM写保护支持以256 Bytes为单位进行,最大RAM写保护范围为32K,不同封装RAM大小可能不同,请以具体芯片型号为准。

\sphinxAtStartPar
写保护空间大小支持256Bytes、512Bytes、1024Bytes。


\subsubsection{模拟电路配置寄存器写保护}
\label{\detokenize{SWM241/_u529f_u80fd_u63cf_u8ff0/SAFETY_u6a21_u5757:id3}}
\sphinxAtStartPar
模拟电路配置寄存器写保护模块包括:HRCCR、LRCCR、BODCR、XTALCR。

\sphinxAtStartPar
当safety使能寄存器中模拟电路配置寄存器写保护控制位为1时,模拟电路配置寄存器不可写,只能读,防止程序运行过程中改写模拟电路配置。


\subsubsection{时钟配置寄存器写保护}
\label{\detokenize{SWM241/_u529f_u80fd_u63cf_u8ff0/SAFETY_u6a21_u5757:id4}}
\sphinxAtStartPar
时钟配置寄存器写保护模块包括:CLKSEL、CLKDIV、CLKEN0、CLKEN1。

\sphinxAtStartPar
当safety使能寄存器中时钟配置寄存器写保护控制位为1时,时钟配置寄存器不可写,只能读,防止程序运行过程中改写配置。


\subsubsection{IO配置寄存器写保护}
\label{\detokenize{SWM241/_u529f_u80fd_u63cf_u8ff0/SAFETY_u6a21_u5757:io}}
\sphinxAtStartPar
IO配置寄存器写保护模块包括:PORTX\_FUNCX、PULLU\_X、PULLD\_X、INEN\_X、OPEND\_X。

\sphinxAtStartPar
当safety使能寄存器中IO配置寄存器写保护控制位为1时,IO配置寄存器不可写,只能读,防止程序运行过程中改写配置。


\subsubsection{中断使能寄存器写保护}
\label{\detokenize{SWM241/_u529f_u80fd_u63cf_u8ff0/SAFETY_u6a21_u5757:id5}}
\sphinxAtStartPar
中断使能寄存器写保护包括所有模块的中断使能寄存器。当safety使能寄存器中中断使能寄存器写保护控制位为1时,所有模块的中断使能寄存器不可写,只可读,防止程序运行过程中改写配置。


\subsubsection{非法地址访问检测}
\label{\detokenize{SWM241/_u529f_u80fd_u63cf_u8ff0/SAFETY_u6a21_u5757:id6}}
\sphinxAtStartPar
当safety使能寄存器中非法地址访问检测使能位为1时,safety模块启动对非法地址访问的检测,此时IW相关的寄存器生效。


\subsubsection{中断配置与清除}
\label{\detokenize{SWM241/_u529f_u80fd_u63cf_u8ff0/SAFETY_u6a21_u5757:id7}}
\sphinxAtStartPar
Safety模块中,当通过非法访问复位中断选择寄存器配置为异常访问产生中断时,可通过配置中断使能寄存器IE相应使能位使能中断,此时,当产生非法访问,中断标志寄存器IF相应位中置1。如需清除此标志,需在相应标志位中写1清零(R/W1C),否则中断在开启状态下会一直进入。


\subsection{寄存器映射}
\label{\detokenize{SWM241/_u529f_u80fd_u63cf_u8ff0/SAFETY_u6a21_u5757:id8}}

\begin{savenotes}\sphinxattablestart
\sphinxthistablewithglobalstyle
\centering
\begin{tabular}[t]{\X{20}{100}\X{20}{100}\X{20}{100}\X{20}{100}\X{20}{100}}
\sphinxtoprule
\sphinxtableatstartofbodyhook
\sphinxAtStartPar
名称   |
&
\begin{DUlineblock}{0em}
\item[] 偏移 |
\end{DUlineblock}
&
\begin{DUlineblock}{0em}
\item[] 
\item[] |
|
\end{DUlineblock}
&
\begin{DUlineblock}{0em}
\item[] 
\end{DUlineblock}
\begin{quote}

\begin{DUlineblock}{0em}
\item[] 
\item[] 
\end{DUlineblock}
\end{quote}
&
\sphinxAtStartPar
描述                       | | | |
\\
\sphinxhline
\sphinxAtStartPar
SAFETYBASE:0 {\color{red}\bfseries{}|}x400AB000
&
\begin{DUlineblock}{0em}
\item[] 
\end{DUlineblock}
&&&\\
\sphinxhline
\sphinxAtStartPar
PERWP
&
\sphinxAtStartPar
0x0
&&
\sphinxAtStartPar
0x 00000
&
\sphinxAtStartPar
外设写保护使能寄存器       |
\\
\sphinxhline
\sphinxAtStartPar
RAMWP
&
\sphinxAtStartPar
0x4
&&
\sphinxAtStartPar
0x 00000
&
\sphinxAtStartPar
RAM写保护配置寄存器        |
\\
\sphinxhline
\sphinxAtStartPar
IAACR
&
\sphinxAtStartPar
0x8
&&
\sphinxAtStartPar
0x 00000
&
\sphinxAtStartPar
非法访问复位中断选择寄存器 |
\\
\sphinxhline
\sphinxAtStartPar
IF
&
\sphinxAtStartPar
0x0C
&&
\sphinxAtStartPar
0x 00000
&
\sphinxAtStartPar
中断标志寄存器             |
\\
\sphinxhline
\sphinxAtStartPar
IE
&
\sphinxAtStartPar
0x10
&&
\sphinxAtStartPar
0x 00000
&
\sphinxAtStartPar
中断使能寄存器             |
\\
\sphinxhline
\sphinxAtStartPar
BADDR0
&
\sphinxAtStartPar
0x20
&&
\sphinxAtStartPar
0x 10000
&
\sphinxAtStartPar
CPU非法访问底地址寄存器0   |
\\
\sphinxhline
\sphinxAtStartPar
TADDR0
&
\sphinxAtStartPar
0x24
&&
\sphinxAtStartPar
0x 00000
&
\sphinxAtStartPar
CPU非法访问顶地址寄存器0   |
\\
\sphinxhline
\sphinxAtStartPar
BADDR1
&
\sphinxAtStartPar
0x28
&&
\sphinxAtStartPar
0x 80000
&
\sphinxAtStartPar
CPU非法访问底地址寄存器1   |
\\
\sphinxhline
\sphinxAtStartPar
TADDR1
&
\sphinxAtStartPar
0x2C
&&
\sphinxAtStartPar
0x 00000
&
\sphinxAtStartPar
CPU非法访问顶地址寄存器1   |
\\
\sphinxhline
\sphinxAtStartPar
BADDR2
&
\sphinxAtStartPar
0x30
&&
\sphinxAtStartPar
0x 04000
&
\sphinxAtStartPar
CPU非法访问底地址寄存器2   |
\\
\sphinxhline
\sphinxAtStartPar
TADDR2
&
\sphinxAtStartPar
0x34
&&
\sphinxAtStartPar
0x 00000
&
\sphinxAtStartPar
CPU非法访问顶地址寄存器2   |
\\
\sphinxhline
\sphinxAtStartPar
BADDR3
&
\sphinxAtStartPar
0x38
&&
\sphinxAtStartPar
0x 00000
&
\sphinxAtStartPar
CPU非法访问底地址寄存器3   |
\\
\sphinxhline
\sphinxAtStartPar
TADDR3
&
\sphinxAtStartPar
0x3C
&&
\sphinxAtStartPar
0x 00000
&
\sphinxAtStartPar
CPU非法访问顶地址寄存器3   |
\\
\sphinxbottomrule
\end{tabular}
\sphinxtableafterendhook\par
\sphinxattableend\end{savenotes}


\subsection{寄存器描述}
\label{\detokenize{SWM241/_u529f_u80fd_u63cf_u8ff0/SAFETY_u6a21_u5757:id11}}

\subsubsection{外设写保护使能寄存器PERWP}
\label{\detokenize{SWM241/_u529f_u80fd_u63cf_u8ff0/SAFETY_u6a21_u5757:perwp}}

\begin{savenotes}\sphinxattablestart
\sphinxthistablewithglobalstyle
\centering
\begin{tabular}[t]{\X{20}{100}\X{20}{100}\X{20}{100}\X{20}{100}\X{20}{100}}
\sphinxtoprule
\sphinxtableatstartofbodyhook
\sphinxAtStartPar
寄存器 |
&
\begin{DUlineblock}{0em}
\item[] 偏移 |
\end{DUlineblock}
&
\begin{DUlineblock}{0em}
\item[] 
\item[] {\color{red}\bfseries{}|}
\end{DUlineblock}
&
\sphinxAtStartPar
复位值 |    描 | |
&
\begin{DUlineblock}{0em}
\item[] |
  |
\end{DUlineblock}
\\
\sphinxhline
\sphinxAtStartPar
PERWP
&
\sphinxAtStartPar
0x0
&&
\sphinxAtStartPar
0 000000
&
\sphinxAtStartPar
外设写保护使能寄存器       |
\\
\sphinxbottomrule
\end{tabular}
\sphinxtableafterendhook\par
\sphinxattableend\end{savenotes}


\begin{savenotes}\sphinxattablestart
\sphinxthistablewithglobalstyle
\centering
\begin{tabular}[t]{\X{12}{96}\X{12}{96}\X{12}{96}\X{12}{96}\X{12}{96}\X{12}{96}\X{12}{96}\X{12}{96}}
\sphinxtoprule
\sphinxtableatstartofbodyhook
\sphinxAtStartPar
31
&
\sphinxAtStartPar
30
&
\sphinxAtStartPar
29
&
\sphinxAtStartPar
28
&
\sphinxAtStartPar
27
&
\sphinxAtStartPar
26
&
\sphinxAtStartPar
25
&
\sphinxAtStartPar
24
\\
\sphinxhline\begin{itemize}
\item {} 
\end{itemize}
&&&&&&&\\
\sphinxhline
\sphinxAtStartPar
23
&
\sphinxAtStartPar
22
&
\sphinxAtStartPar
21
&
\sphinxAtStartPar
20
&
\sphinxAtStartPar
19
&
\sphinxAtStartPar
18
&
\sphinxAtStartPar
17
&
\sphinxAtStartPar
16
\\
\sphinxhline\begin{itemize}
\item {} 
\end{itemize}
&&&&&&&\\
\sphinxhline
\sphinxAtStartPar
15
&
\sphinxAtStartPar
14
&
\sphinxAtStartPar
13
&
\sphinxAtStartPar
12
&
\sphinxAtStartPar
11
&
\sphinxAtStartPar
10
&
\sphinxAtStartPar
9
&
\sphinxAtStartPar
8
\\
\sphinxhline\begin{itemize}
\item {} 
\end{itemize}
&&&&&&&\\
\sphinxhline
\sphinxAtStartPar
7
&
\sphinxAtStartPar
6
&
\sphinxAtStartPar
5
&
\sphinxAtStartPar
4
&
\sphinxAtStartPar
3
&
\sphinxAtStartPar
2
&
\sphinxAtStartPar
1
&
\sphinxAtStartPar
0
\\
\sphinxhline\begin{itemize}
\item {} 
\end{itemize}
&&&
\sphinxAtStartPar
A FGR
&
\sphinxAtStartPar
CL GR
&&&\\
\sphinxbottomrule
\end{tabular}
\sphinxtableafterendhook\par
\sphinxattableend\end{savenotes}


\begin{savenotes}\sphinxattablestart
\sphinxthistablewithglobalstyle
\centering
\begin{tabular}[t]{\X{33}{99}\X{33}{99}\X{33}{99}}
\sphinxtoprule
\sphinxtableatstartofbodyhook
\sphinxAtStartPar
位域 |
&
\sphinxAtStartPar
名称     | |
&
\sphinxAtStartPar
描述                                        | |
\\
\sphinxhline
\sphinxAtStartPar
31:5
&\begin{itemize}
\item {} 
\end{itemize}
&\begin{itemize}
\item {} 
\end{itemize}
\\
\sphinxhline
\sphinxAtStartPar
4
&
\sphinxAtStartPar
ANACFGR
&
\sphinxAtStartPar
模拟电路配置寄存器写保护控制                |

\sphinxAtStartPar
0:模拟电路配置寄存器可写                   |

\sphinxAtStartPar
1:模拟电路配置寄存器不可写,只能读         |

\sphinxAtStartPar
包括:HRCCR、LRCCR、BODCR、XTALCR           |
\\
\sphinxhline
\sphinxAtStartPar
3
&
\sphinxAtStartPar
CLKCFGR
&
\sphinxAtStartPar
时钟配置寄存器写保护控制                    |

\sphinxAtStartPar
0:时钟配置寄存器可写                       |

\sphinxAtStartPar
1:时钟配置寄存器不可写,只能读             |

\sphinxAtStartPar
包括:CLKSEL、CLKDIV、CLKEN0、CLKEN1        |
\\
\sphinxhline
\sphinxAtStartPar
2
&
\sphinxAtStartPar
IOCFGR
&
\sphinxAtStartPar
IO配置寄存器写保护控制                      |

\sphinxAtStartPar
0:IO配置寄存器可写                         |

\sphinxAtStartPar
1:IO配置寄存器不可写,只能读               |

\sphinxAtStartPar
包括:                                      | TX\_FUNCX、PULLU\_X、PULLD\_X、INEN\_X、OPEND\_X |
\\
\sphinxhline
\sphinxAtStartPar
1
&
\sphinxAtStartPar
IER
&
\sphinxAtStartPar
中断使能寄存器写保护控制                    |

\sphinxAtStartPar
0:所有模块的中断使能寄存器可写。           |

\sphinxAtStartPar
1:所有模块的中断使能寄存器不可写,只可读。 |
\\
\sphinxhline
\sphinxAtStartPar
0
&
\sphinxAtStartPar
IAADEN
&
\sphinxAtStartPar
非法地址访问检测使能                        |

\sphinxAtStartPar
0:非法地址访问检测不使能                   |

\sphinxAtStartPar
1:非法地址访问检测使能                     |
\\
\sphinxbottomrule
\end{tabular}
\sphinxtableafterendhook\par
\sphinxattableend\end{savenotes}


\subsubsection{RAM写保护配置寄存器RAMWP}
\label{\detokenize{SWM241/_u529f_u80fd_u63cf_u8ff0/SAFETY_u6a21_u5757:ramramwp}}

\begin{savenotes}\sphinxattablestart
\sphinxthistablewithglobalstyle
\centering
\begin{tabular}[t]{\X{20}{100}\X{20}{100}\X{20}{100}\X{20}{100}\X{20}{100}}
\sphinxtoprule
\sphinxtableatstartofbodyhook
\sphinxAtStartPar
寄存器 |
&
\begin{DUlineblock}{0em}
\item[] 偏移 |
\end{DUlineblock}
&
\begin{DUlineblock}{0em}
\item[] 
\item[] {\color{red}\bfseries{}|}
\end{DUlineblock}
&
\sphinxAtStartPar
复位值 |    描 | |
&
\begin{DUlineblock}{0em}
\item[] |
  |
\end{DUlineblock}
\\
\sphinxhline
\sphinxAtStartPar
RAMWP
&
\sphinxAtStartPar
0x4
&&
\sphinxAtStartPar
0 000000
&
\sphinxAtStartPar
RAM写保护配置寄存器        |
\\
\sphinxbottomrule
\end{tabular}
\sphinxtableafterendhook\par
\sphinxattableend\end{savenotes}


\begin{savenotes}\sphinxattablestart
\sphinxthistablewithglobalstyle
\centering
\begin{tabular}[t]{\X{12}{96}\X{12}{96}\X{12}{96}\X{12}{96}\X{12}{96}\X{12}{96}\X{12}{96}\X{12}{96}}
\sphinxtoprule
\sphinxtableatstartofbodyhook
\sphinxAtStartPar
31
&
\sphinxAtStartPar
30
&
\sphinxAtStartPar
29
&
\sphinxAtStartPar
28
&
\sphinxAtStartPar
27
&
\sphinxAtStartPar
26
&
\sphinxAtStartPar
25
&
\sphinxAtStartPar
24
\\
\sphinxhline\begin{itemize}
\item {} 
\end{itemize}
&&&&&&&\\
\sphinxhline
\sphinxAtStartPar
23
&
\sphinxAtStartPar
22
&
\sphinxAtStartPar
21
&
\sphinxAtStartPar
20
&
\sphinxAtStartPar
19
&
\sphinxAtStartPar
18
&
\sphinxAtStartPar
17
&
\sphinxAtStartPar
16
\\
\sphinxhline\begin{itemize}
\item {} 
\end{itemize}
&&&&&&&\\
\sphinxhline
\sphinxAtStartPar
15
&
\sphinxAtStartPar
14
&
\sphinxAtStartPar
13
&
\sphinxAtStartPar
12
&
\sphinxAtStartPar
11
&
\sphinxAtStartPar
10
&
\sphinxAtStartPar
9
&
\sphinxAtStartPar
8
\\
\sphinxhline\begin{itemize}
\item {} 
\end{itemize}
&
\sphinxAtStartPar
ADDR
&&&&&&\\
\sphinxhline
\sphinxAtStartPar
7
&
\sphinxAtStartPar
6
&
\sphinxAtStartPar
5
&
\sphinxAtStartPar
4
&
\sphinxAtStartPar
3
&
\sphinxAtStartPar
2
&
\sphinxAtStartPar
1
&
\sphinxAtStartPar
0
\\
\sphinxhline\begin{itemize}
\item {} 
\end{itemize}
&&&&&&&\\
\sphinxbottomrule
\end{tabular}
\sphinxtableafterendhook\par
\sphinxattableend\end{savenotes}


\begin{savenotes}\sphinxattablestart
\sphinxthistablewithglobalstyle
\centering
\begin{tabular}[t]{\X{33}{99}\X{33}{99}\X{33}{99}}
\sphinxtoprule
\sphinxtableatstartofbodyhook
\sphinxAtStartPar
位域 |
&
\sphinxAtStartPar
名称     | |
&
\sphinxAtStartPar
描述                                        | |
\\
\sphinxhline
\sphinxAtStartPar
31:15
&\begin{itemize}
\item {} 
\end{itemize}
&\begin{itemize}
\item {} 
\end{itemize}
\\
\sphinxhline
\sphinxAtStartPar
14:8
&
\sphinxAtStartPar
ADDR
&
\sphinxAtStartPar
RAM写保护的起始地址设置寄存器               |

\sphinxAtStartPar
该寄存器配置的地址是以256 Bytes为单位的。   |

\sphinxAtStartPar
0:表示起始地址为RAM物理地址0x0000          |

\sphinxAtStartPar
1:表示起始地址为RAM物理地址0x0100          |

\sphinxAtStartPar
2:表示起始地址为RAM物理地址0x0200          |

\sphinxAtStartPar
3:表示起始地址为RAM物理地址0x0300          |

\sphinxAtStartPar
以此类推,最大起始地址可配置为32K Bytes\sphinxhyphen{}256 | Bytes。                                     |

\sphinxAtStartPar
当RAM最大为32K时,可配                      | 始地址为RAM物理地址0x7F00(0x8000\sphinxhyphen{}0x100)  |

\sphinxAtStartPar
在32K RAM的范围内,以256                    | Bytes为单位可进行写保护。                   |
\\
\sphinxhline
\sphinxAtStartPar
7:2
&\begin{itemize}
\item {} 
\end{itemize}
&\begin{itemize}
\item {} 
\end{itemize}
\\
\sphinxhline
\sphinxAtStartPar
1:0
&
\sphinxAtStartPar
SIZE
&
\sphinxAtStartPar
RAM写保护空间控制                           |

\sphinxAtStartPar
00:无效,RAM的所有空间都可以写             |

\sphinxAtStartPar
01:从设定的起始地址起始的256               | Bytes被保护,不能写                         |

\sphinxAtStartPar
10:从设定的起始地址起始的512               | Bytes被保护,不能写                         |

\sphinxAtStartPar
11:从设定的起始地址起始的1024              | Bytes被保护,不能写                         |
\\
\sphinxbottomrule
\end{tabular}
\sphinxtableafterendhook\par
\sphinxattableend\end{savenotes}


\subsubsection{非法访问复位中断选择寄存器IAACR}
\label{\detokenize{SWM241/_u529f_u80fd_u63cf_u8ff0/SAFETY_u6a21_u5757:iaacr}}

\begin{savenotes}\sphinxattablestart
\sphinxthistablewithglobalstyle
\centering
\begin{tabular}[t]{\X{20}{100}\X{20}{100}\X{20}{100}\X{20}{100}\X{20}{100}}
\sphinxtoprule
\sphinxtableatstartofbodyhook
\sphinxAtStartPar
寄存器 |
&
\begin{DUlineblock}{0em}
\item[] 偏移 |
\end{DUlineblock}
&
\begin{DUlineblock}{0em}
\item[] 
\item[] {\color{red}\bfseries{}|}
\end{DUlineblock}
&
\sphinxAtStartPar
复位值 |    描 | |
&
\begin{DUlineblock}{0em}
\item[] |
  |
\end{DUlineblock}
\\
\sphinxhline
\sphinxAtStartPar
IAACR
&
\sphinxAtStartPar
0x8
&&
\sphinxAtStartPar
0 000000
&
\sphinxAtStartPar
非法访问复位中断选择寄存器 |
\\
\sphinxbottomrule
\end{tabular}
\sphinxtableafterendhook\par
\sphinxattableend\end{savenotes}


\begin{savenotes}\sphinxattablestart
\sphinxthistablewithglobalstyle
\centering
\begin{tabular}[t]{\X{12}{96}\X{12}{96}\X{12}{96}\X{12}{96}\X{12}{96}\X{12}{96}\X{12}{96}\X{12}{96}}
\sphinxtoprule
\sphinxtableatstartofbodyhook
\sphinxAtStartPar
31
&
\sphinxAtStartPar
30
&
\sphinxAtStartPar
29
&
\sphinxAtStartPar
28
&
\sphinxAtStartPar
27
&
\sphinxAtStartPar
26
&
\sphinxAtStartPar
25
&
\sphinxAtStartPar
24
\\
\sphinxhline\begin{itemize}
\item {} 
\end{itemize}
&&&&&&&\\
\sphinxhline
\sphinxAtStartPar
23
&
\sphinxAtStartPar
22
&
\sphinxAtStartPar
21
&
\sphinxAtStartPar
20
&
\sphinxAtStartPar
19
&
\sphinxAtStartPar
18
&
\sphinxAtStartPar
17
&
\sphinxAtStartPar
16
\\
\sphinxhline\begin{itemize}
\item {} 
\end{itemize}
&&&&&&&\\
\sphinxhline
\sphinxAtStartPar
15
&
\sphinxAtStartPar
14
&
\sphinxAtStartPar
13
&
\sphinxAtStartPar
12
&
\sphinxAtStartPar
11
&
\sphinxAtStartPar
10
&
\sphinxAtStartPar
9
&
\sphinxAtStartPar
8
\\
\sphinxhline\begin{itemize}
\item {} 
\end{itemize}
&&&&&&&\\
\sphinxhline
\sphinxAtStartPar
7
&
\sphinxAtStartPar
6
&
\sphinxAtStartPar
5
&
\sphinxAtStartPar
4
&
\sphinxAtStartPar
3
&
\sphinxAtStartPar
2
&
\sphinxAtStartPar
1
&
\sphinxAtStartPar
0
\\
\sphinxhline\begin{itemize}
\item {} 
\end{itemize}
&&&&&&&\\
\sphinxbottomrule
\end{tabular}
\sphinxtableafterendhook\par
\sphinxattableend\end{savenotes}


\begin{savenotes}\sphinxattablestart
\sphinxthistablewithglobalstyle
\centering
\begin{tabular}[t]{\X{33}{99}\X{33}{99}\X{33}{99}}
\sphinxtoprule
\sphinxtableatstartofbodyhook
\sphinxAtStartPar
位域 |
&
\sphinxAtStartPar
名称     | |
&
\sphinxAtStartPar
描述                                        | |
\\
\sphinxhline
\sphinxAtStartPar
31:4
&\begin{itemize}
\item {} 
\end{itemize}
&\begin{itemize}
\item {} 
\end{itemize}
\\
\sphinxhline
\sphinxAtStartPar
3
&
\sphinxAtStartPar
R3INT
&
\sphinxAtStartPar
CPU访问Region3区异常,复位中断选择。        |

\sphinxAtStartPar
0:异常访问产生复位                         |

\sphinxAtStartPar
1:异常访问产生中断                         |
\\
\sphinxhline
\sphinxAtStartPar
2
&
\sphinxAtStartPar
R2INT
&
\sphinxAtStartPar
CPU访问Region2区异常,复位中断选择。        |

\sphinxAtStartPar
0:异常访问产生复位                         |

\sphinxAtStartPar
1:异常访问产生中断                         |
\\
\sphinxhline
\sphinxAtStartPar
1
&
\sphinxAtStartPar
R1INT
&
\sphinxAtStartPar
CPU访问Region1区异常,复位中断选择。        |

\sphinxAtStartPar
0:异常访问产生复位                         |

\sphinxAtStartPar
1:异常访问产生中断                         |
\\
\sphinxhline
\sphinxAtStartPar
0
&
\sphinxAtStartPar
R0INT
&
\sphinxAtStartPar
CPU访问Region0区异常,复位中断选择。        |

\sphinxAtStartPar
0:异常访问产生复位                         |

\sphinxAtStartPar
1:异常访问产生中断                         |
\\
\sphinxbottomrule
\end{tabular}
\sphinxtableafterendhook\par
\sphinxattableend\end{savenotes}


\subsubsection{中断标志寄存器IF}
\label{\detokenize{SWM241/_u529f_u80fd_u63cf_u8ff0/SAFETY_u6a21_u5757:if}}

\begin{savenotes}\sphinxattablestart
\sphinxthistablewithglobalstyle
\centering
\begin{tabular}[t]{\X{20}{100}\X{20}{100}\X{20}{100}\X{20}{100}\X{20}{100}}
\sphinxtoprule
\sphinxtableatstartofbodyhook
\sphinxAtStartPar
寄存器 |
&
\begin{DUlineblock}{0em}
\item[] 偏移 |
\end{DUlineblock}
&
\begin{DUlineblock}{0em}
\item[] 
\item[] {\color{red}\bfseries{}|}
\end{DUlineblock}
&
\sphinxAtStartPar
复位值 |    描 | |
&
\begin{DUlineblock}{0em}
\item[] |
  |
\end{DUlineblock}
\\
\sphinxhline
\sphinxAtStartPar
IF
&
\sphinxAtStartPar
0x0C
&&
\sphinxAtStartPar
0 000000
&
\sphinxAtStartPar
中断标志寄存器             |
\\
\sphinxbottomrule
\end{tabular}
\sphinxtableafterendhook\par
\sphinxattableend\end{savenotes}


\begin{savenotes}\sphinxattablestart
\sphinxthistablewithglobalstyle
\centering
\begin{tabular}[t]{\X{12}{96}\X{12}{96}\X{12}{96}\X{12}{96}\X{12}{96}\X{12}{96}\X{12}{96}\X{12}{96}}
\sphinxtoprule
\sphinxtableatstartofbodyhook
\sphinxAtStartPar
31
&
\sphinxAtStartPar
30
&
\sphinxAtStartPar
29
&
\sphinxAtStartPar
28
&
\sphinxAtStartPar
27
&
\sphinxAtStartPar
26
&
\sphinxAtStartPar
25
&
\sphinxAtStartPar
24
\\
\sphinxhline\begin{itemize}
\item {} 
\end{itemize}
&&&&&&&\\
\sphinxhline
\sphinxAtStartPar
23
&
\sphinxAtStartPar
22
&
\sphinxAtStartPar
21
&
\sphinxAtStartPar
20
&
\sphinxAtStartPar
19
&
\sphinxAtStartPar
18
&
\sphinxAtStartPar
17
&
\sphinxAtStartPar
16
\\
\sphinxhline\begin{itemize}
\item {} 
\end{itemize}
&&&&&&&\\
\sphinxhline
\sphinxAtStartPar
15
&
\sphinxAtStartPar
14
&
\sphinxAtStartPar
13
&
\sphinxAtStartPar
12
&
\sphinxAtStartPar
11
&
\sphinxAtStartPar
10
&
\sphinxAtStartPar
9
&
\sphinxAtStartPar
8
\\
\sphinxhline\begin{itemize}
\item {} 
\end{itemize}
&&&&&&&\\
\sphinxhline
\sphinxAtStartPar
7
&
\sphinxAtStartPar
6
&
\sphinxAtStartPar
5
&
\sphinxAtStartPar
4
&
\sphinxAtStartPar
3
&
\sphinxAtStartPar
2
&
\sphinxAtStartPar
1
&
\sphinxAtStartPar
0
\\
\sphinxhline\begin{itemize}
\item {} 
\end{itemize}
&&&&
\sphinxAtStartPar
R3
&
\sphinxAtStartPar
R2
&
\sphinxAtStartPar
R1
&
\sphinxAtStartPar
R0
\\
\sphinxbottomrule
\end{tabular}
\sphinxtableafterendhook\par
\sphinxattableend\end{savenotes}


\begin{savenotes}\sphinxattablestart
\sphinxthistablewithglobalstyle
\centering
\begin{tabular}[t]{\X{33}{99}\X{33}{99}\X{33}{99}}
\sphinxtoprule
\sphinxtableatstartofbodyhook
\sphinxAtStartPar
位域 |
&
\sphinxAtStartPar
名称     | |
&
\sphinxAtStartPar
描述                                        | |
\\
\sphinxhline
\sphinxAtStartPar
31:4
&\begin{itemize}
\item {} 
\end{itemize}
&\begin{itemize}
\item {} 
\end{itemize}
\\
\sphinxhline
\sphinxAtStartPar
3
&
\sphinxAtStartPar
R3
&
\sphinxAtStartPar
CPU访问Region3区异常中断标志                |

\sphinxAtStartPar
0:无中断                                   |

\sphinxAtStartPar
1:有中断                                   |
\\
\sphinxhline
\sphinxAtStartPar
2
&
\sphinxAtStartPar
R2
&
\sphinxAtStartPar
CPU访问Region2区异常中断标志                |

\sphinxAtStartPar
0:无中断                                   |

\sphinxAtStartPar
1:有中断                                   |
\\
\sphinxhline
\sphinxAtStartPar
1
&
\sphinxAtStartPar
R1
&
\sphinxAtStartPar
CPU访问Region1区异常中断标志                |

\sphinxAtStartPar
0:无中断                                   |

\sphinxAtStartPar
1:有中断                                   |
\\
\sphinxhline
\sphinxAtStartPar
0
&
\sphinxAtStartPar
R0
&
\sphinxAtStartPar
CPU访问Region0区异常中断标志                |

\sphinxAtStartPar
0:无中断                                   |

\sphinxAtStartPar
1:有中断                                   |
\\
\sphinxbottomrule
\end{tabular}
\sphinxtableafterendhook\par
\sphinxattableend\end{savenotes}


\subsubsection{中断使能寄存器IE}
\label{\detokenize{SWM241/_u529f_u80fd_u63cf_u8ff0/SAFETY_u6a21_u5757:ie}}

\begin{savenotes}\sphinxattablestart
\sphinxthistablewithglobalstyle
\centering
\begin{tabular}[t]{\X{20}{100}\X{20}{100}\X{20}{100}\X{20}{100}\X{20}{100}}
\sphinxtoprule
\sphinxtableatstartofbodyhook
\sphinxAtStartPar
寄存器 |
&
\begin{DUlineblock}{0em}
\item[] 偏移 |
\end{DUlineblock}
&
\begin{DUlineblock}{0em}
\item[] 
\item[] {\color{red}\bfseries{}|}
\end{DUlineblock}
&
\sphinxAtStartPar
复位值 |    描 | |
&
\begin{DUlineblock}{0em}
\item[] |
  |
\end{DUlineblock}
\\
\sphinxhline
\sphinxAtStartPar
IE
&
\sphinxAtStartPar
0x10
&&
\sphinxAtStartPar
0 000000
&
\sphinxAtStartPar
中断使能寄存器             |
\\
\sphinxbottomrule
\end{tabular}
\sphinxtableafterendhook\par
\sphinxattableend\end{savenotes}


\begin{savenotes}\sphinxattablestart
\sphinxthistablewithglobalstyle
\centering
\begin{tabular}[t]{\X{12}{96}\X{12}{96}\X{12}{96}\X{12}{96}\X{12}{96}\X{12}{96}\X{12}{96}\X{12}{96}}
\sphinxtoprule
\sphinxtableatstartofbodyhook
\sphinxAtStartPar
31
&
\sphinxAtStartPar
30
&
\sphinxAtStartPar
29
&
\sphinxAtStartPar
28
&
\sphinxAtStartPar
27
&
\sphinxAtStartPar
26
&
\sphinxAtStartPar
25
&
\sphinxAtStartPar
24
\\
\sphinxhline\begin{itemize}
\item {} 
\end{itemize}
&&&&&&&\\
\sphinxhline
\sphinxAtStartPar
23
&
\sphinxAtStartPar
22
&
\sphinxAtStartPar
21
&
\sphinxAtStartPar
20
&
\sphinxAtStartPar
19
&
\sphinxAtStartPar
18
&
\sphinxAtStartPar
17
&
\sphinxAtStartPar
16
\\
\sphinxhline\begin{itemize}
\item {} 
\end{itemize}
&&&&&&&\\
\sphinxhline
\sphinxAtStartPar
15
&
\sphinxAtStartPar
14
&
\sphinxAtStartPar
13
&
\sphinxAtStartPar
12
&
\sphinxAtStartPar
11
&
\sphinxAtStartPar
10
&
\sphinxAtStartPar
9
&
\sphinxAtStartPar
8
\\
\sphinxhline\begin{itemize}
\item {} 
\end{itemize}
&&&&&&&\\
\sphinxhline
\sphinxAtStartPar
7
&
\sphinxAtStartPar
6
&
\sphinxAtStartPar
5
&
\sphinxAtStartPar
4
&
\sphinxAtStartPar
3
&
\sphinxAtStartPar
2
&
\sphinxAtStartPar
1
&
\sphinxAtStartPar
0
\\
\sphinxhline\begin{itemize}
\item {} 
\end{itemize}
&&&&
\sphinxAtStartPar
R3
&
\sphinxAtStartPar
R2
&
\sphinxAtStartPar
R1
&
\sphinxAtStartPar
R0
\\
\sphinxbottomrule
\end{tabular}
\sphinxtableafterendhook\par
\sphinxattableend\end{savenotes}


\begin{savenotes}\sphinxattablestart
\sphinxthistablewithglobalstyle
\centering
\begin{tabular}[t]{\X{33}{99}\X{33}{99}\X{33}{99}}
\sphinxtoprule
\sphinxtableatstartofbodyhook
\sphinxAtStartPar
位域 |
&
\sphinxAtStartPar
名称     | |
&
\sphinxAtStartPar
描述                                        | |
\\
\sphinxhline
\sphinxAtStartPar
31:4
&\begin{itemize}
\item {} 
\end{itemize}
&\begin{itemize}
\item {} 
\end{itemize}
\\
\sphinxhline
\sphinxAtStartPar
3
&
\sphinxAtStartPar
R3
&
\sphinxAtStartPar
CPU访问                                     | ion3区异常中断使能。当R3INT选择中断时有效。 |

\sphinxAtStartPar
0:中断不使能。                             |

\sphinxAtStartPar
1:中断使能。                               |
\\
\sphinxhline
\sphinxAtStartPar
2
&
\sphinxAtStartPar
R2
&
\sphinxAtStartPar
CPU访问                                     | ion2区异常中断使能。当R2INT选择中断时有效。 |

\sphinxAtStartPar
0:中断不使能。                             |

\sphinxAtStartPar
1:中断使能。                               |
\\
\sphinxhline
\sphinxAtStartPar
1
&
\sphinxAtStartPar
R1
&
\sphinxAtStartPar
CPU访问                                     | ion1区异常中断使能。当R1INT选择中断时有效。 |

\sphinxAtStartPar
0:中断不使能。                             |

\sphinxAtStartPar
1:中断使能。                               |
\\
\sphinxhline
\sphinxAtStartPar
0
&
\sphinxAtStartPar
R0
&
\sphinxAtStartPar
CPU访问                                     | ion0区异常中断使能。当R0INT选择中断时有效。 |

\sphinxAtStartPar
0:中断不使能。                             |

\sphinxAtStartPar
1:中断使能。                               |
\\
\sphinxbottomrule
\end{tabular}
\sphinxtableafterendhook\par
\sphinxattableend\end{savenotes}


\subsubsection{CPU非法访问底地址寄存器0 BADDR0}
\label{\detokenize{SWM241/_u529f_u80fd_u63cf_u8ff0/SAFETY_u6a21_u5757:cpu0-baddr0}}

\begin{savenotes}\sphinxattablestart
\sphinxthistablewithglobalstyle
\centering
\begin{tabular}[t]{\X{20}{100}\X{20}{100}\X{20}{100}\X{20}{100}\X{20}{100}}
\sphinxtoprule
\sphinxtableatstartofbodyhook
\sphinxAtStartPar
寄存器 |
&
\begin{DUlineblock}{0em}
\item[] 偏移 |
\end{DUlineblock}
&
\begin{DUlineblock}{0em}
\item[] 
\item[] {\color{red}\bfseries{}|}
\end{DUlineblock}
&
\sphinxAtStartPar
复位值 |    描 | |
&
\begin{DUlineblock}{0em}
\item[] |
  |
\end{DUlineblock}
\\
\sphinxhline
\sphinxAtStartPar
BADDR0
&
\sphinxAtStartPar
0x20
&&
\sphinxAtStartPar
0 010000
&
\sphinxAtStartPar
CPU非法访问底地址寄存器0   |
\\
\sphinxbottomrule
\end{tabular}
\sphinxtableafterendhook\par
\sphinxattableend\end{savenotes}


\begin{savenotes}\sphinxattablestart
\sphinxthistablewithglobalstyle
\centering
\begin{tabular}[t]{\X{12}{96}\X{12}{96}\X{12}{96}\X{12}{96}\X{12}{96}\X{12}{96}\X{12}{96}\X{12}{96}}
\sphinxtoprule
\sphinxtableatstartofbodyhook
\sphinxAtStartPar
31
&
\sphinxAtStartPar
30
&
\sphinxAtStartPar
29
&
\sphinxAtStartPar
28
&
\sphinxAtStartPar
27
&
\sphinxAtStartPar
26
&
\sphinxAtStartPar
25
&
\sphinxAtStartPar
24
\\
\sphinxhline
\sphinxAtStartPar
BADDR0
&&&&&&&\\
\sphinxhline
\sphinxAtStartPar
23
&
\sphinxAtStartPar
22
&
\sphinxAtStartPar
21
&
\sphinxAtStartPar
20
&
\sphinxAtStartPar
19
&
\sphinxAtStartPar
18
&
\sphinxAtStartPar
17
&
\sphinxAtStartPar
16
\\
\sphinxhline
\sphinxAtStartPar
BADDR0
&&&&&&&\\
\sphinxhline
\sphinxAtStartPar
15
&
\sphinxAtStartPar
14
&
\sphinxAtStartPar
13
&
\sphinxAtStartPar
12
&
\sphinxAtStartPar
11
&
\sphinxAtStartPar
10
&
\sphinxAtStartPar
9
&
\sphinxAtStartPar
8
\\
\sphinxhline
\sphinxAtStartPar
BADDR0
&&&&&&&\\
\sphinxhline
\sphinxAtStartPar
7
&
\sphinxAtStartPar
6
&
\sphinxAtStartPar
5
&
\sphinxAtStartPar
4
&
\sphinxAtStartPar
3
&
\sphinxAtStartPar
2
&
\sphinxAtStartPar
1
&
\sphinxAtStartPar
0
\\
\sphinxhline
\sphinxAtStartPar
Ignore
&&&&&&&\\
\sphinxbottomrule
\end{tabular}
\sphinxtableafterendhook\par
\sphinxattableend\end{savenotes}


\begin{savenotes}\sphinxattablestart
\sphinxthistablewithglobalstyle
\centering
\begin{tabular}[t]{\X{33}{99}\X{33}{99}\X{33}{99}}
\sphinxtoprule
\sphinxtableatstartofbodyhook
\sphinxAtStartPar
位域 |
&
\sphinxAtStartPar
名称     | |
&
\sphinxAtStartPar
描述                                        | |
\\
\sphinxhline
\sphinxAtStartPar
31:8
&
\sphinxAtStartPar
BADDR0
&
\sphinxAtStartPar
CPU非法访问地址窗口1底地址{[}30:8{]}位。        |
\\
\sphinxhline
\sphinxAtStartPar
7:0
&
\sphinxAtStartPar
Ignore
&
\sphinxAtStartPar
向这些位的写操作会被忽略,这些位总是为0。   |
\\
\sphinxbottomrule
\end{tabular}
\sphinxtableafterendhook\par
\sphinxattableend\end{savenotes}


\subsubsection{CPU非法访问顶地址寄存器0 TADDR0}
\label{\detokenize{SWM241/_u529f_u80fd_u63cf_u8ff0/SAFETY_u6a21_u5757:cpu0-taddr0}}

\begin{savenotes}\sphinxattablestart
\sphinxthistablewithglobalstyle
\centering
\begin{tabular}[t]{\X{20}{100}\X{20}{100}\X{20}{100}\X{20}{100}\X{20}{100}}
\sphinxtoprule
\sphinxtableatstartofbodyhook
\sphinxAtStartPar
寄存器 |
&
\begin{DUlineblock}{0em}
\item[] 偏移 |
\end{DUlineblock}
&
\begin{DUlineblock}{0em}
\item[] 
\item[] {\color{red}\bfseries{}|}
\end{DUlineblock}
&
\sphinxAtStartPar
复位值 |    描 | |
&
\begin{DUlineblock}{0em}
\item[] |
  |
\end{DUlineblock}
\\
\sphinxhline
\sphinxAtStartPar
TADDR0
&
\sphinxAtStartPar
0x24
&&
\sphinxAtStartPar
0 000000
&
\sphinxAtStartPar
CPU非法访问顶地址寄存器0   |
\\
\sphinxbottomrule
\end{tabular}
\sphinxtableafterendhook\par
\sphinxattableend\end{savenotes}


\begin{savenotes}\sphinxattablestart
\sphinxthistablewithglobalstyle
\centering
\begin{tabular}[t]{\X{12}{96}\X{12}{96}\X{12}{96}\X{12}{96}\X{12}{96}\X{12}{96}\X{12}{96}\X{12}{96}}
\sphinxtoprule
\sphinxtableatstartofbodyhook
\sphinxAtStartPar
31
&
\sphinxAtStartPar
30
&
\sphinxAtStartPar
29
&
\sphinxAtStartPar
28
&
\sphinxAtStartPar
27
&
\sphinxAtStartPar
26
&
\sphinxAtStartPar
25
&
\sphinxAtStartPar
24
\\
\sphinxhline
\sphinxAtStartPar
TADDR0
&&&&&&&\\
\sphinxhline
\sphinxAtStartPar
23
&
\sphinxAtStartPar
22
&
\sphinxAtStartPar
21
&
\sphinxAtStartPar
20
&
\sphinxAtStartPar
19
&
\sphinxAtStartPar
18
&
\sphinxAtStartPar
17
&
\sphinxAtStartPar
16
\\
\sphinxhline
\sphinxAtStartPar
TADDR0
&&&&&&&\\
\sphinxhline
\sphinxAtStartPar
15
&
\sphinxAtStartPar
14
&
\sphinxAtStartPar
13
&
\sphinxAtStartPar
12
&
\sphinxAtStartPar
11
&
\sphinxAtStartPar
10
&
\sphinxAtStartPar
9
&
\sphinxAtStartPar
8
\\
\sphinxhline
\sphinxAtStartPar
TADDR0
&&&&&&&\\
\sphinxhline
\sphinxAtStartPar
7
&
\sphinxAtStartPar
6
&
\sphinxAtStartPar
5
&
\sphinxAtStartPar
4
&
\sphinxAtStartPar
3
&
\sphinxAtStartPar
2
&
\sphinxAtStartPar
1
&
\sphinxAtStartPar
0
\\
\sphinxhline
\sphinxAtStartPar
Ignore
&&&&&&&\\
\sphinxbottomrule
\end{tabular}
\sphinxtableafterendhook\par
\sphinxattableend\end{savenotes}


\begin{savenotes}\sphinxattablestart
\sphinxthistablewithglobalstyle
\centering
\begin{tabular}[t]{\X{33}{99}\X{33}{99}\X{33}{99}}
\sphinxtoprule
\sphinxtableatstartofbodyhook
\sphinxAtStartPar
位域 |
&
\sphinxAtStartPar
名称     | |
&
\sphinxAtStartPar
描述                                        | |
\\
\sphinxhline
\sphinxAtStartPar
31:8
&
\sphinxAtStartPar
TADDR0
&
\sphinxAtStartPar
CPU非法访问地址窗口0顶地址{[}30:8{]}位。        |
\\
\sphinxhline
\sphinxAtStartPar
7:0
&
\sphinxAtStartPar
Ignore
&
\sphinxAtStartPar
向这些位的写操作会被忽略,这些位总是为0。   |
\\
\sphinxbottomrule
\end{tabular}
\sphinxtableafterendhook\par
\sphinxattableend\end{savenotes}


\subsubsection{CPU非法访问底地址寄存器1 BADDR1}
\label{\detokenize{SWM241/_u529f_u80fd_u63cf_u8ff0/SAFETY_u6a21_u5757:cpu1-baddr1}}

\begin{savenotes}\sphinxattablestart
\sphinxthistablewithglobalstyle
\centering
\begin{tabular}[t]{\X{20}{100}\X{20}{100}\X{20}{100}\X{20}{100}\X{20}{100}}
\sphinxtoprule
\sphinxtableatstartofbodyhook
\sphinxAtStartPar
寄存器 |
&
\begin{DUlineblock}{0em}
\item[] 偏移 |
\end{DUlineblock}
&
\begin{DUlineblock}{0em}
\item[] 
\item[] {\color{red}\bfseries{}|}
\end{DUlineblock}
&
\sphinxAtStartPar
复位值 |    描 | |
&
\begin{DUlineblock}{0em}
\item[] |
  |
\end{DUlineblock}
\\
\sphinxhline
\sphinxAtStartPar
BADDR1
&
\sphinxAtStartPar
0x28
&&
\sphinxAtStartPar
0 080000
&
\sphinxAtStartPar
CPU非法访问底地址寄存器1   |
\\
\sphinxbottomrule
\end{tabular}
\sphinxtableafterendhook\par
\sphinxattableend\end{savenotes}


\begin{savenotes}\sphinxattablestart
\sphinxthistablewithglobalstyle
\centering
\begin{tabular}[t]{\X{12}{96}\X{12}{96}\X{12}{96}\X{12}{96}\X{12}{96}\X{12}{96}\X{12}{96}\X{12}{96}}
\sphinxtoprule
\sphinxtableatstartofbodyhook
\sphinxAtStartPar
31
&
\sphinxAtStartPar
30
&
\sphinxAtStartPar
29
&
\sphinxAtStartPar
28
&
\sphinxAtStartPar
27
&
\sphinxAtStartPar
26
&
\sphinxAtStartPar
25
&
\sphinxAtStartPar
24
\\
\sphinxhline
\sphinxAtStartPar
BADDR1
&&&&&&&\\
\sphinxhline
\sphinxAtStartPar
23
&
\sphinxAtStartPar
22
&
\sphinxAtStartPar
21
&
\sphinxAtStartPar
20
&
\sphinxAtStartPar
19
&
\sphinxAtStartPar
18
&
\sphinxAtStartPar
17
&
\sphinxAtStartPar
16
\\
\sphinxhline
\sphinxAtStartPar
BADDR1
&&&&&&&\\
\sphinxhline
\sphinxAtStartPar
15
&
\sphinxAtStartPar
14
&
\sphinxAtStartPar
13
&
\sphinxAtStartPar
12
&
\sphinxAtStartPar
11
&
\sphinxAtStartPar
10
&
\sphinxAtStartPar
9
&
\sphinxAtStartPar
8
\\
\sphinxhline
\sphinxAtStartPar
BADDR1
&&&&&&&\\
\sphinxhline
\sphinxAtStartPar
7
&
\sphinxAtStartPar
6
&
\sphinxAtStartPar
5
&
\sphinxAtStartPar
4
&
\sphinxAtStartPar
3
&
\sphinxAtStartPar
2
&
\sphinxAtStartPar
1
&
\sphinxAtStartPar
0
\\
\sphinxhline
\sphinxAtStartPar
Ignore
&&&&&&&\\
\sphinxbottomrule
\end{tabular}
\sphinxtableafterendhook\par
\sphinxattableend\end{savenotes}


\begin{savenotes}\sphinxattablestart
\sphinxthistablewithglobalstyle
\centering
\begin{tabular}[t]{\X{33}{99}\X{33}{99}\X{33}{99}}
\sphinxtoprule
\sphinxtableatstartofbodyhook
\sphinxAtStartPar
位域 |
&
\sphinxAtStartPar
名称     | |
&
\sphinxAtStartPar
描述                                        | |
\\
\sphinxhline
\sphinxAtStartPar
31:8
&
\sphinxAtStartPar
BADDR1
&
\sphinxAtStartPar
CPU非法访问地址窗口1底地址{[}30:8{]}位。        |
\\
\sphinxhline
\sphinxAtStartPar
7:0
&
\sphinxAtStartPar
Ignore
&
\sphinxAtStartPar
向这些位的写操作会被忽略,这些位总是为0。   |
\\
\sphinxbottomrule
\end{tabular}
\sphinxtableafterendhook\par
\sphinxattableend\end{savenotes}


\subsubsection{CPU非法访问顶地址寄存器1 TADDR1}
\label{\detokenize{SWM241/_u529f_u80fd_u63cf_u8ff0/SAFETY_u6a21_u5757:cpu1-taddr1}}

\begin{savenotes}\sphinxattablestart
\sphinxthistablewithglobalstyle
\centering
\begin{tabular}[t]{\X{20}{100}\X{20}{100}\X{20}{100}\X{20}{100}\X{20}{100}}
\sphinxtoprule
\sphinxtableatstartofbodyhook
\sphinxAtStartPar
寄存器 |
&
\begin{DUlineblock}{0em}
\item[] 偏移 |
\end{DUlineblock}
&
\begin{DUlineblock}{0em}
\item[] 
\item[] {\color{red}\bfseries{}|}
\end{DUlineblock}
&
\sphinxAtStartPar
复位值 |    描 | |
&
\begin{DUlineblock}{0em}
\item[] |
  |
\end{DUlineblock}
\\
\sphinxhline
\sphinxAtStartPar
TADDR1
&
\sphinxAtStartPar
0x2C
&&
\sphinxAtStartPar
0 000000
&
\sphinxAtStartPar
CPU非法访问顶地址寄存器1   |
\\
\sphinxbottomrule
\end{tabular}
\sphinxtableafterendhook\par
\sphinxattableend\end{savenotes}


\begin{savenotes}\sphinxattablestart
\sphinxthistablewithglobalstyle
\centering
\begin{tabular}[t]{\X{12}{96}\X{12}{96}\X{12}{96}\X{12}{96}\X{12}{96}\X{12}{96}\X{12}{96}\X{12}{96}}
\sphinxtoprule
\sphinxtableatstartofbodyhook
\sphinxAtStartPar
31
&
\sphinxAtStartPar
30
&
\sphinxAtStartPar
29
&
\sphinxAtStartPar
28
&
\sphinxAtStartPar
27
&
\sphinxAtStartPar
26
&
\sphinxAtStartPar
25
&
\sphinxAtStartPar
24
\\
\sphinxhline
\sphinxAtStartPar
TADDR1
&&&&&&&\\
\sphinxhline
\sphinxAtStartPar
23
&
\sphinxAtStartPar
22
&
\sphinxAtStartPar
21
&
\sphinxAtStartPar
20
&
\sphinxAtStartPar
19
&
\sphinxAtStartPar
18
&
\sphinxAtStartPar
17
&
\sphinxAtStartPar
16
\\
\sphinxhline
\sphinxAtStartPar
TADDR1
&&&&&&&\\
\sphinxhline
\sphinxAtStartPar
15
&
\sphinxAtStartPar
14
&
\sphinxAtStartPar
13
&
\sphinxAtStartPar
12
&
\sphinxAtStartPar
11
&
\sphinxAtStartPar
10
&
\sphinxAtStartPar
9
&
\sphinxAtStartPar
8
\\
\sphinxhline
\sphinxAtStartPar
TADDR1
&&&&&&&\\
\sphinxhline
\sphinxAtStartPar
7
&
\sphinxAtStartPar
6
&
\sphinxAtStartPar
5
&
\sphinxAtStartPar
4
&
\sphinxAtStartPar
3
&
\sphinxAtStartPar
2
&
\sphinxAtStartPar
1
&
\sphinxAtStartPar
0
\\
\sphinxhline
\sphinxAtStartPar
Ignore
&&&&&&&\\
\sphinxbottomrule
\end{tabular}
\sphinxtableafterendhook\par
\sphinxattableend\end{savenotes}


\begin{savenotes}\sphinxattablestart
\sphinxthistablewithglobalstyle
\centering
\begin{tabular}[t]{\X{33}{99}\X{33}{99}\X{33}{99}}
\sphinxtoprule
\sphinxtableatstartofbodyhook
\sphinxAtStartPar
位域 |
&
\sphinxAtStartPar
名称     | |
&
\sphinxAtStartPar
描述                                        | |
\\
\sphinxhline
\sphinxAtStartPar
31:8
&
\sphinxAtStartPar
TADDR1
&
\sphinxAtStartPar
CPU非法访问地址窗口1顶地址{[}30:8{]}位。        |
\\
\sphinxhline
\sphinxAtStartPar
7:0
&
\sphinxAtStartPar
Ignore
&
\sphinxAtStartPar
向这些位的写操作会被忽略,这些位总是为0。   |
\\
\sphinxbottomrule
\end{tabular}
\sphinxtableafterendhook\par
\sphinxattableend\end{savenotes}


\subsubsection{CPU非法访问底地址寄存器2 BADDR2}
\label{\detokenize{SWM241/_u529f_u80fd_u63cf_u8ff0/SAFETY_u6a21_u5757:cpu2-baddr2}}

\begin{savenotes}\sphinxattablestart
\sphinxthistablewithglobalstyle
\centering
\begin{tabular}[t]{\X{20}{100}\X{20}{100}\X{20}{100}\X{20}{100}\X{20}{100}}
\sphinxtoprule
\sphinxtableatstartofbodyhook
\sphinxAtStartPar
寄存器 |
&
\begin{DUlineblock}{0em}
\item[] 偏移 |
\end{DUlineblock}
&
\begin{DUlineblock}{0em}
\item[] 
\item[] {\color{red}\bfseries{}|}
\end{DUlineblock}
&
\sphinxAtStartPar
复位值 |    描 | |
&
\begin{DUlineblock}{0em}
\item[] |
  |
\end{DUlineblock}
\\
\sphinxhline
\sphinxAtStartPar
BADDR2
&
\sphinxAtStartPar
0x30
&&
\sphinxAtStartPar
0 004000
&
\sphinxAtStartPar
CPU非法访问底地址寄存器2   |
\\
\sphinxbottomrule
\end{tabular}
\sphinxtableafterendhook\par
\sphinxattableend\end{savenotes}


\begin{savenotes}\sphinxattablestart
\sphinxthistablewithglobalstyle
\centering
\begin{tabular}[t]{\X{12}{96}\X{12}{96}\X{12}{96}\X{12}{96}\X{12}{96}\X{12}{96}\X{12}{96}\X{12}{96}}
\sphinxtoprule
\sphinxtableatstartofbodyhook
\sphinxAtStartPar
31
&
\sphinxAtStartPar
30
&
\sphinxAtStartPar
29
&
\sphinxAtStartPar
28
&
\sphinxAtStartPar
27
&
\sphinxAtStartPar
26
&
\sphinxAtStartPar
25
&
\sphinxAtStartPar
24
\\
\sphinxhline
\sphinxAtStartPar
BADDR2
&&&&&&&\\
\sphinxhline
\sphinxAtStartPar
23
&
\sphinxAtStartPar
22
&
\sphinxAtStartPar
21
&
\sphinxAtStartPar
20
&
\sphinxAtStartPar
19
&
\sphinxAtStartPar
18
&
\sphinxAtStartPar
17
&
\sphinxAtStartPar
16
\\
\sphinxhline
\sphinxAtStartPar
BADDR2
&&&&&&&\\
\sphinxhline
\sphinxAtStartPar
15
&
\sphinxAtStartPar
14
&
\sphinxAtStartPar
13
&
\sphinxAtStartPar
12
&
\sphinxAtStartPar
11
&
\sphinxAtStartPar
10
&
\sphinxAtStartPar
9
&
\sphinxAtStartPar
8
\\
\sphinxhline
\sphinxAtStartPar
BADDR2
&&&&&&&\\
\sphinxhline
\sphinxAtStartPar
7
&
\sphinxAtStartPar
6
&
\sphinxAtStartPar
5
&
\sphinxAtStartPar
4
&
\sphinxAtStartPar
3
&
\sphinxAtStartPar
2
&
\sphinxAtStartPar
1
&
\sphinxAtStartPar
0
\\
\sphinxhline
\sphinxAtStartPar
Ignore
&&&&&&&\\
\sphinxbottomrule
\end{tabular}
\sphinxtableafterendhook\par
\sphinxattableend\end{savenotes}


\begin{savenotes}\sphinxattablestart
\sphinxthistablewithglobalstyle
\centering
\begin{tabular}[t]{\X{33}{99}\X{33}{99}\X{33}{99}}
\sphinxtoprule
\sphinxtableatstartofbodyhook
\sphinxAtStartPar
位域 |
&
\sphinxAtStartPar
名称     | |
&
\sphinxAtStartPar
描述                                        | |
\\
\sphinxhline
\sphinxAtStartPar
31:8
&
\sphinxAtStartPar
BADDR2
&
\sphinxAtStartPar
CPU非法访问地址窗口2底地址{[}30:8{]}位。        |
\\
\sphinxhline
\sphinxAtStartPar
7:0
&
\sphinxAtStartPar
Ignore
&
\sphinxAtStartPar
向这些位的写操作会被忽略,这些位总是为0。   |
\\
\sphinxbottomrule
\end{tabular}
\sphinxtableafterendhook\par
\sphinxattableend\end{savenotes}


\subsubsection{CPU非法访问顶地址寄存器2 TADDR2}
\label{\detokenize{SWM241/_u529f_u80fd_u63cf_u8ff0/SAFETY_u6a21_u5757:cpu2-taddr2}}

\begin{savenotes}\sphinxattablestart
\sphinxthistablewithglobalstyle
\centering
\begin{tabular}[t]{\X{20}{100}\X{20}{100}\X{20}{100}\X{20}{100}\X{20}{100}}
\sphinxtoprule
\sphinxtableatstartofbodyhook
\sphinxAtStartPar
寄存器 |
&
\begin{DUlineblock}{0em}
\item[] 偏移 |
\end{DUlineblock}
&
\begin{DUlineblock}{0em}
\item[] 
\item[] {\color{red}\bfseries{}|}
\end{DUlineblock}
&
\sphinxAtStartPar
复位值 |    描 | |
&
\begin{DUlineblock}{0em}
\item[] |
  |
\end{DUlineblock}
\\
\sphinxhline
\sphinxAtStartPar
TADDR2
&
\sphinxAtStartPar
0x34
&&
\sphinxAtStartPar
0 000000
&
\sphinxAtStartPar
CPU非法访问顶地址寄存器2   |
\\
\sphinxbottomrule
\end{tabular}
\sphinxtableafterendhook\par
\sphinxattableend\end{savenotes}


\begin{savenotes}\sphinxattablestart
\sphinxthistablewithglobalstyle
\centering
\begin{tabular}[t]{\X{12}{96}\X{12}{96}\X{12}{96}\X{12}{96}\X{12}{96}\X{12}{96}\X{12}{96}\X{12}{96}}
\sphinxtoprule
\sphinxtableatstartofbodyhook
\sphinxAtStartPar
31
&
\sphinxAtStartPar
30
&
\sphinxAtStartPar
29
&
\sphinxAtStartPar
28
&
\sphinxAtStartPar
27
&
\sphinxAtStartPar
26
&
\sphinxAtStartPar
25
&
\sphinxAtStartPar
24
\\
\sphinxhline
\sphinxAtStartPar
TADDR2
&&&&&&&\\
\sphinxhline
\sphinxAtStartPar
23
&
\sphinxAtStartPar
22
&
\sphinxAtStartPar
21
&
\sphinxAtStartPar
20
&
\sphinxAtStartPar
19
&
\sphinxAtStartPar
18
&
\sphinxAtStartPar
17
&
\sphinxAtStartPar
16
\\
\sphinxhline
\sphinxAtStartPar
TADDR2
&&&&&&&\\
\sphinxhline
\sphinxAtStartPar
15
&
\sphinxAtStartPar
14
&
\sphinxAtStartPar
13
&
\sphinxAtStartPar
12
&
\sphinxAtStartPar
11
&
\sphinxAtStartPar
10
&
\sphinxAtStartPar
9
&
\sphinxAtStartPar
8
\\
\sphinxhline
\sphinxAtStartPar
TADDR2
&&&&&&&\\
\sphinxhline
\sphinxAtStartPar
7
&
\sphinxAtStartPar
6
&
\sphinxAtStartPar
5
&
\sphinxAtStartPar
4
&
\sphinxAtStartPar
3
&
\sphinxAtStartPar
2
&
\sphinxAtStartPar
1
&
\sphinxAtStartPar
0
\\
\sphinxhline
\sphinxAtStartPar
Ignore
&&&&&&&\\
\sphinxbottomrule
\end{tabular}
\sphinxtableafterendhook\par
\sphinxattableend\end{savenotes}


\begin{savenotes}\sphinxattablestart
\sphinxthistablewithglobalstyle
\centering
\begin{tabular}[t]{\X{33}{99}\X{33}{99}\X{33}{99}}
\sphinxtoprule
\sphinxtableatstartofbodyhook
\sphinxAtStartPar
位域 |
&
\sphinxAtStartPar
名称     | |
&
\sphinxAtStartPar
描述                                        | |
\\
\sphinxhline
\sphinxAtStartPar
31:8
&
\sphinxAtStartPar
TADDR2
&
\sphinxAtStartPar
CPU非法访问地址窗口2顶地址{[}30:8{]}位。        |
\\
\sphinxhline
\sphinxAtStartPar
7:0
&
\sphinxAtStartPar
Ignore
&
\sphinxAtStartPar
向这些位的写操作会被忽略,这些位总是为0。   |
\\
\sphinxbottomrule
\end{tabular}
\sphinxtableafterendhook\par
\sphinxattableend\end{savenotes}


\subsubsection{CPU非法访问底地址寄存器3 BADDR3}
\label{\detokenize{SWM241/_u529f_u80fd_u63cf_u8ff0/SAFETY_u6a21_u5757:cpu3-baddr3}}

\begin{savenotes}\sphinxattablestart
\sphinxthistablewithglobalstyle
\centering
\begin{tabular}[t]{\X{20}{100}\X{20}{100}\X{20}{100}\X{20}{100}\X{20}{100}}
\sphinxtoprule
\sphinxtableatstartofbodyhook
\sphinxAtStartPar
寄存器 |
&
\begin{DUlineblock}{0em}
\item[] 偏移 |
\end{DUlineblock}
&
\begin{DUlineblock}{0em}
\item[] 
\item[] {\color{red}\bfseries{}|}
\end{DUlineblock}
&
\sphinxAtStartPar
复位值 |    描 | |
&
\begin{DUlineblock}{0em}
\item[] |
  |
\end{DUlineblock}
\\
\sphinxhline
\sphinxAtStartPar
BADDR3
&
\sphinxAtStartPar
0x38
&&
\sphinxAtStartPar
0 100000
&
\sphinxAtStartPar
CPU非法访问底地址寄存器3   |
\\
\sphinxbottomrule
\end{tabular}
\sphinxtableafterendhook\par
\sphinxattableend\end{savenotes}


\begin{savenotes}\sphinxattablestart
\sphinxthistablewithglobalstyle
\centering
\begin{tabular}[t]{\X{12}{96}\X{12}{96}\X{12}{96}\X{12}{96}\X{12}{96}\X{12}{96}\X{12}{96}\X{12}{96}}
\sphinxtoprule
\sphinxtableatstartofbodyhook
\sphinxAtStartPar
31
&
\sphinxAtStartPar
30
&
\sphinxAtStartPar
29
&
\sphinxAtStartPar
28
&
\sphinxAtStartPar
27
&
\sphinxAtStartPar
26
&
\sphinxAtStartPar
25
&
\sphinxAtStartPar
24
\\
\sphinxhline
\sphinxAtStartPar
BADDR3
&&&&&&&\\
\sphinxhline
\sphinxAtStartPar
23
&
\sphinxAtStartPar
22
&
\sphinxAtStartPar
21
&
\sphinxAtStartPar
20
&
\sphinxAtStartPar
19
&
\sphinxAtStartPar
18
&
\sphinxAtStartPar
17
&
\sphinxAtStartPar
16
\\
\sphinxhline
\sphinxAtStartPar
BADDR3
&&&&&&&\\
\sphinxhline
\sphinxAtStartPar
15
&
\sphinxAtStartPar
14
&
\sphinxAtStartPar
13
&
\sphinxAtStartPar
12
&
\sphinxAtStartPar
11
&
\sphinxAtStartPar
10
&
\sphinxAtStartPar
9
&
\sphinxAtStartPar
8
\\
\sphinxhline
\sphinxAtStartPar
BADDR3
&&&&&&&\\
\sphinxhline
\sphinxAtStartPar
7
&
\sphinxAtStartPar
6
&
\sphinxAtStartPar
5
&
\sphinxAtStartPar
4
&
\sphinxAtStartPar
3
&
\sphinxAtStartPar
2
&
\sphinxAtStartPar
1
&
\sphinxAtStartPar
0
\\
\sphinxhline
\sphinxAtStartPar
Ignore
&&&&&&&\\
\sphinxbottomrule
\end{tabular}
\sphinxtableafterendhook\par
\sphinxattableend\end{savenotes}


\begin{savenotes}\sphinxattablestart
\sphinxthistablewithglobalstyle
\centering
\begin{tabular}[t]{\X{33}{99}\X{33}{99}\X{33}{99}}
\sphinxtoprule
\sphinxtableatstartofbodyhook
\sphinxAtStartPar
位域 |
&
\sphinxAtStartPar
名称     | |
&
\sphinxAtStartPar
描述                                        | |
\\
\sphinxhline
\sphinxAtStartPar
31:8
&
\sphinxAtStartPar
BADDR3
&
\sphinxAtStartPar
CPU非法访问地址窗口3底地址{[}30:8{]}位。        |
\\
\sphinxhline
\sphinxAtStartPar
7:0
&
\sphinxAtStartPar
Ignore
&
\sphinxAtStartPar
向这些位的写操作会被忽略,这些位总是为0。   |
\\
\sphinxbottomrule
\end{tabular}
\sphinxtableafterendhook\par
\sphinxattableend\end{savenotes}


\subsubsection{CPU非法访问顶地址寄存器3 TADDR3}
\label{\detokenize{SWM241/_u529f_u80fd_u63cf_u8ff0/SAFETY_u6a21_u5757:cpu3-taddr3}}

\begin{savenotes}\sphinxattablestart
\sphinxthistablewithglobalstyle
\centering
\begin{tabular}[t]{\X{20}{100}\X{20}{100}\X{20}{100}\X{20}{100}\X{20}{100}}
\sphinxtoprule
\sphinxtableatstartofbodyhook
\sphinxAtStartPar
寄存器 |
&
\begin{DUlineblock}{0em}
\item[] 偏移 |
\end{DUlineblock}
&
\begin{DUlineblock}{0em}
\item[] 
\item[] {\color{red}\bfseries{}|}
\end{DUlineblock}
&
\sphinxAtStartPar
复位值 |    描 | |
&
\begin{DUlineblock}{0em}
\item[] |
  |
\end{DUlineblock}
\\
\sphinxhline
\sphinxAtStartPar
BADDR3
&
\sphinxAtStartPar
0x3C
&&
\sphinxAtStartPar
0 000000
&
\sphinxAtStartPar
CPU非法访问底地址寄存器2   |
\\
\sphinxbottomrule
\end{tabular}
\sphinxtableafterendhook\par
\sphinxattableend\end{savenotes}


\begin{savenotes}\sphinxattablestart
\sphinxthistablewithglobalstyle
\centering
\begin{tabular}[t]{\X{12}{96}\X{12}{96}\X{12}{96}\X{12}{96}\X{12}{96}\X{12}{96}\X{12}{96}\X{12}{96}}
\sphinxtoprule
\sphinxtableatstartofbodyhook
\sphinxAtStartPar
31
&
\sphinxAtStartPar
30
&
\sphinxAtStartPar
29
&
\sphinxAtStartPar
28
&
\sphinxAtStartPar
27
&
\sphinxAtStartPar
26
&
\sphinxAtStartPar
25
&
\sphinxAtStartPar
24
\\
\sphinxhline
\sphinxAtStartPar
TADDR3
&&&&&&&\\
\sphinxhline
\sphinxAtStartPar
23
&
\sphinxAtStartPar
22
&
\sphinxAtStartPar
21
&
\sphinxAtStartPar
20
&
\sphinxAtStartPar
19
&
\sphinxAtStartPar
18
&
\sphinxAtStartPar
17
&
\sphinxAtStartPar
16
\\
\sphinxhline
\sphinxAtStartPar
TADDR3
&&&&&&&\\
\sphinxhline
\sphinxAtStartPar
15
&
\sphinxAtStartPar
14
&
\sphinxAtStartPar
13
&
\sphinxAtStartPar
12
&
\sphinxAtStartPar
11
&
\sphinxAtStartPar
10
&
\sphinxAtStartPar
9
&
\sphinxAtStartPar
8
\\
\sphinxhline
\sphinxAtStartPar
TADDR3
&&&&&&&\\
\sphinxhline
\sphinxAtStartPar
7
&
\sphinxAtStartPar
6
&
\sphinxAtStartPar
5
&
\sphinxAtStartPar
4
&
\sphinxAtStartPar
3
&
\sphinxAtStartPar
2
&
\sphinxAtStartPar
1
&
\sphinxAtStartPar
0
\\
\sphinxhline
\sphinxAtStartPar
Ignore
&&&&&&&\\
\sphinxbottomrule
\end{tabular}
\sphinxtableafterendhook\par
\sphinxattableend\end{savenotes}


\begin{savenotes}\sphinxattablestart
\sphinxthistablewithglobalstyle
\centering
\begin{tabular}[t]{\X{33}{99}\X{33}{99}\X{33}{99}}
\sphinxtoprule
\sphinxtableatstartofbodyhook
\sphinxAtStartPar
位域 |
&
\sphinxAtStartPar
名称     | |
&
\sphinxAtStartPar
描述                                        | |
\\
\sphinxhline
\sphinxAtStartPar
31:8
&
\sphinxAtStartPar
BADDR3
&
\sphinxAtStartPar
CPU非法访问地址窗口3底地址{[}30:8{]}位。        |
\\
\sphinxhline
\sphinxAtStartPar
7:0
&
\sphinxAtStartPar
Ignore
&
\sphinxAtStartPar
向这些位的写操作会被忽略,这些位总是为0。   |
\\
\sphinxbottomrule
\end{tabular}
\sphinxtableafterendhook\par
\sphinxattableend\end{savenotes}

\sphinxstepscope


\section{FLASH控制器与ISP操作}
\label{\detokenize{SWM241/_u529f_u80fd_u63cf_u8ff0/FLASH_u63a7_u5236_u5668_u4e0eISP_u64cd_u4f5c:flashisp}}\label{\detokenize{SWM241/_u529f_u80fd_u63cf_u8ff0/FLASH_u63a7_u5236_u5668_u4e0eISP_u64cd_u4f5c::doc}}
\sphinxAtStartPar
概述
\textasciitilde{}\textasciitilde{}

\sphinxAtStartPar
SWM241系列内置FLASH为串行SPI接口FLASH。通过调用IAP函数进行FLASH操作。

\sphinxAtStartPar
操作FLASH前,需要关闭中断,防止打断造成写入错误。

\sphinxAtStartPar
特性
\textasciitilde{}\textasciitilde{}
\begin{itemize}
\item {} 
\sphinxAtStartPar
支持ISP程序定制

\item {} 
\sphinxAtStartPar
支持FLASH编程

\item {} 
\sphinxAtStartPar
支持BOOT自定义

\item {} 
\sphinxAtStartPar
支持加密

\end{itemize}


\subsection{功能描述}
\label{\detokenize{SWM241/_u529f_u80fd_u63cf_u8ff0/FLASH_u63a7_u5236_u5668_u4e0eISP_u64cd_u4f5c:id1}}

\subsubsection{FLASH操作}
\label{\detokenize{SWM241/_u529f_u80fd_u63cf_u8ff0/FLASH_u63a7_u5236_u5668_u4e0eISP_u64cd_u4f5c:flash}}
\sphinxAtStartPar
FLASH操作可以通过寄存器进行操作,也可以通过IAP函数进行擦除及写入。

\sphinxAtStartPar
\sphinxstylestrong{寄存器操作}
\begin{itemize}
\item {} 
\sphinxAtStartPar
ERASE操作:
\begin{itemize}
\item {} 
\sphinxAtStartPar
使能FLASH擦写使能位

\item {} 
\sphinxAtStartPar
配置擦page的地址

\item {} 
\sphinxAtStartPar
查询ERASE位等待擦完成,直至状态从1变为0,擦除完成。当Flash完成擦除操作后,方可进行其他操作

\end{itemize}

\item {} 
\sphinxAtStartPar
PROGRAM 操作:

\item {} 
\sphinxAtStartPar
使能FLASH擦写使能位

\item {} 
\sphinxAtStartPar
配置FLASH写地址,必须字对齐

\item {} 
\sphinxAtStartPar
配置FLASH要写的数据

\item {} 
\sphinxAtStartPar
查询FLASHIDLE位等待写完成

\end{itemize}

\sphinxAtStartPar
\sphinxstyleemphasis{注1:以上操作流程均可在FLASH或SRAM中执行}

\sphinxAtStartPar
\sphinxstyleemphasis{注2:每PAGE为1K,每次最少写1word}

\sphinxAtStartPar
\sphinxstylestrong{IAP操作}

\sphinxAtStartPar
IAP 函数作为片内驻留程序,其提供了针对flash 的相关操作IAP 函数为Thumb 代码,分为擦除函数(驻留地址为0x1000400)和写入函数(驻留地址为0x1000450),建议使用如下方式调用:

\sphinxAtStartPar
擦除函数:

\sphinxAtStartPar
定义函数类型:

\sphinxAtStartPar
typedef uint32\_t (*IAPFunc1)(uint32\_t PageNum);

\sphinxAtStartPar
IAPFunc1 FLASH\_PageErase = (IAPFunc1)0x1000401;

\sphinxAtStartPar
变量定义如下:

\sphinxAtStartPar
PageNum: flash 擦除目标页码,以页为单位,0 为首地址,N 为页*N 对应地址

\sphinxAtStartPar
返回值:

\sphinxAtStartPar
0:擦除成功

\sphinxAtStartPar
1:擦除失败,参数错误

\sphinxAtStartPar
调用:

\sphinxAtStartPar
Result = FLASH\_PageErase(10);

\sphinxAtStartPar
擦除第20KB 内容。Result 返回0 表示成功。

\sphinxAtStartPar
写入函数:

\sphinxAtStartPar
定义函数类型:

\sphinxAtStartPar
typedef void (*IAPFunc2)(uint32\_t faddr, uint32\_t raddr, uint32\_t cnt);

\sphinxAtStartPar
IAPFunc2 FLASH\_PageWrite = (IAPFunc2)0x1000451;

\sphinxAtStartPar
变量定义如下:

\sphinxAtStartPar
faddr: flash 写入目标地址,字对齐

\sphinxAtStartPar
raddr: ram 写入目标地址,字对齐

\sphinxAtStartPar
cnt: 写入数量,字为单位,最大为512个字(2KB 长度)

\sphinxAtStartPar
返回值:

\sphinxAtStartPar
0:写入成功

\sphinxAtStartPar
1:写入失败,参数错误

\sphinxAtStartPar
调用:

\sphinxAtStartPar
Result = FLASH\_PageWrite(0x400,0x20000400,8);

\sphinxAtStartPar
将ram 地址0x20000400 开始的8*4 个字节写入flash 地址0x400 起始。Result 返回0 表示成功。

\sphinxAtStartPar
调用IAP 函数时,应保证栈空间剩余24 个字节(byte)以上。执行写操作前,需确认对应目标地址已经执行过擦除操作。

\sphinxAtStartPar
详细操作请参阅库函数。


\subsubsection{ISP模式}
\label{\detokenize{SWM241/_u529f_u80fd_u63cf_u8ff0/FLASH_u63a7_u5236_u5668_u4e0eISP_u64cd_u4f5c:isp}}
\sphinxAtStartPar
ISP(在系统编程)操作说明:当芯片上电后检测到ISP引脚持续5ms 以上的高电平后,将会进入ISP(在应用编程)模式。配合上位机及串口可执行程序更新操作,默认使用B6(RX)/B5(TX)作为串口通讯使用。

\sphinxAtStartPar
详细操作请参阅应用文档及库函数。


\subsubsection{BOOT自定义}
\label{\detokenize{SWM241/_u529f_u80fd_u63cf_u8ff0/FLASH_u63a7_u5236_u5668_u4e0eISP_u64cd_u4f5c:boot}}
\sphinxAtStartPar
FLASH地址空间支持将指定地址的2K数据映射至0x00空间,通过REMAP寄存器实现。将地址(2KB对齐)写入REMAP寄存器BASEADD,并将EN位置1,则指定地址内容将被映射至0x00空间,可通过此功能实现向量表的重映射。

\sphinxAtStartPar
例如

\sphinxAtStartPar
BOOT: 0x00 \textasciitilde{} 0x4000

\sphinxAtStartPar
USER: 0x4000 \textasciitilde{} 0x8000

\sphinxAtStartPar
在BOOT中配置REMAP寄存器地址为0x4000并使能,并跳转至USER执行,当读取0x00地址时,返回内容为0x4000地址内容。


\subsubsection{加密方式}
\label{\detokenize{SWM241/_u529f_u80fd_u63cf_u8ff0/FLASH_u63a7_u5236_u5668_u4e0eISP_u64cd_u4f5c:id2}}
\sphinxAtStartPar
加密支持三种级别


\begin{savenotes}\sphinxattablestart
\sphinxthistablewithglobalstyle
\centering
\begin{tabular}[t]{\X{33}{99}\X{33}{99}\X{33}{99}}
\sphinxtoprule
\sphinxtableatstartofbodyhook
\sphinxAtStartPar
级别 |
&
\sphinxAtStartPar
说明                             | |
&
\sphinxAtStartPar
关键字值           | |
\\
\sphinxhline
\sphinxAtStartPar
级别1 |
&
\sphinxAtStartPar
不加密,SWD可正常读写            |    0x00 |
&
\begin{DUlineblock}{0em}
\item[] {\color{red}\bfseries{}|}
\end{DUlineblock}
\\
\sphinxhline
\sphinxAtStartPar
级别2 |
&
\sphinxAtStartPar
SW 读取加密,SWD无法读取FLASH,只能执 | 行擦除操作,连接SWD后,FLASH无法执  | 行读操作,读取FLASH会进入Hardfault  |
&
\sphinxAtStartPar
0x43211234 | | |
\\
\sphinxhline
\sphinxAtStartPar
级别3 |
&
\sphinxAtStartPar
SWD封锁,SWD无法                 | 行读取及擦除工作,只能通过ISP读取 |
&
\sphinxAtStartPar
xABCD1234         | |
\\
\sphinxbottomrule
\end{tabular}
\sphinxtableafterendhook\par
\sphinxattableend\end{savenotes}

\sphinxAtStartPar
通过在用户程序中将0x1C偏移地址初始化为指定关键字,即可实现指定级别的加密。程序下载后再次上电后,芯片将处于指定加密级别的状态。


\subsection{寄存器映射}
\label{\detokenize{SWM241/_u529f_u80fd_u63cf_u8ff0/FLASH_u63a7_u5236_u5668_u4e0eISP_u64cd_u4f5c:id5}}

\begin{savenotes}\sphinxattablestart
\sphinxthistablewithglobalstyle
\centering
\begin{tabular}[t]{\X{20}{100}\X{20}{100}\X{20}{100}\X{20}{100}\X{20}{100}}
\sphinxtoprule
\sphinxtableatstartofbodyhook
\sphinxAtStartPar
名称   |
&
\begin{DUlineblock}{0em}
\item[] 偏移 |
\end{DUlineblock}
&
\begin{DUlineblock}{0em}
\item[] 
\item[] |
|
\end{DUlineblock}
&
\begin{DUlineblock}{0em}
\item[] 
\end{DUlineblock}
\begin{quote}

\begin{DUlineblock}{0em}
\item[] 
\item[] 
\end{DUlineblock}
\end{quote}
&
\sphinxAtStartPar
描述                       | | | |
\\
\sphinxhline
\sphinxAtStartPar
FLASHCTLBASE:0 {\color{red}\bfseries{}|}x4004A000
&
\begin{DUlineblock}{0em}
\item[] 
\end{DUlineblock}
&&&\\
\sphinxhline
\sphinxAtStartPar
DATA
&
\sphinxAtStartPar
0x00
&&
\sphinxAtStartPar
0x 00000
&
\sphinxAtStartPar
写数据寄存器               |
\\
\sphinxhline
\sphinxAtStartPar
ADDR
&
\sphinxAtStartPar
0x04
&&
\sphinxAtStartPar
0x 00000
&
\sphinxAtStartPar
写地址寄存器               |
\\
\sphinxhline
\sphinxAtStartPar
ERASE
&
\sphinxAtStartPar
0x08
&&
\sphinxAtStartPar
0x 00000
&
\sphinxAtStartPar
擦除寄存器                 |
\\
\sphinxhline
\sphinxAtStartPar
PROGEN
&
\sphinxAtStartPar
0x0C
&&
\sphinxAtStartPar
0x 00000
&
\sphinxAtStartPar
Program使能寄存器          |
\\
\sphinxhline
\sphinxAtStartPar
STAT
&
\sphinxAtStartPar
0x18
&&
\sphinxAtStartPar
0x 00028
&
\sphinxAtStartPar
状态寄存器                 |
\\
\sphinxhline
\sphinxAtStartPar
REMAP
&
\sphinxAtStartPar
0x28
&&
\sphinxAtStartPar
0x 00000
&
\sphinxAtStartPar
地址映射寄存器             |
\\
\sphinxbottomrule
\end{tabular}
\sphinxtableafterendhook\par
\sphinxattableend\end{savenotes}


\subsection{寄存器描述}
\label{\detokenize{SWM241/_u529f_u80fd_u63cf_u8ff0/FLASH_u63a7_u5236_u5668_u4e0eISP_u64cd_u4f5c:id8}}

\subsubsection{写数据寄存器DATA}
\label{\detokenize{SWM241/_u529f_u80fd_u63cf_u8ff0/FLASH_u63a7_u5236_u5668_u4e0eISP_u64cd_u4f5c:data}}

\begin{savenotes}\sphinxattablestart
\sphinxthistablewithglobalstyle
\centering
\begin{tabular}[t]{\X{20}{100}\X{20}{100}\X{20}{100}\X{20}{100}\X{20}{100}}
\sphinxtoprule
\sphinxtableatstartofbodyhook
\sphinxAtStartPar
寄存器 |
&
\begin{DUlineblock}{0em}
\item[] 偏移 |
\end{DUlineblock}
&
\begin{DUlineblock}{0em}
\item[] 
\item[] {\color{red}\bfseries{}|}
\end{DUlineblock}
&
\sphinxAtStartPar
复位值 |    描 | |
&
\begin{DUlineblock}{0em}
\item[] |
  |
\end{DUlineblock}
\\
\sphinxhline
\sphinxAtStartPar
DATA
&
\sphinxAtStartPar
0x00
&&
\sphinxAtStartPar
0 000000
&
\sphinxAtStartPar
写数据寄存器               |
\\
\sphinxbottomrule
\end{tabular}
\sphinxtableafterendhook\par
\sphinxattableend\end{savenotes}


\begin{savenotes}\sphinxattablestart
\sphinxthistablewithglobalstyle
\centering
\begin{tabular}[t]{\X{12}{96}\X{12}{96}\X{12}{96}\X{12}{96}\X{12}{96}\X{12}{96}\X{12}{96}\X{12}{96}}
\sphinxtoprule
\sphinxtableatstartofbodyhook
\sphinxAtStartPar
31
&
\sphinxAtStartPar
30
&
\sphinxAtStartPar
29
&
\sphinxAtStartPar
28
&
\sphinxAtStartPar
27
&
\sphinxAtStartPar
26
&
\sphinxAtStartPar
25
&
\sphinxAtStartPar
24
\\
\sphinxhline
\sphinxAtStartPar
DATA
&&&&&&&\\
\sphinxhline
\sphinxAtStartPar
23
&
\sphinxAtStartPar
22
&
\sphinxAtStartPar
21
&
\sphinxAtStartPar
20
&
\sphinxAtStartPar
19
&
\sphinxAtStartPar
18
&
\sphinxAtStartPar
17
&
\sphinxAtStartPar
16
\\
\sphinxhline
\sphinxAtStartPar
DATA
&&&&&&&\\
\sphinxhline
\sphinxAtStartPar
15
&
\sphinxAtStartPar
14
&
\sphinxAtStartPar
13
&
\sphinxAtStartPar
12
&
\sphinxAtStartPar
11
&
\sphinxAtStartPar
10
&
\sphinxAtStartPar
9
&
\sphinxAtStartPar
8
\\
\sphinxhline
\sphinxAtStartPar
DATA
&&&&&&&\\
\sphinxhline
\sphinxAtStartPar
7
&
\sphinxAtStartPar
6
&
\sphinxAtStartPar
5
&
\sphinxAtStartPar
4
&
\sphinxAtStartPar
3
&
\sphinxAtStartPar
2
&
\sphinxAtStartPar
1
&
\sphinxAtStartPar
0
\\
\sphinxhline
\sphinxAtStartPar
DATA
&&&&&&&\\
\sphinxbottomrule
\end{tabular}
\sphinxtableafterendhook\par
\sphinxattableend\end{savenotes}


\begin{savenotes}\sphinxattablestart
\sphinxthistablewithglobalstyle
\centering
\begin{tabular}[t]{\X{33}{99}\X{33}{99}\X{33}{99}}
\sphinxtoprule
\sphinxtableatstartofbodyhook
\sphinxAtStartPar
位域 |
&
\sphinxAtStartPar
名称     | |
&
\sphinxAtStartPar
描述                                        | |
\\
\sphinxhline
\sphinxAtStartPar
31:0
&
\sphinxAtStartPar
DATA
&
\sphinxAtStartPar
需要写的数据                                |

\sphinxAtStartPar
为保证FIFO不会溢出,只要write               | FIFO非满就可以写入新的数据。                |
\\
\sphinxbottomrule
\end{tabular}
\sphinxtableafterendhook\par
\sphinxattableend\end{savenotes}


\subsubsection{写地址寄存器ADDR}
\label{\detokenize{SWM241/_u529f_u80fd_u63cf_u8ff0/FLASH_u63a7_u5236_u5668_u4e0eISP_u64cd_u4f5c:addr}}

\begin{savenotes}\sphinxattablestart
\sphinxthistablewithglobalstyle
\centering
\begin{tabular}[t]{\X{20}{100}\X{20}{100}\X{20}{100}\X{20}{100}\X{20}{100}}
\sphinxtoprule
\sphinxtableatstartofbodyhook
\sphinxAtStartPar
寄存器 |
&
\begin{DUlineblock}{0em}
\item[] 偏移 |
\end{DUlineblock}
&
\begin{DUlineblock}{0em}
\item[] 
\item[] {\color{red}\bfseries{}|}
\end{DUlineblock}
&
\sphinxAtStartPar
复位值 |    描 | |
&
\begin{DUlineblock}{0em}
\item[] |
  |
\end{DUlineblock}
\\
\sphinxhline
\sphinxAtStartPar
ADDR
&
\sphinxAtStartPar
0x04
&&
\sphinxAtStartPar
0 000000
&
\sphinxAtStartPar
写地址寄存器               |
\\
\sphinxbottomrule
\end{tabular}
\sphinxtableafterendhook\par
\sphinxattableend\end{savenotes}


\begin{savenotes}\sphinxattablestart
\sphinxthistablewithglobalstyle
\centering
\begin{tabular}[t]{\X{12}{96}\X{12}{96}\X{12}{96}\X{12}{96}\X{12}{96}\X{12}{96}\X{12}{96}\X{12}{96}}
\sphinxtoprule
\sphinxtableatstartofbodyhook
\sphinxAtStartPar
31
&
\sphinxAtStartPar
30
&
\sphinxAtStartPar
29
&
\sphinxAtStartPar
28
&
\sphinxAtStartPar
27
&
\sphinxAtStartPar
26
&
\sphinxAtStartPar
25
&
\sphinxAtStartPar
24
\\
\sphinxhline
\sphinxAtStartPar
ADDR
&&&&&&&\\
\sphinxhline
\sphinxAtStartPar
23
&
\sphinxAtStartPar
22
&
\sphinxAtStartPar
21
&
\sphinxAtStartPar
20
&
\sphinxAtStartPar
19
&
\sphinxAtStartPar
18
&
\sphinxAtStartPar
17
&
\sphinxAtStartPar
16
\\
\sphinxhline
\sphinxAtStartPar
ADDR
&&&&&&&\\
\sphinxhline
\sphinxAtStartPar
15
&
\sphinxAtStartPar
14
&
\sphinxAtStartPar
13
&
\sphinxAtStartPar
12
&
\sphinxAtStartPar
11
&
\sphinxAtStartPar
10
&
\sphinxAtStartPar
9
&
\sphinxAtStartPar
8
\\
\sphinxhline
\sphinxAtStartPar
ADDR
&&&&&&&\\
\sphinxhline
\sphinxAtStartPar
7
&
\sphinxAtStartPar
6
&
\sphinxAtStartPar
5
&
\sphinxAtStartPar
4
&
\sphinxAtStartPar
3
&
\sphinxAtStartPar
2
&
\sphinxAtStartPar
1
&
\sphinxAtStartPar
0
\\
\sphinxhline
\sphinxAtStartPar
ADDR
&&&&&&&\\
\sphinxbottomrule
\end{tabular}
\sphinxtableafterendhook\par
\sphinxattableend\end{savenotes}


\begin{savenotes}\sphinxattablestart
\sphinxthistablewithglobalstyle
\centering
\begin{tabular}[t]{\X{33}{99}\X{33}{99}\X{33}{99}}
\sphinxtoprule
\sphinxtableatstartofbodyhook
\sphinxAtStartPar
位域 |
&
\sphinxAtStartPar
名称     | |
&
\sphinxAtStartPar
描述                                        | |
\\
\sphinxhline
\sphinxAtStartPar
31:0
&
\sphinxAtStartPar
ADDR
&
\sphinxAtStartPar
Flash写入起始地址                           |
\\
\sphinxbottomrule
\end{tabular}
\sphinxtableafterendhook\par
\sphinxattableend\end{savenotes}


\subsubsection{擦除寄存器ERASE}
\label{\detokenize{SWM241/_u529f_u80fd_u63cf_u8ff0/FLASH_u63a7_u5236_u5668_u4e0eISP_u64cd_u4f5c:erase}}

\begin{savenotes}\sphinxattablestart
\sphinxthistablewithglobalstyle
\centering
\begin{tabular}[t]{\X{20}{100}\X{20}{100}\X{20}{100}\X{20}{100}\X{20}{100}}
\sphinxtoprule
\sphinxtableatstartofbodyhook
\sphinxAtStartPar
寄存器 |
&
\begin{DUlineblock}{0em}
\item[] 偏移 |
\end{DUlineblock}
&
\begin{DUlineblock}{0em}
\item[] 
\item[] {\color{red}\bfseries{}|}
\end{DUlineblock}
&
\sphinxAtStartPar
复位值 |    描 | |
&
\begin{DUlineblock}{0em}
\item[] |
  |
\end{DUlineblock}
\\
\sphinxhline
\sphinxAtStartPar
ERASE
&
\sphinxAtStartPar
0x08
&&
\sphinxAtStartPar
0 000000
&
\sphinxAtStartPar
擦除寄存器                 |
\\
\sphinxbottomrule
\end{tabular}
\sphinxtableafterendhook\par
\sphinxattableend\end{savenotes}


\begin{savenotes}\sphinxattablestart
\sphinxthistablewithglobalstyle
\centering
\begin{tabular}[t]{\X{12}{96}\X{12}{96}\X{12}{96}\X{12}{96}\X{12}{96}\X{12}{96}\X{12}{96}\X{12}{96}}
\sphinxtoprule
\sphinxtableatstartofbodyhook
\sphinxAtStartPar
31
&
\sphinxAtStartPar
30
&
\sphinxAtStartPar
29
&
\sphinxAtStartPar
28
&
\sphinxAtStartPar
27
&
\sphinxAtStartPar
26
&
\sphinxAtStartPar
25
&
\sphinxAtStartPar
24
\\
\sphinxhline
\sphinxAtStartPar
REQ
&&&&&&&\\
\sphinxhline
\sphinxAtStartPar
23
&
\sphinxAtStartPar
22
&
\sphinxAtStartPar
21
&
\sphinxAtStartPar
20
&
\sphinxAtStartPar
19
&
\sphinxAtStartPar
18
&
\sphinxAtStartPar
17
&
\sphinxAtStartPar
16
\\
\sphinxhline\begin{itemize}
\item {} 
\end{itemize}
&&&&&&&
\sphinxAtStartPar
ADDR
\\
\sphinxhline
\sphinxAtStartPar
15
&
\sphinxAtStartPar
14
&
\sphinxAtStartPar
13
&
\sphinxAtStartPar
12
&
\sphinxAtStartPar
11
&
\sphinxAtStartPar
10
&
\sphinxAtStartPar
9
&
\sphinxAtStartPar
8
\\
\sphinxhline
\sphinxAtStartPar
ADDR
&&&&&&&\\
\sphinxhline
\sphinxAtStartPar
7
&
\sphinxAtStartPar
6
&
\sphinxAtStartPar
5
&
\sphinxAtStartPar
4
&
\sphinxAtStartPar
3
&
\sphinxAtStartPar
2
&
\sphinxAtStartPar
1
&
\sphinxAtStartPar
0
\\
\sphinxhline
\sphinxAtStartPar
ADDR
&&&&&&&\\
\sphinxbottomrule
\end{tabular}
\sphinxtableafterendhook\par
\sphinxattableend\end{savenotes}


\begin{savenotes}\sphinxattablestart
\sphinxthistablewithglobalstyle
\centering
\begin{tabular}[t]{\X{33}{99}\X{33}{99}\X{33}{99}}
\sphinxtoprule
\sphinxtableatstartofbodyhook
\sphinxAtStartPar
位域 |
&
\sphinxAtStartPar
名称     | |
&
\sphinxAtStartPar
描述                                        | |
\\
\sphinxhline
\sphinxAtStartPar
31:24
&
\sphinxAtStartPar
REQ
&
\sphinxAtStartPar
擦使能,8位全为1有效                         |
\\
\sphinxhline
\sphinxAtStartPar
23:17
&\begin{itemize}
\item {} 
\end{itemize}
&\begin{itemize}
\item {} 
\end{itemize}
\\
\sphinxhline
\sphinxAtStartPar
16:0
&
\sphinxAtStartPar
ADDR
&
\sphinxAtStartPar
擦除地址,全1表示整个eFlash 擦除            |

\sphinxAtStartPar
每次page擦可擦除1K空间                      |
\\
\sphinxbottomrule
\end{tabular}
\sphinxtableafterendhook\par
\sphinxattableend\end{savenotes}


\subsubsection{PROGEN寄存器 PROGEN}
\label{\detokenize{SWM241/_u529f_u80fd_u63cf_u8ff0/FLASH_u63a7_u5236_u5668_u4e0eISP_u64cd_u4f5c:progen-progen}}

\begin{savenotes}\sphinxattablestart
\sphinxthistablewithglobalstyle
\centering
\begin{tabular}[t]{\X{20}{100}\X{20}{100}\X{20}{100}\X{20}{100}\X{20}{100}}
\sphinxtoprule
\sphinxtableatstartofbodyhook
\sphinxAtStartPar
寄存器 |
&
\begin{DUlineblock}{0em}
\item[] 偏移 |
\end{DUlineblock}
&
\begin{DUlineblock}{0em}
\item[] 
\item[] {\color{red}\bfseries{}|}
\end{DUlineblock}
&
\sphinxAtStartPar
复位值 |    描 | |
&
\begin{DUlineblock}{0em}
\item[] |
  |
\end{DUlineblock}
\\
\sphinxhline
\sphinxAtStartPar
PROGEN
&
\sphinxAtStartPar
0x0C
&&
\sphinxAtStartPar
0 000000
&
\sphinxAtStartPar
Program使能寄存器          |
\\
\sphinxbottomrule
\end{tabular}
\sphinxtableafterendhook\par
\sphinxattableend\end{savenotes}


\begin{savenotes}\sphinxattablestart
\sphinxthistablewithglobalstyle
\centering
\begin{tabular}[t]{\X{12}{96}\X{12}{96}\X{12}{96}\X{12}{96}\X{12}{96}\X{12}{96}\X{12}{96}\X{12}{96}}
\sphinxtoprule
\sphinxtableatstartofbodyhook
\sphinxAtStartPar
31
&
\sphinxAtStartPar
30
&
\sphinxAtStartPar
29
&
\sphinxAtStartPar
28
&
\sphinxAtStartPar
27
&
\sphinxAtStartPar
26
&
\sphinxAtStartPar
25
&
\sphinxAtStartPar
24
\\
\sphinxhline\begin{itemize}
\item {} 
\end{itemize}
&&&&&&&\\
\sphinxhline
\sphinxAtStartPar
23
&
\sphinxAtStartPar
22
&
\sphinxAtStartPar
21
&
\sphinxAtStartPar
20
&
\sphinxAtStartPar
19
&
\sphinxAtStartPar
18
&
\sphinxAtStartPar
17
&
\sphinxAtStartPar
16
\\
\sphinxhline\begin{itemize}
\item {} 
\end{itemize}
&&&&&&&\\
\sphinxhline
\sphinxAtStartPar
15
&
\sphinxAtStartPar
14
&
\sphinxAtStartPar
13
&
\sphinxAtStartPar
12
&
\sphinxAtStartPar
11
&
\sphinxAtStartPar
10
&
\sphinxAtStartPar
9
&
\sphinxAtStartPar
8
\\
\sphinxhline\begin{itemize}
\item {} 
\end{itemize}
&&&&&&&\\
\sphinxhline
\sphinxAtStartPar
7
&
\sphinxAtStartPar
6
&
\sphinxAtStartPar
5
&
\sphinxAtStartPar
4
&
\sphinxAtStartPar
3
&
\sphinxAtStartPar
2
&
\sphinxAtStartPar
1
&
\sphinxAtStartPar
0
\\
\sphinxhline
\sphinxAtStartPar
PROGEN
&&&&&&&\\
\sphinxbottomrule
\end{tabular}
\sphinxtableafterendhook\par
\sphinxattableend\end{savenotes}


\begin{savenotes}\sphinxattablestart
\sphinxthistablewithglobalstyle
\centering
\begin{tabular}[t]{\X{33}{99}\X{33}{99}\X{33}{99}}
\sphinxtoprule
\sphinxtableatstartofbodyhook
\sphinxAtStartPar
位域 |
&
\sphinxAtStartPar
名称     | |
&
\sphinxAtStartPar
描述                                        | |
\\
\sphinxhline
\sphinxAtStartPar
31:1
&\begin{itemize}
\item {} 
\end{itemize}
&\begin{itemize}
\item {} 
\end{itemize}
\\
\sphinxhline
\sphinxAtStartPar
0
&
\sphinxAtStartPar
PROGEN
&
\sphinxAtStartPar
1:eflash program en                        |
\\
\sphinxbottomrule
\end{tabular}
\sphinxtableafterendhook\par
\sphinxattableend\end{savenotes}


\subsubsection{状态寄存器STAT}
\label{\detokenize{SWM241/_u529f_u80fd_u63cf_u8ff0/FLASH_u63a7_u5236_u5668_u4e0eISP_u64cd_u4f5c:stat}}

\begin{savenotes}\sphinxattablestart
\sphinxthistablewithglobalstyle
\centering
\begin{tabular}[t]{\X{20}{100}\X{20}{100}\X{20}{100}\X{20}{100}\X{20}{100}}
\sphinxtoprule
\sphinxtableatstartofbodyhook
\sphinxAtStartPar
寄存器 |
&
\begin{DUlineblock}{0em}
\item[] 偏移 |
\end{DUlineblock}
&
\begin{DUlineblock}{0em}
\item[] 
\item[] {\color{red}\bfseries{}|}
\end{DUlineblock}
&
\sphinxAtStartPar
复位值 |    描 | |
&
\begin{DUlineblock}{0em}
\item[] |
  |
\end{DUlineblock}
\\
\sphinxhline
\sphinxAtStartPar
STAT
&
\sphinxAtStartPar
0x18
&&
\sphinxAtStartPar
0 000028
&
\sphinxAtStartPar
状态寄存器                 |
\\
\sphinxbottomrule
\end{tabular}
\sphinxtableafterendhook\par
\sphinxattableend\end{savenotes}


\begin{savenotes}\sphinxattablestart
\sphinxthistablewithglobalstyle
\centering
\begin{tabular}[t]{\X{12}{96}\X{12}{96}\X{12}{96}\X{12}{96}\X{12}{96}\X{12}{96}\X{12}{96}\X{12}{96}}
\sphinxtoprule
\sphinxtableatstartofbodyhook
\sphinxAtStartPar
31
&
\sphinxAtStartPar
30
&
\sphinxAtStartPar
29
&
\sphinxAtStartPar
28
&
\sphinxAtStartPar
27
&
\sphinxAtStartPar
26
&
\sphinxAtStartPar
25
&
\sphinxAtStartPar
24
\\
\sphinxhline
\sphinxAtStartPar
FLASHIDLE
&\begin{itemize}
\item {} 
\end{itemize}
&&&&&&\\
\sphinxhline
\sphinxAtStartPar
23
&
\sphinxAtStartPar
22
&
\sphinxAtStartPar
21
&
\sphinxAtStartPar
20
&
\sphinxAtStartPar
19
&
\sphinxAtStartPar
18
&
\sphinxAtStartPar
17
&
\sphinxAtStartPar
16
\\
\sphinxhline\begin{itemize}
\item {} 
\end{itemize}
&&&&&&&\\
\sphinxhline
\sphinxAtStartPar
15
&
\sphinxAtStartPar
14
&
\sphinxAtStartPar
13
&
\sphinxAtStartPar
12
&
\sphinxAtStartPar
11
&
\sphinxAtStartPar
10
&
\sphinxAtStartPar
9
&
\sphinxAtStartPar
8
\\
\sphinxhline\begin{itemize}
\item {} 
\end{itemize}
&&&&&&&\\
\sphinxhline
\sphinxAtStartPar
7
&
\sphinxAtStartPar
6
&
\sphinxAtStartPar
5
&
\sphinxAtStartPar
4
&
\sphinxAtStartPar
3
&
\sphinxAtStartPar
2
&
\sphinxAtStartPar
1
&
\sphinxAtStartPar
0
\\
\sphinxhline\begin{itemize}
\item {} 
\end{itemize}
&&&
\sphinxAtStartPar
FI FOFULL
&
\sphinxAtStartPar
FIF OEMPTY
&
\sphinxAtStartPar
RE ADBUSY
&
\sphinxAtStartPar
PR OGBUSY
&
\sphinxAtStartPar
ERA SEBUSY
\\
\sphinxbottomrule
\end{tabular}
\sphinxtableafterendhook\par
\sphinxattableend\end{savenotes}


\begin{savenotes}\sphinxattablestart
\sphinxthistablewithglobalstyle
\centering
\begin{tabular}[t]{\X{33}{99}\X{33}{99}\X{33}{99}}
\sphinxtoprule
\sphinxtableatstartofbodyhook
\sphinxAtStartPar
位域 |
&
\sphinxAtStartPar
名称     | |
&
\sphinxAtStartPar
描述                                        | |
\\
\sphinxhline
\sphinxAtStartPar
31
&
\sphinxAtStartPar
FLASHIDLE
&
\sphinxAtStartPar
1:flash空闲                                |

\sphinxAtStartPar
0:flash 忙                                 |
\\
\sphinxhline
\sphinxAtStartPar
30:5
&\begin{itemize}
\item {} 
\end{itemize}
&\begin{itemize}
\item {} 
\end{itemize}
\\
\sphinxhline
\sphinxAtStartPar
4
&
\sphinxAtStartPar
FIFOFULL
&
\sphinxAtStartPar
write FIFO满                                |
\\
\sphinxhline
\sphinxAtStartPar
3
&
\sphinxAtStartPar
FIFOEMPTY
&
\sphinxAtStartPar
write fifo 空                               |
\\
\sphinxhline
\sphinxAtStartPar
2
&
\sphinxAtStartPar
READBUSY
&
\sphinxAtStartPar
read操作进行中,RO                          |
\\
\sphinxhline
\sphinxAtStartPar
1
&
\sphinxAtStartPar
PROGBUSY
&
\sphinxAtStartPar
program操作进行中,RO                       |
\\
\sphinxhline
\sphinxAtStartPar
0
&
\sphinxAtStartPar
ERASEBUSY
&
\sphinxAtStartPar
erase操作进行中,RO                         |
\\
\sphinxbottomrule
\end{tabular}
\sphinxtableafterendhook\par
\sphinxattableend\end{savenotes}


\subsubsection{地址映射寄存器REMAP}
\label{\detokenize{SWM241/_u529f_u80fd_u63cf_u8ff0/FLASH_u63a7_u5236_u5668_u4e0eISP_u64cd_u4f5c:remap}}

\begin{savenotes}\sphinxattablestart
\sphinxthistablewithglobalstyle
\centering
\begin{tabular}[t]{\X{20}{100}\X{20}{100}\X{20}{100}\X{20}{100}\X{20}{100}}
\sphinxtoprule
\sphinxtableatstartofbodyhook
\sphinxAtStartPar
寄存器 |
&
\begin{DUlineblock}{0em}
\item[] 偏移 |
\end{DUlineblock}
&
\begin{DUlineblock}{0em}
\item[] 
\item[] {\color{red}\bfseries{}|}
\end{DUlineblock}
&
\sphinxAtStartPar
复位值 |    描 | |
&
\begin{DUlineblock}{0em}
\item[] |
  |
\end{DUlineblock}
\\
\sphinxhline
\sphinxAtStartPar
REMAP
&
\sphinxAtStartPar
0x28
&&
\sphinxAtStartPar
0 00\_000
&
\sphinxAtStartPar
地址映射寄存器             |
\\
\sphinxbottomrule
\end{tabular}
\sphinxtableafterendhook\par
\sphinxattableend\end{savenotes}


\begin{savenotes}\sphinxattablestart
\sphinxthistablewithglobalstyle
\centering
\begin{tabular}[t]{\X{12}{96}\X{12}{96}\X{12}{96}\X{12}{96}\X{12}{96}\X{12}{96}\X{12}{96}\X{12}{96}}
\sphinxtoprule
\sphinxtableatstartofbodyhook
\sphinxAtStartPar
31
&
\sphinxAtStartPar
30
&
\sphinxAtStartPar
29
&
\sphinxAtStartPar
28
&
\sphinxAtStartPar
27
&
\sphinxAtStartPar
26
&
\sphinxAtStartPar
25
&
\sphinxAtStartPar
24
\\
\sphinxhline\begin{itemize}
\item {} 
\end{itemize}
&&&&&&&\\
\sphinxhline
\sphinxAtStartPar
23
&
\sphinxAtStartPar
22
&
\sphinxAtStartPar
21
&
\sphinxAtStartPar
20
&
\sphinxAtStartPar
19
&
\sphinxAtStartPar
18
&
\sphinxAtStartPar
17
&
\sphinxAtStartPar
16
\\
\sphinxhline\begin{itemize}
\item {} 
\end{itemize}
&&&&&&&\\
\sphinxhline
\sphinxAtStartPar
15
&
\sphinxAtStartPar
14
&
\sphinxAtStartPar
13
&
\sphinxAtStartPar
12
&
\sphinxAtStartPar
11
&
\sphinxAtStartPar
10
&
\sphinxAtStartPar
9
&
\sphinxAtStartPar
8
\\
\sphinxhline\begin{itemize}
\item {} 
\end{itemize}
&&&&&&&\\
\sphinxhline
\sphinxAtStartPar
7
&
\sphinxAtStartPar
6
&
\sphinxAtStartPar
5
&
\sphinxAtStartPar
4
&
\sphinxAtStartPar
3
&
\sphinxAtStartPar
2
&
\sphinxAtStartPar
1
&
\sphinxAtStartPar
0
\\
\sphinxhline\begin{itemize}
\item {} 
\end{itemize}
&
\sphinxAtStartPar
OFFSET
&&&&&&
\sphinxAtStartPar
ON
\\
\sphinxbottomrule
\end{tabular}
\sphinxtableafterendhook\par
\sphinxattableend\end{savenotes}


\begin{savenotes}\sphinxattablestart
\sphinxthistablewithglobalstyle
\centering
\begin{tabular}[t]{\X{33}{99}\X{33}{99}\X{33}{99}}
\sphinxtoprule
\sphinxtableatstartofbodyhook
\sphinxAtStartPar
位域 |
&
\sphinxAtStartPar
名称     | |
&
\sphinxAtStartPar
描述                                        | |
\\
\sphinxhline
\sphinxAtStartPar
31:7
&\begin{itemize}
\item {} 
\end{itemize}
&\begin{itemize}
\item {} 
\end{itemize}
\\
\sphinxhline
\sphinxAtStartPar
6:1
&
\sphinxAtStartPar
OFFSET
&
\sphinxAtStartPar
BASEADD地址                                 |

\sphinxAtStartPar
将0地址开头的                               | 地址的访问都映射到BASEADD基地址对应的2K地址 |
\\
\sphinxhline
\sphinxAtStartPar
0
&
\sphinxAtStartPar
ON
&
\sphinxAtStartPar
REMAP

\sphinxAtStartPar
1:打开                                     |

\sphinxAtStartPar
0:关闭                                     |
\\
\sphinxbottomrule
\end{tabular}
\sphinxtableafterendhook\par
\sphinxattableend\end{savenotes}

\sphinxstepscope


\chapter{典型应用电路}
\label{\detokenize{SWM241/_u5178_u578b_u5e94_u7528_u7535_u8def:id1}}\label{\detokenize{SWM241/_u5178_u578b_u5e94_u7528_u7535_u8def::doc}}
\sphinxAtStartPar
\sphinxincludegraphics{{SWM241/media典型应用电路002}.emf}

\sphinxAtStartPar
图 7‑1典型应用电路图

\sphinxstepscope


\chapter{电气特性}
\label{\detokenize{SWM241/_u7535_u6c14_u7279_u6027:id1}}\label{\detokenize{SWM241/_u7535_u6c14_u7279_u6027::doc}}
\sphinxAtStartPar
本章提供了SWM241系列电气参数,包括额定值,DC参数及AC参数。


\section{绝对最大额定值}
\label{\detokenize{SWM241/_u7535_u6c14_u7279_u6027:id2}}
\sphinxAtStartPar
表格 8‑1绝对最大额定值


\begin{savenotes}\sphinxattablestart
\sphinxthistablewithglobalstyle
\centering
\begin{tabular}[t]{\X{20}{100}\X{20}{100}\X{20}{100}\X{20}{100}\X{20}{100}}
\sphinxtoprule
\sphinxtableatstartofbodyhook
\sphinxAtStartPar
符号  |
&
\sphinxAtStartPar
参数                 |
&
\sphinxAtStartPar
最大值 |    最小
&
\begin{DUlineblock}{0em}
\item[] {\color{red}\bfseries{}|}
\end{DUlineblock}
&
\sphinxAtStartPar
单位 |
\\
\sphinxhline
\sphinxAtStartPar
Vdd\sphinxhyphen{}Vss
&
\sphinxAtStartPar
直流电源电压         |    6
&
\begin{DUlineblock}{0em}
\item[] \sphinxhyphen{}
\end{DUlineblock}
&
\sphinxAtStartPar
.3   |    V
&
\begin{DUlineblock}{0em}
\item[] 
\end{DUlineblock}
\\
\sphinxhline
\sphinxAtStartPar
T:sub:\sphinxtitleref{w}
&
\sphinxAtStartPar
工作温度             |
&
\sphinxAtStartPar
105    |
&
\sphinxAtStartPar
\sphinxhyphen{}40    |
&
\sphinxAtStartPar
℃   |
\\
\sphinxhline
\sphinxAtStartPar
T:sub:\sphinxtitleref{S}
&
\sphinxAtStartPar
贮存温度             |
&
\sphinxAtStartPar
150    |
&
\sphinxAtStartPar
\sphinxhyphen{}50    |
&
\sphinxAtStartPar
℃   |
\\
\sphinxhline
\sphinxAtStartPar
I$_{\text{OL}}$
&
\sphinxAtStartPar
单一管脚最大灌电流   |    20
&
\begin{DUlineblock}{0em}
\item[] —
\end{DUlineblock}
&
\begin{DUlineblock}{0em}
\item[] mA
\end{DUlineblock}
&\\
\sphinxhline
\sphinxAtStartPar
I$_{\text{OH}}$
&
\sphinxAtStartPar
单一管脚最大输出电流 |    20
&
\begin{DUlineblock}{0em}
\item[] —
\end{DUlineblock}
&
\begin{DUlineblock}{0em}
\item[] mA  |
\end{DUlineblock}
&\\
\sphinxhline
\sphinxAtStartPar
I$_{\text{LED}}$
&
\sphinxAtStartPar
大驱动IO灌电流能力   |    150
&
\begin{DUlineblock}{0em}
\item[] 
\end{DUlineblock}
&
\begin{DUlineblock}{0em}
\item[] mA
\end{DUlineblock}
&
\begin{DUlineblock}{0em}
\item[] 
\end{DUlineblock}
\\
\sphinxhline
\sphinxAtStartPar
ΣI$_{\text{OL}}$
&
\sphinxAtStartPar
所有管脚输入电流和   |    250
&
\begin{DUlineblock}{0em}
\item[] —
\end{DUlineblock}
&
\begin{DUlineblock}{0em}
\item[] mA
\end{DUlineblock}
&\\
\sphinxhline
\sphinxAtStartPar
ΣI$_{\text{OH}}$
&
\sphinxAtStartPar
所有管脚输出电流和   |    120
&
\begin{DUlineblock}{0em}
\item[] —
\end{DUlineblock}
&
\begin{DUlineblock}{0em}
\item[] mA
\end{DUlineblock}
&\\
\sphinxhline
\sphinxAtStartPar
V$_{\text{ESD}}$
&
\sphinxAtStartPar
静电保护 (HBM)       |
&
\sphinxAtStartPar
8000   |
&
\sphinxAtStartPar
—      |
&
\sphinxAtStartPar
V   |
\\
\sphinxbottomrule
\end{tabular}
\sphinxtableafterendhook\par
\sphinxattableend\end{savenotes}

\sphinxAtStartPar
注意: 绝对最大额定值是可能给器件带来物理性损伤或者影响稳定性的额定值,必须在不超过额定值的情况下使用此器件。


\section{直流电气特性}
\label{\detokenize{SWM241/_u7535_u6c14_u7279_u6027:id5}}
\sphinxAtStartPar
表格 8‑2 直流电气特性(Vdd = 2.5V \textasciitilde{} 5.0V, Vss = 0V, T$_{\text{w}}$ =25℃))


\begin{savenotes}\sphinxattablestart
\sphinxthistablewithglobalstyle
\centering
\begin{tabular}[t]{\X{14}{98}\X{14}{98}\X{14}{98}\X{14}{98}\X{14}{98}\X{14}{98}\X{14}{98}}
\sphinxtoprule
\sphinxtableatstartofbodyhook
\sphinxAtStartPar
参数       |
&
\begin{DUlineblock}{0em}
\item[] 最 | 大  | 值  |
\end{DUlineblock}
&
\begin{DUlineblock}{0em}
\item[] 
\item[] 
\item[] 
\item[] 
\end{DUlineblock}
&
\begin{DUlineblock}{0em}
\item[] 符
\end{DUlineblock}
&
\begin{DUlineblock}{0em}
\item[] 
\end{DUlineblock}
\begin{quote}

\sphinxAtStartPar
单 位
\end{quote}
&
\begin{DUlineblock}{0em}
\item[] 
\end{DUlineblock}
&
\sphinxAtStartPar
测试条件         | | | |
\\
\sphinxhline
\sphinxAtStartPar
工作电压   |
&
\begin{DUlineblock}{0em}
\item[] 5.5
\end{DUlineblock}
&
\begin{DUlineblock}{0em}
\item[] 
\end{DUlineblock}

\sphinxAtStartPar
5.0
&
\begin{DUlineblock}{0em}
\item[] 
\end{DUlineblock}

\sphinxAtStartPar
2.5
&
\begin{DUlineblock}{0em}
\item[] 
\end{DUlineblock}

\sphinxAtStartPar
Vdd
&
\begin{DUlineblock}{0em}
\item[] V
\end{DUlineblock}
&
\sphinxAtStartPar
—                |
\\
\sphinxhline
\sphinxAtStartPar
模拟工作电压 | Vdd
&
\sphinxAtStartPar
—
&&&&&
\sphinxAtStartPar
— |
\\
\sphinxhline
\sphinxAtStartPar
模拟参考电压 |   —
&
\sphinxAtStartPar
A
&&&&&
\sphinxAtStartPar
— |
\\
\sphinxhline
\sphinxAtStartPar
普通工     {\color{red}\bfseries{}|}作模式下电流  |   —
&
\begin{DUlineblock}{0em}
\item[] 6.5
\end{DUlineblock}
&
\begin{DUlineblock}{0em}
\item[] —
\end{DUlineblock}
&
\begin{DUlineblock}{0em}
\item[] mA
\end{DUlineblock}
&
\begin{DUlineblock}{0em}
\item[] {\color{red}\bfseries{}|}
\end{DUlineblock}
&
\begin{DUlineblock}{0em}
\item[] 4
\end{DUlineblock}

\sphinxAtStartPar
Id
d1
&
\sphinxAtStartPar
F$_{\text{sys}}$ =  | Hz, Vdd = 5V, | 所有             | 引脚无负载,输入不  | 使能,外设时钟关闭  |
\\
\sphinxhline&
\sphinxAtStartPar
—
&&&&&
\sphinxAtStartPar
F$_{\text{sys}}$ = 6MHz, Vdd = 5V, 所有             | 负载,输入不  | 外设时钟关闭  |
\\
\sphinxhline
\sphinxAtStartPar
普通工     {\color{red}\bfseries{}|}作模式下电流  |   —(32KHz)  |
&
\begin{DUlineblock}{0em}
\item[] 700  |
\end{DUlineblock}
&
\begin{DUlineblock}{0em}
\item[] —
\end{DUlineblock}
\begin{quote}

\begin{DUlineblock}{0em}
\item[] 
\end{DUlineblock}
\end{quote}
&
\begin{DUlineblock}{0em}
\item[] uA
\end{DUlineblock}
\begin{quote}

\begin{DUlineblock}{0em}
\item[] 
\end{DUlineblock}
\end{quote}
&
\begin{DUlineblock}{0em}
\item[] {\color{red}\bfseries{}|}
\end{DUlineblock}
\begin{quote}

\begin{DUlineblock}{0em}
\item[] 
\end{DUlineblock}
\end{quote}
&
\begin{DUlineblock}{0em}
\item[] 3
\end{DUlineblock}
\begin{description}
\sphinxlineitem{Id}
\begin{DUlineblock}{0em}
\item[] 
\end{DUlineblock}

\end{description}
&
\sphinxAtStartPar
F$_{\text{sys}}$ =  | Hz, Vdd = 5V, | 所有             | 无负载,输入不  | 使能,外设时钟关闭, |
\begin{quote}

\sphinxAtStartPar
内部高频振荡器关闭 |
\end{quote}
\\
\sphinxhline
\sphinxAtStartPar
SLEEP模式  |
&
\begin{DUlineblock}{0em}
\item[] —
\end{DUlineblock}
&
\begin{DUlineblock}{0em}
\item[] 
\end{DUlineblock}

\sphinxAtStartPar
5
&
\begin{DUlineblock}{0em}
\item[] 
\end{DUlineblock}

\sphinxAtStartPar
—
&
\begin{DUlineblock}{0em}
\item[] 
\end{DUlineblock}

\sphinxAtStartPar
A
&
\begin{DUlineblock}{0em}
\item[] 
\end{DUlineblock}

\sphinxAtStartPar
d 4
&
\sphinxAtStartPar
Vdd = 5.0V       |
\\
\sphinxhline
\sphinxAtStartPar
输入低电压 |
&
\begin{DUlineblock}{0em}
\item[] 0.3 Vdd
\end{DUlineblock}
&
\begin{DUlineblock}{0em}
\item[] —
\end{DUlineblock}
&
\begin{DUlineblock}{0em}
\item[] —
\end{DUlineblock}
&
\begin{DUlineblock}{0em}
\item[] 
\end{DUlineblock}
&&
\sphinxAtStartPar
nput Enable     |
\\
\sphinxhline
\sphinxAtStartPar
输入高电压 |
&
\begin{DUlineblock}{0em}
\item[] —
\end{DUlineblock}
&
\begin{DUlineblock}{0em}
\item[] —
\end{DUlineblock}
&
\begin{DUlineblock}{0em}
\item[] 0.7
Vdd
\end{DUlineblock}
&
\begin{DUlineblock}{0em}
\item[] 
\end{DUlineblock}
&&
\sphinxAtStartPar
nput Enable     |
\\
\sphinxhline
\sphinxAtStartPar
输出低电压 |
&
\begin{DUlineblock}{0em}
\item[] 0.4
\end{DUlineblock}
&
\begin{DUlineblock}{0em}
\item[] —
\end{DUlineblock}
&
\begin{DUlineblock}{0em}
\item[] —
\end{DUlineblock}
&
\begin{DUlineblock}{0em}
\item[] 
\end{DUlineblock}
&&
\sphinxAtStartPar
.5V≤Vdd\textless{}3.3V    |
\\
\sphinxhline&
\sphinxAtStartPar
0.6
&&&
\sphinxAtStartPar
V
&&
\sphinxAtStartPar
3.3V≤Vdd≤5V
\\
\sphinxhline
\sphinxAtStartPar
输出高电压 |
&
\begin{DUlineblock}{0em}
\item[] —
\end{DUlineblock}
&
\begin{DUlineblock}{0em}
\item[] —
\end{DUlineblock}
&
\begin{DUlineblock}{0em}
\item[]   V
dd\sphinxhyphen{}
0.4
\end{DUlineblock}
&
\begin{DUlineblock}{0em}
\item[] 
\end{DUlineblock}
&&
\sphinxAtStartPar
.5V≤Vdd\textless{}3.3V    |
\\
\sphinxhline&
\sphinxAtStartPar
—
&&&
\sphinxAtStartPar
V
&&
\sphinxAtStartPar
3.3V≤Vdd≤5V
\\
\sphinxhline
\sphinxAtStartPar
内置上拉电阻 |
&
\sphinxAtStartPar
TBD
&&&&&\\
\sphinxhline
\sphinxAtStartPar
内置下拉电阻 |
&
\sphinxAtStartPar
TBD
&&&&&\\
\sphinxbottomrule
\end{tabular}
\sphinxtableafterendhook\par
\sphinxattableend\end{savenotes}


\section{交流电气特性}
\label{\detokenize{SWM241/_u7535_u6c14_u7279_u6027:id14}}

\subsection{内部高频RC振荡器}
\label{\detokenize{SWM241/_u7535_u6c14_u7279_u6027:rc}}
\sphinxAtStartPar
表格 8‑3内部高频RC振荡器特征值


\begin{savenotes}\sphinxattablestart
\sphinxthistablewithglobalstyle
\centering
\begin{tabular}[t]{\X{16}{96}\X{16}{96}\X{16}{96}\X{16}{96}\X{16}{96}\X{16}{96}}
\sphinxtoprule
\sphinxtableatstartofbodyhook
\sphinxAtStartPar
参数         |
&
\begin{DUlineblock}{0em}
\item[] 最 | 大值 |
\end{DUlineblock}
&
\begin{DUlineblock}{0em}
\item[] 
\item[] 
\end{DUlineblock}

\sphinxAtStartPar
小值
&
\begin{DUlineblock}{0em}
\item[] 单
\end{DUlineblock}
&
\begin{DUlineblock}{0em}
\item[] 
\end{DUlineblock}
&
\sphinxAtStartPar
条件                | | |
\\
\sphinxhline
\sphinxAtStartPar
电压         |
&
\begin{DUlineblock}{0em}
\item[] 5.5
\end{DUlineblock}
&
\begin{DUlineblock}{0em}
\item[] 
\end{DUlineblock}

\sphinxAtStartPar
0
&
\begin{DUlineblock}{0em}
\item[] 
\end{DUlineblock}

\sphinxAtStartPar
5
&
\sphinxAtStartPar
V |
&
\sphinxAtStartPar
—                   |
\\
\sphinxhline
\sphinxAtStartPar
中心频率     |
&
\sphinxAtStartPar
— |
&
\begin{DUlineblock}{0em}
\item[] 48
\end{DUlineblock}
&
\sphinxAtStartPar
— |
&
\begin{DUlineblock}{0em}
\item[] 
\end{DUlineblock}

\sphinxAtStartPar
MHz
&
\sphinxAtStartPar
—                   |
\\
\sphinxhline
\sphinxAtStartPar
内部震荡矫正 |    1
&
\begin{DUlineblock}{0em}
\item[] —
\end{DUlineblock}
&
\begin{DUlineblock}{0em}
\item[] 
\end{DUlineblock}
&
\begin{DUlineblock}{0em}
\item[] \%
\sphinxhyphen{}1
\end{DUlineblock}
&
\begin{DUlineblock}{0em}
\item[] T
\end{DUlineblock}
&
\sphinxAtStartPar
$_{\text{w}}$ = 25℃   | Vdd = 5.0V
\\
\sphinxhline&
\sphinxAtStartPar
3
&
\sphinxAtStartPar
—
&&
\sphinxAtStartPar
\%
&
\sphinxAtStartPar
T$_{\text{w}}$ = \sphinxhyphen{}40℃\textasciitilde{}105℃, Vdd = 2.5V\textasciitilde{}5.5V
\\
\sphinxbottomrule
\end{tabular}
\sphinxtableafterendhook\par
\sphinxattableend\end{savenotes}


\subsection{内部低频RC振荡器}
\label{\detokenize{SWM241/_u7535_u6c14_u7279_u6027:id15}}
\sphinxAtStartPar
表格 8‑4内部低频RC振荡器特征值


\begin{savenotes}\sphinxattablestart
\sphinxthistablewithglobalstyle
\centering
\begin{tabular}[t]{\X{16}{96}\X{16}{96}\X{16}{96}\X{16}{96}\X{16}{96}\X{16}{96}}
\sphinxtoprule
\sphinxtableatstartofbodyhook
\sphinxAtStartPar
参数         |
&
\begin{DUlineblock}{0em}
\item[] 最 | 大值 |
\end{DUlineblock}
&
\begin{DUlineblock}{0em}
\item[] 
\item[] 
\end{DUlineblock}

\sphinxAtStartPar
小值
&
\begin{DUlineblock}{0em}
\item[] 单
\end{DUlineblock}
&
\begin{DUlineblock}{0em}
\item[] 
\end{DUlineblock}
&
\sphinxAtStartPar
条件                | | |
\\
\sphinxhline
\sphinxAtStartPar
电压         |
&
\begin{DUlineblock}{0em}
\item[] 5.5
\end{DUlineblock}
&
\begin{DUlineblock}{0em}
\item[] 
\end{DUlineblock}

\sphinxAtStartPar
0
&
\begin{DUlineblock}{0em}
\item[] 
\end{DUlineblock}

\sphinxAtStartPar
5
&
\sphinxAtStartPar
V |
&
\sphinxAtStartPar
—                   |
\\
\sphinxhline
\sphinxAtStartPar
中心频率     |
&
\sphinxAtStartPar
— |
&
\begin{DUlineblock}{0em}
\item[] 32
\end{DUlineblock}
&
\sphinxAtStartPar
— |
&
\begin{DUlineblock}{0em}
\item[] 
\end{DUlineblock}

\sphinxAtStartPar
kHz
&
\sphinxAtStartPar
—                   |
\\
\sphinxhline
\sphinxAtStartPar
频率误差     |
&
\begin{DUlineblock}{0em}
\item[] 20
\end{DUlineblock}
&
\sphinxAtStartPar
— |
&
\begin{DUlineblock}{0em}
\item[] 
\end{DUlineblock}

\sphinxAtStartPar
\sphinxhyphen{}20
&
\sphinxAtStartPar
\% |
&
\sphinxAtStartPar
T$_{\text{w}}$ =       | \sphinxhyphen{}40℃\textasciitilde{}105℃, Vdd = 2.5V\textasciitilde{}5.5V
\\
\sphinxbottomrule
\end{tabular}
\sphinxtableafterendhook\par
\sphinxattableend\end{savenotes}


\subsection{外部2\sphinxhyphen{}32MHz晶体振荡器}
\label{\detokenize{SWM241/_u7535_u6c14_u7279_u6027:mhz}}
\sphinxAtStartPar
表格 8‑5外部2\sphinxhyphen{}32MHz晶体振荡器


\begin{savenotes}\sphinxattablestart
\sphinxthistablewithglobalstyle
\centering
\begin{tabular}[t]{\X{16}{96}\X{16}{96}\X{16}{96}\X{16}{96}\X{16}{96}\X{16}{96}}
\sphinxtoprule
\sphinxtableatstartofbodyhook
\sphinxAtStartPar
参数         |
&
\begin{DUlineblock}{0em}
\item[] 最 | 大值 |
\end{DUlineblock}
&
\begin{DUlineblock}{0em}
\item[] 
\item[] 
\end{DUlineblock}

\sphinxAtStartPar
小值
&
\begin{DUlineblock}{0em}
\item[] 单
\end{DUlineblock}
&
\begin{DUlineblock}{0em}
\item[] 
\end{DUlineblock}
&
\sphinxAtStartPar
测试条件           | | |
\\
\sphinxhline
\sphinxAtStartPar
工作电压     |
&
\begin{DUlineblock}{0em}
\item[] 5.5
\end{DUlineblock}
&\begin{itemize}
\item {} 
\begin{DUlineblock}{0em}
\item[] 
\end{DUlineblock}

\end{itemize}
&
\begin{DUlineblock}{0em}
\item[] 
\end{DUlineblock}

\sphinxAtStartPar
2.5
&
\sphinxAtStartPar
V |
&\begin{itemize}
\item {} 
\begin{DUlineblock}{0em}
\item[] 
\end{DUlineblock}

\end{itemize}
\\
\sphinxhline
\sphinxAtStartPar
温度         |
&
\begin{DUlineblock}{0em}
\item[] 105
\end{DUlineblock}
&\begin{itemize}
\item {} 
\begin{DUlineblock}{0em}
\item[] 
\end{DUlineblock}

\end{itemize}
&
\begin{DUlineblock}{0em}
\item[] 
\end{DUlineblock}

\sphinxAtStartPar
0
&
\sphinxAtStartPar
℃ |
&\begin{itemize}
\item {} 
\begin{DUlineblock}{0em}
\item[] 
\end{DUlineblock}

\end{itemize}
\\
\sphinxhline
\sphinxAtStartPar
工作电流     |
&\begin{itemize}
\item {} 
\begin{DUlineblock}{0em}
\item[] 
\end{DUlineblock}

\end{itemize}
&
\begin{DUlineblock}{0em}
\item[] 
\end{DUlineblock}

\sphinxAtStartPar
TBD
&\begin{itemize}
\item {} 
\begin{DUlineblock}{0em}
\item[] 
\end{DUlineblock}

\end{itemize}
&
\begin{DUlineblock}{0em}
\item[] mA
\end{DUlineblock}
&
\sphinxAtStartPar
12 MHz, VDD = 5.0V |
\\
\sphinxhline
\sphinxAtStartPar
时钟频率     |
&
\begin{DUlineblock}{0em}
\item[] 32
\end{DUlineblock}
&\begin{itemize}
\item {} 
\begin{DUlineblock}{0em}
\item[] 
\end{DUlineblock}

\end{itemize}
&
\sphinxAtStartPar
2 |
&
\begin{DUlineblock}{0em}
\item[] 
\end{DUlineblock}

\sphinxAtStartPar
MHz
&\begin{itemize}
\item {} 
\begin{DUlineblock}{0em}
\item[] 
\end{DUlineblock}

\end{itemize}
\\
\sphinxbottomrule
\end{tabular}
\sphinxtableafterendhook\par
\sphinxattableend\end{savenotes}

\sphinxAtStartPar
表格 8‑6外部振荡器典型电路


\begin{savenotes}\sphinxattablestart
\sphinxthistablewithglobalstyle
\centering
\begin{tabular}[t]{\X{33}{99}\X{33}{99}\X{33}{99}}
\sphinxtoprule
\sphinxtableatstartofbodyhook
\sphinxAtStartPar
晶振频率           |
&
\sphinxAtStartPar
C1                |
&
\sphinxAtStartPar
C2                |
\\
\sphinxhline
\sphinxAtStartPar
2MHz \textasciitilde{} 32 MHz
&
\sphinxAtStartPar
10\textasciitilde{}20 pF
&
\sphinxAtStartPar
10\textasciitilde{}20 pF
\\
\sphinxbottomrule
\end{tabular}
\sphinxtableafterendhook\par
\sphinxattableend\end{savenotes}

\sphinxAtStartPar
\sphinxincludegraphics{{SWM241/media电气特性002}.emf}


\section{模拟器件特性}
\label{\detokenize{SWM241/_u7535_u6c14_u7279_u6027:id16}}

\subsection{SARADC特性}
\label{\detokenize{SWM241/_u7535_u6c14_u7279_u6027:saradc}}
\sphinxAtStartPar
表格 8‑7 SAR ADC特征值


\begin{savenotes}\sphinxattablestart
\sphinxthistablewithglobalstyle
\centering
\begin{tabular}[t]{\X{16}{96}\X{16}{96}\X{16}{96}\X{16}{96}\X{16}{96}\X{16}{96}}
\sphinxtoprule
\sphinxtableatstartofbodyhook
\sphinxAtStartPar
参数         |
&
\begin{DUlineblock}{0em}
\item[] 最大值 | 典
\end{DUlineblock}
&
\begin{DUlineblock}{0em}
\item[] 
\end{DUlineblock}

\sphinxAtStartPar
最小
&
\begin{DUlineblock}{0em}
\item[] 
\end{DUlineblock}
&
\sphinxAtStartPar
符号  |
&
\sphinxAtStartPar
单位 | |
\\
\sphinxhline
\sphinxAtStartPar
分辨率       |
&
\sphinxAtStartPar
12    |
&
\sphinxAtStartPar
—   |
&
\sphinxAtStartPar
—     |
&
\sphinxAtStartPar
—     |
&
\sphinxAtStartPar
bit  |
\\
\sphinxhline
\sphinxAtStartPar
工           {\color{red}\bfseries{}|}作电流(平均)  {\color{red}\bfseries{}|}非线性差分误差 |
&
\sphinxAtStartPar
—     | |   |
&
\sphinxAtStartPar
7   |
&
\sphinxAtStartPar
—     | |

\begin{DUlineblock}{0em}
\item[] 
\end{DUlineblock}
&
\sphinxAtStartPar
Idda  | |

\begin{DUlineblock}{0em}
\item[] 
\end{DUlineblock}
&
\sphinxAtStartPar
mA   |
\\
\sphinxhline&
\sphinxAtStartPar
TBD
&
\sphinxAtStartPar
—
&
\sphinxAtStartPar
—
&
\sphinxAtStartPar
DNL
&
\sphinxAtStartPar
LSB
\\
\sphinxhline
\sphinxAtStartPar
非线性积分误差 |
&
\sphinxAtStartPar
TBD |
&
\sphinxAtStartPar
—
&
\sphinxAtStartPar
—
&
\sphinxAtStartPar
INL
&
\sphinxAtStartPar
LSB
\\
\sphinxhline
\sphinxAtStartPar
补偿错误     |
&
\sphinxAtStartPar
—     |
&
\sphinxAtStartPar
TBD |
&
\sphinxAtStartPar
—     |
&
\sphinxAtStartPar
EO    |
&
\sphinxAtStartPar
mV   |
\\
\sphinxhline
\sphinxAtStartPar
采样速率     |
&
\sphinxAtStartPar
—     |
&
\sphinxAtStartPar
1   |
&
\sphinxAtStartPar
0.05  |
&
\sphinxAtStartPar
FS    |
&
\sphinxAtStartPar
MHz  |
\\
\sphinxhline
\sphinxAtStartPar
工作时钟频率 |    —
&
\begin{DUlineblock}{0em}
\item[] 1
\end{DUlineblock}
&
\begin{DUlineblock}{0em}
\item[] 0
\end{DUlineblock}
&
\sphinxAtStartPar
05  |    F
&
\sphinxAtStartPar
LK  |    M
&
\sphinxAtStartPar
z  |
\\
\sphinxhline
\sphinxAtStartPar
采样延时     |
&
\sphinxAtStartPar
—     |
&
\sphinxAtStartPar
1   |
&
\sphinxAtStartPar
—     |
&
\sphinxAtStartPar
TADC  |
&
\begin{DUlineblock}{0em}
\item[] 
\end{DUlineblock}

\sphinxAtStartPar
les
\\
\sphinxhline
\sphinxAtStartPar
参考电压     |
&
\sphinxAtStartPar
AVDD  |
&
\begin{DUlineblock}{0em}
\item[] 
\end{DUlineblock}

\sphinxAtStartPar
VDD
&
\sphinxAtStartPar
3.0   |
&
\begin{DUlineblock}{0em}
\item[] 
\end{DUlineblock}

\sphinxAtStartPar
FIN
&
\sphinxAtStartPar
V    |
\\
\sphinxhline
\sphinxAtStartPar
电           {\color{red}\bfseries{}|}容值(每通道)  |
&
\sphinxAtStartPar
TBD   | |
&
\sphinxAtStartPar
—   |
&
\sphinxAtStartPar
—     | |
&
\sphinxAtStartPar
—     | |
&
\sphinxAtStartPar
pF   |
\\
\sphinxhline
\sphinxAtStartPar
工作电压     |
&
\sphinxAtStartPar
5.5   |
&
\sphinxAtStartPar
5.0 |
&
\sphinxAtStartPar
2.5   |
&
\sphinxAtStartPar
AVdd  |
&
\sphinxAtStartPar
V    |
\\
\sphinxbottomrule
\end{tabular}
\sphinxtableafterendhook\par
\sphinxattableend\end{savenotes}


\subsection{LDO特性}
\label{\detokenize{SWM241/_u7535_u6c14_u7279_u6027:ldo}}
\sphinxAtStartPar
表格 8‑8 LDO特征值


\begin{savenotes}\sphinxattablestart
\sphinxthistablewithglobalstyle
\centering
\begin{tabular}[t]{\X{16}{96}\X{16}{96}\X{16}{96}\X{16}{96}\X{16}{96}\X{16}{96}}
\sphinxtoprule
\sphinxtableatstartofbodyhook
\sphinxAtStartPar
参数         |
&
\begin{DUlineblock}{0em}
\item[] 最大值 | 典
\end{DUlineblock}
&
\begin{DUlineblock}{0em}
\item[] 
\end{DUlineblock}

\sphinxAtStartPar
最小
&
\begin{DUlineblock}{0em}
\item[] 
\end{DUlineblock}
&
\sphinxAtStartPar
符号  |
&
\sphinxAtStartPar
单位 | |
\\
\sphinxhline
\sphinxAtStartPar
工作电压     |
&
\sphinxAtStartPar
5.5   |
&\begin{itemize}
\item {} 
\begin{DUlineblock}{0em}
\item[] 
\end{DUlineblock}

\end{itemize}
&
\sphinxAtStartPar
2.5   |
&
\sphinxAtStartPar
VDD   |
&
\sphinxAtStartPar
V    |
\\
\sphinxhline
\sphinxAtStartPar
输出电压     |
&
\sphinxAtStartPar
TBD   |
&
\begin{DUlineblock}{0em}
\item[] 
\end{DUlineblock}

\sphinxAtStartPar
.55
&
\sphinxAtStartPar
TBD   |
&
\sphinxAtStartPar
VLDO  |
&
\sphinxAtStartPar
V    |
\\
\sphinxhline
\sphinxAtStartPar
工作温度范围 |    1
&
\sphinxAtStartPar
5   |    2
&
\begin{DUlineblock}{0em}
\item[] \sphinxhyphen{}
\end{DUlineblock}
&
\sphinxAtStartPar
0   |    T
&
\begin{DUlineblock}{0em}
\item[] ℃
\end{DUlineblock}
&
\begin{DUlineblock}{0em}
\item[] 
\end{DUlineblock}
\\
\sphinxbottomrule
\end{tabular}
\sphinxtableafterendhook\par
\sphinxattableend\end{savenotes}


\subsection{欠压检测 (Brown\sphinxhyphen{}out Detector)}
\label{\detokenize{SWM241/_u7535_u6c14_u7279_u6027:brown-out-detector}}
\sphinxAtStartPar
表格 8‑9 欠压检测特性


\begin{savenotes}\sphinxattablestart
\sphinxthistablewithglobalstyle
\centering
\begin{tabular}[t]{\X{14}{98}\X{14}{98}\X{14}{98}\X{14}{98}\X{14}{98}\X{14}{98}\X{14}{98}}
\sphinxtoprule
\sphinxtableatstartofbodyhook
\sphinxAtStartPar
参数 |
&
\sphinxAtStartPar
符号 |  最
&&&&&\\
\sphinxhline
\sphinxAtStartPar
工  {\color{red}\bfseries{}|}作电压 |
&
\begin{DUlineblock}{0em}
\item[] VDD |
\end{DUlineblock}
&
\sphinxAtStartPar
5.5  | |
&\begin{itemize}
\item {} 
\begin{DUlineblock}{0em}
\item[] {\color{red}\bfseries{}|}
\end{DUlineblock}

\end{itemize}
&
\sphinxAtStartPar
1.55  | |
&
\sphinxAtStartPar
V  | |
&\begin{itemize}
\item {} 
\begin{DUlineblock}{0em}
\item[] {\color{red}\bfseries{}|}
\end{DUlineblock}

\end{itemize}
\\
\sphinxhline
\sphinxAtStartPar
温度 |
&
\sphinxAtStartPar
TA |
&
\sphinxAtStartPar
105 |
&
\sphinxAtStartPar
25 |
&
\sphinxAtStartPar
\sphinxhyphen{}40 |
&
\sphinxAtStartPar
℃ |
&\begin{itemize}
\item {} 
\begin{DUlineblock}{0em}
\item[] 
\end{DUlineblock}

\end{itemize}
\\
\sphinxhline
\sphinxAtStartPar
静  {\color{red}\bfseries{}|}态电流 |
&
\begin{DUlineblock}{0em}
\item[] BOD |
\end{DUlineblock}
&
\sphinxAtStartPar
TBD  | |
&\begin{itemize}
\item {} 
\begin{DUlineblock}{0em}
\item[] {\color{red}\bfseries{}|}
\end{DUlineblock}

\end{itemize}
&\begin{itemize}
\item {} 
\begin{DUlineblock}{0em}
\item[] {\color{red}\bfseries{}|}
\end{DUlineblock}

\end{itemize}
&
\sphinxAtStartPar
μA | |
&
\sphinxAtStartPar
VDD =  | 5 V  |
\\
\sphinxhline
\sphinxAtStartPar
欠压复 {\color{red}\bfseries{}|}位阈值 | D
&
\sphinxAtStartPar
VBO | RST |
&
\sphinxAtStartPar
TBD | |
&
\sphinxAtStartPar
1.7 | |
&
\sphinxAtStartPar
TBD | |
&
\sphinxAtStartPar
V | |
&
\sphinxAtStartPar
RSTLVL 00  | |
\\
\sphinxhline&&
\sphinxAtStartPar
TBD
&
\sphinxAtStartPar
1.9
&
\sphinxAtStartPar
TBD
&
\sphinxAtStartPar
V
&
\sphinxAtStartPar
RSTLVL = 001
\\
\sphinxhline&&
\sphinxAtStartPar
TBD
&
\sphinxAtStartPar
2.1
&
\sphinxAtStartPar
TBD
&
\sphinxAtStartPar
V
&
\sphinxAtStartPar
RSTLVL = 010
\\
\sphinxhline&&
\sphinxAtStartPar
TBD
&
\sphinxAtStartPar
2.7
&
\sphinxAtStartPar
TBD
&
\sphinxAtStartPar
V
&
\sphinxAtStartPar
RSTLVL = 011
\\
\sphinxhline&&
\sphinxAtStartPar
TBD
&
\sphinxAtStartPar
3.5
&
\sphinxAtStartPar
TBD
&
\sphinxAtStartPar
V
&
\sphinxAtStartPar
RSTLVL = 100
\\
\sphinxhline
\sphinxAtStartPar
欠压中 {\color{red}\bfseries{}|}断阈值 | D
&
\sphinxAtStartPar
VBO | INT |
&
\sphinxAtStartPar
TBD | |
&
\sphinxAtStartPar
1.9 | |
&
\sphinxAtStartPar
TBD | |
&
\sphinxAtStartPar
V | |
&
\sphinxAtStartPar
INTLVL 00  | |
\\
\sphinxhline&&
\sphinxAtStartPar
TBD
&
\sphinxAtStartPar
2.1
&
\sphinxAtStartPar
TBD
&
\sphinxAtStartPar
V
&
\sphinxAtStartPar
INTLVL = 001
\\
\sphinxhline&&
\sphinxAtStartPar
TBD
&
\sphinxAtStartPar
2.3
&
\sphinxAtStartPar
TBD
&
\sphinxAtStartPar
V
&
\sphinxAtStartPar
INTLVL = 010
\\
\sphinxhline&&
\sphinxAtStartPar
TBD
&
\sphinxAtStartPar
2.5
&
\sphinxAtStartPar
TBD
&
\sphinxAtStartPar
V
&
\sphinxAtStartPar
INTLVL = 011
\\
\sphinxhline&&
\sphinxAtStartPar
TBD
&
\sphinxAtStartPar
2.7
&
\sphinxAtStartPar
TBD
&
\sphinxAtStartPar
V
&
\sphinxAtStartPar
INTLVL = 100
\\
\sphinxhline&&
\sphinxAtStartPar
TBD
&
\sphinxAtStartPar
3.5
&
\sphinxAtStartPar
TBD
&
\sphinxAtStartPar
V
&
\sphinxAtStartPar
INTLVL = 101
\\
\sphinxhline&&
\sphinxAtStartPar
TBD
&
\sphinxAtStartPar
4.1
&
\sphinxAtStartPar
TBD
&
\sphinxAtStartPar
V
&
\sphinxAtStartPar
INTLVL = 110
\\
\sphinxbottomrule
\end{tabular}
\sphinxtableafterendhook\par
\sphinxattableend\end{savenotes}


\subsection{上电复位(Power\sphinxhyphen{}on Reset)}
\label{\detokenize{SWM241/_u7535_u6c14_u7279_u6027:power-on-reset}}
\sphinxAtStartPar
表格 8‑10 上电复位特性


\begin{savenotes}\sphinxattablestart
\sphinxthistablewithglobalstyle
\centering
\begin{tabular}[t]{\X{16}{96}\X{16}{96}\X{16}{96}\X{16}{96}\X{16}{96}\X{16}{96}}
\sphinxtoprule
\sphinxtableatstartofbodyhook
\sphinxAtStartPar
参数                 |
&
\sphinxAtStartPar
最 | 大值  |
&
\sphinxAtStartPar
典 | | 小值
&
\sphinxAtStartPar
最 | |   符号
&
\begin{DUlineblock}{0em}
\item[] 单位 |
\end{DUlineblock}
&
\begin{DUlineblock}{0em}
\item[] 
\end{DUlineblock}
\\
\sphinxhline
\sphinxAtStartPar
温度                 |
&
\begin{DUlineblock}{0em}
\item[] 105
\end{DUlineblock}
&
\sphinxAtStartPar
25 |
&
\begin{DUlineblock}{0em}
\item[] 
\end{DUlineblock}

\sphinxAtStartPar
0
&
\sphinxAtStartPar
TA  |
&
\sphinxAtStartPar
℃   |
\\
\sphinxhline
\sphinxAtStartPar
复位电压             |
&
\begin{DUlineblock}{0em}
\item[] TBD
\end{DUlineblock}
&
\begin{DUlineblock}{0em}
\item[] 
\end{DUlineblock}

\sphinxAtStartPar
1.6
&
\begin{DUlineblock}{0em}
\item[] 
\end{DUlineblock}

\sphinxAtStartPar
TBD
&
\begin{DUlineblock}{0em}
\item[] 
\end{DUlineblock}

\sphinxAtStartPar
POR
&
\sphinxAtStartPar
V   |
\\
\sphinxhline
\sphinxAtStartPar
VDD起始电压来确保上电复位 |       |
&
\sphinxAtStartPar
200 |
&\begin{itemize}
\item {} 
\end{itemize}

\begin{DUlineblock}{0em}
\item[] 
\end{DUlineblock}
&\begin{itemize}
\item {} 
\begin{DUlineblock}{0em}
\item[] 
\end{DUlineblock}

\end{itemize}
&&
\sphinxAtStartPar
mV
\\
\sphinxhline
\sphinxAtStartPar
VDD上升率来确保上电复位 |       |
&\begin{itemize}
\item {} 
\begin{DUlineblock}{0em}
\item[] 
\end{DUlineblock}

\end{itemize}
&\begin{itemize}
\item {} 
\begin{DUlineblock}{0em}
\item[] 
\end{DUlineblock}

\end{itemize}
&&&\\
\sphinxbottomrule
\end{tabular}
\sphinxtableafterendhook\par
\sphinxattableend\end{savenotes}


\subsection{上电VDD上升率要求}
\label{\detokenize{SWM241/_u7535_u6c14_u7279_u6027:vdd}}

\begin{savenotes}\sphinxattablestart
\sphinxthistablewithglobalstyle
\centering
\begin{tabular}[t]{\X{16}{96}\X{16}{96}\X{16}{96}\X{16}{96}\X{16}{96}\X{16}{96}}
\sphinxtoprule
\sphinxtableatstartofbodyhook
\sphinxAtStartPar
参数         |
&
\begin{DUlineblock}{0em}
\item[] 最大值 | 典
\end{DUlineblock}
&
\begin{DUlineblock}{0em}
\item[] 
\end{DUlineblock}

\sphinxAtStartPar
最小
&
\begin{DUlineblock}{0em}
\item[] 
\end{DUlineblock}
&
\sphinxAtStartPar
单位  |
&
\sphinxAtStartPar
符号 | |
\\
\sphinxhline
\sphinxAtStartPar
电           {\color{red}\bfseries{}|}源供电上升时间  |
&
\sphinxAtStartPar
2.0   | |
&\begin{itemize}
\item {} 
\begin{DUlineblock}{0em}
\item[] 
\end{DUlineblock}

\end{itemize}
&\begin{itemize}
\item {} 
\begin{DUlineblock}{0em}
\item[] {\color{red}\bfseries{}|}
\end{DUlineblock}

\end{itemize}
&
\sphinxAtStartPar
ms    | |
&
\sphinxAtStartPar
Tr   |
\\
\sphinxbottomrule
\end{tabular}
\sphinxtableafterendhook\par
\sphinxattableend\end{savenotes}

\sphinxAtStartPar
注:当电源电压上升较慢时,需要通过reset引脚保证上电稳定性;或通过BOD复位保证上电稳定性,BOD复位为电平复位,内部默认一直开启,且对上电速度要求较低。

\sphinxAtStartPar
上电

\sphinxAtStartPar
\sphinxincludegraphics{{SWM241/media电气特性003}.emf}

\sphinxAtStartPar
图 8‑1 上电复位时间示意图


\subsection{FLASH DC电气特性}
\label{\detokenize{SWM241/_u7535_u6c14_u7279_u6027:flash-dc}}

\begin{savenotes}\sphinxattablestart
\sphinxthistablewithglobalstyle
\centering
\begin{tabular}[t]{\X{16}{96}\X{16}{96}\X{16}{96}\X{16}{96}\X{16}{96}\X{16}{96}}
\sphinxtoprule
\sphinxtableatstartofbodyhook
\sphinxAtStartPar
参数         |
&
\begin{DUlineblock}{0em}
\item[] 最大值 | 典
\end{DUlineblock}
&
\begin{DUlineblock}{0em}
\item[] 
\end{DUlineblock}

\sphinxAtStartPar
最小
&
\begin{DUlineblock}{0em}
\item[] 
\end{DUlineblock}
&
\sphinxAtStartPar
符号  |
&
\sphinxAtStartPar
单位 | |
\\
\sphinxhline
\sphinxAtStartPar
擦写次数     |
&
\sphinxAtStartPar
—     |
&
\sphinxAtStartPar
—   |
&
\sphinxAtStartPar
20K   |
&
\begin{DUlineblock}{0em}
\item[] 
\end{DUlineblock}

\sphinxAtStartPar
sub
UR\textasciigrave{}
&
\begin{DUlineblock}{0em}
\item[] 
\end{DUlineblock}

\sphinxAtStartPar
les
\\
\sphinxhline
\sphinxAtStartPar
数据保留     |
&
\sphinxAtStartPar
—     |
&
\sphinxAtStartPar
—   |
&
\sphinxAtStartPar
100   |
&
\sphinxAtStartPar
T:s | ub:\sphinxtitleref{RET}
&
\begin{DUlineblock}{0em}
\item[] 
\end{DUlineblock}

\sphinxAtStartPar
ars
\\
\sphinxhline
\sphinxAtStartPar
全片擦除时间 |    —
&
\begin{DUlineblock}{0em}
\item[] 6
\end{DUlineblock}
&
\begin{DUlineblock}{0em}
\item[] —
\end{DUlineblock}
&
\begin{DUlineblock}{0em}
\item[] 
\end{DUlineblock}
&
\begin{DUlineblock}{0em}
\item[] m
\end{DUlineblock}
\begin{quote}\begin{description}
\sphinxlineitem{sub}
\end{description}\end{quote}

\sphinxAtStartPar
RASE\textasciigrave{}
&
\begin{DUlineblock}{0em}
\item[] 
\end{DUlineblock}
\\
\sphinxhline
\sphinxAtStartPar
页擦除时间   |
&
\begin{DUlineblock}{0em}
\item[] 1.5
\end{DUlineblock}
&
\begin{DUlineblock}{0em}
\item[] 
\end{DUlineblock}
&
\begin{DUlineblock}{0em}
\item[] 
\end{DUlineblock}
&
\begin{DUlineblock}{0em}
\item[] 
\end{DUlineblock}

\sphinxAtStartPar
:sub
ASE\textasciigrave{}
&
\sphinxAtStartPar
s   |
\\
\sphinxhline
\sphinxAtStartPar
字编程时间   |
&
\begin{DUlineblock}{0em}
\item[] 
\end{DUlineblock}
&
\sphinxAtStartPar
0  |
&
\begin{DUlineblock}{0em}
\item[] 
\end{DUlineblock}
&
\begin{DUlineblock}{0em}
\item[] :su
\end{DUlineblock}

\sphinxAtStartPar
ROG\textasciigrave{}
&
\sphinxAtStartPar
s   |
\\
\sphinxbottomrule
\end{tabular}
\sphinxtableafterendhook\par
\sphinxattableend\end{savenotes}

\sphinxstepscope


\chapter{封装尺寸}
\label{\detokenize{SWM241/_u5c01_u88c5_u5c3a_u5bf8:id1}}\label{\detokenize{SWM241/_u5c01_u88c5_u5c3a_u5bf8::doc}}

\section{LQFP44}
\label{\detokenize{SWM241/_u5c01_u88c5_u5c3a_u5bf8:lqfp44}}\phantomsection\label{\detokenize{SWM241/_u5c01_u88c5_u5c3a_u5bf8:id2}}
\sphinxAtStartPar
\sphinxincludegraphics[width=2.49in,height=2.37in]{{SWM241/media封装尺寸002}.png}:

\sphinxAtStartPar
\sphinxincludegraphics[width=1.27in,height=1.59in]{{SWM241/media封装尺寸003}.png}

\phantomsection\label{\detokenize{SWM241/_u5c01_u88c5_u5c3a_u5bf8:id3}}
\sphinxAtStartPar
\sphinxincludegraphics[width=5.02in,height=0.7in]{{SWM241/media封装尺寸004}.png}:

\phantomsection\label{\detokenize{SWM241/_u5c01_u88c5_u5c3a_u5bf8:symbol-mil-limeter-min-nom-max-a-1-60-a1-0-05-0-15-a2-1-35-1-40-1-45-a3-0-59-0-64-0-69-b-0-28-0-36-c-0-13-0-17-d-11-80-12-00-12-20-d1-9-90-10-00-10-10-e-11-80-12-00-12-20-e1-9-90-10-00-10-10-e-0-80b-sc-1-eb-11-05-11-25-l-0-45-0-75-l1-1-00ref-0-7-su-p-o}}
\sphinxAtStartPar
+++++|symbol|mil|||||limeter|||+========+=========+=========+=======+||min|nom|max|+++++|a|—{\color{red}\bfseries{}|—|1.60|+++++|a1|0.05|}—{\color{red}\bfseries{}|0.15|+++++|a2|1.35|1.40|1.45|+++++|a3|0.59|0.64|0.69|+++++|b|0.28|}—{\color{red}\bfseries{}|0.36|+++++|c|0.13|}—{\color{red}\bfseries{}|0.17|+++++|d|11.80|12.00|12.20|+++++|d1|9.90|10.00|10.10|+++++|e|11.80|12.00|12.20|+++++|e1|9.90|10.00|10.10|+++++|e|0.80b|||||sc{[}1{]}\_|||+++++|eb|11.05|}—{\color{red}\bfseries{}|11.25|+++++|l|0.45|}—{\color{red}\bfseries{}|0.75|+++++|l1|1.00ref|||+++++|θ|0|}—{\color{red}\bfseries{}|7|||||}:su|||||p:{\color{red}\bfseries{}\textasciigrave{}}o\textasciigrave{}|+++++:

\phantomsection\label{\detokenize{SWM241/_u5c01_u88c5_u5c3a_u5bf8:id6}}
\sphinxAtStartPar
图9‑1lqfp44封装尺寸图|:

\begin{DUlineblock}{0em}
\item[] 
\end{DUlineblock}


\section{LQFP32}
\label{\detokenize{SWM241/_u5c01_u88c5_u5c3a_u5bf8:lqfp32}}\phantomsection\label{\detokenize{SWM241/_u5c01_u88c5_u5c3a_u5bf8:id7}}
\sphinxAtStartPar
\sphinxincludegraphics[width=2.63in,height=2.59in]{{SWM241/media封装尺寸005}.png}:

\sphinxAtStartPar
\sphinxincludegraphics[width=1.45in,height=1.48in]{{SWM241/media封装尺寸006}.png}

\phantomsection\label{\detokenize{SWM241/_u5c01_u88c5_u5c3a_u5bf8:id8}}
\sphinxAtStartPar
\sphinxincludegraphics[width=4.62in,height=0.81in]{{SWM241/media封装尺寸007}.png}:

\phantomsection\label{\detokenize{SWM241/_u5c01_u88c5_u5c3a_u5bf8:symbol-mill-imeter-min-nom-max-a-1-60-a1-0-05-0-15-a2-1-35-1-40-1-45-a3-0-59-0-64-0-69-b-0-33-0-41-c-0-13-0-17-d-8-80-9-00-9-20-d1-6-90-7-00-7-10-e-8-80-9-00-9-20-e1-6-90-7-00-7-10-e-0-80bs-c-2-eb-8-10-8-25-l-0-45-0-75-l1-1-00ref-0-7-su-p-o}}
\sphinxAtStartPar
+++++|symbol|mill|||||imeter|||+========+========+========+=======+||min|nom|max|+++++|a|—{\color{red}\bfseries{}|—|1.60|+++++|a1|0.05|}—{\color{red}\bfseries{}|0.15|+++++|a2|1.35|1.40|1.45|+++++|a3|0.59|0.64|0.69|+++++|b|0.33|}—{\color{red}\bfseries{}|0.41|+++++|c|0.13|}—{\color{red}\bfseries{}|0.17|+++++|d|8.80|9.00|9.20|+++++|d1|6.90|7.00|7.10|+++++|e|8.80|9.00|9.20|+++++|e1|6.90|7.00|7.10|+++++|e|0.80bs|||||c{[}2{]}\_|||+++++|eb|8.10|}—{\color{red}\bfseries{}|8.25|+++++|l|0.45|}—{\color{red}\bfseries{}|0.75|+++++|l1|1|||||}.00ref|||+++++|θ|0|—{\color{red}\bfseries{}|7|||||}:su|||||p:{\color{red}\bfseries{}\textasciigrave{}}o\textasciigrave{}|+++++:

\phantomsection\label{\detokenize{SWM241/_u5c01_u88c5_u5c3a_u5bf8:id11}}
\sphinxAtStartPar
图9‑2lqfp32封装尺寸图|:

\begin{DUlineblock}{0em}
\item[] 
\end{DUlineblock}

\sphinxstepscope


\chapter{版本记录}
\label{\detokenize{SWM241/_u7248_u672c_u8bb0_u5f55:id1}}\label{\detokenize{SWM241/_u7248_u672c_u8bb0_u5f55::doc}}

\begin{savenotes}\sphinxattablestart
\sphinxthistablewithglobalstyle
\centering
\begin{tabular}[t]{\X{33}{99}\X{33}{99}\X{33}{99}}
\sphinxtoprule
\sphinxtableatstartofbodyhook
\sphinxAtStartPar
版本
&
\sphinxAtStartPar
日期
&
\sphinxAtStartPar
说明
\\
\sphinxhline
\sphinxAtStartPar
V1.0
&
\sphinxAtStartPar
2023.11.6
&
\sphinxAtStartPar
初始版本
\\
\sphinxbottomrule
\end{tabular}
\sphinxtableafterendhook\par
\sphinxattableend\end{savenotes}

\sphinxAtStartPar
\sphinxstylestrong{Important Notice}

\sphinxAtStartPar
Synwit Products are neither intended nor warranted for usage in systems or equipment, any malfunction or failure of which may cause loss of human
life, bodily injury or severe property damage.
Such applications are deemed, “Insecure Usage”.

\sphinxAtStartPar
Insecure usage includes, but is not limited to: equipment for surgical implementation, atomic energy control instruments, airplane or spaceship
instruments, the control or operation of dynamic, brake or safety systems designed for vehicular use, traffic signal instruments, all types of safety
devices, and other applications intended to support or sustain life.

\sphinxAtStartPar
All Insecure Usage shall be made at customer’s risk, and in the event that third parties lay claims to Synwit as a result of customer’s Insecure
Usage, customer shall indemnify the damages and liabilities thus incurred by Synwit.


\chapter{Indices and tables}
\label{\detokenize{index:indices-and-tables}}\begin{itemize}
\item {} 
\sphinxAtStartPar
\DUrole{xref,std,std-ref}{genindex}

\item {} 
\sphinxAtStartPar
\DUrole{xref,std,std-ref}{modindex}

\item {} 
\sphinxAtStartPar
\DUrole{xref,std,std-ref}{search}

\end{itemize}



\renewcommand{\indexname}{Index}
\printindex
\end{document}